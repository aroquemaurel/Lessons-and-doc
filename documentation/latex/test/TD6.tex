\documentclass{slides}

\usepackage{lmodern}
\usepackage{xcolor}
\usepackage[utf8]{inputenc}
\usepackage[T1]{fontenc}
\usepackage[francais]{babel}
\usepackage[top=1.2cm, bottom=1.2cm, left=1.2cm, right=1.2cm]{geometry}
%\usepackage[frenchb]{babel}
%\usepackage{layout}
%\usepackage{setspace}
%\usepackage{soul}
%\usepackage{ulem}
%\usepackage{eurosym}
%\usepackage{bookman}
%\usepackage{charter}
%\usepackage{newcent}
%\usepackage{lmodern}
%\usepackage{mathpazo}
%\usepackage{mathptmx}
%\usepackage{url}
%\usepackage{verbatim}
%\usepackage{moreverb}
%\usepackage{wrapfig}
%\usepackage{amsmath}
%\usepackage{mathrsfs}
%\usepackage{asmthm}
%\usepackage{makeidx}
%\usepackage{tikz} %Vectoriel
\usepackage{listings}
\usepackage{fancyhdr}
\usepackage{multido}
\usepackage{amssymb}


\input{/home/satenske/cours/listings.tex} %prise en charge du langage algo

\title{}
\date{TAD\\ Semestre 2}

\lhead{La religion peut elle survivre avec la science?}
\chead{}
\rhead{\thepage}

\lfoot{UPS}
\cfoot{}
\rfoot{Antoine de ROQUEMAUREL - Kévin SÉGUY}

\pagestyle{fancy}
\begin{document}
	\maketitle
	Question 1 à 4: spécification fonctionnelle du TAD\\
	Question 5 à 7: spécification algorithmique du TAD 
			(type abstrait $\rightarrow$ type concret)

	\section{Identifier les opérations}
		\begin{itemize}
			\item nombre d'éléments (cardinal)
			\item insérer un élément à un ensemble
			\item supprimer un élément d'un ensemble
			\item Tester l'appartenance d'un élément à un ensemble
			\item créer un ensemble vide
		\end{itemize}
	\section{Syntaxe des opérations}
		\begin{eqnarray*}
			ensembleVide&:& \rightarrow Ensemble[T] \\
			estVide&:& Ensemble[T] \rightarrow Booléen\\ 
			appartient&:& Ensemble[T] \times T \rightarrow Booléen \\
			ajouter&:& Ensemble[T] \times T \rightarrow Ensemble[T]\\
			supprimer&:& Ensemble[T] \times T \rightarrow Ensemble[T]\\
			cardinal&:& Ensemble[T] \rightarrow Entier\\
		\end{eqnarray*}
	 \section{Préconditions des opérations}
	 \begin{itemize}
		\item L'opération d'ajout n'est possible que si l'élément n'appartient
		pas déjà à l'ensemble.
		\item L'opération de suppression nécessite que l'élément soit présent
		dans l'ensemble.
	\end{itemize}	
	\subsection{Précondition}
	Pour ens de type Ensemble[T] et e de type T.\\ \\
		ajouter(ens, e) est défini si et seulement si non appartient(ens, e);\\
		supprimer(ens, e) est défini si et seulement si appartient(ens, e);\\
	\section{Sémantique des opérations}
	\begin{tabular}{|c|c|}
		\hline
			ensembleVide & Générateur de base\\
		\hline 
			estVide & Observateur \\
		\hline
			apartient & Observateur \\
		\hline
			ajouter & Générateur de base\\
		\hline
			suprimer & Générateur secondaire\\
		\hline
			cardinal & Observateur \\
		\hline	
	\end{tabular}
	\newpage
	Pour ens de type Ensemble[T] et e de type T.
	\begin{eqnarray}
		estVide(ensembleVide) &=& VRAI \\
		appartient(ensembleVide, e) &=& FAUX\\
		cardinal(ensembleVide) &=& 0\\
		estVide(ajouter(ens,~e)) &=& FAUX\\
		appartient(ajouter(ens,~e),~e') &=& si~e=e'~alors~VRAI~sinon~appartient
				(ens,~e')\\
		cardinal(ajouter(ens,~e)) &=& cardinal(ens)+1\\
		suprimer(ensembleVide,~e) &=& NON~VALIDE.~A~SUPPRIMER\\
		suprimer(ajouter(ens,~e),~e') &=& si~e=e'~alors~ens~sinon~ajouter
				(suprimer(ens,e'),e) 
	\end{eqnarray}	
	
	\section{Incarner le type abstrait en un type concret}
	\lstinputlisting[caption=Entête des opératoins]{1.algo}	
	\lstinputlisting[caption=Entête des opératoins]{2.algo}	


\end{document}

