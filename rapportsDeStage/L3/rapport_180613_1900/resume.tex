\cleardoublepage
\begingroup
	\thispagestyle{empty}
	\par\normalfont\sffamily\selectfont
	 \noindent\Large{Résumé du stage}
	\vspace*{5mm}
\endgroup
	
	Étant encore incertaine quant à ma future orientation professionnelle, j'ai décidé de profiter du stage destiné à valider ma troisième année de licence afin de découvrir le monde de la recherche. Pour cela j'ai contacté Guillaume Cabanac qui m'a proposé un stage en scientométrie\footnote{Étude quantitative de la science par une démarche scientifique.} dans son laboratoire de recherche, l'IRIT\footnote{Institut de Recherche en Informatique de Toulouse.}.
	
	
	
	La première partie de ce stage consistait à valoriser Inforsid, un congrès francophone réunissant chaque année depuis 1983 les chercheurs en Sciences de l'Information (SI). Pour cela je disposais d'une base de données contenant, pour chaque édition du congrès, son comité de programme\footnote{Liste de chercheurs chargés de sélectionner les articles présentés au congrès.}, la ville l'ayant accueilli et les articles présentés. Ce travail était destiné à me familiariser avec le vocabulaire de la recherche scientifique et à me permettre de me documenter au sujet des méthodes scientométriques. J'ai commencé par intégrer les données des éditions 2012 et 2013. Afin de mettre en évidence les pôles majeurs du congrès, j'ai ensuite réalisé deux cartes, une de France et une d'Europe. Pour visualiser les principaux thèmes du congrès et leur évolution j'ai par la suite créé trois nuages de mots (un nuage par décennie) représentant les concepts\footnote{Mots ou expressions de plusieurs mots -- telles que «~base de données~» ou «~recherche d'information~».} les plus fréquents dans les titres des articles présentés. J'ai ensuite réalisé une carte représentant 77 revues scientifiques du domaine SI  sur laquelle étaient mises en évidence les revues comprenant des membres d'Inforsid dans leur comité de rédaction\footnote{Liste de chercheurs chargés de sélectionner les articles à publier dans le journal.}. Enfin j'ai réalisé une carte représentant les différentes villes ayant successivement accueilli le congrès.
	
	
	
	La seconde mission portait sur une étude de genre\footnote{Analyse des différences et des similitudes entre hommes et femmes.} des membres de comités de rédaction des 77 revues scientifiques évoquées précédemment. Les résultats de cette étude étaient destinés à alimenter la rédaction d'un article scientifique. Je disposais des bibliographies de 2\,850 membres~: 422 femmes, 2\,402 hommes et 26 membres dont le sexe n'avait pas pu être déterminé. J'ai calculé plusieurs indicateurs scientométriques, tels que la production (mesurant la quantité de documents publiés par le membre), l'homophilie (mesurant la propension du membre à collaborer avec des femmes) ou le $\varphi$-index (mesurant la capacité du membre à maintenir des collaborations).
	
	Malheureusement nous avons rapidement réalisé que notre calcul de l'homophilie était incorrect à cause d'un manque d'information concernant le genre des coauteurs -- et incalculable à cause de l'énorme quantité de données que nous aurions dû annoter manuellement. J'ai donc décidé de me documenter en consultant les articles scientifiques existants présentant des études de genre. Ceux-ci m'ont décidé à orienter mon étude vers une comparaison de générations : j'ai divisé l'échantillon en deux -- les membres ayant publié leur premier article avant 2000 et les autres -- et observé l'évolution des différences de productivité et de $\varphi$-index entre hommes et femmes. Dans les deux cas on remarque que les femmes de l'ancienne génération produisent moins que les hommes et ont plus de difficultés à collaborer ou à maintenir des collaborations avec leurs pairs. Cependant ces différences tendent à disparaître pour la nouvelle génération -- pour laquelle, sur ces métriques, aucune différence significative n'est mesurable entre hommes et femmes.
	
	Au moment où je rédige ce résumé, cette étude de genre n'est pas encore finalisée mais je pense orienter mon article vers cette différence entre générations.
	
	
	
	Ce stage aura été une expérience profondément enrichissante pour moi. J'ai eu la chance de réaliser un travail de recherche avec un maître de stage ouvert à mes suggestions et toujours présent pour me conseiller. De plus j'ai appris à utiliser de nouveaux outils, concepts et méthodes qui pourront m'être utiles dans le futur.
	
	