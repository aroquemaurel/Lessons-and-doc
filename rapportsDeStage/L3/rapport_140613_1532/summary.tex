\cleardoublepage
\begingroup
	\thispagestyle{empty}
	\par\normalfont\sffamily\selectfont
	\noindent\Large{Internship summary}
	\vspace*{5mm}
\endgroup
	
Being uncertain about my future career, I decided to experience the world of research during the internship scheduled at the end of my undergraduate degree. I liaised with Guillaume Cabanac who offered me to do an internship in scientometrics\footnote{Quantitative study of science by a scientific approach.} at his research laboratory, the IRIT.\footnote{Institut de Recherche en Informatique de Toulouse.}


My work was comprised of two tasks. The first task was to promote Inforsid, a congress gathering  researchers in Information Systems (IS) since 1983, using modern computing facilities.  I used a database containing, for each edition of the congress, the program committee,\footnote{Panel of researchers responsible for selecting articles to be presented at the congress.} the city having hosted and the papers presented. This work was intended to familiarise myself with the language of scientific research and allow me to learn about scientometric methods.

First, I integrated data from the 2012 and 2013 editions. Second, to highlight the congress's main contributors I produced two maps, one for France and one for Europe. Third, to stress the main themes and their evolution, I devised three word clouds (one cloud per decade) representing the most recurrent concepts\footnote{Words or term of several words -- such as ``information retrieval."} in the titles of presented papers. Fourth, I produced a chart representing 77 scientific journals from the IS field. In this chart I emphasized the journals comprising members of Inforsid in their editorial board.\footnote{Panel of researchers responsible for selecting articles for publication in the journal.} Fifth, I made a map showing the cities where the 31 editions were held.


The second task concerned a gender study\footnote{An analysis of the differences and similarities between women and men.} using as input the members of the editorial boards of the 77 aforementioned scientific journals. The main findings of this work were intended to appear in an academic article. There were 2,850 records~: 422 women, 2,402 men and 26 members whose gender could not be determined. I calculated several indicators, such as researchers' production (measuring the amount of material produced by the member), homophily (measuring the propensity of members to collaborate with women) or $\varphi$-index (measuring the ability of the member to maintain collaborations).

Alas, we soon realized that our calculation of homophily was wrong because of a lack of data about the gender of coauthors -- and incalculable because of the huge amount of data that we would have to manually annotate. I then decided to review the literature about gender studies. I read about forty papers spanning a large period (1992--2013). This reading suggested to me the idea to direct my study to a comparison of generations: I divided the aforementioned sample to define two sub-groups -- members who published their first article before 2000 and the others -- and studied the evolution of differences in productivity and $\varphi$-index between men and women. In both cases we find that women from the older generation produce less than men and have more difficulty working or maintaining collaborations with their peers. However these differences tend to disappear for the contemporary generation -- on which, on these metrics, no significant differences were noted between men and women. Thus it seems that well-reported gender gap in academia tends to disappear.

At the time I write this summary, this gender study is not yet finalized, however I think I will focus my article on these differences between generations.


This placement has proved to be a deeply rewarding experience for me. I was given the opportunity to conduct a research study with an advisor open to my suggestions and always available to discuss my progress. Furthermore I learnt to use new tools, concepts and methods that will certainly prove to be useful to me in the future.