\documentclass{report}

\usepackage[utf8]{inputenc}
\usepackage[T1]{fontenc}
\usepackage[francais]{babel}
\usepackage[top=2cm, bottom=2cm, left=2cm, right=2cm]{geometry} %marges
\usepackage{fancyhdr}
\usepackage{graphicx}
\usepackage{verbatim}

\makeatletter
\def\clap#1{\hbox to 0pt{\hss #1\hss}}%
\def\ligne#1{%
\hbox to \hsize{%
\vbox{\centering #1}}}%
\def\haut#1#2#3{%
\hbox to \hsize{%
\rlap{\vtop{\raggedright #1}}%
\hss
\clap{\vtop{\centering #2}}%
\hss
\llap{\vtop{\raggedleft #3}}}}%
\def\bas#1#2#3{%
\hbox to \hsize{%
\rlap{\vbox{\raggedright #1}}%
\hss
\clap{\vbox{\centering #2}}%
\hss
\llap{\vbox{\raggedleft #3}}}}%
\def\maketitle{%
\thispagestyle{empty}\vbox to \vsize{%
\haut{}{\@blurb}{}
\vfill
\vspace{1cm}
\begin{flushleft}
\usefont{OT1}{ptm}{m}{n}
\huge \@title
\end{flushleft}
\hrule height 4pt
\begin{flushright}
\usefont{OT1}{phv}{m}{n}
\Large \@author
\end{flushright}
\vspace{1cm}
\vfill
\vfill
\bas{}{\@location}{}
}%
\cleardoublepage
}
\def\author#1{\def\@author{#1}}
\def\title#1{\def\@title{#1}}
\def\location#1{\def\@location{#1}}
\def\blurb#1{\def\@blurb{#1}}
\location{Amiens}\blurb{}
\makeatother
\title{Système 1}
\author{A.S.R}
\location{Semestre 1}
\blurb{%
Université Toulouse III
Paul sabatier - IUT A\\
\textbf{DUT Informatique}\\[1em]
}%

\begin{document}
	\section{Exercice de maths}
		\subsection{Déterminer les rééls a, b et c tel que $f(x)=ax+b\frac{c}{x+1}$}
			yep, alors pour que ça soit plus clair:

			$$f(x)=\frac{x^{2} + x + 4}{x + 1}$$\\ \\
			
			$$ax+b+\frac{c}{x+1}$$
			$$=\frac{(ax + b)(x + 1)}{x + 1} + \frac{c}{x + 1}$$
			$$=\frac{ax^2 + (a + b)x + (b + c)}{x + 1}$$\\ \\
			
			donc tu veux que $\frac{ax^{2} + (a+b)x + (b+c)}{x + 1} = \frac{x^{2} + x + 4}{x + 1}$\\
			Tu obtiens donc les équations suivantes: 
			$$a = 1$$
			$$b + a == 1$$
			$$b+c = 4$$
	
			
			

\end{document}

