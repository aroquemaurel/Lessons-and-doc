\documentclass[12pt,a4paper,openany]{report}

\usepackage{lmodern}
\usepackage{xcolor}
\usepackage[utf8]{inputenc}
\usepackage[T1]{fontenc}
\usepackage[francais]{babel}
\usepackage[top=1.7cm, bottom=1.7cm, left=1.7cm, right=1.7cm]{geometry}
\usepackage{verbatim}
\usepackage{tikz} %Vectoriel
\usepackage{listings}
\usepackage{fancyhdr}
\usepackage{multido}
\usepackage{amssymb}

\newcommand{\titre}{CRM et Référencement}

\newcommand{\module}{CRM et Référencement}
\newcommand{\sigle}{crm}

\newcommand{\semestre}{3}

\input{/home/satenske/cours/listings.tex} %prise en charge du langage algo
\input{/home/satenske/cours/entete_iut-cours.tex}

\begin{document}
	\maketitle
	\chapter{CRM}
	\section{Qu'est ce qu'un CRM ?}
	Le CRM \footnote{Customer Relationship Management ou Gestionnaire de la Relation Client} 
	permet de cibler, attirer et conserver un client. \\
	Le CRM combine marketing, stratégie et technologies de l'information.
	\subsection{Le marketing transactionnel et relationnel}
	Le CRM s'appuie sur les concepts du \textbf{marketing relationnel}. \\
	\begin{center}
	\begin{tabular}{|c|c|c|}
		\hline
		& Marketing transactionnel & Marketing relationnel \\
		\hline
		Suivi du client & Court terme & Long terme\\
		\hline
		Stratégie ETP & Centrée sur le produit & Centrée sur le client \\
		\hline
		Mesure de la satisfaction & & \\
		\hline
	\end{tabular}	
\end{center}
	\begin{itemize}
		\item Qui sont mes prospects et mes clients: Savoir quel type de client on a est 
			important. \\
		\item Quel est le produit qui conviendra le mieux à mon client ? : Donner l'impression que 
			le client est ``unique''
		\item Comment maintenir le lien avec mon client ? : Campagne de phonie/mailing
	\end{itemize}
	\section{Les principales fonctionnalités du CRM}
	\begin{itemize}
		\item  La prospection commerciale
		\item L'aide à la vente
		\item L'analyse des données
	\end{itemize}
	\section{La mise en place du CRM en entreprise}
	\begin{itemize}
		\item L'environnement organisationnel de l'entreprise
		\item L'environnement technique existant
		\item Le choix d'une solution intranet ou extranet
	\end{itemize}
	\section{CRM: limites et mises en garde}
	\begin{itemize}
		\item Ne pas négliger le facteur humain et la résistance au changement lors de la mise en 
			place du CRM dans l'entreprise
		\item Sensibiliser régulièrement les acteurs au fonctionnement du CRM, donner du sens au 
			projet. 
		\item Vérifier que les procédures de saisie soient respectées
		\item Prendre du recul par rapport aux résultats générés par le CRM.
	\end{itemize}
	\section{Principaux acteurs des CRM}
	\begin{itemize}
		\item Oracle
		\item SAP
		\item SAS
		\item Emailvision
	\end{itemize}
	\chapter{Référencement}
	\begin{itemize}
		\item L'hégémonie de Google
		\item Référencement et positionnement
		\item Quelques techniques de référencement naturel
		\item Le référencement payant avec Google Adwords
	\end{itemize}
	\section{L'hégémonie actuelle de Google}
	\begin{itemize}
		\item Google fût fondé en 1998 par Larry Page (25 ans) et Sergueï Brin (25 ans)
		\item Google est utilisé par 91\% des internautes Français
		\item Le moteur supporte 88 milliards de recherches par mois et dans le monde
		\item 97\% du chiffre d'affaire provient de la publicité
		\item En 2010 Google réalise un chiffre d'affaire de 21 milliards d'euros et un bénéfice 
			de 6 milliards d'euros.
	\end{itemize}
	\section{Référencement et positionnement}
	\subsection{Qu'est ce que le référencement?} C'est le fait d'inscrire un site sur un moteur de recherche.
	\subsection{Référencement naturel et payant}
	\begin{tabular}{cc}
		Référencement naturel & Référecenement payant\\
		Gratuit & Payant\\
		Positionnement plus long & Positionement plus court \\
	\end{tabular}

	\subsection{Le triangle d'or de Google} 
	\begin{itemize}
		\item Position 1: 100\% 
		\item Position 2: 100\% 
		\item Position 3: 100\% 
		\item Position 4: 85\% 
		\item Position 5:  
		\item Position 6: 50\%
		\item Position 7: 50\%
		\item Position 8: 30\%
		\item Position 9: 30\%
		\item Position 10: 20\%
	\end{itemize}
	\section{Quelques techniques de référenceent}
	\subsection{1$^{er}$ niveau : conception technique du site}
	\begin{itemize}
		\item La conformité au recommandations du W3C\footnote{World Wide Web Consumption}
		\item Les mots clefs dans les URL (url rewriting)
		\item L'otimisation des balises <title> et <description> (<keywords> totalement dépassée)
		\item La carte dynamique du site (sitemap.xml)
	\end{itemize}
	\subsection{2$^{nd}$ niveau : communication autour du site}
	\begin{itemize}
		\item Le choix du nom de domaine
		\item La rédaction des contenus, les générateurs de mots clefs
		\item Les échanges de liens (partenariats)
		\item L'inscription dans les moteurs et les annuaires
		\item Les stratégies marketing évoluées de référencement
	\end{itemize}
	\section{Les référencements payant avec AdWords}
	\subsection{Les étapes de lancement d'une campagne AdWords}
	\begin{itemize}
		\item Création de l'annonce textuelle
		\item Choix des mots clefs
		\item Fixation du prix du clic par mot clef
		\item Diffusion de l'annonce
	\end{itemize}

\end{document}
