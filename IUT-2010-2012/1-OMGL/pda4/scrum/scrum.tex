\documentclass[12pt,a4paper,openany]{article}


\usepackage{lmodern}
\usepackage{xcolor}
\usepackage[utf8]{inputenc}
\usepackage[T1]{fontenc}
\usepackage[francais]{babel}
\usepackage[top=1.7cm, bottom=1.7cm, left=1.7cm, right=1.7cm]{geometry}
\usepackage{verbatim}
\usepackage{tikz} %Vectoriel
\usepackage{listings}
\usepackage{fancyhdr}
\usepackage{multido}
\usepackage{amssymb}

\newcommand{\titre}{Méthode \bsc{SCRUM}}

\newcommand{\module}{Processus de développement Agile}
\newcommand{\sigle}{pda}

\newcommand{\semestre}{4}

\newcommand{\footCentre}{}
\newcommand{\premierDestinataire}{Monsieur\bsc{Fernandez}}
\newcommand{\rolePremierDestinataire}{}

\newcommand{\secondDestinaire}{}
\newcommand{\roleSecondDestinaire}{}

\newcommand{\troisiemeDestinaire}{}
\newcommand{\roleTroisiemeDestinaire}{}

\newcommand{\quatriemeDestinaire}{}
\newcommand{\roleQuatriemeDestinaire}{}

\newcommand{\cinquiemeDestinaire}{}
\newcommand{\roleCinquiereDestinaire}{}

\newcommand{\titreDocument}{Méthode \bsc{Scrum}}


\usepackage{verbatim}
\usepackage{datatool}

\date{\today}

\chead{}
\rhead{Projet \#20}
\lhead{Bibliothèque d'objets grahiques UML}
\makeindex
\lfoot{Université Paul Sabatier Toulouse III}
\rfoot{--~\thepage~--}
\cfoot{\footCentre}
\makeglossary
\makeatletter
\def\clap#1{\hbox to 0pt{\hss #1\hss}}%
\def\ligne#1{%
\hbox to \hsize{%
\vbox{\centering #1}}}%
\def\haut#1#2#3{%
\hbox to \hsize{%
\rlap{\vtop{\raggedright #1}}%
\hss
\clap{\vtop{\centering #2}}%
\hss
\llap{\vtop{\raggedleft #3}}}}%
\def\bas#1#2#3{%
\hbox to \hsize{%
\rlap{\vbox{\raggedright #1}}%
\hss \clap{\vbox{\centering #2}}%
\hss
\llap{\vbox{\raggedleft #3}}}}%
\def\maketitle{%
\thispagestyle{empty}\vbox to \vsize{%
\haut{}{\@blurb}{}
\begin{flushleft}
	\vspace{1cm}
	Antoine de \bsc{Roquemaurel}\\ 
	Mathieu \bsc{Soum}\\
	Geoffroy \bsc{Subias}\\
	Marie-Ly \bsc{Tang}\\
	\textit{Groupe B}\\
\end{flushleft}
\begin{flushright}
	\vspace{-3cm}
\begin{tabular}{r@{~}l}
	\ifthenelse{\equal{\premierDestinataire}{}}{
	}
	{
		Pour \premierDestinataire & (\rolePremierDestinataire) \\
	}
	\ifthenelse{\equal{\secondDestinaire}{}}{
	}
	{
		\secondDestinaire & (\roleSecondDestinaire) \\
	}
	\ifthenelse{\equal{\troisiemeDestinaire}{}}{
	}
	{
		\troisiemeDestinaire & (\roleTroisiemeDestinaire) \\
	}
	\ifthenelse{\equal{\quatriemeDestinaire}{}}{
	}
	{
		\quatriemeDestinaire & (\roleQuatriemeDestinaire) \\
	}
	\ifthenelse{\equal{\cinquiemeDestinaire}{}}{
	}
	{
		\cinquiemeDestinaire & (\roleCinquiereDestinaire) \\
	}
\end{tabular}
\end{flushright}
\vfill
\vspace{1cm}
\begin{flushleft}
\usefont{OT1}{ptm}{m}{n}
\huge \@title
\end{flushleft}
\par
\hrule height 4pt
\par
\begin{flushright}
\usefont{OT1}{phv}{m}{n}
\Large \@author
\par
\end{flushright}
\vspace{1cm}
\vfill
\vfill
\bas{}{\@location, le \@date}{}
}%
\cleardoublepage
}
\def\date#1{\def\@date{#1}}
\def\author#1{\def\@author{#1}}
\def\title#1{\def\@title{#1}}
\def\location#1{\def\@location{#1}}
\def\blurb#1{\def\@blurb{#1}}
\date{\today}
\author{}
\title{}
\location{Amiens}\blurb{}
\makeatother
\title{\titreDocument}
\author{Bibliothèque d'objets graphiques UML}

\location{Toulouse}
\blurb{%
Université Paul Sabatier -- Toulouse III\\
IUT A - Toulouse Rangueil\\
\textbf{Projet tuteuré \#20}\\[1em]
}%

\makeatletter
\newcommand{\sortitem}[3]{%
	\DTLnewrow{list}%
	\DTLnewdbentry{list}{nom}{#1}%
	\DTLnewdbentry{list}{page}{#2}%
	\DTLnewdbentry{list}{definition}{#3}%
}

\newenvironment{sortedlist}%
{%
\DTLifdbexists{list}{\DTLcleardb{list}}{\DTLnewdb{list}}%
}%
{%
\DTLsort{nom,page,definition}{list}%a
	\DTLforeach*{list}{\theNom=nom,\laPage=page,\theDefinition=definition}{%
	\paragraph{\theNom}\hspace{-8px}(p \laPage)~--~\theDefinition\hfill 
}%
}

%\newwrite{\verbatim@out@one}
%\newcommand\initiateglossary[1]{\immediate\openout \verbatim@out@one #1}

%\def\terminateglossary{\immediate\closeout\verbatim@out@one\@esphack}
%\DeclareTextFontCommand{\policeGlossaire}{\fontfamily{cmvtt}\selectfont}
\DeclareTextFontCommand{\policeGlossaire}{\fontfamily{lmss}\selectfont}
\DeclareTextFontCommand{\policePackage}{\fontfamily{phv}\selectfont}

\newwrite\glossaireVar
\openout\glossaireVar=glossaire
\write\glossaireVar{\noexpand}
\newcommand{\glo}[3]{
\policeGlossaire{\hspace{-4px}#1\hspace{-6px}}
	\write\glossaireVar{\noexpand\sortitem{#2}{\thepage}{#3}}
}

\makeatother
\newcommand{\nouveauChapitre}{ \thispagestyle{fancy} }
\def\sectionautorefname{Section}
\pagestyle{fancy}

\definecolor{gris1}{gray}{0.40}
\definecolor{gris2}{gray}{0.55}
\definecolor{gris3}{gray}{0.65}
\definecolor{gris4}{gray}{0.50}
\definecolor{vert}{rgb}{0,0.4,0}
\definecolor{violet}{rgb}{0.65, 0.2, 0.65}
\definecolor{bleu1}{rgb}{0,0,0.8}
\definecolor{bleu2}{rgb}{0,0.2,0.6}
\definecolor{bleu3}{rgb}{0,0.2,0.2}
\definecolor{rouge}{HTML}{F93928}


\lstdefinelanguage{algo}{%
   morekeywords={%
    %%% couleur 1
		importer, programme, glossaire, fonction, procedure, constante, type, 
	%%% IMPORT & Co.
		si, sinon, alors, fin, tantque, debut, faire, lorsque, fin lorsque, 
		declenche, declencher, enregistrement, tableau, retourne, retourner, =, pour, a,
		/=, <, >, traite,exception, 
	%%% types 
		Entier, Reel, Booleen, Caractere, Réél, Booléen, Caractère,
	%%% types 
		entree, maj, sortie,entrée,
	%%% types 
		et, ou, non,
	},
  sensitive=true,
  morecomment=[l]{--},
  morestring=[b]',
}

\lstset{language=algo,
    %%% BOUCLE, TEST & Co.
      emph={importer, programme, glossaire, fonction, procedure, constante, type},
      emphstyle=\color{bleu2},
    %%% IMPORT & Co.  
	emph={[2]
		si, sinon, alors, fin , tantque, debut, faire, lorsque, fin lorsque, 
		declencher, retourner, et, ou, non,enregistrement, retourner, retourne, 
		tableau, /=, <, =, >, traite,exception, pour, a
	},
      emphstyle=[2]\color{bleu1},
    %%% FONCTIONS NUMERIQUES
      emph={[3]Entier, Reel, Booleen, Caractere, Booléen, Réél, Caractère},
      emphstyle=[3]\color{gris1},
    %%% FONCTIONS NUMERIQUES
      emph={[4]entree, maj, sortie, entrée},	
      emphstyle=[4]\color{gris1},
}
\lstdefinelanguage{wl}{%
   morekeywords={%
    %%% couleur 1
		importer, programme, glossaire, fonction, procedure, constante, type, 
	%%% IMPORT & Co.
		si, sinon, alors, fin, TANTQUE, tantque, FIN, PROCEDURE, debut, faire, lorsque, 
		fin lorsque, declenche, declencher, enregistrement, tableau, retourne, retourner, =, 
		/=, <, >, traite,exception, 
	%%% types 
		Entier, Reel, Booleen, Caractere, Réél, Booléen, Caractère,
	%%% types 
		entree, maj, sortie,entrée,
	%%% types 
		et, ou, non,
	},
  sensitive=true,
  morecomment=[l]{//},
  morestring=[b]',
}

\lstset{language=wl,
    %%% BOUCLE, TEST & Co.
      emph={importer, programme, glossaire, fonction, procedure, constante, type},
      emphstyle=\color{bleu2},
    %%% IMPORT & Co.  
	emph={[2]
		si, sinon, alors, fin , tantque, debut, faire, lorsque, fin lorsque, 
		declencher, retourner, et, ou, non,enregistrement, retourner, retourne, 
		tableau, /=, <, =, >, traite,exception
	},
      emphstyle=[2]\color{bleu1},
    %%% FONCTIONS NUMERIQUES
      emph={[3]Entier, Reel, Booleen, Caractere, Booléen, Réél, Caractère},
      emphstyle=[3]\color{gris1},
    %%% FONCTIONS NUMERIQUES
      emph={[4]entree, maj, sortie, entrée},	
      emphstyle=[4]\color{gris1},
}
\lstdefinelanguage{css}{%
   morekeywords={%
    %%% couleur 1
		background, image, repeat, position, index, color, border, font, 
		size, url, family, style, variant, weight, letter, spacing, line, 
		height, text, decoration, align, indent, transform, shadow, 
		background, image, repeat, position, index, color, border, font, 
		size, url, family, style, variant, weight, letter, spacing, line, 
		height, text, decoration, align, indent, transform, shadow, 
		vertical, align, white, space, word, spacing,attachment, width, 
		max, min, margin, padding, clip, direction, display, overflow,
		visibility, clear, float, top, right, bottom, left, list, type, 
		collapse, side, empty, cells, table, layout, cursor, marks, page, break,
		before, after, inside, orphans, windows, azimuth, after, before, cue, 
		elevation, pause, play, during, pitch, range, richness, spek, header, 
		numeral, punctuation, rate, stress, voice, volume,
	%%% types 
		left, right, bottom, top, none, center, solid, black, blue, red, green,
	},
  sensitive=true,
  sensitive=true,
  morecomment=[s]{/*}{*/},
  morestring=[b]',
}
\lstset{language=css,
    %%% BOUCLE, TEST & Co.
      emph={
		background, image, repeat, position, index, color, border, font, 
		size, url, family, style, variant, weight, letter, spacing, line, 
		height, text, decoration, align, indent, transform, shadow, 
		background, image, repeat, position, index, color, border, font, 
		size, url, family, style, variant, weight, letter, spacing, line, 
		height, text, decoration, align, indent, transform, shadow, 
		vertical, align, white, space, word, spacing,attachment, width, 
		max, min, margin, padding, clip, direction, display, overflow,
		visibility, clear, float, top, right, bottom, left, list, type, 
		collapse, side, empty, cells, table, layout, cursor, marks, page, break,
		before, after, inside, orphans, windows, azimuth, after, before, cue, 
		elevation, pause, play, during, pitch, range, richness, spek, header, 
		numeral, punctuation, rate, stress, voice, volume,
	  },
      emphstyle=\color{bleu2},
    %%% FONCTIONS NUMERIQUES
      emph={[3]
		left, right, bottom, top,none, solid, black, blue, green,
		  },
      emphstyle=[3]\color{bleu3},
    %%% FONCTIONS NUMERIQUES
}

\lstset{language=SQL,
    %%% BOUCLE, TEST & Co.
      emph={INSERT, UPDATE, DELETE, WHERE, SET, GROUP, BY, ORDER, REFERENCES},
      emphstyle=\color{bleu2},
    %%% IMPORT & Co.  
	emph={[2]
		if, end, begin, then, for, each, else, after, of, on, to
	},
      emphstyle=[2]\color{bleu1},
    %%% FONCTIONS NUMERIQUES
      emph={[3]Entier, Reel, Booleen, Caractere, Booléen, Réél, Caractère},
      emphstyle=[3]\color{gris1},
    %%% FONCTIONS NUMERIQUES
      emph={[4]entree, maj, sortie, entrée},	
      emphstyle=[4]\color{gris1},
}
\lstdefinelanguage{ARM}{%
   morekeywords={%
   ADD, SUB, MOV, MUL, RSB,CMP, BLS, BLE, B,BHI,LDR,
   BGE, RSBLT, BGT, BEQ, BNE,BLT,BHS,STR,STRB
	},
  sensitive=true,
  morecomment=[l]{@},
  morestring=[b]',
}

\lstset{ % general style for listings 
   numbers=left 
   , literate={é}{{\'e}}1 {è}{{\`e}}1 {à}{{\`a}}1 {ê}{{\^e}}1 {É}{{\'E}}1 {ô}{{\^o}}1 {€}{{\euro}}1{°}{{$^{\circ}$}}1 {ç}{ {c}}1 {ù}{u}1
	, extendedchars=\true
   , tabsize=2 
   , frame=l
   , framerule=1.1pt
   , linewidth=520px
   , breaklines=true 
   , basicstyle=\footnotesize\ttfamily 
   , numberstyle=\tiny\ttfamily 
   , framexleftmargin=0mm 
   , xleftmargin=0mm 
   , captionpos=b 
	, keywordstyle=\color{bleu2}
	, commentstyle=\color{vert}
	, stringstyle=\color{rouge}
	, showstringspaces=false
	, extendedchars=true
	, mathescape=true
} 
%	\lstlistoflistings
%	\addcontentsline{toc}{part}{List of code examples}
 %prise en charge du langage algo

\begin{document}
	\maketitle
	\section*{Scrum, un truc qui marche}
	On est naturellement tenté de parler de méthode agile ou de processus agile pour Scrum. En fait, la définition officielle, celle donnée par la Scrum Alliance\footnote{http://www.scrumalliance.org} et son fondateur
	Ken \bsc{Schwaber} et légèrement différente. Scrum n'est présenté ni comme un processus ni comme une méthode.

	Le plus souvent, Ken \bsc{Schwaber} décrit Scrum comme un cadre (framework) ; à d'autre occasions il en parle comme d'une voie ç suivre (path) ou d'un outil et il revient à procesus. 
	Un spécialiste des processus parlerait pour Scrum de pattern de processus\ldots Un spécialise des processus parlerait, pour Scrum, de pattern de processus, orienté gestion de projet,
	qui peut incorporer différentes méthodes ou pratiques d'ingénierie. 

	Qu'on le désigne comme un cadre, un pattern de processus une méthode, voire un truc, Scrum définit des éléments qui feront partie du processus apliqué poru développer un produit. Ces éléments
	sont en petit nombre, le cadre imposé par Scrum étant très léger : guère plus que des itérations des réunoins au début et à la fin de chacune, un hacing de produit et trois rôles;

	Ce coté minimaliste, plus les succès sur le terrain, donnent à croire que Scrum est un truc qui marche.

	\section*{Scrum en bref}
	Si la vraie nature de Scrum est difficile à définir il est beaucoup plus simple d'expliquer la mécanique de mise en \oe{}uvre :
	\begin{itemize}
		\item \textbf{Scrum} sert à développer des produits, généralement en quelques mois. Les fonctionnalités souhaitées sont collectées dans le backlog de produit et classées
			par priorité. C'est le produit Owner qui est responsable de la gestion de ce haking.
		\item Une \textbf{version (release)} est produite par une série d'itérations d'un mois\footnote{On peut remarquer que l'usage de Scrum évolue : par exemple, un epratique courante aujourd'hui est
			d'avoir des sprints de deux semaines, alors que la durée initialie était un mois} appelées des sprints. Le contenu d'un sprint est défini par l'équipe, avec le Product Owner, 
			en tenant compte de spriorités et de la capacité de l'équipe. À partir de ce contenu, l'équipe identifie les tâches nécessaire et s'engage pour réaliser
			les fonctionnalités sélectionnées pour le sprint.
		\item Pendant un psrint des poitns de contrôle sur le déroulement des tâches sont effectués lors des mêlées quotidienne (scrums). Cela permet au \textbf{ScrumMaster} l'animateur chargé de faire
			appliquer Scrum, de déterminer l'avancement par rapport aux engagements et d'appliquer avec l'équipe, des ajustements pour assurer le succés du sprint.
		\item À la fin de chaque spirnt l'équipe obtient un \textbf{produit partiel} (un incrméent) qui fonctionne. Cet incrément du produit est potentiellement livrable et son
			évaluation permet d'ajuster le backlog pour le sprint suivant.
	\end{itemize}
	\section{Théorie}
	Les premières expérimentations de Scrum datent de 1993et le premier article\footnote{htt://jeffsutherland.com/oopsla/schwapub.pdf} est paru en 1995, pour la conférence \bsc{OOPSLA}\footnote{Object-Oriented Programming,
	Systems, Languages \& Applications} ; signé de Ken \bsc{Schwaber}, il présent Scrum comme un processus empirique adapté aux développements de produits complexes.

	Scrum a son origine dans la théorie de contrôle empirique des processus. Les trois piliers de la théorie sont la transparence, l'inspection et l'adaptation du processus dont Scrum fournit le cadre :
	\begin{itemize}
		\item \textbf{Transparence} -- La transparence garantit que tous les indicateurs relatifs à l'état du développement sont visibles par tout ceux qui sont intéréssés
			par le résultat du roduit. Non seulement la transparence pousse à la visibilité mais ce qui est rendu visible doit être bien compris ; 
			cela signifie que ce qui est vue est bien le reflet de la réalité. Par exemle, si un indicateur annonce que le produit est finit, cela doit être
			strictement équivalent à la signification de fini définie par l'équipe.
		\item \textbf{Inspection} -- Les différentes facettes du développement doivent être inspectée suffisamment souvent pour que des variations 
			excessives dans les indicateurs puissent être détectées à temps.
		\item \textbf{Adaptation}
			\begin{itemize}
				\item\textbf{Le scrum quotidien} permet d'inspecter la progression par rapport au but du sprint et de faire des adaptations qui optimisnet la valeur du travail du jour suivant.
				\item\textbf{La planification et la revue de sprint}  sont utilisées pour inspecter l'avancement du développement par rapport au but de la release et faire des adaptations sur le contenu du produit.
				\item \textbf{La rétrospective} inspecte la façon de travailler dans le sprint pour déterminer quelles améliorations du processus euvent être faites
					dans le prochain sprint.
			\end{itemize}
	\end{itemize}
	%%% TODO
	\section{Éléments}
	Le cadre Scrum consiste en une équipe avec des rôles bien définis, des blocs de temps(timeboxes) et des artefacts).\\
	\begin{center}
	\begin{tabular}{|c|p{0.5cm}|c|p{0.5cm}|c|}
		\hline
		\textbf{Rôles} && \textbf{Timeboxes} && \textbf{Artefacts}\\
		Product Owner && Plannification de release && Backlog de produit\\
		ScrumMaster && Planification de release&&Plan de release \\
		Équipe && Plannification de sprint && plan de release\\
		&& Scrum quotidien && plan de sprint\\
		&&Revue de sprint && Burndown de sprint\\
		&&Rétrospective && Burndown de release\\
		\hline
	\end{tabular}\\
\end{center}
	\begin{itemize}
		\item\textbf{ Équipe de rôles} -- L'équipe a un rôle capital dans Scrum : elle est consituée avec le but d'optimiser la flexibilité et la productivité ; pour cela, elle s'organise elle-même et doit
			avoir toutes les compétneces nécessaires au dévelopement de produit. Elle est investie avec le pouvoir et l'autorité pour faire ce qu'elle a à faire.
		\item \textbf{Timeboxes} -- Scrum utilise des blocs de temps pour créer de la régularité. Le c\oe{}ur du rythme de Scrum est le sprint, une itréation d'un mois ou moins. Dans chaque sprint, le cadre est donné par un cérémonial léger mais précis basé des réunions.
		\item \textbf{Artefacts} -- Scrum exige peu d'artefacts lors du développeemnt : le plus remarquable est le backlog de produits pivot des différentes activités.
	\end{itemize}
	Quelques règles liant les éléments complètent ce cadre simple. Toutefois, derrière l'aparente simplicité de Scrum se cache une grande puissanc epour mettre en évidence le degré d'efficacacité des pratiques
	de dévelopements utilisées.
			
\end{document}






