\chapter{Sprint 1}
Au cours de ce sprint nous n'avons prévu que 3 User Stories, en effet, le sprint 1 devait être finit pour le 08 Mars 2012, or
nous devions effectués toutes les étapes de conception préliminaire\footnote{cf chapitre \ref{sprintzero} page \pageref{sprintzero}.} 


%%%%%%%%%%%%%%%%%%%
%%%%%%%%%%%%%%%%%%%



L'user-story « Quota surveillance » (sprint 2) a pu être réalisée lors de ce sprint (avancement).
Nous avons déterminé à partir de ces besoins ainsi que des user-stories prévues les technical-stories suivantes : 


US prévues pour le sprint : 

Saisir contrôle								[Must - 5]
En tant que		responsable de matière,
Je désire	saisir la date d'un contrôle, son type et sa durée
Afin de		planifier les contrôles



Saisir salle								[Would - 5]
En tant que		responsable des plannings, 
Je désire	renseigner une salle de surveillance
Afin de		préciser sa capacité et le nombre de surveillants nécessaires



Visualiser contrôles de l'années en cours				[Could - 3]
En tant que		tout le monde,
Je souhaite		visualiser les contrôles de l'année universitaire en cours
Afin de


Technical stories induites : 

Conception de l'IHM
Conception de la BD
Création de la BD
Exceptions personnalisées


Tâches : (hiérarchie)

User-story
Technical-story
Tâche


Conception de  l'IHM
Création de la fenêtre principale
Connexion de l'utilisateur
Saisir contrôle (entrée : type de contrôle, matière, salle, pôle, date, heure\ldots)
Editer spec contrôles
Ajouter type de contrôle
Ajouter matière
Ajouter salle
Ajouter pôle
Imprimer (création de l'interface)
Changer mot de passe (interface)
Dock filtre (interface : dock permettant d'ajouter des filtres sur la liste des contrôles)
[root] Ajouter utilisateur 
Visualiser contrôle

Conception de la BD
Création de la BD
Listing des requêtes
Code des requêtes
Création jeu d'essai BD

Exceptions personnalisées

Difficultés rencontrées :

Formation à la bibliothèque utilisée (Qt).
Mise en place de l'architecture.

Bilan du sprint :

