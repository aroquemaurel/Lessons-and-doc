\section*{Avant-propos}
\addcontentsline{toc}{chapter}{Avant-propos}
L'enseignement Agile a été mis en oeuvre à travers la création d'une application de gestion de 
surveillance d'examens, ce logiciel fut baptisé ``\textit{Rubidium}''.

Ce logiciel doit permettre aux enseignants de s'auto-affecter à la surveillance de partiels\footnote{
En effet, ils ont un quota d'heure de surveillance à respecter, ce logiciel permettra de les aider à savoir où ils en sont dans ce quota}, aux responsables de matières de créer et d'éditer des partiels et aux administrateurs de gérer en intégralité l'organisation de ces derniers.

Nous avons décidé de développer cette application en C++ grâce à la bibliothèque Qt pour des raisons pratiques\footnote{La flexibilité du code notamment, mais également une rapidité d'exécution, et enfin, le fait qu'elle soit multi-plateforme.} et du fait de l'expérience de certains membres du groupe dans cette bibliothèque.

Planning-poker : attribution de poids aux user-stories données par le client.
Répartition des tâches en 3 sprints en fonction de leurs poids et priorités.

