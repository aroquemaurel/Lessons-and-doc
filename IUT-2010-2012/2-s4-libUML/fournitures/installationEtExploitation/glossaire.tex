
\sortitem {Biblioth\IeC {\`e}que}{3}{Composant programm\IeC {\'e} dans un langage donn\IeC {\'e} fournissant des m\IeC {\'e}thodes permettant d'effectuer des t\IeC {\^a}ches voulut}
\sortitem {UML}{3}{(Unified Modeling Language) Langage de mod\IeC {\'e}lisation graphique \IeC {\`a} base de pictogramme. Il est apparu dans le monde du g\IeC {\'e}nie logiciel dans le cadre de la conception orient\IeC {\'e}e objet. Ce langage est compos\IeC {\'e} de diff\IeC {\'e}rents diagrammes, allant du d\IeC {\'e}veloppement \IeC {\`a} la simple analyse des besoins.}
\sortitem {Java}{3}{Langage de programmation orient\IeC {\'e} objet moderne, il compile le programme pour ensuite l'ex\IeC {\'e}cuter sur une machine Java, ainsi le programme une fois compil\IeC {\'e} peut \IeC {\^e}tre ex\IeC {\'e}cut\IeC {\'e} sur diff\IeC {\'e}rentes plateformes (Windows, Linux, Mac OS X, \dots  ).}
