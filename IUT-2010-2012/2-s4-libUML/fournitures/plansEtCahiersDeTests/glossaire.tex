
\sortitem {Biblioth\IeC {\`e}que}{5}{Un composant ind\IeC {\'e}pendant fournissant des m\IeC {\'e}thodes d\IeC {\'e}j\IeC {\`a} impl\IeC {\'e}ment\IeC {\'e}es, facilitant le d\IeC {\'e}veloppement de d'une application plus importante.}
\sortitem {UML}{5}{(Unified Modeling Language) Langage de mod\IeC {\'e}lisation graphique \IeC {\`a} base de pictogramme. Il est apparu dans le monde du g\IeC {\'e}nie logiciel dans le cadre de la conception orient\IeC {\'e}e objet. Ce langage est compos\IeC {\'e} de diff\IeC {\'e}rents diagrammes, allant du d\IeC {\'e}veloppement \IeC {\`a} la simple analyse des besoins. }
\sortitem {D\IeC {\'e}monstrateur}{5}{Un d\IeC {\'e}monstrateur est un logiciel simple, permettant de montrer les possibilit\IeC {\'e}s d'une biblioth\IeC {\`e}que}
\sortitem {Diagramme de classes}{5}{Sch\IeC {\'e}ma utilis\IeC {\'e} en g\IeC {\'e}nie logiciel pour repr\IeC {\'e}senter les classes et les interfaces des syst\IeC {\`e}mes ainsi que les diff\IeC {\'e}rentes relations entre celles-ci.}
\sortitem {Diagramme de s\IeC {\'e}quence}{5}{ Repr\IeC {\'e}sentation graphique des interactions entre les acteurs et le syst\IeC {\`e}me selon un ordre chronologique. Ce diagramme est inclus dans la partie dynamique d'UML.}
\sortitem {Diagramme de cas d'utilisation}{5}{Repr\IeC {\'e}sentation graphique permettant de d\IeC {\'e}crire les int\IeC {\'e}ractions entre les acteurs et le syst\IeC {\`e}me.}
\sortitem {JUnit}{7}{Biblioth\IeC {\`e}que permettant de cr\IeC {\'e}er des tests unitaires simplement}
