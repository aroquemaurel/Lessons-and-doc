	\chapter{Analyse des risques}
%<<<<<<< HEAD
%		%lister risques potentiels w/ impact (délai, cout, résultat et contre mesure
%		Nous avons dégager trucpourafficherlecompteurditem risques majeurs : % TODO <--- compteur
%		\begin{itemize}
%			\item Le premier risque réside dans le fait que les personnes
%				participant au projet sont, à l'heure actuelle, étudiantes.
%				Bien que les technologies informatiques nécessaires soient
%				maîtrisées, la contrainte de temps n'est pas à négliger.
%				En effet, lors d'un précedent esaie, le projet n'avait pas
%				abouti à cause d'un manque de temps. Cependant, ce facteur 
%				de isque a été pris en compte lors de la construction de la
%				nouvelle équipe et, les moyens humais et le temps imparti,
%				sont donc devenu plus important.
%
%			\item Le deuxième risque est la fait qu'il est nécessaire qu'un
%				nombre important de téléchargement soit atteint avant d'engager
%				des démarches commerciales. Cependant, la gratuité du produit,
%				le besoin conséquent en matière de logiciel de bilans thermiques
%				et la communication prévue devraient assurer la visibilité du
%				site internet et par conséquent du produit. 
% 
%			\item Le troisième riqsue est que notre offre de support média
%				n'intéresse pas les annonceurs, toutefois, il s'agit d'un
%				support innovant permettant de cibler précisemment leurs clients,
%				doté d'une iteractivité et d'un temps de visibilité sans précédent.
%				De plus, notre politique de prix permet d'avoir une offre abordable
%				pour les annonceurs visés qui ont plusieurs centaines de milliers
%				d'euro de budget mensuel de communication.
%		\end{itemize}
%=======

		%lister risques potentiels / impact (délai, cout, résultat et contre mesure)
		%% TODO Compléter le tableau
		%% TODO Ajouter des risques pour que Marquie soit content
  \begin{table}[H]
	  \centering
	  \rowcolors{2}{grisfonce}{grisclair}
	  \begin{tabular}{|p{5.5cm}|c|c|p{6.5cm}|}
		  \hline
		  \textbf{Risques} & \textbf{Pertinence} & \textbf{Coût} & \textbf{Solution} \\
		  \hline
			Contrainte de temps &
			Haute & 
			&\\
		\hline
			Il est nécessaire qu'un nombre important de téléchargement soit atteint avant d'engager des démarches commerciales. &
			Moyenne & 
			&\\
		\hline
			Les annonceurs peuvent ne pas être intéressés par notre offre &
			Haute&
			&\\
		\hline

	  \end{tabular}
	  \caption{Tarifs des espaces publicitaires en fonction du temps d'affichage par heure}
	  \label{tab:risques}
  \end{table}
%>>>>>>> 26a396224663c33c5853ef04f026e938fc5ae45c
