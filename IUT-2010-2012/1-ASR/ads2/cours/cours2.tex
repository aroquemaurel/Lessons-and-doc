\documentclass[12pt,a4paper,openany]{article}

\usepackage{lmodern}
\usepackage{xcolor}
\usepackage[utf8]{inputenc}
\usepackage[T1]{fontenc}
\usepackage[francais]{babel}
\usepackage[top=1.7cm, bottom=1.7cm, left=1.7cm, right=1.7cm]{geometry}
%\usepackage[frenchb]{babel}
%\usepackage{layout}
%\usepackage{setspace}
%\usepackage{soul}
%\usepackage{ulem}
%\usepackage{eurosym}
%\usepackage{bookman}
%\usepackage{charter}
%\usepackage{newcent}
%\usepackage{lmodern}
%\usepackage{mathpazo}
%\usepackage{mathptmx}
%\usepackage{url}
%\usepackage{verbatim}
%\usepackage{moreverb}
%\usepackage{wrapfig}
%\usepackage{amsmath}
%\usepackage{mathrsfs}
%\usepackage{asmthm}
%\usepackage{makeidx}
%\usepackage{tikz} %Vectoriel
\usepackage{listings}
%\usepackage{fancyhdr}
%\usepackage{multido}
%\usepackage{amssymb}


%\lstset{ % general style for listings 
 %  numbers=left 
%	, extendedchars=\true
  % , tabsize=2 
 %  , frame=single 
%   , breaklines=true 
   %, basicstyle=\ttfamily 
  % , numberstyle=\tiny\ttfamily 
 %  , framexleftmargin=13mm 
%  , xleftmargin=12mm 
 % , captionpos=b 
%	, language=html
%	, keywordstyle=\color{blue}
%	, commentstyle=\color{vert}
%	, showstringspaces=false
%	, extendedchars=true
%	, mathescape=true
%} 

\title{Cours\\ Programmation Internet}
\date{PGI\\ Semestre 2}

\pagestyle{fancy}
\begin{document}
	\maketitle
	\chapter{HTML}
		\begin{itemize}
			\item Balises
			\item Doctype
			\item Formulaires
		\end{itemize}
%		\lstinputlisting[language=html]{test.html}	
		\paragraph{CMS} Content Managment System. Créer des sites sans utiliser 
		du code HTML, automatisé, souvent logiciels libre, rapide et plus facile.
		\section{Formulaires}
		\paragraph{} Permet d'entrer des données par l'utilisateurs, lié à une 
		page dynamique, forum; chat etc...\\
		%% Graphic for TeX using PGF
% Title: /usr/home/satenske/Diagram1.dia
% Creator: Dia v0.97.1
% CreationDate: Wed Mar 30 09:12:35 2011
% For: satenske
% \usepackage{tikz}
% The following commands are not supported in PSTricks at present
% We define them conditionally, so when they are implemented,
% this pgf file will use them.
\ifx\du\undefined
  \newlength{\du}
\fi
\setlength{\du}{15\unitlength}
\begin{tikzpicture}
\pgftransformxscale{1.000000}
\pgftransformyscale{-1.000000}
\definecolor{dialinecolor}{rgb}{0.000000, 0.000000, 0.000000}
\pgfsetstrokecolor{dialinecolor}
\definecolor{dialinecolor}{rgb}{1.000000, 1.000000, 1.000000}
\pgfsetfillcolor{dialinecolor}
\pgfsetlinewidth{0.100000\du}
\pgfsetdash{}{0pt}
\pgfsetdash{}{0pt}
\pgfsetbuttcap
\pgfsetmiterjoin
\pgfsetlinewidth{0.001000\du}
\pgfsetbuttcap
\pgfsetmiterjoin
\pgfsetdash{}{0pt}
\definecolor{dialinecolor}{rgb}{0.717647, 0.717647, 0.615686}
\pgfsetfillcolor{dialinecolor}
\pgfpathmoveto{\pgfpoint{9.544836\du}{9.940680\du}}
\pgfpathlineto{\pgfpoint{11.543262\du}{9.940680\du}}
\pgfpathlineto{\pgfpoint{11.543262\du}{10.310013\du}}
\pgfpathlineto{\pgfpoint{9.544836\du}{10.310013\du}}
\pgfpathlineto{\pgfpoint{9.544836\du}{9.940680\du}}
\pgfusepath{fill}
\pgfsetbuttcap
\pgfsetmiterjoin
\pgfsetdash{}{0pt}
\definecolor{dialinecolor}{rgb}{0.286275, 0.286275, 0.211765}
\pgfsetstrokecolor{dialinecolor}
\pgfpathmoveto{\pgfpoint{9.544836\du}{9.940680\du}}
\pgfpathlineto{\pgfpoint{11.543262\du}{9.940680\du}}
\pgfpathlineto{\pgfpoint{11.543262\du}{10.310013\du}}
\pgfpathlineto{\pgfpoint{9.544836\du}{10.310013\du}}
\pgfpathlineto{\pgfpoint{9.544836\du}{9.940680\du}}
\pgfusepath{stroke}
\pgfsetbuttcap
\pgfsetmiterjoin
\pgfsetdash{}{0pt}
\definecolor{dialinecolor}{rgb}{0.788235, 0.788235, 0.713726}
\pgfsetfillcolor{dialinecolor}
\pgfpathmoveto{\pgfpoint{9.544836\du}{9.940680\du}}
\pgfpathlineto{\pgfpoint{9.756738\du}{9.739798\du}}
\pgfpathlineto{\pgfpoint{11.755164\du}{9.739798\du}}
\pgfpathlineto{\pgfpoint{11.543262\du}{9.940680\du}}
\pgfpathlineto{\pgfpoint{9.544836\du}{9.940680\du}}
\pgfusepath{fill}
\pgfsetbuttcap
\pgfsetmiterjoin
\pgfsetdash{}{0pt}
\definecolor{dialinecolor}{rgb}{0.286275, 0.286275, 0.211765}
\pgfsetstrokecolor{dialinecolor}
\pgfpathmoveto{\pgfpoint{9.544836\du}{9.940680\du}}
\pgfpathlineto{\pgfpoint{9.756738\du}{9.739798\du}}
\pgfpathlineto{\pgfpoint{11.755164\du}{9.739798\du}}
\pgfpathlineto{\pgfpoint{11.543262\du}{9.940680\du}}
\pgfpathlineto{\pgfpoint{9.544836\du}{9.940680\du}}
\pgfusepath{stroke}
\pgfsetlinewidth{0.106000\du}
\pgfsetbuttcap
\pgfsetmiterjoin
\pgfsetdash{}{0pt}
\definecolor{dialinecolor}{rgb}{0.000000, 0.000000, 0.000000}
\pgfsetstrokecolor{dialinecolor}
\pgfpathmoveto{\pgfpoint{11.431486\du}{10.108816\du}}
\pgfpathlineto{\pgfpoint{10.951952\du}{10.108816\du}}
\pgfusepath{stroke}
\pgfsetlinewidth{0.001000\du}
\pgfsetbuttcap
\pgfsetmiterjoin
\pgfsetdash{}{0pt}
\definecolor{dialinecolor}{rgb}{0.478431, 0.478431, 0.352941}
\pgfsetfillcolor{dialinecolor}
\pgfpathmoveto{\pgfpoint{11.543262\du}{10.310013\du}}
\pgfpathlineto{\pgfpoint{11.755164\du}{10.097481\du}}
\pgfpathlineto{\pgfpoint{11.755164\du}{9.739798\du}}
\pgfpathlineto{\pgfpoint{11.543262\du}{9.940680\du}}
\pgfpathlineto{\pgfpoint{11.543262\du}{10.310013\du}}
\pgfusepath{fill}
\pgfsetbuttcap
\pgfsetmiterjoin
\pgfsetdash{}{0pt}
\definecolor{dialinecolor}{rgb}{0.286275, 0.286275, 0.211765}
\pgfsetstrokecolor{dialinecolor}
\pgfpathmoveto{\pgfpoint{11.543262\du}{10.310013\du}}
\pgfpathlineto{\pgfpoint{11.755164\du}{10.097481\du}}
\pgfpathlineto{\pgfpoint{11.755164\du}{9.739798\du}}
\pgfpathlineto{\pgfpoint{11.543262\du}{9.940680\du}}
\pgfpathlineto{\pgfpoint{11.543262\du}{10.310013\du}}
\pgfusepath{stroke}
\pgfsetbuttcap
\pgfsetmiterjoin
\pgfsetdash{}{0pt}
\definecolor{dialinecolor}{rgb}{0.788235, 0.788235, 0.713726}
\pgfsetfillcolor{dialinecolor}
\pgfpathmoveto{\pgfpoint{9.556171\du}{10.544270\du}}
\pgfpathlineto{\pgfpoint{9.779093\du}{10.264987\du}}
\pgfpathlineto{\pgfpoint{11.320025\du}{10.264987\du}}
\pgfpathlineto{\pgfpoint{11.097103\du}{10.544270\du}}
\pgfpathlineto{\pgfpoint{9.556171\du}{10.544270\du}}
\pgfusepath{fill}
\pgfsetbuttcap
\pgfsetmiterjoin
\pgfsetdash{}{0pt}
\definecolor{dialinecolor}{rgb}{0.286275, 0.286275, 0.211765}
\pgfsetstrokecolor{dialinecolor}
\pgfpathmoveto{\pgfpoint{9.556171\du}{10.544270\du}}
\pgfpathlineto{\pgfpoint{9.779093\du}{10.264987\du}}
\pgfpathlineto{\pgfpoint{11.320025\du}{10.264987\du}}
\pgfpathlineto{\pgfpoint{11.097103\du}{10.544270\du}}
\pgfpathlineto{\pgfpoint{9.556171\du}{10.544270\du}}
\pgfusepath{stroke}
\pgfsetbuttcap
\pgfsetmiterjoin
\pgfsetdash{}{0pt}
\definecolor{dialinecolor}{rgb}{0.478431, 0.478431, 0.352941}
\pgfsetfillcolor{dialinecolor}
\pgfpathmoveto{\pgfpoint{11.097103\du}{10.600000\du}}
\pgfpathlineto{\pgfpoint{11.320025\du}{10.365743\du}}
\pgfpathlineto{\pgfpoint{11.320025\du}{10.264987\du}}
\pgfpathlineto{\pgfpoint{11.097103\du}{10.544270\du}}
\pgfpathlineto{\pgfpoint{11.097103\du}{10.600000\du}}
\pgfusepath{fill}
\pgfsetbuttcap
\pgfsetmiterjoin
\pgfsetdash{}{0pt}
\definecolor{dialinecolor}{rgb}{0.286275, 0.286275, 0.211765}
\pgfsetstrokecolor{dialinecolor}
\pgfpathmoveto{\pgfpoint{11.097103\du}{10.600000\du}}
\pgfpathlineto{\pgfpoint{11.320025\du}{10.365743\du}}
\pgfpathlineto{\pgfpoint{11.320025\du}{10.264987\du}}
\pgfpathlineto{\pgfpoint{11.097103\du}{10.544270\du}}
\pgfpathlineto{\pgfpoint{11.097103\du}{10.600000\du}}
\pgfusepath{stroke}
\pgfsetbuttcap
\pgfsetmiterjoin
\pgfsetdash{}{0pt}
\definecolor{dialinecolor}{rgb}{0.717647, 0.717647, 0.615686}
\pgfsetfillcolor{dialinecolor}
\pgfpathmoveto{\pgfpoint{9.556171\du}{10.544270\du}}
\pgfpathlineto{\pgfpoint{11.097103\du}{10.544270\du}}
\pgfpathlineto{\pgfpoint{11.097103\du}{10.600000\du}}
\pgfpathlineto{\pgfpoint{9.556171\du}{10.600000\du}}
\pgfpathlineto{\pgfpoint{9.556171\du}{10.544270\du}}
\pgfusepath{fill}
\pgfsetbuttcap
\pgfsetmiterjoin
\pgfsetdash{}{0pt}
\definecolor{dialinecolor}{rgb}{0.286275, 0.286275, 0.211765}
\pgfsetstrokecolor{dialinecolor}
\pgfpathmoveto{\pgfpoint{9.556171\du}{10.544270\du}}
\pgfpathlineto{\pgfpoint{11.097103\du}{10.544270\du}}
\pgfpathlineto{\pgfpoint{11.097103\du}{10.600000\du}}
\pgfpathlineto{\pgfpoint{9.556171\du}{10.600000\du}}
\pgfpathlineto{\pgfpoint{9.556171\du}{10.544270\du}}
\pgfusepath{stroke}
\pgfsetbuttcap
\pgfsetmiterjoin
\pgfsetdash{}{0pt}
\definecolor{dialinecolor}{rgb}{0.000000, 0.000000, 0.000000}
\pgfsetfillcolor{dialinecolor}
\pgfpathmoveto{\pgfpoint{9.846159\du}{9.896285\du}}
\pgfpathlineto{\pgfpoint{10.013980\du}{9.739798\du}}
\pgfpathlineto{\pgfpoint{11.431486\du}{9.739798\du}}
\pgfpathlineto{\pgfpoint{11.275630\du}{9.896285\du}}
\pgfpathlineto{\pgfpoint{9.846159\du}{9.896285\du}}
\pgfusepath{fill}
\pgfsetbuttcap
\pgfsetmiterjoin
\pgfsetdash{}{0pt}
\definecolor{dialinecolor}{rgb}{0.000000, 0.000000, 0.000000}
\pgfsetstrokecolor{dialinecolor}
\pgfpathmoveto{\pgfpoint{9.846159\du}{9.896285\du}}
\pgfpathlineto{\pgfpoint{10.013980\du}{9.739798\du}}
\pgfpathlineto{\pgfpoint{11.431486\du}{9.739798\du}}
\pgfpathlineto{\pgfpoint{11.275630\du}{9.896285\du}}
\pgfpathlineto{\pgfpoint{9.846159\du}{9.896285\du}}
\pgfusepath{stroke}
\pgfsetbuttcap
\pgfsetmiterjoin
\pgfsetdash{}{0pt}
\definecolor{dialinecolor}{rgb}{0.788235, 0.788235, 0.713726}
\pgfsetfillcolor{dialinecolor}
\pgfpathmoveto{\pgfpoint{9.834824\du}{8.745151\du}}
\pgfpathlineto{\pgfpoint{9.991625\du}{8.600000\du}}
\pgfpathlineto{\pgfpoint{11.409761\du}{8.600000\du}}
\pgfpathlineto{\pgfpoint{11.252960\du}{8.745151\du}}
\pgfpathlineto{\pgfpoint{9.834824\du}{8.745151\du}}
\pgfusepath{fill}
\pgfsetbuttcap
\pgfsetmiterjoin
\pgfsetdash{}{0pt}
\definecolor{dialinecolor}{rgb}{0.286275, 0.286275, 0.211765}
\pgfsetstrokecolor{dialinecolor}
\pgfpathmoveto{\pgfpoint{9.834824\du}{8.745151\du}}
\pgfpathlineto{\pgfpoint{9.991625\du}{8.600000\du}}
\pgfpathlineto{\pgfpoint{11.409761\du}{8.600000\du}}
\pgfpathlineto{\pgfpoint{11.252960\du}{8.745151\du}}
\pgfpathlineto{\pgfpoint{9.834824\du}{8.745151\du}}
\pgfusepath{stroke}
\pgfsetbuttcap
\pgfsetmiterjoin
\pgfsetdash{}{0pt}
\definecolor{dialinecolor}{rgb}{0.717647, 0.717647, 0.615686}
\pgfsetfillcolor{dialinecolor}
\pgfpathmoveto{\pgfpoint{9.834824\du}{8.745151\du}}
\pgfpathlineto{\pgfpoint{11.264295\du}{8.745151\du}}
\pgfpathlineto{\pgfpoint{11.264295\du}{9.873615\du}}
\pgfpathlineto{\pgfpoint{9.834824\du}{9.873615\du}}
\pgfpathlineto{\pgfpoint{9.834824\du}{8.745151\du}}
\pgfusepath{fill}
\pgfsetbuttcap
\pgfsetmiterjoin
\pgfsetdash{}{0pt}
\definecolor{dialinecolor}{rgb}{0.286275, 0.286275, 0.211765}
\pgfsetstrokecolor{dialinecolor}
\pgfpathmoveto{\pgfpoint{9.834824\du}{8.745151\du}}
\pgfpathlineto{\pgfpoint{11.263665\du}{8.745151\du}}
\pgfpathlineto{\pgfpoint{11.263665\du}{9.873300\du}}
\pgfpathlineto{\pgfpoint{9.834824\du}{9.873300\du}}
\pgfpathlineto{\pgfpoint{9.834824\du}{8.745151\du}}
\pgfusepath{stroke}
\pgfsetbuttcap
\pgfsetmiterjoin
\pgfsetdash{}{0pt}
\definecolor{dialinecolor}{rgb}{1.000000, 1.000000, 1.000000}
\pgfsetfillcolor{dialinecolor}
\pgfpathmoveto{\pgfpoint{9.957620\du}{8.889987\du}}
\pgfpathlineto{\pgfpoint{11.141184\du}{8.889987\du}}
\pgfpathlineto{\pgfpoint{11.141184\du}{9.761839\du}}
\pgfpathlineto{\pgfpoint{9.957620\du}{9.761839\du}}
\pgfpathlineto{\pgfpoint{9.957620\du}{8.889987\du}}
\pgfusepath{fill}
\pgfsetbuttcap
\pgfsetmiterjoin
\pgfsetdash{}{0pt}
\definecolor{dialinecolor}{rgb}{0.286275, 0.286275, 0.211765}
\pgfsetstrokecolor{dialinecolor}
\pgfpathmoveto{\pgfpoint{9.957620\du}{8.889987\du}}
\pgfpathlineto{\pgfpoint{11.141184\du}{8.889987\du}}
\pgfpathlineto{\pgfpoint{11.141184\du}{9.761524\du}}
\pgfpathlineto{\pgfpoint{9.957620\du}{9.761524\du}}
\pgfpathlineto{\pgfpoint{9.957620\du}{8.889987\du}}
\pgfusepath{stroke}
\pgfsetbuttcap
\pgfsetmiterjoin
\pgfsetdash{}{0pt}
\definecolor{dialinecolor}{rgb}{0.478431, 0.478431, 0.352941}
\pgfsetfillcolor{dialinecolor}
\pgfpathmoveto{\pgfpoint{11.252960\du}{9.862909\du}}
\pgfpathlineto{\pgfpoint{11.409761\du}{9.706423\du}}
\pgfpathlineto{\pgfpoint{11.409761\du}{8.600000\du}}
\pgfpathlineto{\pgfpoint{11.252960\du}{8.745151\du}}
\pgfpathlineto{\pgfpoint{11.252960\du}{9.862909\du}}
\pgfusepath{fill}
\pgfsetbuttcap
\pgfsetmiterjoin
\pgfsetdash{}{0pt}
\definecolor{dialinecolor}{rgb}{0.286275, 0.286275, 0.211765}
\pgfsetstrokecolor{dialinecolor}
\pgfpathmoveto{\pgfpoint{11.252960\du}{9.862909\du}}
\pgfpathlineto{\pgfpoint{11.409761\du}{9.706423\du}}
\pgfpathlineto{\pgfpoint{11.409761\du}{8.600000\du}}
\pgfpathlineto{\pgfpoint{11.252960\du}{8.745151\du}}
\pgfpathlineto{\pgfpoint{11.252960\du}{9.862909\du}}
\pgfusepath{stroke}
\pgfsetlinewidth{0.100000\du}
\pgfsetdash{}{0pt}
\pgfsetdash{}{0pt}
\pgfsetbuttcap
\pgfsetmiterjoin
\pgfsetlinewidth{0.001000\du}
\pgfsetbuttcap
\pgfsetmiterjoin
\pgfsetdash{}{0pt}
\definecolor{dialinecolor}{rgb}{0.717647, 0.717647, 0.615686}
\pgfsetfillcolor{dialinecolor}
\pgfpathmoveto{\pgfpoint{20.628989\du}{8.998912\du}}
\pgfpathlineto{\pgfpoint{20.628989\du}{10.850000\du}}
\pgfpathlineto{\pgfpoint{21.723081\du}{10.850000\du}}
\pgfpathlineto{\pgfpoint{21.723081\du}{8.998912\du}}
\pgfpathlineto{\pgfpoint{20.628989\du}{8.998912\du}}
\pgfusepath{fill}
\pgfsetbuttcap
\pgfsetmiterjoin
\pgfsetdash{}{0pt}
\definecolor{dialinecolor}{rgb}{0.286275, 0.286275, 0.211765}
\pgfsetstrokecolor{dialinecolor}
\pgfpathmoveto{\pgfpoint{20.628989\du}{8.998912\du}}
\pgfpathlineto{\pgfpoint{20.628989\du}{10.850000\du}}
\pgfpathlineto{\pgfpoint{21.723081\du}{10.850000\du}}
\pgfpathlineto{\pgfpoint{21.723081\du}{8.998912\du}}
\pgfpathlineto{\pgfpoint{20.628989\du}{8.998912\du}}
\pgfusepath{stroke}
\pgfsetbuttcap
\pgfsetmiterjoin
\pgfsetdash{}{0pt}
\definecolor{dialinecolor}{rgb}{0.788235, 0.788235, 0.713726}
\pgfsetfillcolor{dialinecolor}
\pgfpathmoveto{\pgfpoint{20.628989\du}{8.998912\du}}
\pgfpathlineto{\pgfpoint{20.777246\du}{8.850000\du}}
\pgfpathlineto{\pgfpoint{21.871011\du}{8.850000\du}}
\pgfpathlineto{\pgfpoint{21.723081\du}{8.998912\du}}
\pgfpathlineto{\pgfpoint{20.628989\du}{8.998912\du}}
\pgfusepath{fill}
\pgfsetbuttcap
\pgfsetmiterjoin
\pgfsetdash{}{0pt}
\definecolor{dialinecolor}{rgb}{0.286275, 0.286275, 0.211765}
\pgfsetstrokecolor{dialinecolor}
\pgfpathmoveto{\pgfpoint{20.628989\du}{8.998912\du}}
\pgfpathlineto{\pgfpoint{20.777246\du}{8.850000\du}}
\pgfpathlineto{\pgfpoint{21.863811\du}{8.850000\du}}
\pgfusepath{stroke}
\pgfsetbuttcap
\pgfsetmiterjoin
\pgfsetdash{}{0pt}
\definecolor{dialinecolor}{rgb}{0.286275, 0.286275, 0.211765}
\pgfsetstrokecolor{dialinecolor}
\pgfpathmoveto{\pgfpoint{21.863811\du}{8.857527\du}}
\pgfpathlineto{\pgfpoint{21.723081\du}{8.998912\du}}
\pgfpathlineto{\pgfpoint{20.628989\du}{8.998912\du}}
\pgfusepath{stroke}
\pgfsetbuttcap
\pgfsetmiterjoin
\pgfsetdash{}{0pt}
\definecolor{dialinecolor}{rgb}{0.788235, 0.788235, 0.713726}
\pgfsetfillcolor{dialinecolor}
\pgfpathmoveto{\pgfpoint{20.696408\du}{9.106586\du}}
\pgfpathlineto{\pgfpoint{21.196163\du}{9.106586\du}}
\pgfpathlineto{\pgfpoint{21.196163\du}{9.349427\du}}
\pgfpathlineto{\pgfpoint{20.696408\du}{9.349427\du}}
\pgfpathlineto{\pgfpoint{20.696408\du}{9.106586\du}}
\pgfusepath{fill}
\pgfsetbuttcap
\pgfsetmiterjoin
\pgfsetdash{}{0pt}
\definecolor{dialinecolor}{rgb}{0.384314, 0.384314, 0.282353}
\pgfsetstrokecolor{dialinecolor}
\pgfpathmoveto{\pgfpoint{20.696408\du}{9.106586\du}}
\pgfpathlineto{\pgfpoint{21.195835\du}{9.106586\du}}
\pgfpathlineto{\pgfpoint{21.195835\du}{9.349100\du}}
\pgfpathlineto{\pgfpoint{20.696408\du}{9.349100\du}}
\pgfpathlineto{\pgfpoint{20.696408\du}{9.106586\du}}
\pgfusepath{stroke}
\pgfsetlinewidth{0.030000\du}
\pgfsetbuttcap
\pgfsetmiterjoin
\pgfsetdash{}{0pt}
\definecolor{dialinecolor}{rgb}{0.925490, 0.925490, 0.905882}
\pgfsetstrokecolor{dialinecolor}
\pgfpathmoveto{\pgfpoint{20.763828\du}{9.228334\du}}
\pgfpathlineto{\pgfpoint{21.114343\du}{9.228334\du}}
\pgfusepath{stroke}
\pgfsetlinewidth{0.001000\du}
\pgfsetbuttcap
\pgfsetmiterjoin
\pgfsetdash{}{0pt}
\definecolor{dialinecolor}{rgb}{0.478431, 0.478431, 0.352941}
\pgfsetfillcolor{dialinecolor}
\pgfpathmoveto{\pgfpoint{21.723081\du}{10.850000\du}}
\pgfpathlineto{\pgfpoint{21.871011\du}{10.700761\du}}
\pgfpathlineto{\pgfpoint{21.871011\du}{8.850000\du}}
\pgfpathlineto{\pgfpoint{21.723081\du}{8.998912\du}}
\pgfpathlineto{\pgfpoint{21.723081\du}{10.850000\du}}
\pgfusepath{fill}
\pgfsetbuttcap
\pgfsetmiterjoin
\pgfsetdash{}{0pt}
\definecolor{dialinecolor}{rgb}{0.286275, 0.286275, 0.211765}
\pgfsetstrokecolor{dialinecolor}
\pgfpathmoveto{\pgfpoint{21.723081\du}{10.850000\du}}
\pgfpathlineto{\pgfpoint{21.863811\du}{10.708288\du}}
\pgfusepath{stroke}
\pgfsetbuttcap
\pgfsetmiterjoin
\pgfsetdash{}{0pt}
\definecolor{dialinecolor}{rgb}{0.286275, 0.286275, 0.211765}
\pgfsetstrokecolor{dialinecolor}
\pgfpathmoveto{\pgfpoint{21.863811\du}{8.857527\du}}
\pgfpathlineto{\pgfpoint{21.723081\du}{8.998912\du}}
\pgfpathlineto{\pgfpoint{21.723081\du}{10.850000\du}}
\pgfusepath{stroke}
\pgfsetlinewidth{0.030000\du}
\pgfsetbuttcap
\pgfsetmiterjoin
\pgfsetdash{}{0pt}
\definecolor{dialinecolor}{rgb}{0.925490, 0.925490, 0.905882}
\pgfsetstrokecolor{dialinecolor}
\pgfpathmoveto{\pgfpoint{20.642734\du}{10.727925\du}}
\pgfpathlineto{\pgfpoint{21.722754\du}{10.727925\du}}
\pgfusepath{stroke}
\pgfsetbuttcap
\pgfsetmiterjoin
\pgfsetdash{}{0pt}
\definecolor{dialinecolor}{rgb}{0.000000, 0.000000, 0.000000}
\pgfsetstrokecolor{dialinecolor}
\pgfpathmoveto{\pgfpoint{20.642734\du}{9.741834\du}}
\pgfpathlineto{\pgfpoint{21.722754\du}{9.741834\du}}
\pgfusepath{stroke}
\pgfsetbuttcap
\pgfsetmiterjoin
\pgfsetdash{}{0pt}
\definecolor{dialinecolor}{rgb}{0.286275, 0.286275, 0.211765}
\pgfsetstrokecolor{dialinecolor}
\pgfpathmoveto{\pgfpoint{20.628989\du}{10.714507\du}}
\pgfpathlineto{\pgfpoint{21.721772\du}{10.714507\du}}
\pgfusepath{stroke}
\pgfsetbuttcap
\pgfsetmiterjoin
\pgfsetdash{}{0pt}
\definecolor{dialinecolor}{rgb}{0.000000, 0.000000, 0.000000}
\pgfsetstrokecolor{dialinecolor}
\pgfpathmoveto{\pgfpoint{20.628989\du}{9.728089\du}}
\pgfpathlineto{\pgfpoint{21.721772\du}{9.728089\du}}
\pgfusepath{stroke}
\pgfsetlinewidth{0.001000\du}
\pgfsetbuttcap
\pgfsetmiterjoin
\pgfsetdash{}{0pt}
\definecolor{dialinecolor}{rgb}{0.925490, 0.925490, 0.905882}
\pgfsetstrokecolor{dialinecolor}
\pgfpathmoveto{\pgfpoint{20.696408\du}{9.336336\du}}
\pgfpathlineto{\pgfpoint{20.696408\du}{9.106586\du}}
\pgfpathlineto{\pgfpoint{21.182417\du}{9.106586\du}}
\pgfusepath{stroke}
\pgfsetlinewidth{0.100000\du}
\pgfsetdash{}{0pt}
\pgfsetdash{}{0pt}
\pgfsetbuttcap
\pgfsetmiterjoin
\pgfsetlinewidth{0.001000\du}
\pgfsetbuttcap
\pgfsetmiterjoin
\pgfsetdash{}{0pt}
\definecolor{dialinecolor}{rgb}{0.788235, 0.788235, 0.713726}
\pgfsetfillcolor{dialinecolor}
\pgfpathmoveto{\pgfpoint{22.752053\du}{13.227519\du}}
\pgfpathlineto{\pgfpoint{22.999199\du}{13.000000\du}}
\pgfpathlineto{\pgfpoint{25.247947\du}{13.000000\du}}
\pgfpathlineto{\pgfpoint{25.000401\du}{13.227519\du}}
\pgfpathlineto{\pgfpoint{22.752053\du}{13.227519\du}}
\pgfusepath{fill}
\pgfsetbuttcap
\pgfsetmiterjoin
\pgfsetdash{}{0pt}
\definecolor{dialinecolor}{rgb}{0.286275, 0.286275, 0.211765}
\pgfsetstrokecolor{dialinecolor}
\pgfpathmoveto{\pgfpoint{22.763669\du}{13.217104\du}}
\pgfpathlineto{\pgfpoint{22.989986\du}{13.008412\du}}
\pgfpathlineto{\pgfpoint{25.239135\du}{13.008412\du}}
\pgfpathlineto{\pgfpoint{25.000401\du}{13.227519\du}}
\pgfpathlineto{\pgfpoint{22.763669\du}{13.227519\du}}
\pgfpathlineto{\pgfpoint{22.763669\du}{13.217104\du}}
\pgfusepath{stroke}
\pgfsetbuttcap
\pgfsetmiterjoin
\pgfsetdash{}{0pt}
\definecolor{dialinecolor}{rgb}{0.717647, 0.717647, 0.615686}
\pgfsetfillcolor{dialinecolor}
\pgfpathmoveto{\pgfpoint{22.752053\du}{13.227519\du}}
\pgfpathlineto{\pgfpoint{25.018426\du}{13.227519\du}}
\pgfpathlineto{\pgfpoint{25.018426\du}{15.000000\du}}
\pgfpathlineto{\pgfpoint{22.752053\du}{15.000000\du}}
\pgfpathlineto{\pgfpoint{22.752053\du}{13.227519\du}}
\pgfusepath{fill}
\pgfsetbuttcap
\pgfsetmiterjoin
\pgfsetdash{}{0pt}
\definecolor{dialinecolor}{rgb}{0.286275, 0.286275, 0.211765}
\pgfsetstrokecolor{dialinecolor}
\pgfpathmoveto{\pgfpoint{22.763669\du}{13.227519\du}}
\pgfpathlineto{\pgfpoint{25.017625\du}{13.227519\du}}
\pgfpathlineto{\pgfpoint{25.017625\du}{14.999199\du}}
\pgfpathlineto{\pgfpoint{22.763669\du}{14.999199\du}}
\pgfpathlineto{\pgfpoint{22.763669\du}{13.227519\du}}
\pgfusepath{stroke}
\pgfsetbuttcap
\pgfsetmiterjoin
\pgfsetdash{}{0pt}
\definecolor{dialinecolor}{rgb}{0.478431, 0.478431, 0.352941}
\pgfsetfillcolor{dialinecolor}
\pgfpathmoveto{\pgfpoint{25.000401\du}{14.982776\du}}
\pgfpathlineto{\pgfpoint{25.247947\du}{14.736832\du}}
\pgfpathlineto{\pgfpoint{25.247947\du}{13.000000\du}}
\pgfpathlineto{\pgfpoint{25.000401\du}{13.227519\du}}
\pgfpathlineto{\pgfpoint{25.000401\du}{14.982776\du}}
\pgfusepath{fill}
\pgfsetbuttcap
\pgfsetmiterjoin
\pgfsetdash{}{0pt}
\definecolor{dialinecolor}{rgb}{0.286275, 0.286275, 0.211765}
\pgfsetstrokecolor{dialinecolor}
\pgfpathmoveto{\pgfpoint{25.000401\du}{14.982776\du}}
\pgfpathlineto{\pgfpoint{25.247947\du}{14.736832\du}}
\pgfpathlineto{\pgfpoint{25.247947\du}{13.008412\du}}
\pgfpathlineto{\pgfpoint{25.239135\du}{13.008412\du}}
\pgfpathlineto{\pgfpoint{25.000401\du}{13.227519\du}}
\pgfpathlineto{\pgfpoint{25.000401\du}{14.982776\du}}
\pgfpathlineto{\pgfpoint{25.000401\du}{14.982776\du}}
\pgfusepath{stroke}
\pgfsetlinewidth{0.050000\du}
\pgfsetdash{}{0pt}
\pgfsetdash{}{0pt}
\pgfsetbuttcap
{
\definecolor{dialinecolor}{rgb}{0.000000, 0.000000, 0.000000}
\pgfsetfillcolor{dialinecolor}
% was here!!!
\pgfsetarrowsend{to}
\definecolor{dialinecolor}{rgb}{0.000000, 0.000000, 0.000000}
\pgfsetstrokecolor{dialinecolor}
\draw (11.755164\du,9.918640\du)--(20.628989\du,9.924456\du);
}
\pgfsetlinewidth{0.100000\du}
\pgfsetdash{}{0pt}
\pgfsetdash{}{0pt}
\pgfsetbuttcap
{
\definecolor{dialinecolor}{rgb}{0.000000, 0.000000, 0.000000}
\pgfsetfillcolor{dialinecolor}
% was here!!!
\pgfsetarrowsend{to}
\definecolor{dialinecolor}{rgb}{0.000000, 0.000000, 0.000000}
\pgfsetstrokecolor{dialinecolor}
\draw (21.870751\du,10.650831\du)--(23.500000\du,12.750000\du);
}
\pgfsetlinewidth{0.050000\du}
\pgfsetdash{}{0pt}
\pgfsetdash{}{0pt}
\pgfsetbuttcap
{
\definecolor{dialinecolor}{rgb}{0.000000, 0.000000, 0.000000}
\pgfsetfillcolor{dialinecolor}
% was here!!!
\pgfsetarrowsend{to}
\definecolor{dialinecolor}{rgb}{0.000000, 0.000000, 0.000000}
\pgfsetstrokecolor{dialinecolor}
\draw (22.757095\du,13.994376\du)--(17.400000\du,14.000000\du);
}
\pgfsetlinewidth{0.050000\du}
\pgfsetdash{}{0pt}
\pgfsetdash{}{0pt}
\pgfsetbuttcap
\pgfsetmiterjoin
\pgfsetlinewidth{0.050000\du}
\pgfsetbuttcap
\pgfsetmiterjoin
\pgfsetdash{}{0pt}
\definecolor{dialinecolor}{rgb}{1.000000, 1.000000, 1.000000}
\pgfsetfillcolor{dialinecolor}
\fill (14.682258\du,12.900000\du)--(14.682258\du,15.605000\du)--(17.300000\du,15.605000\du)--(17.300000\du,12.900000\du)--cycle;
\definecolor{dialinecolor}{rgb}{0.000000, 0.000000, 0.000000}
\pgfsetstrokecolor{dialinecolor}
\draw (14.682258\du,12.900000\du)--(14.682258\du,15.605000\du)--(17.300000\du,15.605000\du)--(17.300000\du,12.900000\du)--cycle;
\pgfsetbuttcap
\pgfsetmiterjoin
\pgfsetdash{}{0pt}
\definecolor{dialinecolor}{rgb}{0.000000, 0.000000, 0.000000}
\pgfsetstrokecolor{dialinecolor}
\draw (14.682258\du,12.900000\du)--(14.682258\du,15.605000\du)--(17.300000\du,15.605000\du)--(17.300000\du,12.900000\du)--cycle;
% setfont left to latex
\definecolor{dialinecolor}{rgb}{0.000000, 0.000000, 0.000000}
\pgfsetstrokecolor{dialinecolor}
\node[anchor=west] at (15.239919\du,14.659583\du){html};
\pgfsetlinewidth{0.050000\du}
\pgfsetdash{}{0pt}
\pgfsetdash{}{0pt}
\pgfsetbuttcap
{
\definecolor{dialinecolor}{rgb}{0.000000, 0.000000, 0.000000}
\pgfsetfillcolor{dialinecolor}
% was here!!!
\pgfsetarrowsend{to}
\definecolor{dialinecolor}{rgb}{0.000000, 0.000000, 0.000000}
\pgfsetstrokecolor{dialinecolor}
\draw (14.682258\du,14.252500\du)--(10.500000\du,10.750000\du);
}
% setfont left to latex
\definecolor{dialinecolor}{rgb}{0.000000, 0.000000, 0.000000}
\pgfsetstrokecolor{dialinecolor}
\node[anchor=west] at (8.950000\du,14.350000\du){Navigateur};
% setfont left to latex
\definecolor{dialinecolor}{rgb}{0.000000, 0.000000, 0.000000}
\pgfsetstrokecolor{dialinecolor}
\node[anchor=west] at (10.800000\du,7.600000\du){Client};
% setfont left to latex
\definecolor{dialinecolor}{rgb}{0.000000, 0.000000, 0.000000}
\pgfsetstrokecolor{dialinecolor}
\node[anchor=west] at (20.250000\du,7.800000\du){Serveur};
% setfont left to latex
\definecolor{dialinecolor}{rgb}{0.000000, 0.000000, 0.000000}
\pgfsetstrokecolor{dialinecolor}
\node[anchor=west] at (22.950000\du,15.500000\du){Apache};
% setfont left to latex
\definecolor{dialinecolor}{rgb}{0.000000, 0.000000, 0.000000}
\pgfsetstrokecolor{dialinecolor}
\node[anchor=west] at (19.400000\du,15.200000\du){PHP};
\end{tikzpicture}

		La saisie de formulaire se traite grâce au PHP, cependant leurs création
		reste du HTML.\\
		deux méthodes, GET et POST.\\
		\subparagraph{GET} Accessible par l'URL. L'inconveignant étant la taille
		maximum de caractère (buffer clavier)
		\subparagraph{POST} Accessible grâce aux formulaires. L'inconveignant 
		étant de resaisir les informations en cas d'erreur (sauf si utilisation 
		de PHP ou JavaScript) \\
		Les champs name sont toujours associer à un nom de variable en PHP,
		toujours le renseigner, donner des noms claires.
		
		\subsection{Champ de texte monolignes}
		\lstinputlisting[language=html]{1.html}	
		\subsection{Champ de texte multilignes}
		Textarea, permet d'insérer plusieurs lignes de code, avec éventuellement 
		une scrollbar. (exemple: commentaire, message forum...)
		\lstinputlisting[language=html]{2.html}	
		\subsection{Champ de fichiers}
		Permet de poster un fichier sur le serveur, possibilité de limiter le 
		poids de fichier.\\
		ATTENTIION: Toujours vérifier l'extension de fichier et la taille du fichier!
		\lstinputlisting[language=html]{3.html}	
		\subsection{Champ caché}
		Permet de faire passer un champ que ne voit pas l'utilisateur, récupération 
		de la donnée passé grâce au PHP
		\lstinputlisting[language=html]{4.html}	
		\subsection{Bouton radio}
		Le bouton radio permet à l'utilisateur de ne choisir qu'un choix parmis 
		une liste de choix (exemple Monsieur/Madame)
		\lstinputlisting[language=html]{5.html}	
		\subsection{Bouton simple}
		\lstinputlisting[language=html]{6.html}	
		\subsection{Liste}
		\lstinputlisting[language=html]{7.html}	

	\section{En-tête de document (head)}
	La valise <head> contient de nombreux renseignement sur la page, sur l'auteur 
	ainsi que la plupart des scripts.
	\subsection{Titre}
		\lstinputlisting[language=html]{8.html}	
	\subsection{Insértion JavaScript}
		\lstinputlisting[language=html]{9.html}	
	\subsection{Style CSS}	
		\lstinputlisting[language=html]{10.html}	
	\subsection{Méta-données}
		La plupart sont utilisées pour le référencement.
		\lstinputlisting[language=html, caption="Encodage de caractère]{11.html}	
		\begin{itemize}
		\item meta description: description de la page
		\item meta keyword: mots clefs de la page
		\item meta rating: public visé
		\item meta robots: pours les bots de référencement
		\end{itemize}
		\lstinputlisting[language=html]{12.html}	
	\chapter{CSS}
		CSS =  pour compléter pour le HTML: permet de mettre en forme la page.
		Le HTML ne devrait être utilisé que pour le contenu et le html que pour
		le fond.\\
		\section{Insertion CSS}	
		Trois façon de l'insérer comme dit précédemment. 
		\subsection{FIchier à part}
		\lstinputlisting[language=html]{13.html}	
		\subsection{Dans l'en-tête}
		\lstinputlisting[language=html]{14.html}	
		\subsection{Dans le HTML}
		\lstinputlisting[language=html]{15.html}	
		oijoij
		\section{Syntaxe}
			Le CSS possède sa propre << syntaxe >>.\\
			Dans un CSS, on trouve trois élements différents: 
		%	\begin{itemize}
		%		\item Noms de balise
		%		\item Des propriétés CSS
		%		\item Des valeurs
		%	\end{itemize}
			\subsection{une Feuille de style}
			ijlkjlkj
%		\lstinputlisting[language=HTML]{1.html}	
%		\lstinputlisting[language=HTML]{2.html}	

\end{document}


