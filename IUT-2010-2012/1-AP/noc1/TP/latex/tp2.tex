\documentclass{article}

\usepackage[utf8]{inputenc}
\usepackage[T1]{fontenc}
\usepackage[francais]{babel}
\usepackage[top=2cm, bottom=2cm, left=2cm, right=2cm]{geometry} %marges
\usepackage{fancyhdr}
\usepackage{multido}
\usepackage{amssymb}

\title{TP2}
\date{C$++$ de base}
\newcommand{\Pointilles}[1][3]
{
	\multido{}{#1}{\makebox[\linewidth]{\dotfill}\\[\parskip]}
}
\linespread{1.1}

\lhead{TP2 de C$++$ de base}
\chead{}
\rhead{\thepage}

\lfoot{Université paul sabatier}
\cfoot{\thepage}
\rfoot{Toulouse III}

\pagestyle{fancy}
\begin{document}
	\maketitle 
	Recopier à l'aide d'une commande Shell dans votre répertoire de travail les 3 fichiers erato1.C, erato2.C et calculette.C qui se trouvent dans le répertoire 
	/usr /local/public/TPC+.
	\section*{Exercice 1}
		Modifier le fichier erato1.C pour respecter les conventions d'écriture des programmes C$++$. Indiquer ce que fait le programme sans l'exécuter. Même question pour le fichier erato2.C.
	\section*{Exercice 2}
		Compiler le programme erato1.C (création du fichier makefile) puis l'exécuter. Même question pour le programme erato2.C.
	\section*{Exercice 3}
		Etudier, compiler et exécuter le fichier calculette.C. Trouver et corriger les erreurs. Modifier le fichier source pour respecter les conventions. Pour trouver les erreurs, on pourra utiliser le débogueur DDD.
	

\end{document}
