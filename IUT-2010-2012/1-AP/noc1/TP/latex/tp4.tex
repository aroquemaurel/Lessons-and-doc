\documentclass{article}

\usepackage[utf8]{inputenc}
\usepackage[T1]{fontenc}
\usepackage[francais]{babel}
\usepackage[top=2cm, bottom=2cm, left=2cm, right=2cm]{geometry} %marges
\usepackage{fancyhdr}
\usepackage{multido}
\usepackage{amssymb}

\title{TP4\\ Caractères et chaînes de caractères}
\date{C$++$ de base}
\newcommand{\Pointilles}[1][3]
{
	\multido{}{#1}{\makebox[\linewidth]{\dotfill}\\[\parskip]}
}
\linespread{1.1}

\lhead{TP4 de C$++$ de base}
\chead{}
\rhead{\thepage}

\lfoot{Université paul sabatier}
\cfoot{\thepage}
\rfoot{Toulouse III}

\pagestyle{fancy}
\begin{document}
	\maketitle 
	\section*{Exercice 1}
		Ecrire un programme qui demande à un utilisateur de saisir un caractère et qui affiche si c'est un caractère minuscule, majuscule ou un chiffre
		\begin{itemize}
			\item La fonction estMinuscule prend un caractère en entrée et renvoie un booléen. Même chose pour les fonctions estMajuscule et estChiffre
			\item Les trois fonctions précédentes s'écrivent en une seule instruction !
		\end{itemize}
		Etendre les capacités de votre programme de façon à ce que l'utilisateur puisse saisir une suite de caractères terminée par le caractère '.', pour ce faire, vous utiliserez une boucle de la forme while() {} en se remémorant l'algorithme de saisie d'une suite de valeurs terminée par une valeur marqueur.
	\section*{Exercice 2}
		Ecrire un programme qui demande à un utilisateur de saisir un caractère et qui affiche son code en ASCII, puis qui demande à un utilisateur de saisir un code ASCII sous forme entière et affiche le caractère correspondant
		\begin{itemize}
			\item Les codes des conversions entre type caractère et type entier ne doivent pas dépasser une ligne de C++.
		\end{itemize}		
	\section*{Exercice 3}
		Ecrire une fonction changerMinusculesMajuscules qui a un paramètre de type chaîne de caractères (en mise à jour) et qui transforme toutes les minuscules de la chaîne en majuscules et toutes les majuscules en minuscules. Tester en demandant à un utilisateur de saisir une chaîne et afficher alors la chaîne modifiée.
		\begin{itemize}
			\item Vous utiliserez le type Chaine de chaine.h
			\item Les traitements sur les caractères de la chaîne seront encapsulés dans les fonctions minusculeEnMajuscule et majusculeEnMinuscule
			\item Les chaînes de caractères sont indicées de 0 à longueur(chaine) -1			
			\item Vous utiliserez une boucle for pour traiter tous les caractères de la chaîne.			
		\end{itemize}		
	\section*{Exercice 4}
		Ecrire un programme qui demande à l'utilisateur de taper une chaîne de caractères et qui affiche la ou les lettres les plus fréquentes.	
	\section*{Exercice 5}		
		Ecrire une fonction qui prend en paramètres deux chaînes de caractères ch1 et ch2 (paramètres en entrée) et renvoie un booléen indiquant si la chaîne ch2 est contenue dans la chaîne ch1.
\end{document}
