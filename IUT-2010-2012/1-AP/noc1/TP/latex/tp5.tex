\documentclass{article}

\usepackage[utf8]{inputenc}
\usepackage[T1]{fontenc}
\usepackage[francais]{babel}
\usepackage[top=2cm, bottom=2cm, left=2cm, right=2cm]{geometry} %marges
\usepackage{fancyhdr}
\usepackage{multido}
\usepackage{amssymb}

\title{TP5\\ Les instructions alternatives}
\date{C$++$ de base}
\newcommand{\Pointilles}[1][3]
{
	\multido{}{#1}{\makebox[\linewidth]{\dotfill}\\[\parskip]}
}
\linespread{1.1}

\lhead{TP5 de C$++$ de base}
\chead{}
\rhead{\thepage}

\lfoot{Université paul sabatier}
\cfoot{\thepage}
\rfoot{Toulouse III}

\pagestyle{fancy}
\begin{document}
	\maketitle 
	\section*{Exercice 1}
		Ecrire un programme qui demande à un utilisateur de saisir 4 entiers A, B, C et D, représentant respectivement les intervalles d'entiers [A,B] et [C,D] puis qui affiche quelle est l'intersection des intervalles [AB] et [CD].

		\begin{itemize}
			\item Déclarez et utilisez une structure Intervalle comprenant deux entiers inf et sup.
			\item Ecrire une procédure chargée de la saisie d'un intervalle.
			\item Ecrire une procédure chargée de l'affichage d'un intervalle.
			\item Ecrire une fonction intersection prenant deux intervalles en entrée et renvoyant comme résultat l'intervalle représentant l'intersection.
			\item La constante INTERVALLE\_VIDE égale à [0,0] représentera l'intersection vide.			
		\end{itemize}
	\section*{Exercice 2}
		Sachant que : l'eau gèle à $0^{\circ} C$, le fuel gèle à $-5^{\circ} C$, le super gèle à $-23^{\circ} C$, l'ordinaire gèle à $-13^{\circ} C$ et l'eau salée gèle à $-3^{\circ} C$. 
		Ecrire un programme qui demande à un utilisateur de saisir une température et qui affiche la liste des liquides gelés à cette température.
	\section*{Exercice 3}
		Ecrire un programme qui gère un menu. L'affichage de ce menu comporte 4 choix d'actions possible. La dernière de ces actions est la sortie du programme. A chaque action sélectionnée par un utilisateur, 
		il est affiché le nom de l'action considérée, par exemple action1. Puis l'utilisateur tape sur la touche entrée et le menu est de nouveau affiché si le choix n'est pas la sortie du programme.

		\begin{itemize}
			\item La gestion des actions en fonction des choix effectués par l'utilisateur se programmera à l'aide d'une instruction switch(choix).
			\item L'affichage du menu se répètera tant que l'utilisateur n'a pas décidé de sortir du programme.
			\item La gestion des erreurs de saisie sera prise en compte. On supposera que l'utilisateur saisit toujours un entier.	
		\end{itemize}		
		\newpage
	\section*{Exercice 4}
		Une compagnie d'assurance automobile propose à ses clients quatre familles de tarifs identifiables par une couleur, du moins au plus onéreux : tarifs bleu, vert, orange et rouge. Ecrire un programme permettant 
		de saisir les données nécessaires d'un assuré et de lui indiquer son acceptation ou refus ainsi que la famille de tarifs à laquelle il peut prétendre. Le tarif dépend de la situation du conducteur :
		\begin{itemize}
			\item Un conducteur de moins de 25 ans et titulaire du permis depuis moins de deux ans, se voit attribuer le tarif rouge, si toutefois il n'a jamais été responsable d'accident. Sinon, la compagnie refuse de l'assurer
			\item Un conducteur de moins de 25 ans et titulaire du permis depuis plus de deux ans, ou de plus de 25 ans mais titulaire du permis depuis moins de deux ans a le droit au tarif orange s'il n'a jamais provoqué 
				d'accident, au tarif rouge pour un accident, sinon la compagnie refuse de l'assurer. 

			\item Un conducteur de plus de 25 ans titulaire du permis depuis plus de deux ans bénéficie du tarif vert s'il n'est à l'origine d'aucun accident et du tarif orange pour un accident, du tarif rouge pour deux accidents,
					 et refusé au-delà.
			\item De plus, pour encourager la fidélité des clients acceptés, la compagnie propose un contrat de la couleur immédiatement la plus avantageuse s'il est entré dans la maison depuis plus d'un an.

		\end{itemize}		
	\section*{Exercice 5}		
		Ecrire un programme qui résout l’équation ax2+bx+c=0 en envisageant tous les cas particuliers. Vous devez permettre à un utilisateur de saisir les coeficients d'un polynome du second degré, et vous 
		devez afficher l'ensemble des solutions avec une explication
		\begin{itemize}
			\item Que se passe-t'il lorsque au moins l'un des coefficients est nul ?
			\item Rappel : dans le cas général, il faut calculer le discriminant delta égal à b2 - 4ac. 
			\item Si le discriminant est < 0 alors pas de solution, la racine carrée d'un nombre négatif n'étant pas défini pour l'ensemble des nombres réels. 
			\item Si le discriminant est égal à 0, alors l'équation admet une racine double réelle égale à -b/2a.
			\item Si le discriminant est > 0, alors l'équation admet deux solutions réelles distinctes : -b + racine(delta)/2a et -b - racine(delta)/2a.
			\item Attention aux divisions par zéro !
			\item Vous utiliserez la bibliothèque <cmath> et la fonction de prototype double sqrt(const double a).									

		\end{itemize}		
\end{document}
