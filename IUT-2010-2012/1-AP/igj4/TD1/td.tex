\documentclass{article}

\usepackage{lmodern}
\usepackage{xcolor}
\usepackage[utf8]{inputenc}
\usepackage[T1]{fontenc}
\usepackage[francais]{babel}
\usepackage[top=1.7cm, bottom=1.7cm, left=1.7cm, right=1.7cm]{geometry}
\usepackage{verbatim}
\usepackage{tikz} %Vectoriel
\usepackage{listings}
\usepackage{fancyhdr}
\usepackage{multido}
\usepackage{amssymb}

\newcommand{\titre}{Interface graphique et événements}
\newcommand{\numTD}{1}

\newcommand{\module}{Interface graphique en Java}
\newcommand{\sigle}{igj}

\newcommand{\semestre}{4}

\input{/home/satenske/cours/listings.tex} %prise en charge du langage algo

\usepackage{ifthen}
\date{\today}

\chead{}
\rhead{TD\no\numTD}
\lhead{\titre}
%\makeindex

\lfoot{Université Paul Sabatier Toulouse III}
\rfoot{\sigle\semestre}
%\rfoot{}
\cfoot{--~~\thepage~~--}

\makeglossary
\makeatletter
\def\clap#1{\hbox to 0pt{\hss #1\hss}}%

\def\haut#1#2#3{%
	\hbox to \hsize{%
		\rlap{\vtop{\raggedright #1}
	}%
	\hss
	\clap{\vtop{\centering #2}
}%
\hss
\llap{\vtop{\raggedleft #3}}}}%
\def\bas#1#2#3{%
	\hbox to \hsize{%
		\rlap{\vbox{
			\raggedright #1
		}
	}%
	\hss \clap{\vbox{\centering #2}}%
	\hss
	\llap{\vbox{\raggedleft #3}}}
}%
\def\maketitle{%
	\thispagestyle{empty}{%
		\haut{}{\@blurb}{}
		%	
		%\vfill

		\begin{center}
			\vspace{-1.5cm}
			\usefont{OT1}{ptm}{m}{n}
			\huge \@numeroTD \@title
		\end{center}
		\par
		\hrule height 1pt
		\par
		\vspace{1cm}
		\bas{}{}{}
}%
}
\def\date#1{\def\@date{#1}}
\def\author#1{\def\@author{#1}}
\def\numeroTD#1{\def\@numeroTD{#1}}
\def\title#1{\def\@title{#1}}
\def\location#1{\def\@location{#1}}
\def\blurb#1{\def\@blurb{#1}}
\date{\today}
\newboolean{monBool}
\setboolean{monBool}{true}
\author{}
\title{}
\ifthenelse{\equal{\numTD}{}}{
\numeroTD{}
}
{
	\numeroTD{TD \no\numTD~--- }
}
\location{Amiens}\blurb{}
%\makeatother
\title{\titre}
\author{%Semestre \semestre
}

\location{Toulouse}
\blurb{%
\vspace{-35px}
\begin{flushleft}
	Université Paul Sabatier -- Toulouse III\\
	IUT A - Toulouse Rangueil\\
\end{flushleft}
\begin{flushright}
	\vspace{-45px}
	\Large \textbf \module \\
	\normalsize \textit \today\\
	Semestre \semestre
	\vspace{30px}
\end{flushright}
}%



%\title{Cours \\ \titre}
%\date{\today\\ Semestre \semestre}

%\lhead{Cours: \titre}
%\chead{}
%\rhead{\thepage}

%\lfoot{Université Paul Sabatier Toulouse III}
%\cfoot{\thepage}
%\rfoot{\sigle\semestre}

\pagestyle{fancy}


\begin{document}
	\maketitle
	Pour réaliser des interfaces graphiques, nous allons utiliser une architecture MVC\footnote{Modèle Vue Contrôleur}:
	\begin{itemize}
		\item La vue s'occupera de l'affichage de notre application
		\item Le modèle contient les méthodes utiles pour le fonctionnement de l'application en elle même
		\item Le contrôleur liera la vue avec le modèle
	\end{itemize}
	Cette architecture nous permet de pouvoir séparer les différentes choses simplement. Mais Également, cela nous permet de pouvoir éventuellement changer l'aspect visuel sans modifier le fonctionnement du jeu (le modèle).
	\section{Construction du modèle -- Classe \texttt{ModelMorpion}}
	\lstinputlisting[language=Java, caption=Classe \texttt{ModeleMorpion}]{1.java}
	\section{Construction de la vue -- Classe \texttt{VueMorpion}}
	\lstinputlisting[language=Java, caption=Classe \texttt{VueMorpion}]{2.java}
	\newpage
	\section{Construction des contrôleurs}
		\subsection{Classe \texttt{controleurBtGrille}}
	\lstinputlisting[language=Java, caption=Classe \texttt{VueMorpion}]{2-1.java}
		\subsection{Classe \texttt{contoleurBteffacer}}
	\lstinputlisting[language=Java, caption=Classe \texttt{VueMorpion}]{2-1.java}
	\section{Amélioration -- Centralisation des traitements}
	\subsection{Traitement 1}
	\begin{itemize}
		\item Initialiser l'étiquette joueur courant à joueur 1
		\item Initialiser la grille (boutons et modèle)
		\item Initialiser à vide le libellé du résultat
	\end{itemize}
	\subsection{Traitement 2}
	\begin{itemize}
		\item Positionner ``X'' dans le bouton cliqué
		\item Positionner ``JOUEUR1'' dans la grille du modèle
		\item Si le joueur à gagné ou non :  
\begin{lstlisting}[language=Algo, numbers=none]
si 3 cases alignees alors
	mettre à jour le résultat;
sinon
	positionner dans l'étiquette le joueur 2 au joueur;
fin si;
\end{lstlisting}
	\end{itemize}
	\subsection{Traitement 3}
	\begin{itemize}
		\item Positionner ``O'' dans le bouton cliqué
		\item Positionner ``JOUEUR2'' dans la grille du modèle
		\item Si le joueur à gagné ou non :
			
\begin{lstlisting}[language=Algo, numbers=none]
si 3 cases alignees alors
	mettre à jour le résultat;
sinon
	positionner dans l'étiquette le joueur 1 au joueur;
fin si;
\end{lstlisting}
	\end{itemize}

	\subsection{Améliorations du code}
	\lstinputlisting[language=Java, caption=Améliorations dans la classe \texttt{VueModele}]{3.java}
	\section{Application du jeu de Morpion}
	\lstinputlisting[language=Java, caption=Classe \texttt{Application}]{4.java}
	\section{Applet du jeu de Morpion}
	\subsection{Applet}
	\lstinputlisting[language=Java, caption=Classe \texttt{AppletMorpion}]{5.java}
	\subsection{Page \bsc{HTML}}
	\lstinputlisting[language=HTML, caption=Page \bsc{HTML}]{6.html}
	\newpage
	\appendix
	\section{Le code complet}
	\subsection{Enumeration \texttt{TypeCase}}
	\lstinputlisting[language=Java, caption=Énumeration \texttt{TypeCase}]{TypeCase.java}
	\subsection{Classe \texttt{ModelMorpion}}
	\lstinputlisting[language=Java, caption=Classe \texttt{ModeleMorpion}]{ModeleMorpion.java}
	\subsection{Classe \texttt{VueMorpion}}
	\lstinputlisting[language=Java, caption=Classe \texttt{VueMorpion}]{3.java}


\end{document}


