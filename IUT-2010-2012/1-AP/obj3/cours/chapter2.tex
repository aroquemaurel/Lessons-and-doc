\chapter{L'héritage de classes et la composition d'objets}
Deux relations entre classes
\begin{itemize}
	\item La relation ``être''(héritage)
m	\item La relation ``avoir'' ou ``posseder'' ($\Rightarrow$ composition d'objets)
\end{itemize}
\section{L'héritage la relation ``être''}
	\paragraph{Définition} Une classe partage ses propriétés (attributs + méthodes) 
	avec une autre classe apelée super-classe. Relation entre la classe et 
	sa super-classe.
	\paragraph{Exemple} Les objets \textit{laLacoste} et \textit{laBadoit} partagent des propriétés communes
	(désignation, prixHT, getDesignation, \ldots) mais avec des spécifités 
	(couleur et taille pour \textit{laLacoste}, prixDeVente pour \textit{laBadoit} (TVA=5.5\%), \ldots) 
	\subsection{Deux modes d'héritages}
	\begin{itemize}
		\item Le mode par \textbf{enrichissement}: ajout d'attributs et/ou méthodes à la 
			super-classe (appelée aussi classe ancêtre)
		\lstinputlisting[language=java]{ex4.java}	
		\newpage
		\item Le mode par substitution redéfinition dans une sous-classe d'une méthode héritée 
		\lstinputlisting[language=java]{ex5.java}	
	\end{itemize}
	A partir des deux classes Chemise et EauGazeuse, on peut créer les objets 
	laLacoste et laBadoit.
	\lstinputlisting[language=java]{ex6.java}	
	\subsubsection{Remarques}
	\begin{itemize}
		\item Dans une même classe on peut hériter par enrichissement et par 
			substitution
		\item Pour un eau gazeuse, une seule définition de la méthode prixDeVente, 
			celle redéfinie dans la classe chemise (elle masque la méthode héritée)
		\item Dans la classe Chemise l'accès au prixHT par le selecteur getPrixHT 
			d'Article (rincipe d'encapsulation)
	\end{itemize}
\subsection{Sémantique de l'héritage}
\begin{itemize}
	\item Pas de sémantique formelle!  En général hériter = ``être une sorte de''
		\textbf{Attention}: héritage $\neq$ partage de code
\end{itemize}
\subsubsection{Point de vue extension}
Une classe est un ensemble d'objet\\
Si E designe l'ensemble des objets d'une classe alors si B hérite de A (mot clé extends)\\
$$E(B) \in E(A)$$
\paragraph{Remarque} L'ensemble des chemises est un inclus dans l'ensemble des articles
\subsubsection{Point de vue intention}
Une classe est un ensemble de proriétés (attributs + méthodes)\\
si I désigne l'intention d'une classe alors si B hérité de A 
on a $I(A) \in I(B)$
\paragraph{Remarque}Une chemise possède les propriétés d'un article
\paragraph{Remarque} Avec ces deux points de vue complémentaire: \\
	Si B hérite de A, alors B est considéré sous-type de A (d'après 
	classe ~= type) \\
	Conséquence: Tout objet déclaré de type A pourra mémoriser
	dynamiquement un objet de type B (polymorphisme):
	\subsection{Le graphe d'héritage}
	Représentation graphique de la relation d'héritage entre classe
	(cf diagramme de classe en UML par exemple)
	\paragraph{Exemple}Classes Article et ses dérivés\\
%			% Graphic for TeX using PGF
% Title: /home/satenske/cours/AP/obj3/uml2.dia
% Creator: Dia v0.97.1
% CreationDate: Thu Sep 22 09:54:15 2011
% For: satenske
% \usepackage{tikz}
% The following commands are not supported in PSTricks at present
% We define them conditionally, so when they are implemented,
% this pgf file will use them.
\ifx\du\undefined
  \newlength{\du}
\fi
\setlength{\du}{15\unitlength}
\begin{tikzpicture}
\pgftransformxscale{1.000000}
\pgftransformyscale{-1.000000}
\definecolor{dialinecolor}{rgb}{0.000000, 0.000000, 0.000000}
\pgfsetstrokecolor{dialinecolor}
\definecolor{dialinecolor}{rgb}{1.000000, 1.000000, 1.000000}
\pgfsetfillcolor{dialinecolor}
\pgfsetlinewidth{0.020000\du}
\pgfsetdash{}{0pt}
\definecolor{dialinecolor}{rgb}{1.000000, 1.000000, 1.000000}
\pgfsetfillcolor{dialinecolor}
\fill (5.600000\du,2.600000\du)--(5.600000\du,4.000000\du)--(12.260000\du,4.000000\du)--(12.260000\du,2.600000\du)--cycle;
\definecolor{dialinecolor}{rgb}{0.000000, 0.000000, 0.000000}
\pgfsetstrokecolor{dialinecolor}
\draw (5.600000\du,2.600000\du)--(5.600000\du,4.000000\du)--(12.260000\du,4.000000\du)--(12.260000\du,2.600000\du)--cycle;
% setfont left to latex
\definecolor{dialinecolor}{rgb}{0.000000, 0.000000, 0.000000}
\pgfsetstrokecolor{dialinecolor}
\node at (8.930000\du,3.550000\du){Article};
\definecolor{dialinecolor}{rgb}{1.000000, 1.000000, 1.000000}
\pgfsetfillcolor{dialinecolor}
\fill (5.600000\du,4.000000\du)--(5.600000\du,5.000000\du)--(12.260000\du,5.000000\du)--(12.260000\du,4.000000\du)--cycle;
\definecolor{dialinecolor}{rgb}{0.000000, 0.000000, 0.000000}
\pgfsetstrokecolor{dialinecolor}
\draw (5.600000\du,4.000000\du)--(5.600000\du,5.000000\du)--(12.260000\du,5.000000\du)--(12.260000\du,4.000000\du)--cycle;
% setfont left to latex
\definecolor{dialinecolor}{rgb}{0.000000, 0.000000, 0.000000}
\pgfsetstrokecolor{dialinecolor}
\node[anchor=west] at (5.710000\du,4.700000\du){-dateDeLivraison};
\definecolor{dialinecolor}{rgb}{1.000000, 1.000000, 1.000000}
\pgfsetfillcolor{dialinecolor}
\fill (5.600000\du,5.000000\du)--(5.600000\du,6.000000\du)--(12.260000\du,6.000000\du)--(12.260000\du,5.000000\du)--cycle;
\definecolor{dialinecolor}{rgb}{0.000000, 0.000000, 0.000000}
\pgfsetstrokecolor{dialinecolor}
\draw (5.600000\du,5.000000\du)--(5.600000\du,6.000000\du)--(12.260000\du,6.000000\du)--(12.260000\du,5.000000\du)--cycle;
% setfont left to latex
\definecolor{dialinecolor}{rgb}{0.000000, 0.000000, 0.000000}
\pgfsetstrokecolor{dialinecolor}
\node[anchor=west] at (5.710000\du,5.700000\du){+prixDeVente()};
\pgfsetlinewidth{0.020000\du}
\pgfsetdash{}{0pt}
\definecolor{dialinecolor}{rgb}{1.000000, 1.000000, 1.000000}
\pgfsetfillcolor{dialinecolor}
\fill (12.750000\du,10.200000\du)--(12.750000\du,11.600000\du)--(18.892500\du,11.600000\du)--(18.892500\du,10.200000\du)--cycle;
\definecolor{dialinecolor}{rgb}{0.000000, 0.000000, 0.000000}
\pgfsetstrokecolor{dialinecolor}
\draw (12.750000\du,10.200000\du)--(12.750000\du,11.600000\du)--(18.892500\du,11.600000\du)--(18.892500\du,10.200000\du)--cycle;
% setfont left to latex
\definecolor{dialinecolor}{rgb}{0.000000, 0.000000, 0.000000}
\pgfsetstrokecolor{dialinecolor}
\node at (15.821250\du,11.150000\du){Eau gazeuse};
\definecolor{dialinecolor}{rgb}{1.000000, 1.000000, 1.000000}
\pgfsetfillcolor{dialinecolor}
\fill (12.750000\du,11.600000\du)--(12.750000\du,12.000000\du)--(18.892500\du,12.000000\du)--(18.892500\du,11.600000\du)--cycle;
\definecolor{dialinecolor}{rgb}{0.000000, 0.000000, 0.000000}
\pgfsetstrokecolor{dialinecolor}
\draw (12.750000\du,11.600000\du)--(12.750000\du,12.000000\du)--(18.892500\du,12.000000\du)--(18.892500\du,11.600000\du)--cycle;
\definecolor{dialinecolor}{rgb}{1.000000, 1.000000, 1.000000}
\pgfsetfillcolor{dialinecolor}
\fill (12.750000\du,12.000000\du)--(12.750000\du,13.000000\du)--(18.892500\du,13.000000\du)--(18.892500\du,12.000000\du)--cycle;
\definecolor{dialinecolor}{rgb}{0.000000, 0.000000, 0.000000}
\pgfsetstrokecolor{dialinecolor}
\draw (12.750000\du,12.000000\du)--(12.750000\du,13.000000\du)--(18.892500\du,13.000000\du)--(18.892500\du,12.000000\du)--cycle;
% setfont left to latex
\definecolor{dialinecolor}{rgb}{0.000000, 0.000000, 0.000000}
\pgfsetstrokecolor{dialinecolor}
\node[anchor=west] at (12.860000\du,12.700000\du){+prixDeVente()};
\pgfsetlinewidth{0.020000\du}
\pgfsetdash{}{0pt}
\definecolor{dialinecolor}{rgb}{1.000000, 1.000000, 1.000000}
\pgfsetfillcolor{dialinecolor}
\fill (3.350000\du,10.200000\du)--(3.350000\du,11.600000\du)--(9.240000\du,11.600000\du)--(9.240000\du,10.200000\du)--cycle;
\definecolor{dialinecolor}{rgb}{0.000000, 0.000000, 0.000000}
\pgfsetstrokecolor{dialinecolor}
\draw (3.350000\du,10.200000\du)--(3.350000\du,11.600000\du)--(9.240000\du,11.600000\du)--(9.240000\du,10.200000\du)--cycle;
% setfont left to latex
\definecolor{dialinecolor}{rgb}{0.000000, 0.000000, 0.000000}
\pgfsetstrokecolor{dialinecolor}
\node at (6.295000\du,11.150000\du){Chemise};
\definecolor{dialinecolor}{rgb}{1.000000, 1.000000, 1.000000}
\pgfsetfillcolor{dialinecolor}
\fill (3.350000\du,11.600000\du)--(3.350000\du,12.000000\du)--(9.240000\du,12.000000\du)--(9.240000\du,11.600000\du)--cycle;
\definecolor{dialinecolor}{rgb}{0.000000, 0.000000, 0.000000}
\pgfsetstrokecolor{dialinecolor}
\draw (3.350000\du,11.600000\du)--(3.350000\du,12.000000\du)--(9.240000\du,12.000000\du)--(9.240000\du,11.600000\du)--cycle;
\definecolor{dialinecolor}{rgb}{1.000000, 1.000000, 1.000000}
\pgfsetfillcolor{dialinecolor}
\fill (3.350000\du,12.000000\du)--(3.350000\du,13.000000\du)--(9.240000\du,13.000000\du)--(9.240000\du,12.000000\du)--cycle;
\definecolor{dialinecolor}{rgb}{0.000000, 0.000000, 0.000000}
\pgfsetstrokecolor{dialinecolor}
\draw (3.350000\du,12.000000\du)--(3.350000\du,13.000000\du)--(9.240000\du,13.000000\du)--(9.240000\du,12.000000\du)--cycle;
% setfont left to latex
\definecolor{dialinecolor}{rgb}{0.000000, 0.000000, 0.000000}
\pgfsetstrokecolor{dialinecolor}
\node[anchor=west] at (3.460000\du,12.700000\du){+prixDeVente()};
\pgfsetlinewidth{0.020000\du}
\pgfsetdash{}{0pt}
\definecolor{dialinecolor}{rgb}{1.000000, 1.000000, 1.000000}
\pgfsetfillcolor{dialinecolor}
\fill (16.850000\du,6.350000\du)--(20.800000\du,6.350000\du)--(21.400000\du,6.950000\du)--(21.400000\du,8.003533\du)--(16.850000\du,8.003533\du)--cycle;
\definecolor{dialinecolor}{rgb}{0.000000, 0.000000, 0.000000}
\pgfsetstrokecolor{dialinecolor}
\draw (16.850000\du,6.350000\du)--(20.800000\du,6.350000\du)--(21.400000\du,6.950000\du)--(21.400000\du,8.003533\du)--(16.850000\du,8.003533\du)--cycle;
\pgfsetlinewidth{0.010000\du}
\definecolor{dialinecolor}{rgb}{0.000000, 0.000000, 0.000000}
\pgfsetstrokecolor{dialinecolor}
\draw (20.800000\du,6.350000\du)--(20.800000\du,6.950000\du)--(21.400000\du,6.950000\du);
% setfont left to latex
\definecolor{dialinecolor}{rgb}{0.000000, 0.000000, 0.000000}
\pgfsetstrokecolor{dialinecolor}
\node[anchor=west] at (17.160000\du,7.240000\du){hérite de };
% setfont left to latex
\definecolor{dialinecolor}{rgb}{0.000000, 0.000000, 0.000000}
\pgfsetstrokecolor{dialinecolor}
\node[anchor=west] at (17.160000\du,7.616767\du){("est une sorte de")};
\pgfsetlinewidth{0.100000\du}
\pgfsetdash{}{0pt}
\pgfsetmiterjoin
\pgfsetbuttcap
{
\definecolor{dialinecolor}{rgb}{0.000000, 0.000000, 0.000000}
\pgfsetfillcolor{dialinecolor}
% was here!!!
\definecolor{dialinecolor}{rgb}{0.000000, 0.000000, 0.000000}
\pgfsetstrokecolor{dialinecolor}
\draw (8.930000\du,6.000000\du)--(8.930000\du,7.550000\du)--(6.295000\du,7.550000\du)--(6.295000\du,10.200000\du);
}
\definecolor{dialinecolor}{rgb}{0.000000, 0.000000, 0.000000}
\pgfsetstrokecolor{dialinecolor}
\draw (8.930000\du,6.911803\du)--(8.930000\du,7.550000\du)--(6.295000\du,7.550000\du)--(6.295000\du,10.200000\du);
\pgfsetmiterjoin
\definecolor{dialinecolor}{rgb}{1.000000, 1.000000, 1.000000}
\pgfsetfillcolor{dialinecolor}
\fill (9.330000\du,6.911803\du)--(8.930000\du,6.111803\du)--(8.530000\du,6.911803\du)--cycle;
\pgfsetlinewidth{0.100000\du}
\pgfsetdash{}{0pt}
\pgfsetmiterjoin
\definecolor{dialinecolor}{rgb}{0.000000, 0.000000, 0.000000}
\pgfsetstrokecolor{dialinecolor}
\draw (9.330000\du,6.911803\du)--(8.930000\du,6.111803\du)--(8.530000\du,6.911803\du)--cycle;
% setfont left to latex
\pgfsetlinewidth{0.100000\du}
\pgfsetdash{}{0pt}
\pgfsetmiterjoin
\pgfsetbuttcap
{
\definecolor{dialinecolor}{rgb}{0.000000, 0.000000, 0.000000}
\pgfsetfillcolor{dialinecolor}
% was here!!!
\definecolor{dialinecolor}{rgb}{0.000000, 0.000000, 0.000000}
\pgfsetstrokecolor{dialinecolor}
\draw (8.930000\du,6.050000\du)--(8.930000\du,7.535280\du)--(15.821250\du,7.535280\du)--(15.821250\du,10.250000\du);
}
\definecolor{dialinecolor}{rgb}{0.000000, 0.000000, 0.000000}
\pgfsetstrokecolor{dialinecolor}
\draw (8.930000\du,6.961803\du)--(8.930000\du,7.535280\du)--(15.821250\du,7.535280\du)--(15.821250\du,10.250000\du);
\pgfsetmiterjoin
\definecolor{dialinecolor}{rgb}{1.000000, 1.000000, 1.000000}
\pgfsetfillcolor{dialinecolor}
\fill (9.330000\du,6.961803\du)--(8.930000\du,6.161803\du)--(8.530000\du,6.961803\du)--cycle;
\pgfsetlinewidth{0.100000\du}
\pgfsetdash{}{0pt}
\pgfsetmiterjoin
\definecolor{dialinecolor}{rgb}{0.000000, 0.000000, 0.000000}
\pgfsetstrokecolor{dialinecolor}
\draw (9.330000\du,6.961803\du)--(8.930000\du,6.161803\du)--(8.530000\du,6.961803\du)--cycle;
% setfont left to latex
\end{tikzpicture}

% Graphic for TeX using PGF
% Title: /home/satenske/cours/AP/obj3/uml2.dia
% Creator: Dia v0.97.1
% CreationDate: Thu Sep 22 09:54:15 2011
% For: satenske
% \usepackage{tikz}
% The following commands are not supported in PSTricks at present
% We define them conditionally, so when they are implemented,
% this pgf file will use them.
\ifx\du\undefined
  \newlength{\du}
\fi
\setlength{\du}{15\unitlength}
\begin{tikzpicture}
\pgftransformxscale{1.000000}
\pgftransformyscale{-1.000000}
\definecolor{dialinecolor}{rgb}{0.000000, 0.000000, 0.000000}
\pgfsetstrokecolor{dialinecolor}
\definecolor{dialinecolor}{rgb}{1.000000, 1.000000, 1.000000}
\pgfsetfillcolor{dialinecolor}
\pgfsetlinewidth{0.020000\du}
\pgfsetdash{}{0pt}
\definecolor{dialinecolor}{rgb}{1.000000, 1.000000, 1.000000}
\pgfsetfillcolor{dialinecolor}
\fill (5.600000\du,2.600000\du)--(5.600000\du,4.000000\du)--(12.260000\du,4.000000\du)--(12.260000\du,2.600000\du)--cycle;
\definecolor{dialinecolor}{rgb}{0.000000, 0.000000, 0.000000}
\pgfsetstrokecolor{dialinecolor}
\draw (5.600000\du,2.600000\du)--(5.600000\du,4.000000\du)--(12.260000\du,4.000000\du)--(12.260000\du,2.600000\du)--cycle;
% setfont left to latex
\definecolor{dialinecolor}{rgb}{0.000000, 0.000000, 0.000000}
\pgfsetstrokecolor{dialinecolor}
\node at (8.930000\du,3.550000\du){Article};
\definecolor{dialinecolor}{rgb}{1.000000, 1.000000, 1.000000}
\pgfsetfillcolor{dialinecolor}
\fill (5.600000\du,4.000000\du)--(5.600000\du,5.000000\du)--(12.260000\du,5.000000\du)--(12.260000\du,4.000000\du)--cycle;
\definecolor{dialinecolor}{rgb}{0.000000, 0.000000, 0.000000}
\pgfsetstrokecolor{dialinecolor}
\draw (5.600000\du,4.000000\du)--(5.600000\du,5.000000\du)--(12.260000\du,5.000000\du)--(12.260000\du,4.000000\du)--cycle;
% setfont left to latex
\definecolor{dialinecolor}{rgb}{0.000000, 0.000000, 0.000000}
\pgfsetstrokecolor{dialinecolor}
\node[anchor=west] at (5.710000\du,4.700000\du){-dateDeLivraison};
\definecolor{dialinecolor}{rgb}{1.000000, 1.000000, 1.000000}
\pgfsetfillcolor{dialinecolor}
\fill (5.600000\du,5.000000\du)--(5.600000\du,6.000000\du)--(12.260000\du,6.000000\du)--(12.260000\du,5.000000\du)--cycle;
\definecolor{dialinecolor}{rgb}{0.000000, 0.000000, 0.000000}
\pgfsetstrokecolor{dialinecolor}
\draw (5.600000\du,5.000000\du)--(5.600000\du,6.000000\du)--(12.260000\du,6.000000\du)--(12.260000\du,5.000000\du)--cycle;
% setfont left to latex
\definecolor{dialinecolor}{rgb}{0.000000, 0.000000, 0.000000}
\pgfsetstrokecolor{dialinecolor}
\node[anchor=west] at (5.710000\du,5.700000\du){+prixDeVente()};
\pgfsetlinewidth{0.020000\du}
\pgfsetdash{}{0pt}
\definecolor{dialinecolor}{rgb}{1.000000, 1.000000, 1.000000}
\pgfsetfillcolor{dialinecolor}
\fill (12.750000\du,10.200000\du)--(12.750000\du,11.600000\du)--(18.892500\du,11.600000\du)--(18.892500\du,10.200000\du)--cycle;
\definecolor{dialinecolor}{rgb}{0.000000, 0.000000, 0.000000}
\pgfsetstrokecolor{dialinecolor}
\draw (12.750000\du,10.200000\du)--(12.750000\du,11.600000\du)--(18.892500\du,11.600000\du)--(18.892500\du,10.200000\du)--cycle;
% setfont left to latex
\definecolor{dialinecolor}{rgb}{0.000000, 0.000000, 0.000000}
\pgfsetstrokecolor{dialinecolor}
\node at (15.821250\du,11.150000\du){Eau gazeuse};
\definecolor{dialinecolor}{rgb}{1.000000, 1.000000, 1.000000}
\pgfsetfillcolor{dialinecolor}
\fill (12.750000\du,11.600000\du)--(12.750000\du,12.000000\du)--(18.892500\du,12.000000\du)--(18.892500\du,11.600000\du)--cycle;
\definecolor{dialinecolor}{rgb}{0.000000, 0.000000, 0.000000}
\pgfsetstrokecolor{dialinecolor}
\draw (12.750000\du,11.600000\du)--(12.750000\du,12.000000\du)--(18.892500\du,12.000000\du)--(18.892500\du,11.600000\du)--cycle;
\definecolor{dialinecolor}{rgb}{1.000000, 1.000000, 1.000000}
\pgfsetfillcolor{dialinecolor}
\fill (12.750000\du,12.000000\du)--(12.750000\du,13.000000\du)--(18.892500\du,13.000000\du)--(18.892500\du,12.000000\du)--cycle;
\definecolor{dialinecolor}{rgb}{0.000000, 0.000000, 0.000000}
\pgfsetstrokecolor{dialinecolor}
\draw (12.750000\du,12.000000\du)--(12.750000\du,13.000000\du)--(18.892500\du,13.000000\du)--(18.892500\du,12.000000\du)--cycle;
% setfont left to latex
\definecolor{dialinecolor}{rgb}{0.000000, 0.000000, 0.000000}
\pgfsetstrokecolor{dialinecolor}
\node[anchor=west] at (12.860000\du,12.700000\du){+prixDeVente()};
\pgfsetlinewidth{0.020000\du}
\pgfsetdash{}{0pt}
\definecolor{dialinecolor}{rgb}{1.000000, 1.000000, 1.000000}
\pgfsetfillcolor{dialinecolor}
\fill (3.350000\du,10.200000\du)--(3.350000\du,11.600000\du)--(9.240000\du,11.600000\du)--(9.240000\du,10.200000\du)--cycle;
\definecolor{dialinecolor}{rgb}{0.000000, 0.000000, 0.000000}
\pgfsetstrokecolor{dialinecolor}
\draw (3.350000\du,10.200000\du)--(3.350000\du,11.600000\du)--(9.240000\du,11.600000\du)--(9.240000\du,10.200000\du)--cycle;
% setfont left to latex
\definecolor{dialinecolor}{rgb}{0.000000, 0.000000, 0.000000}
\pgfsetstrokecolor{dialinecolor}
\node at (6.295000\du,11.150000\du){Chemise};
\definecolor{dialinecolor}{rgb}{1.000000, 1.000000, 1.000000}
\pgfsetfillcolor{dialinecolor}
\fill (3.350000\du,11.600000\du)--(3.350000\du,12.000000\du)--(9.240000\du,12.000000\du)--(9.240000\du,11.600000\du)--cycle;
\definecolor{dialinecolor}{rgb}{0.000000, 0.000000, 0.000000}
\pgfsetstrokecolor{dialinecolor}
\draw (3.350000\du,11.600000\du)--(3.350000\du,12.000000\du)--(9.240000\du,12.000000\du)--(9.240000\du,11.600000\du)--cycle;
\definecolor{dialinecolor}{rgb}{1.000000, 1.000000, 1.000000}
\pgfsetfillcolor{dialinecolor}
\fill (3.350000\du,12.000000\du)--(3.350000\du,13.000000\du)--(9.240000\du,13.000000\du)--(9.240000\du,12.000000\du)--cycle;
\definecolor{dialinecolor}{rgb}{0.000000, 0.000000, 0.000000}
\pgfsetstrokecolor{dialinecolor}
\draw (3.350000\du,12.000000\du)--(3.350000\du,13.000000\du)--(9.240000\du,13.000000\du)--(9.240000\du,12.000000\du)--cycle;
% setfont left to latex
\definecolor{dialinecolor}{rgb}{0.000000, 0.000000, 0.000000}
\pgfsetstrokecolor{dialinecolor}
\node[anchor=west] at (3.460000\du,12.700000\du){+prixDeVente()};
\pgfsetlinewidth{0.020000\du}
\pgfsetdash{}{0pt}
\definecolor{dialinecolor}{rgb}{1.000000, 1.000000, 1.000000}
\pgfsetfillcolor{dialinecolor}
\fill (16.850000\du,6.350000\du)--(20.800000\du,6.350000\du)--(21.400000\du,6.950000\du)--(21.400000\du,8.003533\du)--(16.850000\du,8.003533\du)--cycle;
\definecolor{dialinecolor}{rgb}{0.000000, 0.000000, 0.000000}
\pgfsetstrokecolor{dialinecolor}
\draw (16.850000\du,6.350000\du)--(20.800000\du,6.350000\du)--(21.400000\du,6.950000\du)--(21.400000\du,8.003533\du)--(16.850000\du,8.003533\du)--cycle;
\pgfsetlinewidth{0.010000\du}
\definecolor{dialinecolor}{rgb}{0.000000, 0.000000, 0.000000}
\pgfsetstrokecolor{dialinecolor}
\draw (20.800000\du,6.350000\du)--(20.800000\du,6.950000\du)--(21.400000\du,6.950000\du);
% setfont left to latex
\definecolor{dialinecolor}{rgb}{0.000000, 0.000000, 0.000000}
\pgfsetstrokecolor{dialinecolor}
\node[anchor=west] at (17.160000\du,7.240000\du){hérite de };
% setfont left to latex
\definecolor{dialinecolor}{rgb}{0.000000, 0.000000, 0.000000}
\pgfsetstrokecolor{dialinecolor}
\node[anchor=west] at (17.160000\du,7.616767\du){("est une sorte de")};
\pgfsetlinewidth{0.100000\du}
\pgfsetdash{}{0pt}
\pgfsetmiterjoin
\pgfsetbuttcap
{
\definecolor{dialinecolor}{rgb}{0.000000, 0.000000, 0.000000}
\pgfsetfillcolor{dialinecolor}
% was here!!!
\definecolor{dialinecolor}{rgb}{0.000000, 0.000000, 0.000000}
\pgfsetstrokecolor{dialinecolor}
\draw (8.930000\du,6.000000\du)--(8.930000\du,7.550000\du)--(6.295000\du,7.550000\du)--(6.295000\du,10.200000\du);
}
\definecolor{dialinecolor}{rgb}{0.000000, 0.000000, 0.000000}
\pgfsetstrokecolor{dialinecolor}
\draw (8.930000\du,6.911803\du)--(8.930000\du,7.550000\du)--(6.295000\du,7.550000\du)--(6.295000\du,10.200000\du);
\pgfsetmiterjoin
\definecolor{dialinecolor}{rgb}{1.000000, 1.000000, 1.000000}
\pgfsetfillcolor{dialinecolor}
\fill (9.330000\du,6.911803\du)--(8.930000\du,6.111803\du)--(8.530000\du,6.911803\du)--cycle;
\pgfsetlinewidth{0.100000\du}
\pgfsetdash{}{0pt}
\pgfsetmiterjoin
\definecolor{dialinecolor}{rgb}{0.000000, 0.000000, 0.000000}
\pgfsetstrokecolor{dialinecolor}
\draw (9.330000\du,6.911803\du)--(8.930000\du,6.111803\du)--(8.530000\du,6.911803\du)--cycle;
% setfont left to latex
\pgfsetlinewidth{0.100000\du}
\pgfsetdash{}{0pt}
\pgfsetmiterjoin
\pgfsetbuttcap
{
\definecolor{dialinecolor}{rgb}{0.000000, 0.000000, 0.000000}
\pgfsetfillcolor{dialinecolor}
% was here!!!
\definecolor{dialinecolor}{rgb}{0.000000, 0.000000, 0.000000}
\pgfsetstrokecolor{dialinecolor}
\draw (8.930000\du,6.050000\du)--(8.930000\du,7.535280\du)--(15.821250\du,7.535280\du)--(15.821250\du,10.250000\du);
}
\definecolor{dialinecolor}{rgb}{0.000000, 0.000000, 0.000000}
\pgfsetstrokecolor{dialinecolor}
\draw (8.930000\du,6.961803\du)--(8.930000\du,7.535280\du)--(15.821250\du,7.535280\du)--(15.821250\du,10.250000\du);
\pgfsetmiterjoin
\definecolor{dialinecolor}{rgb}{1.000000, 1.000000, 1.000000}
\pgfsetfillcolor{dialinecolor}
\fill (9.330000\du,6.961803\du)--(8.930000\du,6.161803\du)--(8.530000\du,6.961803\du)--cycle;
\pgfsetlinewidth{0.100000\du}
\pgfsetdash{}{0pt}
\pgfsetmiterjoin
\definecolor{dialinecolor}{rgb}{0.000000, 0.000000, 0.000000}
\pgfsetstrokecolor{dialinecolor}
\draw (9.330000\du,6.961803\du)--(8.930000\du,6.161803\du)--(8.530000\du,6.961803\du)--cycle;
% setfont left to latex
\end{tikzpicture}

\subsection{Méthodes retardées et classes abstraites}
Une méthode est dite retardée lorsque l'ensemble des sous-classes d'une classe donnée en 
proposent une définition.
\paragraph{Exemple}UML:  \\
% Graphic for TeX using PGF
% Title: /home/satenske/cours/AP/obj3/uml3dia
% Creator: Dia v0.97.1
% CreationDate: Thu Sep 15 10:23:25 2011
% For: satenske
% \usepackage{tikz}
% The following commands are not supported in PSTricks at present
% We define them conditionally, so when they are implemented,
% this pgf file will use them.
\ifx\du\undefined
  \newlength{\du}
\fi
\setlength{\du}{15\unitlength}
\begin{tikzpicture}
\pgftransformxscale{1.000000}
\pgftransformyscale{-1.000000}
\definecolor{dialinecolor}{rgb}{0.000000, 0.000000, 0.000000}
\pgfsetstrokecolor{dialinecolor}
\definecolor{dialinecolor}{rgb}{1.000000, 1.000000, 1.000000}
\pgfsetfillcolor{dialinecolor}
\pgfsetlinewidth{0.020000\du}
\pgfsetdash{}{0pt}
\definecolor{dialinecolor}{rgb}{1.000000, 1.000000, 1.000000}
\pgfsetfillcolor{dialinecolor}
\fill (10.700000\du,2.950000\du)--(10.700000\du,4.350000\du)--(15.042500\du,4.350000\du)--(15.042500\du,2.950000\du)--cycle;
\definecolor{dialinecolor}{rgb}{0.000000, 0.000000, 0.000000}
\pgfsetstrokecolor{dialinecolor}
\draw (10.700000\du,2.950000\du)--(10.700000\du,4.350000\du)--(15.042500\du,4.350000\du)--(15.042500\du,2.950000\du)--cycle;
% setfont left to latex
\definecolor{dialinecolor}{rgb}{0.000000, 0.000000, 0.000000}
\pgfsetstrokecolor{dialinecolor}
\node at (12.871250\du,3.900000\du){Employé};
\definecolor{dialinecolor}{rgb}{1.000000, 1.000000, 1.000000}
\pgfsetfillcolor{dialinecolor}
\fill (10.700000\du,4.350000\du)--(10.700000\du,4.750000\du)--(15.042500\du,4.750000\du)--(15.042500\du,4.350000\du)--cycle;
\definecolor{dialinecolor}{rgb}{0.000000, 0.000000, 0.000000}
\pgfsetstrokecolor{dialinecolor}
\draw (10.700000\du,4.350000\du)--(10.700000\du,4.750000\du)--(15.042500\du,4.750000\du)--(15.042500\du,4.350000\du)--cycle;
\definecolor{dialinecolor}{rgb}{1.000000, 1.000000, 1.000000}
\pgfsetfillcolor{dialinecolor}
\fill (10.700000\du,4.750000\du)--(10.700000\du,5.150000\du)--(15.042500\du,5.150000\du)--(15.042500\du,4.750000\du)--cycle;
\definecolor{dialinecolor}{rgb}{0.000000, 0.000000, 0.000000}
\pgfsetstrokecolor{dialinecolor}
\draw (10.700000\du,4.750000\du)--(10.700000\du,5.150000\du)--(15.042500\du,5.150000\du)--(15.042500\du,4.750000\du)--cycle;
\pgfsetlinewidth{0.020000\du}
\pgfsetdash{}{0pt}
\definecolor{dialinecolor}{rgb}{1.000000, 1.000000, 1.000000}
\pgfsetfillcolor{dialinecolor}
\fill (0.650000\du,10.300000\du)--(0.650000\du,11.700000\du)--(10.975000\du,11.700000\du)--(10.975000\du,10.300000\du)--cycle;
\definecolor{dialinecolor}{rgb}{0.000000, 0.000000, 0.000000}
\pgfsetstrokecolor{dialinecolor}
\draw (0.650000\du,10.300000\du)--(0.650000\du,11.700000\du)--(10.975000\du,11.700000\du)--(10.975000\du,10.300000\du)--cycle;
% setfont left to latex
\definecolor{dialinecolor}{rgb}{0.000000, 0.000000, 0.000000}
\pgfsetstrokecolor{dialinecolor}
\node at (5.812500\du,11.250000\du){Directeur Commercial};
\definecolor{dialinecolor}{rgb}{1.000000, 1.000000, 1.000000}
\pgfsetfillcolor{dialinecolor}
\fill (0.650000\du,11.700000\du)--(0.650000\du,13.500000\du)--(10.975000\du,13.500000\du)--(10.975000\du,11.700000\du)--cycle;
\definecolor{dialinecolor}{rgb}{0.000000, 0.000000, 0.000000}
\pgfsetstrokecolor{dialinecolor}
\draw (0.650000\du,11.700000\du)--(0.650000\du,13.500000\du)--(10.975000\du,13.500000\du)--(10.975000\du,11.700000\du)--cycle;
% setfont left to latex
\definecolor{dialinecolor}{rgb}{0.000000, 0.000000, 0.000000}
\pgfsetstrokecolor{dialinecolor}
\node[anchor=west] at (0.760000\du,12.400000\du){-forfait};
% setfont left to latex
\definecolor{dialinecolor}{rgb}{0.000000, 0.000000, 0.000000}
\pgfsetstrokecolor{dialinecolor}
\node[anchor=west] at (0.760000\du,13.200000\du){-prime};
\definecolor{dialinecolor}{rgb}{1.000000, 1.000000, 1.000000}
\pgfsetfillcolor{dialinecolor}
\fill (0.650000\du,13.500000\du)--(0.650000\du,14.500000\du)--(10.975000\du,14.500000\du)--(10.975000\du,13.500000\du)--cycle;
\definecolor{dialinecolor}{rgb}{0.000000, 0.000000, 0.000000}
\pgfsetstrokecolor{dialinecolor}
\draw (0.650000\du,13.500000\du)--(0.650000\du,14.500000\du)--(10.975000\du,14.500000\du)--(10.975000\du,13.500000\du)--cycle;
% setfont left to latex
\definecolor{dialinecolor}{rgb}{0.000000, 0.000000, 0.000000}
\pgfsetstrokecolor{dialinecolor}
\node[anchor=west] at (0.760000\du,14.200000\du){+salaire()};
\pgfsetlinewidth{0.020000\du}
\pgfsetdash{}{0pt}
\definecolor{dialinecolor}{rgb}{1.000000, 1.000000, 1.000000}
\pgfsetfillcolor{dialinecolor}
\fill (20.810000\du,11.035000\du)--(20.810000\du,12.435000\du)--(25.305000\du,12.435000\du)--(25.305000\du,11.035000\du)--cycle;
\definecolor{dialinecolor}{rgb}{0.000000, 0.000000, 0.000000}
\pgfsetstrokecolor{dialinecolor}
\draw (20.810000\du,11.035000\du)--(20.810000\du,12.435000\du)--(25.305000\du,12.435000\du)--(25.305000\du,11.035000\du)--cycle;
% setfont left to latex
\definecolor{dialinecolor}{rgb}{0.000000, 0.000000, 0.000000}
\pgfsetstrokecolor{dialinecolor}
\node at (23.057500\du,11.985000\du){caissière};
\definecolor{dialinecolor}{rgb}{1.000000, 1.000000, 1.000000}
\pgfsetfillcolor{dialinecolor}
\fill (20.810000\du,12.435000\du)--(20.810000\du,13.435000\du)--(25.305000\du,13.435000\du)--(25.305000\du,12.435000\du)--cycle;
\definecolor{dialinecolor}{rgb}{0.000000, 0.000000, 0.000000}
\pgfsetstrokecolor{dialinecolor}
\draw (20.810000\du,12.435000\du)--(20.810000\du,13.435000\du)--(25.305000\du,13.435000\du)--(25.305000\du,12.435000\du)--cycle;
% setfont left to latex
\definecolor{dialinecolor}{rgb}{0.000000, 0.000000, 0.000000}
\pgfsetstrokecolor{dialinecolor}
\node[anchor=west] at (20.920000\du,13.135000\du){-nbHeures};
\definecolor{dialinecolor}{rgb}{1.000000, 1.000000, 1.000000}
\pgfsetfillcolor{dialinecolor}
\fill (20.810000\du,13.435000\du)--(20.810000\du,13.835000\du)--(25.305000\du,13.835000\du)--(25.305000\du,13.435000\du)--cycle;
\definecolor{dialinecolor}{rgb}{0.000000, 0.000000, 0.000000}
\pgfsetstrokecolor{dialinecolor}
\draw (20.810000\du,13.435000\du)--(20.810000\du,13.835000\du)--(25.305000\du,13.835000\du)--(25.305000\du,13.435000\du)--cycle;
\pgfsetlinewidth{0.020000\du}
\pgfsetdash{}{0pt}
\definecolor{dialinecolor}{rgb}{1.000000, 1.000000, 1.000000}
\pgfsetfillcolor{dialinecolor}
\fill (12.870000\du,10.670000\du)--(12.870000\du,12.070000\du)--(18.225000\du,12.070000\du)--(18.225000\du,10.670000\du)--cycle;
\definecolor{dialinecolor}{rgb}{0.000000, 0.000000, 0.000000}
\pgfsetstrokecolor{dialinecolor}
\draw (12.870000\du,10.670000\du)--(12.870000\du,12.070000\du)--(18.225000\du,12.070000\du)--(18.225000\du,10.670000\du)--cycle;
% setfont left to latex
\definecolor{dialinecolor}{rgb}{0.000000, 0.000000, 0.000000}
\pgfsetstrokecolor{dialinecolor}
\node at (15.547500\du,11.620000\du){Comptable};
\definecolor{dialinecolor}{rgb}{1.000000, 1.000000, 1.000000}
\pgfsetfillcolor{dialinecolor}
\fill (12.870000\du,12.070000\du)--(12.870000\du,13.070000\du)--(18.225000\du,13.070000\du)--(18.225000\du,12.070000\du)--cycle;
\definecolor{dialinecolor}{rgb}{0.000000, 0.000000, 0.000000}
\pgfsetstrokecolor{dialinecolor}
\draw (12.870000\du,12.070000\du)--(12.870000\du,13.070000\du)--(18.225000\du,13.070000\du)--(18.225000\du,12.070000\du)--cycle;
% setfont left to latex
\definecolor{dialinecolor}{rgb}{0.000000, 0.000000, 0.000000}
\pgfsetstrokecolor{dialinecolor}
\node[anchor=west] at (12.980000\du,12.770000\du){-forfait};
\definecolor{dialinecolor}{rgb}{1.000000, 1.000000, 1.000000}
\pgfsetfillcolor{dialinecolor}
\fill (12.870000\du,13.070000\du)--(12.870000\du,14.070000\du)--(18.225000\du,14.070000\du)--(18.225000\du,13.070000\du)--cycle;
\definecolor{dialinecolor}{rgb}{0.000000, 0.000000, 0.000000}
\pgfsetstrokecolor{dialinecolor}
\draw (12.870000\du,13.070000\du)--(12.870000\du,14.070000\du)--(18.225000\du,14.070000\du)--(18.225000\du,13.070000\du)--cycle;
% setfont left to latex
\definecolor{dialinecolor}{rgb}{0.000000, 0.000000, 0.000000}
\pgfsetstrokecolor{dialinecolor}
\node[anchor=west] at (12.980000\du,13.770000\du){+salaire()};
\pgfsetlinewidth{0.100000\du}
\pgfsetdash{}{0pt}
\pgfsetmiterjoin
\pgfsetbuttcap
{
\definecolor{dialinecolor}{rgb}{0.000000, 0.000000, 0.000000}
\pgfsetfillcolor{dialinecolor}
% was here!!!
\definecolor{dialinecolor}{rgb}{0.000000, 0.000000, 0.000000}
\pgfsetstrokecolor{dialinecolor}
\draw (12.871250\du,5.150000\du)--(12.871250\du,7.600000\du)--(6.050000\du,7.600000\du)--(6.050000\du,10.050000\du);
}
\definecolor{dialinecolor}{rgb}{0.000000, 0.000000, 0.000000}
\pgfsetstrokecolor{dialinecolor}
\draw (12.871250\du,6.061803\du)--(12.871250\du,7.600000\du)--(6.050000\du,7.600000\du)--(6.050000\du,10.050000\du);
\pgfsetmiterjoin
\definecolor{dialinecolor}{rgb}{1.000000, 1.000000, 1.000000}
\pgfsetfillcolor{dialinecolor}
\fill (13.271250\du,6.061803\du)--(12.871250\du,5.261803\du)--(12.471250\du,6.061803\du)--cycle;
\pgfsetlinewidth{0.100000\du}
\pgfsetdash{}{0pt}
\pgfsetmiterjoin
\definecolor{dialinecolor}{rgb}{0.000000, 0.000000, 0.000000}
\pgfsetstrokecolor{dialinecolor}
\draw (13.271250\du,6.061803\du)--(12.871250\du,5.261803\du)--(12.471250\du,6.061803\du)--cycle;
% setfont left to latex
\pgfsetlinewidth{0.100000\du}
\pgfsetdash{}{0pt}
\pgfsetmiterjoin
\pgfsetbuttcap
{
\definecolor{dialinecolor}{rgb}{0.000000, 0.000000, 0.000000}
\pgfsetfillcolor{dialinecolor}
% was here!!!
\definecolor{dialinecolor}{rgb}{0.000000, 0.000000, 0.000000}
\pgfsetstrokecolor{dialinecolor}
\draw (13.681250\du,5.185000\du)--(13.681250\du,7.842500\du)--(15.000000\du,7.842500\du)--(15.000000\du,10.500000\du);
}
\definecolor{dialinecolor}{rgb}{0.000000, 0.000000, 0.000000}
\pgfsetstrokecolor{dialinecolor}
\draw (13.681250\du,6.096803\du)--(13.681250\du,7.842500\du)--(15.000000\du,7.842500\du)--(15.000000\du,10.500000\du);
\pgfsetmiterjoin
\definecolor{dialinecolor}{rgb}{1.000000, 1.000000, 1.000000}
\pgfsetfillcolor{dialinecolor}
\fill (14.081250\du,6.096803\du)--(13.681250\du,5.296803\du)--(13.281250\du,6.096803\du)--cycle;
\pgfsetlinewidth{0.100000\du}
\pgfsetdash{}{0pt}
\pgfsetmiterjoin
\definecolor{dialinecolor}{rgb}{0.000000, 0.000000, 0.000000}
\pgfsetstrokecolor{dialinecolor}
\draw (14.081250\du,6.096803\du)--(13.681250\du,5.296803\du)--(13.281250\du,6.096803\du)--cycle;
% setfont left to latex
\pgfsetlinewidth{0.100000\du}
\pgfsetdash{}{0pt}
\pgfsetmiterjoin
\pgfsetbuttcap
{
\definecolor{dialinecolor}{rgb}{0.000000, 0.000000, 0.000000}
\pgfsetfillcolor{dialinecolor}
% was here!!!
\definecolor{dialinecolor}{rgb}{0.000000, 0.000000, 0.000000}
\pgfsetstrokecolor{dialinecolor}
\draw (14.660000\du,5.285000\du)--(14.660000\du,7.600000\du)--(21.850000\du,7.600000\du)--(21.850000\du,10.800000\du);
}
\definecolor{dialinecolor}{rgb}{0.000000, 0.000000, 0.000000}
\pgfsetstrokecolor{dialinecolor}
\draw (14.660000\du,6.196803\du)--(14.660000\du,7.600000\du)--(21.850000\du,7.600000\du)--(21.850000\du,10.800000\du);
\pgfsetmiterjoin
\definecolor{dialinecolor}{rgb}{1.000000, 1.000000, 1.000000}
\pgfsetfillcolor{dialinecolor}
\fill (15.060000\du,6.196803\du)--(14.660000\du,5.396803\du)--(14.260000\du,6.196803\du)--cycle;
\pgfsetlinewidth{0.100000\du}
\pgfsetdash{}{0pt}
\pgfsetmiterjoin
\definecolor{dialinecolor}{rgb}{0.000000, 0.000000, 0.000000}
\pgfsetstrokecolor{dialinecolor}
\draw (15.060000\du,6.196803\du)--(14.660000\du,5.396803\du)--(14.260000\du,6.196803\du)--cycle;
% setfont left to latex
\end{tikzpicture}


\paragraph{Remarques}
\begin{itemize}
\item Appliquer la méthode salaire à un employé n'a pas de sens.
	A l'exécution, l'employé devra référencer(dans le code) un comptable,
	une caissière ou un directeur commercial.
\item Une classe qui hérite d'une classe abstraite reste abstraite
	si elle ne définit pas la méthode retardée.
\item On ne peut pas créer d'objet à partir d'une classe abstraite
	Par contre une classe abstraite peut avoir un constructeur pour ses
	classes descendantes (\textbf{super}) 
\end{itemize}
\subsection{Héritage simple et héritage multiple}
\paragraph{Héritage simple} une classe ne peut hériter directement que 
d'une seule classe ancêtre
\paragraph{Héritage multiple} Sa classe peut hériter directement de plusieurs classe 
intérêt partager plusieurs point de vue complémentaire\\
% Graphic for TeX using PGF
% Title: /home/satenske/Diagram2.dia
% Creator: Dia v0.97.1
% CreationDate: Thu Sep 15 10:36:57 2011
% For: satenske
% \usepackage{tikz}
% The following commands are not supported in PSTricks at present
% We define them conditionally, so when they are implemented,
% this pgf file will use them.
\ifx\du\undefined
  \newlength{\du}
\fi
\setlength{\du}{15\unitlength}
\begin{tikzpicture}
\pgftransformxscale{1.000000}
\pgftransformyscale{-1.000000}
\definecolor{dialinecolor}{rgb}{0.000000, 0.000000, 0.000000}
\pgfsetstrokecolor{dialinecolor}
\definecolor{dialinecolor}{rgb}{1.000000, 1.000000, 1.000000}
\pgfsetfillcolor{dialinecolor}
\pgfsetlinewidth{0.100000\du}
\pgfsetdash{}{0pt}
\definecolor{dialinecolor}{rgb}{1.000000, 1.000000, 1.000000}
\pgfsetfillcolor{dialinecolor}
\fill (4.050000\du,3.550000\du)--(4.050000\du,4.950000\du)--(10.740000\du,4.950000\du)--(10.740000\du,3.550000\du)--cycle;
\definecolor{dialinecolor}{rgb}{0.000000, 0.000000, 0.000000}
\pgfsetstrokecolor{dialinecolor}
\draw (4.050000\du,3.550000\du)--(4.050000\du,4.950000\du)--(10.740000\du,4.950000\du)--(10.740000\du,3.550000\du)--cycle;
% setfont left to latex
\definecolor{dialinecolor}{rgb}{0.000000, 0.000000, 0.000000}
\pgfsetstrokecolor{dialinecolor}
\node at (7.395000\du,4.500000\du){Article fragile};
\definecolor{dialinecolor}{rgb}{1.000000, 1.000000, 1.000000}
\pgfsetfillcolor{dialinecolor}
\fill (4.050000\du,4.950000\du)--(4.050000\du,5.350000\du)--(10.740000\du,5.350000\du)--(10.740000\du,4.950000\du)--cycle;
\definecolor{dialinecolor}{rgb}{0.000000, 0.000000, 0.000000}
\pgfsetstrokecolor{dialinecolor}
\draw (4.050000\du,4.950000\du)--(4.050000\du,5.350000\du)--(10.740000\du,5.350000\du)--(10.740000\du,4.950000\du)--cycle;
\definecolor{dialinecolor}{rgb}{1.000000, 1.000000, 1.000000}
\pgfsetfillcolor{dialinecolor}
\fill (4.050000\du,5.350000\du)--(4.050000\du,5.750000\du)--(10.740000\du,5.750000\du)--(10.740000\du,5.350000\du)--cycle;
\definecolor{dialinecolor}{rgb}{0.000000, 0.000000, 0.000000}
\pgfsetstrokecolor{dialinecolor}
\draw (4.050000\du,5.350000\du)--(4.050000\du,5.750000\du)--(10.740000\du,5.750000\du)--(10.740000\du,5.350000\du)--cycle;
\pgfsetlinewidth{0.100000\du}
\pgfsetdash{}{0pt}
\definecolor{dialinecolor}{rgb}{1.000000, 1.000000, 1.000000}
\pgfsetfillcolor{dialinecolor}
\fill (14.860000\du,3.590000\du)--(14.860000\du,4.990000\du)--(24.295000\du,4.990000\du)--(24.295000\du,3.590000\du)--cycle;
\definecolor{dialinecolor}{rgb}{0.000000, 0.000000, 0.000000}
\pgfsetstrokecolor{dialinecolor}
\draw (14.860000\du,3.590000\du)--(14.860000\du,4.990000\du)--(24.295000\du,4.990000\du)--(24.295000\du,3.590000\du)--cycle;
% setfont left to latex
\definecolor{dialinecolor}{rgb}{0.000000, 0.000000, 0.000000}
\pgfsetstrokecolor{dialinecolor}
\node at (19.577500\du,4.540000\du){Article electronique};
\definecolor{dialinecolor}{rgb}{1.000000, 1.000000, 1.000000}
\pgfsetfillcolor{dialinecolor}
\fill (14.860000\du,4.990000\du)--(14.860000\du,5.390000\du)--(24.295000\du,5.390000\du)--(24.295000\du,4.990000\du)--cycle;
\definecolor{dialinecolor}{rgb}{0.000000, 0.000000, 0.000000}
\pgfsetstrokecolor{dialinecolor}
\draw (14.860000\du,4.990000\du)--(14.860000\du,5.390000\du)--(24.295000\du,5.390000\du)--(24.295000\du,4.990000\du)--cycle;
\definecolor{dialinecolor}{rgb}{1.000000, 1.000000, 1.000000}
\pgfsetfillcolor{dialinecolor}
\fill (14.860000\du,5.390000\du)--(14.860000\du,5.790000\du)--(24.295000\du,5.790000\du)--(24.295000\du,5.390000\du)--cycle;
\definecolor{dialinecolor}{rgb}{0.000000, 0.000000, 0.000000}
\pgfsetstrokecolor{dialinecolor}
\draw (14.860000\du,5.390000\du)--(14.860000\du,5.790000\du)--(24.295000\du,5.790000\du)--(24.295000\du,5.390000\du)--cycle;
\pgfsetlinewidth{0.100000\du}
\pgfsetdash{}{0pt}
\definecolor{dialinecolor}{rgb}{1.000000, 1.000000, 1.000000}
\pgfsetfillcolor{dialinecolor}
\fill (9.960000\du,11.340000\du)--(9.960000\du,12.740000\du)--(15.360000\du,12.740000\du)--(15.360000\du,11.340000\du)--cycle;
\definecolor{dialinecolor}{rgb}{0.000000, 0.000000, 0.000000}
\pgfsetstrokecolor{dialinecolor}
\draw (9.960000\du,11.340000\du)--(9.960000\du,12.740000\du)--(15.360000\du,12.740000\du)--(15.360000\du,11.340000\du)--cycle;
% setfont left to latex
\definecolor{dialinecolor}{rgb}{0.000000, 0.000000, 0.000000}
\pgfsetstrokecolor{dialinecolor}
\node at (12.660000\du,12.290000\du){Chaine Hifi};
\definecolor{dialinecolor}{rgb}{1.000000, 1.000000, 1.000000}
\pgfsetfillcolor{dialinecolor}
\fill (9.960000\du,12.740000\du)--(9.960000\du,13.140000\du)--(15.360000\du,13.140000\du)--(15.360000\du,12.740000\du)--cycle;
\definecolor{dialinecolor}{rgb}{0.000000, 0.000000, 0.000000}
\pgfsetstrokecolor{dialinecolor}
\draw (9.960000\du,12.740000\du)--(9.960000\du,13.140000\du)--(15.360000\du,13.140000\du)--(15.360000\du,12.740000\du)--cycle;
\definecolor{dialinecolor}{rgb}{1.000000, 1.000000, 1.000000}
\pgfsetfillcolor{dialinecolor}
\fill (9.960000\du,13.140000\du)--(9.960000\du,13.540000\du)--(15.360000\du,13.540000\du)--(15.360000\du,13.140000\du)--cycle;
\definecolor{dialinecolor}{rgb}{0.000000, 0.000000, 0.000000}
\pgfsetstrokecolor{dialinecolor}
\draw (9.960000\du,13.140000\du)--(9.960000\du,13.540000\du)--(15.360000\du,13.540000\du)--(15.360000\du,13.140000\du)--cycle;
\pgfsetlinewidth{0.100000\du}
\pgfsetdash{}{0pt}
\pgfsetmiterjoin
\pgfsetbuttcap
{
\definecolor{dialinecolor}{rgb}{0.000000, 0.000000, 0.000000}
\pgfsetfillcolor{dialinecolor}
% was here!!!
\definecolor{dialinecolor}{rgb}{0.000000, 0.000000, 0.000000}
\pgfsetstrokecolor{dialinecolor}
\draw (7.395000\du,5.750000\du)--(7.395000\du,8.545000\du)--(12.660000\du,8.545000\du)--(12.660000\du,11.340000\du);
}
\definecolor{dialinecolor}{rgb}{0.000000, 0.000000, 0.000000}
\pgfsetstrokecolor{dialinecolor}
\draw (7.395000\du,6.661803\du)--(7.395000\du,8.545000\du)--(12.660000\du,8.545000\du)--(12.660000\du,11.340000\du);
\pgfsetmiterjoin
\definecolor{dialinecolor}{rgb}{1.000000, 1.000000, 1.000000}
\pgfsetfillcolor{dialinecolor}
\fill (7.795000\du,6.661803\du)--(7.395000\du,5.861803\du)--(6.995000\du,6.661803\du)--cycle;
\pgfsetlinewidth{0.100000\du}
\pgfsetdash{}{0pt}
\pgfsetmiterjoin
\definecolor{dialinecolor}{rgb}{0.000000, 0.000000, 0.000000}
\pgfsetstrokecolor{dialinecolor}
\draw (7.795000\du,6.661803\du)--(7.395000\du,5.861803\du)--(6.995000\du,6.661803\du)--cycle;
% setfont left to latex
\pgfsetlinewidth{0.100000\du}
\pgfsetdash{}{0pt}
\pgfsetmiterjoin
\pgfsetbuttcap
{
\definecolor{dialinecolor}{rgb}{0.000000, 0.000000, 0.000000}
\pgfsetfillcolor{dialinecolor}
% was here!!!
\definecolor{dialinecolor}{rgb}{0.000000, 0.000000, 0.000000}
\pgfsetstrokecolor{dialinecolor}
\draw (19.577500\du,5.790000\du)--(19.577500\du,8.565000\du)--(12.660000\du,8.565000\du)--(12.660000\du,11.340000\du);
}
\definecolor{dialinecolor}{rgb}{0.000000, 0.000000, 0.000000}
\pgfsetstrokecolor{dialinecolor}
\draw (19.577500\du,6.701803\du)--(19.577500\du,8.565000\du)--(12.660000\du,8.565000\du)--(12.660000\du,11.340000\du);
\pgfsetmiterjoin
\definecolor{dialinecolor}{rgb}{1.000000, 1.000000, 1.000000}
\pgfsetfillcolor{dialinecolor}
\fill (19.977500\du,6.701803\du)--(19.577500\du,5.901803\du)--(19.177500\du,6.701803\du)--cycle;
\pgfsetlinewidth{0.100000\du}
\pgfsetdash{}{0pt}
\pgfsetmiterjoin
\definecolor{dialinecolor}{rgb}{0.000000, 0.000000, 0.000000}
\pgfsetstrokecolor{dialinecolor}
\draw (19.977500\du,6.701803\du)--(19.577500\du,5.901803\du)--(19.177500\du,6.701803\du)--cycle;
% setfont left to latex
\end{tikzpicture}

\lstinputlisting[language=java]{ex7.java}	
\subparagraph{Remarques}
\begin{itemize}
\item En héritage simple le graphie d'héritage est un arbre.
\item Tous les langages à objets ne possèdent pas l'héritage multiple car: 
	\begin{itemize}
		\item Sémantique peut claire
		\item Pue d'exemple pertinent exploitant l'héritage multiple
		\item Des problèmes théorique et pratiques que subsistent
	\end{itemize}
\end{itemize}
\subsection{Conflits d'héritage}
En héritage multiple, une même méthode peut être héritée par une sous classe plusieurs fois!
\paragraph{Problème}pour la sous-classe quelle méthode retenir parmi l'ensemble des méthodes.
\paragraph{Exemple}Coût de livraison pour un fromage\\
% Graphic for TeX using PGF
% Title: /home/satenske/cours/AP/obj3/uml5.dia
% Creator: Dia v0.97.1
% CreationDate: Thu Sep 22 10:08:42 2011
% For: satenske
% \usepackage{tikz}
% The following commands are not supported in PSTricks at present
% We define them conditionally, so when they are implemented,
% this pgf file will use them.
\ifx\du\undefined
  \newlength{\du}
\fi
\setlength{\du}{15\unitlength}
\begin{tikzpicture}
\pgftransformxscale{1.000000}
\pgftransformyscale{-1.000000}
\definecolor{dialinecolor}{rgb}{0.000000, 0.000000, 0.000000}
\pgfsetstrokecolor{dialinecolor}
\definecolor{dialinecolor}{rgb}{1.000000, 1.000000, 1.000000}
\pgfsetfillcolor{dialinecolor}
\pgfsetlinewidth{0.020000\du}
\pgfsetdash{}{0pt}
\definecolor{dialinecolor}{rgb}{1.000000, 1.000000, 1.000000}
\pgfsetfillcolor{dialinecolor}
\fill (9.600000\du,1.050000\du)--(9.600000\du,2.450000\du)--(17.030000\du,2.450000\du)--(17.030000\du,1.050000\du)--cycle;
\definecolor{dialinecolor}{rgb}{0.000000, 0.000000, 0.000000}
\pgfsetstrokecolor{dialinecolor}
\draw (9.600000\du,1.050000\du)--(9.600000\du,2.450000\du)--(17.030000\du,2.450000\du)--(17.030000\du,1.050000\du)--cycle;
% setfont left to latex
\definecolor{dialinecolor}{rgb}{0.000000, 0.000000, 0.000000}
\pgfsetstrokecolor{dialinecolor}
\node at (13.315000\du,2.000000\du){Article};
\definecolor{dialinecolor}{rgb}{1.000000, 1.000000, 1.000000}
\pgfsetfillcolor{dialinecolor}
\fill (9.600000\du,2.450000\du)--(9.600000\du,2.850000\du)--(17.030000\du,2.850000\du)--(17.030000\du,2.450000\du)--cycle;
\definecolor{dialinecolor}{rgb}{0.000000, 0.000000, 0.000000}
\pgfsetstrokecolor{dialinecolor}
\draw (9.600000\du,2.450000\du)--(9.600000\du,2.850000\du)--(17.030000\du,2.850000\du)--(17.030000\du,2.450000\du)--cycle;
\definecolor{dialinecolor}{rgb}{1.000000, 1.000000, 1.000000}
\pgfsetfillcolor{dialinecolor}
\fill (9.600000\du,2.850000\du)--(9.600000\du,3.850000\du)--(17.030000\du,3.850000\du)--(17.030000\du,2.850000\du)--cycle;
\definecolor{dialinecolor}{rgb}{0.000000, 0.000000, 0.000000}
\pgfsetstrokecolor{dialinecolor}
\draw (9.600000\du,2.850000\du)--(9.600000\du,3.850000\du)--(17.030000\du,3.850000\du)--(17.030000\du,2.850000\du)--cycle;
% setfont left to latex
\definecolor{dialinecolor}{rgb}{0.000000, 0.000000, 0.000000}
\pgfsetstrokecolor{dialinecolor}
\node[anchor=west] at (9.710000\du,3.550000\du){+coutDeLivraison()};
\pgfsetlinewidth{0.020000\du}
\pgfsetdash{}{0pt}
\definecolor{dialinecolor}{rgb}{1.000000, 1.000000, 1.000000}
\pgfsetfillcolor{dialinecolor}
\fill (18.050000\du,10.100000\du)--(18.050000\du,11.500000\du)--(27.485000\du,11.500000\du)--(27.485000\du,10.100000\du)--cycle;
\definecolor{dialinecolor}{rgb}{0.000000, 0.000000, 0.000000}
\pgfsetstrokecolor{dialinecolor}
\draw (18.050000\du,10.100000\du)--(18.050000\du,11.500000\du)--(27.485000\du,11.500000\du)--(27.485000\du,10.100000\du)--cycle;
% setfont left to latex
\definecolor{dialinecolor}{rgb}{0.000000, 0.000000, 0.000000}
\pgfsetstrokecolor{dialinecolor}
\node at (22.767500\du,11.050000\du){Article électronique};
\definecolor{dialinecolor}{rgb}{1.000000, 1.000000, 1.000000}
\pgfsetfillcolor{dialinecolor}
\fill (18.050000\du,11.500000\du)--(18.050000\du,11.900000\du)--(27.485000\du,11.900000\du)--(27.485000\du,11.500000\du)--cycle;
\definecolor{dialinecolor}{rgb}{0.000000, 0.000000, 0.000000}
\pgfsetstrokecolor{dialinecolor}
\draw (18.050000\du,11.500000\du)--(18.050000\du,11.900000\du)--(27.485000\du,11.900000\du)--(27.485000\du,11.500000\du)--cycle;
\definecolor{dialinecolor}{rgb}{1.000000, 1.000000, 1.000000}
\pgfsetfillcolor{dialinecolor}
\fill (18.050000\du,11.900000\du)--(18.050000\du,12.300000\du)--(27.485000\du,12.300000\du)--(27.485000\du,11.900000\du)--cycle;
\definecolor{dialinecolor}{rgb}{0.000000, 0.000000, 0.000000}
\pgfsetstrokecolor{dialinecolor}
\draw (18.050000\du,11.900000\du)--(18.050000\du,12.300000\du)--(27.485000\du,12.300000\du)--(27.485000\du,11.900000\du)--cycle;
\pgfsetlinewidth{0.020000\du}
\pgfsetdash{}{0pt}
\definecolor{dialinecolor}{rgb}{1.000000, 1.000000, 1.000000}
\pgfsetfillcolor{dialinecolor}
\fill (3.850000\du,10.200000\du)--(3.850000\du,11.600000\du)--(11.280000\du,11.600000\du)--(11.280000\du,10.200000\du)--cycle;
\definecolor{dialinecolor}{rgb}{0.000000, 0.000000, 0.000000}
\pgfsetstrokecolor{dialinecolor}
\draw (3.850000\du,10.200000\du)--(3.850000\du,11.600000\du)--(11.280000\du,11.600000\du)--(11.280000\du,10.200000\du)--cycle;
% setfont left to latex
\definecolor{dialinecolor}{rgb}{0.000000, 0.000000, 0.000000}
\pgfsetstrokecolor{dialinecolor}
\node at (7.565000\du,11.150000\du){Article fragile};
\definecolor{dialinecolor}{rgb}{1.000000, 1.000000, 1.000000}
\pgfsetfillcolor{dialinecolor}
\fill (3.850000\du,11.600000\du)--(3.850000\du,12.000000\du)--(11.280000\du,12.000000\du)--(11.280000\du,11.600000\du)--cycle;
\definecolor{dialinecolor}{rgb}{0.000000, 0.000000, 0.000000}
\pgfsetstrokecolor{dialinecolor}
\draw (3.850000\du,11.600000\du)--(3.850000\du,12.000000\du)--(11.280000\du,12.000000\du)--(11.280000\du,11.600000\du)--cycle;
\definecolor{dialinecolor}{rgb}{1.000000, 1.000000, 1.000000}
\pgfsetfillcolor{dialinecolor}
\fill (3.850000\du,12.000000\du)--(3.850000\du,13.000000\du)--(11.280000\du,13.000000\du)--(11.280000\du,12.000000\du)--cycle;
\definecolor{dialinecolor}{rgb}{0.000000, 0.000000, 0.000000}
\pgfsetstrokecolor{dialinecolor}
\draw (3.850000\du,12.000000\du)--(3.850000\du,13.000000\du)--(11.280000\du,13.000000\du)--(11.280000\du,12.000000\du)--cycle;
% setfont left to latex
\definecolor{dialinecolor}{rgb}{0.000000, 0.000000, 0.000000}
\pgfsetstrokecolor{dialinecolor}
\node[anchor=west] at (3.960000\du,12.700000\du){+coutDeLivraison()};
\pgfsetlinewidth{0.020000\du}
\pgfsetdash{}{0pt}
\definecolor{dialinecolor}{rgb}{1.000000, 1.000000, 1.000000}
\pgfsetfillcolor{dialinecolor}
\fill (19.800000\du,16.300000\du)--(31.605000\du,16.300000\du)--(32.205000\du,16.900000\du)--(32.205000\du,18.118633\du)--(19.800000\du,18.118633\du)--cycle;
\definecolor{dialinecolor}{rgb}{0.000000, 0.000000, 0.000000}
\pgfsetstrokecolor{dialinecolor}
\draw (19.800000\du,16.300000\du)--(31.605000\du,16.300000\du)--(32.205000\du,16.900000\du)--(32.205000\du,18.118633\du)--(19.800000\du,18.118633\du)--cycle;
\pgfsetlinewidth{0.010000\du}
\definecolor{dialinecolor}{rgb}{0.000000, 0.000000, 0.000000}
\pgfsetstrokecolor{dialinecolor}
\draw (31.605000\du,16.300000\du)--(31.605000\du,16.900000\du)--(32.205000\du,16.900000\du);
% setfont left to latex
\definecolor{dialinecolor}{rgb}{0.000000, 0.000000, 0.000000}
\pgfsetstrokecolor{dialinecolor}
\node[anchor=west] at (20.110000\du,17.137500\du){Méthode de coutLivraison                                                      };
% setfont left to latex
\definecolor{dialinecolor}{rgb}{0.000000, 0.000000, 0.000000}
\pgfsetstrokecolor{dialinecolor}
\node[anchor=west] at (20.110000\du,17.443711\du){};
% setfont left to latex
\definecolor{dialinecolor}{rgb}{0.000000, 0.000000, 0.000000}
\pgfsetstrokecolor{dialinecolor}
\node[anchor=west] at (20.110000\du,17.749922\du){conflit d'héritage                                };
\pgfsetlinewidth{0.100000\du}
\pgfsetdash{}{0pt}
\definecolor{dialinecolor}{rgb}{1.000000, 1.000000, 1.000000}
\pgfsetfillcolor{dialinecolor}
\fill (9.175000\du,16.045000\du)--(9.175000\du,17.445000\du)--(13.605000\du,17.445000\du)--(13.605000\du,16.045000\du)--cycle;
\definecolor{dialinecolor}{rgb}{0.000000, 0.000000, 0.000000}
\pgfsetstrokecolor{dialinecolor}
\draw (9.175000\du,16.045000\du)--(9.175000\du,17.445000\du)--(13.605000\du,17.445000\du)--(13.605000\du,16.045000\du)--cycle;
% setfont left to latex
\definecolor{dialinecolor}{rgb}{0.000000, 0.000000, 0.000000}
\pgfsetstrokecolor{dialinecolor}
\node at (11.390000\du,16.995000\du){Fromage};
\definecolor{dialinecolor}{rgb}{1.000000, 1.000000, 1.000000}
\pgfsetfillcolor{dialinecolor}
\fill (9.175000\du,17.445000\du)--(9.175000\du,17.845000\du)--(13.605000\du,17.845000\du)--(13.605000\du,17.445000\du)--cycle;
\definecolor{dialinecolor}{rgb}{0.000000, 0.000000, 0.000000}
\pgfsetstrokecolor{dialinecolor}
\draw (9.175000\du,17.445000\du)--(9.175000\du,17.845000\du)--(13.605000\du,17.845000\du)--(13.605000\du,17.445000\du)--cycle;
\definecolor{dialinecolor}{rgb}{1.000000, 1.000000, 1.000000}
\pgfsetfillcolor{dialinecolor}
\fill (9.175000\du,17.845000\du)--(9.175000\du,18.245000\du)--(13.605000\du,18.245000\du)--(13.605000\du,17.845000\du)--cycle;
\definecolor{dialinecolor}{rgb}{0.000000, 0.000000, 0.000000}
\pgfsetstrokecolor{dialinecolor}
\draw (9.175000\du,17.845000\du)--(9.175000\du,18.245000\du)--(13.605000\du,18.245000\du)--(13.605000\du,17.845000\du)--cycle;
\pgfsetlinewidth{0.100000\du}
\pgfsetdash{}{0pt}
\pgfsetmiterjoin
\pgfsetbuttcap
{
\definecolor{dialinecolor}{rgb}{0.000000, 0.000000, 0.000000}
\pgfsetfillcolor{dialinecolor}
% was here!!!
\definecolor{dialinecolor}{rgb}{0.000000, 0.000000, 0.000000}
\pgfsetstrokecolor{dialinecolor}
\draw (13.315000\du,3.850000\du)--(13.315000\du,7.025000\du)--(7.565000\du,7.025000\du)--(7.565000\du,10.200000\du);
}
\definecolor{dialinecolor}{rgb}{0.000000, 0.000000, 0.000000}
\pgfsetstrokecolor{dialinecolor}
\draw (13.315000\du,4.761803\du)--(13.315000\du,7.025000\du)--(7.565000\du,7.025000\du)--(7.565000\du,10.200000\du);
\pgfsetmiterjoin
\definecolor{dialinecolor}{rgb}{1.000000, 1.000000, 1.000000}
\pgfsetfillcolor{dialinecolor}
\fill (13.715000\du,4.761803\du)--(13.315000\du,3.961803\du)--(12.915000\du,4.761803\du)--cycle;
\pgfsetlinewidth{0.100000\du}
\pgfsetdash{}{0pt}
\pgfsetmiterjoin
\definecolor{dialinecolor}{rgb}{0.000000, 0.000000, 0.000000}
\pgfsetstrokecolor{dialinecolor}
\draw (13.715000\du,4.761803\du)--(13.315000\du,3.961803\du)--(12.915000\du,4.761803\du)--cycle;
% setfont left to latex
\pgfsetlinewidth{0.100000\du}
\pgfsetdash{}{0pt}
\pgfsetmiterjoin
\pgfsetbuttcap
{
\definecolor{dialinecolor}{rgb}{0.000000, 0.000000, 0.000000}
\pgfsetfillcolor{dialinecolor}
% was here!!!
\definecolor{dialinecolor}{rgb}{0.000000, 0.000000, 0.000000}
\pgfsetstrokecolor{dialinecolor}
\draw (13.315000\du,3.850000\du)--(13.315000\du,6.975000\du)--(22.767500\du,6.975000\du)--(22.767500\du,10.100000\du);
}
\definecolor{dialinecolor}{rgb}{0.000000, 0.000000, 0.000000}
\pgfsetstrokecolor{dialinecolor}
\draw (13.315000\du,4.761803\du)--(13.315000\du,6.975000\du)--(22.767500\du,6.975000\du)--(22.767500\du,10.100000\du);
\pgfsetmiterjoin
\definecolor{dialinecolor}{rgb}{1.000000, 1.000000, 1.000000}
\pgfsetfillcolor{dialinecolor}
\fill (13.715000\du,4.761803\du)--(13.315000\du,3.961803\du)--(12.915000\du,4.761803\du)--cycle;
\pgfsetlinewidth{0.100000\du}
\pgfsetdash{}{0pt}
\pgfsetmiterjoin
\definecolor{dialinecolor}{rgb}{0.000000, 0.000000, 0.000000}
\pgfsetstrokecolor{dialinecolor}
\draw (13.715000\du,4.761803\du)--(13.315000\du,3.961803\du)--(12.915000\du,4.761803\du)--cycle;
% setfont left to latex
\pgfsetlinewidth{0.100000\du}
\pgfsetdash{}{0pt}
\pgfsetmiterjoin
\pgfsetbuttcap
{
\definecolor{dialinecolor}{rgb}{0.000000, 0.000000, 0.000000}
\pgfsetfillcolor{dialinecolor}
% was here!!!
\definecolor{dialinecolor}{rgb}{0.000000, 0.000000, 0.000000}
\pgfsetstrokecolor{dialinecolor}
\draw (7.395100\du,13.087500\du)--(7.395100\du,14.566250\du)--(11.390000\du,14.566250\du)--(11.390000\du,16.045000\du);
}
\definecolor{dialinecolor}{rgb}{0.000000, 0.000000, 0.000000}
\pgfsetstrokecolor{dialinecolor}
\draw (7.395100\du,13.999303\du)--(7.395100\du,14.566250\du)--(11.390000\du,14.566250\du)--(11.390000\du,16.045000\du);
\pgfsetmiterjoin
\definecolor{dialinecolor}{rgb}{1.000000, 1.000000, 1.000000}
\pgfsetfillcolor{dialinecolor}
\fill (7.795100\du,13.999303\du)--(7.395100\du,13.199303\du)--(6.995100\du,13.999303\du)--cycle;
\pgfsetlinewidth{0.100000\du}
\pgfsetdash{}{0pt}
\pgfsetmiterjoin
\definecolor{dialinecolor}{rgb}{0.000000, 0.000000, 0.000000}
\pgfsetstrokecolor{dialinecolor}
\draw (7.795100\du,13.999303\du)--(7.395100\du,13.199303\du)--(6.995100\du,13.999303\du)--cycle;
% setfont left to latex
\pgfsetlinewidth{0.100000\du}
\pgfsetdash{}{0pt}
\pgfsetmiterjoin
\pgfsetbuttcap
{
\definecolor{dialinecolor}{rgb}{0.000000, 0.000000, 0.000000}
\pgfsetfillcolor{dialinecolor}
% was here!!!
\definecolor{dialinecolor}{rgb}{0.000000, 0.000000, 0.000000}
\pgfsetstrokecolor{dialinecolor}
\draw (22.767500\du,12.300000\du)--(22.767500\du,14.587500\du)--(11.390000\du,14.587500\du)--(11.390000\du,16.045000\du);
}
\definecolor{dialinecolor}{rgb}{0.000000, 0.000000, 0.000000}
\pgfsetstrokecolor{dialinecolor}
\draw (22.767500\du,13.211803\du)--(22.767500\du,14.587500\du)--(11.390000\du,14.587500\du)--(11.390000\du,16.045000\du);
\pgfsetmiterjoin
\definecolor{dialinecolor}{rgb}{1.000000, 1.000000, 1.000000}
\pgfsetfillcolor{dialinecolor}
\fill (23.167500\du,13.211803\du)--(22.767500\du,12.411803\du)--(22.367500\du,13.211803\du)--cycle;
\pgfsetlinewidth{0.100000\du}
\pgfsetdash{}{0pt}
\pgfsetmiterjoin
\definecolor{dialinecolor}{rgb}{0.000000, 0.000000, 0.000000}
\pgfsetstrokecolor{dialinecolor}
\draw (23.167500\du,13.211803\du)--(22.767500\du,12.411803\du)--(22.367500\du,13.211803\du)--cycle;
% setfont left to latex
\pgfsetlinewidth{0.100000\du}
\pgfsetdash{{\pgflinewidth}{0.200000\du}}{0cm}
\pgfsetdash{{\pgflinewidth}{0.200000\du}}{0cm}
\pgfsetbuttcap
{
\definecolor{dialinecolor}{rgb}{0.000000, 0.000000, 0.000000}
\pgfsetfillcolor{dialinecolor}
% was here!!!
\definecolor{dialinecolor}{rgb}{0.000000, 0.000000, 0.000000}
\pgfsetstrokecolor{dialinecolor}
\draw (19.800000\du,17.209317\du)--(13.611587\du,17.161990\du);
}
\end{tikzpicture}
\\
Dans les langages pas de relation universelle.
Deux approches:
\begin{itemize}
	\item Le choix est établi par le langage en considérant un parcours de graphe d'héritage
	\item Le programmeur résout (dans son code) le conflit
\end{itemize}
\paragraph{1ère approche}
Différents parcours possibles du graphe: en largeur d'abord, en profondeur d'abord, stratagème mixe\ldots
\subparagraph{exemple:}  
\textbf{Parcours 1:} Fromage, Article fragile, Article Périssable, Article 
$\Rightarrow$ CoutLivraison de la classe Fragile\\
\textbf{Parcours 2: } Résolution du conflit par le programmeur par exemple: 
\begin{itemize}
	\item En forçant la redéfinition
	\item En obligeant à renommer la méthode dans la sous-classe
\end{itemize}
\section{La composition d'objets. La relation \textit{Avoir}}
	Deux manières d'utiliser un classe: 
	\begin{itemize}
		\item L'héritage (relation \textit{Être})
		\item La composition relation (\textit{Avoir})
	\end{itemize}
\subsection{Définition}
Un objet composite (ou agrégation d'objets) est un objet formé de l'assemblage de plusieurs
objets. 
\paragraph{Avantage} Construire des objets complexes à partir d'objets existant.
Les constituants d'un objet composite ne sont accessible que via leurs interfaces (principe
d'encapsulation)
\subsection{Exemple}
\subsubsection{La chaîne hifi}
\begin{center}
	% Graphic for TeX using PGF
% Title: /home/satenske/cours/AP/obj3/uml16.dia
% Creator: Dia v0.97.1
% CreationDate: Thu Sep 22 10:19:53 2011
% For: satenske
% \usepackage{tikz}
% The following commands are not supported in PSTricks at present
% We define them conditionally, so when they are implemented,
% this pgf file will use them.
\ifx\du\undefined
  \newlength{\du}
\fi
\setlength{\du}{15\unitlength}
\begin{tikzpicture}
\pgftransformxscale{1.000000}
\pgftransformyscale{-1.000000}
\definecolor{dialinecolor}{rgb}{0.000000, 0.000000, 0.000000}
\pgfsetstrokecolor{dialinecolor}
\definecolor{dialinecolor}{rgb}{1.000000, 1.000000, 1.000000}
\pgfsetfillcolor{dialinecolor}
\pgfsetlinewidth{0.100000\du}
\pgfsetdash{}{0pt}
\definecolor{dialinecolor}{rgb}{1.000000, 1.000000, 1.000000}
\pgfsetfillcolor{dialinecolor}
\fill (8.200000\du,5.650000\du)--(8.200000\du,7.050000\du)--(11.235000\du,7.050000\du)--(11.235000\du,5.650000\du)--cycle;
\definecolor{dialinecolor}{rgb}{0.000000, 0.000000, 0.000000}
\pgfsetstrokecolor{dialinecolor}
\draw (8.200000\du,5.650000\du)--(8.200000\du,7.050000\du)--(11.235000\du,7.050000\du)--(11.235000\du,5.650000\du)--cycle;
% setfont left to latex
\definecolor{dialinecolor}{rgb}{0.000000, 0.000000, 0.000000}
\pgfsetstrokecolor{dialinecolor}
\node at (9.717500\du,6.600000\du){Tuner};
\definecolor{dialinecolor}{rgb}{1.000000, 1.000000, 1.000000}
\pgfsetfillcolor{dialinecolor}
\fill (8.200000\du,7.050000\du)--(8.200000\du,7.450000\du)--(11.235000\du,7.450000\du)--(11.235000\du,7.050000\du)--cycle;
\definecolor{dialinecolor}{rgb}{0.000000, 0.000000, 0.000000}
\pgfsetstrokecolor{dialinecolor}
\draw (8.200000\du,7.050000\du)--(8.200000\du,7.450000\du)--(11.235000\du,7.450000\du)--(11.235000\du,7.050000\du)--cycle;
\definecolor{dialinecolor}{rgb}{1.000000, 1.000000, 1.000000}
\pgfsetfillcolor{dialinecolor}
\fill (8.200000\du,7.450000\du)--(8.200000\du,7.850000\du)--(11.235000\du,7.850000\du)--(11.235000\du,7.450000\du)--cycle;
\definecolor{dialinecolor}{rgb}{0.000000, 0.000000, 0.000000}
\pgfsetstrokecolor{dialinecolor}
\draw (8.200000\du,7.450000\du)--(8.200000\du,7.850000\du)--(11.235000\du,7.850000\du)--(11.235000\du,7.450000\du)--cycle;
\pgfsetlinewidth{0.100000\du}
\pgfsetdash{}{0pt}
\definecolor{dialinecolor}{rgb}{1.000000, 1.000000, 1.000000}
\pgfsetfillcolor{dialinecolor}
\fill (14.975000\du,4.415000\du)--(14.975000\du,5.815000\du)--(20.422500\du,5.815000\du)--(20.422500\du,4.415000\du)--cycle;
\definecolor{dialinecolor}{rgb}{0.000000, 0.000000, 0.000000}
\pgfsetstrokecolor{dialinecolor}
\draw (14.975000\du,4.415000\du)--(14.975000\du,5.815000\du)--(20.422500\du,5.815000\du)--(20.422500\du,4.415000\du)--cycle;
% setfont left to latex
\definecolor{dialinecolor}{rgb}{0.000000, 0.000000, 0.000000}
\pgfsetstrokecolor{dialinecolor}
\node at (17.698750\du,5.365000\du){Lecteur CD};
\definecolor{dialinecolor}{rgb}{1.000000, 1.000000, 1.000000}
\pgfsetfillcolor{dialinecolor}
\fill (14.975000\du,5.815000\du)--(14.975000\du,6.215000\du)--(20.422500\du,6.215000\du)--(20.422500\du,5.815000\du)--cycle;
\definecolor{dialinecolor}{rgb}{0.000000, 0.000000, 0.000000}
\pgfsetstrokecolor{dialinecolor}
\draw (14.975000\du,5.815000\du)--(14.975000\du,6.215000\du)--(20.422500\du,6.215000\du)--(20.422500\du,5.815000\du)--cycle;
\definecolor{dialinecolor}{rgb}{1.000000, 1.000000, 1.000000}
\pgfsetfillcolor{dialinecolor}
\fill (14.975000\du,6.215000\du)--(14.975000\du,6.615000\du)--(20.422500\du,6.615000\du)--(20.422500\du,6.215000\du)--cycle;
\definecolor{dialinecolor}{rgb}{0.000000, 0.000000, 0.000000}
\pgfsetstrokecolor{dialinecolor}
\draw (14.975000\du,6.215000\du)--(14.975000\du,6.615000\du)--(20.422500\du,6.615000\du)--(20.422500\du,6.215000\du)--cycle;
\pgfsetlinewidth{0.100000\du}
\pgfsetdash{}{0pt}
\definecolor{dialinecolor}{rgb}{1.000000, 1.000000, 1.000000}
\pgfsetfillcolor{dialinecolor}
\fill (23.600000\du,5.680000\du)--(23.600000\du,7.080000\du)--(28.005000\du,7.080000\du)--(28.005000\du,5.680000\du)--cycle;
\definecolor{dialinecolor}{rgb}{0.000000, 0.000000, 0.000000}
\pgfsetstrokecolor{dialinecolor}
\draw (23.600000\du,5.680000\du)--(23.600000\du,7.080000\du)--(28.005000\du,7.080000\du)--(28.005000\du,5.680000\du)--cycle;
% setfont left to latex
\definecolor{dialinecolor}{rgb}{0.000000, 0.000000, 0.000000}
\pgfsetstrokecolor{dialinecolor}
\node at (25.802500\du,6.630000\du){Enceinte};
\definecolor{dialinecolor}{rgb}{1.000000, 1.000000, 1.000000}
\pgfsetfillcolor{dialinecolor}
\fill (23.600000\du,7.080000\du)--(23.600000\du,7.480000\du)--(28.005000\du,7.480000\du)--(28.005000\du,7.080000\du)--cycle;
\definecolor{dialinecolor}{rgb}{0.000000, 0.000000, 0.000000}
\pgfsetstrokecolor{dialinecolor}
\draw (23.600000\du,7.080000\du)--(23.600000\du,7.480000\du)--(28.005000\du,7.480000\du)--(28.005000\du,7.080000\du)--cycle;
\definecolor{dialinecolor}{rgb}{1.000000, 1.000000, 1.000000}
\pgfsetfillcolor{dialinecolor}
\fill (23.600000\du,7.480000\du)--(23.600000\du,7.880000\du)--(28.005000\du,7.880000\du)--(28.005000\du,7.480000\du)--cycle;
\definecolor{dialinecolor}{rgb}{0.000000, 0.000000, 0.000000}
\pgfsetstrokecolor{dialinecolor}
\draw (23.600000\du,7.480000\du)--(23.600000\du,7.880000\du)--(28.005000\du,7.880000\du)--(28.005000\du,7.480000\du)--cycle;
\pgfsetlinewidth{0.100000\du}
\pgfsetdash{}{0pt}
\definecolor{dialinecolor}{rgb}{1.000000, 1.000000, 1.000000}
\pgfsetfillcolor{dialinecolor}
\fill (14.975000\du,12.545000\du)--(14.975000\du,13.945000\du)--(20.375000\du,13.945000\du)--(20.375000\du,12.545000\du)--cycle;
\definecolor{dialinecolor}{rgb}{0.000000, 0.000000, 0.000000}
\pgfsetstrokecolor{dialinecolor}
\draw (14.975000\du,12.545000\du)--(14.975000\du,13.945000\du)--(20.375000\du,13.945000\du)--(20.375000\du,12.545000\du)--cycle;
% setfont left to latex
\definecolor{dialinecolor}{rgb}{0.000000, 0.000000, 0.000000}
\pgfsetstrokecolor{dialinecolor}
\node at (17.675000\du,13.495000\du){Chaine Hifi};
\definecolor{dialinecolor}{rgb}{1.000000, 1.000000, 1.000000}
\pgfsetfillcolor{dialinecolor}
\fill (14.975000\du,13.945000\du)--(14.975000\du,14.345000\du)--(20.375000\du,14.345000\du)--(20.375000\du,13.945000\du)--cycle;
\definecolor{dialinecolor}{rgb}{0.000000, 0.000000, 0.000000}
\pgfsetstrokecolor{dialinecolor}
\draw (14.975000\du,13.945000\du)--(14.975000\du,14.345000\du)--(20.375000\du,14.345000\du)--(20.375000\du,13.945000\du)--cycle;
\definecolor{dialinecolor}{rgb}{1.000000, 1.000000, 1.000000}
\pgfsetfillcolor{dialinecolor}
\fill (14.975000\du,14.345000\du)--(14.975000\du,14.745000\du)--(20.375000\du,14.745000\du)--(20.375000\du,14.345000\du)--cycle;
\definecolor{dialinecolor}{rgb}{0.000000, 0.000000, 0.000000}
\pgfsetstrokecolor{dialinecolor}
\draw (14.975000\du,14.345000\du)--(14.975000\du,14.745000\du)--(20.375000\du,14.745000\du)--(20.375000\du,14.345000\du)--cycle;
\pgfsetlinewidth{0.100000\du}
\pgfsetdash{}{0pt}
\pgfsetmiterjoin
\pgfsetbuttcap
{
\definecolor{dialinecolor}{rgb}{0.000000, 0.000000, 0.000000}
\pgfsetfillcolor{dialinecolor}
% was here!!!
\definecolor{dialinecolor}{rgb}{0.000000, 0.000000, 0.000000}
\pgfsetstrokecolor{dialinecolor}
\draw (17.675000\du,12.545000\du)--(17.675000\du,10.197500\du)--(9.717500\du,10.197500\du)--(9.717500\du,7.850000\du);
}
\definecolor{dialinecolor}{rgb}{0.000000, 0.000000, 0.000000}
\pgfsetstrokecolor{dialinecolor}
\draw (17.675000\du,11.286421\du)--(17.675000\du,10.197500\du)--(9.717500\du,10.197500\du)--(9.717500\du,7.850000\du);
\pgfsetdash{}{0pt}
\pgfsetmiterjoin
\pgfsetbuttcap
\definecolor{dialinecolor}{rgb}{1.000000, 1.000000, 1.000000}
\pgfsetfillcolor{dialinecolor}
\fill (17.675000\du,12.545000\du)--(17.435000\du,11.845000\du)--(17.675000\du,11.145000\du)--(17.915000\du,11.845000\du)--cycle;
\pgfsetlinewidth{0.100000\du}
\pgfsetdash{}{0pt}
\pgfsetmiterjoin
\pgfsetbuttcap
\definecolor{dialinecolor}{rgb}{0.000000, 0.000000, 0.000000}
\pgfsetstrokecolor{dialinecolor}
\draw (17.675000\du,12.545000\du)--(17.435000\du,11.845000\du)--(17.675000\du,11.145000\du)--(17.915000\du,11.845000\du)--cycle;
% setfont left to latex
\definecolor{dialinecolor}{rgb}{0.000000, 0.000000, 0.000000}
\pgfsetstrokecolor{dialinecolor}
\node at (13.696250\du,9.997500\du){};
\definecolor{dialinecolor}{rgb}{0.000000, 0.000000, 0.000000}
\pgfsetstrokecolor{dialinecolor}
\node[anchor=west] at (18.225000\du,12.345000\du){};
\definecolor{dialinecolor}{rgb}{0.000000, 0.000000, 0.000000}
\pgfsetstrokecolor{dialinecolor}
\node[anchor=west] at (9.917500\du,8.450000\du){};
\pgfsetlinewidth{0.100000\du}
\pgfsetdash{}{0pt}
\pgfsetmiterjoin
\pgfsetbuttcap
{
\definecolor{dialinecolor}{rgb}{0.000000, 0.000000, 0.000000}
\pgfsetfillcolor{dialinecolor}
% was here!!!
\definecolor{dialinecolor}{rgb}{0.000000, 0.000000, 0.000000}
\pgfsetstrokecolor{dialinecolor}
\draw (17.675000\du,12.545000\du)--(17.675000\du,10.350000\du)--(17.698800\du,10.350000\du)--(17.698800\du,6.615000\du);
}
\definecolor{dialinecolor}{rgb}{0.000000, 0.000000, 0.000000}
\pgfsetstrokecolor{dialinecolor}
\draw (17.675000\du,11.286421\du)--(17.675000\du,10.350000\du)--(17.698800\du,10.350000\du)--(17.698800\du,6.615000\du);
\pgfsetdash{}{0pt}
\pgfsetmiterjoin
\pgfsetbuttcap
\definecolor{dialinecolor}{rgb}{1.000000, 1.000000, 1.000000}
\pgfsetfillcolor{dialinecolor}
\fill (17.675000\du,12.545000\du)--(17.435000\du,11.845000\du)--(17.675000\du,11.145000\du)--(17.915000\du,11.845000\du)--cycle;
\pgfsetlinewidth{0.100000\du}
\pgfsetdash{}{0pt}
\pgfsetmiterjoin
\pgfsetbuttcap
\definecolor{dialinecolor}{rgb}{0.000000, 0.000000, 0.000000}
\pgfsetstrokecolor{dialinecolor}
\draw (17.675000\du,12.545000\du)--(17.435000\du,11.845000\du)--(17.675000\du,11.145000\du)--(17.915000\du,11.845000\du)--cycle;
% setfont left to latex
\definecolor{dialinecolor}{rgb}{0.000000, 0.000000, 0.000000}
\pgfsetstrokecolor{dialinecolor}
\node at (17.686900\du,10.150000\du){};
\definecolor{dialinecolor}{rgb}{0.000000, 0.000000, 0.000000}
\pgfsetstrokecolor{dialinecolor}
\node[anchor=west] at (18.225000\du,12.345000\du){};
\definecolor{dialinecolor}{rgb}{0.000000, 0.000000, 0.000000}
\pgfsetstrokecolor{dialinecolor}
\node[anchor=west] at (17.898800\du,7.215000\du){};
\pgfsetlinewidth{0.100000\du}
\pgfsetdash{}{0pt}
\pgfsetmiterjoin
\pgfsetbuttcap
{
\definecolor{dialinecolor}{rgb}{0.000000, 0.000000, 0.000000}
\pgfsetfillcolor{dialinecolor}
% was here!!!
\definecolor{dialinecolor}{rgb}{0.000000, 0.000000, 0.000000}
\pgfsetstrokecolor{dialinecolor}
\draw (17.675000\du,12.494719\du)--(17.675000\du,10.187360\du)--(25.802500\du,10.187360\du)--(25.802500\du,7.880000\du);
}
\definecolor{dialinecolor}{rgb}{0.000000, 0.000000, 0.000000}
\pgfsetstrokecolor{dialinecolor}
\draw (17.675000\du,11.236141\du)--(17.675000\du,10.187360\du)--(25.802500\du,10.187360\du)--(25.802500\du,7.880000\du);
\pgfsetdash{}{0pt}
\pgfsetmiterjoin
\pgfsetbuttcap
\definecolor{dialinecolor}{rgb}{1.000000, 1.000000, 1.000000}
\pgfsetfillcolor{dialinecolor}
\fill (17.675000\du,12.494719\du)--(17.435000\du,11.794719\du)--(17.675000\du,11.094719\du)--(17.915000\du,11.794719\du)--cycle;
\pgfsetlinewidth{0.100000\du}
\pgfsetdash{}{0pt}
\pgfsetmiterjoin
\pgfsetbuttcap
\definecolor{dialinecolor}{rgb}{0.000000, 0.000000, 0.000000}
\pgfsetstrokecolor{dialinecolor}
\draw (17.675000\du,12.494719\du)--(17.435000\du,11.794719\du)--(17.675000\du,11.094719\du)--(17.915000\du,11.794719\du)--cycle;
% setfont left to latex
\definecolor{dialinecolor}{rgb}{0.000000, 0.000000, 0.000000}
\pgfsetstrokecolor{dialinecolor}
\node at (21.738750\du,9.987360\du){};
\definecolor{dialinecolor}{rgb}{0.000000, 0.000000, 0.000000}
\pgfsetstrokecolor{dialinecolor}
\node[anchor=west] at (18.225000\du,12.294719\du){};
\definecolor{dialinecolor}{rgb}{0.000000, 0.000000, 0.000000}
\pgfsetstrokecolor{dialinecolor}
\node[anchor=west] at (26.002500\du,8.480000\du){};
\end{tikzpicture}

\end{center}
\lstinputlisting[language=java]{8.java}	
\subsubsection{Remarques}
\paragraph{}Avec un langage à objet, définition d'une bibliothèque de classes pour réutiliser 
ce qui existe déjà pour la classe Tableau 
\paragraph{}
La communication entre un objet composite et ses constituants est réutilisée via l'interface 
des constituants $\Rightarrow$ relation client-fournisseur entre le composite (client) et 
ses constituants (fournisseur)
\paragraph{}Une classe A est cliente d'une classe B (et B est fournisseur de A) si
\begin{itemize}
\item A contient un attribut b de type B
\item A possède une méthode avec un paramètre d'entrée ou de retour de type B
\item A utilise une variable locale de type B pour une de ses méthodes
\end{itemize}
Le premier cas correspond à une \textbf{dépendance structurelle} les deux autres cas à une
dépendance \textbf{non structurelle }

\section{Mise en œuvre des mécanismes de réutilisation}
\subsection{Héritage d'interface}
Mode d'héritage dans lequel toutes les sous-classes peuvent répondre aux services
d'une classe ancêtre
\paragraph{Point de vue}
Une classe est assimilé à un type donc une sous classe est assimilé à un sous type.\\
En principe la classe ancêtre est souvent abstraite.
\paragraph{Exemple: les classes Tableau, Liste, Ensemble d'interface collection}
\begin{center}
	% Graphic for TeX using PGF
% Title: /home/satenske/cours/AP/obj3/uml16.dia
% Creator: Dia v0.97.1
% CreationDate: Thu Sep 22 10:19:31 2011
% For: satenske
% \usepackage{tikz}
% The following commands are not supported in PSTricks at present
% We define them conditionally, so when they are implemented,
% this pgf file will use them.
\ifx\du\undefined
  \newlength{\du}
\fi
\setlength{\du}{15\unitlength}
\begin{tikzpicture}
\pgftransformxscale{1.000000}
\pgftransformyscale{-1.000000}
\definecolor{dialinecolor}{rgb}{0.000000, 0.000000, 0.000000}
\pgfsetstrokecolor{dialinecolor}
\definecolor{dialinecolor}{rgb}{1.000000, 1.000000, 1.000000}
\pgfsetfillcolor{dialinecolor}
\pgfsetlinewidth{0.100000\du}
\pgfsetdash{}{0pt}
\definecolor{dialinecolor}{rgb}{1.000000, 1.000000, 1.000000}
\pgfsetfillcolor{dialinecolor}
\fill (19.675000\du,-5.885000\du)--(19.675000\du,-4.485000\du)--(23.657500\du,-4.485000\du)--(23.657500\du,-5.885000\du)--cycle;
\definecolor{dialinecolor}{rgb}{0.000000, 0.000000, 0.000000}
\pgfsetstrokecolor{dialinecolor}
\draw (19.675000\du,-5.885000\du)--(19.675000\du,-4.485000\du)--(23.657500\du,-4.485000\du)--(23.657500\du,-5.885000\du)--cycle;
% setfont left to latex
\definecolor{dialinecolor}{rgb}{0.000000, 0.000000, 0.000000}
\pgfsetstrokecolor{dialinecolor}
\node at (21.666250\du,-4.935000\du){Tableau};
\definecolor{dialinecolor}{rgb}{1.000000, 1.000000, 1.000000}
\pgfsetfillcolor{dialinecolor}
\fill (19.675000\du,-4.485000\du)--(19.675000\du,-4.085000\du)--(23.657500\du,-4.085000\du)--(23.657500\du,-4.485000\du)--cycle;
\definecolor{dialinecolor}{rgb}{0.000000, 0.000000, 0.000000}
\pgfsetstrokecolor{dialinecolor}
\draw (19.675000\du,-4.485000\du)--(19.675000\du,-4.085000\du)--(23.657500\du,-4.085000\du)--(23.657500\du,-4.485000\du)--cycle;
\definecolor{dialinecolor}{rgb}{1.000000, 1.000000, 1.000000}
\pgfsetfillcolor{dialinecolor}
\fill (19.675000\du,-4.085000\du)--(19.675000\du,-3.685000\du)--(23.657500\du,-3.685000\du)--(23.657500\du,-4.085000\du)--cycle;
\definecolor{dialinecolor}{rgb}{0.000000, 0.000000, 0.000000}
\pgfsetstrokecolor{dialinecolor}
\draw (19.675000\du,-4.085000\du)--(19.675000\du,-3.685000\du)--(23.657500\du,-3.685000\du)--(23.657500\du,-4.085000\du)--cycle;
\pgfsetlinewidth{0.100000\du}
\pgfsetdash{}{0pt}
\definecolor{dialinecolor}{rgb}{1.000000, 1.000000, 1.000000}
\pgfsetfillcolor{dialinecolor}
\fill (24.600000\du,-1.170000\du)--(24.600000\du,0.230000\du)--(27.285000\du,0.230000\du)--(27.285000\du,-1.170000\du)--cycle;
\definecolor{dialinecolor}{rgb}{0.000000, 0.000000, 0.000000}
\pgfsetstrokecolor{dialinecolor}
\draw (24.600000\du,-1.170000\du)--(24.600000\du,0.230000\du)--(27.285000\du,0.230000\du)--(27.285000\du,-1.170000\du)--cycle;
% setfont left to latex
\definecolor{dialinecolor}{rgb}{0.000000, 0.000000, 0.000000}
\pgfsetstrokecolor{dialinecolor}
\node at (25.942500\du,-0.220000\du){Liste};
\definecolor{dialinecolor}{rgb}{1.000000, 1.000000, 1.000000}
\pgfsetfillcolor{dialinecolor}
\fill (24.600000\du,0.230000\du)--(24.600000\du,0.630000\du)--(27.285000\du,0.630000\du)--(27.285000\du,0.230000\du)--cycle;
\definecolor{dialinecolor}{rgb}{0.000000, 0.000000, 0.000000}
\pgfsetstrokecolor{dialinecolor}
\draw (24.600000\du,0.230000\du)--(24.600000\du,0.630000\du)--(27.285000\du,0.630000\du)--(27.285000\du,0.230000\du)--cycle;
\definecolor{dialinecolor}{rgb}{1.000000, 1.000000, 1.000000}
\pgfsetfillcolor{dialinecolor}
\fill (24.600000\du,0.630000\du)--(24.600000\du,1.030000\du)--(27.285000\du,1.030000\du)--(27.285000\du,0.630000\du)--cycle;
\definecolor{dialinecolor}{rgb}{0.000000, 0.000000, 0.000000}
\pgfsetstrokecolor{dialinecolor}
\draw (24.600000\du,0.630000\du)--(24.600000\du,1.030000\du)--(27.285000\du,1.030000\du)--(27.285000\du,0.630000\du)--cycle;
\pgfsetlinewidth{0.100000\du}
\pgfsetdash{}{0pt}
\definecolor{dialinecolor}{rgb}{1.000000, 1.000000, 1.000000}
\pgfsetfillcolor{dialinecolor}
\fill (21.725000\du,4.095000\du)--(21.725000\du,5.495000\du)--(26.582500\du,5.495000\du)--(26.582500\du,4.095000\du)--cycle;
\definecolor{dialinecolor}{rgb}{0.000000, 0.000000, 0.000000}
\pgfsetstrokecolor{dialinecolor}
\draw (21.725000\du,4.095000\du)--(21.725000\du,5.495000\du)--(26.582500\du,5.495000\du)--(26.582500\du,4.095000\du)--cycle;
% setfont left to latex
\definecolor{dialinecolor}{rgb}{0.000000, 0.000000, 0.000000}
\pgfsetstrokecolor{dialinecolor}
\node at (24.153750\du,5.045000\du){Ensemble};
\definecolor{dialinecolor}{rgb}{1.000000, 1.000000, 1.000000}
\pgfsetfillcolor{dialinecolor}
\fill (21.725000\du,5.495000\du)--(21.725000\du,5.895000\du)--(26.582500\du,5.895000\du)--(26.582500\du,5.495000\du)--cycle;
\definecolor{dialinecolor}{rgb}{0.000000, 0.000000, 0.000000}
\pgfsetstrokecolor{dialinecolor}
\draw (21.725000\du,5.495000\du)--(21.725000\du,5.895000\du)--(26.582500\du,5.895000\du)--(26.582500\du,5.495000\du)--cycle;
\definecolor{dialinecolor}{rgb}{1.000000, 1.000000, 1.000000}
\pgfsetfillcolor{dialinecolor}
\fill (21.725000\du,5.895000\du)--(21.725000\du,6.295000\du)--(26.582500\du,6.295000\du)--(26.582500\du,5.895000\du)--cycle;
\definecolor{dialinecolor}{rgb}{0.000000, 0.000000, 0.000000}
\pgfsetstrokecolor{dialinecolor}
\draw (21.725000\du,5.895000\du)--(21.725000\du,6.295000\du)--(26.582500\du,6.295000\du)--(26.582500\du,5.895000\du)--cycle;
\definecolor{dialinecolor}{rgb}{1.000000, 1.000000, 1.000000}
\pgfsetfillcolor{dialinecolor}
\pgfpathellipse{\pgfpoint{10.108750\du}{-0.737500\du}}{\pgfpoint{1.000000\du}{0\du}}{\pgfpoint{0\du}{1.000000\du}}
\pgfusepath{fill}
\pgfsetlinewidth{0.100000\du}
\pgfsetdash{}{0pt}
\definecolor{dialinecolor}{rgb}{0.000000, 0.000000, 0.000000}
\pgfsetstrokecolor{dialinecolor}
\pgfpathellipse{\pgfpoint{10.108750\du}{-0.737500\du}}{\pgfpoint{1.000000\du}{0\du}}{\pgfpoint{0\du}{1.000000\du}}
\pgfusepath{stroke}
\definecolor{dialinecolor}{rgb}{0.000000, 0.000000, 0.000000}
\pgfsetstrokecolor{dialinecolor}
\draw (9.849931\du,-1.703426\du)--(10.249931\du,-1.436759\du);
\definecolor{dialinecolor}{rgb}{0.000000, 0.000000, 0.000000}
\pgfsetstrokecolor{dialinecolor}
\draw (9.849931\du,-1.703426\du)--(10.249931\du,-1.970092\du);
% setfont left to latex
\definecolor{dialinecolor}{rgb}{0.000000, 0.000000, 0.000000}
\pgfsetstrokecolor{dialinecolor}
\node at (10.108750\du,1.607500\du){Collection};
\pgfsetlinewidth{0.100000\du}
\pgfsetdash{}{0pt}
\pgfsetdash{}{0pt}
\pgfsetbuttcap
{
\definecolor{dialinecolor}{rgb}{0.000000, 0.000000, 0.000000}
\pgfsetfillcolor{dialinecolor}
% was here!!!
\definecolor{dialinecolor}{rgb}{0.000000, 0.000000, 0.000000}
\pgfsetstrokecolor{dialinecolor}
\draw (19.625488\du,-3.946712\du)--(11.817615\du,-0.739454\du);
}
\pgfsetlinewidth{0.100000\du}
\pgfsetdash{}{0pt}
\pgfsetdash{}{0pt}
\pgfsetbuttcap
{
\definecolor{dialinecolor}{rgb}{0.000000, 0.000000, 0.000000}
\pgfsetfillcolor{dialinecolor}
% was here!!!
\definecolor{dialinecolor}{rgb}{0.000000, 0.000000, 0.000000}
\pgfsetstrokecolor{dialinecolor}
\draw (24.550379\du,-0.067143\du)--(11.817856\du,-0.041008\du);
}
\pgfsetlinewidth{0.100000\du}
\pgfsetdash{}{0pt}
\pgfsetdash{}{0pt}
\pgfsetbuttcap
{
\definecolor{dialinecolor}{rgb}{0.000000, 0.000000, 0.000000}
\pgfsetfillcolor{dialinecolor}
% was here!!!
\definecolor{dialinecolor}{rgb}{0.000000, 0.000000, 0.000000}
\pgfsetstrokecolor{dialinecolor}
\draw (21.676330\du,4.272031\du)--(11.817227\du,0.598997\du);
}
\pgfsetlinewidth{0.100000\du}
\pgfsetdash{}{0pt}
\definecolor{dialinecolor}{rgb}{1.000000, 1.000000, 1.000000}
\pgfsetfillcolor{dialinecolor}
\fill (2.350000\du,-4.187500\du)--(6.885000\du,-4.187500\du)--(7.485000\du,-3.587500\du)--(7.485000\du,-2.487500\du)--(2.350000\du,-2.487500\du)--cycle;
\definecolor{dialinecolor}{rgb}{0.000000, 0.000000, 0.000000}
\pgfsetstrokecolor{dialinecolor}
\draw (2.350000\du,-4.187500\du)--(6.885000\du,-4.187500\du)--(7.485000\du,-3.587500\du)--(7.485000\du,-2.487500\du)--(2.350000\du,-2.487500\du)--cycle;
\pgfsetlinewidth{0.050000\du}
\definecolor{dialinecolor}{rgb}{0.000000, 0.000000, 0.000000}
\pgfsetstrokecolor{dialinecolor}
\draw (6.885000\du,-4.187500\du)--(6.885000\du,-3.587500\du)--(7.485000\du,-3.587500\du);
% setfont left to latex
\definecolor{dialinecolor}{rgb}{0.000000, 0.000000, 0.000000}
\pgfsetstrokecolor{dialinecolor}
\node[anchor=west] at (2.700000\du,-2.942500\du){Interface  };
\end{tikzpicture}

\end{center}
\lstinputlisting[language=java]{9.java}	

\subsection{Héritage versus composition}
\paragraph{héritage} statique (définit à la compilation). 
\paragraph{Composition} dynamique (à l'exécution)
\paragraph{}
Avec la composition possibilité de combiner dynamiquement des objets entre eux.

\subsubsection{Exemple}
\paragraph{} Les produits laitiers avec uniquement l'héritage
\begin{center}
	% Graphic for TeX using PGF
% Title: /home/satenske/cours/AP/obj3/uml19.dia
% Creator: Dia v0.97.1
% CreationDate: Thu Sep 22 10:31:27 2011
% For: satenske
% \usepackage{tikz}
% The following commands are not supported in PSTricks at present
% We define them conditionally, so when they are implemented,
% this pgf file will use them.
\ifx\du\undefined
  \newlength{\du}
\fi
\setlength{\du}{15\unitlength}
\begin{tikzpicture}
\pgftransformxscale{1.000000}
\pgftransformyscale{-1.000000}
\definecolor{dialinecolor}{rgb}{0.000000, 0.000000, 0.000000}
\pgfsetstrokecolor{dialinecolor}
\definecolor{dialinecolor}{rgb}{1.000000, 1.000000, 1.000000}
\pgfsetfillcolor{dialinecolor}
\pgfsetlinewidth{0.100000\du}
\pgfsetdash{}{0pt}
\definecolor{dialinecolor}{rgb}{1.000000, 1.000000, 1.000000}
\pgfsetfillcolor{dialinecolor}
\fill (16.775000\du,-1.635000\du)--(16.775000\du,-0.235000\du)--(23.805000\du,-0.235000\du)--(23.805000\du,-1.635000\du)--cycle;
\definecolor{dialinecolor}{rgb}{0.000000, 0.000000, 0.000000}
\pgfsetstrokecolor{dialinecolor}
\draw (16.775000\du,-1.635000\du)--(16.775000\du,-0.235000\du)--(23.805000\du,-0.235000\du)--(23.805000\du,-1.635000\du)--cycle;
% setfont left to latex
\definecolor{dialinecolor}{rgb}{0.000000, 0.000000, 0.000000}
\pgfsetstrokecolor{dialinecolor}
\node at (20.290000\du,-0.685000\du){Produit Laitier};
\definecolor{dialinecolor}{rgb}{1.000000, 1.000000, 1.000000}
\pgfsetfillcolor{dialinecolor}
\fill (16.775000\du,-0.235000\du)--(16.775000\du,0.165000\du)--(23.805000\du,0.165000\du)--(23.805000\du,-0.235000\du)--cycle;
\definecolor{dialinecolor}{rgb}{0.000000, 0.000000, 0.000000}
\pgfsetstrokecolor{dialinecolor}
\draw (16.775000\du,-0.235000\du)--(16.775000\du,0.165000\du)--(23.805000\du,0.165000\du)--(23.805000\du,-0.235000\du)--cycle;
\definecolor{dialinecolor}{rgb}{1.000000, 1.000000, 1.000000}
\pgfsetfillcolor{dialinecolor}
\fill (16.775000\du,0.165000\du)--(16.775000\du,0.565000\du)--(23.805000\du,0.565000\du)--(23.805000\du,0.165000\du)--cycle;
\definecolor{dialinecolor}{rgb}{0.000000, 0.000000, 0.000000}
\pgfsetstrokecolor{dialinecolor}
\draw (16.775000\du,0.165000\du)--(16.775000\du,0.565000\du)--(23.805000\du,0.565000\du)--(23.805000\du,0.165000\du)--cycle;
\pgfsetlinewidth{0.100000\du}
\pgfsetdash{}{0pt}
\definecolor{dialinecolor}{rgb}{1.000000, 1.000000, 1.000000}
\pgfsetfillcolor{dialinecolor}
\fill (14.665000\du,5.260000\du)--(14.665000\du,6.660000\du)--(16.872500\du,6.660000\du)--(16.872500\du,5.260000\du)--cycle;
\definecolor{dialinecolor}{rgb}{0.000000, 0.000000, 0.000000}
\pgfsetstrokecolor{dialinecolor}
\draw (14.665000\du,5.260000\du)--(14.665000\du,6.660000\du)--(16.872500\du,6.660000\du)--(16.872500\du,5.260000\du)--cycle;
% setfont left to latex
\definecolor{dialinecolor}{rgb}{0.000000, 0.000000, 0.000000}
\pgfsetstrokecolor{dialinecolor}
\node at (15.768750\du,6.210000\du){Lait};
\definecolor{dialinecolor}{rgb}{1.000000, 1.000000, 1.000000}
\pgfsetfillcolor{dialinecolor}
\fill (14.665000\du,6.660000\du)--(14.665000\du,7.060000\du)--(16.872500\du,7.060000\du)--(16.872500\du,6.660000\du)--cycle;
\definecolor{dialinecolor}{rgb}{0.000000, 0.000000, 0.000000}
\pgfsetstrokecolor{dialinecolor}
\draw (14.665000\du,6.660000\du)--(14.665000\du,7.060000\du)--(16.872500\du,7.060000\du)--(16.872500\du,6.660000\du)--cycle;
\definecolor{dialinecolor}{rgb}{1.000000, 1.000000, 1.000000}
\pgfsetfillcolor{dialinecolor}
\fill (14.665000\du,7.060000\du)--(14.665000\du,7.460000\du)--(16.872500\du,7.460000\du)--(16.872500\du,7.060000\du)--cycle;
\definecolor{dialinecolor}{rgb}{0.000000, 0.000000, 0.000000}
\pgfsetstrokecolor{dialinecolor}
\draw (14.665000\du,7.060000\du)--(14.665000\du,7.460000\du)--(16.872500\du,7.460000\du)--(16.872500\du,7.060000\du)--cycle;
\pgfsetlinewidth{0.100000\du}
\pgfsetdash{}{0pt}
\definecolor{dialinecolor}{rgb}{1.000000, 1.000000, 1.000000}
\pgfsetfillcolor{dialinecolor}
\fill (29.065000\du,5.260000\du)--(29.065000\du,6.660000\du)--(32.510000\du,6.660000\du)--(32.510000\du,5.260000\du)--cycle;
\definecolor{dialinecolor}{rgb}{0.000000, 0.000000, 0.000000}
\pgfsetstrokecolor{dialinecolor}
\draw (29.065000\du,5.260000\du)--(29.065000\du,6.660000\du)--(32.510000\du,6.660000\du)--(32.510000\du,5.260000\du)--cycle;
% setfont left to latex
\definecolor{dialinecolor}{rgb}{0.000000, 0.000000, 0.000000}
\pgfsetstrokecolor{dialinecolor}
\node at (30.787500\du,6.210000\du){Yaourt};
\definecolor{dialinecolor}{rgb}{1.000000, 1.000000, 1.000000}
\pgfsetfillcolor{dialinecolor}
\fill (29.065000\du,6.660000\du)--(29.065000\du,7.060000\du)--(32.510000\du,7.060000\du)--(32.510000\du,6.660000\du)--cycle;
\definecolor{dialinecolor}{rgb}{0.000000, 0.000000, 0.000000}
\pgfsetstrokecolor{dialinecolor}
\draw (29.065000\du,6.660000\du)--(29.065000\du,7.060000\du)--(32.510000\du,7.060000\du)--(32.510000\du,6.660000\du)--cycle;
\definecolor{dialinecolor}{rgb}{1.000000, 1.000000, 1.000000}
\pgfsetfillcolor{dialinecolor}
\fill (29.065000\du,7.060000\du)--(29.065000\du,7.460000\du)--(32.510000\du,7.460000\du)--(32.510000\du,7.060000\du)--cycle;
\definecolor{dialinecolor}{rgb}{0.000000, 0.000000, 0.000000}
\pgfsetstrokecolor{dialinecolor}
\draw (29.065000\du,7.060000\du)--(29.065000\du,7.460000\du)--(32.510000\du,7.460000\du)--(32.510000\du,7.060000\du)--cycle;
\pgfsetlinewidth{0.100000\du}
\pgfsetdash{}{0pt}
\pgfsetmiterjoin
\pgfsetbuttcap
{
\definecolor{dialinecolor}{rgb}{0.000000, 0.000000, 0.000000}
\pgfsetfillcolor{dialinecolor}
% was here!!!
\definecolor{dialinecolor}{rgb}{0.000000, 0.000000, 0.000000}
\pgfsetstrokecolor{dialinecolor}
\draw (20.290000\du,0.615281\du)--(20.290000\du,2.912500\du)--(15.768750\du,2.912500\du)--(15.768750\du,5.209719\du);
}
\definecolor{dialinecolor}{rgb}{0.000000, 0.000000, 0.000000}
\pgfsetstrokecolor{dialinecolor}
\draw (20.290000\du,1.527084\du)--(20.290000\du,2.912500\du)--(15.768750\du,2.912500\du)--(15.768750\du,5.209719\du);
\pgfsetmiterjoin
\definecolor{dialinecolor}{rgb}{1.000000, 1.000000, 1.000000}
\pgfsetfillcolor{dialinecolor}
\fill (20.690000\du,1.527084\du)--(20.290000\du,0.727084\du)--(19.890000\du,1.527084\du)--cycle;
\pgfsetlinewidth{0.100000\du}
\pgfsetdash{}{0pt}
\pgfsetmiterjoin
\definecolor{dialinecolor}{rgb}{0.000000, 0.000000, 0.000000}
\pgfsetstrokecolor{dialinecolor}
\draw (20.690000\du,1.527084\du)--(20.290000\du,0.727084\du)--(19.890000\du,1.527084\du)--cycle;
% setfont left to latex
\pgfsetlinewidth{0.100000\du}
\pgfsetdash{}{0pt}
\pgfsetmiterjoin
\pgfsetbuttcap
{
\definecolor{dialinecolor}{rgb}{0.000000, 0.000000, 0.000000}
\pgfsetfillcolor{dialinecolor}
% was here!!!
\definecolor{dialinecolor}{rgb}{0.000000, 0.000000, 0.000000}
\pgfsetstrokecolor{dialinecolor}
\draw (20.290000\du,0.615281\du)--(20.290000\du,2.912500\du)--(30.787500\du,2.912500\du)--(30.787500\du,5.209719\du);
}
\definecolor{dialinecolor}{rgb}{0.000000, 0.000000, 0.000000}
\pgfsetstrokecolor{dialinecolor}
\draw (20.290000\du,1.527084\du)--(20.290000\du,2.912500\du)--(30.787500\du,2.912500\du)--(30.787500\du,5.209719\du);
\pgfsetmiterjoin
\definecolor{dialinecolor}{rgb}{1.000000, 1.000000, 1.000000}
\pgfsetfillcolor{dialinecolor}
\fill (20.690000\du,1.527084\du)--(20.290000\du,0.727084\du)--(19.890000\du,1.527084\du)--cycle;
\pgfsetlinewidth{0.100000\du}
\pgfsetdash{}{0pt}
\pgfsetmiterjoin
\definecolor{dialinecolor}{rgb}{0.000000, 0.000000, 0.000000}
\pgfsetstrokecolor{dialinecolor}
\draw (20.690000\du,1.527084\du)--(20.290000\du,0.727084\du)--(19.890000\du,1.527084\du)--cycle;
% setfont left to latex
\pgfsetlinewidth{0.100000\du}
\pgfsetdash{}{0pt}
\definecolor{dialinecolor}{rgb}{1.000000, 1.000000, 1.000000}
\pgfsetfillcolor{dialinecolor}
\fill (5.115000\du,10.335000\du)--(5.115000\du,11.735000\du)--(12.975000\du,11.735000\du)--(12.975000\du,10.335000\du)--cycle;
\definecolor{dialinecolor}{rgb}{0.000000, 0.000000, 0.000000}
\pgfsetstrokecolor{dialinecolor}
\draw (5.115000\du,10.335000\du)--(5.115000\du,11.735000\du)--(12.975000\du,11.735000\du)--(12.975000\du,10.335000\du)--cycle;
% setfont left to latex
\definecolor{dialinecolor}{rgb}{0.000000, 0.000000, 0.000000}
\pgfsetstrokecolor{dialinecolor}
\node at (9.045000\du,11.285000\du){Lait en bouteille};
\definecolor{dialinecolor}{rgb}{1.000000, 1.000000, 1.000000}
\pgfsetfillcolor{dialinecolor}
\fill (5.115000\du,11.735000\du)--(5.115000\du,12.135000\du)--(12.975000\du,12.135000\du)--(12.975000\du,11.735000\du)--cycle;
\definecolor{dialinecolor}{rgb}{0.000000, 0.000000, 0.000000}
\pgfsetstrokecolor{dialinecolor}
\draw (5.115000\du,11.735000\du)--(5.115000\du,12.135000\du)--(12.975000\du,12.135000\du)--(12.975000\du,11.735000\du)--cycle;
\definecolor{dialinecolor}{rgb}{1.000000, 1.000000, 1.000000}
\pgfsetfillcolor{dialinecolor}
\fill (5.115000\du,12.135000\du)--(5.115000\du,12.535000\du)--(12.975000\du,12.535000\du)--(12.975000\du,12.135000\du)--cycle;
\definecolor{dialinecolor}{rgb}{0.000000, 0.000000, 0.000000}
\pgfsetstrokecolor{dialinecolor}
\draw (5.115000\du,12.135000\du)--(5.115000\du,12.535000\du)--(12.975000\du,12.535000\du)--(12.975000\du,12.135000\du)--cycle;
\pgfsetlinewidth{0.100000\du}
\pgfsetdash{}{0pt}
\definecolor{dialinecolor}{rgb}{1.000000, 1.000000, 1.000000}
\pgfsetfillcolor{dialinecolor}
\fill (32.030000\du,10.145000\du)--(32.030000\du,11.545000\du)--(38.920000\du,11.545000\du)--(38.920000\du,10.145000\du)--cycle;
\definecolor{dialinecolor}{rgb}{0.000000, 0.000000, 0.000000}
\pgfsetstrokecolor{dialinecolor}
\draw (32.030000\du,10.145000\du)--(32.030000\du,11.545000\du)--(38.920000\du,11.545000\du)--(38.920000\du,10.145000\du)--cycle;
% setfont left to latex
\definecolor{dialinecolor}{rgb}{0.000000, 0.000000, 0.000000}
\pgfsetstrokecolor{dialinecolor}
\node at (35.475000\du,11.095000\du){yaourt à boire};
\definecolor{dialinecolor}{rgb}{1.000000, 1.000000, 1.000000}
\pgfsetfillcolor{dialinecolor}
\fill (32.030000\du,11.545000\du)--(32.030000\du,11.945000\du)--(38.920000\du,11.945000\du)--(38.920000\du,11.545000\du)--cycle;
\definecolor{dialinecolor}{rgb}{0.000000, 0.000000, 0.000000}
\pgfsetstrokecolor{dialinecolor}
\draw (32.030000\du,11.545000\du)--(32.030000\du,11.945000\du)--(38.920000\du,11.945000\du)--(38.920000\du,11.545000\du)--cycle;
\definecolor{dialinecolor}{rgb}{1.000000, 1.000000, 1.000000}
\pgfsetfillcolor{dialinecolor}
\fill (32.030000\du,11.945000\du)--(32.030000\du,12.345000\du)--(38.920000\du,12.345000\du)--(38.920000\du,11.945000\du)--cycle;
\definecolor{dialinecolor}{rgb}{0.000000, 0.000000, 0.000000}
\pgfsetstrokecolor{dialinecolor}
\draw (32.030000\du,11.945000\du)--(32.030000\du,12.345000\du)--(38.920000\du,12.345000\du)--(38.920000\du,11.945000\du)--cycle;
\pgfsetlinewidth{0.100000\du}
\pgfsetdash{}{0pt}
\definecolor{dialinecolor}{rgb}{1.000000, 1.000000, 1.000000}
\pgfsetfillcolor{dialinecolor}
\fill (24.245000\du,10.205000\du)--(24.245000\du,11.605000\du)--(30.877500\du,11.605000\du)--(30.877500\du,10.205000\du)--cycle;
\definecolor{dialinecolor}{rgb}{0.000000, 0.000000, 0.000000}
\pgfsetstrokecolor{dialinecolor}
\draw (24.245000\du,10.205000\du)--(24.245000\du,11.605000\du)--(30.877500\du,11.605000\du)--(30.877500\du,10.205000\du)--cycle;
% setfont left to latex
\definecolor{dialinecolor}{rgb}{0.000000, 0.000000, 0.000000}
\pgfsetstrokecolor{dialinecolor}
\node at (27.561250\du,11.155000\du){yaourt en pot};
\definecolor{dialinecolor}{rgb}{1.000000, 1.000000, 1.000000}
\pgfsetfillcolor{dialinecolor}
\fill (24.245000\du,11.605000\du)--(24.245000\du,12.005000\du)--(30.877500\du,12.005000\du)--(30.877500\du,11.605000\du)--cycle;
\definecolor{dialinecolor}{rgb}{0.000000, 0.000000, 0.000000}
\pgfsetstrokecolor{dialinecolor}
\draw (24.245000\du,11.605000\du)--(24.245000\du,12.005000\du)--(30.877500\du,12.005000\du)--(30.877500\du,11.605000\du)--cycle;
\definecolor{dialinecolor}{rgb}{1.000000, 1.000000, 1.000000}
\pgfsetfillcolor{dialinecolor}
\fill (24.245000\du,12.005000\du)--(24.245000\du,12.405000\du)--(30.877500\du,12.405000\du)--(30.877500\du,12.005000\du)--cycle;
\definecolor{dialinecolor}{rgb}{0.000000, 0.000000, 0.000000}
\pgfsetstrokecolor{dialinecolor}
\draw (24.245000\du,12.005000\du)--(24.245000\du,12.405000\du)--(30.877500\du,12.405000\du)--(30.877500\du,12.005000\du)--cycle;
\pgfsetlinewidth{0.100000\du}
\pgfsetdash{}{0pt}
\definecolor{dialinecolor}{rgb}{1.000000, 1.000000, 1.000000}
\pgfsetfillcolor{dialinecolor}
\fill (14.110000\du,10.315000\du)--(14.110000\du,11.715000\du)--(20.912500\du,11.715000\du)--(20.912500\du,10.315000\du)--cycle;
\definecolor{dialinecolor}{rgb}{0.000000, 0.000000, 0.000000}
\pgfsetstrokecolor{dialinecolor}
\draw (14.110000\du,10.315000\du)--(14.110000\du,11.715000\du)--(20.912500\du,11.715000\du)--(20.912500\du,10.315000\du)--cycle;
% setfont left to latex
\definecolor{dialinecolor}{rgb}{0.000000, 0.000000, 0.000000}
\pgfsetstrokecolor{dialinecolor}
\node at (17.511250\du,11.265000\du){Lait en brique};
\definecolor{dialinecolor}{rgb}{1.000000, 1.000000, 1.000000}
\pgfsetfillcolor{dialinecolor}
\fill (14.110000\du,11.715000\du)--(14.110000\du,12.115000\du)--(20.912500\du,12.115000\du)--(20.912500\du,11.715000\du)--cycle;
\definecolor{dialinecolor}{rgb}{0.000000, 0.000000, 0.000000}
\pgfsetstrokecolor{dialinecolor}
\draw (14.110000\du,11.715000\du)--(14.110000\du,12.115000\du)--(20.912500\du,12.115000\du)--(20.912500\du,11.715000\du)--cycle;
\definecolor{dialinecolor}{rgb}{1.000000, 1.000000, 1.000000}
\pgfsetfillcolor{dialinecolor}
\fill (14.110000\du,12.115000\du)--(14.110000\du,12.515000\du)--(20.912500\du,12.515000\du)--(20.912500\du,12.115000\du)--cycle;
\definecolor{dialinecolor}{rgb}{0.000000, 0.000000, 0.000000}
\pgfsetstrokecolor{dialinecolor}
\draw (14.110000\du,12.115000\du)--(14.110000\du,12.515000\du)--(20.912500\du,12.515000\du)--(20.912500\du,12.115000\du)--cycle;
\pgfsetlinewidth{0.100000\du}
\pgfsetdash{}{0pt}
\pgfsetmiterjoin
\pgfsetbuttcap
{
\definecolor{dialinecolor}{rgb}{0.000000, 0.000000, 0.000000}
\pgfsetfillcolor{dialinecolor}
% was here!!!
\definecolor{dialinecolor}{rgb}{0.000000, 0.000000, 0.000000}
\pgfsetstrokecolor{dialinecolor}
\draw (15.768750\du,7.510281\du)--(15.768750\du,8.897500\du)--(9.045000\du,8.897500\du)--(9.045000\du,10.284719\du);
}
\definecolor{dialinecolor}{rgb}{0.000000, 0.000000, 0.000000}
\pgfsetstrokecolor{dialinecolor}
\draw (15.768750\du,8.422084\du)--(15.768750\du,8.897500\du)--(9.045000\du,8.897500\du)--(9.045000\du,10.284719\du);
\pgfsetmiterjoin
\definecolor{dialinecolor}{rgb}{1.000000, 1.000000, 1.000000}
\pgfsetfillcolor{dialinecolor}
\fill (16.168750\du,8.422084\du)--(15.768750\du,7.622084\du)--(15.368750\du,8.422084\du)--cycle;
\pgfsetlinewidth{0.100000\du}
\pgfsetdash{}{0pt}
\pgfsetmiterjoin
\definecolor{dialinecolor}{rgb}{0.000000, 0.000000, 0.000000}
\pgfsetstrokecolor{dialinecolor}
\draw (16.168750\du,8.422084\du)--(15.768750\du,7.622084\du)--(15.368750\du,8.422084\du)--cycle;
% setfont left to latex
\pgfsetlinewidth{0.100000\du}
\pgfsetdash{}{0pt}
\pgfsetmiterjoin
\pgfsetbuttcap
{
\definecolor{dialinecolor}{rgb}{0.000000, 0.000000, 0.000000}
\pgfsetfillcolor{dialinecolor}
% was here!!!
\definecolor{dialinecolor}{rgb}{0.000000, 0.000000, 0.000000}
\pgfsetstrokecolor{dialinecolor}
\draw (15.768750\du,7.510281\du)--(15.768750\du,8.887500\du)--(17.511250\du,8.887500\du)--(17.511250\du,10.264719\du);
}
\definecolor{dialinecolor}{rgb}{0.000000, 0.000000, 0.000000}
\pgfsetstrokecolor{dialinecolor}
\draw (15.768750\du,8.422084\du)--(15.768750\du,8.887500\du)--(17.511250\du,8.887500\du)--(17.511250\du,10.264719\du);
\pgfsetmiterjoin
\definecolor{dialinecolor}{rgb}{1.000000, 1.000000, 1.000000}
\pgfsetfillcolor{dialinecolor}
\fill (16.168750\du,8.422084\du)--(15.768750\du,7.622084\du)--(15.368750\du,8.422084\du)--cycle;
\pgfsetlinewidth{0.100000\du}
\pgfsetdash{}{0pt}
\pgfsetmiterjoin
\definecolor{dialinecolor}{rgb}{0.000000, 0.000000, 0.000000}
\pgfsetstrokecolor{dialinecolor}
\draw (16.168750\du,8.422084\du)--(15.768750\du,7.622084\du)--(15.368750\du,8.422084\du)--cycle;
% setfont left to latex
\pgfsetlinewidth{0.100000\du}
\pgfsetdash{}{0pt}
\pgfsetmiterjoin
\pgfsetbuttcap
{
\definecolor{dialinecolor}{rgb}{0.000000, 0.000000, 0.000000}
\pgfsetfillcolor{dialinecolor}
% was here!!!
\definecolor{dialinecolor}{rgb}{0.000000, 0.000000, 0.000000}
\pgfsetstrokecolor{dialinecolor}
\draw (30.787500\du,7.510281\du)--(30.787500\du,8.832500\du)--(27.561250\du,8.832500\du)--(27.561250\du,10.154719\du);
}
\definecolor{dialinecolor}{rgb}{0.000000, 0.000000, 0.000000}
\pgfsetstrokecolor{dialinecolor}
\draw (30.787500\du,8.422084\du)--(30.787500\du,8.832500\du)--(27.561250\du,8.832500\du)--(27.561250\du,10.154719\du);
\pgfsetmiterjoin
\definecolor{dialinecolor}{rgb}{1.000000, 1.000000, 1.000000}
\pgfsetfillcolor{dialinecolor}
\fill (31.187500\du,8.422084\du)--(30.787500\du,7.622084\du)--(30.387500\du,8.422084\du)--cycle;
\pgfsetlinewidth{0.100000\du}
\pgfsetdash{}{0pt}
\pgfsetmiterjoin
\definecolor{dialinecolor}{rgb}{0.000000, 0.000000, 0.000000}
\pgfsetstrokecolor{dialinecolor}
\draw (31.187500\du,8.422084\du)--(30.787500\du,7.622084\du)--(30.387500\du,8.422084\du)--cycle;
% setfont left to latex
\pgfsetlinewidth{0.100000\du}
\pgfsetdash{}{0pt}
\pgfsetmiterjoin
\pgfsetbuttcap
{
\definecolor{dialinecolor}{rgb}{0.000000, 0.000000, 0.000000}
\pgfsetfillcolor{dialinecolor}
% was here!!!
\definecolor{dialinecolor}{rgb}{0.000000, 0.000000, 0.000000}
\pgfsetstrokecolor{dialinecolor}
\draw (30.787500\du,7.510281\du)--(30.787500\du,8.802500\du)--(35.475000\du,8.802500\du)--(35.475000\du,10.094719\du);
}
\definecolor{dialinecolor}{rgb}{0.000000, 0.000000, 0.000000}
\pgfsetstrokecolor{dialinecolor}
\draw (30.787500\du,8.422084\du)--(30.787500\du,8.802500\du)--(35.475000\du,8.802500\du)--(35.475000\du,10.094719\du);
\pgfsetmiterjoin
\definecolor{dialinecolor}{rgb}{1.000000, 1.000000, 1.000000}
\pgfsetfillcolor{dialinecolor}
\fill (31.187500\du,8.422084\du)--(30.787500\du,7.622084\du)--(30.387500\du,8.422084\du)--cycle;
\pgfsetlinewidth{0.100000\du}
\pgfsetdash{}{0pt}
\pgfsetmiterjoin
\definecolor{dialinecolor}{rgb}{0.000000, 0.000000, 0.000000}
\pgfsetstrokecolor{dialinecolor}
\draw (31.187500\du,8.422084\du)--(30.787500\du,7.622084\du)--(30.387500\du,8.422084\du)--cycle;
% setfont left to latex
\end{tikzpicture}

\end{center}

\paragraph{} Les produits laitiers avec deux hiérarchies d'héritage: le produit et son conditionnement. 
\begin{center}
	% Graphic for TeX using PGF
% Title: /home/satenske/cours/AP/obj3/uml19.dia
% Creator: Dia v0.97.1
% CreationDate: Thu Sep 22 10:27:24 2011
% For: satenske
% \usepackage{tikz}
% The following commands are not supported in PSTricks at present
% We define them conditionally, so when they are implemented,
% this pgf file will use them.
\ifx\du\undefined
  \newlength{\du}
\fi
\setlength{\du}{15\unitlength}
\begin{tikzpicture}
\pgftransformxscale{1.000000}
\pgftransformyscale{-1.000000}
\definecolor{dialinecolor}{rgb}{0.000000, 0.000000, 0.000000}
\pgfsetstrokecolor{dialinecolor}
\definecolor{dialinecolor}{rgb}{1.000000, 1.000000, 1.000000}
\pgfsetfillcolor{dialinecolor}
\pgfsetlinewidth{0.100000\du}
\pgfsetdash{}{0pt}
\definecolor{dialinecolor}{rgb}{1.000000, 1.000000, 1.000000}
\pgfsetfillcolor{dialinecolor}
\fill (7.525000\du,4.865000\du)--(7.525000\du,6.265000\du)--(14.555000\du,6.265000\du)--(14.555000\du,4.865000\du)--cycle;
\definecolor{dialinecolor}{rgb}{0.000000, 0.000000, 0.000000}
\pgfsetstrokecolor{dialinecolor}
\draw (7.525000\du,4.865000\du)--(7.525000\du,6.265000\du)--(14.555000\du,6.265000\du)--(14.555000\du,4.865000\du)--cycle;
% setfont left to latex
\definecolor{dialinecolor}{rgb}{0.000000, 0.000000, 0.000000}
\pgfsetstrokecolor{dialinecolor}
\node at (11.040000\du,5.815000\du){Produit Laitier};
\definecolor{dialinecolor}{rgb}{1.000000, 1.000000, 1.000000}
\pgfsetfillcolor{dialinecolor}
\fill (7.525000\du,6.265000\du)--(7.525000\du,6.665000\du)--(14.555000\du,6.665000\du)--(14.555000\du,6.265000\du)--cycle;
\definecolor{dialinecolor}{rgb}{0.000000, 0.000000, 0.000000}
\pgfsetstrokecolor{dialinecolor}
\draw (7.525000\du,6.265000\du)--(7.525000\du,6.665000\du)--(14.555000\du,6.665000\du)--(14.555000\du,6.265000\du)--cycle;
\definecolor{dialinecolor}{rgb}{1.000000, 1.000000, 1.000000}
\pgfsetfillcolor{dialinecolor}
\fill (7.525000\du,6.665000\du)--(7.525000\du,7.065000\du)--(14.555000\du,7.065000\du)--(14.555000\du,6.665000\du)--cycle;
\definecolor{dialinecolor}{rgb}{0.000000, 0.000000, 0.000000}
\pgfsetstrokecolor{dialinecolor}
\draw (7.525000\du,6.665000\du)--(7.525000\du,7.065000\du)--(14.555000\du,7.065000\du)--(14.555000\du,6.665000\du)--cycle;
\pgfsetlinewidth{0.100000\du}
\pgfsetdash{}{0pt}
\definecolor{dialinecolor}{rgb}{1.000000, 1.000000, 1.000000}
\pgfsetfillcolor{dialinecolor}
\fill (6.815000\du,11.210000\du)--(6.815000\du,12.610000\du)--(9.022500\du,12.610000\du)--(9.022500\du,11.210000\du)--cycle;
\definecolor{dialinecolor}{rgb}{0.000000, 0.000000, 0.000000}
\pgfsetstrokecolor{dialinecolor}
\draw (6.815000\du,11.210000\du)--(6.815000\du,12.610000\du)--(9.022500\du,12.610000\du)--(9.022500\du,11.210000\du)--cycle;
% setfont left to latex
\definecolor{dialinecolor}{rgb}{0.000000, 0.000000, 0.000000}
\pgfsetstrokecolor{dialinecolor}
\node at (7.918750\du,12.160000\du){Lait};
\definecolor{dialinecolor}{rgb}{1.000000, 1.000000, 1.000000}
\pgfsetfillcolor{dialinecolor}
\fill (6.815000\du,12.610000\du)--(6.815000\du,13.010000\du)--(9.022500\du,13.010000\du)--(9.022500\du,12.610000\du)--cycle;
\definecolor{dialinecolor}{rgb}{0.000000, 0.000000, 0.000000}
\pgfsetstrokecolor{dialinecolor}
\draw (6.815000\du,12.610000\du)--(6.815000\du,13.010000\du)--(9.022500\du,13.010000\du)--(9.022500\du,12.610000\du)--cycle;
\definecolor{dialinecolor}{rgb}{1.000000, 1.000000, 1.000000}
\pgfsetfillcolor{dialinecolor}
\fill (6.815000\du,13.010000\du)--(6.815000\du,13.410000\du)--(9.022500\du,13.410000\du)--(9.022500\du,13.010000\du)--cycle;
\definecolor{dialinecolor}{rgb}{0.000000, 0.000000, 0.000000}
\pgfsetstrokecolor{dialinecolor}
\draw (6.815000\du,13.010000\du)--(6.815000\du,13.410000\du)--(9.022500\du,13.410000\du)--(9.022500\du,13.010000\du)--cycle;
\pgfsetlinewidth{0.100000\du}
\pgfsetdash{}{0pt}
\definecolor{dialinecolor}{rgb}{1.000000, 1.000000, 1.000000}
\pgfsetfillcolor{dialinecolor}
\fill (12.115000\du,11.260000\du)--(12.115000\du,12.660000\du)--(15.560000\du,12.660000\du)--(15.560000\du,11.260000\du)--cycle;
\definecolor{dialinecolor}{rgb}{0.000000, 0.000000, 0.000000}
\pgfsetstrokecolor{dialinecolor}
\draw (12.115000\du,11.260000\du)--(12.115000\du,12.660000\du)--(15.560000\du,12.660000\du)--(15.560000\du,11.260000\du)--cycle;
% setfont left to latex
\definecolor{dialinecolor}{rgb}{0.000000, 0.000000, 0.000000}
\pgfsetstrokecolor{dialinecolor}
\node at (13.837500\du,12.210000\du){Yaourt};
\definecolor{dialinecolor}{rgb}{1.000000, 1.000000, 1.000000}
\pgfsetfillcolor{dialinecolor}
\fill (12.115000\du,12.660000\du)--(12.115000\du,13.060000\du)--(15.560000\du,13.060000\du)--(15.560000\du,12.660000\du)--cycle;
\definecolor{dialinecolor}{rgb}{0.000000, 0.000000, 0.000000}
\pgfsetstrokecolor{dialinecolor}
\draw (12.115000\du,12.660000\du)--(12.115000\du,13.060000\du)--(15.560000\du,13.060000\du)--(15.560000\du,12.660000\du)--cycle;
\definecolor{dialinecolor}{rgb}{1.000000, 1.000000, 1.000000}
\pgfsetfillcolor{dialinecolor}
\fill (12.115000\du,13.060000\du)--(12.115000\du,13.460000\du)--(15.560000\du,13.460000\du)--(15.560000\du,13.060000\du)--cycle;
\definecolor{dialinecolor}{rgb}{0.000000, 0.000000, 0.000000}
\pgfsetstrokecolor{dialinecolor}
\draw (12.115000\du,13.060000\du)--(12.115000\du,13.460000\du)--(15.560000\du,13.460000\du)--(15.560000\du,13.060000\du)--cycle;
\pgfsetlinewidth{0.100000\du}
\pgfsetdash{}{0pt}
\definecolor{dialinecolor}{rgb}{1.000000, 1.000000, 1.000000}
\pgfsetfillcolor{dialinecolor}
\fill (18.865000\du,11.110000\du)--(18.865000\du,12.510000\du)--(23.387500\du,12.510000\du)--(23.387500\du,11.110000\du)--cycle;
\definecolor{dialinecolor}{rgb}{0.000000, 0.000000, 0.000000}
\pgfsetstrokecolor{dialinecolor}
\draw (18.865000\du,11.110000\du)--(18.865000\du,12.510000\du)--(23.387500\du,12.510000\du)--(23.387500\du,11.110000\du)--cycle;
% setfont left to latex
\definecolor{dialinecolor}{rgb}{0.000000, 0.000000, 0.000000}
\pgfsetstrokecolor{dialinecolor}
\node at (21.126250\du,12.060000\du){Bouteille};
\definecolor{dialinecolor}{rgb}{1.000000, 1.000000, 1.000000}
\pgfsetfillcolor{dialinecolor}
\fill (18.865000\du,12.510000\du)--(18.865000\du,12.910000\du)--(23.387500\du,12.910000\du)--(23.387500\du,12.510000\du)--cycle;
\definecolor{dialinecolor}{rgb}{0.000000, 0.000000, 0.000000}
\pgfsetstrokecolor{dialinecolor}
\draw (18.865000\du,12.510000\du)--(18.865000\du,12.910000\du)--(23.387500\du,12.910000\du)--(23.387500\du,12.510000\du)--cycle;
\definecolor{dialinecolor}{rgb}{1.000000, 1.000000, 1.000000}
\pgfsetfillcolor{dialinecolor}
\fill (18.865000\du,12.910000\du)--(18.865000\du,13.310000\du)--(23.387500\du,13.310000\du)--(23.387500\du,12.910000\du)--cycle;
\definecolor{dialinecolor}{rgb}{0.000000, 0.000000, 0.000000}
\pgfsetstrokecolor{dialinecolor}
\draw (18.865000\du,12.910000\du)--(18.865000\du,13.310000\du)--(23.387500\du,13.310000\du)--(23.387500\du,12.910000\du)--cycle;
\pgfsetlinewidth{0.100000\du}
\pgfsetdash{}{0pt}
\definecolor{dialinecolor}{rgb}{1.000000, 1.000000, 1.000000}
\pgfsetfillcolor{dialinecolor}
\fill (27.015000\du,11.160000\du)--(27.015000\du,12.560000\du)--(30.480000\du,12.560000\du)--(30.480000\du,11.160000\du)--cycle;
\definecolor{dialinecolor}{rgb}{0.000000, 0.000000, 0.000000}
\pgfsetstrokecolor{dialinecolor}
\draw (27.015000\du,11.160000\du)--(27.015000\du,12.560000\du)--(30.480000\du,12.560000\du)--(30.480000\du,11.160000\du)--cycle;
% setfont left to latex
\definecolor{dialinecolor}{rgb}{0.000000, 0.000000, 0.000000}
\pgfsetstrokecolor{dialinecolor}
\node at (28.747500\du,12.110000\du){Brique};
\definecolor{dialinecolor}{rgb}{1.000000, 1.000000, 1.000000}
\pgfsetfillcolor{dialinecolor}
\fill (27.015000\du,12.560000\du)--(27.015000\du,12.960000\du)--(30.480000\du,12.960000\du)--(30.480000\du,12.560000\du)--cycle;
\definecolor{dialinecolor}{rgb}{0.000000, 0.000000, 0.000000}
\pgfsetstrokecolor{dialinecolor}
\draw (27.015000\du,12.560000\du)--(27.015000\du,12.960000\du)--(30.480000\du,12.960000\du)--(30.480000\du,12.560000\du)--cycle;
\definecolor{dialinecolor}{rgb}{1.000000, 1.000000, 1.000000}
\pgfsetfillcolor{dialinecolor}
\fill (27.015000\du,12.960000\du)--(27.015000\du,13.360000\du)--(30.480000\du,13.360000\du)--(30.480000\du,12.960000\du)--cycle;
\definecolor{dialinecolor}{rgb}{0.000000, 0.000000, 0.000000}
\pgfsetstrokecolor{dialinecolor}
\draw (27.015000\du,12.960000\du)--(27.015000\du,13.360000\du)--(30.480000\du,13.360000\du)--(30.480000\du,12.960000\du)--cycle;
\pgfsetlinewidth{0.100000\du}
\pgfsetdash{}{0pt}
\definecolor{dialinecolor}{rgb}{1.000000, 1.000000, 1.000000}
\pgfsetfillcolor{dialinecolor}
\fill (33.265000\du,11.160000\du)--(33.265000\du,12.560000\du)--(35.285000\du,12.560000\du)--(35.285000\du,11.160000\du)--cycle;
\definecolor{dialinecolor}{rgb}{0.000000, 0.000000, 0.000000}
\pgfsetstrokecolor{dialinecolor}
\draw (33.265000\du,11.160000\du)--(33.265000\du,12.560000\du)--(35.285000\du,12.560000\du)--(35.285000\du,11.160000\du)--cycle;
% setfont left to latex
\definecolor{dialinecolor}{rgb}{0.000000, 0.000000, 0.000000}
\pgfsetstrokecolor{dialinecolor}
\node at (34.275000\du,12.110000\du){Pot};
\definecolor{dialinecolor}{rgb}{1.000000, 1.000000, 1.000000}
\pgfsetfillcolor{dialinecolor}
\fill (33.265000\du,12.560000\du)--(33.265000\du,12.960000\du)--(35.285000\du,12.960000\du)--(35.285000\du,12.560000\du)--cycle;
\definecolor{dialinecolor}{rgb}{0.000000, 0.000000, 0.000000}
\pgfsetstrokecolor{dialinecolor}
\draw (33.265000\du,12.560000\du)--(33.265000\du,12.960000\du)--(35.285000\du,12.960000\du)--(35.285000\du,12.560000\du)--cycle;
\definecolor{dialinecolor}{rgb}{1.000000, 1.000000, 1.000000}
\pgfsetfillcolor{dialinecolor}
\fill (33.265000\du,12.960000\du)--(33.265000\du,13.360000\du)--(35.285000\du,13.360000\du)--(35.285000\du,12.960000\du)--cycle;
\definecolor{dialinecolor}{rgb}{0.000000, 0.000000, 0.000000}
\pgfsetstrokecolor{dialinecolor}
\draw (33.265000\du,12.960000\du)--(33.265000\du,13.360000\du)--(35.285000\du,13.360000\du)--(35.285000\du,12.960000\du)--cycle;
\pgfsetlinewidth{0.100000\du}
\pgfsetdash{}{0pt}
\definecolor{dialinecolor}{rgb}{1.000000, 1.000000, 1.000000}
\pgfsetfillcolor{dialinecolor}
\fill (24.765000\du,4.960000\du)--(24.765000\du,6.360000\du)--(32.762500\du,6.360000\du)--(32.762500\du,4.960000\du)--cycle;
\definecolor{dialinecolor}{rgb}{0.000000, 0.000000, 0.000000}
\pgfsetstrokecolor{dialinecolor}
\draw (24.765000\du,4.960000\du)--(24.765000\du,6.360000\du)--(32.762500\du,6.360000\du)--(32.762500\du,4.960000\du)--cycle;
% setfont left to latex
\definecolor{dialinecolor}{rgb}{0.000000, 0.000000, 0.000000}
\pgfsetstrokecolor{dialinecolor}
\node at (28.763750\du,5.910000\du){Conditonnement};
\definecolor{dialinecolor}{rgb}{1.000000, 1.000000, 1.000000}
\pgfsetfillcolor{dialinecolor}
\fill (24.765000\du,6.360000\du)--(24.765000\du,6.760000\du)--(32.762500\du,6.760000\du)--(32.762500\du,6.360000\du)--cycle;
\definecolor{dialinecolor}{rgb}{0.000000, 0.000000, 0.000000}
\pgfsetstrokecolor{dialinecolor}
\draw (24.765000\du,6.360000\du)--(24.765000\du,6.760000\du)--(32.762500\du,6.760000\du)--(32.762500\du,6.360000\du)--cycle;
\definecolor{dialinecolor}{rgb}{1.000000, 1.000000, 1.000000}
\pgfsetfillcolor{dialinecolor}
\fill (24.765000\du,6.760000\du)--(24.765000\du,7.160000\du)--(32.762500\du,7.160000\du)--(32.762500\du,6.760000\du)--cycle;
\definecolor{dialinecolor}{rgb}{0.000000, 0.000000, 0.000000}
\pgfsetstrokecolor{dialinecolor}
\draw (24.765000\du,6.760000\du)--(24.765000\du,7.160000\du)--(32.762500\du,7.160000\du)--(32.762500\du,6.760000\du)--cycle;
\pgfsetlinewidth{0.100000\du}
\pgfsetdash{}{0pt}
\pgfsetmiterjoin
\pgfsetbuttcap
{
\definecolor{dialinecolor}{rgb}{0.000000, 0.000000, 0.000000}
\pgfsetfillcolor{dialinecolor}
% was here!!!
\definecolor{dialinecolor}{rgb}{0.000000, 0.000000, 0.000000}
\pgfsetstrokecolor{dialinecolor}
\draw (11.040000\du,7.115281\du)--(11.040000\du,9.137500\du)--(7.918750\du,9.137500\du)--(7.918750\du,11.159719\du);
}
\definecolor{dialinecolor}{rgb}{0.000000, 0.000000, 0.000000}
\pgfsetstrokecolor{dialinecolor}
\draw (11.040000\du,8.027084\du)--(11.040000\du,9.137500\du)--(7.918750\du,9.137500\du)--(7.918750\du,11.159719\du);
\pgfsetmiterjoin
\definecolor{dialinecolor}{rgb}{1.000000, 1.000000, 1.000000}
\pgfsetfillcolor{dialinecolor}
\fill (11.440000\du,8.027084\du)--(11.040000\du,7.227084\du)--(10.640000\du,8.027084\du)--cycle;
\pgfsetlinewidth{0.100000\du}
\pgfsetdash{}{0pt}
\pgfsetmiterjoin
\definecolor{dialinecolor}{rgb}{0.000000, 0.000000, 0.000000}
\pgfsetstrokecolor{dialinecolor}
\draw (11.440000\du,8.027084\du)--(11.040000\du,7.227084\du)--(10.640000\du,8.027084\du)--cycle;
% setfont left to latex
\pgfsetlinewidth{0.100000\du}
\pgfsetdash{}{0pt}
\pgfsetmiterjoin
\pgfsetbuttcap
{
\definecolor{dialinecolor}{rgb}{0.000000, 0.000000, 0.000000}
\pgfsetfillcolor{dialinecolor}
% was here!!!
\definecolor{dialinecolor}{rgb}{0.000000, 0.000000, 0.000000}
\pgfsetstrokecolor{dialinecolor}
\draw (11.040000\du,7.115281\du)--(11.040000\du,9.162500\du)--(13.837500\du,9.162500\du)--(13.837500\du,11.209719\du);
}
\definecolor{dialinecolor}{rgb}{0.000000, 0.000000, 0.000000}
\pgfsetstrokecolor{dialinecolor}
\draw (11.040000\du,8.027084\du)--(11.040000\du,9.162500\du)--(13.837500\du,9.162500\du)--(13.837500\du,11.209719\du);
\pgfsetmiterjoin
\definecolor{dialinecolor}{rgb}{1.000000, 1.000000, 1.000000}
\pgfsetfillcolor{dialinecolor}
\fill (11.440000\du,8.027084\du)--(11.040000\du,7.227084\du)--(10.640000\du,8.027084\du)--cycle;
\pgfsetlinewidth{0.100000\du}
\pgfsetdash{}{0pt}
\pgfsetmiterjoin
\definecolor{dialinecolor}{rgb}{0.000000, 0.000000, 0.000000}
\pgfsetstrokecolor{dialinecolor}
\draw (11.440000\du,8.027084\du)--(11.040000\du,7.227084\du)--(10.640000\du,8.027084\du)--cycle;
% setfont left to latex
\pgfsetlinewidth{0.100000\du}
\pgfsetdash{}{0pt}
\pgfsetmiterjoin
\pgfsetbuttcap
{
\definecolor{dialinecolor}{rgb}{0.000000, 0.000000, 0.000000}
\pgfsetfillcolor{dialinecolor}
% was here!!!
\definecolor{dialinecolor}{rgb}{0.000000, 0.000000, 0.000000}
\pgfsetstrokecolor{dialinecolor}
\draw (28.763750\du,7.210281\du)--(28.763750\du,9.135000\du)--(21.126250\du,9.135000\du)--(21.126250\du,11.059719\du);
}
\definecolor{dialinecolor}{rgb}{0.000000, 0.000000, 0.000000}
\pgfsetstrokecolor{dialinecolor}
\draw (28.763750\du,8.122084\du)--(28.763750\du,9.135000\du)--(21.126250\du,9.135000\du)--(21.126250\du,11.059719\du);
\pgfsetmiterjoin
\definecolor{dialinecolor}{rgb}{1.000000, 1.000000, 1.000000}
\pgfsetfillcolor{dialinecolor}
\fill (29.163750\du,8.122084\du)--(28.763750\du,7.322084\du)--(28.363750\du,8.122084\du)--cycle;
\pgfsetlinewidth{0.100000\du}
\pgfsetdash{}{0pt}
\pgfsetmiterjoin
\definecolor{dialinecolor}{rgb}{0.000000, 0.000000, 0.000000}
\pgfsetstrokecolor{dialinecolor}
\draw (29.163750\du,8.122084\du)--(28.763750\du,7.322084\du)--(28.363750\du,8.122084\du)--cycle;
% setfont left to latex
\pgfsetlinewidth{0.100000\du}
\pgfsetdash{}{0pt}
\pgfsetmiterjoin
\pgfsetbuttcap
{
\definecolor{dialinecolor}{rgb}{0.000000, 0.000000, 0.000000}
\pgfsetfillcolor{dialinecolor}
% was here!!!
\definecolor{dialinecolor}{rgb}{0.000000, 0.000000, 0.000000}
\pgfsetstrokecolor{dialinecolor}
\draw (28.763750\du,7.210281\du)--(28.763750\du,9.160000\du)--(28.747500\du,9.160000\du)--(28.747500\du,11.109719\du);
}
\definecolor{dialinecolor}{rgb}{0.000000, 0.000000, 0.000000}
\pgfsetstrokecolor{dialinecolor}
\draw (28.763750\du,8.122084\du)--(28.763750\du,9.160000\du)--(28.747500\du,9.160000\du)--(28.747500\du,11.109719\du);
\pgfsetmiterjoin
\definecolor{dialinecolor}{rgb}{1.000000, 1.000000, 1.000000}
\pgfsetfillcolor{dialinecolor}
\fill (29.163750\du,8.122084\du)--(28.763750\du,7.322084\du)--(28.363750\du,8.122084\du)--cycle;
\pgfsetlinewidth{0.100000\du}
\pgfsetdash{}{0pt}
\pgfsetmiterjoin
\definecolor{dialinecolor}{rgb}{0.000000, 0.000000, 0.000000}
\pgfsetstrokecolor{dialinecolor}
\draw (29.163750\du,8.122084\du)--(28.763750\du,7.322084\du)--(28.363750\du,8.122084\du)--cycle;
% setfont left to latex
\pgfsetlinewidth{0.100000\du}
\pgfsetdash{}{0pt}
\pgfsetmiterjoin
\pgfsetbuttcap
{
\definecolor{dialinecolor}{rgb}{0.000000, 0.000000, 0.000000}
\pgfsetfillcolor{dialinecolor}
% was here!!!
\definecolor{dialinecolor}{rgb}{0.000000, 0.000000, 0.000000}
\pgfsetstrokecolor{dialinecolor}
\draw (28.763750\du,7.210281\du)--(28.763750\du,9.160000\du)--(34.275000\du,9.160000\du)--(34.275000\du,11.109719\du);
}
\definecolor{dialinecolor}{rgb}{0.000000, 0.000000, 0.000000}
\pgfsetstrokecolor{dialinecolor}
\draw (28.763750\du,8.122084\du)--(28.763750\du,9.160000\du)--(34.275000\du,9.160000\du)--(34.275000\du,11.109719\du);
\pgfsetmiterjoin
\definecolor{dialinecolor}{rgb}{1.000000, 1.000000, 1.000000}
\pgfsetfillcolor{dialinecolor}
\fill (29.163750\du,8.122084\du)--(28.763750\du,7.322084\du)--(28.363750\du,8.122084\du)--cycle;
\pgfsetlinewidth{0.100000\du}
\pgfsetdash{}{0pt}
\pgfsetmiterjoin
\definecolor{dialinecolor}{rgb}{0.000000, 0.000000, 0.000000}
\pgfsetstrokecolor{dialinecolor}
\draw (29.163750\du,8.122084\du)--(28.763750\du,7.322084\du)--(28.363750\du,8.122084\du)--cycle;
% setfont left to latex
\pgfsetlinewidth{0.100000\du}
\pgfsetbuttcap
\pgfsetdash{}{0pt}
{
\definecolor{dialinecolor}{rgb}{0.000000, 0.000000, 0.000000}
\pgfsetfillcolor{dialinecolor}
% was here!!!
\pgfsetarrowsstart{latex}
\definecolor{dialinecolor}{rgb}{0.000000, 0.000000, 0.000000}
\pgfsetstrokecolor{dialinecolor}
\draw (24.715219\du,6.038300\du)--(14.588212\du,5.984019\du);
}
% setfont left to latex
\end{tikzpicture}

\end{center}

\subsection{Héritage versus état d'un objet}
Ne pas confondre héritage (classification) et état d'un objet 
(valeurs des champs de l'objet)

