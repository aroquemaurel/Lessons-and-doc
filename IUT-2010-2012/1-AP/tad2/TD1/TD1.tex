\documentclass{article}

\usepackage{lmodern}
\usepackage{xcolor}
\usepackage[utf8]{inputenc}
\usepackage[T1]{fontenc}
\usepackage[francais]{babel}
\usepackage[top=1.7cm, bottom=1.7cm, left=1.7cm, right=1.7cm]{geometry}
%\usepackage[frenchb]{babel}
%\usepackage{layout}
%\usepackage{setspace}
%\usepackage{soul}
%\usepackage{ulem}
%\usepackage{eurosym}
%\usepackage{bookman}
%\usepackage{charter}
%\usepackage{newcent}
%\usepackage{lmodern}
%\usepackage{mathpazo}
%\usepackage{mathptmx}
%\usepackage{url}
%\usepackage{verbatim}
%\usepackage{moreverb}
%\usepackage{wrapfig}
%\usepackage{amsmath}
%\usepackage{mathrsfs}
%\usepackage{asmthm}
%\usepackage{makeidx}
\usepackage{tikz} %Vectoriel
\usepackage{listings}
\usepackage{fancyhdr}
\usepackage{multido}
\usepackage{amssymb}

\definecolor{gris1}{gray}{0.40}
\definecolor{gris2}{gray}{0.55}
\definecolor{gris3}{gray}{0.65}
\definecolor{gris4}{gray}{0.50}


\lstdefinelanguage{algo}{%
   morekeywords={%
    %%% couleur 1
		importer, programme, glossaire, fonction, procedure, constante, type, 
	%%% IMPORT & Co.
		si, sinon, alors, fin, tantque, debut, faire, lorsque, fin lorsque, declancher, enregistrement, tableau, retourne, retourner, =, /=, <, >, traite,exception, 
	%%% types 
		Entier, Reel, Booleen, Caractere,
	%%% types 
		entree, maj, sortie,	
	%%% types 
		et, ou, non,
	},
  sensitive=true,
  morecomment=[l]{--},
  morestring=[b]',
}

%\lstset{language=algo,
    %%% BOUCLE, TEST & Co.
%      emph={importer, programme, glossaire, fonction, procedure, constante, type},
%      emphstyle=\color{gris2},
    %%% IMPORT & Co.
%      emph={[2]si, sinon, alors, fin , tantque, debut, faire, lorsque, fin lorsque, declancher, retourner, et, ou, non,enregistrement, retourner, retourne, tableau, /=, <, =, >, traite,exception},
%      emphstyle=[2]\color{gris1},
    %%% FONCTIONS NUMERIQUES
%      emph={[3]Entier, Reel, Booleen, Caractere},
%      emphstyle=[3]\color{gris3},
    %%% FONCTIONS NUMERIQUES
%      emph={[4]entree, maj, sortie},	
%      emphstyle=[4]\color{gris4},
%}
\lstset{ % general style for listings 
   numbers=left 
	, extendedchars=\true
   , tabsize=2 
   , frame=single 
   , breaklines=true 
   , basicstyle=\ttfamily 
   , numberstyle=\tiny\ttfamily 
   , framexleftmargin=13mm 
   , xleftmargin=12mm 
   , captionpos=b 
	, language=algo
	, keywordstyle=\color{blue}
	, commentstyle=\color{green}
	, showstringspaces=false
	, extendedchars=true
	, mathescape=true
} 
 %prise en charge du langage algo

\title{TD 1\\ Construction dynamique d'une liste}
\date{TAD\\ Semestre 1}

\lhead{TD 1: Construction dynamique d'une liste}
\chead{}
\rhead{\thepage}

\lfoot{Université Paul sabatier Toulouse III}
\cfoot{\thepage}
\rfoot{tad2}

\pagestyle{fancy}
\begin{document}
	\maketitle
	\section{Représentation de la liste}		
		% Graphic for TeX using PGF
% Title: /usr/home/satenske/Diagram1.dia
% Creator: Dia v0.97.1
% CreationDate: Wed Mar 30 09:12:35 2011
% For: satenske
% \usepackage{tikz}
% The following commands are not supported in PSTricks at present
% We define them conditionally, so when they are implemented,
% this pgf file will use them.
\ifx\du\undefined
  \newlength{\du}
\fi
\setlength{\du}{15\unitlength}
\begin{tikzpicture}
\pgftransformxscale{1.000000}
\pgftransformyscale{-1.000000}
\definecolor{dialinecolor}{rgb}{0.000000, 0.000000, 0.000000}
\pgfsetstrokecolor{dialinecolor}
\definecolor{dialinecolor}{rgb}{1.000000, 1.000000, 1.000000}
\pgfsetfillcolor{dialinecolor}
\pgfsetlinewidth{0.100000\du}
\pgfsetdash{}{0pt}
\pgfsetdash{}{0pt}
\pgfsetbuttcap
\pgfsetmiterjoin
\pgfsetlinewidth{0.001000\du}
\pgfsetbuttcap
\pgfsetmiterjoin
\pgfsetdash{}{0pt}
\definecolor{dialinecolor}{rgb}{0.717647, 0.717647, 0.615686}
\pgfsetfillcolor{dialinecolor}
\pgfpathmoveto{\pgfpoint{9.544836\du}{9.940680\du}}
\pgfpathlineto{\pgfpoint{11.543262\du}{9.940680\du}}
\pgfpathlineto{\pgfpoint{11.543262\du}{10.310013\du}}
\pgfpathlineto{\pgfpoint{9.544836\du}{10.310013\du}}
\pgfpathlineto{\pgfpoint{9.544836\du}{9.940680\du}}
\pgfusepath{fill}
\pgfsetbuttcap
\pgfsetmiterjoin
\pgfsetdash{}{0pt}
\definecolor{dialinecolor}{rgb}{0.286275, 0.286275, 0.211765}
\pgfsetstrokecolor{dialinecolor}
\pgfpathmoveto{\pgfpoint{9.544836\du}{9.940680\du}}
\pgfpathlineto{\pgfpoint{11.543262\du}{9.940680\du}}
\pgfpathlineto{\pgfpoint{11.543262\du}{10.310013\du}}
\pgfpathlineto{\pgfpoint{9.544836\du}{10.310013\du}}
\pgfpathlineto{\pgfpoint{9.544836\du}{9.940680\du}}
\pgfusepath{stroke}
\pgfsetbuttcap
\pgfsetmiterjoin
\pgfsetdash{}{0pt}
\definecolor{dialinecolor}{rgb}{0.788235, 0.788235, 0.713726}
\pgfsetfillcolor{dialinecolor}
\pgfpathmoveto{\pgfpoint{9.544836\du}{9.940680\du}}
\pgfpathlineto{\pgfpoint{9.756738\du}{9.739798\du}}
\pgfpathlineto{\pgfpoint{11.755164\du}{9.739798\du}}
\pgfpathlineto{\pgfpoint{11.543262\du}{9.940680\du}}
\pgfpathlineto{\pgfpoint{9.544836\du}{9.940680\du}}
\pgfusepath{fill}
\pgfsetbuttcap
\pgfsetmiterjoin
\pgfsetdash{}{0pt}
\definecolor{dialinecolor}{rgb}{0.286275, 0.286275, 0.211765}
\pgfsetstrokecolor{dialinecolor}
\pgfpathmoveto{\pgfpoint{9.544836\du}{9.940680\du}}
\pgfpathlineto{\pgfpoint{9.756738\du}{9.739798\du}}
\pgfpathlineto{\pgfpoint{11.755164\du}{9.739798\du}}
\pgfpathlineto{\pgfpoint{11.543262\du}{9.940680\du}}
\pgfpathlineto{\pgfpoint{9.544836\du}{9.940680\du}}
\pgfusepath{stroke}
\pgfsetlinewidth{0.106000\du}
\pgfsetbuttcap
\pgfsetmiterjoin
\pgfsetdash{}{0pt}
\definecolor{dialinecolor}{rgb}{0.000000, 0.000000, 0.000000}
\pgfsetstrokecolor{dialinecolor}
\pgfpathmoveto{\pgfpoint{11.431486\du}{10.108816\du}}
\pgfpathlineto{\pgfpoint{10.951952\du}{10.108816\du}}
\pgfusepath{stroke}
\pgfsetlinewidth{0.001000\du}
\pgfsetbuttcap
\pgfsetmiterjoin
\pgfsetdash{}{0pt}
\definecolor{dialinecolor}{rgb}{0.478431, 0.478431, 0.352941}
\pgfsetfillcolor{dialinecolor}
\pgfpathmoveto{\pgfpoint{11.543262\du}{10.310013\du}}
\pgfpathlineto{\pgfpoint{11.755164\du}{10.097481\du}}
\pgfpathlineto{\pgfpoint{11.755164\du}{9.739798\du}}
\pgfpathlineto{\pgfpoint{11.543262\du}{9.940680\du}}
\pgfpathlineto{\pgfpoint{11.543262\du}{10.310013\du}}
\pgfusepath{fill}
\pgfsetbuttcap
\pgfsetmiterjoin
\pgfsetdash{}{0pt}
\definecolor{dialinecolor}{rgb}{0.286275, 0.286275, 0.211765}
\pgfsetstrokecolor{dialinecolor}
\pgfpathmoveto{\pgfpoint{11.543262\du}{10.310013\du}}
\pgfpathlineto{\pgfpoint{11.755164\du}{10.097481\du}}
\pgfpathlineto{\pgfpoint{11.755164\du}{9.739798\du}}
\pgfpathlineto{\pgfpoint{11.543262\du}{9.940680\du}}
\pgfpathlineto{\pgfpoint{11.543262\du}{10.310013\du}}
\pgfusepath{stroke}
\pgfsetbuttcap
\pgfsetmiterjoin
\pgfsetdash{}{0pt}
\definecolor{dialinecolor}{rgb}{0.788235, 0.788235, 0.713726}
\pgfsetfillcolor{dialinecolor}
\pgfpathmoveto{\pgfpoint{9.556171\du}{10.544270\du}}
\pgfpathlineto{\pgfpoint{9.779093\du}{10.264987\du}}
\pgfpathlineto{\pgfpoint{11.320025\du}{10.264987\du}}
\pgfpathlineto{\pgfpoint{11.097103\du}{10.544270\du}}
\pgfpathlineto{\pgfpoint{9.556171\du}{10.544270\du}}
\pgfusepath{fill}
\pgfsetbuttcap
\pgfsetmiterjoin
\pgfsetdash{}{0pt}
\definecolor{dialinecolor}{rgb}{0.286275, 0.286275, 0.211765}
\pgfsetstrokecolor{dialinecolor}
\pgfpathmoveto{\pgfpoint{9.556171\du}{10.544270\du}}
\pgfpathlineto{\pgfpoint{9.779093\du}{10.264987\du}}
\pgfpathlineto{\pgfpoint{11.320025\du}{10.264987\du}}
\pgfpathlineto{\pgfpoint{11.097103\du}{10.544270\du}}
\pgfpathlineto{\pgfpoint{9.556171\du}{10.544270\du}}
\pgfusepath{stroke}
\pgfsetbuttcap
\pgfsetmiterjoin
\pgfsetdash{}{0pt}
\definecolor{dialinecolor}{rgb}{0.478431, 0.478431, 0.352941}
\pgfsetfillcolor{dialinecolor}
\pgfpathmoveto{\pgfpoint{11.097103\du}{10.600000\du}}
\pgfpathlineto{\pgfpoint{11.320025\du}{10.365743\du}}
\pgfpathlineto{\pgfpoint{11.320025\du}{10.264987\du}}
\pgfpathlineto{\pgfpoint{11.097103\du}{10.544270\du}}
\pgfpathlineto{\pgfpoint{11.097103\du}{10.600000\du}}
\pgfusepath{fill}
\pgfsetbuttcap
\pgfsetmiterjoin
\pgfsetdash{}{0pt}
\definecolor{dialinecolor}{rgb}{0.286275, 0.286275, 0.211765}
\pgfsetstrokecolor{dialinecolor}
\pgfpathmoveto{\pgfpoint{11.097103\du}{10.600000\du}}
\pgfpathlineto{\pgfpoint{11.320025\du}{10.365743\du}}
\pgfpathlineto{\pgfpoint{11.320025\du}{10.264987\du}}
\pgfpathlineto{\pgfpoint{11.097103\du}{10.544270\du}}
\pgfpathlineto{\pgfpoint{11.097103\du}{10.600000\du}}
\pgfusepath{stroke}
\pgfsetbuttcap
\pgfsetmiterjoin
\pgfsetdash{}{0pt}
\definecolor{dialinecolor}{rgb}{0.717647, 0.717647, 0.615686}
\pgfsetfillcolor{dialinecolor}
\pgfpathmoveto{\pgfpoint{9.556171\du}{10.544270\du}}
\pgfpathlineto{\pgfpoint{11.097103\du}{10.544270\du}}
\pgfpathlineto{\pgfpoint{11.097103\du}{10.600000\du}}
\pgfpathlineto{\pgfpoint{9.556171\du}{10.600000\du}}
\pgfpathlineto{\pgfpoint{9.556171\du}{10.544270\du}}
\pgfusepath{fill}
\pgfsetbuttcap
\pgfsetmiterjoin
\pgfsetdash{}{0pt}
\definecolor{dialinecolor}{rgb}{0.286275, 0.286275, 0.211765}
\pgfsetstrokecolor{dialinecolor}
\pgfpathmoveto{\pgfpoint{9.556171\du}{10.544270\du}}
\pgfpathlineto{\pgfpoint{11.097103\du}{10.544270\du}}
\pgfpathlineto{\pgfpoint{11.097103\du}{10.600000\du}}
\pgfpathlineto{\pgfpoint{9.556171\du}{10.600000\du}}
\pgfpathlineto{\pgfpoint{9.556171\du}{10.544270\du}}
\pgfusepath{stroke}
\pgfsetbuttcap
\pgfsetmiterjoin
\pgfsetdash{}{0pt}
\definecolor{dialinecolor}{rgb}{0.000000, 0.000000, 0.000000}
\pgfsetfillcolor{dialinecolor}
\pgfpathmoveto{\pgfpoint{9.846159\du}{9.896285\du}}
\pgfpathlineto{\pgfpoint{10.013980\du}{9.739798\du}}
\pgfpathlineto{\pgfpoint{11.431486\du}{9.739798\du}}
\pgfpathlineto{\pgfpoint{11.275630\du}{9.896285\du}}
\pgfpathlineto{\pgfpoint{9.846159\du}{9.896285\du}}
\pgfusepath{fill}
\pgfsetbuttcap
\pgfsetmiterjoin
\pgfsetdash{}{0pt}
\definecolor{dialinecolor}{rgb}{0.000000, 0.000000, 0.000000}
\pgfsetstrokecolor{dialinecolor}
\pgfpathmoveto{\pgfpoint{9.846159\du}{9.896285\du}}
\pgfpathlineto{\pgfpoint{10.013980\du}{9.739798\du}}
\pgfpathlineto{\pgfpoint{11.431486\du}{9.739798\du}}
\pgfpathlineto{\pgfpoint{11.275630\du}{9.896285\du}}
\pgfpathlineto{\pgfpoint{9.846159\du}{9.896285\du}}
\pgfusepath{stroke}
\pgfsetbuttcap
\pgfsetmiterjoin
\pgfsetdash{}{0pt}
\definecolor{dialinecolor}{rgb}{0.788235, 0.788235, 0.713726}
\pgfsetfillcolor{dialinecolor}
\pgfpathmoveto{\pgfpoint{9.834824\du}{8.745151\du}}
\pgfpathlineto{\pgfpoint{9.991625\du}{8.600000\du}}
\pgfpathlineto{\pgfpoint{11.409761\du}{8.600000\du}}
\pgfpathlineto{\pgfpoint{11.252960\du}{8.745151\du}}
\pgfpathlineto{\pgfpoint{9.834824\du}{8.745151\du}}
\pgfusepath{fill}
\pgfsetbuttcap
\pgfsetmiterjoin
\pgfsetdash{}{0pt}
\definecolor{dialinecolor}{rgb}{0.286275, 0.286275, 0.211765}
\pgfsetstrokecolor{dialinecolor}
\pgfpathmoveto{\pgfpoint{9.834824\du}{8.745151\du}}
\pgfpathlineto{\pgfpoint{9.991625\du}{8.600000\du}}
\pgfpathlineto{\pgfpoint{11.409761\du}{8.600000\du}}
\pgfpathlineto{\pgfpoint{11.252960\du}{8.745151\du}}
\pgfpathlineto{\pgfpoint{9.834824\du}{8.745151\du}}
\pgfusepath{stroke}
\pgfsetbuttcap
\pgfsetmiterjoin
\pgfsetdash{}{0pt}
\definecolor{dialinecolor}{rgb}{0.717647, 0.717647, 0.615686}
\pgfsetfillcolor{dialinecolor}
\pgfpathmoveto{\pgfpoint{9.834824\du}{8.745151\du}}
\pgfpathlineto{\pgfpoint{11.264295\du}{8.745151\du}}
\pgfpathlineto{\pgfpoint{11.264295\du}{9.873615\du}}
\pgfpathlineto{\pgfpoint{9.834824\du}{9.873615\du}}
\pgfpathlineto{\pgfpoint{9.834824\du}{8.745151\du}}
\pgfusepath{fill}
\pgfsetbuttcap
\pgfsetmiterjoin
\pgfsetdash{}{0pt}
\definecolor{dialinecolor}{rgb}{0.286275, 0.286275, 0.211765}
\pgfsetstrokecolor{dialinecolor}
\pgfpathmoveto{\pgfpoint{9.834824\du}{8.745151\du}}
\pgfpathlineto{\pgfpoint{11.263665\du}{8.745151\du}}
\pgfpathlineto{\pgfpoint{11.263665\du}{9.873300\du}}
\pgfpathlineto{\pgfpoint{9.834824\du}{9.873300\du}}
\pgfpathlineto{\pgfpoint{9.834824\du}{8.745151\du}}
\pgfusepath{stroke}
\pgfsetbuttcap
\pgfsetmiterjoin
\pgfsetdash{}{0pt}
\definecolor{dialinecolor}{rgb}{1.000000, 1.000000, 1.000000}
\pgfsetfillcolor{dialinecolor}
\pgfpathmoveto{\pgfpoint{9.957620\du}{8.889987\du}}
\pgfpathlineto{\pgfpoint{11.141184\du}{8.889987\du}}
\pgfpathlineto{\pgfpoint{11.141184\du}{9.761839\du}}
\pgfpathlineto{\pgfpoint{9.957620\du}{9.761839\du}}
\pgfpathlineto{\pgfpoint{9.957620\du}{8.889987\du}}
\pgfusepath{fill}
\pgfsetbuttcap
\pgfsetmiterjoin
\pgfsetdash{}{0pt}
\definecolor{dialinecolor}{rgb}{0.286275, 0.286275, 0.211765}
\pgfsetstrokecolor{dialinecolor}
\pgfpathmoveto{\pgfpoint{9.957620\du}{8.889987\du}}
\pgfpathlineto{\pgfpoint{11.141184\du}{8.889987\du}}
\pgfpathlineto{\pgfpoint{11.141184\du}{9.761524\du}}
\pgfpathlineto{\pgfpoint{9.957620\du}{9.761524\du}}
\pgfpathlineto{\pgfpoint{9.957620\du}{8.889987\du}}
\pgfusepath{stroke}
\pgfsetbuttcap
\pgfsetmiterjoin
\pgfsetdash{}{0pt}
\definecolor{dialinecolor}{rgb}{0.478431, 0.478431, 0.352941}
\pgfsetfillcolor{dialinecolor}
\pgfpathmoveto{\pgfpoint{11.252960\du}{9.862909\du}}
\pgfpathlineto{\pgfpoint{11.409761\du}{9.706423\du}}
\pgfpathlineto{\pgfpoint{11.409761\du}{8.600000\du}}
\pgfpathlineto{\pgfpoint{11.252960\du}{8.745151\du}}
\pgfpathlineto{\pgfpoint{11.252960\du}{9.862909\du}}
\pgfusepath{fill}
\pgfsetbuttcap
\pgfsetmiterjoin
\pgfsetdash{}{0pt}
\definecolor{dialinecolor}{rgb}{0.286275, 0.286275, 0.211765}
\pgfsetstrokecolor{dialinecolor}
\pgfpathmoveto{\pgfpoint{11.252960\du}{9.862909\du}}
\pgfpathlineto{\pgfpoint{11.409761\du}{9.706423\du}}
\pgfpathlineto{\pgfpoint{11.409761\du}{8.600000\du}}
\pgfpathlineto{\pgfpoint{11.252960\du}{8.745151\du}}
\pgfpathlineto{\pgfpoint{11.252960\du}{9.862909\du}}
\pgfusepath{stroke}
\pgfsetlinewidth{0.100000\du}
\pgfsetdash{}{0pt}
\pgfsetdash{}{0pt}
\pgfsetbuttcap
\pgfsetmiterjoin
\pgfsetlinewidth{0.001000\du}
\pgfsetbuttcap
\pgfsetmiterjoin
\pgfsetdash{}{0pt}
\definecolor{dialinecolor}{rgb}{0.717647, 0.717647, 0.615686}
\pgfsetfillcolor{dialinecolor}
\pgfpathmoveto{\pgfpoint{20.628989\du}{8.998912\du}}
\pgfpathlineto{\pgfpoint{20.628989\du}{10.850000\du}}
\pgfpathlineto{\pgfpoint{21.723081\du}{10.850000\du}}
\pgfpathlineto{\pgfpoint{21.723081\du}{8.998912\du}}
\pgfpathlineto{\pgfpoint{20.628989\du}{8.998912\du}}
\pgfusepath{fill}
\pgfsetbuttcap
\pgfsetmiterjoin
\pgfsetdash{}{0pt}
\definecolor{dialinecolor}{rgb}{0.286275, 0.286275, 0.211765}
\pgfsetstrokecolor{dialinecolor}
\pgfpathmoveto{\pgfpoint{20.628989\du}{8.998912\du}}
\pgfpathlineto{\pgfpoint{20.628989\du}{10.850000\du}}
\pgfpathlineto{\pgfpoint{21.723081\du}{10.850000\du}}
\pgfpathlineto{\pgfpoint{21.723081\du}{8.998912\du}}
\pgfpathlineto{\pgfpoint{20.628989\du}{8.998912\du}}
\pgfusepath{stroke}
\pgfsetbuttcap
\pgfsetmiterjoin
\pgfsetdash{}{0pt}
\definecolor{dialinecolor}{rgb}{0.788235, 0.788235, 0.713726}
\pgfsetfillcolor{dialinecolor}
\pgfpathmoveto{\pgfpoint{20.628989\du}{8.998912\du}}
\pgfpathlineto{\pgfpoint{20.777246\du}{8.850000\du}}
\pgfpathlineto{\pgfpoint{21.871011\du}{8.850000\du}}
\pgfpathlineto{\pgfpoint{21.723081\du}{8.998912\du}}
\pgfpathlineto{\pgfpoint{20.628989\du}{8.998912\du}}
\pgfusepath{fill}
\pgfsetbuttcap
\pgfsetmiterjoin
\pgfsetdash{}{0pt}
\definecolor{dialinecolor}{rgb}{0.286275, 0.286275, 0.211765}
\pgfsetstrokecolor{dialinecolor}
\pgfpathmoveto{\pgfpoint{20.628989\du}{8.998912\du}}
\pgfpathlineto{\pgfpoint{20.777246\du}{8.850000\du}}
\pgfpathlineto{\pgfpoint{21.863811\du}{8.850000\du}}
\pgfusepath{stroke}
\pgfsetbuttcap
\pgfsetmiterjoin
\pgfsetdash{}{0pt}
\definecolor{dialinecolor}{rgb}{0.286275, 0.286275, 0.211765}
\pgfsetstrokecolor{dialinecolor}
\pgfpathmoveto{\pgfpoint{21.863811\du}{8.857527\du}}
\pgfpathlineto{\pgfpoint{21.723081\du}{8.998912\du}}
\pgfpathlineto{\pgfpoint{20.628989\du}{8.998912\du}}
\pgfusepath{stroke}
\pgfsetbuttcap
\pgfsetmiterjoin
\pgfsetdash{}{0pt}
\definecolor{dialinecolor}{rgb}{0.788235, 0.788235, 0.713726}
\pgfsetfillcolor{dialinecolor}
\pgfpathmoveto{\pgfpoint{20.696408\du}{9.106586\du}}
\pgfpathlineto{\pgfpoint{21.196163\du}{9.106586\du}}
\pgfpathlineto{\pgfpoint{21.196163\du}{9.349427\du}}
\pgfpathlineto{\pgfpoint{20.696408\du}{9.349427\du}}
\pgfpathlineto{\pgfpoint{20.696408\du}{9.106586\du}}
\pgfusepath{fill}
\pgfsetbuttcap
\pgfsetmiterjoin
\pgfsetdash{}{0pt}
\definecolor{dialinecolor}{rgb}{0.384314, 0.384314, 0.282353}
\pgfsetstrokecolor{dialinecolor}
\pgfpathmoveto{\pgfpoint{20.696408\du}{9.106586\du}}
\pgfpathlineto{\pgfpoint{21.195835\du}{9.106586\du}}
\pgfpathlineto{\pgfpoint{21.195835\du}{9.349100\du}}
\pgfpathlineto{\pgfpoint{20.696408\du}{9.349100\du}}
\pgfpathlineto{\pgfpoint{20.696408\du}{9.106586\du}}
\pgfusepath{stroke}
\pgfsetlinewidth{0.030000\du}
\pgfsetbuttcap
\pgfsetmiterjoin
\pgfsetdash{}{0pt}
\definecolor{dialinecolor}{rgb}{0.925490, 0.925490, 0.905882}
\pgfsetstrokecolor{dialinecolor}
\pgfpathmoveto{\pgfpoint{20.763828\du}{9.228334\du}}
\pgfpathlineto{\pgfpoint{21.114343\du}{9.228334\du}}
\pgfusepath{stroke}
\pgfsetlinewidth{0.001000\du}
\pgfsetbuttcap
\pgfsetmiterjoin
\pgfsetdash{}{0pt}
\definecolor{dialinecolor}{rgb}{0.478431, 0.478431, 0.352941}
\pgfsetfillcolor{dialinecolor}
\pgfpathmoveto{\pgfpoint{21.723081\du}{10.850000\du}}
\pgfpathlineto{\pgfpoint{21.871011\du}{10.700761\du}}
\pgfpathlineto{\pgfpoint{21.871011\du}{8.850000\du}}
\pgfpathlineto{\pgfpoint{21.723081\du}{8.998912\du}}
\pgfpathlineto{\pgfpoint{21.723081\du}{10.850000\du}}
\pgfusepath{fill}
\pgfsetbuttcap
\pgfsetmiterjoin
\pgfsetdash{}{0pt}
\definecolor{dialinecolor}{rgb}{0.286275, 0.286275, 0.211765}
\pgfsetstrokecolor{dialinecolor}
\pgfpathmoveto{\pgfpoint{21.723081\du}{10.850000\du}}
\pgfpathlineto{\pgfpoint{21.863811\du}{10.708288\du}}
\pgfusepath{stroke}
\pgfsetbuttcap
\pgfsetmiterjoin
\pgfsetdash{}{0pt}
\definecolor{dialinecolor}{rgb}{0.286275, 0.286275, 0.211765}
\pgfsetstrokecolor{dialinecolor}
\pgfpathmoveto{\pgfpoint{21.863811\du}{8.857527\du}}
\pgfpathlineto{\pgfpoint{21.723081\du}{8.998912\du}}
\pgfpathlineto{\pgfpoint{21.723081\du}{10.850000\du}}
\pgfusepath{stroke}
\pgfsetlinewidth{0.030000\du}
\pgfsetbuttcap
\pgfsetmiterjoin
\pgfsetdash{}{0pt}
\definecolor{dialinecolor}{rgb}{0.925490, 0.925490, 0.905882}
\pgfsetstrokecolor{dialinecolor}
\pgfpathmoveto{\pgfpoint{20.642734\du}{10.727925\du}}
\pgfpathlineto{\pgfpoint{21.722754\du}{10.727925\du}}
\pgfusepath{stroke}
\pgfsetbuttcap
\pgfsetmiterjoin
\pgfsetdash{}{0pt}
\definecolor{dialinecolor}{rgb}{0.000000, 0.000000, 0.000000}
\pgfsetstrokecolor{dialinecolor}
\pgfpathmoveto{\pgfpoint{20.642734\du}{9.741834\du}}
\pgfpathlineto{\pgfpoint{21.722754\du}{9.741834\du}}
\pgfusepath{stroke}
\pgfsetbuttcap
\pgfsetmiterjoin
\pgfsetdash{}{0pt}
\definecolor{dialinecolor}{rgb}{0.286275, 0.286275, 0.211765}
\pgfsetstrokecolor{dialinecolor}
\pgfpathmoveto{\pgfpoint{20.628989\du}{10.714507\du}}
\pgfpathlineto{\pgfpoint{21.721772\du}{10.714507\du}}
\pgfusepath{stroke}
\pgfsetbuttcap
\pgfsetmiterjoin
\pgfsetdash{}{0pt}
\definecolor{dialinecolor}{rgb}{0.000000, 0.000000, 0.000000}
\pgfsetstrokecolor{dialinecolor}
\pgfpathmoveto{\pgfpoint{20.628989\du}{9.728089\du}}
\pgfpathlineto{\pgfpoint{21.721772\du}{9.728089\du}}
\pgfusepath{stroke}
\pgfsetlinewidth{0.001000\du}
\pgfsetbuttcap
\pgfsetmiterjoin
\pgfsetdash{}{0pt}
\definecolor{dialinecolor}{rgb}{0.925490, 0.925490, 0.905882}
\pgfsetstrokecolor{dialinecolor}
\pgfpathmoveto{\pgfpoint{20.696408\du}{9.336336\du}}
\pgfpathlineto{\pgfpoint{20.696408\du}{9.106586\du}}
\pgfpathlineto{\pgfpoint{21.182417\du}{9.106586\du}}
\pgfusepath{stroke}
\pgfsetlinewidth{0.100000\du}
\pgfsetdash{}{0pt}
\pgfsetdash{}{0pt}
\pgfsetbuttcap
\pgfsetmiterjoin
\pgfsetlinewidth{0.001000\du}
\pgfsetbuttcap
\pgfsetmiterjoin
\pgfsetdash{}{0pt}
\definecolor{dialinecolor}{rgb}{0.788235, 0.788235, 0.713726}
\pgfsetfillcolor{dialinecolor}
\pgfpathmoveto{\pgfpoint{22.752053\du}{13.227519\du}}
\pgfpathlineto{\pgfpoint{22.999199\du}{13.000000\du}}
\pgfpathlineto{\pgfpoint{25.247947\du}{13.000000\du}}
\pgfpathlineto{\pgfpoint{25.000401\du}{13.227519\du}}
\pgfpathlineto{\pgfpoint{22.752053\du}{13.227519\du}}
\pgfusepath{fill}
\pgfsetbuttcap
\pgfsetmiterjoin
\pgfsetdash{}{0pt}
\definecolor{dialinecolor}{rgb}{0.286275, 0.286275, 0.211765}
\pgfsetstrokecolor{dialinecolor}
\pgfpathmoveto{\pgfpoint{22.763669\du}{13.217104\du}}
\pgfpathlineto{\pgfpoint{22.989986\du}{13.008412\du}}
\pgfpathlineto{\pgfpoint{25.239135\du}{13.008412\du}}
\pgfpathlineto{\pgfpoint{25.000401\du}{13.227519\du}}
\pgfpathlineto{\pgfpoint{22.763669\du}{13.227519\du}}
\pgfpathlineto{\pgfpoint{22.763669\du}{13.217104\du}}
\pgfusepath{stroke}
\pgfsetbuttcap
\pgfsetmiterjoin
\pgfsetdash{}{0pt}
\definecolor{dialinecolor}{rgb}{0.717647, 0.717647, 0.615686}
\pgfsetfillcolor{dialinecolor}
\pgfpathmoveto{\pgfpoint{22.752053\du}{13.227519\du}}
\pgfpathlineto{\pgfpoint{25.018426\du}{13.227519\du}}
\pgfpathlineto{\pgfpoint{25.018426\du}{15.000000\du}}
\pgfpathlineto{\pgfpoint{22.752053\du}{15.000000\du}}
\pgfpathlineto{\pgfpoint{22.752053\du}{13.227519\du}}
\pgfusepath{fill}
\pgfsetbuttcap
\pgfsetmiterjoin
\pgfsetdash{}{0pt}
\definecolor{dialinecolor}{rgb}{0.286275, 0.286275, 0.211765}
\pgfsetstrokecolor{dialinecolor}
\pgfpathmoveto{\pgfpoint{22.763669\du}{13.227519\du}}
\pgfpathlineto{\pgfpoint{25.017625\du}{13.227519\du}}
\pgfpathlineto{\pgfpoint{25.017625\du}{14.999199\du}}
\pgfpathlineto{\pgfpoint{22.763669\du}{14.999199\du}}
\pgfpathlineto{\pgfpoint{22.763669\du}{13.227519\du}}
\pgfusepath{stroke}
\pgfsetbuttcap
\pgfsetmiterjoin
\pgfsetdash{}{0pt}
\definecolor{dialinecolor}{rgb}{0.478431, 0.478431, 0.352941}
\pgfsetfillcolor{dialinecolor}
\pgfpathmoveto{\pgfpoint{25.000401\du}{14.982776\du}}
\pgfpathlineto{\pgfpoint{25.247947\du}{14.736832\du}}
\pgfpathlineto{\pgfpoint{25.247947\du}{13.000000\du}}
\pgfpathlineto{\pgfpoint{25.000401\du}{13.227519\du}}
\pgfpathlineto{\pgfpoint{25.000401\du}{14.982776\du}}
\pgfusepath{fill}
\pgfsetbuttcap
\pgfsetmiterjoin
\pgfsetdash{}{0pt}
\definecolor{dialinecolor}{rgb}{0.286275, 0.286275, 0.211765}
\pgfsetstrokecolor{dialinecolor}
\pgfpathmoveto{\pgfpoint{25.000401\du}{14.982776\du}}
\pgfpathlineto{\pgfpoint{25.247947\du}{14.736832\du}}
\pgfpathlineto{\pgfpoint{25.247947\du}{13.008412\du}}
\pgfpathlineto{\pgfpoint{25.239135\du}{13.008412\du}}
\pgfpathlineto{\pgfpoint{25.000401\du}{13.227519\du}}
\pgfpathlineto{\pgfpoint{25.000401\du}{14.982776\du}}
\pgfpathlineto{\pgfpoint{25.000401\du}{14.982776\du}}
\pgfusepath{stroke}
\pgfsetlinewidth{0.050000\du}
\pgfsetdash{}{0pt}
\pgfsetdash{}{0pt}
\pgfsetbuttcap
{
\definecolor{dialinecolor}{rgb}{0.000000, 0.000000, 0.000000}
\pgfsetfillcolor{dialinecolor}
% was here!!!
\pgfsetarrowsend{to}
\definecolor{dialinecolor}{rgb}{0.000000, 0.000000, 0.000000}
\pgfsetstrokecolor{dialinecolor}
\draw (11.755164\du,9.918640\du)--(20.628989\du,9.924456\du);
}
\pgfsetlinewidth{0.100000\du}
\pgfsetdash{}{0pt}
\pgfsetdash{}{0pt}
\pgfsetbuttcap
{
\definecolor{dialinecolor}{rgb}{0.000000, 0.000000, 0.000000}
\pgfsetfillcolor{dialinecolor}
% was here!!!
\pgfsetarrowsend{to}
\definecolor{dialinecolor}{rgb}{0.000000, 0.000000, 0.000000}
\pgfsetstrokecolor{dialinecolor}
\draw (21.870751\du,10.650831\du)--(23.500000\du,12.750000\du);
}
\pgfsetlinewidth{0.050000\du}
\pgfsetdash{}{0pt}
\pgfsetdash{}{0pt}
\pgfsetbuttcap
{
\definecolor{dialinecolor}{rgb}{0.000000, 0.000000, 0.000000}
\pgfsetfillcolor{dialinecolor}
% was here!!!
\pgfsetarrowsend{to}
\definecolor{dialinecolor}{rgb}{0.000000, 0.000000, 0.000000}
\pgfsetstrokecolor{dialinecolor}
\draw (22.757095\du,13.994376\du)--(17.400000\du,14.000000\du);
}
\pgfsetlinewidth{0.050000\du}
\pgfsetdash{}{0pt}
\pgfsetdash{}{0pt}
\pgfsetbuttcap
\pgfsetmiterjoin
\pgfsetlinewidth{0.050000\du}
\pgfsetbuttcap
\pgfsetmiterjoin
\pgfsetdash{}{0pt}
\definecolor{dialinecolor}{rgb}{1.000000, 1.000000, 1.000000}
\pgfsetfillcolor{dialinecolor}
\fill (14.682258\du,12.900000\du)--(14.682258\du,15.605000\du)--(17.300000\du,15.605000\du)--(17.300000\du,12.900000\du)--cycle;
\definecolor{dialinecolor}{rgb}{0.000000, 0.000000, 0.000000}
\pgfsetstrokecolor{dialinecolor}
\draw (14.682258\du,12.900000\du)--(14.682258\du,15.605000\du)--(17.300000\du,15.605000\du)--(17.300000\du,12.900000\du)--cycle;
\pgfsetbuttcap
\pgfsetmiterjoin
\pgfsetdash{}{0pt}
\definecolor{dialinecolor}{rgb}{0.000000, 0.000000, 0.000000}
\pgfsetstrokecolor{dialinecolor}
\draw (14.682258\du,12.900000\du)--(14.682258\du,15.605000\du)--(17.300000\du,15.605000\du)--(17.300000\du,12.900000\du)--cycle;
% setfont left to latex
\definecolor{dialinecolor}{rgb}{0.000000, 0.000000, 0.000000}
\pgfsetstrokecolor{dialinecolor}
\node[anchor=west] at (15.239919\du,14.659583\du){html};
\pgfsetlinewidth{0.050000\du}
\pgfsetdash{}{0pt}
\pgfsetdash{}{0pt}
\pgfsetbuttcap
{
\definecolor{dialinecolor}{rgb}{0.000000, 0.000000, 0.000000}
\pgfsetfillcolor{dialinecolor}
% was here!!!
\pgfsetarrowsend{to}
\definecolor{dialinecolor}{rgb}{0.000000, 0.000000, 0.000000}
\pgfsetstrokecolor{dialinecolor}
\draw (14.682258\du,14.252500\du)--(10.500000\du,10.750000\du);
}
% setfont left to latex
\definecolor{dialinecolor}{rgb}{0.000000, 0.000000, 0.000000}
\pgfsetstrokecolor{dialinecolor}
\node[anchor=west] at (8.950000\du,14.350000\du){Navigateur};
% setfont left to latex
\definecolor{dialinecolor}{rgb}{0.000000, 0.000000, 0.000000}
\pgfsetstrokecolor{dialinecolor}
\node[anchor=west] at (10.800000\du,7.600000\du){Client};
% setfont left to latex
\definecolor{dialinecolor}{rgb}{0.000000, 0.000000, 0.000000}
\pgfsetstrokecolor{dialinecolor}
\node[anchor=west] at (20.250000\du,7.800000\du){Serveur};
% setfont left to latex
\definecolor{dialinecolor}{rgb}{0.000000, 0.000000, 0.000000}
\pgfsetstrokecolor{dialinecolor}
\node[anchor=west] at (22.950000\du,15.500000\du){Apache};
% setfont left to latex
\definecolor{dialinecolor}{rgb}{0.000000, 0.000000, 0.000000}
\pgfsetstrokecolor{dialinecolor}
\node[anchor=west] at (19.400000\du,15.200000\du){PHP};
\end{tikzpicture}

	\section{Définition du type de la liste}		
		\lstinputlisting[caption=Type de la liste]{2.algo}
	\section{Construction de la liste}
		\subsection{}
		\subsection{}
			\subsubsection{Avant la première insertion}
		\subsection{}
			\lstinputlisting{3-3.algo}
	\section{Impression de la liste en détruisant au fur et à mesure 
				ses éléments}
		\subsection{}
			\lstinputlisting{4-1.algo}
		\subsection{}
			\subsubsection{Après la première insertion}
			\subsubsection{Suppresion autre que la première}
				même séquence de code que a (car suppression toujours en tête 
					de la liste)
		\subsection{}
			\lstinputlisting{4-3.algo}
\end{document}
