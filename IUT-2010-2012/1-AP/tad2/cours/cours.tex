\documentclass[12pt,a4paper,openany]{report}

\usepackage{lmodern}
\usepackage{xcolor}
\usepackage[utf8]{inputenc}
\usepackage[T1]{fontenc}
\usepackage[francais]{babel}
\usepackage[top=1.7cm, bottom=1.7cm, left=1.7cm, right=1.7cm]{geometry}
%\usepackage[frenchb]{babel}
%\usepackage{layout}
%\usepackage{setspace}
%\usepackage{soul}
%\usepackage{ulem}
%\usepackage{eurosym}
%\usepackage{bookman}
%\usepackage{charter}
%\usepackage{newcent}
%\usepackage{lmodern}
%\usepackage{mathpazo}
%\usepackage{mathptmx}
%\usepackage{url}
%\usepackage{verbatim}
%\usepackage{moreverb}
%\usepackage{wrapfig}
%\usepackage{amsmath}
%\usepackage{mathrsfs}
%\usepackage{asmthm}
%\usepackage{makeidx}
\usepackage{tikz} %Vectoriel
\usepackage{listingsutf8}
\usepackage{fancyhdr}
\usepackage{multido}
\usepackage{amssymb}


\newcommand{\titre}{Type abstrait de données}

\newcommand{\module}{Type abstrait de données}
\newcommand{\sigle}{tad}

\newcommand{\semestre}{2}

\definecolor{gris1}{gray}{0.40}
\definecolor{gris2}{gray}{0.55}
\definecolor{gris3}{gray}{0.65}
\definecolor{gris4}{gray}{0.50}
\definecolor{vert}{rgb}{0,0.4,0}
\definecolor{violet}{rgb}{0.65, 0.2, 0.65}
\definecolor{bleu1}{rgb}{0,0,0.8}
\definecolor{bleu2}{rgb}{0,0.2,0.6}
\definecolor{bleu3}{rgb}{0,0.2,0.2}
\definecolor{rouge}{HTML}{F93928}


\lstdefinelanguage{algo}{%
   morekeywords={%
    %%% couleur 1
		importer, programme, glossaire, fonction, procedure, constante, type, 
	%%% IMPORT & Co.
		si, sinon, alors, fin, tantque, debut, faire, lorsque, fin lorsque, 
		declenche, declencher, enregistrement, tableau, retourne, retourner, =, pour, a,
		/=, <, >, traite,exception, 
	%%% types 
		Entier, Reel, Booleen, Caractere, Réél, Booléen, Caractère,
	%%% types 
		entree, maj, sortie,entrée,
	%%% types 
		et, ou, non,
	},
  sensitive=true,
  morecomment=[l]{--},
  morestring=[b]',
}

\lstset{language=algo,
    %%% BOUCLE, TEST & Co.
      emph={importer, programme, glossaire, fonction, procedure, constante, type},
      emphstyle=\color{bleu2},
    %%% IMPORT & Co.  
	emph={[2]
		si, sinon, alors, fin , tantque, debut, faire, lorsque, fin lorsque, 
		declencher, retourner, et, ou, non,enregistrement, retourner, retourne, 
		tableau, /=, <, =, >, traite,exception, pour, a
	},
      emphstyle=[2]\color{bleu1},
    %%% FONCTIONS NUMERIQUES
      emph={[3]Entier, Reel, Booleen, Caractere, Booléen, Réél, Caractère},
      emphstyle=[3]\color{gris1},
    %%% FONCTIONS NUMERIQUES
      emph={[4]entree, maj, sortie, entrée},	
      emphstyle=[4]\color{gris1},
}
\lstdefinelanguage{wl}{%
   morekeywords={%
    %%% couleur 1
		importer, programme, glossaire, fonction, procedure, constante, type, 
	%%% IMPORT & Co.
		si, sinon, alors, fin, TANTQUE, tantque, FIN, PROCEDURE, debut, faire, lorsque, 
		fin lorsque, declenche, declencher, enregistrement, tableau, retourne, retourner, =, 
		/=, <, >, traite,exception, 
	%%% types 
		Entier, Reel, Booleen, Caractere, Réél, Booléen, Caractère,
	%%% types 
		entree, maj, sortie,entrée,
	%%% types 
		et, ou, non,
	},
  sensitive=true,
  morecomment=[l]{//},
  morestring=[b]',
}

\lstset{language=wl,
    %%% BOUCLE, TEST & Co.
      emph={importer, programme, glossaire, fonction, procedure, constante, type},
      emphstyle=\color{bleu2},
    %%% IMPORT & Co.  
	emph={[2]
		si, sinon, alors, fin , tantque, debut, faire, lorsque, fin lorsque, 
		declencher, retourner, et, ou, non,enregistrement, retourner, retourne, 
		tableau, /=, <, =, >, traite,exception
	},
      emphstyle=[2]\color{bleu1},
    %%% FONCTIONS NUMERIQUES
      emph={[3]Entier, Reel, Booleen, Caractere, Booléen, Réél, Caractère},
      emphstyle=[3]\color{gris1},
    %%% FONCTIONS NUMERIQUES
      emph={[4]entree, maj, sortie, entrée},	
      emphstyle=[4]\color{gris1},
}
\lstdefinelanguage{css}{%
   morekeywords={%
    %%% couleur 1
		background, image, repeat, position, index, color, border, font, 
		size, url, family, style, variant, weight, letter, spacing, line, 
		height, text, decoration, align, indent, transform, shadow, 
		background, image, repeat, position, index, color, border, font, 
		size, url, family, style, variant, weight, letter, spacing, line, 
		height, text, decoration, align, indent, transform, shadow, 
		vertical, align, white, space, word, spacing,attachment, width, 
		max, min, margin, padding, clip, direction, display, overflow,
		visibility, clear, float, top, right, bottom, left, list, type, 
		collapse, side, empty, cells, table, layout, cursor, marks, page, break,
		before, after, inside, orphans, windows, azimuth, after, before, cue, 
		elevation, pause, play, during, pitch, range, richness, spek, header, 
		numeral, punctuation, rate, stress, voice, volume,
	%%% types 
		left, right, bottom, top, none, center, solid, black, blue, red, green,
	},
  sensitive=true,
  sensitive=true,
  morecomment=[s]{/*}{*/},
  morestring=[b]',
}
\lstset{language=css,
    %%% BOUCLE, TEST & Co.
      emph={
		background, image, repeat, position, index, color, border, font, 
		size, url, family, style, variant, weight, letter, spacing, line, 
		height, text, decoration, align, indent, transform, shadow, 
		background, image, repeat, position, index, color, border, font, 
		size, url, family, style, variant, weight, letter, spacing, line, 
		height, text, decoration, align, indent, transform, shadow, 
		vertical, align, white, space, word, spacing,attachment, width, 
		max, min, margin, padding, clip, direction, display, overflow,
		visibility, clear, float, top, right, bottom, left, list, type, 
		collapse, side, empty, cells, table, layout, cursor, marks, page, break,
		before, after, inside, orphans, windows, azimuth, after, before, cue, 
		elevation, pause, play, during, pitch, range, richness, spek, header, 
		numeral, punctuation, rate, stress, voice, volume,
	  },
      emphstyle=\color{bleu2},
    %%% FONCTIONS NUMERIQUES
      emph={[3]
		left, right, bottom, top,none, solid, black, blue, green,
		  },
      emphstyle=[3]\color{bleu3},
    %%% FONCTIONS NUMERIQUES
}

\lstset{language=SQL,
    %%% BOUCLE, TEST & Co.
      emph={INSERT, UPDATE, DELETE, WHERE, SET, GROUP, BY, ORDER, REFERENCES},
      emphstyle=\color{bleu2},
    %%% IMPORT & Co.  
	emph={[2]
		if, end, begin, then, for, each, else, after, of, on, to
	},
      emphstyle=[2]\color{bleu1},
    %%% FONCTIONS NUMERIQUES
      emph={[3]Entier, Reel, Booleen, Caractere, Booléen, Réél, Caractère},
      emphstyle=[3]\color{gris1},
    %%% FONCTIONS NUMERIQUES
      emph={[4]entree, maj, sortie, entrée},	
      emphstyle=[4]\color{gris1},
}
\lstdefinelanguage{ARM}{%
   morekeywords={%
   ADD, SUB, MOV, MUL, RSB,CMP, BLS, BLE, B,BHI,LDR,
   BGE, RSBLT, BGT, BEQ, BNE,BLT,BHS,STR,STRB
	},
  sensitive=true,
  morecomment=[l]{@},
  morestring=[b]',
}

\lstset{ % general style for listings 
   numbers=left 
   , literate={é}{{\'e}}1 {è}{{\`e}}1 {à}{{\`a}}1 {ê}{{\^e}}1 {É}{{\'E}}1 {ô}{{\^o}}1 {€}{{\euro}}1{°}{{$^{\circ}$}}1 {ç}{ {c}}1 {ù}{u}1
	, extendedchars=\true
   , tabsize=2 
   , frame=l
   , framerule=1.1pt
   , linewidth=520px
   , breaklines=true 
   , basicstyle=\footnotesize\ttfamily 
   , numberstyle=\tiny\ttfamily 
   , framexleftmargin=0mm 
   , xleftmargin=0mm 
   , captionpos=b 
	, keywordstyle=\color{bleu2}
	, commentstyle=\color{vert}
	, stringstyle=\color{rouge}
	, showstringspaces=false
	, extendedchars=true
	, mathescape=true
} 
%	\lstlistoflistings
%	\addcontentsline{toc}{part}{List of code examples}
 %prise en charge du langage algo
\date{\today}

\chead{}
\rhead{--~ \thepage ~--}
\lhead{\titre}
\makeindex
\lfoot{Université Paul Sabatier Toulouse III}
\rfoot{\sigle\semestre}
%\rfoot{}
\cfoot{}
\makeglossary
\makeatletter
\def\clap#1{\hbox to 0pt{\hss #1\hss}}%
\def\ligne#1{%
\hbox to \hsize{%
\vbox{\centering #1}}}%
\def\haut#1#2#3{%
\hbox to \hsize{%
\rlap{\vtop{\raggedright #1}}%
\hss
\clap{\vtop{\centering #2}}%
\hss
\llap{\vtop{\raggedleft #3}}}}%
\def\bas#1#2#3{%
\hbox to \hsize{%
\rlap{\vbox{\raggedright #1}}%
\hss \clap{\vbox{\centering #2}}%
\hss
\llap{\vbox{\raggedleft #3}}}}%
\def\maketitle{%
\thispagestyle{empty}\vbox to \vsize{%
\haut{}{\@blurb}{}

\vfill
\vspace{1cm}
\begin{flushleft}
\usefont{OT1}{ptm}{m}{n}
\huge \@title
\end{flushleft}
\par
\hrule height 4pt
\par
\begin{flushright}
\usefont{OT1}{phv}{m}{n}
\Large \@author
\par
\end{flushright}
\vspace{1cm}
\vfill
\vfill
\bas{}{\@location, le \@date}{}
}%
\cleardoublepage
}
\def\date#1{\def\@date{#1}}
\def\author#1{\def\@author{#1}}
\def\title#1{\def\@title{#1}}
\def\location#1{\def\@location{#1}}
\def\blurb#1{\def\@blurb{#1}}
\date{\today}
\author{}
\title{}
\location{Amiens}\blurb{}
\makeatother
\title{\titre}
\author{Semestre \semestre}

\location{Toulouse}
\blurb{%
Université Paul Sabatier -- Toulouse III\\
IUT A - Toulouse Rangueil\\
}%



%\title{Cours \\ \titre}
%\date{\today\\ Semestre \semestre}

%\lhead{Cours: \titre}
%\chead{}
%\rhead{\thepage}

%\lfoot{Université Paul Sabatier Toulouse III}
%\cfoot{\thepage}
%\rfoot{\sigle\semestre}

\pagestyle{fancy}


\begin{document}
	\maketitle
	\chapter{Les pointeurs}
		\chapter{Cours sur les pointeurs en C}\label{pointeurs}
Déjà vu par le passages de paramètres.
\section{Syntaxe}
	\subsection{Déclaration}
\begin{lstlisting}[language=C, numbers=none,frame=none, caption=Syntaxe de déclaration d'un pointeur]
typePointé* nomPointeur
\end{lstlisting}
\lstinputlisting[caption=Exemple de déclaration, numbers=none,language=C]{annexes/ptr1.c}
	\subsection{Utilisation}
\begin{lstlisting}[language=C, numbers=none,frame=none, caption=Syntaxe utilisation d'un pointeur]
nomPointeur // manipule l'adresse
*nomPointeur //manipule la zone pointée
\end{lstlisting}
\begin{lstlisting}[language=C, numbers=none,frame=none, caption=Exemple d'utilisation d'un pointeur]
pe=&n; //opérateur d'adressage
\end{lstlisting}

\subsection{Constante}
	\texttt{NULL} représente une adresse inexistante.
	\begin{lstlisting}[language=C, numbers=none,frame=none, caption=Exemple d'utilisation de la constante \texttt{NULL}]
pe = NULL;
*pe; // Erreur à l'exécution
	\end{lstlisting}

\section{Opérateur autorisés sur les pointeurs}
\subsection{L'affectation}
\begin{verbatim}
nomPointeur =  expression correspondant à une adresse ou à NULL
\end{verbatim}

\subsection{Addition et la soustraction entre un pointeur et un entier}
\begin{verbatim}
nomPointeur = nomPointeur + 10; 
nomPointeur = nomPointeur - 15; 
\end{verbatim}
On obtient une expression correspondant à une adresse
\begin{verbatim}
pe = pe+10; //pe contient l'adresse du 10e entier après la valeu initalie de pe.
\end{verbatim}
\attention{À utiliser que si pe pointe sur un tableau}

\subsection{Soustraction de deux pointeurs}
Renvoi un entier donnant le nombre d'éléments pointés entre les deux pointeurs
\attention{Uniquement si les deux pointeurs sont sur le même tableau}

\subsection{Comparaison sur des pointeurs}
Ce sont les opérateurs de comparaison classique : \texttt{= =} et \texttt{!=}

\subsection{Allocation dynamique de mémoire}
\begin{lstlisting}[language=C, numbers=none,frame=none, caption=Syntaxe d'allocation dynamique]
nomPointeur = (typePointeur) malloc(sizeof(typePointé));
nomPointeur = (typePointé*) malloc(n*sizeof(typePointé));
\end{lstlisting}
\begin{lstlisting}[language=C, numbers=none,frame=none, caption=Exemple d'allocation dynamique]
	int *e;
	pe = (int*) malloc(sizeof(int));
\end{lstlisting}
\begin{enumerate}
	\item Le programme demande au gestionnaire mémoire à avoir une place de la taille \texttt{sizeof(int)}
	\item Si la place est disponible retourne l'adresse demandée ou la première case du <<tableau>> dynamique
	\item Sinon retourne \texttt{NULL}
\end{enumerate}

\subsection{Libération dynamique de mémoire}
\begin{lstlisting}[language=C, numbers=none,frame=none, caption=Syntaxe de libération de mémoire]
free(nomPointeur);
\end{lstlisting}
\begin{enumerate}
	\item Le programme contact le gestionnaire mémoire 
	\item Le gestionnaire mémoire <<libère>> la place
\end{enumerate}

Cela veut dire que la place n'est plus réservé au programme, elle pourra être alloué à un autre programme.
\attention{Le gestionnaire de mémoire ne met pas à jour la case mémoire, celle-ci contient toujours la valeur, si personne ne récupère la case, il sera toujours
possible d'accéder à la donnée. C'est donc aléatoire, c'est une source d'erreurs.}

	
	\chapter{Les types abstraits de données}
		Une structure de donnée se traduit dans un langage par un type. On distingue, du plus simple au plus complexe: 
\begin{itemize}
	\item Les types élémentaires (cf cours algo)
	\item Les types composés (cf cours algo)
	\item Les Types Abstraits de données (étudiés dans ce cours)
\end{itemize}

\section{Notion de type abstrait}
	\subsection{Définitions}
		\paragraph{Type} Un type est un ensemble de valeurs et un ensemble d'opérations. 
		\paragraph{Type Abstrait} Un type abstrait est un type où l'utilisateur de la donnée ignore la représentation de la donnée mémoire et le codage des opérations.
	\subsection{Différents point de vue d'un TAD}
		\subsubsection{Concepteur}
			Il définit avec précision les opérations et les propriétés du type abstrait.\\
			Ce point de vue est appelé \textbf{spécifications} c'est la définition du type (QUOI?)
		\subsubsection{Programmeur}
			Il propose un codage des opérations du type\\
			Ce point de vue est appelé \textbf{implémentation} (COMMENT?)
		\subsubsection{Utilisateur}
			Il exploite les opérations du type pour son application.
	\subsection{Propriété des TADs}
		\paragraph{Encapsulation} 
			Les détails d'implémentations d'un type abstrait sont cachés (donc non accessible) au client.\\
			\subparagraph{Intérêt}			
				\begin{itemize}
					\item  Le programmeur peut modifier son implémentation sans impacter le client
					\item La localisation, le code du type abstrait est enregistré au même endroit de l'application.
					\item Notion de module, par exemple paquetage ou classe.
				\end{itemize}
	\subsection{En résumé}
		% Graphic for TeX using PGF
% Title: /home/satenske/cours/schema_tad.dia
% Creator: Dia v0.97.1
% CreationDate: Fri Feb  4 12:10:39 2011
% For: satenske
% \usepackage{tikz}
% The following commands are not supported in PSTricks at present
% We define them conditionally, so when they are implemented,
% this pgf file will use them.
\ifx\du\undefined
  \newlength{\du}
\fi
\setlength{\du}{15\unitlength}
\begin{tikzpicture}
\pgftransformxscale{1.000000}
\pgftransformyscale{-1.000000}
\definecolor{dialinecolor}{rgb}{0.000000, 0.000000, 0.000000}
\pgfsetstrokecolor{dialinecolor}
\definecolor{dialinecolor}{rgb}{1.000000, 1.000000, 1.000000}
\pgfsetfillcolor{dialinecolor}
\pgfsetlinewidth{0.050000\du}
\pgfsetdash{}{0pt}
\pgfsetdash{}{0pt}
\pgfsetbuttcap
\pgfsetmiterjoin
\pgfsetlinewidth{0.050000\du}
\pgfsetbuttcap
\pgfsetmiterjoin
\pgfsetdash{}{0pt}
\definecolor{dialinecolor}{rgb}{0.749020, 0.749020, 0.749020}
\pgfsetfillcolor{dialinecolor}
\pgfpathellipse{\pgfpoint{17.075000\du}{17.125000\du}}{\pgfpoint{9.225000\du}{0\du}}{\pgfpoint{0\du}{9.225000\du}}
\pgfusepath{fill}
\definecolor{dialinecolor}{rgb}{0.000000, 0.000000, 0.000000}
\pgfsetstrokecolor{dialinecolor}
\pgfpathellipse{\pgfpoint{17.075000\du}{17.125000\du}}{\pgfpoint{9.225000\du}{0\du}}{\pgfpoint{0\du}{9.225000\du}}
\pgfusepath{stroke}
\pgfsetbuttcap
\pgfsetmiterjoin
\pgfsetdash{}{0pt}
\definecolor{dialinecolor}{rgb}{0.000000, 0.000000, 0.000000}
\pgfsetstrokecolor{dialinecolor}
\pgfpathellipse{\pgfpoint{17.075000\du}{17.125000\du}}{\pgfpoint{9.225000\du}{0\du}}{\pgfpoint{0\du}{9.225000\du}}
\pgfusepath{stroke}
\pgfsetlinewidth{0.050000\du}
\pgfsetdash{}{0pt}
\pgfsetdash{}{0pt}
\pgfsetbuttcap
\pgfsetmiterjoin
\pgfsetlinewidth{0.050000\du}
\pgfsetbuttcap
\pgfsetmiterjoin
\pgfsetdash{}{0pt}
\definecolor{dialinecolor}{rgb}{1.000000, 1.000000, 1.000000}
\pgfsetfillcolor{dialinecolor}
\pgfpathellipse{\pgfpoint{16.756250\du}{16.856250\du}}{\pgfpoint{5.768750\du}{0\du}}{\pgfpoint{0\du}{5.768750\du}}
\pgfusepath{fill}
\definecolor{dialinecolor}{rgb}{0.000000, 0.000000, 0.000000}
\pgfsetstrokecolor{dialinecolor}
\pgfpathellipse{\pgfpoint{16.756250\du}{16.856250\du}}{\pgfpoint{5.768750\du}{0\du}}{\pgfpoint{0\du}{5.768750\du}}
\pgfusepath{stroke}
\pgfsetbuttcap
\pgfsetmiterjoin
\pgfsetdash{}{0pt}
\definecolor{dialinecolor}{rgb}{0.000000, 0.000000, 0.000000}
\pgfsetstrokecolor{dialinecolor}
\pgfpathellipse{\pgfpoint{16.756250\du}{16.856250\du}}{\pgfpoint{5.768750\du}{0\du}}{\pgfpoint{0\du}{5.768750\du}}
\pgfusepath{stroke}
\pgfsetlinewidth{0.050000\du}
\pgfsetdash{}{0pt}
\pgfsetdash{}{0pt}
\pgfsetbuttcap
{
\definecolor{dialinecolor}{rgb}{0.000000, 0.000000, 0.000000}
\pgfsetfillcolor{dialinecolor}
% was here!!!
\definecolor{dialinecolor}{rgb}{0.000000, 0.000000, 0.000000}
\pgfsetstrokecolor{dialinecolor}
\draw (16.756250\du,16.856250\du)--(16.756250\du,16.856250\du);
}
\pgfsetlinewidth{0.050000\du}
\pgfsetdash{}{0pt}
\pgfsetdash{}{0pt}
\pgfsetbuttcap
\pgfsetmiterjoin
\pgfsetlinewidth{0.050000\du}
\pgfsetbuttcap
\pgfsetmiterjoin
\pgfsetdash{}{0pt}
\definecolor{dialinecolor}{rgb}{1.000000, 1.000000, 1.000000}
\pgfsetfillcolor{dialinecolor}
\fill (15.745363\du,12.230417\du)--(15.745363\du,14.500208\du)--(17.941935\du,14.500208\du)--(17.941935\du,12.230417\du)--cycle;
\definecolor{dialinecolor}{rgb}{0.000000, 0.000000, 0.000000}
\pgfsetstrokecolor{dialinecolor}
\draw (15.745363\du,12.230417\du)--(15.745363\du,14.500208\du)--(17.941935\du,14.500208\du)--(17.941935\du,12.230417\du)--cycle;
\pgfsetbuttcap
\pgfsetmiterjoin
\pgfsetdash{}{0pt}
\definecolor{dialinecolor}{rgb}{0.000000, 0.000000, 0.000000}
\pgfsetstrokecolor{dialinecolor}
\draw (15.745363\du,12.230417\du)--(15.745363\du,14.500208\du)--(17.941935\du,14.500208\du)--(17.941935\du,12.230417\du)--cycle;
\pgfsetlinewidth{0.050000\du}
\pgfsetdash{}{0pt}
\pgfsetdash{}{0pt}
\pgfsetbuttcap
{
\definecolor{dialinecolor}{rgb}{0.000000, 0.000000, 0.000000}
\pgfsetfillcolor{dialinecolor}
% was here!!!
\pgfsetarrowsend{to}
\definecolor{dialinecolor}{rgb}{0.000000, 0.000000, 0.000000}
\pgfsetstrokecolor{dialinecolor}
\draw (26.043750\du,18.493750\du)--(29.956250\du,18.168750\du);
}
\pgfsetlinewidth{0.050000\du}
\pgfsetdash{}{0pt}
\pgfsetdash{}{0pt}
\pgfsetbuttcap
{
\definecolor{dialinecolor}{rgb}{0.000000, 0.000000, 0.000000}
\pgfsetfillcolor{dialinecolor}
% was here!!!
\pgfsetarrowsend{to}
\definecolor{dialinecolor}{rgb}{0.000000, 0.000000, 0.000000}
\pgfsetstrokecolor{dialinecolor}
\draw (16.604317\du,21.027085\du)--(14.906250\du,28.968750\du);
}
% setfont left to latex
\definecolor{dialinecolor}{rgb}{0.000000, 0.000000, 0.000000}
\pgfsetstrokecolor{dialinecolor}
\node at (15.393750\du,30.168750\du){Corps des opérations};
% setfont left to latex
\definecolor{dialinecolor}{rgb}{0.000000, 0.000000, 0.000000}
\pgfsetstrokecolor{dialinecolor}
\node at (15.393750\du,30.968750\du){(implémentation)};
% setfont left to latex
\definecolor{dialinecolor}{rgb}{0.000000, 0.000000, 0.000000}
\pgfsetstrokecolor{dialinecolor}
\node at (34.081250\du,18.293750\du){En-tête des opératoins};
% setfont left to latex
\definecolor{dialinecolor}{rgb}{0.000000, 0.000000, 0.000000}
\pgfsetstrokecolor{dialinecolor}
\node at (34.081250\du,19.093750\du){(spécification)};
\pgfsetlinewidth{0.050000\du}
\pgfsetdash{}{0pt}
\pgfsetdash{}{0pt}
\pgfsetbuttcap
\pgfsetmiterjoin
\pgfsetlinewidth{0.050000\du}
\pgfsetbuttcap
\pgfsetmiterjoin
\pgfsetdash{}{0pt}
\definecolor{dialinecolor}{rgb}{1.000000, 1.000000, 1.000000}
\pgfsetfillcolor{dialinecolor}
\fill (15.786250\du,18.888750\du)--(15.786250\du,21.158542\du)--(17.982823\du,21.158542\du)--(17.982823\du,18.888750\du)--cycle;
\definecolor{dialinecolor}{rgb}{0.000000, 0.000000, 0.000000}
\pgfsetstrokecolor{dialinecolor}
\draw (15.786250\du,18.888750\du)--(15.786250\du,21.158542\du)--(17.982823\du,21.158542\du)--(17.982823\du,18.888750\du)--cycle;
\pgfsetbuttcap
\pgfsetmiterjoin
\pgfsetdash{}{0pt}
\definecolor{dialinecolor}{rgb}{0.000000, 0.000000, 0.000000}
\pgfsetstrokecolor{dialinecolor}
\draw (15.786250\du,18.888750\du)--(15.786250\du,21.158542\du)--(17.982823\du,21.158542\du)--(17.982823\du,18.888750\du)--cycle;
\pgfsetlinewidth{0.050000\du}
\pgfsetdash{}{0pt}
\pgfsetdash{}{0pt}
\pgfsetbuttcap
\pgfsetmiterjoin
\pgfsetlinewidth{0.050000\du}
\pgfsetbuttcap
\pgfsetmiterjoin
\pgfsetdash{}{0pt}
\definecolor{dialinecolor}{rgb}{1.000000, 1.000000, 1.000000}
\pgfsetfillcolor{dialinecolor}
\fill (12.041250\du,16.208750\du)--(12.041250\du,18.478542\du)--(14.237823\du,18.478542\du)--(14.237823\du,16.208750\du)--cycle;
\definecolor{dialinecolor}{rgb}{0.000000, 0.000000, 0.000000}
\pgfsetstrokecolor{dialinecolor}
\draw (12.041250\du,16.208750\du)--(12.041250\du,18.478542\du)--(14.237823\du,18.478542\du)--(14.237823\du,16.208750\du)--cycle;
\pgfsetbuttcap
\pgfsetmiterjoin
\pgfsetdash{}{0pt}
\definecolor{dialinecolor}{rgb}{0.000000, 0.000000, 0.000000}
\pgfsetstrokecolor{dialinecolor}
\draw (12.041250\du,16.208750\du)--(12.041250\du,18.478542\du)--(14.237823\du,18.478542\du)--(14.237823\du,16.208750\du)--cycle;
\pgfsetlinewidth{0.050000\du}
\pgfsetdash{}{0pt}
\pgfsetdash{}{0pt}
\pgfsetbuttcap
\pgfsetmiterjoin
\pgfsetlinewidth{0.050000\du}
\pgfsetbuttcap
\pgfsetmiterjoin
\pgfsetdash{}{0pt}
\definecolor{dialinecolor}{rgb}{1.000000, 1.000000, 1.000000}
\pgfsetfillcolor{dialinecolor}
\fill (19.301250\du,15.848750\du)--(19.301250\du,18.118542\du)--(21.497823\du,18.118542\du)--(21.497823\du,15.848750\du)--cycle;
\definecolor{dialinecolor}{rgb}{0.000000, 0.000000, 0.000000}
\pgfsetstrokecolor{dialinecolor}
\draw (19.301250\du,15.848750\du)--(19.301250\du,18.118542\du)--(21.497823\du,18.118542\du)--(21.497823\du,15.848750\du)--cycle;
\pgfsetbuttcap
\pgfsetmiterjoin
\pgfsetdash{}{0pt}
\definecolor{dialinecolor}{rgb}{0.000000, 0.000000, 0.000000}
\pgfsetstrokecolor{dialinecolor}
\draw (19.301250\du,15.848750\du)--(19.301250\du,18.118542\du)--(21.497823\du,18.118542\du)--(21.497823\du,15.848750\du)--cycle;
\end{tikzpicture}
	
\section{Spécification d'un TAD}
	\paragraph{Définition} 
		Ensemble des opérations et des propriétés du type. \\
		Mode d'emploi de la structure de données
	\subsection{Opération}
		Énumération de l'ensemble des opérations selon la syntaxe pour une opération.
		$$ f: x_{1} \times x_{2} \times \dots \times x_{n} \rightarrow y_{1} \times y_{2} \times \dots \times y_{n}$$
		\paragraph{$f$} nom opération (symbole de fonction)
		\paragraph{$x_{1} \times x_{2} \times \dots \times x_{n}$} Domaines d'entrée de l'opération
		\paragraph{$y_{1} \times y_{2} \times \dots \times y_{n}$} Domaine de sortie de l'opération
		\paragraph{} Où les $x_{i}$ et $y_{i}$ sont des types dont l'un au moins est le type $T$ étudié
		\subsubsection{Exemple}
			Soit à définir le TAD Point d'affichage d'un point à l'écran.\\
			\subparagraph{Opération} 	
				\begin{itemize}
					\item Créer un point d'abscisse $x$, d'ordonnée $y$, de couleur $c$ et de taille $t$
					\item Connaître l'abscisse d'un point $p$
					\item Modifier la taille $t$ d'un point $p$
					\item Translater un point $p$ de $tx$ et $ty$
				\end{itemize}
			\subparagraph{Plus formellement}
				Soit Point le type point.
				\begin{eqnarray*}
					unPoint: Reel \times Reel \times Couleur \times Reel &\rightarrow& Point\\
					PointOrigine &\rightarrow& Point \\
					taille : Point &\rightarrow& Reel \\
					modifierTaille : Point \times Reel &\rightarrow& Point \\
					translater : Point \times Reel \times Reel &\rightarrow& Point\\
					\vdots \\ \vdots
				\end{eqnarray*}	
					Suite sur moodle clef = tads2
		\paragraph{Remarque}
			Une opération peut-être partielle\\
			Par exemple la taille $t$ du point doit être positive \\
			On définit une pré condition: pour $p$ de type Point $x$,$y$x et $t$ de type Réel et $c$ de type Couleur.\\
			$unPoint(x,y,c,t)$ est défini par $t > 0$
	\subsection{Propriétés}
		\paragraph{Opération} Syntaxe du Type Abstrait de Données
		\paragraph{Propriétés} Sémantique du Type Abstrait de Données
		\paragraph{Pour définir les propriétés} On combine les opérations entre elles et on indique le résultat de ces combinaisons.
		\subsubsection{On définit pour les opérations}
			% Graphic for TeX using PGF
% Title: /home/satenske/Diagramme1.dia
% Creator: Dia v0.97.1
% CreationDate: Fri Feb  4 20:12:22 2011
% For: satenske
% \usepackage{tikz}
% The following commands are not supported in PSTricks at present
% We define them conditionally, so when they are implemented,
% this pgf file will use them.
\ifx\du\undefined
  \newlength{\du}
\fi
\setlength{\du}{15\unitlength}
\begin{tikzpicture}
\pgftransformxscale{1.000000}
\pgftransformyscale{-1.000000}
\definecolor{dialinecolor}{rgb}{0.000000, 0.000000, 0.000000}
\pgfsetstrokecolor{dialinecolor}
\definecolor{dialinecolor}{rgb}{1.000000, 1.000000, 1.000000}
\pgfsetfillcolor{dialinecolor}
% setfont left to latex
\definecolor{dialinecolor}{rgb}{0.000000, 0.000000, 0.000000}
\pgfsetstrokecolor{dialinecolor}
\node[anchor=west] at (5.550000\du,11.900000\du){Opérations};
% setfont left to latex
\definecolor{dialinecolor}{rgb}{0.000000, 0.000000, 0.000000}
\pgfsetstrokecolor{dialinecolor}
\node[anchor=west] at (12.950000\du,8.600000\du){Observateur};
% setfont left to latex
\definecolor{dialinecolor}{rgb}{0.000000, 0.000000, 0.000000}
\pgfsetstrokecolor{dialinecolor}
\node[anchor=west] at (13.550000\du,15.400000\du){Générateurs};
% setfont left to latex
\definecolor{dialinecolor}{rgb}{0.000000, 0.000000, 0.000000}
\pgfsetstrokecolor{dialinecolor}
\node[anchor=west] at (21.350000\du,12.800000\du){Générateurs de base};
% setfont left to latex
\definecolor{dialinecolor}{rgb}{0.000000, 0.000000, 0.000000}
\pgfsetstrokecolor{dialinecolor}
\node[anchor=west] at (21.550000\du,18.350000\du){Générateurs secondaires};
\pgfsetlinewidth{0.050000\du}
\pgfsetdash{}{0pt}
\pgfsetdash{}{0pt}
\pgfsetbuttcap
{
\definecolor{dialinecolor}{rgb}{0.000000, 0.000000, 0.000000}
\pgfsetfillcolor{dialinecolor}
% was here!!!
\pgfsetarrowsend{stealth}
\definecolor{dialinecolor}{rgb}{0.000000, 0.000000, 0.000000}
\pgfsetstrokecolor{dialinecolor}
\draw (8.850000\du,11.050000\du)--(12.500000\du,8.650000\du);
}
\pgfsetlinewidth{0.050000\du}
\pgfsetdash{}{0pt}
\pgfsetdash{}{0pt}
\pgfsetbuttcap
{
\definecolor{dialinecolor}{rgb}{0.000000, 0.000000, 0.000000}
\pgfsetfillcolor{dialinecolor}
% was here!!!
\pgfsetarrowsend{stealth}
\definecolor{dialinecolor}{rgb}{0.000000, 0.000000, 0.000000}
\pgfsetstrokecolor{dialinecolor}
\draw (8.700000\du,12.550000\du)--(12.850000\du,15.300000\du);
}
\pgfsetlinewidth{0.050000\du}
\pgfsetdash{}{0pt}
\pgfsetdash{}{0pt}
\pgfsetbuttcap
{
\definecolor{dialinecolor}{rgb}{0.000000, 0.000000, 0.000000}
\pgfsetfillcolor{dialinecolor}
% was here!!!
\pgfsetarrowsend{stealth}
\definecolor{dialinecolor}{rgb}{0.000000, 0.000000, 0.000000}
\pgfsetstrokecolor{dialinecolor}
\draw (17.950000\du,15.100000\du)--(20.550000\du,13.050000\du);
}
\pgfsetlinewidth{0.050000\du}
\pgfsetdash{}{0pt}
\pgfsetdash{}{0pt}
\pgfsetbuttcap
{
\definecolor{dialinecolor}{rgb}{0.000000, 0.000000, 0.000000}
\pgfsetfillcolor{dialinecolor}
% was here!!!
\pgfsetarrowsend{stealth}
\definecolor{dialinecolor}{rgb}{0.000000, 0.000000, 0.000000}
\pgfsetstrokecolor{dialinecolor}
\draw (18.350000\du,16.200000\du)--(21.250000\du,18.150000\du);
}
\end{tikzpicture}
	
			\subparagraph{Observateur} Opération qui fournit une caractéristique de la donnée (sans la modifier)
			\subparagraph{Opérateur} Opération qui fournit une valeur du type abstrait étudié
			\subparagraph{Générateur de base} Générateur qui permet de construire toutes les valeurs du type
			\subparagraph{Générateur secondaire} Générateur autre qu'un générateur de base
			\subparagraph{Pour écrire les propriétés}
				\begin{itemize}
					\item on fournit les valeurs des observateurs et des générateurs secondaires appliqués au générateurs de base
					\item On peut aussi procéder par équivalence avec le générateur de base
				\end{itemize}
			\subsubsection{Pour le TAD point}
				\begin{tabular}{c|c}
					Générateur de base & unPoint\\
					\hline
										  & pointOrigine;\\
					Générateur Secondaire & modifierTaille;\\
										  & translater;\\
					\hline
					Observateur & taille
				\end{tabular}
			
			\subsubsection{Propriété du TAD Point}
				Pour $x$, $y$, $tx$, $ty$, de type Réel.\\
				Pour $c$ de type Couleur\\
				\begin{eqnarray}
					pointOrigine&=&unPoint(0.0, 0.0, noir, 1.0)\\	
					taille(unPoint(x, y, c, t))&=&t\\
					modifierTaille(unPoint(x,y,c,tx),t2)&=&untPoint(x,y,c,t2)\\
					translater(unPoint(x,y,c,t),tx,ty&=&unPoint(x \times ty, y \times ty, c, t)
				\end{eqnarray}
		\subsection{}		
			% Graphic for TeX using PGF
% Title: /usr/home/satenske/Diagram1.dia
% Creator: Dia v0.97.1
% CreationDate: Wed Mar 30 09:12:35 2011
% For: satenske
% \usepackage{tikz}
% The following commands are not supported in PSTricks at present
% We define them conditionally, so when they are implemented,
% this pgf file will use them.
\ifx\du\undefined
  \newlength{\du}
\fi
\setlength{\du}{15\unitlength}
\begin{tikzpicture}
\pgftransformxscale{1.000000}
\pgftransformyscale{-1.000000}
\definecolor{dialinecolor}{rgb}{0.000000, 0.000000, 0.000000}
\pgfsetstrokecolor{dialinecolor}
\definecolor{dialinecolor}{rgb}{1.000000, 1.000000, 1.000000}
\pgfsetfillcolor{dialinecolor}
\pgfsetlinewidth{0.100000\du}
\pgfsetdash{}{0pt}
\pgfsetdash{}{0pt}
\pgfsetbuttcap
\pgfsetmiterjoin
\pgfsetlinewidth{0.001000\du}
\pgfsetbuttcap
\pgfsetmiterjoin
\pgfsetdash{}{0pt}
\definecolor{dialinecolor}{rgb}{0.717647, 0.717647, 0.615686}
\pgfsetfillcolor{dialinecolor}
\pgfpathmoveto{\pgfpoint{9.544836\du}{9.940680\du}}
\pgfpathlineto{\pgfpoint{11.543262\du}{9.940680\du}}
\pgfpathlineto{\pgfpoint{11.543262\du}{10.310013\du}}
\pgfpathlineto{\pgfpoint{9.544836\du}{10.310013\du}}
\pgfpathlineto{\pgfpoint{9.544836\du}{9.940680\du}}
\pgfusepath{fill}
\pgfsetbuttcap
\pgfsetmiterjoin
\pgfsetdash{}{0pt}
\definecolor{dialinecolor}{rgb}{0.286275, 0.286275, 0.211765}
\pgfsetstrokecolor{dialinecolor}
\pgfpathmoveto{\pgfpoint{9.544836\du}{9.940680\du}}
\pgfpathlineto{\pgfpoint{11.543262\du}{9.940680\du}}
\pgfpathlineto{\pgfpoint{11.543262\du}{10.310013\du}}
\pgfpathlineto{\pgfpoint{9.544836\du}{10.310013\du}}
\pgfpathlineto{\pgfpoint{9.544836\du}{9.940680\du}}
\pgfusepath{stroke}
\pgfsetbuttcap
\pgfsetmiterjoin
\pgfsetdash{}{0pt}
\definecolor{dialinecolor}{rgb}{0.788235, 0.788235, 0.713726}
\pgfsetfillcolor{dialinecolor}
\pgfpathmoveto{\pgfpoint{9.544836\du}{9.940680\du}}
\pgfpathlineto{\pgfpoint{9.756738\du}{9.739798\du}}
\pgfpathlineto{\pgfpoint{11.755164\du}{9.739798\du}}
\pgfpathlineto{\pgfpoint{11.543262\du}{9.940680\du}}
\pgfpathlineto{\pgfpoint{9.544836\du}{9.940680\du}}
\pgfusepath{fill}
\pgfsetbuttcap
\pgfsetmiterjoin
\pgfsetdash{}{0pt}
\definecolor{dialinecolor}{rgb}{0.286275, 0.286275, 0.211765}
\pgfsetstrokecolor{dialinecolor}
\pgfpathmoveto{\pgfpoint{9.544836\du}{9.940680\du}}
\pgfpathlineto{\pgfpoint{9.756738\du}{9.739798\du}}
\pgfpathlineto{\pgfpoint{11.755164\du}{9.739798\du}}
\pgfpathlineto{\pgfpoint{11.543262\du}{9.940680\du}}
\pgfpathlineto{\pgfpoint{9.544836\du}{9.940680\du}}
\pgfusepath{stroke}
\pgfsetlinewidth{0.106000\du}
\pgfsetbuttcap
\pgfsetmiterjoin
\pgfsetdash{}{0pt}
\definecolor{dialinecolor}{rgb}{0.000000, 0.000000, 0.000000}
\pgfsetstrokecolor{dialinecolor}
\pgfpathmoveto{\pgfpoint{11.431486\du}{10.108816\du}}
\pgfpathlineto{\pgfpoint{10.951952\du}{10.108816\du}}
\pgfusepath{stroke}
\pgfsetlinewidth{0.001000\du}
\pgfsetbuttcap
\pgfsetmiterjoin
\pgfsetdash{}{0pt}
\definecolor{dialinecolor}{rgb}{0.478431, 0.478431, 0.352941}
\pgfsetfillcolor{dialinecolor}
\pgfpathmoveto{\pgfpoint{11.543262\du}{10.310013\du}}
\pgfpathlineto{\pgfpoint{11.755164\du}{10.097481\du}}
\pgfpathlineto{\pgfpoint{11.755164\du}{9.739798\du}}
\pgfpathlineto{\pgfpoint{11.543262\du}{9.940680\du}}
\pgfpathlineto{\pgfpoint{11.543262\du}{10.310013\du}}
\pgfusepath{fill}
\pgfsetbuttcap
\pgfsetmiterjoin
\pgfsetdash{}{0pt}
\definecolor{dialinecolor}{rgb}{0.286275, 0.286275, 0.211765}
\pgfsetstrokecolor{dialinecolor}
\pgfpathmoveto{\pgfpoint{11.543262\du}{10.310013\du}}
\pgfpathlineto{\pgfpoint{11.755164\du}{10.097481\du}}
\pgfpathlineto{\pgfpoint{11.755164\du}{9.739798\du}}
\pgfpathlineto{\pgfpoint{11.543262\du}{9.940680\du}}
\pgfpathlineto{\pgfpoint{11.543262\du}{10.310013\du}}
\pgfusepath{stroke}
\pgfsetbuttcap
\pgfsetmiterjoin
\pgfsetdash{}{0pt}
\definecolor{dialinecolor}{rgb}{0.788235, 0.788235, 0.713726}
\pgfsetfillcolor{dialinecolor}
\pgfpathmoveto{\pgfpoint{9.556171\du}{10.544270\du}}
\pgfpathlineto{\pgfpoint{9.779093\du}{10.264987\du}}
\pgfpathlineto{\pgfpoint{11.320025\du}{10.264987\du}}
\pgfpathlineto{\pgfpoint{11.097103\du}{10.544270\du}}
\pgfpathlineto{\pgfpoint{9.556171\du}{10.544270\du}}
\pgfusepath{fill}
\pgfsetbuttcap
\pgfsetmiterjoin
\pgfsetdash{}{0pt}
\definecolor{dialinecolor}{rgb}{0.286275, 0.286275, 0.211765}
\pgfsetstrokecolor{dialinecolor}
\pgfpathmoveto{\pgfpoint{9.556171\du}{10.544270\du}}
\pgfpathlineto{\pgfpoint{9.779093\du}{10.264987\du}}
\pgfpathlineto{\pgfpoint{11.320025\du}{10.264987\du}}
\pgfpathlineto{\pgfpoint{11.097103\du}{10.544270\du}}
\pgfpathlineto{\pgfpoint{9.556171\du}{10.544270\du}}
\pgfusepath{stroke}
\pgfsetbuttcap
\pgfsetmiterjoin
\pgfsetdash{}{0pt}
\definecolor{dialinecolor}{rgb}{0.478431, 0.478431, 0.352941}
\pgfsetfillcolor{dialinecolor}
\pgfpathmoveto{\pgfpoint{11.097103\du}{10.600000\du}}
\pgfpathlineto{\pgfpoint{11.320025\du}{10.365743\du}}
\pgfpathlineto{\pgfpoint{11.320025\du}{10.264987\du}}
\pgfpathlineto{\pgfpoint{11.097103\du}{10.544270\du}}
\pgfpathlineto{\pgfpoint{11.097103\du}{10.600000\du}}
\pgfusepath{fill}
\pgfsetbuttcap
\pgfsetmiterjoin
\pgfsetdash{}{0pt}
\definecolor{dialinecolor}{rgb}{0.286275, 0.286275, 0.211765}
\pgfsetstrokecolor{dialinecolor}
\pgfpathmoveto{\pgfpoint{11.097103\du}{10.600000\du}}
\pgfpathlineto{\pgfpoint{11.320025\du}{10.365743\du}}
\pgfpathlineto{\pgfpoint{11.320025\du}{10.264987\du}}
\pgfpathlineto{\pgfpoint{11.097103\du}{10.544270\du}}
\pgfpathlineto{\pgfpoint{11.097103\du}{10.600000\du}}
\pgfusepath{stroke}
\pgfsetbuttcap
\pgfsetmiterjoin
\pgfsetdash{}{0pt}
\definecolor{dialinecolor}{rgb}{0.717647, 0.717647, 0.615686}
\pgfsetfillcolor{dialinecolor}
\pgfpathmoveto{\pgfpoint{9.556171\du}{10.544270\du}}
\pgfpathlineto{\pgfpoint{11.097103\du}{10.544270\du}}
\pgfpathlineto{\pgfpoint{11.097103\du}{10.600000\du}}
\pgfpathlineto{\pgfpoint{9.556171\du}{10.600000\du}}
\pgfpathlineto{\pgfpoint{9.556171\du}{10.544270\du}}
\pgfusepath{fill}
\pgfsetbuttcap
\pgfsetmiterjoin
\pgfsetdash{}{0pt}
\definecolor{dialinecolor}{rgb}{0.286275, 0.286275, 0.211765}
\pgfsetstrokecolor{dialinecolor}
\pgfpathmoveto{\pgfpoint{9.556171\du}{10.544270\du}}
\pgfpathlineto{\pgfpoint{11.097103\du}{10.544270\du}}
\pgfpathlineto{\pgfpoint{11.097103\du}{10.600000\du}}
\pgfpathlineto{\pgfpoint{9.556171\du}{10.600000\du}}
\pgfpathlineto{\pgfpoint{9.556171\du}{10.544270\du}}
\pgfusepath{stroke}
\pgfsetbuttcap
\pgfsetmiterjoin
\pgfsetdash{}{0pt}
\definecolor{dialinecolor}{rgb}{0.000000, 0.000000, 0.000000}
\pgfsetfillcolor{dialinecolor}
\pgfpathmoveto{\pgfpoint{9.846159\du}{9.896285\du}}
\pgfpathlineto{\pgfpoint{10.013980\du}{9.739798\du}}
\pgfpathlineto{\pgfpoint{11.431486\du}{9.739798\du}}
\pgfpathlineto{\pgfpoint{11.275630\du}{9.896285\du}}
\pgfpathlineto{\pgfpoint{9.846159\du}{9.896285\du}}
\pgfusepath{fill}
\pgfsetbuttcap
\pgfsetmiterjoin
\pgfsetdash{}{0pt}
\definecolor{dialinecolor}{rgb}{0.000000, 0.000000, 0.000000}
\pgfsetstrokecolor{dialinecolor}
\pgfpathmoveto{\pgfpoint{9.846159\du}{9.896285\du}}
\pgfpathlineto{\pgfpoint{10.013980\du}{9.739798\du}}
\pgfpathlineto{\pgfpoint{11.431486\du}{9.739798\du}}
\pgfpathlineto{\pgfpoint{11.275630\du}{9.896285\du}}
\pgfpathlineto{\pgfpoint{9.846159\du}{9.896285\du}}
\pgfusepath{stroke}
\pgfsetbuttcap
\pgfsetmiterjoin
\pgfsetdash{}{0pt}
\definecolor{dialinecolor}{rgb}{0.788235, 0.788235, 0.713726}
\pgfsetfillcolor{dialinecolor}
\pgfpathmoveto{\pgfpoint{9.834824\du}{8.745151\du}}
\pgfpathlineto{\pgfpoint{9.991625\du}{8.600000\du}}
\pgfpathlineto{\pgfpoint{11.409761\du}{8.600000\du}}
\pgfpathlineto{\pgfpoint{11.252960\du}{8.745151\du}}
\pgfpathlineto{\pgfpoint{9.834824\du}{8.745151\du}}
\pgfusepath{fill}
\pgfsetbuttcap
\pgfsetmiterjoin
\pgfsetdash{}{0pt}
\definecolor{dialinecolor}{rgb}{0.286275, 0.286275, 0.211765}
\pgfsetstrokecolor{dialinecolor}
\pgfpathmoveto{\pgfpoint{9.834824\du}{8.745151\du}}
\pgfpathlineto{\pgfpoint{9.991625\du}{8.600000\du}}
\pgfpathlineto{\pgfpoint{11.409761\du}{8.600000\du}}
\pgfpathlineto{\pgfpoint{11.252960\du}{8.745151\du}}
\pgfpathlineto{\pgfpoint{9.834824\du}{8.745151\du}}
\pgfusepath{stroke}
\pgfsetbuttcap
\pgfsetmiterjoin
\pgfsetdash{}{0pt}
\definecolor{dialinecolor}{rgb}{0.717647, 0.717647, 0.615686}
\pgfsetfillcolor{dialinecolor}
\pgfpathmoveto{\pgfpoint{9.834824\du}{8.745151\du}}
\pgfpathlineto{\pgfpoint{11.264295\du}{8.745151\du}}
\pgfpathlineto{\pgfpoint{11.264295\du}{9.873615\du}}
\pgfpathlineto{\pgfpoint{9.834824\du}{9.873615\du}}
\pgfpathlineto{\pgfpoint{9.834824\du}{8.745151\du}}
\pgfusepath{fill}
\pgfsetbuttcap
\pgfsetmiterjoin
\pgfsetdash{}{0pt}
\definecolor{dialinecolor}{rgb}{0.286275, 0.286275, 0.211765}
\pgfsetstrokecolor{dialinecolor}
\pgfpathmoveto{\pgfpoint{9.834824\du}{8.745151\du}}
\pgfpathlineto{\pgfpoint{11.263665\du}{8.745151\du}}
\pgfpathlineto{\pgfpoint{11.263665\du}{9.873300\du}}
\pgfpathlineto{\pgfpoint{9.834824\du}{9.873300\du}}
\pgfpathlineto{\pgfpoint{9.834824\du}{8.745151\du}}
\pgfusepath{stroke}
\pgfsetbuttcap
\pgfsetmiterjoin
\pgfsetdash{}{0pt}
\definecolor{dialinecolor}{rgb}{1.000000, 1.000000, 1.000000}
\pgfsetfillcolor{dialinecolor}
\pgfpathmoveto{\pgfpoint{9.957620\du}{8.889987\du}}
\pgfpathlineto{\pgfpoint{11.141184\du}{8.889987\du}}
\pgfpathlineto{\pgfpoint{11.141184\du}{9.761839\du}}
\pgfpathlineto{\pgfpoint{9.957620\du}{9.761839\du}}
\pgfpathlineto{\pgfpoint{9.957620\du}{8.889987\du}}
\pgfusepath{fill}
\pgfsetbuttcap
\pgfsetmiterjoin
\pgfsetdash{}{0pt}
\definecolor{dialinecolor}{rgb}{0.286275, 0.286275, 0.211765}
\pgfsetstrokecolor{dialinecolor}
\pgfpathmoveto{\pgfpoint{9.957620\du}{8.889987\du}}
\pgfpathlineto{\pgfpoint{11.141184\du}{8.889987\du}}
\pgfpathlineto{\pgfpoint{11.141184\du}{9.761524\du}}
\pgfpathlineto{\pgfpoint{9.957620\du}{9.761524\du}}
\pgfpathlineto{\pgfpoint{9.957620\du}{8.889987\du}}
\pgfusepath{stroke}
\pgfsetbuttcap
\pgfsetmiterjoin
\pgfsetdash{}{0pt}
\definecolor{dialinecolor}{rgb}{0.478431, 0.478431, 0.352941}
\pgfsetfillcolor{dialinecolor}
\pgfpathmoveto{\pgfpoint{11.252960\du}{9.862909\du}}
\pgfpathlineto{\pgfpoint{11.409761\du}{9.706423\du}}
\pgfpathlineto{\pgfpoint{11.409761\du}{8.600000\du}}
\pgfpathlineto{\pgfpoint{11.252960\du}{8.745151\du}}
\pgfpathlineto{\pgfpoint{11.252960\du}{9.862909\du}}
\pgfusepath{fill}
\pgfsetbuttcap
\pgfsetmiterjoin
\pgfsetdash{}{0pt}
\definecolor{dialinecolor}{rgb}{0.286275, 0.286275, 0.211765}
\pgfsetstrokecolor{dialinecolor}
\pgfpathmoveto{\pgfpoint{11.252960\du}{9.862909\du}}
\pgfpathlineto{\pgfpoint{11.409761\du}{9.706423\du}}
\pgfpathlineto{\pgfpoint{11.409761\du}{8.600000\du}}
\pgfpathlineto{\pgfpoint{11.252960\du}{8.745151\du}}
\pgfpathlineto{\pgfpoint{11.252960\du}{9.862909\du}}
\pgfusepath{stroke}
\pgfsetlinewidth{0.100000\du}
\pgfsetdash{}{0pt}
\pgfsetdash{}{0pt}
\pgfsetbuttcap
\pgfsetmiterjoin
\pgfsetlinewidth{0.001000\du}
\pgfsetbuttcap
\pgfsetmiterjoin
\pgfsetdash{}{0pt}
\definecolor{dialinecolor}{rgb}{0.717647, 0.717647, 0.615686}
\pgfsetfillcolor{dialinecolor}
\pgfpathmoveto{\pgfpoint{20.628989\du}{8.998912\du}}
\pgfpathlineto{\pgfpoint{20.628989\du}{10.850000\du}}
\pgfpathlineto{\pgfpoint{21.723081\du}{10.850000\du}}
\pgfpathlineto{\pgfpoint{21.723081\du}{8.998912\du}}
\pgfpathlineto{\pgfpoint{20.628989\du}{8.998912\du}}
\pgfusepath{fill}
\pgfsetbuttcap
\pgfsetmiterjoin
\pgfsetdash{}{0pt}
\definecolor{dialinecolor}{rgb}{0.286275, 0.286275, 0.211765}
\pgfsetstrokecolor{dialinecolor}
\pgfpathmoveto{\pgfpoint{20.628989\du}{8.998912\du}}
\pgfpathlineto{\pgfpoint{20.628989\du}{10.850000\du}}
\pgfpathlineto{\pgfpoint{21.723081\du}{10.850000\du}}
\pgfpathlineto{\pgfpoint{21.723081\du}{8.998912\du}}
\pgfpathlineto{\pgfpoint{20.628989\du}{8.998912\du}}
\pgfusepath{stroke}
\pgfsetbuttcap
\pgfsetmiterjoin
\pgfsetdash{}{0pt}
\definecolor{dialinecolor}{rgb}{0.788235, 0.788235, 0.713726}
\pgfsetfillcolor{dialinecolor}
\pgfpathmoveto{\pgfpoint{20.628989\du}{8.998912\du}}
\pgfpathlineto{\pgfpoint{20.777246\du}{8.850000\du}}
\pgfpathlineto{\pgfpoint{21.871011\du}{8.850000\du}}
\pgfpathlineto{\pgfpoint{21.723081\du}{8.998912\du}}
\pgfpathlineto{\pgfpoint{20.628989\du}{8.998912\du}}
\pgfusepath{fill}
\pgfsetbuttcap
\pgfsetmiterjoin
\pgfsetdash{}{0pt}
\definecolor{dialinecolor}{rgb}{0.286275, 0.286275, 0.211765}
\pgfsetstrokecolor{dialinecolor}
\pgfpathmoveto{\pgfpoint{20.628989\du}{8.998912\du}}
\pgfpathlineto{\pgfpoint{20.777246\du}{8.850000\du}}
\pgfpathlineto{\pgfpoint{21.863811\du}{8.850000\du}}
\pgfusepath{stroke}
\pgfsetbuttcap
\pgfsetmiterjoin
\pgfsetdash{}{0pt}
\definecolor{dialinecolor}{rgb}{0.286275, 0.286275, 0.211765}
\pgfsetstrokecolor{dialinecolor}
\pgfpathmoveto{\pgfpoint{21.863811\du}{8.857527\du}}
\pgfpathlineto{\pgfpoint{21.723081\du}{8.998912\du}}
\pgfpathlineto{\pgfpoint{20.628989\du}{8.998912\du}}
\pgfusepath{stroke}
\pgfsetbuttcap
\pgfsetmiterjoin
\pgfsetdash{}{0pt}
\definecolor{dialinecolor}{rgb}{0.788235, 0.788235, 0.713726}
\pgfsetfillcolor{dialinecolor}
\pgfpathmoveto{\pgfpoint{20.696408\du}{9.106586\du}}
\pgfpathlineto{\pgfpoint{21.196163\du}{9.106586\du}}
\pgfpathlineto{\pgfpoint{21.196163\du}{9.349427\du}}
\pgfpathlineto{\pgfpoint{20.696408\du}{9.349427\du}}
\pgfpathlineto{\pgfpoint{20.696408\du}{9.106586\du}}
\pgfusepath{fill}
\pgfsetbuttcap
\pgfsetmiterjoin
\pgfsetdash{}{0pt}
\definecolor{dialinecolor}{rgb}{0.384314, 0.384314, 0.282353}
\pgfsetstrokecolor{dialinecolor}
\pgfpathmoveto{\pgfpoint{20.696408\du}{9.106586\du}}
\pgfpathlineto{\pgfpoint{21.195835\du}{9.106586\du}}
\pgfpathlineto{\pgfpoint{21.195835\du}{9.349100\du}}
\pgfpathlineto{\pgfpoint{20.696408\du}{9.349100\du}}
\pgfpathlineto{\pgfpoint{20.696408\du}{9.106586\du}}
\pgfusepath{stroke}
\pgfsetlinewidth{0.030000\du}
\pgfsetbuttcap
\pgfsetmiterjoin
\pgfsetdash{}{0pt}
\definecolor{dialinecolor}{rgb}{0.925490, 0.925490, 0.905882}
\pgfsetstrokecolor{dialinecolor}
\pgfpathmoveto{\pgfpoint{20.763828\du}{9.228334\du}}
\pgfpathlineto{\pgfpoint{21.114343\du}{9.228334\du}}
\pgfusepath{stroke}
\pgfsetlinewidth{0.001000\du}
\pgfsetbuttcap
\pgfsetmiterjoin
\pgfsetdash{}{0pt}
\definecolor{dialinecolor}{rgb}{0.478431, 0.478431, 0.352941}
\pgfsetfillcolor{dialinecolor}
\pgfpathmoveto{\pgfpoint{21.723081\du}{10.850000\du}}
\pgfpathlineto{\pgfpoint{21.871011\du}{10.700761\du}}
\pgfpathlineto{\pgfpoint{21.871011\du}{8.850000\du}}
\pgfpathlineto{\pgfpoint{21.723081\du}{8.998912\du}}
\pgfpathlineto{\pgfpoint{21.723081\du}{10.850000\du}}
\pgfusepath{fill}
\pgfsetbuttcap
\pgfsetmiterjoin
\pgfsetdash{}{0pt}
\definecolor{dialinecolor}{rgb}{0.286275, 0.286275, 0.211765}
\pgfsetstrokecolor{dialinecolor}
\pgfpathmoveto{\pgfpoint{21.723081\du}{10.850000\du}}
\pgfpathlineto{\pgfpoint{21.863811\du}{10.708288\du}}
\pgfusepath{stroke}
\pgfsetbuttcap
\pgfsetmiterjoin
\pgfsetdash{}{0pt}
\definecolor{dialinecolor}{rgb}{0.286275, 0.286275, 0.211765}
\pgfsetstrokecolor{dialinecolor}
\pgfpathmoveto{\pgfpoint{21.863811\du}{8.857527\du}}
\pgfpathlineto{\pgfpoint{21.723081\du}{8.998912\du}}
\pgfpathlineto{\pgfpoint{21.723081\du}{10.850000\du}}
\pgfusepath{stroke}
\pgfsetlinewidth{0.030000\du}
\pgfsetbuttcap
\pgfsetmiterjoin
\pgfsetdash{}{0pt}
\definecolor{dialinecolor}{rgb}{0.925490, 0.925490, 0.905882}
\pgfsetstrokecolor{dialinecolor}
\pgfpathmoveto{\pgfpoint{20.642734\du}{10.727925\du}}
\pgfpathlineto{\pgfpoint{21.722754\du}{10.727925\du}}
\pgfusepath{stroke}
\pgfsetbuttcap
\pgfsetmiterjoin
\pgfsetdash{}{0pt}
\definecolor{dialinecolor}{rgb}{0.000000, 0.000000, 0.000000}
\pgfsetstrokecolor{dialinecolor}
\pgfpathmoveto{\pgfpoint{20.642734\du}{9.741834\du}}
\pgfpathlineto{\pgfpoint{21.722754\du}{9.741834\du}}
\pgfusepath{stroke}
\pgfsetbuttcap
\pgfsetmiterjoin
\pgfsetdash{}{0pt}
\definecolor{dialinecolor}{rgb}{0.286275, 0.286275, 0.211765}
\pgfsetstrokecolor{dialinecolor}
\pgfpathmoveto{\pgfpoint{20.628989\du}{10.714507\du}}
\pgfpathlineto{\pgfpoint{21.721772\du}{10.714507\du}}
\pgfusepath{stroke}
\pgfsetbuttcap
\pgfsetmiterjoin
\pgfsetdash{}{0pt}
\definecolor{dialinecolor}{rgb}{0.000000, 0.000000, 0.000000}
\pgfsetstrokecolor{dialinecolor}
\pgfpathmoveto{\pgfpoint{20.628989\du}{9.728089\du}}
\pgfpathlineto{\pgfpoint{21.721772\du}{9.728089\du}}
\pgfusepath{stroke}
\pgfsetlinewidth{0.001000\du}
\pgfsetbuttcap
\pgfsetmiterjoin
\pgfsetdash{}{0pt}
\definecolor{dialinecolor}{rgb}{0.925490, 0.925490, 0.905882}
\pgfsetstrokecolor{dialinecolor}
\pgfpathmoveto{\pgfpoint{20.696408\du}{9.336336\du}}
\pgfpathlineto{\pgfpoint{20.696408\du}{9.106586\du}}
\pgfpathlineto{\pgfpoint{21.182417\du}{9.106586\du}}
\pgfusepath{stroke}
\pgfsetlinewidth{0.100000\du}
\pgfsetdash{}{0pt}
\pgfsetdash{}{0pt}
\pgfsetbuttcap
\pgfsetmiterjoin
\pgfsetlinewidth{0.001000\du}
\pgfsetbuttcap
\pgfsetmiterjoin
\pgfsetdash{}{0pt}
\definecolor{dialinecolor}{rgb}{0.788235, 0.788235, 0.713726}
\pgfsetfillcolor{dialinecolor}
\pgfpathmoveto{\pgfpoint{22.752053\du}{13.227519\du}}
\pgfpathlineto{\pgfpoint{22.999199\du}{13.000000\du}}
\pgfpathlineto{\pgfpoint{25.247947\du}{13.000000\du}}
\pgfpathlineto{\pgfpoint{25.000401\du}{13.227519\du}}
\pgfpathlineto{\pgfpoint{22.752053\du}{13.227519\du}}
\pgfusepath{fill}
\pgfsetbuttcap
\pgfsetmiterjoin
\pgfsetdash{}{0pt}
\definecolor{dialinecolor}{rgb}{0.286275, 0.286275, 0.211765}
\pgfsetstrokecolor{dialinecolor}
\pgfpathmoveto{\pgfpoint{22.763669\du}{13.217104\du}}
\pgfpathlineto{\pgfpoint{22.989986\du}{13.008412\du}}
\pgfpathlineto{\pgfpoint{25.239135\du}{13.008412\du}}
\pgfpathlineto{\pgfpoint{25.000401\du}{13.227519\du}}
\pgfpathlineto{\pgfpoint{22.763669\du}{13.227519\du}}
\pgfpathlineto{\pgfpoint{22.763669\du}{13.217104\du}}
\pgfusepath{stroke}
\pgfsetbuttcap
\pgfsetmiterjoin
\pgfsetdash{}{0pt}
\definecolor{dialinecolor}{rgb}{0.717647, 0.717647, 0.615686}
\pgfsetfillcolor{dialinecolor}
\pgfpathmoveto{\pgfpoint{22.752053\du}{13.227519\du}}
\pgfpathlineto{\pgfpoint{25.018426\du}{13.227519\du}}
\pgfpathlineto{\pgfpoint{25.018426\du}{15.000000\du}}
\pgfpathlineto{\pgfpoint{22.752053\du}{15.000000\du}}
\pgfpathlineto{\pgfpoint{22.752053\du}{13.227519\du}}
\pgfusepath{fill}
\pgfsetbuttcap
\pgfsetmiterjoin
\pgfsetdash{}{0pt}
\definecolor{dialinecolor}{rgb}{0.286275, 0.286275, 0.211765}
\pgfsetstrokecolor{dialinecolor}
\pgfpathmoveto{\pgfpoint{22.763669\du}{13.227519\du}}
\pgfpathlineto{\pgfpoint{25.017625\du}{13.227519\du}}
\pgfpathlineto{\pgfpoint{25.017625\du}{14.999199\du}}
\pgfpathlineto{\pgfpoint{22.763669\du}{14.999199\du}}
\pgfpathlineto{\pgfpoint{22.763669\du}{13.227519\du}}
\pgfusepath{stroke}
\pgfsetbuttcap
\pgfsetmiterjoin
\pgfsetdash{}{0pt}
\definecolor{dialinecolor}{rgb}{0.478431, 0.478431, 0.352941}
\pgfsetfillcolor{dialinecolor}
\pgfpathmoveto{\pgfpoint{25.000401\du}{14.982776\du}}
\pgfpathlineto{\pgfpoint{25.247947\du}{14.736832\du}}
\pgfpathlineto{\pgfpoint{25.247947\du}{13.000000\du}}
\pgfpathlineto{\pgfpoint{25.000401\du}{13.227519\du}}
\pgfpathlineto{\pgfpoint{25.000401\du}{14.982776\du}}
\pgfusepath{fill}
\pgfsetbuttcap
\pgfsetmiterjoin
\pgfsetdash{}{0pt}
\definecolor{dialinecolor}{rgb}{0.286275, 0.286275, 0.211765}
\pgfsetstrokecolor{dialinecolor}
\pgfpathmoveto{\pgfpoint{25.000401\du}{14.982776\du}}
\pgfpathlineto{\pgfpoint{25.247947\du}{14.736832\du}}
\pgfpathlineto{\pgfpoint{25.247947\du}{13.008412\du}}
\pgfpathlineto{\pgfpoint{25.239135\du}{13.008412\du}}
\pgfpathlineto{\pgfpoint{25.000401\du}{13.227519\du}}
\pgfpathlineto{\pgfpoint{25.000401\du}{14.982776\du}}
\pgfpathlineto{\pgfpoint{25.000401\du}{14.982776\du}}
\pgfusepath{stroke}
\pgfsetlinewidth{0.050000\du}
\pgfsetdash{}{0pt}
\pgfsetdash{}{0pt}
\pgfsetbuttcap
{
\definecolor{dialinecolor}{rgb}{0.000000, 0.000000, 0.000000}
\pgfsetfillcolor{dialinecolor}
% was here!!!
\pgfsetarrowsend{to}
\definecolor{dialinecolor}{rgb}{0.000000, 0.000000, 0.000000}
\pgfsetstrokecolor{dialinecolor}
\draw (11.755164\du,9.918640\du)--(20.628989\du,9.924456\du);
}
\pgfsetlinewidth{0.100000\du}
\pgfsetdash{}{0pt}
\pgfsetdash{}{0pt}
\pgfsetbuttcap
{
\definecolor{dialinecolor}{rgb}{0.000000, 0.000000, 0.000000}
\pgfsetfillcolor{dialinecolor}
% was here!!!
\pgfsetarrowsend{to}
\definecolor{dialinecolor}{rgb}{0.000000, 0.000000, 0.000000}
\pgfsetstrokecolor{dialinecolor}
\draw (21.870751\du,10.650831\du)--(23.500000\du,12.750000\du);
}
\pgfsetlinewidth{0.050000\du}
\pgfsetdash{}{0pt}
\pgfsetdash{}{0pt}
\pgfsetbuttcap
{
\definecolor{dialinecolor}{rgb}{0.000000, 0.000000, 0.000000}
\pgfsetfillcolor{dialinecolor}
% was here!!!
\pgfsetarrowsend{to}
\definecolor{dialinecolor}{rgb}{0.000000, 0.000000, 0.000000}
\pgfsetstrokecolor{dialinecolor}
\draw (22.757095\du,13.994376\du)--(17.400000\du,14.000000\du);
}
\pgfsetlinewidth{0.050000\du}
\pgfsetdash{}{0pt}
\pgfsetdash{}{0pt}
\pgfsetbuttcap
\pgfsetmiterjoin
\pgfsetlinewidth{0.050000\du}
\pgfsetbuttcap
\pgfsetmiterjoin
\pgfsetdash{}{0pt}
\definecolor{dialinecolor}{rgb}{1.000000, 1.000000, 1.000000}
\pgfsetfillcolor{dialinecolor}
\fill (14.682258\du,12.900000\du)--(14.682258\du,15.605000\du)--(17.300000\du,15.605000\du)--(17.300000\du,12.900000\du)--cycle;
\definecolor{dialinecolor}{rgb}{0.000000, 0.000000, 0.000000}
\pgfsetstrokecolor{dialinecolor}
\draw (14.682258\du,12.900000\du)--(14.682258\du,15.605000\du)--(17.300000\du,15.605000\du)--(17.300000\du,12.900000\du)--cycle;
\pgfsetbuttcap
\pgfsetmiterjoin
\pgfsetdash{}{0pt}
\definecolor{dialinecolor}{rgb}{0.000000, 0.000000, 0.000000}
\pgfsetstrokecolor{dialinecolor}
\draw (14.682258\du,12.900000\du)--(14.682258\du,15.605000\du)--(17.300000\du,15.605000\du)--(17.300000\du,12.900000\du)--cycle;
% setfont left to latex
\definecolor{dialinecolor}{rgb}{0.000000, 0.000000, 0.000000}
\pgfsetstrokecolor{dialinecolor}
\node[anchor=west] at (15.239919\du,14.659583\du){html};
\pgfsetlinewidth{0.050000\du}
\pgfsetdash{}{0pt}
\pgfsetdash{}{0pt}
\pgfsetbuttcap
{
\definecolor{dialinecolor}{rgb}{0.000000, 0.000000, 0.000000}
\pgfsetfillcolor{dialinecolor}
% was here!!!
\pgfsetarrowsend{to}
\definecolor{dialinecolor}{rgb}{0.000000, 0.000000, 0.000000}
\pgfsetstrokecolor{dialinecolor}
\draw (14.682258\du,14.252500\du)--(10.500000\du,10.750000\du);
}
% setfont left to latex
\definecolor{dialinecolor}{rgb}{0.000000, 0.000000, 0.000000}
\pgfsetstrokecolor{dialinecolor}
\node[anchor=west] at (8.950000\du,14.350000\du){Navigateur};
% setfont left to latex
\definecolor{dialinecolor}{rgb}{0.000000, 0.000000, 0.000000}
\pgfsetstrokecolor{dialinecolor}
\node[anchor=west] at (10.800000\du,7.600000\du){Client};
% setfont left to latex
\definecolor{dialinecolor}{rgb}{0.000000, 0.000000, 0.000000}
\pgfsetstrokecolor{dialinecolor}
\node[anchor=west] at (20.250000\du,7.800000\du){Serveur};
% setfont left to latex
\definecolor{dialinecolor}{rgb}{0.000000, 0.000000, 0.000000}
\pgfsetstrokecolor{dialinecolor}
\node[anchor=west] at (22.950000\du,15.500000\du){Apache};
% setfont left to latex
\definecolor{dialinecolor}{rgb}{0.000000, 0.000000, 0.000000}
\pgfsetstrokecolor{dialinecolor}
\node[anchor=west] at (19.400000\du,15.200000\du){PHP};
\end{tikzpicture}

			\paragraph{Remarques}
				\subparagraph{}
				Propriété = Axiome définit par le constructeur \\ 
				Par exemple on aurait pu imaginer:\\
				$$modifierTaille(unPoint(x,y,c,t1),t2)$$
				$$unPoint(x,y,c,t1+t2)$$
				\subparagraph{}
					Pré conditions valides lors de l'écriture des propriétés.
				\subparagraph{}
					Les types élémentaires (entier, Réel, booléen ...) sont des types abstraits de données déjà définis dans le langage. \\
					Pour Booléen: Vrai $\rightarrow$ Booléen \\
									Faux $\rightarrow$ Booléen \\
									Non: Booléen $\rightarrow$ Booléen
									Et: Booléen $\times$ Booléen $\rightarrow$ Booléen \\ ...\\ \\
					Propriétés: Non(vrai) = Faux\\
								Non(faux) = vrai\\
								Et(vrai,vrai) = vrai\\
								Et(vrai,faux) = faux.\\
				\subparagraph{}
					Les types composés (Tableau, enregistrement) sont aussi des Types Abstrait de Données.	\\
					Pour tableau:\\
						unTableau : Entier $\times$ Entier $\rightarrow$ Tableau[T]\\
						ième : Tableau[T] $\times$ Entier $\rightarrow$ T\\
						changeIème : Tableau[T] $\times$ Entier $\times$ T$ \rightarrow$ Tableau[T]\\ \\
						$ième(tab, i) \equiv tab[i]$
						$changerIème(tab, i, el) equiv tab[i]<-e$\\
					Propriétés:\\
			\lstinputlisting[caption=Propriété]{1.algo}
				\subparagraph{}
					Il existe des TADs fondamentaux les tables, listes, piles, fils, arbres, graphe\\
					En général ces TADs sont génériques.\\
					Par exemple pour le TAD liste on introduit le TAD $liste[T]$
	\section{Du type Abstrait au type concret}
		\paragraph{Définition}
			Un type concret est la transition dans un langage d'un type abstrait.
		\paragraph{Schéma de traduction}
	%			% Graphic for TeX using PGF
% Title: /usr/home/satenske/Diagram1.dia
% Creator: Dia v0.97.1
% CreationDate: Wed Mar 30 09:12:35 2011
% For: satenske
% \usepackage{tikz}
% The following commands are not supported in PSTricks at present
% We define them conditionally, so when they are implemented,
% this pgf file will use them.
\ifx\du\undefined
  \newlength{\du}
\fi
\setlength{\du}{15\unitlength}
\begin{tikzpicture}
\pgftransformxscale{1.000000}
\pgftransformyscale{-1.000000}
\definecolor{dialinecolor}{rgb}{0.000000, 0.000000, 0.000000}
\pgfsetstrokecolor{dialinecolor}
\definecolor{dialinecolor}{rgb}{1.000000, 1.000000, 1.000000}
\pgfsetfillcolor{dialinecolor}
\pgfsetlinewidth{0.100000\du}
\pgfsetdash{}{0pt}
\pgfsetdash{}{0pt}
\pgfsetbuttcap
\pgfsetmiterjoin
\pgfsetlinewidth{0.001000\du}
\pgfsetbuttcap
\pgfsetmiterjoin
\pgfsetdash{}{0pt}
\definecolor{dialinecolor}{rgb}{0.717647, 0.717647, 0.615686}
\pgfsetfillcolor{dialinecolor}
\pgfpathmoveto{\pgfpoint{9.544836\du}{9.940680\du}}
\pgfpathlineto{\pgfpoint{11.543262\du}{9.940680\du}}
\pgfpathlineto{\pgfpoint{11.543262\du}{10.310013\du}}
\pgfpathlineto{\pgfpoint{9.544836\du}{10.310013\du}}
\pgfpathlineto{\pgfpoint{9.544836\du}{9.940680\du}}
\pgfusepath{fill}
\pgfsetbuttcap
\pgfsetmiterjoin
\pgfsetdash{}{0pt}
\definecolor{dialinecolor}{rgb}{0.286275, 0.286275, 0.211765}
\pgfsetstrokecolor{dialinecolor}
\pgfpathmoveto{\pgfpoint{9.544836\du}{9.940680\du}}
\pgfpathlineto{\pgfpoint{11.543262\du}{9.940680\du}}
\pgfpathlineto{\pgfpoint{11.543262\du}{10.310013\du}}
\pgfpathlineto{\pgfpoint{9.544836\du}{10.310013\du}}
\pgfpathlineto{\pgfpoint{9.544836\du}{9.940680\du}}
\pgfusepath{stroke}
\pgfsetbuttcap
\pgfsetmiterjoin
\pgfsetdash{}{0pt}
\definecolor{dialinecolor}{rgb}{0.788235, 0.788235, 0.713726}
\pgfsetfillcolor{dialinecolor}
\pgfpathmoveto{\pgfpoint{9.544836\du}{9.940680\du}}
\pgfpathlineto{\pgfpoint{9.756738\du}{9.739798\du}}
\pgfpathlineto{\pgfpoint{11.755164\du}{9.739798\du}}
\pgfpathlineto{\pgfpoint{11.543262\du}{9.940680\du}}
\pgfpathlineto{\pgfpoint{9.544836\du}{9.940680\du}}
\pgfusepath{fill}
\pgfsetbuttcap
\pgfsetmiterjoin
\pgfsetdash{}{0pt}
\definecolor{dialinecolor}{rgb}{0.286275, 0.286275, 0.211765}
\pgfsetstrokecolor{dialinecolor}
\pgfpathmoveto{\pgfpoint{9.544836\du}{9.940680\du}}
\pgfpathlineto{\pgfpoint{9.756738\du}{9.739798\du}}
\pgfpathlineto{\pgfpoint{11.755164\du}{9.739798\du}}
\pgfpathlineto{\pgfpoint{11.543262\du}{9.940680\du}}
\pgfpathlineto{\pgfpoint{9.544836\du}{9.940680\du}}
\pgfusepath{stroke}
\pgfsetlinewidth{0.106000\du}
\pgfsetbuttcap
\pgfsetmiterjoin
\pgfsetdash{}{0pt}
\definecolor{dialinecolor}{rgb}{0.000000, 0.000000, 0.000000}
\pgfsetstrokecolor{dialinecolor}
\pgfpathmoveto{\pgfpoint{11.431486\du}{10.108816\du}}
\pgfpathlineto{\pgfpoint{10.951952\du}{10.108816\du}}
\pgfusepath{stroke}
\pgfsetlinewidth{0.001000\du}
\pgfsetbuttcap
\pgfsetmiterjoin
\pgfsetdash{}{0pt}
\definecolor{dialinecolor}{rgb}{0.478431, 0.478431, 0.352941}
\pgfsetfillcolor{dialinecolor}
\pgfpathmoveto{\pgfpoint{11.543262\du}{10.310013\du}}
\pgfpathlineto{\pgfpoint{11.755164\du}{10.097481\du}}
\pgfpathlineto{\pgfpoint{11.755164\du}{9.739798\du}}
\pgfpathlineto{\pgfpoint{11.543262\du}{9.940680\du}}
\pgfpathlineto{\pgfpoint{11.543262\du}{10.310013\du}}
\pgfusepath{fill}
\pgfsetbuttcap
\pgfsetmiterjoin
\pgfsetdash{}{0pt}
\definecolor{dialinecolor}{rgb}{0.286275, 0.286275, 0.211765}
\pgfsetstrokecolor{dialinecolor}
\pgfpathmoveto{\pgfpoint{11.543262\du}{10.310013\du}}
\pgfpathlineto{\pgfpoint{11.755164\du}{10.097481\du}}
\pgfpathlineto{\pgfpoint{11.755164\du}{9.739798\du}}
\pgfpathlineto{\pgfpoint{11.543262\du}{9.940680\du}}
\pgfpathlineto{\pgfpoint{11.543262\du}{10.310013\du}}
\pgfusepath{stroke}
\pgfsetbuttcap
\pgfsetmiterjoin
\pgfsetdash{}{0pt}
\definecolor{dialinecolor}{rgb}{0.788235, 0.788235, 0.713726}
\pgfsetfillcolor{dialinecolor}
\pgfpathmoveto{\pgfpoint{9.556171\du}{10.544270\du}}
\pgfpathlineto{\pgfpoint{9.779093\du}{10.264987\du}}
\pgfpathlineto{\pgfpoint{11.320025\du}{10.264987\du}}
\pgfpathlineto{\pgfpoint{11.097103\du}{10.544270\du}}
\pgfpathlineto{\pgfpoint{9.556171\du}{10.544270\du}}
\pgfusepath{fill}
\pgfsetbuttcap
\pgfsetmiterjoin
\pgfsetdash{}{0pt}
\definecolor{dialinecolor}{rgb}{0.286275, 0.286275, 0.211765}
\pgfsetstrokecolor{dialinecolor}
\pgfpathmoveto{\pgfpoint{9.556171\du}{10.544270\du}}
\pgfpathlineto{\pgfpoint{9.779093\du}{10.264987\du}}
\pgfpathlineto{\pgfpoint{11.320025\du}{10.264987\du}}
\pgfpathlineto{\pgfpoint{11.097103\du}{10.544270\du}}
\pgfpathlineto{\pgfpoint{9.556171\du}{10.544270\du}}
\pgfusepath{stroke}
\pgfsetbuttcap
\pgfsetmiterjoin
\pgfsetdash{}{0pt}
\definecolor{dialinecolor}{rgb}{0.478431, 0.478431, 0.352941}
\pgfsetfillcolor{dialinecolor}
\pgfpathmoveto{\pgfpoint{11.097103\du}{10.600000\du}}
\pgfpathlineto{\pgfpoint{11.320025\du}{10.365743\du}}
\pgfpathlineto{\pgfpoint{11.320025\du}{10.264987\du}}
\pgfpathlineto{\pgfpoint{11.097103\du}{10.544270\du}}
\pgfpathlineto{\pgfpoint{11.097103\du}{10.600000\du}}
\pgfusepath{fill}
\pgfsetbuttcap
\pgfsetmiterjoin
\pgfsetdash{}{0pt}
\definecolor{dialinecolor}{rgb}{0.286275, 0.286275, 0.211765}
\pgfsetstrokecolor{dialinecolor}
\pgfpathmoveto{\pgfpoint{11.097103\du}{10.600000\du}}
\pgfpathlineto{\pgfpoint{11.320025\du}{10.365743\du}}
\pgfpathlineto{\pgfpoint{11.320025\du}{10.264987\du}}
\pgfpathlineto{\pgfpoint{11.097103\du}{10.544270\du}}
\pgfpathlineto{\pgfpoint{11.097103\du}{10.600000\du}}
\pgfusepath{stroke}
\pgfsetbuttcap
\pgfsetmiterjoin
\pgfsetdash{}{0pt}
\definecolor{dialinecolor}{rgb}{0.717647, 0.717647, 0.615686}
\pgfsetfillcolor{dialinecolor}
\pgfpathmoveto{\pgfpoint{9.556171\du}{10.544270\du}}
\pgfpathlineto{\pgfpoint{11.097103\du}{10.544270\du}}
\pgfpathlineto{\pgfpoint{11.097103\du}{10.600000\du}}
\pgfpathlineto{\pgfpoint{9.556171\du}{10.600000\du}}
\pgfpathlineto{\pgfpoint{9.556171\du}{10.544270\du}}
\pgfusepath{fill}
\pgfsetbuttcap
\pgfsetmiterjoin
\pgfsetdash{}{0pt}
\definecolor{dialinecolor}{rgb}{0.286275, 0.286275, 0.211765}
\pgfsetstrokecolor{dialinecolor}
\pgfpathmoveto{\pgfpoint{9.556171\du}{10.544270\du}}
\pgfpathlineto{\pgfpoint{11.097103\du}{10.544270\du}}
\pgfpathlineto{\pgfpoint{11.097103\du}{10.600000\du}}
\pgfpathlineto{\pgfpoint{9.556171\du}{10.600000\du}}
\pgfpathlineto{\pgfpoint{9.556171\du}{10.544270\du}}
\pgfusepath{stroke}
\pgfsetbuttcap
\pgfsetmiterjoin
\pgfsetdash{}{0pt}
\definecolor{dialinecolor}{rgb}{0.000000, 0.000000, 0.000000}
\pgfsetfillcolor{dialinecolor}
\pgfpathmoveto{\pgfpoint{9.846159\du}{9.896285\du}}
\pgfpathlineto{\pgfpoint{10.013980\du}{9.739798\du}}
\pgfpathlineto{\pgfpoint{11.431486\du}{9.739798\du}}
\pgfpathlineto{\pgfpoint{11.275630\du}{9.896285\du}}
\pgfpathlineto{\pgfpoint{9.846159\du}{9.896285\du}}
\pgfusepath{fill}
\pgfsetbuttcap
\pgfsetmiterjoin
\pgfsetdash{}{0pt}
\definecolor{dialinecolor}{rgb}{0.000000, 0.000000, 0.000000}
\pgfsetstrokecolor{dialinecolor}
\pgfpathmoveto{\pgfpoint{9.846159\du}{9.896285\du}}
\pgfpathlineto{\pgfpoint{10.013980\du}{9.739798\du}}
\pgfpathlineto{\pgfpoint{11.431486\du}{9.739798\du}}
\pgfpathlineto{\pgfpoint{11.275630\du}{9.896285\du}}
\pgfpathlineto{\pgfpoint{9.846159\du}{9.896285\du}}
\pgfusepath{stroke}
\pgfsetbuttcap
\pgfsetmiterjoin
\pgfsetdash{}{0pt}
\definecolor{dialinecolor}{rgb}{0.788235, 0.788235, 0.713726}
\pgfsetfillcolor{dialinecolor}
\pgfpathmoveto{\pgfpoint{9.834824\du}{8.745151\du}}
\pgfpathlineto{\pgfpoint{9.991625\du}{8.600000\du}}
\pgfpathlineto{\pgfpoint{11.409761\du}{8.600000\du}}
\pgfpathlineto{\pgfpoint{11.252960\du}{8.745151\du}}
\pgfpathlineto{\pgfpoint{9.834824\du}{8.745151\du}}
\pgfusepath{fill}
\pgfsetbuttcap
\pgfsetmiterjoin
\pgfsetdash{}{0pt}
\definecolor{dialinecolor}{rgb}{0.286275, 0.286275, 0.211765}
\pgfsetstrokecolor{dialinecolor}
\pgfpathmoveto{\pgfpoint{9.834824\du}{8.745151\du}}
\pgfpathlineto{\pgfpoint{9.991625\du}{8.600000\du}}
\pgfpathlineto{\pgfpoint{11.409761\du}{8.600000\du}}
\pgfpathlineto{\pgfpoint{11.252960\du}{8.745151\du}}
\pgfpathlineto{\pgfpoint{9.834824\du}{8.745151\du}}
\pgfusepath{stroke}
\pgfsetbuttcap
\pgfsetmiterjoin
\pgfsetdash{}{0pt}
\definecolor{dialinecolor}{rgb}{0.717647, 0.717647, 0.615686}
\pgfsetfillcolor{dialinecolor}
\pgfpathmoveto{\pgfpoint{9.834824\du}{8.745151\du}}
\pgfpathlineto{\pgfpoint{11.264295\du}{8.745151\du}}
\pgfpathlineto{\pgfpoint{11.264295\du}{9.873615\du}}
\pgfpathlineto{\pgfpoint{9.834824\du}{9.873615\du}}
\pgfpathlineto{\pgfpoint{9.834824\du}{8.745151\du}}
\pgfusepath{fill}
\pgfsetbuttcap
\pgfsetmiterjoin
\pgfsetdash{}{0pt}
\definecolor{dialinecolor}{rgb}{0.286275, 0.286275, 0.211765}
\pgfsetstrokecolor{dialinecolor}
\pgfpathmoveto{\pgfpoint{9.834824\du}{8.745151\du}}
\pgfpathlineto{\pgfpoint{11.263665\du}{8.745151\du}}
\pgfpathlineto{\pgfpoint{11.263665\du}{9.873300\du}}
\pgfpathlineto{\pgfpoint{9.834824\du}{9.873300\du}}
\pgfpathlineto{\pgfpoint{9.834824\du}{8.745151\du}}
\pgfusepath{stroke}
\pgfsetbuttcap
\pgfsetmiterjoin
\pgfsetdash{}{0pt}
\definecolor{dialinecolor}{rgb}{1.000000, 1.000000, 1.000000}
\pgfsetfillcolor{dialinecolor}
\pgfpathmoveto{\pgfpoint{9.957620\du}{8.889987\du}}
\pgfpathlineto{\pgfpoint{11.141184\du}{8.889987\du}}
\pgfpathlineto{\pgfpoint{11.141184\du}{9.761839\du}}
\pgfpathlineto{\pgfpoint{9.957620\du}{9.761839\du}}
\pgfpathlineto{\pgfpoint{9.957620\du}{8.889987\du}}
\pgfusepath{fill}
\pgfsetbuttcap
\pgfsetmiterjoin
\pgfsetdash{}{0pt}
\definecolor{dialinecolor}{rgb}{0.286275, 0.286275, 0.211765}
\pgfsetstrokecolor{dialinecolor}
\pgfpathmoveto{\pgfpoint{9.957620\du}{8.889987\du}}
\pgfpathlineto{\pgfpoint{11.141184\du}{8.889987\du}}
\pgfpathlineto{\pgfpoint{11.141184\du}{9.761524\du}}
\pgfpathlineto{\pgfpoint{9.957620\du}{9.761524\du}}
\pgfpathlineto{\pgfpoint{9.957620\du}{8.889987\du}}
\pgfusepath{stroke}
\pgfsetbuttcap
\pgfsetmiterjoin
\pgfsetdash{}{0pt}
\definecolor{dialinecolor}{rgb}{0.478431, 0.478431, 0.352941}
\pgfsetfillcolor{dialinecolor}
\pgfpathmoveto{\pgfpoint{11.252960\du}{9.862909\du}}
\pgfpathlineto{\pgfpoint{11.409761\du}{9.706423\du}}
\pgfpathlineto{\pgfpoint{11.409761\du}{8.600000\du}}
\pgfpathlineto{\pgfpoint{11.252960\du}{8.745151\du}}
\pgfpathlineto{\pgfpoint{11.252960\du}{9.862909\du}}
\pgfusepath{fill}
\pgfsetbuttcap
\pgfsetmiterjoin
\pgfsetdash{}{0pt}
\definecolor{dialinecolor}{rgb}{0.286275, 0.286275, 0.211765}
\pgfsetstrokecolor{dialinecolor}
\pgfpathmoveto{\pgfpoint{11.252960\du}{9.862909\du}}
\pgfpathlineto{\pgfpoint{11.409761\du}{9.706423\du}}
\pgfpathlineto{\pgfpoint{11.409761\du}{8.600000\du}}
\pgfpathlineto{\pgfpoint{11.252960\du}{8.745151\du}}
\pgfpathlineto{\pgfpoint{11.252960\du}{9.862909\du}}
\pgfusepath{stroke}
\pgfsetlinewidth{0.100000\du}
\pgfsetdash{}{0pt}
\pgfsetdash{}{0pt}
\pgfsetbuttcap
\pgfsetmiterjoin
\pgfsetlinewidth{0.001000\du}
\pgfsetbuttcap
\pgfsetmiterjoin
\pgfsetdash{}{0pt}
\definecolor{dialinecolor}{rgb}{0.717647, 0.717647, 0.615686}
\pgfsetfillcolor{dialinecolor}
\pgfpathmoveto{\pgfpoint{20.628989\du}{8.998912\du}}
\pgfpathlineto{\pgfpoint{20.628989\du}{10.850000\du}}
\pgfpathlineto{\pgfpoint{21.723081\du}{10.850000\du}}
\pgfpathlineto{\pgfpoint{21.723081\du}{8.998912\du}}
\pgfpathlineto{\pgfpoint{20.628989\du}{8.998912\du}}
\pgfusepath{fill}
\pgfsetbuttcap
\pgfsetmiterjoin
\pgfsetdash{}{0pt}
\definecolor{dialinecolor}{rgb}{0.286275, 0.286275, 0.211765}
\pgfsetstrokecolor{dialinecolor}
\pgfpathmoveto{\pgfpoint{20.628989\du}{8.998912\du}}
\pgfpathlineto{\pgfpoint{20.628989\du}{10.850000\du}}
\pgfpathlineto{\pgfpoint{21.723081\du}{10.850000\du}}
\pgfpathlineto{\pgfpoint{21.723081\du}{8.998912\du}}
\pgfpathlineto{\pgfpoint{20.628989\du}{8.998912\du}}
\pgfusepath{stroke}
\pgfsetbuttcap
\pgfsetmiterjoin
\pgfsetdash{}{0pt}
\definecolor{dialinecolor}{rgb}{0.788235, 0.788235, 0.713726}
\pgfsetfillcolor{dialinecolor}
\pgfpathmoveto{\pgfpoint{20.628989\du}{8.998912\du}}
\pgfpathlineto{\pgfpoint{20.777246\du}{8.850000\du}}
\pgfpathlineto{\pgfpoint{21.871011\du}{8.850000\du}}
\pgfpathlineto{\pgfpoint{21.723081\du}{8.998912\du}}
\pgfpathlineto{\pgfpoint{20.628989\du}{8.998912\du}}
\pgfusepath{fill}
\pgfsetbuttcap
\pgfsetmiterjoin
\pgfsetdash{}{0pt}
\definecolor{dialinecolor}{rgb}{0.286275, 0.286275, 0.211765}
\pgfsetstrokecolor{dialinecolor}
\pgfpathmoveto{\pgfpoint{20.628989\du}{8.998912\du}}
\pgfpathlineto{\pgfpoint{20.777246\du}{8.850000\du}}
\pgfpathlineto{\pgfpoint{21.863811\du}{8.850000\du}}
\pgfusepath{stroke}
\pgfsetbuttcap
\pgfsetmiterjoin
\pgfsetdash{}{0pt}
\definecolor{dialinecolor}{rgb}{0.286275, 0.286275, 0.211765}
\pgfsetstrokecolor{dialinecolor}
\pgfpathmoveto{\pgfpoint{21.863811\du}{8.857527\du}}
\pgfpathlineto{\pgfpoint{21.723081\du}{8.998912\du}}
\pgfpathlineto{\pgfpoint{20.628989\du}{8.998912\du}}
\pgfusepath{stroke}
\pgfsetbuttcap
\pgfsetmiterjoin
\pgfsetdash{}{0pt}
\definecolor{dialinecolor}{rgb}{0.788235, 0.788235, 0.713726}
\pgfsetfillcolor{dialinecolor}
\pgfpathmoveto{\pgfpoint{20.696408\du}{9.106586\du}}
\pgfpathlineto{\pgfpoint{21.196163\du}{9.106586\du}}
\pgfpathlineto{\pgfpoint{21.196163\du}{9.349427\du}}
\pgfpathlineto{\pgfpoint{20.696408\du}{9.349427\du}}
\pgfpathlineto{\pgfpoint{20.696408\du}{9.106586\du}}
\pgfusepath{fill}
\pgfsetbuttcap
\pgfsetmiterjoin
\pgfsetdash{}{0pt}
\definecolor{dialinecolor}{rgb}{0.384314, 0.384314, 0.282353}
\pgfsetstrokecolor{dialinecolor}
\pgfpathmoveto{\pgfpoint{20.696408\du}{9.106586\du}}
\pgfpathlineto{\pgfpoint{21.195835\du}{9.106586\du}}
\pgfpathlineto{\pgfpoint{21.195835\du}{9.349100\du}}
\pgfpathlineto{\pgfpoint{20.696408\du}{9.349100\du}}
\pgfpathlineto{\pgfpoint{20.696408\du}{9.106586\du}}
\pgfusepath{stroke}
\pgfsetlinewidth{0.030000\du}
\pgfsetbuttcap
\pgfsetmiterjoin
\pgfsetdash{}{0pt}
\definecolor{dialinecolor}{rgb}{0.925490, 0.925490, 0.905882}
\pgfsetstrokecolor{dialinecolor}
\pgfpathmoveto{\pgfpoint{20.763828\du}{9.228334\du}}
\pgfpathlineto{\pgfpoint{21.114343\du}{9.228334\du}}
\pgfusepath{stroke}
\pgfsetlinewidth{0.001000\du}
\pgfsetbuttcap
\pgfsetmiterjoin
\pgfsetdash{}{0pt}
\definecolor{dialinecolor}{rgb}{0.478431, 0.478431, 0.352941}
\pgfsetfillcolor{dialinecolor}
\pgfpathmoveto{\pgfpoint{21.723081\du}{10.850000\du}}
\pgfpathlineto{\pgfpoint{21.871011\du}{10.700761\du}}
\pgfpathlineto{\pgfpoint{21.871011\du}{8.850000\du}}
\pgfpathlineto{\pgfpoint{21.723081\du}{8.998912\du}}
\pgfpathlineto{\pgfpoint{21.723081\du}{10.850000\du}}
\pgfusepath{fill}
\pgfsetbuttcap
\pgfsetmiterjoin
\pgfsetdash{}{0pt}
\definecolor{dialinecolor}{rgb}{0.286275, 0.286275, 0.211765}
\pgfsetstrokecolor{dialinecolor}
\pgfpathmoveto{\pgfpoint{21.723081\du}{10.850000\du}}
\pgfpathlineto{\pgfpoint{21.863811\du}{10.708288\du}}
\pgfusepath{stroke}
\pgfsetbuttcap
\pgfsetmiterjoin
\pgfsetdash{}{0pt}
\definecolor{dialinecolor}{rgb}{0.286275, 0.286275, 0.211765}
\pgfsetstrokecolor{dialinecolor}
\pgfpathmoveto{\pgfpoint{21.863811\du}{8.857527\du}}
\pgfpathlineto{\pgfpoint{21.723081\du}{8.998912\du}}
\pgfpathlineto{\pgfpoint{21.723081\du}{10.850000\du}}
\pgfusepath{stroke}
\pgfsetlinewidth{0.030000\du}
\pgfsetbuttcap
\pgfsetmiterjoin
\pgfsetdash{}{0pt}
\definecolor{dialinecolor}{rgb}{0.925490, 0.925490, 0.905882}
\pgfsetstrokecolor{dialinecolor}
\pgfpathmoveto{\pgfpoint{20.642734\du}{10.727925\du}}
\pgfpathlineto{\pgfpoint{21.722754\du}{10.727925\du}}
\pgfusepath{stroke}
\pgfsetbuttcap
\pgfsetmiterjoin
\pgfsetdash{}{0pt}
\definecolor{dialinecolor}{rgb}{0.000000, 0.000000, 0.000000}
\pgfsetstrokecolor{dialinecolor}
\pgfpathmoveto{\pgfpoint{20.642734\du}{9.741834\du}}
\pgfpathlineto{\pgfpoint{21.722754\du}{9.741834\du}}
\pgfusepath{stroke}
\pgfsetbuttcap
\pgfsetmiterjoin
\pgfsetdash{}{0pt}
\definecolor{dialinecolor}{rgb}{0.286275, 0.286275, 0.211765}
\pgfsetstrokecolor{dialinecolor}
\pgfpathmoveto{\pgfpoint{20.628989\du}{10.714507\du}}
\pgfpathlineto{\pgfpoint{21.721772\du}{10.714507\du}}
\pgfusepath{stroke}
\pgfsetbuttcap
\pgfsetmiterjoin
\pgfsetdash{}{0pt}
\definecolor{dialinecolor}{rgb}{0.000000, 0.000000, 0.000000}
\pgfsetstrokecolor{dialinecolor}
\pgfpathmoveto{\pgfpoint{20.628989\du}{9.728089\du}}
\pgfpathlineto{\pgfpoint{21.721772\du}{9.728089\du}}
\pgfusepath{stroke}
\pgfsetlinewidth{0.001000\du}
\pgfsetbuttcap
\pgfsetmiterjoin
\pgfsetdash{}{0pt}
\definecolor{dialinecolor}{rgb}{0.925490, 0.925490, 0.905882}
\pgfsetstrokecolor{dialinecolor}
\pgfpathmoveto{\pgfpoint{20.696408\du}{9.336336\du}}
\pgfpathlineto{\pgfpoint{20.696408\du}{9.106586\du}}
\pgfpathlineto{\pgfpoint{21.182417\du}{9.106586\du}}
\pgfusepath{stroke}
\pgfsetlinewidth{0.100000\du}
\pgfsetdash{}{0pt}
\pgfsetdash{}{0pt}
\pgfsetbuttcap
\pgfsetmiterjoin
\pgfsetlinewidth{0.001000\du}
\pgfsetbuttcap
\pgfsetmiterjoin
\pgfsetdash{}{0pt}
\definecolor{dialinecolor}{rgb}{0.788235, 0.788235, 0.713726}
\pgfsetfillcolor{dialinecolor}
\pgfpathmoveto{\pgfpoint{22.752053\du}{13.227519\du}}
\pgfpathlineto{\pgfpoint{22.999199\du}{13.000000\du}}
\pgfpathlineto{\pgfpoint{25.247947\du}{13.000000\du}}
\pgfpathlineto{\pgfpoint{25.000401\du}{13.227519\du}}
\pgfpathlineto{\pgfpoint{22.752053\du}{13.227519\du}}
\pgfusepath{fill}
\pgfsetbuttcap
\pgfsetmiterjoin
\pgfsetdash{}{0pt}
\definecolor{dialinecolor}{rgb}{0.286275, 0.286275, 0.211765}
\pgfsetstrokecolor{dialinecolor}
\pgfpathmoveto{\pgfpoint{22.763669\du}{13.217104\du}}
\pgfpathlineto{\pgfpoint{22.989986\du}{13.008412\du}}
\pgfpathlineto{\pgfpoint{25.239135\du}{13.008412\du}}
\pgfpathlineto{\pgfpoint{25.000401\du}{13.227519\du}}
\pgfpathlineto{\pgfpoint{22.763669\du}{13.227519\du}}
\pgfpathlineto{\pgfpoint{22.763669\du}{13.217104\du}}
\pgfusepath{stroke}
\pgfsetbuttcap
\pgfsetmiterjoin
\pgfsetdash{}{0pt}
\definecolor{dialinecolor}{rgb}{0.717647, 0.717647, 0.615686}
\pgfsetfillcolor{dialinecolor}
\pgfpathmoveto{\pgfpoint{22.752053\du}{13.227519\du}}
\pgfpathlineto{\pgfpoint{25.018426\du}{13.227519\du}}
\pgfpathlineto{\pgfpoint{25.018426\du}{15.000000\du}}
\pgfpathlineto{\pgfpoint{22.752053\du}{15.000000\du}}
\pgfpathlineto{\pgfpoint{22.752053\du}{13.227519\du}}
\pgfusepath{fill}
\pgfsetbuttcap
\pgfsetmiterjoin
\pgfsetdash{}{0pt}
\definecolor{dialinecolor}{rgb}{0.286275, 0.286275, 0.211765}
\pgfsetstrokecolor{dialinecolor}
\pgfpathmoveto{\pgfpoint{22.763669\du}{13.227519\du}}
\pgfpathlineto{\pgfpoint{25.017625\du}{13.227519\du}}
\pgfpathlineto{\pgfpoint{25.017625\du}{14.999199\du}}
\pgfpathlineto{\pgfpoint{22.763669\du}{14.999199\du}}
\pgfpathlineto{\pgfpoint{22.763669\du}{13.227519\du}}
\pgfusepath{stroke}
\pgfsetbuttcap
\pgfsetmiterjoin
\pgfsetdash{}{0pt}
\definecolor{dialinecolor}{rgb}{0.478431, 0.478431, 0.352941}
\pgfsetfillcolor{dialinecolor}
\pgfpathmoveto{\pgfpoint{25.000401\du}{14.982776\du}}
\pgfpathlineto{\pgfpoint{25.247947\du}{14.736832\du}}
\pgfpathlineto{\pgfpoint{25.247947\du}{13.000000\du}}
\pgfpathlineto{\pgfpoint{25.000401\du}{13.227519\du}}
\pgfpathlineto{\pgfpoint{25.000401\du}{14.982776\du}}
\pgfusepath{fill}
\pgfsetbuttcap
\pgfsetmiterjoin
\pgfsetdash{}{0pt}
\definecolor{dialinecolor}{rgb}{0.286275, 0.286275, 0.211765}
\pgfsetstrokecolor{dialinecolor}
\pgfpathmoveto{\pgfpoint{25.000401\du}{14.982776\du}}
\pgfpathlineto{\pgfpoint{25.247947\du}{14.736832\du}}
\pgfpathlineto{\pgfpoint{25.247947\du}{13.008412\du}}
\pgfpathlineto{\pgfpoint{25.239135\du}{13.008412\du}}
\pgfpathlineto{\pgfpoint{25.000401\du}{13.227519\du}}
\pgfpathlineto{\pgfpoint{25.000401\du}{14.982776\du}}
\pgfpathlineto{\pgfpoint{25.000401\du}{14.982776\du}}
\pgfusepath{stroke}
\pgfsetlinewidth{0.050000\du}
\pgfsetdash{}{0pt}
\pgfsetdash{}{0pt}
\pgfsetbuttcap
{
\definecolor{dialinecolor}{rgb}{0.000000, 0.000000, 0.000000}
\pgfsetfillcolor{dialinecolor}
% was here!!!
\pgfsetarrowsend{to}
\definecolor{dialinecolor}{rgb}{0.000000, 0.000000, 0.000000}
\pgfsetstrokecolor{dialinecolor}
\draw (11.755164\du,9.918640\du)--(20.628989\du,9.924456\du);
}
\pgfsetlinewidth{0.100000\du}
\pgfsetdash{}{0pt}
\pgfsetdash{}{0pt}
\pgfsetbuttcap
{
\definecolor{dialinecolor}{rgb}{0.000000, 0.000000, 0.000000}
\pgfsetfillcolor{dialinecolor}
% was here!!!
\pgfsetarrowsend{to}
\definecolor{dialinecolor}{rgb}{0.000000, 0.000000, 0.000000}
\pgfsetstrokecolor{dialinecolor}
\draw (21.870751\du,10.650831\du)--(23.500000\du,12.750000\du);
}
\pgfsetlinewidth{0.050000\du}
\pgfsetdash{}{0pt}
\pgfsetdash{}{0pt}
\pgfsetbuttcap
{
\definecolor{dialinecolor}{rgb}{0.000000, 0.000000, 0.000000}
\pgfsetfillcolor{dialinecolor}
% was here!!!
\pgfsetarrowsend{to}
\definecolor{dialinecolor}{rgb}{0.000000, 0.000000, 0.000000}
\pgfsetstrokecolor{dialinecolor}
\draw (22.757095\du,13.994376\du)--(17.400000\du,14.000000\du);
}
\pgfsetlinewidth{0.050000\du}
\pgfsetdash{}{0pt}
\pgfsetdash{}{0pt}
\pgfsetbuttcap
\pgfsetmiterjoin
\pgfsetlinewidth{0.050000\du}
\pgfsetbuttcap
\pgfsetmiterjoin
\pgfsetdash{}{0pt}
\definecolor{dialinecolor}{rgb}{1.000000, 1.000000, 1.000000}
\pgfsetfillcolor{dialinecolor}
\fill (14.682258\du,12.900000\du)--(14.682258\du,15.605000\du)--(17.300000\du,15.605000\du)--(17.300000\du,12.900000\du)--cycle;
\definecolor{dialinecolor}{rgb}{0.000000, 0.000000, 0.000000}
\pgfsetstrokecolor{dialinecolor}
\draw (14.682258\du,12.900000\du)--(14.682258\du,15.605000\du)--(17.300000\du,15.605000\du)--(17.300000\du,12.900000\du)--cycle;
\pgfsetbuttcap
\pgfsetmiterjoin
\pgfsetdash{}{0pt}
\definecolor{dialinecolor}{rgb}{0.000000, 0.000000, 0.000000}
\pgfsetstrokecolor{dialinecolor}
\draw (14.682258\du,12.900000\du)--(14.682258\du,15.605000\du)--(17.300000\du,15.605000\du)--(17.300000\du,12.900000\du)--cycle;
% setfont left to latex
\definecolor{dialinecolor}{rgb}{0.000000, 0.000000, 0.000000}
\pgfsetstrokecolor{dialinecolor}
\node[anchor=west] at (15.239919\du,14.659583\du){html};
\pgfsetlinewidth{0.050000\du}
\pgfsetdash{}{0pt}
\pgfsetdash{}{0pt}
\pgfsetbuttcap
{
\definecolor{dialinecolor}{rgb}{0.000000, 0.000000, 0.000000}
\pgfsetfillcolor{dialinecolor}
% was here!!!
\pgfsetarrowsend{to}
\definecolor{dialinecolor}{rgb}{0.000000, 0.000000, 0.000000}
\pgfsetstrokecolor{dialinecolor}
\draw (14.682258\du,14.252500\du)--(10.500000\du,10.750000\du);
}
% setfont left to latex
\definecolor{dialinecolor}{rgb}{0.000000, 0.000000, 0.000000}
\pgfsetstrokecolor{dialinecolor}
\node[anchor=west] at (8.950000\du,14.350000\du){Navigateur};
% setfont left to latex
\definecolor{dialinecolor}{rgb}{0.000000, 0.000000, 0.000000}
\pgfsetstrokecolor{dialinecolor}
\node[anchor=west] at (10.800000\du,7.600000\du){Client};
% setfont left to latex
\definecolor{dialinecolor}{rgb}{0.000000, 0.000000, 0.000000}
\pgfsetstrokecolor{dialinecolor}
\node[anchor=west] at (20.250000\du,7.800000\du){Serveur};
% setfont left to latex
\definecolor{dialinecolor}{rgb}{0.000000, 0.000000, 0.000000}
\pgfsetstrokecolor{dialinecolor}
\node[anchor=west] at (22.950000\du,15.500000\du){Apache};
% setfont left to latex
\definecolor{dialinecolor}{rgb}{0.000000, 0.000000, 0.000000}
\pgfsetstrokecolor{dialinecolor}
\node[anchor=west] at (19.400000\du,15.200000\du){PHP};
\end{tikzpicture}
	
		\subsection{Étapes de traductions}
			Soit Ta un TAD et Tc le type concret correspondant. \\
			\paragraph{Étape 1(Concepteur)}
				Définition de Tc un entête de sous programme pour chaque opérations de Ta.	
			\paragraph{Étape 2 (Programmeur)}
				Définir une représentation mémoire d'une valeur de Tc (Tableau, enregistrement, pointeur...)
			\paragraph{Étape 3 (Programmeur)}
				Codes les corps des sous programmes dans Tc (connaissant la représentation mémoire)
			\paragraph{}
				Les points 2 et 3 seront développés au chapitre suivant (implémentation d'un TAD)	
		\subsection{En tête des sous programmes}
			Dans Ta, opération = fonctions (au sens mathématique)\\
			Dans Tc, opération = procédure ou fonction \\
			Dans Tc, on distingue :
			\begin{itemize}
				\item Les opérations de construction
				\item Les opérations de modification 
				\item Les opérations d'évaluation
			\end{itemize}
			\subsubsection{Opération de construction}
				\paragraph{Rôle}
					Construire une valeur du type étudié, éventuellement à partir de valeur d'autres types.
					Opération appelée aussi \textbf{constructeur} 
				\paragraph{Caractéristique}
					Le type Ta étudié n'apparait que dans le domaine de sortie de l'opération.
				\paragraph{Règle}
					Une opération de construction dans Ta se code par une fonction dans Tc(en général)	
				\paragraph{Exemple 1}
					Dans Ta:
					unPoint Réel $\times$ Réel $\times$ Couleur $\times$ Réel $\rightarrow$ Point \\
					Dans Tc: 	
					\lstinputlisting[caption=Tc]{2.algo}
				\paragraph{Exemple 2}
					Dans Ta: pointOrigine $\rightarrow$ Point\\
					Dans Tc
					\lstinputlisting[caption=Tc]{2-1.algo}
					\subparagraph{}
						Le non respect d'une pré condition se traduit par une levée d'exception (cf unPoint)
					\subparagraph{}
						Un constructeur sans paramètre d'entrée est appelé constante du type abstrait
						pour un client 
					\lstinputlisting{2-2.algo}
				\subsection{Opération de consultation}
					\paragraph{Rôle}
						Fournit une caractéristique d'une valeur du type opération aussi appelé observateur
						.
					\paragraph{Caractéristique}
						Le type Ta étudié n'apparait que dans le domaine d'entrée de l'opération. \\
					 \paragraph{Règles}
						Une opération de consultation dans Ta se traduit toujours par une fonction dans TC
					\paragraph{Exemple}
						Dans Ta \\
						taille : Point $\rightarrow$ Réel	\\ \\
						Dans Tc\\
					\lstinputlisting{2-3.algo}
				\subsection{Opération de modification}
					\paragraph{Rôle}
						Modifier une caractéristique d'une valeur d'un type.
					\paragraph{Caractéristique}
						Le type Ta apparait à la fin dans le domaine d'entrée et le domaine de sortie de l'opérateur. 
					\paragraph{Règle}
						Une opération de modification dans Ta se code par une procédure dans Tc. 
						(avec le mode mise à jour pour la valeur à modifier!)
					\paragraph{Exemple}
						Dans Ta: modifierTaille: Point $\times$ Réel $\rightarrow$ Point \\
						Dans Tc
						\lstinputlisting{2-4.algo}
				\subsection{Les opérations d'évaluation}
					\paragraph{Rôle}
						Construit une nouvelle valeur du type abstrait à partir de la valeur existante du
						type. Opération avec ainsi similaire aux constructeur.
					\paragraph{Caractéristique}
						Le type Ta apparaît à la fin dans le domaine d'entrée et de sortie de l'opération.
						(comme une opération de modification).
					\paragraph{Règle}
						Une opération d'évaluation se traduit dans Ta par une fonction dans Tc.
					\paragraph{Exemple}
						Dans Ta: Point $\times$ Réel $\rightarrow$ Point\\
						Dans Tc
						\lstinputlisting{2-5.algo}
				\subsection{Spécification d'un type concret}
					\paragraph{Définition}
						Regroupement des en-têtes des sous programme de la spécification (cf 3.2), avec
						en commentaire les propriétés du type (cf 2). \\
						Cette spécification est aussi appelée spécification algorithmique du type abstrait
						. (cf moodle pour Point et Tableau[T])
\section{Utilisation d'un TAD}
		Un client d'un Type Abstrait de Données peut:
			\begin{itemize}	
				\item Définir des variables de type T
				\item Définir des paramètres de sous-programme de type T
				\item Définir de nouveaux type en utilisant le type T
				\item Appeler des sous-programme définis par le type T
			\end{itemize}
			\paragraph{Remarque}
				Le client n'a pas accès à la représentation mémoire et au codage des sous-programme définis
				dans T!
			\subsubsection{Exemple}
				Soit à calculer le point milieu d'un segment. \\
				En tant que client:	
				\lstinputlisting{2-6.algo}
%				\relax 
\@setckpt{annexes/tachesRedmine}{
\setcounter{page}{2}
\setcounter{equation}{0}
\setcounter{enumi}{0}
\setcounter{enumii}{0}
\setcounter{enumiii}{0}
\setcounter{enumiv}{0}
\setcounter{footnote}{0}
\setcounter{mpfootnote}{0}
\setcounter{part}{0}
\setcounter{chapter}{0}
\setcounter{section}{0}
\setcounter{subsection}{0}
\setcounter{subsubsection}{0}
\setcounter{paragraph}{0}
\setcounter{subparagraph}{0}
\setcounter{figure}{0}
\setcounter{table}{0}
\setcounter{Item}{0}
\setcounter{Hfootnote}{2}
\setcounter{lstnumber}{1}
\setcounter{float@type}{8}
\setcounter{su@anzahl}{0}
\setcounter{DTLrowi}{0}
\setcounter{DTLrowii}{0}
\setcounter{DTLrowiii}{0}
\setcounter{DTLrow}{0}
\setcounter{section@level}{0}
\setcounter{lstlisting}{0}
}

				\lstinputlisting{2-7.algo}
				où abscisse, ordonnée et unPoint sont des opérations du type Point (cf Moodle)					
\section{Processus d'élaboration d'un TAD}
	\subsection{Étape}
		 \begin{enumerate}
			\item Énumérer l'ensemble des opérations du type
			\item Pour chaque opération, préciser son profil (nom de l'opération, domaine d'entrée et 
				domaine de sortie)
			\item lister l'ensemble des propriétés du type
			\item Définir un entête de sous-programme pour chacune des opérations du type
			\item Choisir une représentation mémoire pour coder les opérations et les valeurs du type
			\item Coder avec la représentation mémoire choisie les corps des différents sous-programmes
		\end{enumerate} 	
		\paragraph{Remarque}
			\subparagraph{} Les étapes 1 à 3 sont relatives au type abstrait. (formalisme)	
			\subparagraph{} Les étapes 4 à 6 concernent les types concrets (programmation)
			\subparagraph{} L'étape 3 permet:
				\begin{itemize} 
					\item De donner la sémantique des opérations du type
					\item D'aider au codage des opérations du type concret
					\item De définir des jeux de tests pour ces opérations
				\end{itemize}

	
	\chapter{L'implémentation d'un Type Abstrait de Données}
		\section{Implémentation d'un type concret}
	\paragraph{Définition}
		Mise en œuvre informatique de la spécification algorithmique du type concret.\\
		Tâche du ressort du programmeur du type.\\
		L'implémentation doit respecter la spécification (spécification = cahier des charges pour le
			programmeur)

	\subsection{Tâche du programmeur}
		\begin{enumerate}
			\item Choisir une représentation mémoire pour coder les opérations du type
			\item Coder les corps des sous-programmes conformément à la spécification.
			\item Regrouper au sein d'un module la représentation mémoire et le codage des opérations. 
		\end{enumerate}

	\subsection{Conteneur d'un module d'implémentation}
		Un module peut contenir: 
		\begin{itemize}
			\item des déclarations de constantes
			\item des déclarations de types dont l'un au moins correspond à la définition du type étudié.
			\item les corps des sous-programmes définies par la spécification.
			\item tout sous-programmes nécessaire à la mise en œuvre du type. 
		\end{itemize}

	\subsection{Exemple}
		\subsubsection{1}
		Pour le TAD Point en représentation statique.
		\lstinputlisting{4-1.algo}

	\subsection{Implémentation du TAD Point en représentation dynamique}
		En représentation statique: \\
%			\relax 
\@setckpt{annexes/tachesRedmine}{
\setcounter{page}{2}
\setcounter{equation}{0}
\setcounter{enumi}{0}
\setcounter{enumii}{0}
\setcounter{enumiii}{0}
\setcounter{enumiv}{0}
\setcounter{footnote}{0}
\setcounter{mpfootnote}{0}
\setcounter{part}{0}
\setcounter{chapter}{0}
\setcounter{section}{0}
\setcounter{subsection}{0}
\setcounter{subsubsection}{0}
\setcounter{paragraph}{0}
\setcounter{subparagraph}{0}
\setcounter{figure}{0}
\setcounter{table}{0}
\setcounter{Item}{0}
\setcounter{Hfootnote}{2}
\setcounter{lstnumber}{1}
\setcounter{float@type}{8}
\setcounter{su@anzahl}{0}
\setcounter{DTLrowi}{0}
\setcounter{DTLrowii}{0}
\setcounter{DTLrowiii}{0}
\setcounter{DTLrow}{0}
\setcounter{section@level}{0}
\setcounter{lstlisting}{0}
}

		En représentation dynamique \\
%			\relax 
\@setckpt{annexes/tachesRedmine}{
\setcounter{page}{2}
\setcounter{equation}{0}
\setcounter{enumi}{0}
\setcounter{enumii}{0}
\setcounter{enumiii}{0}
\setcounter{enumiv}{0}
\setcounter{footnote}{0}
\setcounter{mpfootnote}{0}
\setcounter{part}{0}
\setcounter{chapter}{0}
\setcounter{section}{0}
\setcounter{subsection}{0}
\setcounter{subsubsection}{0}
\setcounter{paragraph}{0}
\setcounter{subparagraph}{0}
\setcounter{figure}{0}
\setcounter{table}{0}
\setcounter{Item}{0}
\setcounter{Hfootnote}{2}
\setcounter{lstnumber}{1}
\setcounter{float@type}{8}
\setcounter{su@anzahl}{0}
\setcounter{DTLrowi}{0}
\setcounter{DTLrowii}{0}
\setcounter{DTLrowiii}{0}
\setcounter{DTLrow}{0}
\setcounter{section@level}{0}
\setcounter{lstlisting}{0}
}

		(Un pointeur vers un enregistrement)
		Dans cette représentation dynamique		
		\lstinputlisting{4-2.algo}


\section{Sémantique de valeur et sémantique de référence}
	Dans le type concret, deux nouvelles opération s'ajoutent: l'affectation(<-) et l'égalité(=).
	\paragraph{Problème} Quelle signification (sémantique) donner à ces opérations pour un client? 
	\paragraph{Exemple} Pour un client
		\lstinputlisting{4-3.algo}

	\subsection{Sémantique de valeurs}
		\paragraph{Définition}
			Une affectation x<-y à une sémantique de valeurs si le conteneur (la valeur) de y est 
				recopiée dans x.
		\paragraph{Remarque}
			Avec une sémantique de valeur, toute opération de modification sur y après l'affectation y<-y
			n'affecte pas la valeur de x.
		\subsubsection{Exemple}
			Soit le TAD Point en représentation statique. (c'est-à-dire un enregistrement)\\
			Cette représentation statique à une sémantique de valeur pour le TAD Point
			Car la copie (<-) et la comparaison (= et /=) de deux enregistrements travaillant sur les
			valeur est un enregistrement.
			de champs d'un enregistrement. \\
			% Graphic for TeX using PGF
% Title: /usr/home/satenske/Diagram1.dia
% Creator: Dia v0.97.1
% CreationDate: Wed Feb 16 09:47:41 2011
% For: satenske
% \usepackage{tikz}
% The following commands are not supported in PSTricks at present
% We define them conditionally, so when they are implemented,
% this pgf file will use them.
\ifx\du\undefined
  \newlength{\du}
\fi
\setlength{\du}{15\unitlength}
\begin{tikzpicture}
\pgftransformxscale{1.000000}
\pgftransformyscale{-1.000000}
\definecolor{dialinecolor}{rgb}{0.000000, 0.000000, 0.000000}
\pgfsetstrokecolor{dialinecolor}
\definecolor{dialinecolor}{rgb}{1.000000, 1.000000, 1.000000}
\pgfsetfillcolor{dialinecolor}
\definecolor{dialinecolor}{rgb}{1.000000, 1.000000, 1.000000}
\pgfsetfillcolor{dialinecolor}
\fill (4.650000\du,9.100000\du)--(4.650000\du,13.300000\du)--(18.300000\du,13.300000\du)--(18.300000\du,9.100000\du)--cycle;
\pgfsetlinewidth{0.100000\du}
\pgfsetdash{}{0pt}
\pgfsetdash{}{0pt}
\pgfsetmiterjoin
\definecolor{dialinecolor}{rgb}{0.000000, 0.000000, 0.000000}
\pgfsetstrokecolor{dialinecolor}
\draw (4.650000\du,9.100000\du)--(4.650000\du,13.300000\du)--(18.300000\du,13.300000\du)--(18.300000\du,9.100000\du)--cycle;
% setfont left to latex
\definecolor{dialinecolor}{rgb}{0.000000, 0.000000, 0.000000}
\pgfsetstrokecolor{dialinecolor}
\node at (11.475000\du,11.395000\du){};
\definecolor{dialinecolor}{rgb}{1.000000, 1.000000, 1.000000}
\pgfsetfillcolor{dialinecolor}
\fill (5.791250\du,10.200000\du)--(5.791250\du,12.100000\du)--(7.908750\du,12.100000\du)--(7.908750\du,10.200000\du)--cycle;
\pgfsetlinewidth{0.100000\du}
\pgfsetdash{}{0pt}
\pgfsetdash{}{0pt}
\pgfsetmiterjoin
\definecolor{dialinecolor}{rgb}{0.000000, 0.000000, 0.000000}
\pgfsetstrokecolor{dialinecolor}
\draw (5.791250\du,10.200000\du)--(5.791250\du,12.100000\du)--(7.908750\du,12.100000\du)--(7.908750\du,10.200000\du)--cycle;
% setfont left to latex
\definecolor{dialinecolor}{rgb}{0.000000, 0.000000, 0.000000}
\pgfsetstrokecolor{dialinecolor}
\node at (6.850000\du,11.345000\du){0.0};
\definecolor{dialinecolor}{rgb}{1.000000, 1.000000, 1.000000}
\pgfsetfillcolor{dialinecolor}
\fill (9.000000\du,10.155000\du)--(9.000000\du,12.055000\du)--(11.117500\du,12.055000\du)--(11.117500\du,10.155000\du)--cycle;
\pgfsetlinewidth{0.100000\du}
\pgfsetdash{}{0pt}
\pgfsetdash{}{0pt}
\pgfsetmiterjoin
\definecolor{dialinecolor}{rgb}{0.000000, 0.000000, 0.000000}
\pgfsetstrokecolor{dialinecolor}
\draw (9.000000\du,10.155000\du)--(9.000000\du,12.055000\du)--(11.117500\du,12.055000\du)--(11.117500\du,10.155000\du)--cycle;
% setfont left to latex
\definecolor{dialinecolor}{rgb}{0.000000, 0.000000, 0.000000}
\pgfsetstrokecolor{dialinecolor}
\node at (10.058750\du,11.300000\du){1.0};
\definecolor{dialinecolor}{rgb}{1.000000, 1.000000, 1.000000}
\pgfsetfillcolor{dialinecolor}
\fill (11.890000\du,10.110000\du)--(11.890000\du,12.010000\du)--(14.227500\du,12.010000\du)--(14.227500\du,10.110000\du)--cycle;
\pgfsetlinewidth{0.100000\du}
\pgfsetdash{}{0pt}
\pgfsetdash{}{0pt}
\pgfsetmiterjoin
\definecolor{dialinecolor}{rgb}{0.000000, 0.000000, 0.000000}
\pgfsetstrokecolor{dialinecolor}
\draw (11.890000\du,10.110000\du)--(11.890000\du,12.010000\du)--(14.227500\du,12.010000\du)--(14.227500\du,10.110000\du)--cycle;
% setfont left to latex
\definecolor{dialinecolor}{rgb}{0.000000, 0.000000, 0.000000}
\pgfsetstrokecolor{dialinecolor}
\node at (13.058750\du,11.255000\du){noir};
\definecolor{dialinecolor}{rgb}{1.000000, 1.000000, 1.000000}
\pgfsetfillcolor{dialinecolor}
\fill (15.500000\du,10.165000\du)--(15.500000\du,12.065000\du)--(17.617500\du,12.065000\du)--(17.617500\du,10.165000\du)--cycle;
\pgfsetlinewidth{0.100000\du}
\pgfsetdash{}{0pt}
\pgfsetdash{}{0pt}
\pgfsetmiterjoin
\definecolor{dialinecolor}{rgb}{0.000000, 0.000000, 0.000000}
\pgfsetstrokecolor{dialinecolor}
\draw (15.500000\du,10.165000\du)--(15.500000\du,12.065000\du)--(17.617500\du,12.065000\du)--(17.617500\du,10.165000\du)--cycle;
% setfont left to latex
\definecolor{dialinecolor}{rgb}{0.000000, 0.000000, 0.000000}
\pgfsetstrokecolor{dialinecolor}
\node at (16.558750\du,11.310000\du){3.0};
% setfont left to latex
\definecolor{dialinecolor}{rgb}{0.000000, 0.000000, 0.000000}
\pgfsetstrokecolor{dialinecolor}
\node[anchor=west] at (3.000000\du,10.850000\du){P1};
\end{tikzpicture}
\\
			% Graphic for TeX using PGF
% Title: /usr/home/satenske/Diagram1.dia
% Creator: Dia v0.97.1
% CreationDate: Wed Feb 16 09:48:49 2011
% For: satenske
% \usepackage{tikz}
% The following commands are not supported in PSTricks at present
% We define them conditionally, so when they are implemented,
% this pgf file will use them.
\ifx\du\undefined
  \newlength{\du}
\fi
\setlength{\du}{15\unitlength}
\begin{tikzpicture}
\pgftransformxscale{1.000000}
\pgftransformyscale{-1.000000}
\definecolor{dialinecolor}{rgb}{0.000000, 0.000000, 0.000000}
\pgfsetstrokecolor{dialinecolor}
\definecolor{dialinecolor}{rgb}{1.000000, 1.000000, 1.000000}
\pgfsetfillcolor{dialinecolor}
\definecolor{dialinecolor}{rgb}{1.000000, 1.000000, 1.000000}
\pgfsetfillcolor{dialinecolor}
\fill (4.650000\du,9.100000\du)--(4.650000\du,13.300000\du)--(18.300000\du,13.300000\du)--(18.300000\du,9.100000\du)--cycle;
\pgfsetlinewidth{0.100000\du}
\pgfsetdash{}{0pt}
\pgfsetdash{}{0pt}
\pgfsetmiterjoin
\definecolor{dialinecolor}{rgb}{0.000000, 0.000000, 0.000000}
\pgfsetstrokecolor{dialinecolor}
\draw (4.650000\du,9.100000\du)--(4.650000\du,13.300000\du)--(18.300000\du,13.300000\du)--(18.300000\du,9.100000\du)--cycle;
% setfont left to latex
\definecolor{dialinecolor}{rgb}{0.000000, 0.000000, 0.000000}
\pgfsetstrokecolor{dialinecolor}
\node at (11.475000\du,11.395000\du){};
\definecolor{dialinecolor}{rgb}{1.000000, 1.000000, 1.000000}
\pgfsetfillcolor{dialinecolor}
\fill (5.791250\du,10.200000\du)--(5.791250\du,12.100000\du)--(7.908750\du,12.100000\du)--(7.908750\du,10.200000\du)--cycle;
\pgfsetlinewidth{0.100000\du}
\pgfsetdash{}{0pt}
\pgfsetdash{}{0pt}
\pgfsetmiterjoin
\definecolor{dialinecolor}{rgb}{0.000000, 0.000000, 0.000000}
\pgfsetstrokecolor{dialinecolor}
\draw (5.791250\du,10.200000\du)--(5.791250\du,12.100000\du)--(7.908750\du,12.100000\du)--(7.908750\du,10.200000\du)--cycle;
% setfont left to latex
\definecolor{dialinecolor}{rgb}{0.000000, 0.000000, 0.000000}
\pgfsetstrokecolor{dialinecolor}
\node at (6.850000\du,11.345000\du){1.5};
\definecolor{dialinecolor}{rgb}{1.000000, 1.000000, 1.000000}
\pgfsetfillcolor{dialinecolor}
\fill (9.000000\du,10.155000\du)--(9.000000\du,12.055000\du)--(11.117500\du,12.055000\du)--(11.117500\du,10.155000\du)--cycle;
\pgfsetlinewidth{0.100000\du}
\pgfsetdash{}{0pt}
\pgfsetdash{}{0pt}
\pgfsetmiterjoin
\definecolor{dialinecolor}{rgb}{0.000000, 0.000000, 0.000000}
\pgfsetstrokecolor{dialinecolor}
\draw (9.000000\du,10.155000\du)--(9.000000\du,12.055000\du)--(11.117500\du,12.055000\du)--(11.117500\du,10.155000\du)--cycle;
% setfont left to latex
\definecolor{dialinecolor}{rgb}{0.000000, 0.000000, 0.000000}
\pgfsetstrokecolor{dialinecolor}
\node at (10.058750\du,11.300000\du){3.5};
\definecolor{dialinecolor}{rgb}{1.000000, 1.000000, 1.000000}
\pgfsetfillcolor{dialinecolor}
\fill (11.502500\du,10.110000\du)--(11.502500\du,12.010000\du)--(14.615000\du,12.010000\du)--(14.615000\du,10.110000\du)--cycle;
\pgfsetlinewidth{0.100000\du}
\pgfsetdash{}{0pt}
\pgfsetdash{}{0pt}
\pgfsetmiterjoin
\definecolor{dialinecolor}{rgb}{0.000000, 0.000000, 0.000000}
\pgfsetstrokecolor{dialinecolor}
\draw (11.502500\du,10.110000\du)--(11.502500\du,12.010000\du)--(14.615000\du,12.010000\du)--(14.615000\du,10.110000\du)--cycle;
% setfont left to latex
\definecolor{dialinecolor}{rgb}{0.000000, 0.000000, 0.000000}
\pgfsetstrokecolor{dialinecolor}
\node at (13.058750\du,11.255000\du){vert};
\definecolor{dialinecolor}{rgb}{1.000000, 1.000000, 1.000000}
\pgfsetfillcolor{dialinecolor}
\fill (15.500000\du,10.165000\du)--(15.500000\du,12.065000\du)--(17.617500\du,12.065000\du)--(17.617500\du,10.165000\du)--cycle;
\pgfsetlinewidth{0.100000\du}
\pgfsetdash{}{0pt}
\pgfsetdash{}{0pt}
\pgfsetmiterjoin
\definecolor{dialinecolor}{rgb}{0.000000, 0.000000, 0.000000}
\pgfsetstrokecolor{dialinecolor}
\draw (15.500000\du,10.165000\du)--(15.500000\du,12.065000\du)--(17.617500\du,12.065000\du)--(17.617500\du,10.165000\du)--cycle;
% setfont left to latex
\definecolor{dialinecolor}{rgb}{0.000000, 0.000000, 0.000000}
\pgfsetstrokecolor{dialinecolor}
\node at (16.558750\du,11.310000\du){1.0};
% setfont left to latex
\definecolor{dialinecolor}{rgb}{0.000000, 0.000000, 0.000000}
\pgfsetstrokecolor{dialinecolor}
\node[anchor=west] at (3.000000\du,10.850000\du){P2};
% setfont left to latex
\definecolor{dialinecolor}{rgb}{0.000000, 0.000000, 0.000000}
\pgfsetstrokecolor{dialinecolor}
\node[anchor=west] at (3.650000\du,10.500000\du){};
\end{tikzpicture}

			\lstinputlisting{4-4-1.algo}
			\lstinputlisting{4-4.algo}
			On obtient l'affichage: \\
			3.5\\1.0

	\subsection{Sémantique de référence}
		\paragraph{Définition}
			L'affectation x <- y désigne le même contenu mémoire pouvant être référencé a la fois par x et 
			par y.
		\paragraph{Remarque}
			Après l'affectation x <- y avec un sémantique de référence toute modification sur y se 
			répercute sur x! \\
			On dit que x et y sont des alias pour désigner le même contenu.	

		\subsubsection{Exemple}
			Soit le TAD Point en représentation dynamique (c'est-à-dire par un pointeur vers un
			enregistrement).\\
			$\Rightarrow$ Sémantique de référence pour le TAD Point\\
			(Car affectation et comparaison de deux pointeurs!)\\
			% Graphic for TeX using PGF
% Title: /usr/home/satenske/cours/AP/tad2/cours/Diagram4.dia
% Creator: Dia v0.97.1
% CreationDate: Wed Feb 16 10:05:29 2011
% For: satenske
% \usepackage{tikz}
% The following commands are not supported in PSTricks at present
% We define them conditionally, so when they are implemented,
% this pgf file will use them.
\ifx\du\undefined
  \newlength{\du}
\fi
\setlength{\du}{15\unitlength}
\begin{tikzpicture}
\pgftransformxscale{1.000000}
\pgftransformyscale{-1.000000}
\definecolor{dialinecolor}{rgb}{0.000000, 0.000000, 0.000000}
\pgfsetstrokecolor{dialinecolor}
\definecolor{dialinecolor}{rgb}{1.000000, 1.000000, 1.000000}
\pgfsetfillcolor{dialinecolor}
\definecolor{dialinecolor}{rgb}{1.000000, 1.000000, 1.000000}
\pgfsetfillcolor{dialinecolor}
\fill (4.650000\du,9.100000\du)--(4.650000\du,13.300000\du)--(18.300000\du,13.300000\du)--(18.300000\du,9.100000\du)--cycle;
\pgfsetlinewidth{0.050000\du}
\pgfsetdash{}{0pt}
\pgfsetdash{}{0pt}
\pgfsetmiterjoin
\definecolor{dialinecolor}{rgb}{0.000000, 0.000000, 0.000000}
\pgfsetstrokecolor{dialinecolor}
\draw (4.650000\du,9.100000\du)--(4.650000\du,13.300000\du)--(18.300000\du,13.300000\du)--(18.300000\du,9.100000\du)--cycle;
% setfont left to latex
\definecolor{dialinecolor}{rgb}{0.000000, 0.000000, 0.000000}
\pgfsetstrokecolor{dialinecolor}
\node at (11.475000\du,11.395000\du){};
\definecolor{dialinecolor}{rgb}{1.000000, 1.000000, 1.000000}
\pgfsetfillcolor{dialinecolor}
\fill (5.791250\du,10.200000\du)--(5.791250\du,12.100000\du)--(7.908750\du,12.100000\du)--(7.908750\du,10.200000\du)--cycle;
\pgfsetlinewidth{0.050000\du}
\pgfsetdash{}{0pt}
\pgfsetdash{}{0pt}
\pgfsetmiterjoin
\definecolor{dialinecolor}{rgb}{0.000000, 0.000000, 0.000000}
\pgfsetstrokecolor{dialinecolor}
\draw (5.791250\du,10.200000\du)--(5.791250\du,12.100000\du)--(7.908750\du,12.100000\du)--(7.908750\du,10.200000\du)--cycle;
% setfont left to latex
\definecolor{dialinecolor}{rgb}{0.000000, 0.000000, 0.000000}
\pgfsetstrokecolor{dialinecolor}
\node at (6.850000\du,11.345000\du){1.5};
\definecolor{dialinecolor}{rgb}{1.000000, 1.000000, 1.000000}
\pgfsetfillcolor{dialinecolor}
\fill (9.000000\du,10.155000\du)--(9.000000\du,12.055000\du)--(11.117500\du,12.055000\du)--(11.117500\du,10.155000\du)--cycle;
\pgfsetlinewidth{0.050000\du}
\pgfsetdash{}{0pt}
\pgfsetdash{}{0pt}
\pgfsetmiterjoin
\definecolor{dialinecolor}{rgb}{0.000000, 0.000000, 0.000000}
\pgfsetstrokecolor{dialinecolor}
\draw (9.000000\du,10.155000\du)--(9.000000\du,12.055000\du)--(11.117500\du,12.055000\du)--(11.117500\du,10.155000\du)--cycle;
% setfont left to latex
\definecolor{dialinecolor}{rgb}{0.000000, 0.000000, 0.000000}
\pgfsetstrokecolor{dialinecolor}
\node at (10.058750\du,11.300000\du){3.5};
\definecolor{dialinecolor}{rgb}{1.000000, 1.000000, 1.000000}
\pgfsetfillcolor{dialinecolor}
\fill (11.502500\du,10.110000\du)--(11.502500\du,12.010000\du)--(14.615000\du,12.010000\du)--(14.615000\du,10.110000\du)--cycle;
\pgfsetlinewidth{0.050000\du}
\pgfsetdash{}{0pt}
\pgfsetdash{}{0pt}
\pgfsetmiterjoin
\definecolor{dialinecolor}{rgb}{0.000000, 0.000000, 0.000000}
\pgfsetstrokecolor{dialinecolor}
\draw (11.502500\du,10.110000\du)--(11.502500\du,12.010000\du)--(14.615000\du,12.010000\du)--(14.615000\du,10.110000\du)--cycle;
% setfont left to latex
\definecolor{dialinecolor}{rgb}{0.000000, 0.000000, 0.000000}
\pgfsetstrokecolor{dialinecolor}
\node at (13.058750\du,11.255000\du){vert};
\definecolor{dialinecolor}{rgb}{1.000000, 1.000000, 1.000000}
\pgfsetfillcolor{dialinecolor}
\fill (15.500000\du,10.165000\du)--(15.500000\du,12.065000\du)--(17.617500\du,12.065000\du)--(17.617500\du,10.165000\du)--cycle;
\pgfsetlinewidth{0.050000\du}
\pgfsetdash{}{0pt}
\pgfsetdash{}{0pt}
\pgfsetmiterjoin
\definecolor{dialinecolor}{rgb}{0.000000, 0.000000, 0.000000}
\pgfsetstrokecolor{dialinecolor}
\draw (15.500000\du,10.165000\du)--(15.500000\du,12.065000\du)--(17.617500\du,12.065000\du)--(17.617500\du,10.165000\du)--cycle;
% setfont left to latex
\definecolor{dialinecolor}{rgb}{0.000000, 0.000000, 0.000000}
\pgfsetstrokecolor{dialinecolor}
\node at (16.558750\du,11.310000\du){1.0};
% setfont left to latex
\definecolor{dialinecolor}{rgb}{0.000000, 0.000000, 0.000000}
\pgfsetstrokecolor{dialinecolor}
\node[anchor=west] at (-0.350000\du,13.550000\du){P2};
% setfont left to latex
\definecolor{dialinecolor}{rgb}{0.000000, 0.000000, 0.000000}
\pgfsetstrokecolor{dialinecolor}
\node[anchor=west] at (3.650000\du,10.500000\du){};
\definecolor{dialinecolor}{rgb}{1.000000, 1.000000, 1.000000}
\pgfsetfillcolor{dialinecolor}
\fill (-0.550000\du,10.000000\du)--(-0.550000\du,11.900000\du)--(1.450000\du,11.900000\du)--(1.450000\du,10.000000\du)--cycle;
\pgfsetlinewidth{0.050000\du}
\pgfsetdash{}{0pt}
\pgfsetdash{}{0pt}
\pgfsetmiterjoin
\definecolor{dialinecolor}{rgb}{0.000000, 0.000000, 0.000000}
\pgfsetstrokecolor{dialinecolor}
\draw (-0.550000\du,10.000000\du)--(-0.550000\du,11.900000\du)--(1.450000\du,11.900000\du)--(1.450000\du,10.000000\du)--cycle;
% setfont left to latex
\definecolor{dialinecolor}{rgb}{0.000000, 0.000000, 0.000000}
\pgfsetstrokecolor{dialinecolor}
\node at (0.450000\du,11.145000\du){};
\pgfsetlinewidth{0.050000\du}
\pgfsetdash{}{0pt}
\pgfsetdash{}{0pt}
\pgfsetmiterjoin
\pgfsetbuttcap
{
\definecolor{dialinecolor}{rgb}{0.000000, 0.000000, 0.000000}
\pgfsetfillcolor{dialinecolor}
% was here!!!
\pgfsetarrowsend{to}
{\pgfsetcornersarced{\pgfpoint{0.000000\du}{0.000000\du}}\definecolor{dialinecolor}{rgb}{0.000000, 0.000000, 0.000000}
\pgfsetstrokecolor{dialinecolor}
\draw (1.475250\du,10.950000\du)--(3.062625\du,10.950000\du)--(3.062625\du,11.200000\du)--(4.650000\du,11.200000\du);
}}
\definecolor{dialinecolor}{rgb}{1.000000, 1.000000, 1.000000}
\pgfsetfillcolor{dialinecolor}
\fill (4.275000\du,-0.932500\du)--(4.275000\du,3.267500\du)--(17.925000\du,3.267500\du)--(17.925000\du,-0.932500\du)--cycle;
\pgfsetlinewidth{0.050000\du}
\pgfsetdash{}{0pt}
\pgfsetdash{}{0pt}
\pgfsetmiterjoin
\definecolor{dialinecolor}{rgb}{0.000000, 0.000000, 0.000000}
\pgfsetstrokecolor{dialinecolor}
\draw (4.275000\du,-0.932500\du)--(4.275000\du,3.267500\du)--(17.925000\du,3.267500\du)--(17.925000\du,-0.932500\du)--cycle;
% setfont left to latex
\definecolor{dialinecolor}{rgb}{0.000000, 0.000000, 0.000000}
\pgfsetstrokecolor{dialinecolor}
\node at (11.100000\du,1.362500\du){};
\definecolor{dialinecolor}{rgb}{1.000000, 1.000000, 1.000000}
\pgfsetfillcolor{dialinecolor}
\fill (5.416250\du,0.167500\du)--(5.416250\du,2.067500\du)--(7.533750\du,2.067500\du)--(7.533750\du,0.167500\du)--cycle;
\pgfsetlinewidth{0.050000\du}
\pgfsetdash{}{0pt}
\pgfsetdash{}{0pt}
\pgfsetmiterjoin
\definecolor{dialinecolor}{rgb}{0.000000, 0.000000, 0.000000}
\pgfsetstrokecolor{dialinecolor}
\draw (5.416250\du,0.167500\du)--(5.416250\du,2.067500\du)--(7.533750\du,2.067500\du)--(7.533750\du,0.167500\du)--cycle;
% setfont left to latex
\definecolor{dialinecolor}{rgb}{0.000000, 0.000000, 0.000000}
\pgfsetstrokecolor{dialinecolor}
\node at (6.475000\du,1.312500\du){0.0};
\definecolor{dialinecolor}{rgb}{1.000000, 1.000000, 1.000000}
\pgfsetfillcolor{dialinecolor}
\fill (8.625000\du,0.122500\du)--(8.625000\du,2.022500\du)--(10.742500\du,2.022500\du)--(10.742500\du,0.122500\du)--cycle;
\pgfsetlinewidth{0.050000\du}
\pgfsetdash{}{0pt}
\pgfsetdash{}{0pt}
\pgfsetmiterjoin
\definecolor{dialinecolor}{rgb}{0.000000, 0.000000, 0.000000}
\pgfsetstrokecolor{dialinecolor}
\draw (8.625000\du,0.122500\du)--(8.625000\du,2.022500\du)--(10.742500\du,2.022500\du)--(10.742500\du,0.122500\du)--cycle;
% setfont left to latex
\definecolor{dialinecolor}{rgb}{0.000000, 0.000000, 0.000000}
\pgfsetstrokecolor{dialinecolor}
\node at (9.683750\du,1.267500\du){1.0};
\definecolor{dialinecolor}{rgb}{1.000000, 1.000000, 1.000000}
\pgfsetfillcolor{dialinecolor}
\fill (11.127500\du,0.077500\du)--(11.127500\du,1.977500\du)--(14.240000\du,1.977500\du)--(14.240000\du,0.077500\du)--cycle;
\pgfsetlinewidth{0.050000\du}
\pgfsetdash{}{0pt}
\pgfsetdash{}{0pt}
\pgfsetmiterjoin
\definecolor{dialinecolor}{rgb}{0.000000, 0.000000, 0.000000}
\pgfsetstrokecolor{dialinecolor}
\draw (11.127500\du,0.077500\du)--(11.127500\du,1.977500\du)--(14.240000\du,1.977500\du)--(14.240000\du,0.077500\du)--cycle;
% setfont left to latex
\definecolor{dialinecolor}{rgb}{0.000000, 0.000000, 0.000000}
\pgfsetstrokecolor{dialinecolor}
\node at (12.683750\du,1.222500\du){Noir};
\definecolor{dialinecolor}{rgb}{1.000000, 1.000000, 1.000000}
\pgfsetfillcolor{dialinecolor}
\fill (15.125000\du,0.132500\du)--(15.125000\du,2.032500\du)--(17.242500\du,2.032500\du)--(17.242500\du,0.132500\du)--cycle;
\pgfsetlinewidth{0.050000\du}
\pgfsetdash{}{0pt}
\pgfsetdash{}{0pt}
\pgfsetmiterjoin
\definecolor{dialinecolor}{rgb}{0.000000, 0.000000, 0.000000}
\pgfsetstrokecolor{dialinecolor}
\draw (15.125000\du,0.132500\du)--(15.125000\du,2.032500\du)--(17.242500\du,2.032500\du)--(17.242500\du,0.132500\du)--cycle;
% setfont left to latex
\definecolor{dialinecolor}{rgb}{0.000000, 0.000000, 0.000000}
\pgfsetstrokecolor{dialinecolor}
\node at (16.183750\du,1.277500\du){3.0};
% setfont left to latex
\definecolor{dialinecolor}{rgb}{0.000000, 0.000000, 0.000000}
\pgfsetstrokecolor{dialinecolor}
\node[anchor=west] at (-0.725000\du,3.517500\du){P1};
% setfont left to latex
\definecolor{dialinecolor}{rgb}{0.000000, 0.000000, 0.000000}
\pgfsetstrokecolor{dialinecolor}
\node[anchor=west] at (3.275000\du,0.467500\du){};
\definecolor{dialinecolor}{rgb}{1.000000, 1.000000, 1.000000}
\pgfsetfillcolor{dialinecolor}
\fill (-0.925000\du,-0.032500\du)--(-0.925000\du,1.867500\du)--(1.075000\du,1.867500\du)--(1.075000\du,-0.032500\du)--cycle;
\pgfsetlinewidth{0.050000\du}
\pgfsetdash{}{0pt}
\pgfsetdash{}{0pt}
\pgfsetmiterjoin
\definecolor{dialinecolor}{rgb}{0.000000, 0.000000, 0.000000}
\pgfsetstrokecolor{dialinecolor}
\draw (-0.925000\du,-0.032500\du)--(-0.925000\du,1.867500\du)--(1.075000\du,1.867500\du)--(1.075000\du,-0.032500\du)--cycle;
% setfont left to latex
\definecolor{dialinecolor}{rgb}{0.000000, 0.000000, 0.000000}
\pgfsetstrokecolor{dialinecolor}
\node at (0.075000\du,1.112500\du){};
\pgfsetlinewidth{0.050000\du}
\pgfsetdash{}{0pt}
\pgfsetdash{}{0pt}
\pgfsetmiterjoin
\pgfsetbuttcap
{
\definecolor{dialinecolor}{rgb}{0.000000, 0.000000, 0.000000}
\pgfsetfillcolor{dialinecolor}
% was here!!!
\pgfsetarrowsend{to}
{\pgfsetcornersarced{\pgfpoint{0.000000\du}{0.000000\du}}\definecolor{dialinecolor}{rgb}{0.000000, 0.000000, 0.000000}
\pgfsetstrokecolor{dialinecolor}
\draw (0.650250\du,0.817500\du)--(2.237625\du,0.817500\du)--(2.237625\du,1.067500\du)--(3.825000\du,1.067500\du);
}}
\pgfsetlinewidth{0.050000\du}
\pgfsetdash{}{0pt}
\pgfsetdash{}{0pt}
\pgfsetbuttcap
{
\definecolor{dialinecolor}{rgb}{0.000000, 0.000000, 0.000000}
\pgfsetfillcolor{dialinecolor}
% was here!!!
\definecolor{dialinecolor}{rgb}{0.000000, 0.000000, 0.000000}
\pgfsetstrokecolor{dialinecolor}
\draw (3.500000\du,-0.612500\du)--(2.200000\du,2.287500\du);
}
\pgfsetlinewidth{0.050000\du}
\pgfsetdash{}{0pt}
\pgfsetdash{}{0pt}
\pgfsetbuttcap
{
\definecolor{dialinecolor}{rgb}{0.000000, 0.000000, 0.000000}
\pgfsetfillcolor{dialinecolor}
% was here!!!
\definecolor{dialinecolor}{rgb}{0.000000, 0.000000, 0.000000}
\pgfsetstrokecolor{dialinecolor}
\draw (1.616078\du,-0.741422\du)--(3.400000\du,2.637500\du);
}
\pgfsetlinewidth{0.050000\du}
\pgfsetdash{}{0pt}
\pgfsetdash{}{0pt}
\pgfsetmiterjoin
\pgfsetbuttcap
{
\definecolor{dialinecolor}{rgb}{0.000000, 0.000000, 0.000000}
\pgfsetfillcolor{dialinecolor}
% was here!!!
\pgfsetarrowsend{to}
{\pgfsetcornersarced{\pgfpoint{0.000000\du}{0.000000\du}}\definecolor{dialinecolor}{rgb}{0.000000, 0.000000, 0.000000}
\pgfsetstrokecolor{dialinecolor}
\draw (0.325000\du,1.467500\du)--(2.262500\du,1.467500\du)--(2.262500\du,9.987500\du)--(4.200000\du,9.987500\du);
}}
\end{tikzpicture}
				
			\lstinputlisting{4-4-1.algo}
			\lstinputlisting{4-5.algo}
			On obtient à l'affichage\\ 3.5\\3.5	
			\paragraph{Remarque}
				\subparagraph{1- }
					Avec une représentation par pointeur (et donc avec un sémantique de référence) 
					on peut définir un type de sémantique de valeur\\
					$\Rightarrow$ ajouter dans le TAD une opération de copie et une opération de 
					comparaison.\\
					Pour le TAD Point en représentation dynamique on définit
					\lstinputlisting{4-6.algo}
				\subparagraph{2- }
					En C, un tableau est représenté par un pointeur constant vers son premier élément. 	\\
					$\Rightarrow$ Affectation de deux tableaux n'est pas autorisée!\\
					Comparaison de deux tableaux sont autorisés: comparaisons de deux adresses (différent
					de la comparaison des élément des deux tableaux).


\section{Exportation des opérateurs <-, = et /=}
	\subsection{Implémentation sans exportation des opérateurs}
		La spécification du TAD limite les opérations en interdisant l'usage de l'affectation et des 
		comparaisons.\\
		$\Rightarrow$ la spécification n'inclut pas d'en tête pour <-, = et /=
		\subsubsection{Exemple}
			Pour le TAD Point en implémentation statique ou dynamique. \\
			Pas d'indication de la spécification vis-à-vis des opérateurs implique l'interdiction au client
			d'utiliser les opérateurs. (voir spécification du TAD du chapitre 2) 
			\lstinputlisting[caption=Pour un client]{5-1.algo}
		\paragraph{Remarque} 
			Le statut d'exportation permet de protéger les données d'un type abstrait 
			(renforce l'encapsulation)\\
			Par exemple soit le TAD compteInformatique représente
			\lstinputlisting{5-2.algo}
	\subsection{Implémentation sans exportation}
		La spécification du type abstrait avec un en-tête par opérateur selon la syntaxe pour le type T.	
			\lstinputlisting[caption=Spécification]{5-3.algo}
			\lstinputlisting[caption=Cotès client]{5-4.algo}
			Pour l'implémentation du type deux possibilités.
			\begin{itemize}
				\item Pas de corps pour l'opérateur si la sémantique donné par le langage correspond à
					celle du type.
				\item Écriture d'un corps pour l'opérateur à la sémantique du langage ne correspond plus
					à celle du type abstrait
			\end{itemize}	
	\subsection{Synthèse}
		\begin{tabular}{|p{3cm}|p{7cm}|p{7cm}|}
			\hline
				& Implémentation sans exportations & Implémentation avec exportations\\ % à fusionner.... 
			\hline
				Utilisation (client) & opérateur non définie & Opérateur définie\\
			\hline 
				Spécification (concepteur) & Pas d'entête pour l'opérateur & Entête pour l'opérateur \\ 
			\hline
				Implémentation (programmeur) & Pas de corps pour l'opérateur (il n'y a pas d'entête) &pas de corps opérateur supporté par le langage. Présence d'un corps, redéfinition de l'opérateur\\
			\hline
		\end{tabular}
		\subsubsection{Remarque}
			\paragraph{1- } 
				On peut accorder un statut d'exportation différent selon l'opérateur. 
			\paragraph{2- } 
				L'opérateur inégalité est toujours défini implicitement à partir de l'opérateur égalité.	


\section{Type fonctionnelle V.S type impératif}
	\subsection{Type fonctionnelle}
		Type qui ne propose que des fonctions algorithmique. En général par d'affectation.  
	\subsection{Type impératif}
		Type qui possède au moins une opération de modification avec une procédure et un mode mise à jour pour une 
		variable du type. En général l'affectation est autorisée.
							
		\newpage
\end{document}
