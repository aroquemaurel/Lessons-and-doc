\documentclass{article}

\usepackage{lmodern}
\usepackage{xcolor}
\usepackage[utf8]{inputenc}
\usepackage[T1]{fontenc}
\usepackage[francais]{babel}
\usepackage[top=1.7cm, bottom=1.7cm, left=1.7cm, right=1.7cm]{geometry}
%\usepackage[frenchb]{babel}
%\usepackage{layout}
%\usepackage{setspace}
%\usepackage{soul}
%\usepackage{ulem}
%\usepackage{eurosym}
%\usepackage{bookman}
%\usepackage{charter}
%\usepackage{newcent}
%\usepackage{lmodern}
%\usepackage{mathpazo}
%\usepackage{mathptmx}
%\usepackage{url}
%\usepackage{verbatim}
%\usepackage{moreverb}
%\usepackage{wrapfig}
%\usepackage{amsmath}
%\usepackage{mathrsfs}
%\usepackage{asmthm}
%\usepackage{makeidx}
\usepackage{listings}
\usepackage{fancyhdr}
\usepackage{multido}
\usepackage{amssymb}

\definecolor{gris1}{gray}{0.40}
\definecolor{gris2}{gray}{0.55}
\definecolor{gris3}{gray}{0.65}
\definecolor{gris4}{gray}{0.50}


\lstdefinelanguage{algo}{%
   morekeywords={%
    %%% couleur 1
		importer, programme, glossaire, fonction, procedure, constante, type, 
	%%% IMPORT & Co.
		si, sinon, alors, fin, tantque, debut, faire, lorsque, fin lorsque, declancher, enregistrement, tableau, retourne, retourner, =, /=, <, >, traite,exception, 
	%%% types 
		Entier, Reel, Booleen, Caractere,
	%%% types 
		entree, maj, sortie,	
	%%% types 
		et, ou, non,
	},
  sensitive=true,
  morecomment=[l]{--},
  morestring=[b]',
}

%\lstset{language=algo,
    %%% BOUCLE, TEST & Co.
%      emph={importer, programme, glossaire, fonction, procedure, constante, type},
%      emphstyle=\color{gris2},
    %%% IMPORT & Co.
%      emph={[2]si, sinon, alors, fin , tantque, debut, faire, lorsque, fin lorsque, declancher, retourner, et, ou, non,enregistrement, retourner, retourne, tableau, /=, <, =, >, traite,exception},
%      emphstyle=[2]\color{gris1},
    %%% FONCTIONS NUMERIQUES
%      emph={[3]Entier, Reel, Booleen, Caractere},
%      emphstyle=[3]\color{gris3},
    %%% FONCTIONS NUMERIQUES
%      emph={[4]entree, maj, sortie},	
%      emphstyle=[4]\color{gris4},
%}
\lstset{ % general style for listings 
   numbers=left 
	, extendedchars=\true
   , tabsize=2 
   , frame=single 
   , breaklines=true 
   , basicstyle=\ttfamily 
   , numberstyle=\tiny\ttfamily 
   , framexleftmargin=13mm 
   , xleftmargin=12mm 
   , captionpos=b 
	, language=algo
	, keywordstyle=\color{blue}
	, commentstyle=\color{green}
	, showstringspaces=false
	, extendedchars=true
	, mathescape=true
} 
 %prise en charge du langage algo

\title{TD 12\\ Recherche dichotomique}
\date{Algorithmique\\ Semestre 1}

\lhead{TD 12: Recherche dichotomique}
\chead{}
\rhead{\thepage}

\lfoot{Université paul sabatier Toulouse III}
\cfoot{\thepage}
\rfoot{Alg1}

\pagestyle{fancy}

\begin{document}
	\maketitle
	\section{Spécification}
		\subsection{}
			\lstinputlisting[caption=Entête de rechercherParDichotomie]{1-1.algo}
		\subsection{}
			\lstinputlisting[caption=Entête de rechercherParDichotomie avec précondition et postcondition]{1-2.algo}
	\section{Algorithme}
		\subsection{}
			On divise le tableau en deux sous ensemble si on n'a pas trouvé l'élément on répéte(=>boucle) le processus dans la tranche où il peut se trouver.
		\subsection{}
			\lstinputlisting[caption=Algorithme général de la recherche dichotomique]{2-2.algo}
	\section{Programmation}
		\subsection{}
			Tranche non valide:  $iDeb \leq iFin$\\
			Élément non trouvé:  $tab[iMil] \neq x$
		\subsection{}
			Réduire à la partie gauche: iFin <- iMil - 1;\\
			Réduire à la partie droite: iDeb <- iMil + 1;	
		\subsection{}
			\lstinputlisting[caption=Corps de rechercherParDichotomie]{3-3.algo}
		\subsection{}
			\subsubsection{Pour x = 5}
			\begin{tabular}{|c|c|c|c|c|c|c|}
				\hline
					\textbf{Situation} & \textbf{iDeb} & \textbf{iFin} & \textbf{iMi} & \textbf{tab[iMil]} & \textbf{trouvé} & \textbf{rang}\\
				\hline
					1 & 1 & 11 & 6 & 35 & &\\
				\hline
					2 & 1 & 5 & 6 & 35 & &\\ 
				\hline
					4 & 1 & 5 & 3 & 20 & &\\ 
				\hline
					2 & 1 & 2 & 3 & 20 & &\\ 
				\hline
					4 & 1 & 2 & 1 & 10 & &\\ 
				\hline
					2 & 1 & 0 & 1 & 10 & &\\ 
				\hline
					4 & 1 & 0 & 0 & & &\\ 
				\hline
					5 & 1 & 0 & 0 & & FAUX &\\ 
				\hline
			\end{tabular}
			\subsubsection{Pour x = 10}
			\begin{tabular}{|c|c|c|c|c|c|c|}
				\hline
					\textbf{Situation} & \textbf{iDeb} & \textbf{iFin} & \textbf{iMi} & \textbf{tab[iMil]} & \textbf{trouvé} & \textbf{rang}\\
				\hline
					1 & 1 & 11 & 6 & 35 & &\\
				\hline
					2 & 1 & 5 & 6 & 35 & &\\ 
				\hline
					4 & 1 & 5 & 3 & 20 & &\\ 
				\hline
					2 & 1 & 2 & 3 & 20 & &\\ 
				\hline
					4 & 1 & 2 & 1 & 10 & &\\ 
				\hline
					6 & 1 & 2 & 1 & 10 & VRAI & 1\\ 
				\hline
			\end{tabular}
			\subsubsection{Pour x = 20}
			\begin{tabular}{|c|c|c|c|c|c|c|}
				\hline
					\textbf{Situation} & \textbf{iDeb} & \textbf{iFin} & \textbf{iMi} & \textbf{tab[iMil]} & \textbf{trouvé} & \textbf{rang}\\
				\hline
					1 & 1 & 11 & 6 & 35 & &\\
				\hline
					2 & 1 & 5 & 6 & 35 & &\\ 
				\hline
					4 & 1 & 5 & 3 & 20 & &\\ 
				\hline
					6 & 1 & 5 & 3 & 20 & VRAI & 3\\ 
				\hline
			\end{tabular}
			\subsubsection{Pour x = 5}
			\begin{tabular}{|c|c|c|c|c|c|c|}
				\hline
					\textbf{Situation} & \textbf{iDeb} & \textbf{iFin} & \textbf{iMi} & \textbf{tab[iMil]} & \textbf{trouvé} & \textbf{rang}\\
				\hline
					1 & 1 & 11 & 6 & 35 & &\\
				\hline
					3 & 7 & 11 & 6 & 35 & &\\ 
				\hline
					4 & 7 & 11 & 9 & 65 & &\\ 
				\hline
					3 & 10 & 11 & 9 & 65 & &\\ 
				\hline
					4 & 10 & 11 & 10 & 70 & &\\ 
				\hline
					3 & 11 & 11 & 10 & 70 & &\\ 
				\hline
					4 & 11 & 11 & 11 & 75 & &\\ 
				\hline
					3 & 12 & 11 & 11 & 75 & &\\ 
				\hline
					5 & 12 & 11 & 11 & 75 & FAUX &\\ 
				\hline
			\end{tabular}
			\subsubsection{Pour x = 5}
			\begin{tabular}{|c|c|c|c|c|c|c|}
				\hline
					\textbf{Situation} & \textbf{iDeb} & \textbf{iFin} & \textbf{iMi} & \textbf{tab[iMil]} & \textbf{trouvé} & \textbf{rang}\\
				\hline
					1 & 1 & 11 & 6 & 35 & &\\
				\hline
					2 & 1 & 5 & 6 & 35 & &\\ 
				\hline
					4 & 1 & 5 & 3 & 20 & &\\ 
				\hline
					2 & 1 & 2 & 3 & 20 & &\\ 
				\hline
					4 & 1 & 2 & 1 & 10 & &\\ 
				\hline
					2 & 1 & 0 & 1 & 10 & &\\ 
				\hline
					4 & 1 & 0 & 0 & & &\\ 
				\hline
					5 & 1 & 0 & 0 & & FAUX &\\ 
				\hline
			\end{tabular}
			\subsubsection{Pour x = 80}
			\begin{tabular}{|c|c|c|c|c|c|c|}
				\hline
					\textbf{Situation} & \textbf{iDeb} & \textbf{iFin} & \textbf{iMi} & \textbf{tab[iMil]} & \textbf{trouvé} & \textbf{rang}\\
				\hline
					1 & 1 & 11 & 6 & 35 & &\\
				\hline
					3 & 7 & 1 & 6 & 35 & &\\ 
				\hline
					4 & 1 & 5 & 3 & 20 & &\\ 
				\hline
					3 & 1 & 2 & 3 & 20 & &\\ 
				\hline
					4 & 1 & 2 & 1 & 10 & &\\ 
				\hline
					3 & 1 & 0 & 1 & 10 & &\\ 
				\hline
					4 & 1 & 0 & 0 & & &\\ 
				\hline
					3 & 1 & 0 & 0 & & FAUX &\\ 
				\hline
			\end{tabular}
		\subsection{}
			$$2^{k-1} < n < 2^{k}$$
			$$k-1 < \log(n) \leq k$$
			$$O(\log(n))$$
			\begin{tabular}{c|c|c}
					\textbf{n} & \textbf{Recherche séquentielle} & \textbf{Recherche dichotomique}\\
				\hline
					10 & 10 & 4($2^{4}$)\\
				\hline
					100 & 100 & 7 \\
				\hline
					\vdots & \vdots & \vdots\\
			\end{tabular}	
		\subsection{}
			\lstinputlisting[caption=Fonction estTrié]{3-6.algo}
	\section{Ecriture récursive}	
		\lstinputlisting[caption=procédure rechercherParDichotomie en récurisf]{4.algo}
\end{document}

