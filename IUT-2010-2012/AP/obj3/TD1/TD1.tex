\documentclass{article}

\usepackage{lmodern}
\usepackage{xcolor}
\usepackage[utf8]{inputenc}
\usepackage[T1]{fontenc}
\usepackage[francais]{babel}
\usepackage[top=1.7cm, bottom=1.7cm, left=1.7cm, right=1.7cm]{geometry}
\usepackage{verbatim}
\usepackage{tikz} %Vectoriel
\usepackage{listings}
\usepackage{fancyhdr}
\usepackage{multido}
\usepackage{amssymb}

\newcommand{\titre}{Au zoo}
\newcommand{\numTD}{1}

\newcommand{\module}{Concepts de la programmation par objet}
\newcommand{\sigle}{obj}

\newcommand{\semestre}{3}

\definecolor{gris1}{gray}{0.40}
\definecolor{gris2}{gray}{0.55}
\definecolor{gris3}{gray}{0.65}
\definecolor{gris4}{gray}{0.50}
\definecolor{vert}{rgb}{0,0.4,0}
\definecolor{violet}{rgb}{0.65, 0.2, 0.65}
\definecolor{bleu1}{rgb}{0,0,0.8}
\definecolor{bleu2}{rgb}{0,0.2,0.6}
\definecolor{bleu3}{rgb}{0,0.2,0.2}
\definecolor{rouge}{HTML}{F93928}


\lstdefinelanguage{algo}{%
   morekeywords={%
    %%% couleur 1
		importer, programme, glossaire, fonction, procedure, constante, type, 
	%%% IMPORT & Co.
		si, sinon, alors, fin, tantque, debut, faire, lorsque, fin lorsque, 
		declenche, declencher, enregistrement, tableau, retourne, retourner, =, pour, a,
		/=, <, >, traite,exception, 
	%%% types 
		Entier, Reel, Booleen, Caractere, Réél, Booléen, Caractère,
	%%% types 
		entree, maj, sortie,entrée,
	%%% types 
		et, ou, non,
	},
  sensitive=true,
  morecomment=[l]{--},
  morestring=[b]',
}

\lstset{language=algo,
    %%% BOUCLE, TEST & Co.
      emph={importer, programme, glossaire, fonction, procedure, constante, type},
      emphstyle=\color{bleu2},
    %%% IMPORT & Co.  
	emph={[2]
		si, sinon, alors, fin , tantque, debut, faire, lorsque, fin lorsque, 
		declencher, retourner, et, ou, non,enregistrement, retourner, retourne, 
		tableau, /=, <, =, >, traite,exception, pour, a
	},
      emphstyle=[2]\color{bleu1},
    %%% FONCTIONS NUMERIQUES
      emph={[3]Entier, Reel, Booleen, Caractere, Booléen, Réél, Caractère},
      emphstyle=[3]\color{gris1},
    %%% FONCTIONS NUMERIQUES
      emph={[4]entree, maj, sortie, entrée},	
      emphstyle=[4]\color{gris1},
}
\lstdefinelanguage{wl}{%
   morekeywords={%
    %%% couleur 1
		importer, programme, glossaire, fonction, procedure, constante, type, 
	%%% IMPORT & Co.
		si, sinon, alors, fin, TANTQUE, tantque, FIN, PROCEDURE, debut, faire, lorsque, 
		fin lorsque, declenche, declencher, enregistrement, tableau, retourne, retourner, =, 
		/=, <, >, traite,exception, 
	%%% types 
		Entier, Reel, Booleen, Caractere, Réél, Booléen, Caractère,
	%%% types 
		entree, maj, sortie,entrée,
	%%% types 
		et, ou, non,
	},
  sensitive=true,
  morecomment=[l]{//},
  morestring=[b]',
}

\lstset{language=wl,
    %%% BOUCLE, TEST & Co.
      emph={importer, programme, glossaire, fonction, procedure, constante, type},
      emphstyle=\color{bleu2},
    %%% IMPORT & Co.  
	emph={[2]
		si, sinon, alors, fin , tantque, debut, faire, lorsque, fin lorsque, 
		declencher, retourner, et, ou, non,enregistrement, retourner, retourne, 
		tableau, /=, <, =, >, traite,exception
	},
      emphstyle=[2]\color{bleu1},
    %%% FONCTIONS NUMERIQUES
      emph={[3]Entier, Reel, Booleen, Caractere, Booléen, Réél, Caractère},
      emphstyle=[3]\color{gris1},
    %%% FONCTIONS NUMERIQUES
      emph={[4]entree, maj, sortie, entrée},	
      emphstyle=[4]\color{gris1},
}
\lstdefinelanguage{css}{%
   morekeywords={%
    %%% couleur 1
		background, image, repeat, position, index, color, border, font, 
		size, url, family, style, variant, weight, letter, spacing, line, 
		height, text, decoration, align, indent, transform, shadow, 
		background, image, repeat, position, index, color, border, font, 
		size, url, family, style, variant, weight, letter, spacing, line, 
		height, text, decoration, align, indent, transform, shadow, 
		vertical, align, white, space, word, spacing,attachment, width, 
		max, min, margin, padding, clip, direction, display, overflow,
		visibility, clear, float, top, right, bottom, left, list, type, 
		collapse, side, empty, cells, table, layout, cursor, marks, page, break,
		before, after, inside, orphans, windows, azimuth, after, before, cue, 
		elevation, pause, play, during, pitch, range, richness, spek, header, 
		numeral, punctuation, rate, stress, voice, volume,
	%%% types 
		left, right, bottom, top, none, center, solid, black, blue, red, green,
	},
  sensitive=true,
  sensitive=true,
  morecomment=[s]{/*}{*/},
  morestring=[b]',
}
\lstset{language=css,
    %%% BOUCLE, TEST & Co.
      emph={
		background, image, repeat, position, index, color, border, font, 
		size, url, family, style, variant, weight, letter, spacing, line, 
		height, text, decoration, align, indent, transform, shadow, 
		background, image, repeat, position, index, color, border, font, 
		size, url, family, style, variant, weight, letter, spacing, line, 
		height, text, decoration, align, indent, transform, shadow, 
		vertical, align, white, space, word, spacing,attachment, width, 
		max, min, margin, padding, clip, direction, display, overflow,
		visibility, clear, float, top, right, bottom, left, list, type, 
		collapse, side, empty, cells, table, layout, cursor, marks, page, break,
		before, after, inside, orphans, windows, azimuth, after, before, cue, 
		elevation, pause, play, during, pitch, range, richness, spek, header, 
		numeral, punctuation, rate, stress, voice, volume,
	  },
      emphstyle=\color{bleu2},
    %%% FONCTIONS NUMERIQUES
      emph={[3]
		left, right, bottom, top,none, solid, black, blue, green,
		  },
      emphstyle=[3]\color{bleu3},
    %%% FONCTIONS NUMERIQUES
}

\lstset{language=SQL,
    %%% BOUCLE, TEST & Co.
      emph={INSERT, UPDATE, DELETE, WHERE, SET, GROUP, BY, ORDER, REFERENCES},
      emphstyle=\color{bleu2},
    %%% IMPORT & Co.  
	emph={[2]
		if, end, begin, then, for, each, else, after, of, on, to
	},
      emphstyle=[2]\color{bleu1},
    %%% FONCTIONS NUMERIQUES
      emph={[3]Entier, Reel, Booleen, Caractere, Booléen, Réél, Caractère},
      emphstyle=[3]\color{gris1},
    %%% FONCTIONS NUMERIQUES
      emph={[4]entree, maj, sortie, entrée},	
      emphstyle=[4]\color{gris1},
}
\lstdefinelanguage{ARM}{%
   morekeywords={%
   ADD, SUB, MOV, MUL, RSB,CMP, BLS, BLE, B,BHI,LDR,
   BGE, RSBLT, BGT, BEQ, BNE,BLT,BHS,STR,STRB
	},
  sensitive=true,
  morecomment=[l]{@},
  morestring=[b]',
}

\lstset{ % general style for listings 
   numbers=left 
   , literate={é}{{\'e}}1 {è}{{\`e}}1 {à}{{\`a}}1 {ê}{{\^e}}1 {É}{{\'E}}1 {ô}{{\^o}}1 {€}{{\euro}}1{°}{{$^{\circ}$}}1 {ç}{ {c}}1 {ù}{u}1
	, extendedchars=\true
   , tabsize=2 
   , frame=l
   , framerule=1.1pt
   , linewidth=520px
   , breaklines=true 
   , basicstyle=\footnotesize\ttfamily 
   , numberstyle=\tiny\ttfamily 
   , framexleftmargin=0mm 
   , xleftmargin=0mm 
   , captionpos=b 
	, keywordstyle=\color{bleu2}
	, commentstyle=\color{vert}
	, stringstyle=\color{rouge}
	, showstringspaces=false
	, extendedchars=true
	, mathescape=true
} 
%	\lstlistoflistings
%	\addcontentsline{toc}{part}{List of code examples}
 %prise en charge du langage algo

\usepackage{ifthen}
\date{\today}

\chead{}
\rhead{TD\no\numTD}
\lhead{\titre}
%\makeindex

\lfoot{Université Paul Sabatier Toulouse III}
\rfoot{\sigle\semestre}
%\rfoot{}
\cfoot{--~~\thepage~~--}

\makeglossary
\makeatletter
\def\clap#1{\hbox to 0pt{\hss #1\hss}}%

\def\haut#1#2#3{%
	\hbox to \hsize{%
		\rlap{\vtop{\raggedright #1}
	}%
	\hss
	\clap{\vtop{\centering #2}
}%
\hss
\llap{\vtop{\raggedleft #3}}}}%
\def\bas#1#2#3{%
	\hbox to \hsize{%
		\rlap{\vbox{
			\raggedright #1
		}
	}%
	\hss \clap{\vbox{\centering #2}}%
	\hss
	\llap{\vbox{\raggedleft #3}}}
}%
\def\maketitle{%
	\thispagestyle{empty}{%
		\haut{}{\@blurb}{}
		%	
		%\vfill

		\begin{center}
			\vspace{-1.5cm}
			\usefont{OT1}{ptm}{m}{n}
			\huge \@numeroTD \@title
		\end{center}
		\par
		\hrule height 1pt
		\par
		\vspace{1cm}
		\bas{}{}{}
}%
}
\def\date#1{\def\@date{#1}}
\def\author#1{\def\@author{#1}}
\def\numeroTD#1{\def\@numeroTD{#1}}
\def\title#1{\def\@title{#1}}
\def\location#1{\def\@location{#1}}
\def\blurb#1{\def\@blurb{#1}}
\date{\today}
\newboolean{monBool}
\setboolean{monBool}{true}
\author{}
\title{}
\ifthenelse{\equal{\numTD}{}}{
\numeroTD{}
}
{
	\numeroTD{TD \no\numTD~--- }
}
\location{Amiens}\blurb{}
%\makeatother
\title{\titre}
\author{%Semestre \semestre
}

\location{Toulouse}
\blurb{%
\vspace{-35px}
\begin{flushleft}
	Université Paul Sabatier -- Toulouse III\\
	IUT A - Toulouse Rangueil\\
\end{flushleft}
\begin{flushright}
	\vspace{-45px}
	\Large \textbf \module \\
	\normalsize \textit \today\\
	Semestre \semestre
	\vspace{30px}
\end{flushright}
}%



%\title{Cours \\ \titre}
%\date{\today\\ Semestre \semestre}

%\lhead{Cours: \titre}
%\chead{}
%\rhead{\thepage}

%\lfoot{Université Paul Sabatier Toulouse III}
%\cfoot{\thepage}
%\rfoot{\sigle\semestre}

\pagestyle{fancy}


\begin{document}
	\maketitle
	\section{Le perroquet}
	\lstinputlisting[language=java, caption=Classe Animal]{Animal.java}	
	\lstinputlisting[language=java, caption=Classe Oiseau]{Oiseau.java}	
	\newpage
	\lstinputlisting[language=java, caption=Classe Perroquet]{Perroquet.java}	
	\lstinputlisting[language=java, caption=Classe Propriétaire]{Proprietaire.java}	
	\paragraph{1-4} En faisant hériter Perroquet de Oiseau, et faire hériter Oiseau de Animals
	\lstinputlisting[language=java, caption=Instanciation]{main.java}	
	\newpage
	\section{Le koala}
	\lstinputlisting[language=java, caption=Classe Marsupial]{Marsupial.java}	
	\lstinputlisting[language=java, caption=Classe Koala]{koala.java}	
	\newpage
	\section{Les anguilles}
	\lstinputlisting[language=java, caption=Classe Ovipare]{Ovipare.java}	
	\lstinputlisting[language=java, caption=Classe Poisson]{Poisson.java}	
	\lstinputlisting[language=java, caption=Classe Anguille]{Anguille.java}	
	\lstinputlisting[language=java, caption=Classe AnimalEauDouce]{AnimalEauDouce.java}	
	\lstinputlisting[language=java, caption=Classe AnimalMarin]{AnimalMarin.java}	
	\newpage
	\section{Le mulet -- Conflit en héritage multiple}
	Deux méthodes pour résoudre le conflit:
	\begin{itemize}
		\item Parcours du graphe d'héritage (smalltalk par exemple)
		\item Annotations du programmeur (C++ par exemple)
	\end{itemize}
	\begin{center} % Graphic for TeX using PGF
% Title: /usr/home/satenske/Diagram1.dia
% Creator: Dia v0.97.1
% CreationDate: Wed Mar 30 09:12:35 2011
% For: satenske
% \usepackage{tikz}
% The following commands are not supported in PSTricks at present
% We define them conditionally, so when they are implemented,
% this pgf file will use them.
\ifx\du\undefined
  \newlength{\du}
\fi
\setlength{\du}{15\unitlength}
\begin{tikzpicture}
\pgftransformxscale{1.000000}
\pgftransformyscale{-1.000000}
\definecolor{dialinecolor}{rgb}{0.000000, 0.000000, 0.000000}
\pgfsetstrokecolor{dialinecolor}
\definecolor{dialinecolor}{rgb}{1.000000, 1.000000, 1.000000}
\pgfsetfillcolor{dialinecolor}
\pgfsetlinewidth{0.100000\du}
\pgfsetdash{}{0pt}
\pgfsetdash{}{0pt}
\pgfsetbuttcap
\pgfsetmiterjoin
\pgfsetlinewidth{0.001000\du}
\pgfsetbuttcap
\pgfsetmiterjoin
\pgfsetdash{}{0pt}
\definecolor{dialinecolor}{rgb}{0.717647, 0.717647, 0.615686}
\pgfsetfillcolor{dialinecolor}
\pgfpathmoveto{\pgfpoint{9.544836\du}{9.940680\du}}
\pgfpathlineto{\pgfpoint{11.543262\du}{9.940680\du}}
\pgfpathlineto{\pgfpoint{11.543262\du}{10.310013\du}}
\pgfpathlineto{\pgfpoint{9.544836\du}{10.310013\du}}
\pgfpathlineto{\pgfpoint{9.544836\du}{9.940680\du}}
\pgfusepath{fill}
\pgfsetbuttcap
\pgfsetmiterjoin
\pgfsetdash{}{0pt}
\definecolor{dialinecolor}{rgb}{0.286275, 0.286275, 0.211765}
\pgfsetstrokecolor{dialinecolor}
\pgfpathmoveto{\pgfpoint{9.544836\du}{9.940680\du}}
\pgfpathlineto{\pgfpoint{11.543262\du}{9.940680\du}}
\pgfpathlineto{\pgfpoint{11.543262\du}{10.310013\du}}
\pgfpathlineto{\pgfpoint{9.544836\du}{10.310013\du}}
\pgfpathlineto{\pgfpoint{9.544836\du}{9.940680\du}}
\pgfusepath{stroke}
\pgfsetbuttcap
\pgfsetmiterjoin
\pgfsetdash{}{0pt}
\definecolor{dialinecolor}{rgb}{0.788235, 0.788235, 0.713726}
\pgfsetfillcolor{dialinecolor}
\pgfpathmoveto{\pgfpoint{9.544836\du}{9.940680\du}}
\pgfpathlineto{\pgfpoint{9.756738\du}{9.739798\du}}
\pgfpathlineto{\pgfpoint{11.755164\du}{9.739798\du}}
\pgfpathlineto{\pgfpoint{11.543262\du}{9.940680\du}}
\pgfpathlineto{\pgfpoint{9.544836\du}{9.940680\du}}
\pgfusepath{fill}
\pgfsetbuttcap
\pgfsetmiterjoin
\pgfsetdash{}{0pt}
\definecolor{dialinecolor}{rgb}{0.286275, 0.286275, 0.211765}
\pgfsetstrokecolor{dialinecolor}
\pgfpathmoveto{\pgfpoint{9.544836\du}{9.940680\du}}
\pgfpathlineto{\pgfpoint{9.756738\du}{9.739798\du}}
\pgfpathlineto{\pgfpoint{11.755164\du}{9.739798\du}}
\pgfpathlineto{\pgfpoint{11.543262\du}{9.940680\du}}
\pgfpathlineto{\pgfpoint{9.544836\du}{9.940680\du}}
\pgfusepath{stroke}
\pgfsetlinewidth{0.106000\du}
\pgfsetbuttcap
\pgfsetmiterjoin
\pgfsetdash{}{0pt}
\definecolor{dialinecolor}{rgb}{0.000000, 0.000000, 0.000000}
\pgfsetstrokecolor{dialinecolor}
\pgfpathmoveto{\pgfpoint{11.431486\du}{10.108816\du}}
\pgfpathlineto{\pgfpoint{10.951952\du}{10.108816\du}}
\pgfusepath{stroke}
\pgfsetlinewidth{0.001000\du}
\pgfsetbuttcap
\pgfsetmiterjoin
\pgfsetdash{}{0pt}
\definecolor{dialinecolor}{rgb}{0.478431, 0.478431, 0.352941}
\pgfsetfillcolor{dialinecolor}
\pgfpathmoveto{\pgfpoint{11.543262\du}{10.310013\du}}
\pgfpathlineto{\pgfpoint{11.755164\du}{10.097481\du}}
\pgfpathlineto{\pgfpoint{11.755164\du}{9.739798\du}}
\pgfpathlineto{\pgfpoint{11.543262\du}{9.940680\du}}
\pgfpathlineto{\pgfpoint{11.543262\du}{10.310013\du}}
\pgfusepath{fill}
\pgfsetbuttcap
\pgfsetmiterjoin
\pgfsetdash{}{0pt}
\definecolor{dialinecolor}{rgb}{0.286275, 0.286275, 0.211765}
\pgfsetstrokecolor{dialinecolor}
\pgfpathmoveto{\pgfpoint{11.543262\du}{10.310013\du}}
\pgfpathlineto{\pgfpoint{11.755164\du}{10.097481\du}}
\pgfpathlineto{\pgfpoint{11.755164\du}{9.739798\du}}
\pgfpathlineto{\pgfpoint{11.543262\du}{9.940680\du}}
\pgfpathlineto{\pgfpoint{11.543262\du}{10.310013\du}}
\pgfusepath{stroke}
\pgfsetbuttcap
\pgfsetmiterjoin
\pgfsetdash{}{0pt}
\definecolor{dialinecolor}{rgb}{0.788235, 0.788235, 0.713726}
\pgfsetfillcolor{dialinecolor}
\pgfpathmoveto{\pgfpoint{9.556171\du}{10.544270\du}}
\pgfpathlineto{\pgfpoint{9.779093\du}{10.264987\du}}
\pgfpathlineto{\pgfpoint{11.320025\du}{10.264987\du}}
\pgfpathlineto{\pgfpoint{11.097103\du}{10.544270\du}}
\pgfpathlineto{\pgfpoint{9.556171\du}{10.544270\du}}
\pgfusepath{fill}
\pgfsetbuttcap
\pgfsetmiterjoin
\pgfsetdash{}{0pt}
\definecolor{dialinecolor}{rgb}{0.286275, 0.286275, 0.211765}
\pgfsetstrokecolor{dialinecolor}
\pgfpathmoveto{\pgfpoint{9.556171\du}{10.544270\du}}
\pgfpathlineto{\pgfpoint{9.779093\du}{10.264987\du}}
\pgfpathlineto{\pgfpoint{11.320025\du}{10.264987\du}}
\pgfpathlineto{\pgfpoint{11.097103\du}{10.544270\du}}
\pgfpathlineto{\pgfpoint{9.556171\du}{10.544270\du}}
\pgfusepath{stroke}
\pgfsetbuttcap
\pgfsetmiterjoin
\pgfsetdash{}{0pt}
\definecolor{dialinecolor}{rgb}{0.478431, 0.478431, 0.352941}
\pgfsetfillcolor{dialinecolor}
\pgfpathmoveto{\pgfpoint{11.097103\du}{10.600000\du}}
\pgfpathlineto{\pgfpoint{11.320025\du}{10.365743\du}}
\pgfpathlineto{\pgfpoint{11.320025\du}{10.264987\du}}
\pgfpathlineto{\pgfpoint{11.097103\du}{10.544270\du}}
\pgfpathlineto{\pgfpoint{11.097103\du}{10.600000\du}}
\pgfusepath{fill}
\pgfsetbuttcap
\pgfsetmiterjoin
\pgfsetdash{}{0pt}
\definecolor{dialinecolor}{rgb}{0.286275, 0.286275, 0.211765}
\pgfsetstrokecolor{dialinecolor}
\pgfpathmoveto{\pgfpoint{11.097103\du}{10.600000\du}}
\pgfpathlineto{\pgfpoint{11.320025\du}{10.365743\du}}
\pgfpathlineto{\pgfpoint{11.320025\du}{10.264987\du}}
\pgfpathlineto{\pgfpoint{11.097103\du}{10.544270\du}}
\pgfpathlineto{\pgfpoint{11.097103\du}{10.600000\du}}
\pgfusepath{stroke}
\pgfsetbuttcap
\pgfsetmiterjoin
\pgfsetdash{}{0pt}
\definecolor{dialinecolor}{rgb}{0.717647, 0.717647, 0.615686}
\pgfsetfillcolor{dialinecolor}
\pgfpathmoveto{\pgfpoint{9.556171\du}{10.544270\du}}
\pgfpathlineto{\pgfpoint{11.097103\du}{10.544270\du}}
\pgfpathlineto{\pgfpoint{11.097103\du}{10.600000\du}}
\pgfpathlineto{\pgfpoint{9.556171\du}{10.600000\du}}
\pgfpathlineto{\pgfpoint{9.556171\du}{10.544270\du}}
\pgfusepath{fill}
\pgfsetbuttcap
\pgfsetmiterjoin
\pgfsetdash{}{0pt}
\definecolor{dialinecolor}{rgb}{0.286275, 0.286275, 0.211765}
\pgfsetstrokecolor{dialinecolor}
\pgfpathmoveto{\pgfpoint{9.556171\du}{10.544270\du}}
\pgfpathlineto{\pgfpoint{11.097103\du}{10.544270\du}}
\pgfpathlineto{\pgfpoint{11.097103\du}{10.600000\du}}
\pgfpathlineto{\pgfpoint{9.556171\du}{10.600000\du}}
\pgfpathlineto{\pgfpoint{9.556171\du}{10.544270\du}}
\pgfusepath{stroke}
\pgfsetbuttcap
\pgfsetmiterjoin
\pgfsetdash{}{0pt}
\definecolor{dialinecolor}{rgb}{0.000000, 0.000000, 0.000000}
\pgfsetfillcolor{dialinecolor}
\pgfpathmoveto{\pgfpoint{9.846159\du}{9.896285\du}}
\pgfpathlineto{\pgfpoint{10.013980\du}{9.739798\du}}
\pgfpathlineto{\pgfpoint{11.431486\du}{9.739798\du}}
\pgfpathlineto{\pgfpoint{11.275630\du}{9.896285\du}}
\pgfpathlineto{\pgfpoint{9.846159\du}{9.896285\du}}
\pgfusepath{fill}
\pgfsetbuttcap
\pgfsetmiterjoin
\pgfsetdash{}{0pt}
\definecolor{dialinecolor}{rgb}{0.000000, 0.000000, 0.000000}
\pgfsetstrokecolor{dialinecolor}
\pgfpathmoveto{\pgfpoint{9.846159\du}{9.896285\du}}
\pgfpathlineto{\pgfpoint{10.013980\du}{9.739798\du}}
\pgfpathlineto{\pgfpoint{11.431486\du}{9.739798\du}}
\pgfpathlineto{\pgfpoint{11.275630\du}{9.896285\du}}
\pgfpathlineto{\pgfpoint{9.846159\du}{9.896285\du}}
\pgfusepath{stroke}
\pgfsetbuttcap
\pgfsetmiterjoin
\pgfsetdash{}{0pt}
\definecolor{dialinecolor}{rgb}{0.788235, 0.788235, 0.713726}
\pgfsetfillcolor{dialinecolor}
\pgfpathmoveto{\pgfpoint{9.834824\du}{8.745151\du}}
\pgfpathlineto{\pgfpoint{9.991625\du}{8.600000\du}}
\pgfpathlineto{\pgfpoint{11.409761\du}{8.600000\du}}
\pgfpathlineto{\pgfpoint{11.252960\du}{8.745151\du}}
\pgfpathlineto{\pgfpoint{9.834824\du}{8.745151\du}}
\pgfusepath{fill}
\pgfsetbuttcap
\pgfsetmiterjoin
\pgfsetdash{}{0pt}
\definecolor{dialinecolor}{rgb}{0.286275, 0.286275, 0.211765}
\pgfsetstrokecolor{dialinecolor}
\pgfpathmoveto{\pgfpoint{9.834824\du}{8.745151\du}}
\pgfpathlineto{\pgfpoint{9.991625\du}{8.600000\du}}
\pgfpathlineto{\pgfpoint{11.409761\du}{8.600000\du}}
\pgfpathlineto{\pgfpoint{11.252960\du}{8.745151\du}}
\pgfpathlineto{\pgfpoint{9.834824\du}{8.745151\du}}
\pgfusepath{stroke}
\pgfsetbuttcap
\pgfsetmiterjoin
\pgfsetdash{}{0pt}
\definecolor{dialinecolor}{rgb}{0.717647, 0.717647, 0.615686}
\pgfsetfillcolor{dialinecolor}
\pgfpathmoveto{\pgfpoint{9.834824\du}{8.745151\du}}
\pgfpathlineto{\pgfpoint{11.264295\du}{8.745151\du}}
\pgfpathlineto{\pgfpoint{11.264295\du}{9.873615\du}}
\pgfpathlineto{\pgfpoint{9.834824\du}{9.873615\du}}
\pgfpathlineto{\pgfpoint{9.834824\du}{8.745151\du}}
\pgfusepath{fill}
\pgfsetbuttcap
\pgfsetmiterjoin
\pgfsetdash{}{0pt}
\definecolor{dialinecolor}{rgb}{0.286275, 0.286275, 0.211765}
\pgfsetstrokecolor{dialinecolor}
\pgfpathmoveto{\pgfpoint{9.834824\du}{8.745151\du}}
\pgfpathlineto{\pgfpoint{11.263665\du}{8.745151\du}}
\pgfpathlineto{\pgfpoint{11.263665\du}{9.873300\du}}
\pgfpathlineto{\pgfpoint{9.834824\du}{9.873300\du}}
\pgfpathlineto{\pgfpoint{9.834824\du}{8.745151\du}}
\pgfusepath{stroke}
\pgfsetbuttcap
\pgfsetmiterjoin
\pgfsetdash{}{0pt}
\definecolor{dialinecolor}{rgb}{1.000000, 1.000000, 1.000000}
\pgfsetfillcolor{dialinecolor}
\pgfpathmoveto{\pgfpoint{9.957620\du}{8.889987\du}}
\pgfpathlineto{\pgfpoint{11.141184\du}{8.889987\du}}
\pgfpathlineto{\pgfpoint{11.141184\du}{9.761839\du}}
\pgfpathlineto{\pgfpoint{9.957620\du}{9.761839\du}}
\pgfpathlineto{\pgfpoint{9.957620\du}{8.889987\du}}
\pgfusepath{fill}
\pgfsetbuttcap
\pgfsetmiterjoin
\pgfsetdash{}{0pt}
\definecolor{dialinecolor}{rgb}{0.286275, 0.286275, 0.211765}
\pgfsetstrokecolor{dialinecolor}
\pgfpathmoveto{\pgfpoint{9.957620\du}{8.889987\du}}
\pgfpathlineto{\pgfpoint{11.141184\du}{8.889987\du}}
\pgfpathlineto{\pgfpoint{11.141184\du}{9.761524\du}}
\pgfpathlineto{\pgfpoint{9.957620\du}{9.761524\du}}
\pgfpathlineto{\pgfpoint{9.957620\du}{8.889987\du}}
\pgfusepath{stroke}
\pgfsetbuttcap
\pgfsetmiterjoin
\pgfsetdash{}{0pt}
\definecolor{dialinecolor}{rgb}{0.478431, 0.478431, 0.352941}
\pgfsetfillcolor{dialinecolor}
\pgfpathmoveto{\pgfpoint{11.252960\du}{9.862909\du}}
\pgfpathlineto{\pgfpoint{11.409761\du}{9.706423\du}}
\pgfpathlineto{\pgfpoint{11.409761\du}{8.600000\du}}
\pgfpathlineto{\pgfpoint{11.252960\du}{8.745151\du}}
\pgfpathlineto{\pgfpoint{11.252960\du}{9.862909\du}}
\pgfusepath{fill}
\pgfsetbuttcap
\pgfsetmiterjoin
\pgfsetdash{}{0pt}
\definecolor{dialinecolor}{rgb}{0.286275, 0.286275, 0.211765}
\pgfsetstrokecolor{dialinecolor}
\pgfpathmoveto{\pgfpoint{11.252960\du}{9.862909\du}}
\pgfpathlineto{\pgfpoint{11.409761\du}{9.706423\du}}
\pgfpathlineto{\pgfpoint{11.409761\du}{8.600000\du}}
\pgfpathlineto{\pgfpoint{11.252960\du}{8.745151\du}}
\pgfpathlineto{\pgfpoint{11.252960\du}{9.862909\du}}
\pgfusepath{stroke}
\pgfsetlinewidth{0.100000\du}
\pgfsetdash{}{0pt}
\pgfsetdash{}{0pt}
\pgfsetbuttcap
\pgfsetmiterjoin
\pgfsetlinewidth{0.001000\du}
\pgfsetbuttcap
\pgfsetmiterjoin
\pgfsetdash{}{0pt}
\definecolor{dialinecolor}{rgb}{0.717647, 0.717647, 0.615686}
\pgfsetfillcolor{dialinecolor}
\pgfpathmoveto{\pgfpoint{20.628989\du}{8.998912\du}}
\pgfpathlineto{\pgfpoint{20.628989\du}{10.850000\du}}
\pgfpathlineto{\pgfpoint{21.723081\du}{10.850000\du}}
\pgfpathlineto{\pgfpoint{21.723081\du}{8.998912\du}}
\pgfpathlineto{\pgfpoint{20.628989\du}{8.998912\du}}
\pgfusepath{fill}
\pgfsetbuttcap
\pgfsetmiterjoin
\pgfsetdash{}{0pt}
\definecolor{dialinecolor}{rgb}{0.286275, 0.286275, 0.211765}
\pgfsetstrokecolor{dialinecolor}
\pgfpathmoveto{\pgfpoint{20.628989\du}{8.998912\du}}
\pgfpathlineto{\pgfpoint{20.628989\du}{10.850000\du}}
\pgfpathlineto{\pgfpoint{21.723081\du}{10.850000\du}}
\pgfpathlineto{\pgfpoint{21.723081\du}{8.998912\du}}
\pgfpathlineto{\pgfpoint{20.628989\du}{8.998912\du}}
\pgfusepath{stroke}
\pgfsetbuttcap
\pgfsetmiterjoin
\pgfsetdash{}{0pt}
\definecolor{dialinecolor}{rgb}{0.788235, 0.788235, 0.713726}
\pgfsetfillcolor{dialinecolor}
\pgfpathmoveto{\pgfpoint{20.628989\du}{8.998912\du}}
\pgfpathlineto{\pgfpoint{20.777246\du}{8.850000\du}}
\pgfpathlineto{\pgfpoint{21.871011\du}{8.850000\du}}
\pgfpathlineto{\pgfpoint{21.723081\du}{8.998912\du}}
\pgfpathlineto{\pgfpoint{20.628989\du}{8.998912\du}}
\pgfusepath{fill}
\pgfsetbuttcap
\pgfsetmiterjoin
\pgfsetdash{}{0pt}
\definecolor{dialinecolor}{rgb}{0.286275, 0.286275, 0.211765}
\pgfsetstrokecolor{dialinecolor}
\pgfpathmoveto{\pgfpoint{20.628989\du}{8.998912\du}}
\pgfpathlineto{\pgfpoint{20.777246\du}{8.850000\du}}
\pgfpathlineto{\pgfpoint{21.863811\du}{8.850000\du}}
\pgfusepath{stroke}
\pgfsetbuttcap
\pgfsetmiterjoin
\pgfsetdash{}{0pt}
\definecolor{dialinecolor}{rgb}{0.286275, 0.286275, 0.211765}
\pgfsetstrokecolor{dialinecolor}
\pgfpathmoveto{\pgfpoint{21.863811\du}{8.857527\du}}
\pgfpathlineto{\pgfpoint{21.723081\du}{8.998912\du}}
\pgfpathlineto{\pgfpoint{20.628989\du}{8.998912\du}}
\pgfusepath{stroke}
\pgfsetbuttcap
\pgfsetmiterjoin
\pgfsetdash{}{0pt}
\definecolor{dialinecolor}{rgb}{0.788235, 0.788235, 0.713726}
\pgfsetfillcolor{dialinecolor}
\pgfpathmoveto{\pgfpoint{20.696408\du}{9.106586\du}}
\pgfpathlineto{\pgfpoint{21.196163\du}{9.106586\du}}
\pgfpathlineto{\pgfpoint{21.196163\du}{9.349427\du}}
\pgfpathlineto{\pgfpoint{20.696408\du}{9.349427\du}}
\pgfpathlineto{\pgfpoint{20.696408\du}{9.106586\du}}
\pgfusepath{fill}
\pgfsetbuttcap
\pgfsetmiterjoin
\pgfsetdash{}{0pt}
\definecolor{dialinecolor}{rgb}{0.384314, 0.384314, 0.282353}
\pgfsetstrokecolor{dialinecolor}
\pgfpathmoveto{\pgfpoint{20.696408\du}{9.106586\du}}
\pgfpathlineto{\pgfpoint{21.195835\du}{9.106586\du}}
\pgfpathlineto{\pgfpoint{21.195835\du}{9.349100\du}}
\pgfpathlineto{\pgfpoint{20.696408\du}{9.349100\du}}
\pgfpathlineto{\pgfpoint{20.696408\du}{9.106586\du}}
\pgfusepath{stroke}
\pgfsetlinewidth{0.030000\du}
\pgfsetbuttcap
\pgfsetmiterjoin
\pgfsetdash{}{0pt}
\definecolor{dialinecolor}{rgb}{0.925490, 0.925490, 0.905882}
\pgfsetstrokecolor{dialinecolor}
\pgfpathmoveto{\pgfpoint{20.763828\du}{9.228334\du}}
\pgfpathlineto{\pgfpoint{21.114343\du}{9.228334\du}}
\pgfusepath{stroke}
\pgfsetlinewidth{0.001000\du}
\pgfsetbuttcap
\pgfsetmiterjoin
\pgfsetdash{}{0pt}
\definecolor{dialinecolor}{rgb}{0.478431, 0.478431, 0.352941}
\pgfsetfillcolor{dialinecolor}
\pgfpathmoveto{\pgfpoint{21.723081\du}{10.850000\du}}
\pgfpathlineto{\pgfpoint{21.871011\du}{10.700761\du}}
\pgfpathlineto{\pgfpoint{21.871011\du}{8.850000\du}}
\pgfpathlineto{\pgfpoint{21.723081\du}{8.998912\du}}
\pgfpathlineto{\pgfpoint{21.723081\du}{10.850000\du}}
\pgfusepath{fill}
\pgfsetbuttcap
\pgfsetmiterjoin
\pgfsetdash{}{0pt}
\definecolor{dialinecolor}{rgb}{0.286275, 0.286275, 0.211765}
\pgfsetstrokecolor{dialinecolor}
\pgfpathmoveto{\pgfpoint{21.723081\du}{10.850000\du}}
\pgfpathlineto{\pgfpoint{21.863811\du}{10.708288\du}}
\pgfusepath{stroke}
\pgfsetbuttcap
\pgfsetmiterjoin
\pgfsetdash{}{0pt}
\definecolor{dialinecolor}{rgb}{0.286275, 0.286275, 0.211765}
\pgfsetstrokecolor{dialinecolor}
\pgfpathmoveto{\pgfpoint{21.863811\du}{8.857527\du}}
\pgfpathlineto{\pgfpoint{21.723081\du}{8.998912\du}}
\pgfpathlineto{\pgfpoint{21.723081\du}{10.850000\du}}
\pgfusepath{stroke}
\pgfsetlinewidth{0.030000\du}
\pgfsetbuttcap
\pgfsetmiterjoin
\pgfsetdash{}{0pt}
\definecolor{dialinecolor}{rgb}{0.925490, 0.925490, 0.905882}
\pgfsetstrokecolor{dialinecolor}
\pgfpathmoveto{\pgfpoint{20.642734\du}{10.727925\du}}
\pgfpathlineto{\pgfpoint{21.722754\du}{10.727925\du}}
\pgfusepath{stroke}
\pgfsetbuttcap
\pgfsetmiterjoin
\pgfsetdash{}{0pt}
\definecolor{dialinecolor}{rgb}{0.000000, 0.000000, 0.000000}
\pgfsetstrokecolor{dialinecolor}
\pgfpathmoveto{\pgfpoint{20.642734\du}{9.741834\du}}
\pgfpathlineto{\pgfpoint{21.722754\du}{9.741834\du}}
\pgfusepath{stroke}
\pgfsetbuttcap
\pgfsetmiterjoin
\pgfsetdash{}{0pt}
\definecolor{dialinecolor}{rgb}{0.286275, 0.286275, 0.211765}
\pgfsetstrokecolor{dialinecolor}
\pgfpathmoveto{\pgfpoint{20.628989\du}{10.714507\du}}
\pgfpathlineto{\pgfpoint{21.721772\du}{10.714507\du}}
\pgfusepath{stroke}
\pgfsetbuttcap
\pgfsetmiterjoin
\pgfsetdash{}{0pt}
\definecolor{dialinecolor}{rgb}{0.000000, 0.000000, 0.000000}
\pgfsetstrokecolor{dialinecolor}
\pgfpathmoveto{\pgfpoint{20.628989\du}{9.728089\du}}
\pgfpathlineto{\pgfpoint{21.721772\du}{9.728089\du}}
\pgfusepath{stroke}
\pgfsetlinewidth{0.001000\du}
\pgfsetbuttcap
\pgfsetmiterjoin
\pgfsetdash{}{0pt}
\definecolor{dialinecolor}{rgb}{0.925490, 0.925490, 0.905882}
\pgfsetstrokecolor{dialinecolor}
\pgfpathmoveto{\pgfpoint{20.696408\du}{9.336336\du}}
\pgfpathlineto{\pgfpoint{20.696408\du}{9.106586\du}}
\pgfpathlineto{\pgfpoint{21.182417\du}{9.106586\du}}
\pgfusepath{stroke}
\pgfsetlinewidth{0.100000\du}
\pgfsetdash{}{0pt}
\pgfsetdash{}{0pt}
\pgfsetbuttcap
\pgfsetmiterjoin
\pgfsetlinewidth{0.001000\du}
\pgfsetbuttcap
\pgfsetmiterjoin
\pgfsetdash{}{0pt}
\definecolor{dialinecolor}{rgb}{0.788235, 0.788235, 0.713726}
\pgfsetfillcolor{dialinecolor}
\pgfpathmoveto{\pgfpoint{22.752053\du}{13.227519\du}}
\pgfpathlineto{\pgfpoint{22.999199\du}{13.000000\du}}
\pgfpathlineto{\pgfpoint{25.247947\du}{13.000000\du}}
\pgfpathlineto{\pgfpoint{25.000401\du}{13.227519\du}}
\pgfpathlineto{\pgfpoint{22.752053\du}{13.227519\du}}
\pgfusepath{fill}
\pgfsetbuttcap
\pgfsetmiterjoin
\pgfsetdash{}{0pt}
\definecolor{dialinecolor}{rgb}{0.286275, 0.286275, 0.211765}
\pgfsetstrokecolor{dialinecolor}
\pgfpathmoveto{\pgfpoint{22.763669\du}{13.217104\du}}
\pgfpathlineto{\pgfpoint{22.989986\du}{13.008412\du}}
\pgfpathlineto{\pgfpoint{25.239135\du}{13.008412\du}}
\pgfpathlineto{\pgfpoint{25.000401\du}{13.227519\du}}
\pgfpathlineto{\pgfpoint{22.763669\du}{13.227519\du}}
\pgfpathlineto{\pgfpoint{22.763669\du}{13.217104\du}}
\pgfusepath{stroke}
\pgfsetbuttcap
\pgfsetmiterjoin
\pgfsetdash{}{0pt}
\definecolor{dialinecolor}{rgb}{0.717647, 0.717647, 0.615686}
\pgfsetfillcolor{dialinecolor}
\pgfpathmoveto{\pgfpoint{22.752053\du}{13.227519\du}}
\pgfpathlineto{\pgfpoint{25.018426\du}{13.227519\du}}
\pgfpathlineto{\pgfpoint{25.018426\du}{15.000000\du}}
\pgfpathlineto{\pgfpoint{22.752053\du}{15.000000\du}}
\pgfpathlineto{\pgfpoint{22.752053\du}{13.227519\du}}
\pgfusepath{fill}
\pgfsetbuttcap
\pgfsetmiterjoin
\pgfsetdash{}{0pt}
\definecolor{dialinecolor}{rgb}{0.286275, 0.286275, 0.211765}
\pgfsetstrokecolor{dialinecolor}
\pgfpathmoveto{\pgfpoint{22.763669\du}{13.227519\du}}
\pgfpathlineto{\pgfpoint{25.017625\du}{13.227519\du}}
\pgfpathlineto{\pgfpoint{25.017625\du}{14.999199\du}}
\pgfpathlineto{\pgfpoint{22.763669\du}{14.999199\du}}
\pgfpathlineto{\pgfpoint{22.763669\du}{13.227519\du}}
\pgfusepath{stroke}
\pgfsetbuttcap
\pgfsetmiterjoin
\pgfsetdash{}{0pt}
\definecolor{dialinecolor}{rgb}{0.478431, 0.478431, 0.352941}
\pgfsetfillcolor{dialinecolor}
\pgfpathmoveto{\pgfpoint{25.000401\du}{14.982776\du}}
\pgfpathlineto{\pgfpoint{25.247947\du}{14.736832\du}}
\pgfpathlineto{\pgfpoint{25.247947\du}{13.000000\du}}
\pgfpathlineto{\pgfpoint{25.000401\du}{13.227519\du}}
\pgfpathlineto{\pgfpoint{25.000401\du}{14.982776\du}}
\pgfusepath{fill}
\pgfsetbuttcap
\pgfsetmiterjoin
\pgfsetdash{}{0pt}
\definecolor{dialinecolor}{rgb}{0.286275, 0.286275, 0.211765}
\pgfsetstrokecolor{dialinecolor}
\pgfpathmoveto{\pgfpoint{25.000401\du}{14.982776\du}}
\pgfpathlineto{\pgfpoint{25.247947\du}{14.736832\du}}
\pgfpathlineto{\pgfpoint{25.247947\du}{13.008412\du}}
\pgfpathlineto{\pgfpoint{25.239135\du}{13.008412\du}}
\pgfpathlineto{\pgfpoint{25.000401\du}{13.227519\du}}
\pgfpathlineto{\pgfpoint{25.000401\du}{14.982776\du}}
\pgfpathlineto{\pgfpoint{25.000401\du}{14.982776\du}}
\pgfusepath{stroke}
\pgfsetlinewidth{0.050000\du}
\pgfsetdash{}{0pt}
\pgfsetdash{}{0pt}
\pgfsetbuttcap
{
\definecolor{dialinecolor}{rgb}{0.000000, 0.000000, 0.000000}
\pgfsetfillcolor{dialinecolor}
% was here!!!
\pgfsetarrowsend{to}
\definecolor{dialinecolor}{rgb}{0.000000, 0.000000, 0.000000}
\pgfsetstrokecolor{dialinecolor}
\draw (11.755164\du,9.918640\du)--(20.628989\du,9.924456\du);
}
\pgfsetlinewidth{0.100000\du}
\pgfsetdash{}{0pt}
\pgfsetdash{}{0pt}
\pgfsetbuttcap
{
\definecolor{dialinecolor}{rgb}{0.000000, 0.000000, 0.000000}
\pgfsetfillcolor{dialinecolor}
% was here!!!
\pgfsetarrowsend{to}
\definecolor{dialinecolor}{rgb}{0.000000, 0.000000, 0.000000}
\pgfsetstrokecolor{dialinecolor}
\draw (21.870751\du,10.650831\du)--(23.500000\du,12.750000\du);
}
\pgfsetlinewidth{0.050000\du}
\pgfsetdash{}{0pt}
\pgfsetdash{}{0pt}
\pgfsetbuttcap
{
\definecolor{dialinecolor}{rgb}{0.000000, 0.000000, 0.000000}
\pgfsetfillcolor{dialinecolor}
% was here!!!
\pgfsetarrowsend{to}
\definecolor{dialinecolor}{rgb}{0.000000, 0.000000, 0.000000}
\pgfsetstrokecolor{dialinecolor}
\draw (22.757095\du,13.994376\du)--(17.400000\du,14.000000\du);
}
\pgfsetlinewidth{0.050000\du}
\pgfsetdash{}{0pt}
\pgfsetdash{}{0pt}
\pgfsetbuttcap
\pgfsetmiterjoin
\pgfsetlinewidth{0.050000\du}
\pgfsetbuttcap
\pgfsetmiterjoin
\pgfsetdash{}{0pt}
\definecolor{dialinecolor}{rgb}{1.000000, 1.000000, 1.000000}
\pgfsetfillcolor{dialinecolor}
\fill (14.682258\du,12.900000\du)--(14.682258\du,15.605000\du)--(17.300000\du,15.605000\du)--(17.300000\du,12.900000\du)--cycle;
\definecolor{dialinecolor}{rgb}{0.000000, 0.000000, 0.000000}
\pgfsetstrokecolor{dialinecolor}
\draw (14.682258\du,12.900000\du)--(14.682258\du,15.605000\du)--(17.300000\du,15.605000\du)--(17.300000\du,12.900000\du)--cycle;
\pgfsetbuttcap
\pgfsetmiterjoin
\pgfsetdash{}{0pt}
\definecolor{dialinecolor}{rgb}{0.000000, 0.000000, 0.000000}
\pgfsetstrokecolor{dialinecolor}
\draw (14.682258\du,12.900000\du)--(14.682258\du,15.605000\du)--(17.300000\du,15.605000\du)--(17.300000\du,12.900000\du)--cycle;
% setfont left to latex
\definecolor{dialinecolor}{rgb}{0.000000, 0.000000, 0.000000}
\pgfsetstrokecolor{dialinecolor}
\node[anchor=west] at (15.239919\du,14.659583\du){html};
\pgfsetlinewidth{0.050000\du}
\pgfsetdash{}{0pt}
\pgfsetdash{}{0pt}
\pgfsetbuttcap
{
\definecolor{dialinecolor}{rgb}{0.000000, 0.000000, 0.000000}
\pgfsetfillcolor{dialinecolor}
% was here!!!
\pgfsetarrowsend{to}
\definecolor{dialinecolor}{rgb}{0.000000, 0.000000, 0.000000}
\pgfsetstrokecolor{dialinecolor}
\draw (14.682258\du,14.252500\du)--(10.500000\du,10.750000\du);
}
% setfont left to latex
\definecolor{dialinecolor}{rgb}{0.000000, 0.000000, 0.000000}
\pgfsetstrokecolor{dialinecolor}
\node[anchor=west] at (8.950000\du,14.350000\du){Navigateur};
% setfont left to latex
\definecolor{dialinecolor}{rgb}{0.000000, 0.000000, 0.000000}
\pgfsetstrokecolor{dialinecolor}
\node[anchor=west] at (10.800000\du,7.600000\du){Client};
% setfont left to latex
\definecolor{dialinecolor}{rgb}{0.000000, 0.000000, 0.000000}
\pgfsetstrokecolor{dialinecolor}
\node[anchor=west] at (20.250000\du,7.800000\du){Serveur};
% setfont left to latex
\definecolor{dialinecolor}{rgb}{0.000000, 0.000000, 0.000000}
\pgfsetstrokecolor{dialinecolor}
\node[anchor=west] at (22.950000\du,15.500000\du){Apache};
% setfont left to latex
\definecolor{dialinecolor}{rgb}{0.000000, 0.000000, 0.000000}
\pgfsetstrokecolor{dialinecolor}
\node[anchor=west] at (19.400000\du,15.200000\du){PHP};
\end{tikzpicture}
 \end{center}
	\subsection{Dormir spécifique à tout équidé}
	Méthode retardée
	% Graphic for TeX using PGF
% Title: /home/satenske/cours/pointeur.dia
% Creator: Dia v0.97.1
% CreationDate: Wed Feb  2 19:10:48 2011
% For: satenske
% \usepackage{tikz}
% The following commands are not supported in PSTricks at present
% We define them conditionally, so when they are implemented,
% this pgf file will use them.
\ifx\du\undefined
  \newlength{\du}
\fi
\setlength{\du}{15\unitlength}
\begin{tikzpicture}
\pgftransformxscale{1.000000}
\pgftransformyscale{-1.000000}
\definecolor{dialinecolor}{rgb}{0.000000, 0.000000, 0.000000}
\pgfsetstrokecolor{dialinecolor}
\definecolor{dialinecolor}{rgb}{1.000000, 1.000000, 1.000000}
\pgfsetfillcolor{dialinecolor}
\definecolor{dialinecolor}{rgb}{1.000000, 1.000000, 1.000000}
\pgfsetfillcolor{dialinecolor}
\fill (1.824422\du,5.950672\du)--(1.824422\du,9.517078\du)--(10.066610\du,9.517078\du)--(10.066610\du,5.950672\du)--cycle;
\pgfsetlinewidth{0.050000\du}
\pgfsetdash{}{0pt}
\pgfsetdash{}{0pt}
\pgfsetmiterjoin
\definecolor{dialinecolor}{rgb}{0.000000, 0.000000, 0.000000}
\pgfsetstrokecolor{dialinecolor}
\draw (1.824422\du,5.950672\du)--(1.824422\du,9.517078\du)--(10.066610\du,9.517078\du)--(10.066610\du,5.950672\du)--cycle;
% setfont left to latex
\definecolor{dialinecolor}{rgb}{0.000000, 0.000000, 0.000000}
\pgfsetstrokecolor{dialinecolor}
\node at (5.945516\du,7.928875\du){};
\definecolor{dialinecolor}{rgb}{1.000000, 1.000000, 1.000000}
\pgfsetfillcolor{dialinecolor}
\fill (3.040672\du,6.556141\du)--(3.040672\du,8.556141\du)--(5.108172\du,8.556141\du)--(5.108172\du,6.556141\du)--cycle;
\pgfsetlinewidth{0.050000\du}
\pgfsetdash{}{0pt}
\pgfsetdash{}{0pt}
\pgfsetmiterjoin
\definecolor{dialinecolor}{rgb}{0.000000, 0.000000, 0.000000}
\pgfsetstrokecolor{dialinecolor}
\draw (3.040672\du,6.556141\du)--(3.040672\du,8.556141\du)--(5.108172\du,8.556141\du)--(5.108172\du,6.556141\du)--cycle;
% setfont left to latex
\definecolor{dialinecolor}{rgb}{0.000000, 0.000000, 0.000000}
\pgfsetstrokecolor{dialinecolor}
\node at (4.074422\du,7.751141\du){0.0};
\definecolor{dialinecolor}{rgb}{1.000000, 1.000000, 1.000000}
\pgfsetfillcolor{dialinecolor}
\fill (7.012938\du,6.600672\du)--(7.012938\du,8.600672\du)--(9.080438\du,8.600672\du)--(9.080438\du,6.600672\du)--cycle;
\pgfsetlinewidth{0.050000\du}
\pgfsetdash{}{0pt}
\pgfsetdash{}{0pt}
\pgfsetmiterjoin
\definecolor{dialinecolor}{rgb}{0.000000, 0.000000, 0.000000}
\pgfsetstrokecolor{dialinecolor}
\draw (7.012938\du,6.600672\du)--(7.012938\du,8.600672\du)--(9.080438\du,8.600672\du)--(9.080438\du,6.600672\du)--cycle;
% setfont left to latex
\definecolor{dialinecolor}{rgb}{0.000000, 0.000000, 0.000000}
\pgfsetstrokecolor{dialinecolor}
\node at (8.046688\du,7.795672\du){1.0};
\end{tikzpicture}

	\lstinputlisting[language=java, caption=Classe Equide]{Equide.java}	
	\lstinputlisting[language=java, caption=Classe Cheval]{Cheval.java}	
	\lstinputlisting[language=java, caption=Classe Ane]{Ane.java}	
	\lstinputlisting[language=java, caption=Classe Mulet]{Mulet.java}	
	\newpage
	\section{Le papillon}
	\begin{center} % Graphic for TeX using PGF
% Title: /home/satenske/cours/tableau.dia
% Creator: Dia v0.97.1
% CreationDate: Wed Feb  2 18:56:27 2011
% For: satenske
% \usepackage{tikz}
% The following commands are not supported in PSTricks at present
% We define them conditionally, so when they are implemented,
% this pgf file will use them.
\ifx\du\undefined
  \newlength{\du}
\fi
\setlength{\du}{15\unitlength}
\begin{tikzpicture}
\pgftransformxscale{1.000000}
\pgftransformyscale{-1.000000}
\definecolor{dialinecolor}{rgb}{0.000000, 0.000000, 0.000000}
\pgfsetstrokecolor{dialinecolor}
\definecolor{dialinecolor}{rgb}{1.000000, 1.000000, 1.000000}
\pgfsetfillcolor{dialinecolor}
\definecolor{dialinecolor}{rgb}{1.000000, 1.000000, 1.000000}
\pgfsetfillcolor{dialinecolor}
\fill (0.702473\du,7.291250\du)--(0.702473\du,9.291250\du)--(12.787600\du,9.291250\du)--(12.787600\du,7.291250\du)--cycle;
\pgfsetlinewidth{0.050000\du}
\pgfsetdash{}{0pt}
\pgfsetdash{}{0pt}
\pgfsetmiterjoin
\definecolor{dialinecolor}{rgb}{0.000000, 0.000000, 0.000000}
\pgfsetstrokecolor{dialinecolor}
\draw (0.702473\du,7.291250\du)--(0.702473\du,9.291250\du)--(12.787600\du,9.291250\du)--(12.787600\du,7.291250\du)--cycle;
% setfont left to latex
\definecolor{dialinecolor}{rgb}{0.000000, 0.000000, 0.000000}
\pgfsetstrokecolor{dialinecolor}
\node at (6.745036\du,8.486250\du){};
\definecolor{dialinecolor}{rgb}{1.000000, 1.000000, 1.000000}
\pgfsetfillcolor{dialinecolor}
\fill (0.674213\du,7.297500\du)--(0.674213\du,9.297500\du)--(2.550100\du,9.297500\du)--(2.550100\du,7.297500\du)--cycle;
\pgfsetlinewidth{0.050000\du}
\pgfsetdash{}{0pt}
\pgfsetdash{}{0pt}
\pgfsetmiterjoin
\definecolor{dialinecolor}{rgb}{0.000000, 0.000000, 0.000000}
\pgfsetstrokecolor{dialinecolor}
\draw (0.674213\du,7.297500\du)--(0.674213\du,9.297500\du)--(2.550100\du,9.297500\du)--(2.550100\du,7.297500\du)--cycle;
% setfont left to latex
\definecolor{dialinecolor}{rgb}{0.000000, 0.000000, 0.000000}
\pgfsetstrokecolor{dialinecolor}
\node at (1.612156\du,8.492500\du){L};
\definecolor{dialinecolor}{rgb}{1.000000, 1.000000, 1.000000}
\pgfsetfillcolor{dialinecolor}
\fill (2.546350\du,7.295000\du)--(2.546350\du,9.295000\du)--(4.546350\du,9.295000\du)--(4.546350\du,7.295000\du)--cycle;
\pgfsetlinewidth{0.050000\du}
\pgfsetdash{}{0pt}
\pgfsetdash{}{0pt}
\pgfsetmiterjoin
\definecolor{dialinecolor}{rgb}{0.000000, 0.000000, 0.000000}
\pgfsetstrokecolor{dialinecolor}
\draw (2.546350\du,7.295000\du)--(2.546350\du,9.295000\du)--(4.546350\du,9.295000\du)--(4.546350\du,7.295000\du)--cycle;
% setfont left to latex
\definecolor{dialinecolor}{rgb}{0.000000, 0.000000, 0.000000}
\pgfsetstrokecolor{dialinecolor}
\node at (3.546350\du,8.490000\du){E};
\definecolor{dialinecolor}{rgb}{1.000000, 1.000000, 1.000000}
\pgfsetfillcolor{dialinecolor}
\fill (4.580933\du,7.290625\du)--(4.580933\du,9.290625\du)--(16.805933\du,9.290625\du)--(16.805933\du,7.290625\du)--cycle;
\pgfsetlinewidth{0.050000\du}
\pgfsetdash{}{0pt}
\pgfsetdash{}{0pt}
\pgfsetmiterjoin
\definecolor{dialinecolor}{rgb}{0.000000, 0.000000, 0.000000}
\pgfsetstrokecolor{dialinecolor}
\draw (4.580933\du,7.290625\du)--(4.580933\du,9.290625\du)--(16.805933\du,9.290625\du)--(16.805933\du,7.290625\du)--cycle;
% setfont left to latex
\definecolor{dialinecolor}{rgb}{0.000000, 0.000000, 0.000000}
\pgfsetstrokecolor{dialinecolor}
\node at (10.693433\du,8.485625\du){...};
\definecolor{dialinecolor}{rgb}{1.000000, 1.000000, 1.000000}
\pgfsetfillcolor{dialinecolor}
\fill (4.568433\du,7.296875\du)--(4.568433\du,9.296875\du)--(6.568433\du,9.296875\du)--(6.568433\du,7.296875\du)--cycle;
\pgfsetlinewidth{0.050000\du}
\pgfsetdash{}{0pt}
\pgfsetdash{}{0pt}
\pgfsetmiterjoin
\definecolor{dialinecolor}{rgb}{0.000000, 0.000000, 0.000000}
\pgfsetstrokecolor{dialinecolor}
\draw (4.568433\du,7.296875\du)--(4.568433\du,9.296875\du)--(6.568433\du,9.296875\du)--(6.568433\du,7.296875\du)--cycle;
% setfont left to latex
\definecolor{dialinecolor}{rgb}{0.000000, 0.000000, 0.000000}
\pgfsetstrokecolor{dialinecolor}
\node at (5.568433\du,8.491875\du){};
\definecolor{dialinecolor}{rgb}{1.000000, 1.000000, 1.000000}
\pgfsetfillcolor{dialinecolor}
\fill (6.564683\du,7.294375\du)--(6.564683\du,9.294375\du)--(8.564683\du,9.294375\du)--(8.564683\du,7.294375\du)--cycle;
\pgfsetlinewidth{0.050000\du}
\pgfsetdash{}{0pt}
\pgfsetdash{}{0pt}
\pgfsetmiterjoin
\definecolor{dialinecolor}{rgb}{0.000000, 0.000000, 0.000000}
\pgfsetstrokecolor{dialinecolor}
\draw (6.564683\du,7.294375\du)--(6.564683\du,9.294375\du)--(8.564683\du,9.294375\du)--(8.564683\du,7.294375\du)--cycle;
% setfont left to latex
\definecolor{dialinecolor}{rgb}{0.000000, 0.000000, 0.000000}
\pgfsetstrokecolor{dialinecolor}
\node at (7.564683\du,8.489375\du){T};
\definecolor{dialinecolor}{rgb}{1.000000, 1.000000, 1.000000}
\pgfsetfillcolor{dialinecolor}
\fill (12.838116\du,7.291250\du)--(12.838116\du,9.291250\du)--(14.838116\du,9.291250\du)--(14.838116\du,7.291250\du)--cycle;
\pgfsetlinewidth{0.050000\du}
\pgfsetdash{}{0pt}
\pgfsetdash{}{0pt}
\pgfsetmiterjoin
\definecolor{dialinecolor}{rgb}{0.000000, 0.000000, 0.000000}
\pgfsetstrokecolor{dialinecolor}
\draw (12.838116\du,7.291250\du)--(12.838116\du,9.291250\du)--(14.838116\du,9.291250\du)--(14.838116\du,7.291250\du)--cycle;
% setfont left to latex
\definecolor{dialinecolor}{rgb}{0.000000, 0.000000, 0.000000}
\pgfsetstrokecolor{dialinecolor}
\node at (13.838116\du,8.486250\du){};
\definecolor{dialinecolor}{rgb}{1.000000, 1.000000, 1.000000}
\pgfsetfillcolor{dialinecolor}
\fill (14.834366\du,7.288750\du)--(14.834366\du,9.288750\du)--(16.834366\du,9.288750\du)--(16.834366\du,7.288750\du)--cycle;
\pgfsetlinewidth{0.050000\du}
\pgfsetdash{}{0pt}
\pgfsetdash{}{0pt}
\pgfsetmiterjoin
\definecolor{dialinecolor}{rgb}{0.000000, 0.000000, 0.000000}
\pgfsetstrokecolor{dialinecolor}
\draw (14.834366\du,7.288750\du)--(14.834366\du,9.288750\du)--(16.834366\du,9.288750\du)--(16.834366\du,7.288750\du)--cycle;
% setfont left to latex
\definecolor{dialinecolor}{rgb}{0.000000, 0.000000, 0.000000}
\pgfsetstrokecolor{dialinecolor}
\node at (15.834366\du,8.483750\du){};
% setfont left to latex
\definecolor{dialinecolor}{rgb}{0.000000, 0.000000, 0.000000}
\pgfsetstrokecolor{dialinecolor}
\node[anchor=west] at (3.325129\du,6.068906\du){};
% setfont left to latex
\definecolor{dialinecolor}{rgb}{0.000000, 0.000000, 0.000000}
\pgfsetstrokecolor{dialinecolor}
\node[anchor=west] at (3.475129\du,6.218906\du){};
% setfont left to latex
\definecolor{dialinecolor}{rgb}{0.000000, 0.000000, 0.000000}
\pgfsetstrokecolor{dialinecolor}
\node[anchor=west] at (1.472860\du,6.860828\du){1};
% setfont left to latex
\definecolor{dialinecolor}{rgb}{0.000000, 0.000000, 0.000000}
\pgfsetstrokecolor{dialinecolor}
\node[anchor=west] at (3.347860\du,6.841297\du){2};
% setfont left to latex
\definecolor{dialinecolor}{rgb}{0.000000, 0.000000, 0.000000}
\pgfsetstrokecolor{dialinecolor}
\node[anchor=west] at (5.340047\du,6.860828\du){3};
% setfont left to latex
\definecolor{dialinecolor}{rgb}{0.000000, 0.000000, 0.000000}
\pgfsetstrokecolor{dialinecolor}
\node[anchor=west] at (7.254110\du,6.821766\du){4};
% setfont left to latex
\definecolor{dialinecolor}{rgb}{0.000000, 0.000000, 0.000000}
\pgfsetstrokecolor{dialinecolor}
\node[anchor=west] at (10.144735\du,6.704578\du){...};
% setfont left to latex
\definecolor{dialinecolor}{rgb}{0.000000, 0.000000, 0.000000}
\pgfsetstrokecolor{dialinecolor}
\node[anchor=west] at (13.211141\du,6.802234\du){999};
% setfont left to latex
\definecolor{dialinecolor}{rgb}{0.000000, 0.000000, 0.000000}
\pgfsetstrokecolor{dialinecolor}
\node[anchor=west] at (14.988485\du,6.782703\du){1000};
\pgfsetlinewidth{0.050000\du}
\pgfsetdash{}{0pt}
\pgfsetdash{}{0pt}
\pgfsetbuttcap
{
\definecolor{dialinecolor}{rgb}{0.000000, 0.000000, 0.000000}
\pgfsetfillcolor{dialinecolor}
% was here!!!
\definecolor{dialinecolor}{rgb}{0.000000, 0.000000, 0.000000}
\pgfsetstrokecolor{dialinecolor}
\pgfpathmoveto{\pgfpoint{0.983703\du}{9.438583\du}}
\pgfpatharc{113}{68}{13.147676\du and 13.147676\du}
\pgfusepath{stroke}
}
\pgfsetlinewidth{0.050000\du}
\pgfsetdash{}{0pt}
\pgfsetdash{}{0pt}
\pgfsetbuttcap
{
\definecolor{dialinecolor}{rgb}{0.000000, 0.000000, 0.000000}
\pgfsetfillcolor{dialinecolor}
% was here!!!
\definecolor{dialinecolor}{rgb}{0.000000, 0.000000, 0.000000}
\pgfsetstrokecolor{dialinecolor}
\pgfpathmoveto{\pgfpoint{11.179738\du}{9.555982\du}}
\pgfpatharc{137}{46}{3.362644\du and 3.362644\du}
\pgfusepath{stroke}
}
% setfont left to latex
\definecolor{dialinecolor}{rgb}{0.000000, 0.000000, 0.000000}
\pgfsetstrokecolor{dialinecolor}
\node[anchor=west] at (4.109579\du,11.079578\du){Partie utilisée};
% setfont left to latex
\definecolor{dialinecolor}{rgb}{0.000000, 0.000000, 0.000000}
\pgfsetstrokecolor{dialinecolor}
\node[anchor=west] at (12.918172\du,11.255359\du){inutilisée};
\end{tikzpicture}
 \end{center}
	\lstinputlisting[language=java, caption=Classe Papillon]{Papillon.java}	
	\lstinputlisting[language=java, caption=Classe Stade]{Stade.java}	
	\lstinputlisting[language=java, caption=Classe Chenille]{Chenille.java}	
	\newpage
	\lstinputlisting[language=java, caption=Classe Chrysalide]{Chrysalide.java}	
	\lstinputlisting[language=java, caption=Classe Lepidoptere]{Lepidoptere.java}	
	\newpage
	\section{Le zoo}
	\lstinputlisting[language=java, caption=Interface Iterateur]{Iterateur.java}	
	\lstinputlisting[language=java, caption=Méthode endormirAnimaux]{Zoo.java}	
	\paragraph{Remarque Importante} 
	\begin{itemize}
		\item Le code est indépendant du nombre d'animaux du zoo
		\item Le code est indépendant du nombre d'espèce du zoo
		\item Le code est indépendant d'ajout ou de retrait de nouvelle espèce 
		\item Le code est indépendant de la représentation du zoo (ici un ensemble)
	\end{itemize}
	\newpage
	\section{Le logis des animaux}
	\lstinputlisting[language=java, caption=Constructeur Animal]{constructeurAnimal.java}	
	\lstinputlisting[language=java, caption=Classe Ours]{Ours.java}	
	\lstinputlisting[language=java, caption=Classe OursBrun]{OursBrun.java}	
	\lstinputlisting[language=java, caption=Classe OursBlanc]{OursBlanc.java}	
	\lstinputlisting[language=java, caption=Knut est un ours blanc du zoo]{knut.java}	
\end{document}

