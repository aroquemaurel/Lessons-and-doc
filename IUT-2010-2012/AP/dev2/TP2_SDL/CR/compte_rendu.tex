\documentclass[12pt,a4paper,openany]{article}

\usepackage{lmodern}
\usepackage{xcolor}
\usepackage[utf8]{inputenc}
\usepackage[T1]{fontenc}
\usepackage[francais]{babel}
\usepackage[top=1.7cm, bottom=1.7cm, left=1.7cm, right=1.7cm]{geometry}
%\usepackage[frenchb]{babel}
%\usepackage{graphicx}
%\usepackage{layout}
%\usepackage{setspace}
%\usepackage{soul}
%\usepackage{ulem}
%\usepackage{eurosym}
%\usepackage{bookman}
%\usepackage{charter}
%\usepackage{newcent}
%\usepackage{lmodern}
%\usepackage{mathpazo}
%\usepackage{mathptmx}
%\usepackage{url}
%\usepackage{verbatim}
%\usepackage{moreverb}
%\usepackage{wrapfig}
%\usepackage{amsmath}
%\usepackage{mathrsfs}
%\usepackage{asmthm}
%\usepackage{makeidx}
%\usepackage{tikz} %Vectoriel
\usepackage{listings}
\usepackage{fancyhdr}
\usepackage{multido}
\usepackage{amssymb}


\definecolor{gris1}{gray}{0.40}
\definecolor{gris2}{gray}{0.55}
\definecolor{gris3}{gray}{0.65}
\definecolor{gris4}{gray}{0.50}


\lstdefinelanguage{algo}{%
   morekeywords={%
    %%% couleur 1
		importer, programme, glossaire, fonction, procedure, constante, type, 
	%%% IMPORT & Co.
		si, sinon, alors, fin, tantque, debut, faire, lorsque, fin lorsque, declancher, enregistrement, tableau, retourne, retourner, =, /=, <, >, traite,exception, 
	%%% types 
		Entier, Reel, Booleen, Caractere,
	%%% types 
		entree, maj, sortie,	
	%%% types 
		et, ou, non,
	},
  sensitive=true,
  morecomment=[l]{--},
  morestring=[b]',
}

%\lstset{language=algo,
    %%% BOUCLE, TEST & Co.
%      emph={importer, programme, glossaire, fonction, procedure, constante, type},
%      emphstyle=\color{gris2},
    %%% IMPORT & Co.
%      emph={[2]si, sinon, alors, fin , tantque, debut, faire, lorsque, fin lorsque, declancher, retourner, et, ou, non,enregistrement, retourner, retourne, tableau, /=, <, =, >, traite,exception},
%      emphstyle=[2]\color{gris1},
    %%% FONCTIONS NUMERIQUES
%      emph={[3]Entier, Reel, Booleen, Caractere},
%      emphstyle=[3]\color{gris3},
    %%% FONCTIONS NUMERIQUES
%      emph={[4]entree, maj, sortie},	
%      emphstyle=[4]\color{gris4},
%}
\lstset{ % general style for listings 
   numbers=left 
	, extendedchars=\true
   , tabsize=2 
   , frame=single 
   , breaklines=true 
   , basicstyle=\ttfamily 
   , numberstyle=\tiny\ttfamily 
   , framexleftmargin=13mm 
   , xleftmargin=12mm 
   , captionpos=b 
	, language=algo
	, keywordstyle=\color{blue}
	, commentstyle=\color{green}
	, showstringspaces=false
	, extendedchars=true
	, mathescape=true
} 
 %prise en charge du langage algo

\title{Compte rendu TP2\\ Dessiner un beau dessin}
\date{Développement en C++\\ Semestre 2\\}
\author{}
\lhead{Compte rendu TP2: Dessiner un beau dessin}
\chead{}
\rhead{Valleix - de Roquemaurel (Groupe F)}

\lfoot{Université paul sabatier Toulouse III}
\cfoot{\thepage}
\rfoot{dev2}

\pagestyle{fancy}
\begin{document}
	\maketitle
	De ce TP, nous pouvons retenir qu'un programmeur, ce doit d'être vigilant dans la rédaction de son
	travail. 	
	En effet il ne faut pas se limiter au fait que le programme marche, mais soigner son code, 
	pour qu'il soit compréhensible par les autres, mais également pour nous même ulterieurement.\\
	En effet il peut arriver qu'après coup on ne comprenne plus ce qu'on a voulut faire.\\ \\ \\
	
	Ainsi, voici différentes conventions que nous avons essayé de corriger, cela rend le code plus propre,
	et plus lisible.

	\section{Variable}
		Les variables sont très importantes dans un programme, ainsi, il faut faire attention à leurs
		utilitées, leurs appellations, et leurs types.
		\subsection{Noms}
			Les noms de variables sont extrement important.\\
			 Dans le code fournit par l'IUT, tous les noms 
			de variables s'appellaient $vlx$, c'était catastrophique pour la compréhension du code: \\
			il était impossible de comprendre à quoi servait une variable.
		\subsection{Unicité d'utilité}
			On ne doit pas avoir une variable qui fait plusieurs choses différentes, cela permet d'améliorer
			la compréhension.  
		\subsection{Variables globales}
			Les variables globales sont à proscrire, en effet, elles rendent le code sale, et utilisent
			beaucoup de mémoire, étant donné qu'elles sont disponibles dans tous les sous-programmes, même
			si on ne s'en sert pas.
	\section{Boucles}
		Il est préférable d'utiliser une boucle for lorsque que l'ont doit répéter un nombre d'actions
		connu à l'avance, plutôt que d'utiliser une boucle while. La boucle for est plus lisible, elle 
		permet de déclarer une variable uniquement dans la boucle, et les informations sont condensés, donc
		plus claires, on saura tout de suite qu'une boucle for veut dire que l'on va répéter des actions un
		certain nombre de fois.
	\section{Sous-programmes}
		\begin{center}
		\large{\textit{<< Diviser pour mieux régner >>}}
		\end{center}
		Il faut diviser le code avec le plus de sous-programmes possible: un sous-programme ne doit pas
			dépasser 40 lignes. \\
		En effet il est plus simple de comprendre ce que fait le sous-programme, 
			que de devoir comprendre tout un code, situé dans le programme principal.\\ \\
		Il faut également penser à donner un nom clair à son sous-programme, quitte à ce qu'il soit
		un peu long à écrire.
	\section{Indentation}
		L'indentation d'un code est indispensable: en effet un code bien indenté est beaucoup plus facilement 
		compréhensible, on voit tout de suite quand commence une boucle ou une condition et quand elle
		finit. \\
		Sans indentation, un code est tout simplement illisible.  
	\section{Commentaires}
		\begin{center}
		\large{\textit{<< Si après avoir lu uniquement les commentaires 
			d'un programme vous n'en comprenez pas l'utilité, jetez-le tout! >>}} IBM
		\end{center}

		Les commentaires d'un programmes sont eux aussi utiles, ils servent à aider les programmeurs
		dans le cas ou il relit sont code quelques temps plus tard, ou lorsqu'un autre programmeur lit
		le code, grâce aux commentaires, il compredra le code.\\
		Cependant, il n'est pas utile de commenter toutes les lignes, en effet certaines lignes 
			sont évidente, de plus si les fonctions ont des noms explicites, elles parleront toutes seules! 
%		\lstinputlisting[caption=Code en C++n language=C++]{dessinSDL.cpp}	
			
\end{document}
