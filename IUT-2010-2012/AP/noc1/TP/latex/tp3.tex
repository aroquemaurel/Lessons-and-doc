\documentclass{article}

\usepackage[utf8]{inputenc}
\usepackage[T1]{fontenc}
\usepackage[francais]{babel}
\usepackage[top=2cm, bottom=2cm, left=2cm, right=2cm]{geometry} %marges
\usepackage{fancyhdr}
\usepackage{multido}
\usepackage{amssymb}

\title{TP3\\ Les variables : déclaration et utilisation}
\date{C$++$ de base}
\newcommand{\Pointilles}[1][3]
{
	\multido{}{#1}{\makebox[\linewidth]{\dotfill}\\[\parskip]}
}
\linespread{1.1}

\lhead{TP3 de C$++$ de base}
\chead{}
\rhead{\thepage}

\lfoot{Université paul sabatier}
\cfoot{\thepage}
\rfoot{Toulouse III}

\pagestyle{fancy}
\begin{document}
	\maketitle 
	\section*{Exercice 1}
		Écrire un programme qui permet à l'utilisateur de saisir la largeur et la longueur d'un champ et qui affiche le périmètre et la surface de celui-ci.
		\begin{itemize}
			\item Les variables largeur et longueur seront de type réels double précision
			\item Le calcul du périmètre se fera dans une fonction appelée perimetre, de même que le calcul de la surface se fera dans une fonction appelée surface.
		\end{itemize}
	\section*{Exercice 2}
		Écrire un programme qui demande à l'utilisateur de renseigner 5 entiers et qui affiche leur moyenne.
		\begin{itemize}
			\item Le programme ne devra utiliser que 2 variables.
		\end{itemize}		
	\section*{Exercice 3}
		Écrire un programme qui demande à l'utilisateur de renseigner 2 entiers et qui affiche leur somme.
		\begin{itemize}
			\item Le calcul de la somme se fera dans une procédure appelée calculerSomme ayant les 2 entiers lus en entrée et l'entier résultat en sortie.
		\end{itemize}		
	

\end{document}
