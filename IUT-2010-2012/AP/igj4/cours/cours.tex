\documentclass[12pt,a4paper,openany]{report}

\usepackage{lmodern}
\usepackage{xcolor}
\usepackage[utf8]{inputenc}
\usepackage[T1]{fontenc}
\usepackage[francais]{babel}
\usepackage[top=1.7cm, bottom=1.7cm, left=1.7cm, right=1.7cm]{geometry}
\usepackage{verbatim}
\usepackage{tikz} %Vectoriel
\usepackage{listings}
\usepackage{fancyhdr}
\usepackage{multido}
\usepackage{amssymb}

\newcommand{\titre}{Interfaces Grahiques en Java}

\newcommand{\module}{}
\newcommand{\sigle}{igj}

\newcommand{\semestre}{4}

\input{/home/satenske/cours/listings.tex} %prise en charge du langage algo
\input{/home/satenske/cours/entete_iut-cours.tex}

\begin{document}
	\maketitle
	\section{Généralités}
		Pour convcevoir une interface graphique il faut:
		\begin{itemize}
			\item Disposer de composants graphiques (Fenêtres, boutons, champs, \ldots)
			\item Disposer d'un mécanisme pour agencer les composants graphiques
			\item Associer un traitement à un composant
		\end{itemize}
	\section{Les composants de l'AWT}
	\begin{itemize}
		\item Package \texttt{java.awt}
		\item 3 ensemble de classes
			\begin{itemize}
				\item Les composants simples (Héritant de \texttt{Component}
				\item Les composants de réceptacle (Héritant de \texttt{Container}
				\item Les composants de menu (Héritant de \texttt{MenuComponent}
			\end{itemize}
		\item Possibilité d'imbriquer des composants de réceptacle
			\begin{itemize}
				\item Notion de parenté
			\end{itemize}
	\end{itemize}
	\subsection{Exemple de composant graphique : \texttt{TextField}}
	\begin{tabular}{cc}
		Méthode & Description\\
		\texttt{String getSelectedText()} & Renvoie le texte sélectionné\\
		\texttt{int getSelectionStart()} & Renvoie la position de début de sélection \\
		\texttt{int getSelectionEnd()} & Renvoie la position de fin de séléction\\
		\texttt{String getText()} & Renvoie le texte contenu dans l'objet\\
		\texttt{boolean isEditable()} & Indique si le texte est modifiable \\
		\texttt{void setEditable(boolean b)} & Autoriser ou interdire la modification de texte\\
	\end{tabular}
	\section{Les gestionnaires de mise en page}
	Un gestionnaire de mise en page est un \texttt{LayoutManager}, c'est un objet associé à un \texttt{Container}.
	\subsection{Différents gestionnaires}
	\begin{itemize}
		\item FlowLayout : Ajout les composants les uns après les autres
		\item BorderLayout : Ajoute les composants en fonction d'une position géographique
		\item GridLayout : Ajoute les composants les uns après les autres dans un quadrillage de n lignes et m colonnes\footnote{Taille identique}
		\item GridBagLayout :Ajoute les composants les uns après les autres dans un quadrillage de n lignes et m colonnes\footnote{Taille non identique} 
		\item CardLayout : Permet de positionner des onglets pour visualiser plusieurs feuilles
	\end{itemize}
	\section{La gestion des événements}
	\begin{itemize}
		\item Un objet de type \þexttt{EventObject} est généré par un source à chaque fois qu'un événement se produit.
	\end{itemize}
	\section{Les composants SWING}
	\section{Le contexte graphique}
	\section{Les appelts}
	\section{Les environnements de développements intégrés}
	
\end{document}


