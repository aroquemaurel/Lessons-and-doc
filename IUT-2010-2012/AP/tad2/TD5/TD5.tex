\documentclass{article}

\usepackage{lmodern}
\usepackage{xcolor}
\usepackage[utf8]{inputenc}
\usepackage[T1]{fontenc}
\usepackage[francais]{babel}
\usepackage[top=1.7cm, bottom=1.7cm, left=1.7cm, right=1.7cm]{geometry}
%\usepackage[frenchb]{babel}
%\usepackage{layout}
%\usepackage{setspace}
%\usepackage{soul}
%\usepackage{ulem}
%\usepackage{eurosym}
%\usepackage{bookman}
%\usepackage{charter}
%\usepackage{newcent}
%\usepackage{lmodern}
%\usepackage{mathpazo}
%\usepackage{mathptmx}
%\usepackage{url}
%\usepackage{verbatim}
%\usepackage{moreverb}
%\usepackage{wrapfig}
%\usepackage{amsmath}
%\usepackage{mathrsfs}
%\usepackage{asmthm}
%\usepackage{makeidx}
%\usepackage{tikz} %Vectoriel
\usepackage{listings}
\usepackage{fancyhdr}
\usepackage{multido}
\usepackage{amssymb}


\input{/home/satenske/cours/listings.tex} %prise en charge du langage algo

\title{TD 5\\ Entiers de Guiseppe Peano}
\date{TAD\\ Semestre 2}

\lhead{TD5: Entiers de Guiseppe Peano}
\chead{}
\rhead{\thepage}

\lfoot{Université Paul Sabatier Toulouse III}
\cfoot{\thepage}
\rfoot{tad2}

\pagestyle{fancy}
\begin{document}
	\maketitle
	\section{rôle des différents acteurs}
		\paragraph{Utilisateur}
			Je suis l'utilisateur (ou le client) du TAD je connais les 
			\textbf{en têtes}, les \textbf{propriétés} mais ne connais pas 
			le corps de l'implémentation.
		\paragraph{Concepteur}
			Je suis le concepteur du TAD. Je spécifie les \textbf{opérations}
			du type et les \textbf{propriétés} du type, définie les 
			\textbf{entêtes} des sous programmes et et je dis si l'affectation 
			et la comparaison sont (ou pas) autorisés.
		\paragraph{Programmeur}
			Je suis le programmeur du TAD. J'ai en charge la définition du TAD
			(\textbf{implémentation}) définie la \textbf{représentation mémoire}
			du type. Il code les différents sous programmes du concepteur.

	\section{Spécification fonctionnelle du TAD Peano}
		\begin{tabular}{|c|c|}
			\hline
			\textbf{Peano} & \textbf{Entier} \\
			\hline
			zéro & 0 \\
			\hline
			succ & +1\\
			\hline
			add & +\\
			\hline
			mult & *\\
			\hline
			inf & <\\
			\hline
			egal & =\\
			\hline
		\end{tabular}
		\subsection{}
			\begin{eqnarray*}
				(p1) 0+p&=&p\\
				(p2) (p+1)+q&=&(p+q)+1\\
				(p3) 0*p&=&0\\
				(p4) (p+1)*q=(p*q)+q\\
				(p5) \neg (0 < 0) 
				(p6) \neg 0 < p+1\\
				(p7) \neg (p + 1 < 0)\\
				(p7) p + 1 < q + 1 \equiv p < q \\
				(p8) p+1 < q+1 \equiv p < q\\
				(p9) 0=0\\
				(p10) \neg(0=p+1)\\
				(p11) \neg(p+1=0)\\
				(p12) p+1 = q+1 \equiv p=q\\
			\end{eqnarray*}	
		\subsection{}
			\lstinputlisting{2-1.algo}	
			Utile au client
	\section{Spécification algorithmique du TAD Peano}
		\subsection{}
			\lstinputlisting{3-1.algo}	
		\subsection{}
			En-tête (cf 3.1) + Propriétés (cf sujet, page 2) + en-tête de l'affectation
		\subsection{}
			\lstinputlisting{3-2.algo}	
	\section{Utilisation du TAD Peano}
		\newpage
		\subsection{}
			\lstinputlisting{4-1.algo}	
		\subsection{}
			\lstinputlisting{4-2.algo}	
	\section{Implémentation du TAD Peano}
		\subsection{}
			\lstinputlisting{5-1.algo}	
			\newpage
		\subsection{}
			\lstinputlisting{5-2.algo}	
		\subsection{opération add}
			\lstinputlisting[caption=Algorithme général]{5-3.algo}	
			\newpage
			\lstinputlisting[caption=Programme]{5-3-2.algo}	
		\subsection{opération mult}
			\lstinputlisting[caption=Algorithme général]{5-4.algo}	
			\newpage
			\lstinputlisting[caption=Programme]{5-4-1.algo}	
		\subsection{opération inf}
			\paragraph{Principe} parcourir en parallèle les deux entiers p et q et s'arrêter
			dès qu'un des deux entiers est épuisé (!);
		\subsubsection{}	
			\lstinputlisting[caption=Algorithme général]{5-5.algo}	
			\newpage
		\subsubsection{}	
			\lstinputlisting[caption=Programme]{5-5-1.algo}	
	\subsection{Implémentation du TAD Peano}		
		\subsubsection{}
		\subsubsection{Définition du type Peano}
		\begin{itemize}
			\item Corps des sous-programmes zero, succ, add, mult et inf
			\item Corps du sous-programme "=" (à écrire)
			\item Corps du sous-programme "<-" (à étudier)
		\end{itemize}	
		
\end{document}

