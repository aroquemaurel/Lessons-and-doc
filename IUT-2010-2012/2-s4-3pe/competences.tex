\documentclass[12pt,a4paper,openany]{article}

\usepackage{lmodern}
\usepackage{xcolor}
\usepackage[utf8]{inputenc}
\usepackage[T1]{fontenc}
\usepackage[francais]{babel}
\usepackage[top=1.7cm, bottom=1.7cm, left=1.7cm, right=1.7cm]{geometry}
\usepackage{verbatim}
\usepackage{tikz} %Vectoriel
\usepackage{listings}
\usepackage{fancyhdr}
\usepackage{multido}
\usepackage{amssymb}
\usepackage{multicol}

\newcommand{\titre}{Analyste programmeur}

\newcommand{\module}{Projet Profesionnel et Personnel de l'Étudiant}
\newcommand{\sigle}{3pe}

\newcommand{\semestre}{3} 

%\input{~/cours/listings.tex} %prise en charge du langage algo
\date{\today}

\chead{}
\rhead{Antoine de \bsc{Roquemaurel}}
\lhead{\titre}
\makeindex
\lfoot{Université Paul Sabatier Toulouse III}
\rfoot{\sigle\semestre}
%\rfoot{}
\cfoot{--~\thepage~--}
\makeglossary
\makeatletter
\def\clap#1{\hbox to 0pt{\hss #1\hss}}%
\def\ligne#1{%
\hbox to \hsize{%
\vbox{\centering #1}}}%
\def\haut#1#2#3{%
\hbox to \hsize{%
\rlap{\vtop{\raggedright #1}}%
\hss
\clap{\vtop{\centering #2}}%
\hss
\llap{\vtop{\raggedleft #3}}}}%
\def\bas#1#2#3{%
\hbox to \hsize{%
\rlap{\vbox{\raggedright #1}}%
\hss \clap{\vbox{\centering #2}}%
\hss
\llap{\vbox{\raggedleft #3}}}}%
\def\maketitle{%
\thispagestyle{empty}\vbox to \vsize{%
\haut{}{\@blurb}{}
\begin{flushleft}
	\vspace{1cm}
	Antoine de \bsc{Roquemaurel}\\ 
	\textit{Groupe B}\\
\end{flushleft}
\begin{flushright}
	\vspace{-1.5cm}
	Pour Madame Muriel \bsc{Nottin}
\end{flushright}
\vfill
\vspace{1cm}
\begin{flushleft}
\usefont{OT1}{ptm}{m}{n}
\huge \@title
\end{flushleft}
\par
\hrule height 4pt
\par
\begin{flushright}
\usefont{OT1}{phv}{m}{n}
\Large \@author
\par
\end{flushright}
\vspace{1cm}
\vfill
\vfill
\bas{}{\@location, le \@date}{}
}%
\cleardoublepage
}
\def\date#1{\def\@date{#1}}
\def\author#1{\def\@author{#1}}
\def\title#1{\def\@title{#1}}
\def\location#1{\def\@location{#1}}
\def\blurb#1{\def\@blurb{#1}}
\date{\today}
\author{}
\title{}
\location{Amiens}\blurb{}
\makeatother
\title{\titre}
\author{Projet Professionnel et Personnel de l'Étudiant}

\location{Toulouse}
\blurb{%
Université Paul Sabatier -- Toulouse III\\
IUT A - Toulouse Rangueil\\
}%



%\title{Cours \\ \titre}
%\date{\today\\ Semestre \semestre}

%\lhead{Cours: \titre}
%\chead{}
%\rhead{\thepage}

%\lfoot{Université Paul Sabatier Toulouse III}
%\cfoot{\thepage}
%\rfoot{\sigle\semestre}

\pagestyle{fancy}

\begin{document}
	\maketitle
	\tableofcontents
	\newpage
	\section{Déclinaison des compétences \textit{nécessaire} en savoir}
	\begin{tabular}{|p{3.8cm}|p{4.0cm}|p{4.0cm}|p{4.0cm}|p{4.0cm}|}
		\hline
		\begin{center}\textbf{Activité}\end{center} & 
		\begin{center}``\textbf{Savoir}'' \end{center}&
		\begin{center}``\textbf{Savoir faire}'' \end{center}& 
		\begin{center}``\textbf{Savoir être}''\end{center}\\
		\hline
		Analyse des besoins & Connaissance du métier du client& Compréhension du client, vulgarisation & Être attentif\\
		\hline
		Rédaction des~spécifications techniques & & &\\
		\hline
										& Connaître des langages de programmation & Savoir utiliser un EDI\footnotemark~-- Savoir utiliser un compilateur & Être patient  \\
		Conception et déve\-loppement   & Connaître l'algorithmique && Être logique \\
										& Connaître la modélisation & Connaître des langages de modélisation & Être rigoureux\\
										& Travailler en équipe & Communiquer avec l'équipe (vive voie, mail, forums, \ldots) & Être cohérent dans ses propos\\
		\hline
			Rédaction de documents techniques 
				& Connaître les prin\-cipes de l'expression écrite
				& Savoir utiliser un logiciel de traitement de texte --
				Adapter son style au destinataire\footnotemark & \\
		\hline
	\end{tabular} 

	\addtocounter{footnote}{-1}
	\footnotetext{Environnement de Développement Intégré}
	\addtocounter{footnote}{1}
	\footnotetext{Notamment vulgariser les termes techniques} 

	\section{Compétence en lien avec le projet dans le Programme Pédagogique National}
	\begin{multicols}{2}
		\subsection{AP}
		\subsubsection{Algorithmique}
			Concepts transversaux aux différents champs de l’informatique en
				terme de \textbf{raisonnement}, d’\textbf{abstraction} et de mise en \oe{}uvre de \textbf{solutions},
				\begin{itemize}
					\item Structures de données 
				\end{itemize}
		\subsubsection{Programmation}
		\begin{itemize}
			\item Compétences permettant de \textbf{comprendre}, faire \textbf{évoluer}, d’assurer
				la \textbf{maintenance} et de \textbf{déployer} une application logicielle
			\item Participer à un \textbf{travail d’équipe} en charge d’un projet et à être
				autonome dans la réalisation d’une mission.
			\item Langages C, C++, Java, PHP	
		\end{itemize}
		\subsection{ASR} 
		\subsubsection{Architectures des ordinateurs}
		Connaissance de base de l'ordinateur, permettant de mieux comprendre son fonctionnement et ainsi de programmer
		plus efficacement.
		\begin{itemize}
			\item Assembleur \bsc{ARM} 
		\end{itemize}
			\subsubsection{Les réseaux} 
				Mise en \oe{}uvre d’\textbf{applications réparties} ou distribuées entre des ordinateurs proches ou lointains.
				\begin{itemize}
					\item Proxy \bsc{FTP}
				\end{itemize}
		\subsection{OMGL} 
			\subsubsection{Production de logiciel}
			\begin{itemize}
				\item Notion sur la qualité du code
				\item Langage de modélisation UML\footnote{Unified Modelling Language}
			\end{itemize}
			\subsubsection{Interface Homme Machine}
			\begin{itemize}
				\item Conception d'interface graphique
				\item Notion d'ergonomie
			\end{itemize}
		\subsection{EGO} 
		\begin{itemize}
			\item Droit du travail
			\item Droit de l'informatique
		\end{itemize}
		\subsection{LEC} 
		\subsubsection{Langues}
		\begin{itemize}
			\item Anglais technique
		\end{itemize}
		\subsubsection{Expression Communication}
		\begin{itemize}
			\item Rédaction de rapports
			\item CV et lettre de motivation
			\item Conduite de réunion
		\end{itemize}
		\subsection{Maths} 
		\begin{itemize}
			\item Graphes, matrices, algèbre de Bool
		\end{itemize}
		\subsection{PPPE}
		\begin{itemize}
			\item Mieux me connaître
			\item Visualiser les compétences acquises et non acquises
			\item Choix professionnel post-\bsc{DUT}
		\end{itemize}
	\end{multicols}
	\section{Déclinaison des compétences \textit{apprises} en savoir}	
	\begin{tabular}{|p{1.2cm}|p{6.0cm}|p{6.0cm}|p{3.4cm}|}
		\hline
		\begin{center}``\textbf{Pôle}'' \end{center}&
		\begin{center}``\textbf{Savoir}'' \end{center}&
		\begin{center}``\textbf{Savoir faire}'' \end{center}& 
		\begin{center}``\textbf{Savoir être}''\end{center} \\
		\hline
		AP & Structures de données & Utiliser un EDI&\\ 
			& Langages de programmation & Utiliser un compilateur &\\
			
			&\textbf{comprendre}, faire \textbf{évoluer}, 
			d’assurer la \textbf{maintenance} et de déployer une application logicielle & Concevoir un algorithme &\\
			& Conception d'interfaces gra\-phique & &\\
		\hline
			ASR& Connaissance de base d'un ordinateur& Développement d'applications réparties & \\
		\hline
			OMGL & Notion sur la qualité du code & Utilisation d'outils analysant la qualité & \\
			& Faire la conception d'un programme & Utiliser le langage \bsc{UML} & \\
		\hline
			EGO &	Connaissances en droit du travail et de l'informatique &&\\
		\hline
		LEC &	Rédaction d'un document & Dans le cadre du \bsc{C2I}, apprentissage d'un traitement de texte
			et d'un tableur & Sens de la communication \\
			 & Connaissance de l'anglais technique & &\\
		\hline
		MATH & & Rigueur&\\
		\hline
	\end{tabular} 
	\newpage
	\section{Bilan des compétences}
	\subsection{Savoir}
	À l'\bsc{IUT} nous avons eu beaucoup de savoir, cependant une partie ne sont que des notions, pour certains, cela n'est pas
	très grave de ne pas être allé au bout pour le métier d'analyste programmeur (connaissance en droit de l'informatique par exemple), cependant
	ses notions nous permettent quand même d'avoir un esprit plus ouvert, ce qui est important. 

	D'autres notions doivent être aprofondis pour pouvoir faire ce métier, notamment la qualité du code, faire du code de meilleur qualité peut se
	faire de différentes façon:
	\begin{itemize}
		\item Poursuivre en licence professionnel Qualité 
		\item Utiliser régulièrement des outils qui aident à la qualité (Sonar, PMD, \ldots)
		\item Pratiquer avec une personne plus expérimentée qui nous conseil
	\end{itemize}
	Pour ma part, je pense pas poursuivre en licence professionnel, je préfererait faire des étude plus longues, j'apprendrai donc 
	la qualité du code durant cette poursuite d'étude, cependant j'ai également appris beaucoup de chose récemment dans le 
	cadre des projets tuteurés, notre client relisant régulièrement le code et nous expliquant ce qui ne va pas. 
	\subsection{Savoir faire}
		L'\bsc{IUT} est une excellente formation, ainsi une fois le diplôme obtenu, nous avons beaucoup de savoir faire. Pour être développeur, 
		cela est amplement suffisant, cependant pour être analyse programmeur, je pense que cela pourrait être poussé un peu plus loin.
		
		Notamment sur la modélisation \bsc{UML}, nous n'en connaissons qu'une partie, et il est difficile à notre niveau de faire une
		conception de bonne qualité, ce qu'un analyse programmeur doit être capable de faire. Il faudrait donc que je progresse de ce cotés
		là, je peut me renseigner via Internet sur des compléments de la norme, et pour ce qui est de la qualité de la conception,
		cela viendra avec l'expérience, en effet après avoir fait plusieurs conception, je m'améliorerai et aurait donc un savoir faire
		plus important.
	\subsection{Savoir Être}
	À l'\bsc{IUT}\footnote{Institut Universitaire de Technologies} nous n'apprenons aucun savoir être, en effet, cela est notre personne,
	cela changera peut être avec le temps, cependant ce ne sont pas les professeurs qui vont nous l'apprendre. 

	Actuellement, je pense avoir la majorité de ses savoir être, cela à toujours été dans ma nature d'être calme, et patient, rigoureux, je ne
	pense pas a avoir apprendre de savoir être, cependant, je vais devoir avoir un attention particulière lors de l'analyse des besoins
	pour que je sois attentifs au client et que je me fasse bien comprendre. 
\end{document}

% Les décliner en termes de savoir savoir faire savoir être
% Faire un bilan entre les compétences métier et formation 
		% Compétences acquises
		% Compétences à acquérir => Comment ?!

% Fiche compétence à remplir après le stage (moddle rubrique ``stage obligatoire S4)
% Oral entretien indivuduel avec CV 
	% Note =
		% Dossier résultat du travail : /10
		% Oral : /10


