\chapter{Glossaire}
\paragraph{Adobe AIR} Adobe Integrated Runtime, anciennement nommé Apollo, est une machine virtuelle multiplateforme, multilangage, multi interface qui s'exécute sur le système d'exploitation, qui permet de développer une application native avec des technologies Web.
\paragraph{AJAX}
Asynchronous Javascript and XML, c'est un ensemble de technologies destinées à réaliser de rapides mises à jour du contenu d'une page Web, sans qu'elles nécessitent le moindre rechargement visible par l'utilisateur de la page Web. Les technologies employées sont diverses et dépendent du type de requêtes que l'on souhaite utiliser, mais d'une manière générale le Javascript est constamment présent.
\paragraph{CSS} Cascading Style Sheets, un langage permettant de décrire la présentation des documents HTML.
\paragraph{Expressions régulières} Chaîne de caractères qui décrit un ensemble de chaînes de caractères possibles selon une syntaxe précise. Elles sont aujourd’hui utilisées par les informaticiens dans l’édition et le contrôle de texte ainsi que dans la manipulation des langues.
\paragraph{GFT} Gestion Financière des Télécoms(TEM en anglais). Ensemble des outils et ressources qui ont pour objectif d’optimiser et de rationaliser la gestion des dépenses télécoms (Mobile / Fixe, Voix / data, VoIP) des entreprises.
\paragraph{JavaScript} Langage de programmation de scripts principalement utilisé dans les pages web interactives côté client.
\paragraph{HTML} Hypertext Markup Language, format de données conçu pour représenter les
pages web
\paragraph{Méthodes Agiles}
Groupes de pratiques pouvant s'appliquer aux projets de développement en informatique. Les méthodes agiles impliquent au maximum le client et permettent une grande réactivité à ses demandes. Elles visent la satisfaction réelle du besoin du client en priorité aux termes d'un contrat de développement.
\paragraph{MySQL} Système de gestion de base de données libres, principalement couplé à des serveurs web.
\paragraph{PHP} Hypertext Preprocessor, langage de scripts libre principalement utilisé pour
produire des pages web dynamiques via un serveur HTTP.
\paragraph{SaaS} Software as a Service, technologie consistant à fournir des services ou des logiciels informatiques par le biais du Web et non plus dans le cadre d'une application de bureau ou client-serveur.
\paragraph{SQL} Structured Query Language, pseudo
langage informatique (de type requête) standard et normalisé, destiné à interroger ou
à manipuler une base de données relationnelle




