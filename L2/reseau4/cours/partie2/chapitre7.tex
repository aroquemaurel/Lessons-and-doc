\chapter{Codage de l'information}
\section{Pourquoi coder ?}
La communication nécessite la \textbf{compréhension} entre les deux entités communicantes. L'émetteur
envoie de l'information au récepteur qui doit savoir l'\textbf{interpréter} pour la comprendre. Ainsi,
le codage de l'information est la première étape de toute communication.
\section{Comment coder ?}
L'être humain mit en place des langages et créa l'écrit. Au début, l'écrit consistait
essentiellement dans des dessins puis vint un alphabet plus simple à utiliser qui offrait de
multiples combinaisons pour une plus grande richesse de l'expression. En réalité, les
caractères de l'écrit ne sont que des symboles interprétables. L'écrit se développa et la
communication par la voie écrite fut institutionnalisé en France par la création de la poste
royale en 1464 par Louis XI.

L'invention du télégraphe électrique (1832) par P. Shilling va révolutionner le monde de la
communication qui allait s'effectuer par la voie des ondes. Le ``fil qui chante'' et qui va
plus vite que le messsager était si important que le ministère de l'intérieur allait créer
l'Administration du Télégraphe en 1837. Mais, il était nécessaire de codifier les caractères à
	transmettre. Ce fut l'oeuvre de S. Morse qui, en 1837, développa l'alphabet télégraphique. Le
	``Morse'' codait chaque caractère en une suite de signaux électriques de courte (point) ou de
	longue (trait) durée.

	L'autorisation d'accès privé au télégraphe ne fut donnée qu'en 1851 par Napoléon III. En 1879,
	le premier ministère des postes et du télégraphe (P et T) fut créé. Le téléphone allait
	apparaître à cette époque.

	En 1917, E. Baudot mit au point un code qui allait être utilisé sur le réseau télégraphique
	commuté (Télex). Des machines spéciales munies d'un clavier permettait de générer le signal
	correspondant au texte tapé. Ainsi, si dans le ``Morse'', il fallait que la personne connaisse
	le codage pour le générer, ceci n'était pas nécessaire avec le télex. Les premiers terminaux
	``numériques'' apparaissaient.

	Lorsque l'Informatique se développa, il était nécessaire de constituer un codage adapté. En
	effet, la machine ne comprend que des éléments binaires (bits) : 0 et 1. Le codage consistera
	à combiner plusieurs bits. Ainsi, une dissociation allait se faire entre l'information et le
	signal généré.

	Le codage de l'information s'effectue en deux étapes:

	\begin{itemize}
		\item codage sous forme binaire (ASCII, EBCDIC, DCB\ldots);
		\item codage de l'élément binaire par un état physique (tension, fréquence\ldots);
	\end{itemize}

	Dans le cas où deux équipements veulent communiquer en utilisant des codes différents, il est
	nécessaire de disposer alors de fonctions de transcodage.

	Nous nous intéressons dans ce paragraphe à la première étape, la seconde étant traitée dans le
	paragraphe <<Techniques de transmission>>.

	\section{Différents codes}
	\subsection{Le code morse}
	Le code Morse est un des premiers codes développés. Chaque caractère est codé par une
	combinaison de points et de traits. Le code est un code statistique: les caractères les plus
	fréquemment utilisés sont codés avec peu d'éléments tandis que les caractères les moins
	fréquemment utilisé auront une longueur plus importante.

	\begin{tabular}{cccccccccc}
		A&.--&B&--\ldots&C&--.--.&D&--\ldots&E&.
	\end{tabular}

	L'inconvénient technique de ce code est que DE peut être confondu avec B (-\ldots). C'est
	pourquoi, un silence (pause) entre deux caractères était nécessaire.

	\subsection{Le code Baudot}
	Le code Baudot est un des premiers codes utilisés avec une machine. Chaque caractère est codé
	par une combinaison de 0 et de 1. Le code est un code à 5 bits. Il est aussi appelé code
	télégraphique ou Alphabet International (AI) \no 2 ou code CCITT \no 2.

	Avec 5 bits, nous ne pouvons avoir que 32 combinaisons. Or si on désire coder les lettres er
	les chiffres, nous ne disposons pas d'assez de combinaisons. C'est pourquoi le code Baudot
	contient deux jeux de caractères appelés ``Lettres'' (``Lower Case'') et ``Chiffres'' (``Upper
	Case''). En fait, l'ensemble ``Chiffres'' contient aussi d'autres symboles (ponctuation,
	\&,\#\ldots). Deux caractères ``Inversion Lettres'' (code 29) et ``Inversion Chiffres'' (code
	30) permet de commuter entre les deux ensembles. Evidemment, l'inconvénient réside dans des
	commutations fréquentes. D'autre part, ce code bien qu'il soit plus riche que le code Morse ne
	traite pas les minuscules et certains symboles.


	\subsection{Le code ASCII}
	L'apparition de l'informatique et la nécessité de disposer de codes plus riches et plus
	fonctionnels va mettre en évidence les limitations des codes précédents et va donner naissance
	à des codes contenant plus de bits. En 1963, un code à 7 bits va être développé aux Etats-Unis
	par l'ANSI. Ce code est connu sous le nom d'ASCII\footnote{American Standard Code for Information
Interchange}ou Alphabet International \no 5 ou Code CCITT \no 5 ou ISO 646.

Avec 7 bits, le code ASCII permet la représentation des lettres (majuscules et minuscules),
des chiffres, de différents symboles (nationaux,\ldots) et des caractères de commandes (de
terminal et de communication). C'est un code réellement conçu pour l'échange de données et la
gestion de la communication.

Le codage des lettres et des chiffres facilite le tri et le passage de majuscule aux
minuscules (et vice versa).

Des caractères ont été prévus pour :

\begin{itemize}
	\item les commandes de mise en page (Retour Chariot, Nouvelle Ligne\ldots);
	\item les commandes de périphériques (Device Control 1 à 4);
	\item les commandes de communication et de gestion de la liaion (ACK, NAK\ldots);
	\item \ldots
\end{itemize}

Enfin, pour munir ce code de mécanisme de détection d'erreur, un bit a été rajouté permettant
ce contrôle. Ce bit est appelé bit de parité en raison du mécanisme mis en oeuvre. C'est donc
un code sur 8 bits (7+1).

Le jeu de caractères ASCII a trouvé ses limites lorsqu'on a voulu coder des symboles
graphiques supplémentaires. C'est pourquoi, il fut étendu avec la norme ISO 4873.

L'usage du code ASCII (7+1) dans des protocoles de transmission pose un problème lors des
transmissions de données codées ASCII 8 bits ou lors des transmissions de données non ASCII
(Images, voix, exécutables\ldots).


