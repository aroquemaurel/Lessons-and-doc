\chapter{\'Evolution des réseaux}
		De la même manière que la téléphonie et le télégraphe, nous sommes passé d'une phase expérimentale à une phase d'utilisation. Ainsi l'Informatique à beaucoup évolué. Cette évolution à été progressive, il y a eu plusieurs étapes qui ont marqués les réseaux de communication.
		\paragraph{Coûts des équipements Informatiques / Coûts de la Communication} À l'origine seul les grands comptes étaient capable d'avoir des équipements informatiques. Ainsi les SSI\footnote{Société de Service en Informatiques} sont nées.
		\paragraph{Système de Télétraitement} Ces systèmes ont été destiné aux entreprise, afin qu'a distance elles puissent utiliser la puissance d'un calculateur qui était géographiquement loin. Une première structure de réseau Informatique fut créée.
		\remarque{Nous sommes en train de revenir à cette solution créée 40 ans auparavant: Le cloud computing}

		\section{Les équipements créés}
		Afin de construire ces structures de réseaux de communication nous avons mis en place des équipements :
		\begin{description}
			\item[Processeur Frontal de Communication\footnote{FEP: Front End Processor}]
			\item[Multiplexeurs et concentrateurs] Équipement de partage du support de communication, permettent d'avoir des nœuds de communication.
			\item[Liaisons Spécialisées] Nous avions besoin d'un réseau spécialisé afin d'interconnecter les appareils, pour les connections point à point.
			\item[Modem] Pour les trafics de grande ligne, il fut choisir d'utiliser un réseau déjà existant, le téléphone.
		Cependant, le signal à transmettre doit être adapté au support de transmission, on va donc utiliser un adaptateur qui permettra de faire 
		passer le signal sur le réseau téléphonique : le modem.
			\item[Commutateurs] Pour avoir une connexion la plus rapide possible, nous avions besoin d'un algorithme de routage afin de passer par un chemin en fonction du trafic présent sur la ligne: le routeur.
			\item[Protocole de communication] Permet de faire dialoguer deux machines entre elles, elle doivent utiliser le même protocole afin de se comprendre syntaxiquement et sémantiquement.
		\end{description}
		
		\section{Démocratisation de l'Informatique}
		\begin{description}
			\item[1970] La genèse des protocoles de communication date des années 1970. En réseau, rien n'a été inventé de nouveau, cela à surtout été des progrès technologiques : rapidité, miniaturisation, coûts et donc démocratisation. Les premiers mini-calculateurs.
			\item[1980] Début de l'informatique personnelle et mise en \oe{}uvre des réseaux locaux.
			\item[1990] Applications de l'Internet, premiers mobiles et satellites. 
		\end{description}

