\documentclass[a4paper, 11pt]{book}

\usepackage{lmodern}
\usepackage{xcolor}
\usepackage[utf8]{inputenc}
\usepackage[T1]{fontenc}
\usepackage[francais]{babel}
\usepackage[top=1.7cm, bottom=1.7cm, left=2.5cm, right=2.5cm]{geometry}
\usepackage{verbatim}
\usepackage{tikz} %Vectoriel
\usepackage{listings}
\usepackage{fancyhdr}
\usepackage{multido}
\usepackage{amssymb}
\usepackage{multicol}

\newcommand{\titre}{Probabilités}
\newcommand{\numero}{1}

\newcommand{\typeDoc}{Devoir Maison}

\newcommand{\module}{Probabilités}
\newcommand{\sigle}{proba}

\newcommand{\semestre}{3}


\usepackage{ifthen}
\date{\today}

\chead{Antoine de \bsc{Roquemaurel}}
\rhead{TP\no\typeDoc}
\lhead{\titre}
%\makeindex

\lfoot{Université Toulouse III -- Paul Sabatier}
\rfoot{\sigle\semestre}
%\rfoot{}
\cfoot{--~~\thepage~~--}

\makeglossary
\makeatletter
\def\clap#1{\hbox to 0pt{\hss #1\hss}}%

\def\haut#1#2#3{%
	\hbox to \hsize{%
		\rlap{\vtop{\raggedright #1}
	}%
	\hss
	\clap{\vtop{\centering #2}
}%
\hss
\llap{\vtop{\raggedleft #3}}}}%
\def\bas#1#2#3{%
	\hbox to \hsize{%
		\rlap{\vbox{
			\raggedright #1
		}
	}%
	\hss \clap{\vbox{\centering #2}}%
	\hss
	\llap{\vbox{\raggedleft #3}}}
}%
\def\maketitle{%
	\thispagestyle{empty}{%
		\haut{}{\@blurb}{}
		%	
		%\vfill

		\begin{center}
			\vspace{-2.0cm}
			\usefont{OT1}{ptm}{m}{n}
			\huge \@type \@title
		\end{center}
		\par
		\hrule height 1pt
		\par
		\vspace{1cm}
		\bas{}{}{}
}%
}
\def\date#1{\def\@date{#1}}
\def\author#1{\def\@author{#1}}
\def\type#1{\def\@type{#1}}
\def\title#1{\def\@title{#1}}
\def\location#1{\def\@location{#1}}
\def\blurb#1{\def\@blurb{#1}}
\date{\today}
\newboolean{monBool}
\setboolean{monBool}{true}
\author{}
\title{}
\ifthenelse{\equal{\typeDoc}{}}{
\numeroTD{}
}
{
	\type{\typeDoc~--- }
}
\location{Amiens}\blurb{}
%\makeatother
\title{\titre}
\author{%Semestre \semestre
}

\location{Toulouse}
\blurb{%
\vspace{-35px}
\begin{flushleft}
	Université Toulouse III -- Paul Sabatier\\
	L2 Informatique\\
\end{flushleft}
\begin{flushright}
	\vspace{-45px}
	\Large \textbf \module \\
	\normalsize \textit \today\\
	Semestre \semestre
	\vspace{30px}
\end{flushright}
Antoine de \bsc{Roquemaurel}
}%



%\title{Cours \\ \titre}
%\date{\today\\ Semestre \semestre}

%\lhead{Cours: \titre}
%\chead{}
%\rhead{\thepage}

%\lfoot{Université Paul Sabatier Toulouse III}
%\cfoot{\thepage}
%\rfoot{\sigle\semestre}

\pagestyle{fancy}

\input{/home/aroquemaurel/cours/includesLaTeX/listings.tex} %prise en charge du langage C 




%----------------------------------------------------------------------------------------
%	DEFINITION OF COLORED BOXES
%----------------------------------------------------------------------------------------

\RequirePackage[framemethod=default]{mdframed} % Required for creating the theorem, definition, exercise and corollary boxes

% Theorem box
\newmdenv[skipabove=7pt,
skipbelow=7pt,
backgroundcolor=black!5,
linecolor=ocre,
innerleftmargin=5pt,
innerrightmargin=5pt,
innertopmargin=5pt,
leftmargin=0cm,
rightmargin=0cm,
innerbottommargin=5pt]{tBox}

% Exercise box	  
\newmdenv[skipabove=7pt,
skipbelow=7pt,
rightline=false,
leftline=true,
topline=false,
bottomline=false,
backgroundcolor=ocre!10,
linecolor=ocre,
innerleftmargin=5pt,
innerrightmargin=5pt,
innertopmargin=5pt,
innerbottommargin=5pt,
leftmargin=0cm,
rightmargin=0cm,
linewidth=4pt]{eBox}	

% Definition box
\newmdenv[skipabove=10pt,
skipbelow=10pt,
rightline=false,
leftline=true,
topline=false,
bottomline=false,
linecolor=ocre,
innerleftmargin=5pt,
innerrightmargin=5pt,
innertopmargin=0pt,
leftmargin=0cm,
rightmargin=0cm,
linewidth=4pt,
innerbottommargin=0pt]{dBox}	

% Corollary box
\newmdenv[skipabove=7pt,
skipbelow=7pt,
rightline=false,
leftline=true,
topline=false,
bottomline=false,
linecolor=gray,
backgroundcolor=black!5,
innerleftmargin=5pt,
innerrightmargin=5pt,
innertopmargin=5pt,
leftmargin=0cm,
rightmargin=0cm,
linewidth=4pt,
innerbottommargin=5pt]{cBox}		

% Corollary box
\newmdenv[skipabove=7pt,
skipbelow=7pt,
rightline=true,
leftline=false,
topline=false,
bottomline=true,
linecolor=gray,
backgroundcolor=black!5,
innerleftmargin=5pt,
innerrightmargin=5pt,
innertopmargin=5pt,
leftmargin=0cm,
rightmargin=0cm,
linewidth=1pt,
innerbottommargin=5pt]{rBox}				  
		  

% Creates an environment for each type of theorem and assigns it a theorem text style from the "Theorem Styles" section above and a colored box from above
\newenvironment{theorem}{\begin{tBox}\begin{theoremeT}}{\end{theoremeT}\end{tBox}}
\newenvironment{example}{\begin{exampleT}}{\hfill{\tiny\ensuremath{\blacksquare}}\end{exampleT}}
\newenvironment{definition}{\begin{dBox}\begin{definitionT}}{\end{definitionT}\end{dBox}}
\newenvironment{attention}{\begin{eBox}\small}{\end{eBox}}				  	
\newenvironment{exemple}{\begin{cBox}\small}{\end{cBox}}	

%----------------------------------------------------------------------------------------
%	REMARK ENVIRONMENT
%----------------------------------------------------------------------------------------

\newenvironment{remarque}{\par\vskip10pt\small
\begin{rBox}
\begin{list}{}{
\leftmargin=35pt % Indentation on the left
\rightmargin=25pt}\item\ignorespaces % Indentation on the right
\makebox[-2.5pt]{\begin{tikzpicture}[overlay]
\node[draw=ocre!60,line width=1pt,circle,fill=ocre!25,font=\sffamily\bfseries,inner sep=2pt,outer sep=0pt] at (-15pt,0pt){\textcolor{ocre}{R}};\end{tikzpicture}} % Orange R in a circle
\advance\baselineskip -1pt}
{\end{list}\vskip1mm\end{rBox}\vskip5pt} % Tighter line spacing and white space after remark



\input{/home/aroquemaurel/cours/includesLaTeX/polices.tex}
\input{/home/aroquemaurel/cours/includesLaTeX/affichageChapitre.tex}

\makeatother
\newcommand{\var}{\textrm{var}}
\newcommand{\corr}{\textrm{corr}}
\newcommand{\cov}{\textrm{cov}}
\begin{document}
	\maketitle
	\section*{Exercice 1}
	\subsection*{1 --- Ensemble $\Omega$ et tribu à laquelle appartiennent $M$, $D$ et $F$}
		A est une tribu si les propriétés suivantes sont vérifiées : 
\begin{itemize}
	\item $\varnothing \in A$ et $\Omega \in A$
	\item $A$ est stable par passage au complémentaire : $A \in A \Rightarrow \overline A \in A$
	\item $A$ est stable par réunion et interesection dénombrable : si $(A_n)_n$ est une suite dénobmrable d'éléments de $A$ alors $\bigcup_{n\in N} A_n$ et $\bigcap_{n\in A}A_n$ sont des éléments de $A$
\end{itemize}
	\begin{eqnarray*}
		\Omega&=&\{(\overline M, \overline D, \overline F), (\overline M, \overline D, F), (\overline M, D, \overline F), (\overline M, D, F), (M, \overline D, \overline F), (M, \overline D, F), (M, D, \overline F), (M, D, F)\}\\
	A &=& \{\varnothing, \Omega\}
	\end{eqnarray*}
	\subsection*{2 --- Calculs de probabilités}
	\begin{multicols}{4}
	\begin{eqnarray*}
		P(M) &=& \frac{2}{3}\\
		P(D/M) &=&  \frac{7}{10}\\
		P(F/M) &=&  \frac{1}{5}\\
		P(D/\overline M) &=&  \frac{1}{5}\\
		P(F/\overline M) &=&  \frac{9}{10}\\
		P(\overline M) &=&  \frac{1}{3}\\
		P(\overline D/M) &=&  \frac{3}{10}\\
		P(\overline F / M) &=&  \frac{4}{5}\\
		P(\overline D / \overline M) &=&  \frac{4}{5}\\
		P(\overline F / \overline M) &=&  \frac{1}{10}
	\end{eqnarray*}
\end{multicols}
	\subsection*{3 --- Probabilité que l'étudiant $e$ aime les maths}
	\begin{eqnarray*}
		P(M/D\cap F) &=& \frac{P(D\cap F/M)P(M)}{P(D\cap F)} = \frac{P(D/M)P(F/M)P(M)}{P(D\cap F)}\\
		&=& \frac{P(D/M)P(F/M)P(M)}{P(D \cap F \cap M) + P(D\cap F\cap \overline M)}\\
		&=&  \frac{P(D/M)P(F/M)P(M)}{P(D\cap F/M)P(M) + P(D\cap F/\overline M)P(\overline M)}\\
		&=&  \frac{P(D/M)P(F/M)P(M)}{P(D/M)P(F/M)P(M) + P(D/\overline M) + P(F/\overline M)P(\overline M)}\\
		&=&  \frac{P(D/M)P(F/M)P(M)}{\frac{7}{10} \times \frac{2}{10} \times \frac{2}{3} + \frac{1}{5} \times \frac{9}{10} \times \frac{1}{3}}\\
		P(M/D\cap F) &=&  \frac{\frac{7}{10} \times \frac{1}{5} \times \frac{2}{3}}{\frac{23}{150}} = \frac{14}{23}
	\end{eqnarray*}
	\section*{Exercice 2}
	\subsection*{1 --- Probabilités $P(B_1)$ et $(P_B2)$}
	\begin{eqnarray*}
		P(B_1) &=& \frac{k}{k+m}\\
		P(R_1) &=&  \frac{m}{k+m}
	\end{eqnarray*}
	\subsection*{2 --- Calcul de probabilités}
	\begin{eqnarray*}
		P(B_2/B_1) &=& \frac{k+1}{k+m+1}\\
	P(R_2/B_1) &=&  \frac{m}{k+m+1}\\
	P(B_1 \cap B_2) &=& P(B_2/B_1)P(B_1) = \frac{k(k+1)}{(k+m+1)(k+m)} = \frac{k^2+k}{k(k+2m+1)+m+m^2} \\
	P(B_1 \cap R_2) &=&  P(R_2 / B_1) P(B_1) = \frac{m}{k+m+1} \times \frac{k}{k+m} = \frac{km}{2k+2m}
	\end{eqnarray*}
	\subsection*{3 --- Indépendance}
		P(A) et P(A) sont indépendants si et seulement si $P(A\cap B) = P(A) \times P(B)$

		Je pense que $B_1$ et $B_2$ sont dépendants, en effet, $P(A\cap B) \neq P(A) \times P(B)$
	\subsection*{4}
	\subsection*{5 --- Probabilités conditionnelles}
\begin{eqnarray*}
	P(B_2/R_1) &=& \frac{k}{k+m}\\
	P(R_2/R_1) &=& \frac{m}{k+m}\\
	P(B_2) &=& (P(B_1) \times p(B_2/B_1)) + (P(R_1) \times p(B_2/R_1)) = \frac{k \times (k+1)}{(k+m) \times (k+m+1)}  + \frac{m^2}{(k+m)^2}\\
	&=& \frac{k^2+k}{k^2+2mk+k+m^2+m} + \frac{m^2}{k^2+2km+m^2}\\
	&=& \frac{k^2+k+mk(1+\frac{1}{k+m})}{(k+m)^2+k+m}\\
	P(B_1) \times P(B_2) &=& \frac{k}{k+m} \times \frac{k^2+k+mk(1+\frac{1}{k+m})}{(k+m)^2+k+m} \neq P(B_1/B_2)\\
\end{eqnarray*}
\subsection*{6}
\begin{eqnarray*}
	P(R_1/B_2) &=& \frac{P(R_1\cap B_2)}{P(B_2} = \\
	P(B_1/B_2) &=& \frac{P(B_1\cap B_2)}{P(B_2} =  \frac{\frac{k^2+k}{k(k+2m+1)+m+m^2}}{P(B_2)} \\ 
\end{eqnarray*}
\subsection*{7}
\begin{eqnarray*}
	P(B_1) &=&  \frac{2}{2+3} = \frac{2}{5}\\
	P(B_2) &=&  \frac{11}{25}\\
	P(B_1/B_2) 
\end{eqnarray*}
	\section*{Exercice 3}
	\subsection*{1}
	\begin{eqnarray*}
		\Omega&=&\{00,01,10,11\}\\
		\textrm{espaceArrive} &=& \{00,01,10,11,21,12\}
	\end{eqnarray*}
	\subsection*{2}

	\subsection*{3-4 --- Loi conjointe et marginales}
	\begin{tabular}{c|c|c|c|c|}
		X/Y&0&1&2&p(X=xj)\\
		\hline
		0&$\frac{1}{4}$&0&0&$\frac{1}{4}$\\&&&&\\
		\hline
		1&0&$\frac{2}{4}$&0&$\frac{2}{4}$\\&&&&\\
		\hline
		2&0&0&$\frac{1}{4}$&$\frac{1}{4}$\\&&&&\\
		\hline
		P(y=yj)&$\frac{1}{4}$&$\frac{2}{4}$&$\frac{1}{4}$&\\&&&&\\
		\hline
	\end{tabular}

	\subsection*{5 --- Esperance et variance}
	\begin{eqnarray*}
		E(X)&=& 0 \times \frac{1}{4} + 1 \times \frac{2}{4} + 2 \times \frac{1}{4} = 1\\
		E(Y)&=& 0 \times \frac{1}{4} + 1 \times \frac{2}{4} + 2 \times \frac{1}{4} = 1\\
		\var(X) &=& E(X)^2 - E(X^2) = \frac{3}{2} - 1 = \frac{1}{2}\\
		\var(Y) &=& E(Y)^2 - E(Y^2) = \frac{3}{2} - 1 = \frac{1}{2}\\
	\end{eqnarray*}
	\subsection*{6 --- Corrélation}
	\begin{eqnarray*}
		\corr(X,Y) &=& \frac{\cov(X,Y)}{\sqrt{\var(X) \times \var(Y)}} = \frac{E(X,Y) - E(X) \times E(Y)}{\sqrt{\var(X)\times\var(Y)}} = \frac{\frac{3}{2} - 1 \times 1}{\frac{1}{4}} = 1\\
	\end{eqnarray*}
\end{document}





