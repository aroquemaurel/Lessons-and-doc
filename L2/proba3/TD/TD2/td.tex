\documentclass[a4paper, 11pt]{article}

\usepackage{lmodern}
\usepackage{xcolor}
\usepackage[utf8]{inputenc}
\usepackage[T1]{fontenc}
\usepackage[francais]{babel}
\usepackage[top=1.7cm, bottom=1.7cm, left=2.5cm, right=2.5cm]{geometry}
\usepackage{verbatim}
\usepackage{tikz} %Vectoriel
\usepackage{listings}
\usepackage{fancyhdr}
\usepackage{multido}
\usepackage{amssymb}

\newcommand{\titre}{Variables aléatoires discrètes}
\newcommand{\typeDoc}{TD}

\newcommand{\module}{Probablité}
\newcommand{\sigle}{proba}

\newcommand{\semestre}{3}

\input{/home/satenske/cours/includesLaTeX/listings.tex} %prise en charge du langage algo

\usepackage{ifthen}
\date{\today}

\chead{Antoine de \bsc{Roquemaurel}}
\rhead{TP\no\typeDoc}
\lhead{\titre}
%\makeindex

\lfoot{Université Toulouse III -- Paul Sabatier}
\rfoot{\sigle\semestre}
%\rfoot{}
\cfoot{--~~\thepage~~--}

\makeglossary
\makeatletter
\def\clap#1{\hbox to 0pt{\hss #1\hss}}%

\def\haut#1#2#3{%
	\hbox to \hsize{%
		\rlap{\vtop{\raggedright #1}
	}%
	\hss
	\clap{\vtop{\centering #2}
}%
\hss
\llap{\vtop{\raggedleft #3}}}}%
\def\bas#1#2#3{%
	\hbox to \hsize{%
		\rlap{\vbox{
			\raggedright #1
		}
	}%
	\hss \clap{\vbox{\centering #2}}%
	\hss
	\llap{\vbox{\raggedleft #3}}}
}%
\def\maketitle{%
	\thispagestyle{empty}{%
		\haut{}{\@blurb}{}
		%	
		%\vfill

		\begin{center}
			\vspace{-2.0cm}
			\usefont{OT1}{ptm}{m}{n}
			\huge \@type \@title
		\end{center}
		\par
		\hrule height 1pt
		\par
		\vspace{1cm}
		\bas{}{}{}
}%
}
\def\date#1{\def\@date{#1}}
\def\author#1{\def\@author{#1}}
\def\type#1{\def\@type{#1}}
\def\title#1{\def\@title{#1}}
\def\location#1{\def\@location{#1}}
\def\blurb#1{\def\@blurb{#1}}
\date{\today}
\newboolean{monBool}
\setboolean{monBool}{true}
\author{}
\title{}
\ifthenelse{\equal{\typeDoc}{}}{
\numeroTD{}
}
{
	\type{\typeDoc~--- }
}
\location{Amiens}\blurb{}
%\makeatother
\title{\titre}
\author{%Semestre \semestre
}

\location{Toulouse}
\blurb{%
\vspace{-35px}
\begin{flushleft}
	Université Toulouse III -- Paul Sabatier\\
	L2 Informatique\\
\end{flushleft}
\begin{flushright}
	\vspace{-45px}
	\Large \textbf \module \\
	\normalsize \textit \today\\
	Semestre \semestre
	\vspace{30px}
\end{flushright}
Antoine de \bsc{Roquemaurel}
}%



%\title{Cours \\ \titre}
%\date{\today\\ Semestre \semestre}

%\lhead{Cours: \titre}
%\chead{}
%\rhead{\thepage}

%\lfoot{Université Paul Sabatier Toulouse III}
%\cfoot{\thepage}
%\rfoot{\sigle\semestre}

\pagestyle{fancy}


\begin{document}
	\maketitle
	\section{1}
	\begin{eqnarray*}
		E(x) &=& \int xf(x) dx\\
		&=& \int^b_a x\frac{1}{b-a}dx\\
		&=& \frac{1}{b-a}\int^b_a xdx\\
		&=& \frac{1}{b-a} [\frac{x^2}{2}]^b_a\\
		&=& \frac{[\frac{b^2}{2}] - \frac{a^2}{2}]}{b-a} = \frac{b^2 -a^2}{2(b-a)}\\
		&=& \frac{(b-a)(b+a)}{2(b-a)}
	\end{eqnarray*}

	\begin{eqnarray*}
		var(X) = E(X^2)-E(X)^2\\
		E(X) &=& \int x^2f(x)dx\\
		&=& \int^b_a x^2\frac{1}{b-a} dx\\
		&=& \frac{1}{b-a} \int^b_a x^2 dx\\
		&=& \frac{1}{b-a}[\frac{1}{3} x^3]^b_a\\
		&=& \frac{1}{b-a} \frac{b^3-a^3}{3} = \frac{(b-a)(b^2+a^2+ab)}{3(b-a)}=\frac{b^2+a^2+ab}{3}
	\end{eqnarray*}
	\begin{eqnarray*}
		var(X) &=&  \frac{b^2+a^2+ab}{3} - \frac{a^2+2ab+b^2}{4}\\
		&=& \frac{(4b^2 + 4a^2 + 4ab)-(3a^2	+6ab+3b^2)}{12}\\
		&=& \frac{a^2+b^2-2ab}{12}\\
		&=&\frac{(a-b)^2}{12}
	\end{eqnarray*}
	\section{Exercice 1}
%	\begin{itemize}
%		\item 
%\begin{eqnarray*}
%	x > 0, F(x) = 0\%\%
%	0 \leq x \leq 1, F_x(x) = \int^x_{-\infty}f(u) du\\%
%	= 0+\int^x_0 \frac{u}{2} du = \frac{1}{2}\int^x_0udu = \frac{1}{2}[\frac{u^2}{2}]^x_0 = \frac{1}{2}(\frac{x^2}{2}) = \frac{x^2}{4}
%\end{eqnarray*}
%\item 
%\begin{eqnarray*}
%	F_x(x) &=&  \frac{1}{2}+[u^2 ( \frac{5}{2}u]^x_{\frac{3}{2}}	\\
%	&=& \frac{1}{2} + x^2 - \frac{5{2}x - \frac{9}{4} + \frac{15}{4}\\
%	&=& x^2 - \frac{5}{2}x + 2
%\end{eqnarray*}
%	\end{itemize}
	\section{Exercice 2 --- TD3}
	T\footnote{v.a réélle à densité normale de m = 175 et $\sigma =6$}: taille moyenne d'un homme de 25 ans. 
	\subsection{$E(y)$ et $\sigma(y)$}
	\begin{eqnarray*}
		E(y) &=& E(\alpha X + \beta)\\
		&=& E(\alpha X) + \beta\\
		&=& \alpha E(X) + \beta
	\end{eqnarray*}
	\begin{eqnarray*}
		\sigma (y) &=& \sqrt{var(y)} = \sqrt{var(\alpha X + \beta)} = \sqrt{E( ( (\alpha X + \beta) - E(\alpha X + \beta))^2)} = \sqrt{E
	\end{eqnarray*}
\end{document}


