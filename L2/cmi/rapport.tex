\documentclass{article}

\usepackage{xcolor}
\usepackage[12pt]{extsizes}
\input{/home/aroquemaurel/cours/includesLaTeX/couleurs.tex}
\usepackage{lmodern}
\usepackage[utf8]{inputenc}
\usepackage[T1]{fontenc}
\usepackage[francais]{babel}
\usepackage[top=2.0cm, bottom=2.0cm, left=3.0cm, right=3.0cm]{geometry}
\usepackage{verbatim}
\usepackage{tikz} %Vectoriel
\usepackage{listings}
\usepackage{fancyhdr}
\usepackage{multido}
\usepackage{amssymb}
\usepackage{multicol}
\usepackage{float}
\usepackage[urlbordercolor={1 1 1}, linkbordercolor={1 1 1}, linkcolor=vert1, urlcolor=bleu, colorlinks=true]{hyperref}

\newcommand{\titre}{Entreprise Oracle}
\newcommand{\numero}{}
\newcommand{\typeDoc}{Gestion}
\newcommand{\module}{cmi}
\newcommand{\sigle}{cmi}
\newcommand{\semestre}{4}


\usepackage{ifthen}
\date{\today}

\chead{Antoine de \bsc{Roquemaurel}}
\rhead{TP\no\typeDoc}
\lhead{\titre}
%\makeindex

\lfoot{Université Toulouse III -- Paul Sabatier}
\rfoot{\sigle\semestre}
%\rfoot{}
\cfoot{--~~\thepage~~--}

\makeglossary
\makeatletter
\def\clap#1{\hbox to 0pt{\hss #1\hss}}%

\def\haut#1#2#3{%
	\hbox to \hsize{%
		\rlap{\vtop{\raggedright #1}
	}%
	\hss
	\clap{\vtop{\centering #2}
}%
\hss
\llap{\vtop{\raggedleft #3}}}}%
\def\bas#1#2#3{%
	\hbox to \hsize{%
		\rlap{\vbox{
			\raggedright #1
		}
	}%
	\hss \clap{\vbox{\centering #2}}%
	\hss
	\llap{\vbox{\raggedleft #3}}}
}%
\def\maketitle{%
	\thispagestyle{empty}{%
		\haut{}{\@blurb}{}
		%	
		%\vfill

		\begin{center}
			\vspace{-2.0cm}
			\usefont{OT1}{ptm}{m}{n}
			\huge \@type \@title
		\end{center}
		\par
		\hrule height 1pt
		\par
		\vspace{1cm}
		\bas{}{}{}
}%
}
\def\date#1{\def\@date{#1}}
\def\author#1{\def\@author{#1}}
\def\type#1{\def\@type{#1}}
\def\title#1{\def\@title{#1}}
\def\location#1{\def\@location{#1}}
\def\blurb#1{\def\@blurb{#1}}
\date{\today}
\newboolean{monBool}
\setboolean{monBool}{true}
\author{}
\title{}
\ifthenelse{\equal{\typeDoc}{}}{
\numeroTD{}
}
{
	\type{\typeDoc~--- }
}
\location{Amiens}\blurb{}
%\makeatother
\title{\titre}
\author{%Semestre \semestre
}

\location{Toulouse}
\blurb{%
\vspace{-35px}
\begin{flushleft}
	Université Toulouse III -- Paul Sabatier\\
	L2 Informatique\\
\end{flushleft}
\begin{flushright}
	\vspace{-45px}
	\Large \textbf \module \\
	\normalsize \textit \today\\
	Semestre \semestre
	\vspace{30px}
\end{flushright}
Antoine de \bsc{Roquemaurel}
}%



%\title{Cours \\ \titre}
%\date{\today\\ Semestre \semestre}

%\lhead{Cours: \titre}
%\chead{}
%\rhead{\thepage}

%\lfoot{Université Paul Sabatier Toulouse III}
%\cfoot{\thepage}
%\rfoot{\sigle\semestre}

\pagestyle{fancy}

\input{/home/aroquemaurel/cours/includesLaTeX/listings.tex} %prise en charge du langage C 




%----------------------------------------------------------------------------------------
%	DEFINITION OF COLORED BOXES
%----------------------------------------------------------------------------------------

\RequirePackage[framemethod=default]{mdframed} % Required for creating the theorem, definition, exercise and corollary boxes

% Theorem box
\newmdenv[skipabove=7pt,
skipbelow=7pt,
backgroundcolor=black!5,
linecolor=ocre,
innerleftmargin=5pt,
innerrightmargin=5pt,
innertopmargin=5pt,
leftmargin=0cm,
rightmargin=0cm,
innerbottommargin=5pt]{tBox}

% Exercise box	  
\newmdenv[skipabove=7pt,
skipbelow=7pt,
rightline=false,
leftline=true,
topline=false,
bottomline=false,
backgroundcolor=ocre!10,
linecolor=ocre,
innerleftmargin=5pt,
innerrightmargin=5pt,
innertopmargin=5pt,
innerbottommargin=5pt,
leftmargin=0cm,
rightmargin=0cm,
linewidth=4pt]{eBox}	

% Definition box
\newmdenv[skipabove=10pt,
skipbelow=10pt,
rightline=false,
leftline=true,
topline=false,
bottomline=false,
linecolor=ocre,
innerleftmargin=5pt,
innerrightmargin=5pt,
innertopmargin=0pt,
leftmargin=0cm,
rightmargin=0cm,
linewidth=4pt,
innerbottommargin=0pt]{dBox}	

% Corollary box
\newmdenv[skipabove=7pt,
skipbelow=7pt,
rightline=false,
leftline=true,
topline=false,
bottomline=false,
linecolor=gray,
backgroundcolor=black!5,
innerleftmargin=5pt,
innerrightmargin=5pt,
innertopmargin=5pt,
leftmargin=0cm,
rightmargin=0cm,
linewidth=4pt,
innerbottommargin=5pt]{cBox}		

% Corollary box
\newmdenv[skipabove=7pt,
skipbelow=7pt,
rightline=true,
leftline=false,
topline=false,
bottomline=true,
linecolor=gray,
backgroundcolor=black!5,
innerleftmargin=5pt,
innerrightmargin=5pt,
innertopmargin=5pt,
leftmargin=0cm,
rightmargin=0cm,
linewidth=1pt,
innerbottommargin=5pt]{rBox}				  
		  

% Creates an environment for each type of theorem and assigns it a theorem text style from the "Theorem Styles" section above and a colored box from above
\newenvironment{theorem}{\begin{tBox}\begin{theoremeT}}{\end{theoremeT}\end{tBox}}
\newenvironment{example}{\begin{exampleT}}{\hfill{\tiny\ensuremath{\blacksquare}}\end{exampleT}}
\newenvironment{definition}{\begin{dBox}\begin{definitionT}}{\end{definitionT}\end{dBox}}
\newenvironment{attention}{\begin{eBox}\small}{\end{eBox}}				  	
\newenvironment{exemple}{\begin{cBox}\small}{\end{cBox}}	

%----------------------------------------------------------------------------------------
%	REMARK ENVIRONMENT
%----------------------------------------------------------------------------------------

\newenvironment{remarque}{\par\vskip10pt\small
\begin{rBox}
\begin{list}{}{
\leftmargin=35pt % Indentation on the left
\rightmargin=25pt}\item\ignorespaces % Indentation on the right
\makebox[-2.5pt]{\begin{tikzpicture}[overlay]
\node[draw=ocre!60,line width=1pt,circle,fill=ocre!25,font=\sffamily\bfseries,inner sep=2pt,outer sep=0pt] at (-15pt,0pt){\textcolor{ocre}{R}};\end{tikzpicture}} % Orange R in a circle
\advance\baselineskip -1pt}
{\end{list}\vskip1mm\end{rBox}\vskip5pt} % Tighter line spacing and white space after remark



\input{/home/aroquemaurel/cours/includesLaTeX/polices.tex}
\input{/home/aroquemaurel/cours/includesLaTeX/affichageChapitre.tex}
\makeatother
\begin{document}
	\maketitle
	\setcounter{secnumdepth}{2}
	\section{L'organisation de l'entreprise}
	Oracle Corporation est une multinationale américaine dans les nouvelles technologies dont son siège social est basé à Redwood
	City en Californie. L'entreprise s'est spécialisée dans le développement et le marketing de matériel informatique et les
	logiciels d'entreprise, particulièrement dans la gestion de base de données. \\
	Oracle est la troisième plus grande entreprise de logiciel de part leurs revenues après Microsoft et IBM.

	Oracle possède plusieurs marchés différents, d'une part celui des entreprises ayant besoin d'un système de base de données
	puissant et robuste, ce système étant découpés en plusieurs version à des prix différents\footnote{Par exemple, l'université Paul
	Sabatier paye un abonnement afin d'avoir accès à une base de données Oracle}.

	D'autre part, Oracle à récemment racheté des logiciels qui étaient autre fois libres, notamment un des langages les plus utilisé
	au monde : Java, ainsi que le logiciel de virtualisation Virtualbox et la suite bureautique Openoffice.

	Les principaux concurrents d'Oracle sont Microsoft, Google et IBM, en effet, ils ont les même marchés.

	\section{La structure de l'entreprise}\label{struct}
	Cette entreprise est très grande, ainsi elle doit être extrêmement structurée afin d'être viable.

	Ainsi, elle possède tout d'abord un conseil composé de 8 directeurs, ce sont eux qui donne les grandes lignes de la politique de
	l'entreprise.
	\\Ensuite, tout est structurée par fonctions. Avec le marketing, la sécurité, la communication, les ressources humaines, le web,
	les systèmes, le secrétariat, le juridique, etc\ldots\\
	Toute cette structuration permet à l'entreprise de pouvoir déléguer et organiser toute sa politique.

	L'entreprise possède est une multinationale, ainsi elle doit avoir une structure adaptée à ce contexte, c'est ainsi que
	l'entreprise possède des filiales basées dans les principaux pays intéressés par Oracle.

	\section{Place de l'informatique dans l'entreprise}
	La place de l'informatique dans l'entreprise est très grande. En effet, comme dit précédemment, cette entreprise à une grande
	importance dans l'informatique de nos jours, en effet 
	98 des 100 plus grandes entreprises mondiales utilisent les produits et services développées par Oracle.

	Ainsi l'entreprise est donc lié à ce domaine, elle possède donc une multitude d'ingénieurs et techniciens dans les domaines de
	l'informatique.

	\subsection{De nombreux produits}
	Depuis la création de l'entreprise, celle-ci n'a de cesse de sortir de nouveaux produit, ceci afin d'attirer le maximum de
	clients potentiels et de couvrir le maximum de domaines de l'informatique.\\
	Ainsi, ils ont commencés par la base de données, puis sur des serveurs, puis des logiciels, et enfin le rachat de Sun à pus leur
	apporter le côtés matériel comme le logiciel de visualisation Virtualbox. 

	De nos jours, ils commencent à se tourner vers le cloud computing, en effet cet effet de mode est très demandés par les
	entreprises, oracle possédants déjà un très gros parc de serveur, ils proposent dorénavant aux entreprise la solution ``Cloud
	world'' .

	\subsection{De nombreux salariés}
	Afin de pouvoir proposer des solutions dans tous les domaines, ils doivent donc posséder des ingénieurs qualifiés dans tous les
	domaines également, ainsi l'entreprise possède énormément de salariés afin de satisfaire tous les axes de sa politique. C'est en
	raison de ce nombre de salariés que l'entreprise est importante et à dût utiliser une structure adaptée (cf \ref{struct}).

	\subsection{Une infrastructure informatique solide}
	En raison de la taille de l'entreprise, celle-ci doit également avoir une infrastructure informatique extrêmement grande et
	sécurisée afin de pouvoir avoir toutes les données confidentiels à l'abri d'un problème technique, ou d'un vol, ceci comme toutes
	les entreprises.\\
	Également, tous les ingénieurs doivent travailler sur ce grand parc.
	\section{Analyse personnel}
	Je pense que l'entreprise Oracle est une grande entreprise qui à su apporter beaucoup à l'informatique de nos jours. En effet,
	c'est grâce à celle-ci que les base de données relationnelles sont ce qu'elles sont actuellement. 

	Le rachat récent de Sun par Oracle permet à l'entreprise de s'imposer encore plus sur le marché. Cependant, ils vont devoire
	faire attention afin de garder leur place dominante. En effet, le langage Java connait en ce moment beaucoup de problème qu'ils
	ont du mal à résoudre, ils ont dut vendre Openoffice à la fondation apache, ainsi, Oracle commence à se faire une mauvaise
	réputation suite à plusieurs problèmes avec leurs logiciels. 

À la vitesse à laquelle évolue l'informatique,
	ils ne peuvent se permettre de rester statique, l'innovation à une place très importante dans ce marché.
	
	Cependant, leur place actuellement reste une place dominante sur le marché, en effet, ils n'ont pas encore trouvés de rivales en
	terme de base de données, tant qu'une autre solution aussi puissante qu'Oracle database n'apparaitra pas, l'entreprise à un bel
	avenir devant elle avec ce simple produit, de plus elle possède également Oracle E-Business Suite  un très gros logiciel de
	gestion. 

	Bien qu'Oracle reste a la pointe de la technologie, en s'ouvrant au cloud computing par exemple, elle risque d'être rapidement
	détrônée dans leur domaine d'expertise par des multinationales comme Google.
\end{document}
