\chapter{Complexité d'algorithmes définis par récurrence}
	\section{Exemple introductif : Tri fusion}
\'Etant donné un tableau T, on note T[i:j] le sous tableau de T qui va de la case i à la case j. L'algorithme de tri fusion utilise une procédure
\texttt{fusion(T,i,j,k)}. On suppose que les deux sous tableaux T[i:j] et T[j+1:k] sont déjà triés. En temps $\Theta(n)$, où $n=k-i+1$, la procédure
fusion produit le sous tableau T[i:k] trié à partir de la fusion de ces deux tableaux.

\lstinputlisting[language=algo, caption=Algorithme du tri fusion]{triFusion.algo}
	\section{Méthode naïve d'analyse de complexité}
	Soit un temps maximal d'exécution de tri fusion sur un tableau de longueur $n$.

	D'après l'algorithme, on a $$U_n = U_{\frac{n}{2}} + U_{\frac{n}{2}} \Theta(n)$$
	et $u_1 = 0$

	Pour simplifier la récurrence on suppose que $n$ est pair, et donc $U_n = 2U_{\frac{n}{2}} + \Theta(n)$

	La méthode naïve consiste à deviner la solution, ici on devine $U_n \leq c n \log_2 n$. On suppose $U_{\frac{n}{2}} \leq C \frac{n}{2} \log_2
	\frac{n}{2}$ et on essaye d'en déduire $U_n \leq c n \log_2 n$
	\begin{eqnarray*}
		U_n = 2U_{\frac{n}{2}} + cn &\leq& 2c \frac{n}{2} \log_2 \frac{n}{2} + cn \\&&= cn(\log_2 n -1) + cn = cn\log_2 n
	\end{eqnarray*}

	Puisque $u_1=0 \leq c 1 \log_2 1$, on en déduit $\forall n, i_n \leq cn \log_2 n$

	\subsection{Résumé de la méthode naïve}
	Pour une équation récurrente $u_n = f_n(U_{n-1}, \cdots, u_1)$ où f est une fonction monotone croissante
	\begin{enumerate}
		\item On devine une fonction $g$
		\item On suppose que $\forall n < 1$ on a $U_n \leq g(m)$
		\item On montre $U_n = f_n(U_{n-1},\cdots,u_1 \leq f_n(g(n-1), \cdots, g(1)) \leq g(n)$
		\item On conclut par récurrence que $\forall n$ on a $U_n \leq g(n)$
	\end{enumerate}
	\subsection{Exemples d'application}
	On commence par une \textbf{mauvaise} utilisation. Soit l'équation $U_n = 2 U_{\frac{n}{2}}$. L'intuition $U_n \leq kn$ n'est pas correcte.

	En effet, en remplaçant on obtient : 
	\begin{eqnarray*}
		n_n &=& 2U_{\frac{n}{2}}+1\\
		&=& 2k \frac{n}{2} + 1\\
		&=& kn + 1
	\end{eqnarray*}
	
	La bonne intuition est $u_n \leq kn - b$. En remplaçant on obtient : 
	\begin{eqnarray*}
		u_n = 2U_{\frac{n}{2}} + 1\\
		&\leq& 2(k\frac{n}{2} - b) + 1= kn - 2b + 1\\
		&\leq& kn -b\textrm{ Si } b \geq 1
	\end{eqnarray*}

	\subsection{Réduction à des formes simples}
	Lors de l'analyse d'algorithmes récursifs, on rencontre souvent des équations récurrentes de la forme $$u_n = aU_{\frac{n}{2}}+b,$$ où a et b sont des
	constantes. Par exemple le tri fusion.

	Pour convertir ce type de récurrence en une forme affine $u'_n = a'u'_{n-1}+b'$, on pose
	$$v_k = U_{2^k}$$
	Autrement dit, on étudiera la suite $\{u_n\}_{n \geq 0}$ uniquement sur les puissances de 2.
	\\
	Par exemple, pour le tri fusion, en remplaçant $n$ par $2^k$, 
	\begin{eqnarray*}
		U_{2^k} &=& 2U_{\frac{2^k}{2}} + C2^k\\
		\textrm{donc }V_k &=& 2v_{k-1} + c2^k
	\end{eqnarray*}

	\section{Équation récurrentes linéaires}
	\paragraph{Définition} Une équation récurrente linéaire à coefficients constants d'odre $k$ est une équation de la forme 
	\begin{displaymath}
		\left\{ \begin{array}{llll}
			u_1 &=& C_i (O \leq i \leq k-1) & \textsc{Conditions initiales (CI)}\\
			u_n &=&  \sum^k_i=1 a_i u_{n-i} + g(n) & \textsc{Equation générale}
		\end{array} \right.
	\end{displaymath}

	Une équation est \textbf{homogène} si $\forall n g(n) = 0$. La solution générale est une suite satisfaisant uniquement l'équation générale. Une
	solution particulière est une solution générale satisfaisant aussi des conditions initiales.

	\subsection{Équations récurrentes linéaires homogènes d'ordre 1}
	\paragraph{Proposition} La solution particulière de l'équation : 
	\begin{displaymath}
		\left\{ \begin{array}{lll}
			u_0 &=& c\\
			u_n &=&  a u_{n-1}
		\end{array} \right.
	\end{displaymath}
	est $u_n=C a^n$ (c'est une suite géométrique)

	\subsection{Équations récurrentes linéaires non-homogènes d'ordre 1}
	On ne sait traiter facilement que les équations dans lesquelles le second membre $g(n)$ est un polynôme ou une exponentielle. Pour cela, on
	<<dérive>> l'équation pour faire baisser le degré du polynôme jusqu'à arriver à 0.

	\exemple{\textbf{Le tri fusion}\\
	On a une équation qui n'est pas homogène: $$V_n = 2V_{n-1} + C 2^n$$
	Donc, au rang $n+1$, on a aussi $$V_{n+1} = 2 V_n + C \times 2 ^{n+1}$$

	Pour éliminer la partie non-homogène, on enlève 2 fois la première équation à la seconde.
	\begin{eqnarray*}
		V_{n+1} -2V_n &=& 2V_n - 4 V_{n-1}\\
		V_{n+1} &=& 4V_n - 4V_{n-1}
	\end{eqnarray*}
	}

	\subsection{Recherche d'une solution générale pour les équations récurrentes linéaires homogènes d'ordre 2}
	Une équation récurrente homogène d'ordre 2 est de la forme 
	\begin{displaymath}
		\left\{ \begin{array}{lll}
			u_0 &=&  C_0\\
			u_1 &=&  C_1\\
			u_n &=& a_1 u_{n-1} + a_2 u_{n-2}
		\end{array} \right.
	\end{displaymath}
	On peut obtenir ce type d'équation indirectement lorsque l'on a réduit une équation d'ordre 1 à une équation homogène d'ordre 2.
	\exemple{ L'équation récurrente linéaire homogène d'ordre 2 de Fibonacci
	\begin{displaymath}
		\left\{ \begin{array}{lll}
			U_0 &=&  1\\
			U_1 &=&  1\\
			U_n &=&  U_{n-1} + U_{n-2}
		\end{array} \right.
	\end{displaymath}
	ON résoud ces équations d'ordre 2 comme des équations d'ordre 1 : On cherche une solution générale de la forme $\lambda r^n$. Une telle solution
	vérifie, pour le cas de la suite de Fibonacci : $\forall n \geq 2$, $\lambda r^n = \lambda r^{n-1} + \lambda r^{n-2}$

	Soit, en divisant par $\lambda r^{n-2}$ $$r^2 = r + 1$$

	Autrement dit, $r$ est une racine du polynôme $P(x) = x^3 -x - 1$.
	}
	\paragraph{Définition}
	Le polynôme caractéristique d'une équation récurrente homogène d'ordre $k$
	$$V_{n+k} + a_1 V_{n+k+1} + \cdots + a_kV_n = 0$$ est le polynôme $P(x) = x^k + a_1x^{k-1}+\ldots+a_{k-1}+a_k$

	\paragraph{Théorème} Si $r$ est une racine du polynôme caractéristique d'une équation récurrente linéaire homogène, alors pour toute constante
	$\lambda$, toute suite de la forme $\{\lambda r^n\}_{n \geq 0}$ est une solution générale de cette équation.

	Dans le cas de la suite de Fibonacci, on calcule le discriminant $\Delta=5$ et on trouve les deux racines $r_1 = \frac{1-\sqrt{5}}{2}$ et 
	$r_2=\frac{1+\sqrt{5}}{2}$

	\paragraph{Cas des racines doubles} Si le discriminant $\Delta = 0$, alors le polynome n'a qu'une seule racine (de multiplicité 2). En remarquant
	que $r$ racine double de P(x) implique que $r$ est aussi une racine de $P'(x)$ on peut démontrer que $\{n\lambda r^n\}_{n \geq 0}$ est aussi une
	solution de l'équation récurrente.

	\paragraph{Théorème} Les solutions générales d'une équation récurrente linéaire homogène d'ordre 2 dont le polynmoe cractéristique de deux racines
	$r_1$ et $r_2$ sont :
	\begin{itemize}
		\item Si $r_1 \neq r_2$ : $\{ \lambda_1 r_1^n + \lambda_2 r_2^n\}_{n \geq 0}$ pour toute constantes $\lambda_1$, $\lambda_2$
		\item Si $r_1 = r_2$ :  $\{(\lambda_1 + \lambda_2 \times n)r_1^n\}_{n \geq 0}$ pour toute constantes $\lambda_1$, $\lambda_2$
	\end{itemize}

	\paragraph{Preuve dans le cas d'une racine double}
	Soit l'équation $u_{n+2} + a U_{n+1} + bu_n = 0$ et soit $r$ une racine double du polynome caractéristique.\\  
	$P(x) = x^2 + ax + b$, donc $r$ est aussi une racine de $P'(x)= 2x+a$. La suite $\{nr^n\}_{n\geq 0}$ est une solution de l'équation car
	\begin{eqnarray*}
	(n+2)r^{n+2} + a(n+1)r^{n+1} + bnr^n &=& n(r^{n+2} + ar^{n+1} + br^n) + 2r^{n+2} + ar^{n+1}\\
	&=& r^n[n(r^2 + ar + b) + r(2r+a)]\\
	&=& 0
	\end{eqnarray*}
	\subsection{Recherche de solutions particulières pour les équations récurrentes linéaires homogènes d'ordre 2}<++>
	Dans le cas de la suit ede Fibonacci, on cherche une solution particulières satisfaisant les conditions initialies et qui est de la forme
	$\lambda_1r_1^n + \lambda_2r_2^n$ où $r_1 = \frac{1-\sqrt{5}}{2}$ et $r_2 = \frac{1+\sqrt{5}}{2}$

	Donc on cherche $\lambda_1$, $\lambda_2$ tels que
	\begin{eqnarray*}
		u_0 &=&  1 = \lambda_1 r^0_1 + \lambda_2r_2^0 = \lambda_1 + \lambda_2\\
		u_1&=&  1 = \lambda_1 r^1_1 + \lambda r^1_2 = \frac{\lambda_1 + \lambda_2}{2} + \frac{\lambda_2 - \lambda _ 1}{2} \times \sqrt{5}\\
	\end{eqnarray*}
	\begin{displaymath}
		\Rightarrow 
		\left\{ \begin{array}{lll}
			1 &=&  \lambda_1 + \lambda_2\\
			\frac{1}{2} &=& \frac{\lambda_2 - \lambda_1}{2}\sqrt{5}
		\end{array} \right.
		\Rightarrow 
		\left\{ \begin{array}{lll}
			1 &=&  \lambda_1 + \lambda_2\\
			\frac{1}{\sqrt{5}} &=& \lambda_2 - \lambda_1
		\end{array} \right.
		\Rightarrow 
		\left\{ \begin{array}{lll}
			\lambda_2 &=&  \frac{1+\frac{1}{\sqrt{5}}}{2}\\
			\lambda_1 &=&  \frac{1-\frac{1}{\sqrt{5}}}{2}
		\end{array} \right.
	\end{displaymath}
	Au final, on trouve la solution particulière : 
	$$U_n = \frac{\sqrt{5}-1}{2\sqrt{5}}(\frac{1-\sqrt{5}}{2}) + \frac{\sqrt{5}+1}{2\sqrt{5}}(\frac{1+\sqrt{5}{2})^n$$
	
	\subsubsection{Résumé de la méthode pour les équations homogènes d'ordre 2}
	Pour résoudre l'équation $u_n = aU_{n-1} + bu_{n-2}
	\begin{enumerate}
		\item On calcule le polynome caractéristique $P(x) = x^2 - ax -b$
		\item On calcul les racines (eventuellement complexes)
			$r_1$ et $r_2$ de $P$
		\item On cherche les coefficients lambda_1 \lambda_2 tels que \lambda_1r_1^n + \lambda_2r_2@n
			satisfaisant les CI
	\end{enumerate}<++>


