\chapter{Complexité d'algorithmes définis par récurrence}
	\section{Exemple introductif : Tri fusion}
\'Etant donné un tableau T, on note T[i:j] le sous tableau de T qui va de la case i à la case j. L'algorithme de tri fusion utilise une procédure
\texttt{fusion(T,i,j,k)}. On suppose que les deux sous tableaux T[i:j] et T[j+1:k] sont déjà triés. En temps $\Theta(n)$, où $n=k-i+1$, la procédure
fusion produit le sous tableau T[i:k] trié à partir de la fusion de ces deux tableaux.

\lstinputlisting[language=algo, caption=Algorithme du tri fusion]{triFusion.algo}
	\section{Méthode naïve d'analyse de complexité}
	Soit un temps maximal d'exécution de tri fusion sur un tableau de longueur $n$.

	D'après l'algorithme, on a $$U_n = U_{\frac{n}{2}} + U_{\frac{n}{2}} \Theta(n)$$
	et $u_1 = 0$

	Pour simplifier la récurrence on suppose que $n$ est pair, et donc $U_n = 2U_{\frac{n}{2}} + \Theta(n)$

	La méthode naïve consiste à deviner la solution, ici on devine $U_n \leq c n \log_2 n$. On suppose $U_{\frac{n}{2}} \leq C \frac{n}{2} \log_2
	\frac{n}{2}$ et on essaye d'en déduire $U_n \leq c n \log_2 n$
	\begin{eqnarray*}
		U_n = 2U_{\frac{n}{2}} + cn &\leq& 2c \frac{n}{2} \log_2 \frac{n}{2} + cn \\&&= cn(\log_2 n -1) + cn = cn\log_2 n
	\end{eqnarray*}

	Puisque $u_1=0 \leq c 1 \log_2 1$, on en déduit $\forall n, i_n \leq cn \log_2 n$

	\subsection{Résumé de la méthode naïve}
	Pour une équation récurrente $u_n = f_n(U_{n-1}, \cdots, u_1)$ où f est une fonction monotone croissante
	\begin{enumerate}
		\item On devine une fonction $g$
		\item On suppose que $\forall n < 1$ on a $U_n \leq g(m)$
		\item On montre $U_n = f_n(U_{n-1},\cdots,u_1 \leq f_n(g(n-1), \cdots, g(1)) \leq g(n)$
		\item On conclut par récurrence que $\forall n$ on a $U_n \leq g(n)$
	\end{enumerate}
	\subsection{Exemples d'application}
	On commence par une \textbf{mauvaise} utilisation. Soit l'équation $U_n = 2 U_{\frac{n}{2}}$. L'intuition $U_n \leq kn$ n'est pas correcte.

	En effet, en remplaçant on obtient : 
	\begin{eqnarray*}
		n_n &=& 2U_{\frac{n}{2}}+1\\
		&=& 2k \frac{n}{2} + 1\\
		&=& kn + 1
	\end{eqnarray*}
	
	La bonne intuition est $u_n \leq kn - b$. En remplaçant on obtient : 
	\begin{eqnarray*}
		u_n = 2U_{\frac{n}{2}} + 1\\
		&\leq& 2(k\frac{n}{2} - b) + 1= kn - 2b + 1\\
		&\leq& kn -b\textrm{ Si } b \geq 1
	\end{eqnarray*}

	\subsection{Réduction à des formes simples}
	Lors de l'analyse d'algorithmes récursifs, on rencontre souvent des équations récurrentes de la forme $$u_n = aU_{\frac{n}{2}}+b,$$ où a et b sont des
	constantes. Par exemple le tri fusion.

	Pour convertir ce type de récurrence en une forme affine $u'_n = a'u'_{n-1}+b'$, on pose
	$$v_k = U_{2^k}$$
	Autrement dit, on étudiera la suite $\{u_n\}_{n \geq 0}$ uniquement sur les puissances de 2.
	\\
	Par exemple, pour le tri fusion, en remplaçant $n$ par $2^k$, 
	\begin{eqnarray*}
		U_{2^k} &=& 2U_{\frac{2^k}{2}} + C2^k\\
		\textrm{donc }V_k &=& 2v_{k-1} + c2^k
	\end{eqnarray*}

	\section{Équation récurrentes linéaires}
	\paragraph{Définition} Une équation récurrente linéaire à coefficients constants d'ordre $k$ est une équation de la forme 
	\begin{displaymath}
		\left\{ \begin{array}{llll}
			u_1 &=& C_i (O \leq i \leq k-1) & \textsc{Conditions initiales (CI)}\\
			u_n &=&  \sum^k_i=1 a_i u_{n-i} + g(n) & \textsc{Equation générale}
		\end{array} \right.
	\end{displaymath}

	Une équation est \textbf{homogène} si $\forall n g(n) = 0$. La solution générale est une suite satisfaisant uniquement l'équation générale. Une
	solution particulière est une solution générale satisfaisant aussi des conditions initiales.

	\subsection{Équations récurrentes linéaires homogènes d'ordre 1}
	\paragraph{Proposition} La solution particulière de l'équation : 
	\begin{displaymath}
		\left\{ \begin{array}{lll}
			u_0 &=& c\\
			u_n &=&  a u_{n-1}
		\end{array} \right.
	\end{displaymath}
	est $u_n=C a^n$ (c'est une suite géométrique)

	\subsection{Équations récurrentes linéaires non-homogènes d'ordre 1}
	On ne sait traiter facilement que les équations dans lesquelles le second membre $g(n)$ est un polynôme ou une exponentielle. Pour cela, on
	<<dérive>> l'équation pour faire baisser le degré du polynôme jusqu'à arriver à 0.

	\exemple{\textbf{Le tri fusion}\\
	On a une équation qui n'est pas homogène: $$V_n = 2V_{n-1} + C 2^n$$
	Donc, au rang $n+1$, on a aussi $$V_{n+1} = 2 V_n + C \times 2 ^{n+1}$$

	Pour éliminer la partie non-homogène, on enlève 2 fois la première équation à la seconde.
	\begin{eqnarray*}
		V_{n+1} -2V_n &=& 2V_n - 4 V_{n-1}\\
		V_{n+1} &=& 4V_n - 4V_{n-1}
	\end{eqnarray*}
	}

	\subsection{Recherche d'une solution générale pour les équations récurrentes linéaires homogènes d'ordre 2}
	Une équation récurrente homogène d'ordre 2 est de la forme 
	\begin{displaymath}
		\left\{ \begin{array}{lll}
			u_0 &=&  C_0\\
			u_1 &=&  C_1\\
			u_n &=& a_1 u_{n-1} + a_2 u_{n-2}
		\end{array} \right.
	\end{displaymath}
	On peut obtenir ce type d'équation indirectement lorsque l'on a réduit une équation d'ordre 1 à une équation homogène d'ordre 2.
	\exemple{ L'équation récurrente linéaire homogène d'ordre 2 de Fibonacci
	\begin{displaymath}
		\left\{ \begin{array}{lll}
			U_0 &=&  1\\
			U_1 &=&  1\\
			U_n &=&  U_{n-1} + U_{n-2}
		\end{array} \right.
	\end{displaymath}
	On résout ces équations d'ordre 2 comme des équations d'ordre 1 : On cherche une solution générale de la forme $\lambda r^n$. Une telle solution
	vérifie, pour le cas de la suite de Fibonacci : $\forall n \geq 2$, $\lambda r^n = \lambda r^{n-1} + \lambda r^{n-2}$

	Soit, en divisant par $\lambda r^{n-2}$ $$r^2 = r + 1$$

	Autrement dit, $r$ est une racine du polynôme $P(x) = x^3 -x - 1$.
	}
	\paragraph{Définition}
	Le polynôme caractéristique d'une équation récurrente homogène d'ordre $k$
	$$V_{n+k} + a_1 V_{n+k+1} + \cdots + a_kV_n = 0$$ est le polynôme $P(x) = x^k + a_1x^{k-1}+\ldots+a_{k-1}+a_k$

	\paragraph{Théorème} Si $r$ est une racine du polynôme caractéristique d'une équation récurrente linéaire homogène, alors pour toute constante
	$\lambda$, toute suite de la forme $\{\lambda r^n\}_{n \geq 0}$ est une solution générale de cette équation.

	Dans le cas de la suite de Fibonacci, on calcule le discriminant $\Delta=5$ et on trouve les deux racines $r_1 = \frac{1-\sqrt{5}}{2}$ et 
	$r_2=\frac{1+\sqrt{5}}{2}$

	\paragraph{Cas des racines doubles} Si le discriminant $\Delta = 0$, alors le polynôme n'a qu'une seule racine (de multiplicité 2). En remarquant
	que $r$ racine double de P(x) implique que $r$ est aussi une racine de $P'(x)$ on peut démontrer que $\{n\lambda r^n\}_{n \geq 0}$ est aussi une
	solution de l'équation récurrente.

	\paragraph{Théorème} Les solutions générales d'une équation récurrente linéaire homogène d'ordre 2 dont le polynôme caractéristique de deux racines
	$r_1$ et $r_2$ sont :
	\begin{itemize}
		\item Si $r_1 \neq r_2$ : $\{ \lambda_1 r_1^n + \lambda_2 r_2^n\}_{n \geq 0}$ pour toute constantes $\lambda_1$, $\lambda_2$
		\item Si $r_1 = r_2$ :  $\{(\lambda_1 + \lambda_2 \times n)r_1^n\}_{n \geq 0}$ pour toute constantes $\lambda_1$, $\lambda_2$
	\end{itemize}

	\paragraph{Preuve dans le cas d'une racine double}
	Soit l'équation $u_{n+2} + a U_{n+1} + bu_n = 0$ et soit $r$ une racine double du polynôme caractéristique.\\  
	$P(x) = x^2 + ax + b$, donc $r$ est aussi une racine de $P'(x)= 2x+a$. La suite $\{nr^n\}_{n\geq 0}$ est une solution de l'équation car
	\begin{eqnarray*}
	(n+2)r^{n+2} + a(n+1)r^{n+1} + bnr^n &=& n(r^{n+2} + ar^{n+1} + br^n) + 2r^{n+2} + ar^{n+1}\\
	&=& r^n[n(r^2 + ar + b) + r(2r+a)]\\
	&=& 0
	\end{eqnarray*}
	\subsection{Recherche de solutions particulières pour les équations récurrentes linéaires homogènes d'ordre 2}
	Dans le cas de la suite de Fibonacci, on cherche une solution particulières satisfaisant les conditions initiales et qui est de la forme
	$\lambda_1r_1^n + \lambda_2r_2^n$ où $r_1 = \frac{1-\sqrt{5}}{2}$ et $r_2 = \frac{1+\sqrt{5}}{2}$

	Donc on cherche $\lambda_1$, $\lambda_2$ tels que
	\begin{eqnarray*}
		u_0 &=&  1 = \lambda_1 r^0_1 + \lambda_2r_2^0 = \lambda_1 + \lambda_2\\
		u_1&=&  1 = \lambda_1 r^1_1 + \lambda r^1_2 = \frac{\lambda_1 + \lambda_2}{2} + \frac{\lambda_2 - \lambda _ 1}{2} \times \sqrt{5}\\
	\end{eqnarray*}
	\begin{displaymath}
		\Rightarrow 
		\left\{ \begin{array}{lll}
			1 &=&  \lambda_1 + \lambda_2\\
			\frac{1}{2} &=& \frac{\lambda_2 - \lambda_1}{2}\sqrt{5}
		\end{array} \right.
		\Rightarrow 
		\left\{ \begin{array}{lll}
			1 &=&  \lambda_1 + \lambda_2\\
			\frac{1}{\sqrt{5}} &=& \lambda_2 - \lambda_1
		\end{array} \right.
		\Rightarrow 
		\left\{ \begin{array}{lll}
			\lambda_2 &=&  \frac{1+\frac{1}{\sqrt{5}}}{2}\\
			\lambda_1 &=&  \frac{1-\frac{1}{\sqrt{5}}}{2}
		\end{array} \right.
	\end{displaymath}
	Au final, on trouve la solution particulière : 
	$$U_n = \frac{\sqrt{5}-1}{2\sqrt{5}}(\frac{1-\sqrt{5}}{2}) + \frac{\sqrt{5}+1}{2\sqrt{5}}(\frac{1+\sqrt{5}}{2})^n$$
	
	\subsubsection{Résumé de la méthode pour les équations homogènes d'ordre 2}
	Pour résoudre l'équation $u_n = aU_{n-1} + bu_{n-2}$
	\begin{enumerate}
		\item On calcule le polynôme caractéristique $P(x) = x^2 - ax -b$
		\item On calcul les racines (éventuellement complexes)
			$r_1$ et $r_2$ de $P$
		\item On cherche les coefficients $\lambda_1$ $\lambda_2$ tels que $\lambda_1r_1^n + \lambda_2r_2^n$ satisfaisant les CI
	\end{enumerate}

	\subsection{Équations récurrentes d'ordre $k$}
	Pour les équations récurrentes homogènes d'ordre k, les considérations sur le polynôme caractéristique et ses racines restent valables. La difficulté
	est calculatoire car il faut trouver les racines d'un polynôme de degré k. Mais lorsque l'équation a été obtenu en éliminant  la partie
	non-homogène, les coefficients utilisés sont des solutions. 
	\exemple{Pour l'algorithme de Strasser, on a obtenu l'équation en faisant
	$$E_n-GE_{n-1}$$ où $E_n$ désigne l'équation de rang $n$

	$\Rightarrow$ 4 est une racine du polynôme caractéristique.
	}

	En cas de racine d'ordre $m$, on peut montrer par récurrence que $\{n^j\alpha^n\}_{n\geq 0}$ est une solution de l'équation récurrente homogène pour
	tout $j=0,\cdots,m-1$. Ceci nous permet d'avoir $k$ variables dans le système d'équation linéaires dérivées des CI.

	Le théorème suivant généralise le théorème 2 au cas de récurrences homogènes d'ordre $k > 2$ et prend en compte directement le second membre.
	\paragraph{Théorème 3}
	Supposons que le polynôme caractéristique de la récurrence homogène $u_n = au_{n-1} + \cdots + a_ku_{n-k}$ admet $p$ racines $ri(i=1,\cdots,p)$ de
	multiplicité $mi(i=1,\cdots,p)$. Alors la solution de la récurrence 
	$$u_n = a_1u_{n-1}+\cdots+a_ku_{n-k} + \sum^t_{i=1} b_i^n P_i(n)$$
	où $p_i$ est un polynôme de degré $d_i$) est donnée par $$\sum^t_{i=1}b_i^n Q_i(n)\footnote{Partie de la solution qui prend en compte le second membre} + 
	\sum_{i \in \{1,\ldots,p\}}r_i^n R_i(n)\footnote{Solution pour la récurrence homogène}$$
	tel que $r_i \not\in \{b_1,\cdots,b_t\}$

	Où 
	\begin{displaymath}
		\textrm{deg}(Q_i) = \left\{ \begin{array}{lll}
			d_i & si & b_i \not\in \{r_i,\cdots,r_p \}\\
			d_i+m_j & si& b_i = r_j
		\end{array} \right.
	\end{displaymath}

	Et $\textrm{deg}(R_i) = m_i-1$

	On obtient les polynômes $Q_i$ et $R_i$ à partir des CI et par identification des coefficients des termes $b_i^n n^j$ dans la récurrence.

	\remarque{Dans le théorème 2, il n'y avait de second membre (t=0) et les polynômes $R_i$ étaient de la forme $\lambda_1$ ou $\lambda_1 = \lambda_2 n$
	}
	\begin{displaymath}
		\left. \begin{array}{lll}
			u_n &=&  u_{n-1} + 1n^3\\
			u_{n-1} &=&  u_{n-2} + 1
		\end{array}
		\right\}
		u_n - u_{n-1} = u_{n-1} - u_{n-2}
	\end{displaymath}
	\exemple{
	$$T(n) = 2T(\frac{n}{2}) + n ; T(1) = 1$$
	Après changement de variable $n=2^k$, $u_k = T(n)$, nous avons $u_k = 2u_{k-1}+2^k$.

	Ici le second membre $$\sum^t_{i=1} b_i^k Pi(k) = 2^k$$

	Donc $T=1$, $P_i(k)=1$, $b_i=2$

	Le polynôme caractéristique $P(x)=x-2$. La seule racine est $r_i = 2$.
	Donc la solution particulière  est de la forme $2^k(q_0 + q_i^k)$ car $\textrm{deg}(Q_i) = \textrm{deg}{P_i}  +\textrm{multiplicité}  = 0 + 1$, et
	cette solution satisfait la CI et la récurrence $1 = T(1) = u_0$
	$$2^k(q_0+q_1k) = 2 \times 2^{k-1}(q_0+q_i(k-1))$$	
			D'où $q_0 = 1$, $q_1 = 1$ donc $u_n = 2^k(1+k)$ et $T(n) = u_k = n(1+\log_2n)$
	}

	\subsection{Théorème pour les récurrences par divisions}
	Le théorème suivant nous donne directement l'ordre de grandeur de la solution en fonction des coefficients de l'équation récurrente.

	\paragraph{Théorème 4} Soient $a \geq 1$, $b > 1$ deux constantes, $f(n)$ une fonction, et $\{t(n\}_{n\geq 0}$ une suite vérifiant l'équation 
	$T(n) = aT(\frac{n}{b}) + f(n)$

	On a pour $\epsilon > 0$
\begin{itemize}
	\item Si $f(n) = 0(n^{\log_b a - \epsilon}$ alors $T(n) = \Theta(n^{\log_b a})$
	\item Si $f(n) = \Theta(n^{\log_b a})$ alors $T(n) = \Theta(n^{\log_b a . \log n})$
	\item Si $f(n) = \Omega(n^{\log_b a + \epsilon}$ et $af(\frac{n}{b}) \leq cf(n)$ pour une constante $c >1$, alors $T(n) = \Theta(f(n))$
\end{itemize}
