\chapter{Complexité des boucles}
	\section{Complexité de boucles ``pour''}
\begin{lstlisting}[language=algo]
	pour i:= 1 a n faire
		-- Corps de la boucle
	fin pour;
\end{lstlisting}
Notions $I_i$ la i\ieme{} itération (les instructions executées lors du i\ieme{} passage dans la boucle) et $T(I_i)$ sa complexité temporelle.:

%$$	T_{\moy}(n) = T_\max(n) = \sum^{n}_{i=1} T(I_i)\\$$
Par exemple, $T_\moy(n) = T\max(n) = \Theta(n)$ si $T(I_i)$ constant et $= \Theta(n^2)$ si $T(I_i) = an+b$ (boucle imbriquée.

\subsection{Exemple}
Calculer $A=BC$, le produit de 2 matrics. Rappel:
$$a_{ik} = \sum^n_{j=1} = b_{ij}C_{ji}$$
\begin{lstlisting}[language=algo]
	pour i = 1 a n faire
		pour k = 1 a n faire
				aik  0
				pour j = 1 a n faire
					aik = aik + bij * cjk;
				fin pour;
			fin pour;
		fin pour;
	fin pour;
\end{lstlisting}

$$T_{\moy}(n) = T_{\max}(n) = \sum^n_{i=1}\sum^n_{k=1}(1+n) = \Theta(n^3)$$

\newpage
\section{Complexité de boucles ``tant que''}

\begin{lstlisting}[language=algo]
	tantque C faire
		-- Corps de la boucle
	fin tantque;
\end{lstlisting}
$$T_{\moy} = 1 + \sum^\infty_{i=1}\textrm{Prob}$$
On ajoute 1 pour le test de la condition C lorsque C = faux.

Soit $E_i$ l'événement C = Vrai au début de $i_i$\\
Si $\forall i, j E_i, E_j$ sont indépendantes et $\prob(E_i) = p < 1$, où p est une constante, alors\\ prob(on exécute $I_i$) = $\prob(E_1\cdots E_i) = p^i$ d'où
$$T_{\moy}(n) = 1 + \sum^{\infty}_{i=1}p^i*T(I_i)$$
Si $T(I_i)$ est constante, alors $$T_{\moy}(n) = \Theta (1 + \frac{p}{1-p}) = \Theta (\frac{1}{1-p}) = \Theta(1)$$

\subsection{Exemple}
Comparaison de 2 suites $\{A_i\},\{b_i\}$.
\begin{lstlisting}[language=algo]
	i := 1;
	tantque (ai = bi et i <= n) faire
		i := i + 1;
	fin tantque;
\end{lstlisting}
$T_{\moy}(n) = \Theta(1)$ si les suites sont indépendantes et aléatoires.
