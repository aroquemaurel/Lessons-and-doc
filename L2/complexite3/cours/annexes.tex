\chapter{Exercices}
\section{TD 1}
\subsection{Lesquelles des affirmations suivantes sont vraies ?}
\begin{enumerate}
	\item $n^2.5 = \Theta(n^3)$ : Faux 
	\item $n^2.5 = O(n^3)$ : Vrai 
	\item $n^2.5 = \Omega(n^3)$ : Faux
	\item $log_2(2n) = \Theta(\log n)$ : Vrai
	\item Vrai
	\item Faux
\end{enumerate}
\subsection{Une seule des afirmations suivantes est vraie. Laquelle ?}
Réponse D
\subsection{Une seule des affirmations suivantes est vraie. Laquelle ?}
Réponse C
$n + n\log_2 n \leq 2 n\log_2 n= \Theta(n \log n)$
\remarque{On ne s'occupe pas des facteurs constants}
\subsection{}
Réponse D
\subsection{Laquelle des affirmations suivantes sont vraies}
\begin{enumerate}
	\item $\max{(f(n),g(n))} = \Theta(f(n)+g(n))$ Vrai : $\max (f(n), g(n)) \leq f(n) + n(n) \leq 2 \max(f(n),g(n))$
	\item Vrai : $\frac{1}{c}f(n) \leq g(n)$ et $g(n) \leq \frac{1}{2}f(n)$
	\item Vrai : $\forall n \geq n_0 : f(n) \leq c g(n)$
	\item Faux
	\item Vrai
	\item Faux \\
		$$g(n) = 2n, f(n) = n\\ g(n) = O(f(n))\\ 2^{g(n)} = 2^{2n} = (2^n)^2$$
\end{enumerate}
\subsection{Lesquelles des affirmations suivantes sont vraies ?}
Réponse D. 
\begin{eqnarray*}
	f(n) \leq c_1 g(n)&,& g(n) \leq c_2 f(n)\\
	\frac{1}{c_2} \leq \frac{f(n)}{g(n)} \leq c_1.1 &\Rightarrow& \frac{f(n)}{g(n)} = \Theta(1)
\end{eqnarray*}
\subsection{Simplifiez les expressions suivantes}
\begin{enumerate}
	\item $O(4n^2+3n^2+7\log_2(n^n)) = O(n^3)$ 
	\item $\Theta(n\log_2 n + 17n + 2n^3 = \Theta (n^2)$ 
	\item $\Omega(4n^2 + 3n^3) = \Omega(n^3)$ 
	\item $O(2^{n\log_3}n + 3\log_2 n!) = O(n^2)$
	\item $O(2\log_3 n + 3\log_2 n + 6) = O(\log n)$
\end{enumerate}
\subsection{Classez les fonctions suivantes dans l'ordre croissant d'ordre de grandeur}
\begin{enumerate}
\item $4n \log_2 n + 4n$
\item $2n \log_2 n + 4n$
\item $n^2 \log_e n$
\end{enumerate}

\subsection{}
\begin{eqnarray*}
	\Theta(\frac{1}{1-p})\\
	p = 1 - (\frac{1}{6})^{n-1}\\
	\Theta(\frac{1}{1-(1-\frac{1}{6^{n-1}})}) = \Theta(\frac{1}{\frac{1}{6^{n-1}}}) = \Theta(6^{n-1}) = \Theta(6^n)
\end{eqnarray*}
Donc réponse D.
\subsection{}
\begin{enumerate}
	\item[a]$\Theta(1)$
	\item[b] $\Theta(1)$
	\item[c] $\Theta(\log n)$
	\item[d] $\Theta(n\log n)$
	\item[e] $\Theta(n^3)$
	\item[f] $\Theta(n^4)$
\end{enumerate}


\subsection{16.}
\subsubsection{a}
\begin{displaymath}
		\left. \begin{array}{rrr}
			V_n &=& V_{n-1} + 1\\
			V_{n+1} &=& V_n + 1\\
		\end{array}\right\}
		\Rightarrow \left. \begin{array}{rrr}V_{n+1}-V_n &=& V_n - V_{n-1}\\V_{n+1} &=&  2V_n - V_{n-1}\end{array}\right.
\end{displaymath}

\subsubsection{b}
\begin{displaymath}
	\left. \begin{array}{rrr}
		T(n) &=&  7T(n-1) + 4^n\\
		T(n+1) &=& 7T(n) + 4^{n+1}\\
	\end{array}\right\} 
	T(n+1) - 4T (n) = 7T(n) - 28T(n-1)
\end{displaymath}

\subsection{17.}
\subsubsection{a}
\begin{eqnarray*}
V_{k+1} - 4V_k + 4V_{k-1} &=&  0\\
x^2 - 4x + 4 &=& 0
\end{eqnarray*}
$$\Delta = 16 - 16 = 0$$
$$r = \frac{4}{2} = 0$$
$$(\lambda_1 + \lambda_2 n)2^n$$
\subsubsection{b}
\begin{eqnarray*}
	V_{k+2} - 2V_{k+1} + V_k &=& 0\\
	x^2 - 2x +1
\end{eqnarray*}
$$\Delta = 0$$
$$r = 1$$
$$S = \lambda_1 + \lambda_2 n$$
\subsubsection{c}
