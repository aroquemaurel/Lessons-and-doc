\chapter{Introduction}
\minitoc
	\section{Complexité}
	On cherche à estimer le temps de calcul d'un algorithme A en fonction d'un paramètre n. Pour avoir une mesure indépendante de la machine, on identifie
	le temps de calcul avec le nombre d'instructions exécutées. 

	\exemple{Le paramètre n pourrait être la taille d'un tableau, par exemple.}

	Soit $D_i$ l'ensemble des données possibles telle que $n=i$. Pour $d \in D_i$ on notera $T(A,d)$ le nombre d'instructions exécutée pendant l'exécution de
	$A(d)$.\\
	On notera $\prob(d|i)$ la probabilité que les données soit $d$ étant donné qu'elles sont de taille $i$.

	\subsection{La complexité temporelle maximale} 
	La complexité temporelle maximale\footnote{Complexité dans le pire des cas} d'un algorithme A :
	$$T_{\max}(i) = \max_{d\in D_i}\{T(A,d)\}$$
	\subsection{La complexité temporelle moyenne}
	La complexité temporelle moyenne\footnote{Complexité dans le cas moyen} d'un algorithme A :
	$$T_{\moy} = \sum_{d \in D_i} \prob(d|i) \times T(A,d)$$
	\remarque{Pour pouvoir calculer $T_\moy$, il faut connaître la distribution des données, ce qui n'est pas toujours évident (par exemple en traitement d'image)}
	\subsection{La complexité temporelle minimale}
	La complexité temporelle minimale\footnote{Complexité dans le meilleur des cas} d'un algorithme A :
	$$T_{\min}(i) = \min_{d\in D_i}\{T(A,d)\}$$ 
	\remarque{Peu utilisé, sauf pour prouver qu'un algorithme est mauvais. Si la complexité temporelle minimale est 
		mauvaise même dans le meilleur des cas, alors l'algorithme n'est pas bon.}

	%			\remarque{$T_{\max}$ et $T_{\min}$ nous fournissent des bornes supérieures et inférieures.}

	\subsection{Compairaison de complexités en fonction de la machine}
	\begin{tabular}{| c | p{7cm} | p{7cm}|} 
		\hline
		\textbf{Complexité}& \multicolumn{2}{|c|}{\textbf{Nombre d'instructions pouvant executer la machine}}\\
		\hline
		 & $1\;000\;000$ & $1\;000\;000\;000\;000$\\
		\hline
		$n$ & $1\;000\;000$&$1\;000\;000\;000\;000$\\
		\hline
		$n \log_2 n$ &$64\;000$&$32\;000\;000\;000$\\
		\hline
		$n^2$ & $1\;000$&$1\;000\;000$\\
		\hline
		$n^3$ &$100$&$10\;000$\\
		\hline
		$2^n$ &$20$&$40$\\
		\hline
	\end{tabular}
	\section{Complexité asymptotique}
	Pour comparer des algorithmes, on ne s'intéresse qu'à leur comportements pour n grand. On cherche une mesure de complexité qui soit indépendante du langage de programmation et de la vitesse de la machine. \\ 
	$\Rightarrow$		On ne doit pas perdre en compte des facteurs constants.  \\
	$\Rightarrow$	Ordre de grandeur

	\subsection{La complexité asymptotique}
	La complexité asymptotique\footnote{Que ce soit maximale, moyenne ou minimale} est l'ordre de grandeur de sa limite lorsque $n \rightarrow \infty$

	\subsection{Notation} Soient $T$, $f$ des fonctions positives ou nulles. Rotations de grandeur de fonction asymptotiques.
	\paragraph{Grand O} $T = O(f)$ si $\exists c \in \mathbb{R}^{>0}$ et $n_0 \in \mathbb{N}$ tels que $\forall n \geq n_0$, $T(n) \leq cf(n)$.

	\paragraph{Grand Oméga} $T = \Omega(f)$ si $\exists c \in \mathbb{R}^{>0}$ et $n_0 \in \mathbb{N}$ tels que $i\forall n \geq n_0$, $T(n) \geq cf(n)$
	\paragraph{Petit O} $T = o(f)$ si $\frac{T(n)}{f(n)} \rightarrow O$ lorsque $n \rightarrow \infty$. 
	\remarque{T est négligeable devant f}

	\exemple{
	\begin{enumerate}
	\item $2n^2 + 5n + 10 = O(n^2)$\\
	Dans la définition $n_0 = 5$,$c=4$ : \\
	$\forall n \geq 5,\ 2n^2+5n+10 \leq 4n^2$
	\item $2n^2 + 5n + 10 = \Omega(n^2)$\\
	Dans la définition, $n_0 = 1$, $c = 2$\\
	$\forall n \geq 1,\ 2n^2 + 5n + 10 \geq 2n^2 \cdots$\\
	Donc $2n^2+5n+10 = \Theta (n^2)$
	\item $\frac{1}{5} + n = O(n\log_2 n)\ (n_0 = 2,\ c=2)$
	\item $\frac{1}{5} n \log_2 n + n = \Omega(n \log n)\ (n_0=1,c=\frac{1}{5})$
	\item $\forall k \geq 0$, $n^k = O(n^{k+1})$ mais $n^k \neq \Omega(n^{k+1})$
	\item $\forall a,b >1, \log_a n = \Theta (\log_b n)$ car $\log_a n = \frac{\log_b n}{\log_b a}$ et $\log_b a$ est une constante. $\Rightarrow$ On a pas besoin de préciser la base de logarithme dnas une complexité asymptotique
	\item $2n^2 + 5n + 10 = 2n^2 + 0(n^2)$
	\item Pour toute constante $c>0$, $C = \Theta(1)$
	\item $2^n = o(3^n)$
	\end{enumerate}
	}
	\remarque{\begin{enumerate}
	\item O et $\Omega$ sont des pré-ordres\footnote{Relations reflexives et transitives} :\\ $f = O(f)$ et $f = O(g)$ et $g = O(h)) \Rightarrow f = O(h)$
	\item $\Theta$ est une relation d'équivalence\footnote{relation reflexives, symétrique et transitive} : 
		$f = \Theta(g) \Leftrightarrow g = \Theta (f)$
	\end{enumerate}
	}
	\paragraph{Proposition}
	$$\textrm{Si } \lim_{n \rightarrow \infty} \frac{f(n)}{g(n)} = a > 0 \textrm{ Alors }f = \Theta (g)$$
	\remarque{La réciproque est fausse}

	\paragraph{Notation} $$f \sim g \Rightarrow \lim_{n\rightarrow \infty} \frac{f(n)}{g(n)} = 1$$
	\exemple{$(3n+1)^3 \sim 27n^3$}
	\section{Exemple de complexités d'algorithmes}
	\subsection{Le tri à bulles}
	\begin{eqnarray*}
		T_{\min} (n) &=& \Theta (n) \textrm{ Si le tableau est est déjà trié}\\
		T_{\max}(n) &=& \Theta(n^2) \textrm{ Si le tableau est trié en ordre décroissant}\\
		T_{\moy}(n) &=& T_{\max}(n) = \Theta(n^2)\\
	\end{eqnarray*}
	\subsection{Tri par fusion}
	\begin{eqnarray*}
		T_{\min}(n) = T_{\max}(n) = T_{\moy}(n) = \Theta (n\log n)
	\end{eqnarray*}
	\subsection{Tri rapide}
	\begin{eqnarray*}
		T_{\min}(n) = T_{\moy}(n) &=& \Theta (n \log n)\\
		T_{\max}(n) &=& \Theta(n^2)
	\end{eqnarray*}

	\section{Comportement symptotique de fonctions usuelles}
	Il y a quatre groupes importants de fonction positives croissantes.
	\begin{description}
		\item[Logarithmiques] $(\log n)^\sigma$ (où $\sigma > 0$), $\log\log n, \ldots$
		\item[Polynomiales] $n^\gamma$ (où $\gamma > 0$), $n^\gamma(\log n)^\gamma$(où $\gamma > 0$)
		\item[Exponentielles] $2^{\alpha n^\beta}$ (où $\alpha > 0$ et $0 < \beta \leq 1$), par exemple $2^n$, $4^n$, $2^{\sqrt{n}}$
		\item[Supraexponentielles] $n!$, $n^n$, $2^{n^2}$, \ldots
	\end{description}
	\remarque{
		Il existe des fonctions intermédiaires(par exemple $n^{\log_2 n}$) mais ces fonctions se rencontrent très rarement dans l'analyse de complexité d'algorithmes
	}

	% Courbes. CF Feuille de TD
	\begin{eqnarray*}
		\lim_{n\rightarrow \infty} \frac{n^b}{a^n} = O \textrm{ Pour toutes constantes} a,b \textrm{ avec } a > 1 \textrm{)}\\
		n^b = o(a^n)\\
		\lim_{n\rightarrow \infty} \frac{(\log_2 n)^{\sigma}}{n^\sigma} = O\\
		\Rightarrow (\log_2 n)^\sigma = o(n^\sigma)\\
	\end{eqnarray*}


	\subsection{La formule de Stisling}
	\begin{eqnarray*}
		n! \sim \sqrt{2\pi n}(\frac{n}{e})^n\\
		\Rightarrow n! = o(n^n) \textrm{ et } n! = \Omega(2^n)
	\end{eqnarray*}
	On peut aussi en déduire :
	$$\log(n!) \sim n\log n$$

