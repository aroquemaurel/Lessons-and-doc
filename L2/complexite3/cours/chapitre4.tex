\chapter{Structure de données et complexité}
\section{Les principales structures de données}
Les opérations les plus courantes: 
\begin{description}
	\item[I]Insertion
	\item[A] Test d'appartenance
	\item[S] Suppression
	\item[SD, SP, SM] Suppression du dernier élément du premier élément, de l'élément minimum.
\end{description}

\begin{tabular}{|c|c|c}
	\textbf{TDA}\footnotemark & \textbf{Opération de base} & \textbf{Réalisation pour laquelle ces opérations sont on en O(log n)}\\
	\hline
	Pile & I, SD & liste chainées\\
	\hline
	File & I, SD & liste chainée avec deux pointeurs début et fin\\
	\hline
	index statique & A & tableau trié\\
	\hline
	file de priorité & I, SM & tas\\
	\hline
	ensemble & A, I, S & table d'hâchage\footnotemark\\
	\hline
	ensemble trié & A, I, S, SM & ABR, arbre rouge-noir, arbre AVL, B-arbre\footnotemark\\
\end{tabular} 
\footnotetext{Type abstrait de données}
\footnotetext{Complexité moyenne O(log n)}
\footnotetext{Complexité moyenne O(log n)}
