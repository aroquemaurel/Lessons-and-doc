\documentclass[12pt,a4paper,openany]{report}


\usepackage{lmodern}
\usepackage{xcolor}
\input{/home/satenske/cours/includesLaTeX/couleurs.tex}

\usepackage[utf8]{inputenc}
\usepackage[T1]{fontenc}
\usepackage[francais]{babel}
\usepackage[top=1.7cm, bottom=1.7cm, left=1.7cm, right=1.7cm]{geometry}
\usepackage{verbatim}
\usepackage[urlbordercolor={1 1 1}, linkbordercolor={1 1 1}, linkcolor=vert1, urlcolor=bleu, colorlinks=true]{hyperref}
\usepackage{tikz} %Vectoriel
\usepackage{listings}
\usepackage{fancyhdr}
\usepackage{multido}
\usepackage{amssymb}

\newcommand{\titre}{Logique --- TD}

\newcommand{\pole}{}
\newcommand{\sigle}{logique}

\newcommand{\semestre}{3}

\input{/home/satenske/cours/includesLaTeX/listings.tex} %prise en charge du langage algo
\input{/home/aroquemaurel/cours/includesLaTeX/entete-l2-cours.tex}




%----------------------------------------------------------------------------------------
%	DEFINITION OF COLORED BOXES
%----------------------------------------------------------------------------------------

\RequirePackage[framemethod=default]{mdframed} % Required for creating the theorem, definition, exercise and corollary boxes

% Theorem box
\newmdenv[skipabove=7pt,
skipbelow=7pt,
backgroundcolor=black!5,
linecolor=ocre,
innerleftmargin=5pt,
innerrightmargin=5pt,
innertopmargin=5pt,
leftmargin=0cm,
rightmargin=0cm,
innerbottommargin=5pt]{tBox}

% Exercise box	  
\newmdenv[skipabove=7pt,
skipbelow=7pt,
rightline=false,
leftline=true,
topline=false,
bottomline=false,
backgroundcolor=ocre!10,
linecolor=ocre,
innerleftmargin=5pt,
innerrightmargin=5pt,
innertopmargin=5pt,
innerbottommargin=5pt,
leftmargin=0cm,
rightmargin=0cm,
linewidth=4pt]{eBox}	

% Definition box
\newmdenv[skipabove=10pt,
skipbelow=10pt,
rightline=false,
leftline=true,
topline=false,
bottomline=false,
linecolor=ocre,
innerleftmargin=5pt,
innerrightmargin=5pt,
innertopmargin=0pt,
leftmargin=0cm,
rightmargin=0cm,
linewidth=4pt,
innerbottommargin=0pt]{dBox}	

% Corollary box
\newmdenv[skipabove=7pt,
skipbelow=7pt,
rightline=false,
leftline=true,
topline=false,
bottomline=false,
linecolor=gray,
backgroundcolor=black!5,
innerleftmargin=5pt,
innerrightmargin=5pt,
innertopmargin=5pt,
leftmargin=0cm,
rightmargin=0cm,
linewidth=4pt,
innerbottommargin=5pt]{cBox}		

% Corollary box
\newmdenv[skipabove=7pt,
skipbelow=7pt,
rightline=true,
leftline=false,
topline=false,
bottomline=true,
linecolor=gray,
backgroundcolor=black!5,
innerleftmargin=5pt,
innerrightmargin=5pt,
innertopmargin=5pt,
leftmargin=0cm,
rightmargin=0cm,
linewidth=1pt,
innerbottommargin=5pt]{rBox}				  
		  

% Creates an environment for each type of theorem and assigns it a theorem text style from the "Theorem Styles" section above and a colored box from above
\newenvironment{theorem}{\begin{tBox}\begin{theoremeT}}{\end{theoremeT}\end{tBox}}
\newenvironment{example}{\begin{exampleT}}{\hfill{\tiny\ensuremath{\blacksquare}}\end{exampleT}}
\newenvironment{definition}{\begin{dBox}\begin{definitionT}}{\end{definitionT}\end{dBox}}
\newenvironment{attention}{\begin{eBox}\small}{\end{eBox}}				  	
\newenvironment{exemple}{\begin{cBox}\small}{\end{cBox}}	

%----------------------------------------------------------------------------------------
%	REMARK ENVIRONMENT
%----------------------------------------------------------------------------------------

\newenvironment{remarque}{\par\vskip10pt\small
\begin{rBox}
\begin{list}{}{
\leftmargin=35pt % Indentation on the left
\rightmargin=25pt}\item\ignorespaces % Indentation on the right
\makebox[-2.5pt]{\begin{tikzpicture}[overlay]
\node[draw=ocre!60,line width=1pt,circle,fill=ocre!25,font=\sffamily\bfseries,inner sep=2pt,outer sep=0pt] at (-15pt,0pt){\textcolor{ocre}{R}};\end{tikzpicture}} % Orange R in a circle
\advance\baselineskip -1pt}
{\end{list}\vskip1mm\end{rBox}\vskip5pt} % Tighter line spacing and white space after remark



\input{/home/satenske/cours/includesLaTeX/polices.tex}
\input{/home/satenske/cours/includesLaTeX/affichageChapitre.tex}

\newcommand{\dual}{^\circ}
\begin{document}
	\setcounter{tocdepth}{2}
	\setcounter{secnumdepth}{3}
	\maketitle
	\chapter{TD 1}
		\section{En vrac}
		\section{Les ensemble}
	%	\begin{description}
%			\item[1] $$X \in A$$
%			\item[2 $$(x \in E) \wedge (x \in F)$$
	%		\item[3] $$(x \in E) \wedge (x \not\in F)$$
%			\item[]  $$x \in (C(F)) \wedge(X \in E) = \neg(X \in P) \wedge (x \in E)$$
%			\item[J]  $$(E \cap Fà) C C(C(E) \cup C(F))$$		
%			\item[]  $$x \in(E \cup F)$$
%			\item[5]  
				\begin{eqnarray*}
					x &\in& C(E \cup C(F))\\
					\neg X &\in&(E\cup C(F))\\
					\not(X \in E &\wedge& X \in C(F))\\
					\neg(x \in E &\wedge& \neg (X \in F))\\
				\end{eqnarray*}
	%	\end{description}
		
			

	\subsection{Expression ensembliste}
	$$X \in (A \cup B)$$
	\subsubsection{Compl.}
	\begin{eqnarray*}
		C(C(A)) &=&A\\
		C(a\cup B) &=& C(A) \cap C(B)\\
		C(A \cap B) &=& C(A) \cup C(B)\\
	\end{eqnarray*}
	\subsubsection{Distrib.}
%	A \cap (B \cup C) = (A\cap B) \cup (A \cap C)\\
%	A \cup (B \

%
%
%
%
	\chapter{Métathéorie}
	\section{Induction}
	\subsection{4.}
		\subsubsection{Définition de $^\circ$}
		\begin{eqnarray*}
			(p)^\circ &=& \neg p\\
			(\bot)^\circ &=& \neg \bot\\
			(\neg A)^\circ &=& \neg(A)^\circ\\
		\end{eqnarray*}
		\subsubsection{Test de $^\circ$}
		\subsubsection{Preuve par induction}
		Montrer que $\forall F, F^\circ \equiv \neg F$\\
		\paragraph{Variable propositionnelle}
		\begin{eqnarray*}
			p\dual &\equiv& \neg p\\
			(p)\dual = \neg p &\equiv& \neg p\\
			\bot \dual &\equiv& \neg \bot\\
			(\bot)\dual = \neg \bot &\equiv& \neg \bot
		\end{eqnarray*}

		\paragraph{Négation}
		\begin{eqnarray*}
			A\dual &\equiv&\neg A\ (\textrm{Hypothèse})\\
			(\neg A)\dual &\equiv& \neg \neg A\\
			(\neg A)\dual = \neg(A\dual) &=& \neg \neg A\\
		\end{eqnarray*}

		\paragraph{Disjonction}
		\begin{eqnarray*}
			A\dual \equiv \neg A\\%hypothese
			B\dual \equiv \neg B\\ %hypothese
			\textrm{Montrer que }(A\vee B)\dual \equiv \neg(A \vee B)\\
			(A\vee B)\dual = A\dual \wedge B\dual = (\neg A)\wedge(\neg B) \equiv \neg (A\vee B)\\
		\end{eqnarray*}
		\paragraph{Conjonction}
		\begin{eqnarray*}
			A\dual = \neg A \\
			B\dual = \neg B\\ 
			\textrm{Montrer que} (A\wedge B)\dual = \neg(A \wedge B)\\
			(A\wedge B)\dual = A\dual \vee B\dual = \neg A \vee \neg B \equiv \neg(A\wedge B)
		\end{eqnarray*}

		\subsection{5.}
		\subsubsection{Définition de $\Delta A_n$}
		\begin{eqnarray*}
			\Delta_0 &=& A_0\\
			\Delta_{(n+1)} &=& A_{(n+1)} \wedge \Delta_n\\
			(\Delta_n &=& A_n \wedge \Delta_{(n+1)})\textrm{ Pour }n \neq 0\\
			\Gamma_0(B) &=& A_0 \rightarrow B\\
			\Gamma_{n+1}(B) &=& A_{(n+1)} \rightarrow \Gamma_n\\
		\end{eqnarray*}
		\subsubsection{Preuve par induction}
		Montrer que $\forall n (\Delta_n \rightarrow B) \equiv \Gamma_n (B)$
		\begin{eqnarray*}
			P(n) &=& \Delta_n \rightarrow B \equiv \Gamma_n (B)\\
		\end{eqnarray*}

		\subsection{$\textrm{subst}(A,B,p)$}
		\subsubsection{Définition} $\textrm{subst}(A,B,p)$
		\subsubsection{Trâce d'execution} $( ( p \vee q ) \wedge \neg(r \vee p)) [(r \vee p) / p]$
		\subsubsection{Preuve} $\forall F. nc(F) \leq nc(F[A/p])$~~~---~~~$A[B/p]$
		\paragraph{Cas $\bot$}
		\begin{eqnarray*}
			nc(\bot) &\leq& nc(\bot[A/p]\\
			0 &\leq& (\bot)\\
			0 &\leq& 0
		\end{eqnarray*}
		\remarque{
		nc est le nombre de connecteur d'une expression.
		\begin{eqnarray*}
			nc(p \vee q) &=& 1\\
			nc( (p \vee q)[x \vee y/p]) &=& nc( (x \vee y) \vee q) = 2\\
			nc( (p \vee q)[x \vee y]) &=& nc(r \vee q) = 1
		\end{eqnarray*}
		}

		\paragraph{Cas $F=\neg G$ --- Hypothèse $nc(G) \leq nc(G[A/p])$}
		\begin{eqnarray*}
			nc(G) &\leq& nc(G[A/p])\\
			 nc(\neg G) &\leq& nc(\neg G[A/p])\\
			 &\leq& nc(\neg(G[A/p]))\\
			&\leq& nc(\neg G[A/p])\\
			&\leq& nc(G[A/p])+1\\	
			nc(G) +1 &\leq& nc(G[A/p]) +1 \Rightarrow \textrm{ Vrai par hypothèse}
		\end{eqnarray*}

		\paragraph{Cas $F = G \wedge H$ --- Hypothèse : $nc(G) \leq nc(G[A/p])$ et $nc(H) \leq nc(H[A/p])$}
		\begin{eqnarray*}
			nc(G \wedge H) &\leq& nc( (G \wedge H)[A/p])\\
			nc(G) + nc(H) + 1 &\leq& nc(G[A/p] \wedge H[A/p])\\
			&\leq& nc(G[A/p]) + nc(H[A/p]) + 1 \Rightarrow \textrm{ Vrai par hypothèse}
		\end{eqnarray*}
		\paragraph{Cas $F=v$}
		\subparagraph{Sous cas $v=p$}
		\begin{eqnarray*}
			nc(p) &\leq& nc(p[A/p])\\
			0 &\leq& nc(A)
		\end{eqnarray*}

		\subparagraph{Souscas $v \neq p$}
		\begin{eqnarray*}
			nc(v) &\leq& nc(v[A/p])\\
			0 &\leq& nc(v) = 0
		\end{eqnarray*}

		\section{Logique du premier ordre}
		\subsection{Syntaxe}
		\subsubsection{Exercice 13}
		\begin{enumerate}
			\item $\forall x(\neg(Rx (x)))$ ou $\neg \exists x.(Rx (x))$
			\item $\forall x(Bl(x) \leftrightarrow \neg Br(x))$ ou $\neg \exists x.(Bl(x) \wedge Br(x))$ 
			\item $\forall x.F(x) \rightarrow (\exists y.H(y) \wedge (Gc(y) \wedge Dte(x,y))$ 
		\end{enumerate}
%%%

\end{document}







