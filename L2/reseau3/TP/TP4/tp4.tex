\documentclass[a4paper, 11pt]{article}

\usepackage{lmodern}
\usepackage{xcolor}
\usepackage[utf8]{inputenc}
\usepackage[T1]{fontenc}
\usepackage[francais]{babel}
\usepackage[top=1.7cm, bottom=1.7cm, left=2.5cm, right=2.5cm]{geometry}
\usepackage{verbatim}
\usepackage{tikz} %Vectoriel
\usepackage{listings}
\usepackage{fancyhdr}
\usepackage{multido}
\usepackage{amssymb}
\usepackage{multicol}
\usepackage{float}

\newcommand{\titre}{Réseau}
\newcommand{\numero}{3}
\newcommand{\typeDoc}{TDM}
\newcommand{\module}{Réseau}
\newcommand{\sigle}{reseau}
\newcommand{\semestre}{3}


\usepackage{ifthen}
\date{\today}

\chead{Antoine de \bsc{Roquemaurel}}
\rhead{TP\no\typeDoc}
\lhead{\titre}
%\makeindex

\lfoot{Université Toulouse III -- Paul Sabatier}
\rfoot{\sigle\semestre}
%\rfoot{}
\cfoot{--~~\thepage~~--}

\makeglossary
\makeatletter
\def\clap#1{\hbox to 0pt{\hss #1\hss}}%

\def\haut#1#2#3{%
	\hbox to \hsize{%
		\rlap{\vtop{\raggedright #1}
	}%
	\hss
	\clap{\vtop{\centering #2}
}%
\hss
\llap{\vtop{\raggedleft #3}}}}%
\def\bas#1#2#3{%
	\hbox to \hsize{%
		\rlap{\vbox{
			\raggedright #1
		}
	}%
	\hss \clap{\vbox{\centering #2}}%
	\hss
	\llap{\vbox{\raggedleft #3}}}
}%
\def\maketitle{%
	\thispagestyle{empty}{%
		\haut{}{\@blurb}{}
		%	
		%\vfill

		\begin{center}
			\vspace{-2.0cm}
			\usefont{OT1}{ptm}{m}{n}
			\huge \@type \@title
		\end{center}
		\par
		\hrule height 1pt
		\par
		\vspace{1cm}
		\bas{}{}{}
}%
}
\def\date#1{\def\@date{#1}}
\def\author#1{\def\@author{#1}}
\def\type#1{\def\@type{#1}}
\def\title#1{\def\@title{#1}}
\def\location#1{\def\@location{#1}}
\def\blurb#1{\def\@blurb{#1}}
\date{\today}
\newboolean{monBool}
\setboolean{monBool}{true}
\author{}
\title{}
\ifthenelse{\equal{\typeDoc}{}}{
\numeroTD{}
}
{
	\type{\typeDoc~--- }
}
\location{Amiens}\blurb{}
%\makeatother
\title{\titre}
\author{%Semestre \semestre
}

\location{Toulouse}
\blurb{%
\vspace{-35px}
\begin{flushleft}
	Université Toulouse III -- Paul Sabatier\\
	L2 Informatique\\
\end{flushleft}
\begin{flushright}
	\vspace{-45px}
	\Large \textbf \module \\
	\normalsize \textit \today\\
	Semestre \semestre
	\vspace{30px}
\end{flushright}
Antoine de \bsc{Roquemaurel}
}%



%\title{Cours \\ \titre}
%\date{\today\\ Semestre \semestre}

%\lhead{Cours: \titre}
%\chead{}
%\rhead{\thepage}

%\lfoot{Université Paul Sabatier Toulouse III}
%\cfoot{\thepage}
%\rfoot{\sigle\semestre}

\pagestyle{fancy}

\input{/home/aroquemaurel/cours/includesLaTeX/listings.tex} %prise en charge du langage C 




%----------------------------------------------------------------------------------------
%	DEFINITION OF COLORED BOXES
%----------------------------------------------------------------------------------------

\RequirePackage[framemethod=default]{mdframed} % Required for creating the theorem, definition, exercise and corollary boxes

% Theorem box
\newmdenv[skipabove=7pt,
skipbelow=7pt,
backgroundcolor=black!5,
linecolor=ocre,
innerleftmargin=5pt,
innerrightmargin=5pt,
innertopmargin=5pt,
leftmargin=0cm,
rightmargin=0cm,
innerbottommargin=5pt]{tBox}

% Exercise box	  
\newmdenv[skipabove=7pt,
skipbelow=7pt,
rightline=false,
leftline=true,
topline=false,
bottomline=false,
backgroundcolor=ocre!10,
linecolor=ocre,
innerleftmargin=5pt,
innerrightmargin=5pt,
innertopmargin=5pt,
innerbottommargin=5pt,
leftmargin=0cm,
rightmargin=0cm,
linewidth=4pt]{eBox}	

% Definition box
\newmdenv[skipabove=10pt,
skipbelow=10pt,
rightline=false,
leftline=true,
topline=false,
bottomline=false,
linecolor=ocre,
innerleftmargin=5pt,
innerrightmargin=5pt,
innertopmargin=0pt,
leftmargin=0cm,
rightmargin=0cm,
linewidth=4pt,
innerbottommargin=0pt]{dBox}	

% Corollary box
\newmdenv[skipabove=7pt,
skipbelow=7pt,
rightline=false,
leftline=true,
topline=false,
bottomline=false,
linecolor=gray,
backgroundcolor=black!5,
innerleftmargin=5pt,
innerrightmargin=5pt,
innertopmargin=5pt,
leftmargin=0cm,
rightmargin=0cm,
linewidth=4pt,
innerbottommargin=5pt]{cBox}		

% Corollary box
\newmdenv[skipabove=7pt,
skipbelow=7pt,
rightline=true,
leftline=false,
topline=false,
bottomline=true,
linecolor=gray,
backgroundcolor=black!5,
innerleftmargin=5pt,
innerrightmargin=5pt,
innertopmargin=5pt,
leftmargin=0cm,
rightmargin=0cm,
linewidth=1pt,
innerbottommargin=5pt]{rBox}				  
		  

% Creates an environment for each type of theorem and assigns it a theorem text style from the "Theorem Styles" section above and a colored box from above
\newenvironment{theorem}{\begin{tBox}\begin{theoremeT}}{\end{theoremeT}\end{tBox}}
\newenvironment{example}{\begin{exampleT}}{\hfill{\tiny\ensuremath{\blacksquare}}\end{exampleT}}
\newenvironment{definition}{\begin{dBox}\begin{definitionT}}{\end{definitionT}\end{dBox}}
\newenvironment{attention}{\begin{eBox}\small}{\end{eBox}}				  	
\newenvironment{exemple}{\begin{cBox}\small}{\end{cBox}}	

%----------------------------------------------------------------------------------------
%	REMARK ENVIRONMENT
%----------------------------------------------------------------------------------------

\newenvironment{remarque}{\par\vskip10pt\small
\begin{rBox}
\begin{list}{}{
\leftmargin=35pt % Indentation on the left
\rightmargin=25pt}\item\ignorespaces % Indentation on the right
\makebox[-2.5pt]{\begin{tikzpicture}[overlay]
\node[draw=ocre!60,line width=1pt,circle,fill=ocre!25,font=\sffamily\bfseries,inner sep=2pt,outer sep=0pt] at (-15pt,0pt){\textcolor{ocre}{R}};\end{tikzpicture}} % Orange R in a circle
\advance\baselineskip -1pt}
{\end{list}\vskip1mm\end{rBox}\vskip5pt} % Tighter line spacing and white space after remark



\input{/home/aroquemaurel/cours/includesLaTeX/polices.tex}
\input{/home/aroquemaurel/cours/includesLaTeX/affichageChapitre.tex}

\makeatother
\newcommand{\http}{\bsc{http}}
\newcommand{\url}{\bsc{url}}
\newcommand{\ssh}{\bsc{ssh}}
\newcommand{\https}{\bsc{https}}
\newcommand{\osi}{\bsc{osi}}
\newcommand{\dns}{\bsc{dns}}
\newcommand{\udp}{\bsc{udp}}
\newcommand{\ip}{\bsc{ip}}
\newcommand{\tcp}{\bsc{tcp}}
\newcommand{\ftp}{\bsc{ftp}}
\newcommand{\mac}{\bsc{mac}}
\newcommand{\arp}{\bsc{ftp}}
\begin{document}
	\maketitle
	\section{Commande et outils de tests réseau}
	\subsection{Commandes utiles}
	\subsubsection{\texttt{ifconfig} ou \texttt{ipconfig}}
	\texttt{ifconfig} et \texttt{ipconfig} permettent d'affiches les informations de chacunes des interfaces résea : L'adresse IP, et l'adresse \mac étant les
	principales informations utilisées.
	\subsubsection{\texttt{host} ou \texttt{nslookup}}
	\texttt{host} est une commande permettant d'afficher les redirections DNS
	\subsubsection{\texttt{whois}}
	\texttt{whois} est une commande permettant d'afficher des informations sur le propriétaire d'un nom de domaine.
	\subsection{Le \texttt{ping}}
	Le \texttt{ping} permet de tester la connexion réseau entre deux équipements, le premier équipement envoie un paquet jusqu'à un autre équipement, celui-ci
	doit répondre en renvoyant le paquet, si ce paquet nous revient, la connexion est effective, sinon il y a un problème, ou l'équipement refuse de renvoyer ce
	paquet même s'il l'a bien reçus.
	\subsection{}
\begin{lstlisting}[language=bash, basicstyle=\scriptsize\ttfamily]
C:\Users\21003495.MIG.000>ping 130.120.8.170
Envoi d'une requête Ping 130.120.8.170 avec 32 octets de données :
Réponse de 130.120.8.170 : octets=32 temps<1ms TTL=128
Réponse de 130.120.8.170 : octets=32 temps<1ms TTL=128
Réponse de 130.120.8.170 : octets=32 temps<1ms TTL=128
Réponse de 130.120.8.170 : octets=32 temps<1ms TTL=128

Statistiques Ping pour 130.120.8.170:
Paquets : envoyés = 4, recus = 4, perdus = 0 (perte 0%),
Durée approximative des boucles en millisecondes :
Minimum = 0ms, Maximum = 0ms, Moyenne = 0ms

C:\Users\21003495.MIG.000>ping www.univ-tlse3.fr
Envoi d'une requête ping sur servcms.ups-tlse.fr [195.220.59.77] avec 32 octets de données :
Réponse de 195.220.59.77 : octets=32 temps<1ms TTL=62
Réponse de 195.220.59.77 : octets=32 temps<1ms TTL=62
Réponse de 195.220.59.77 : octets=32 temps<1ms TTL=62
Réponse de 195.220.59.77 : octets=32 temps<1ms TTL=62

Statistiques Ping pour 195.220.59.77:
Paquets : envoyés = 4, recus = 4, perdus = 0 (perte 0%),
Durée approximative des boucles en millisecondes :
Minimum = 0ms, Maximum = 0ms, Moyenne = 0ms

C:\Users\21003495.MIG.000>ping google.fr
Envoi d'une requête ping sur google.fr [74.125.230.248] avec 32 octets de données :
Réponse de 74.125.230.248 : octets=32 temps=14 ms TTL=51
Réponse de 74.125.230.248 : octets=32 temps=14 ms TTL=51
Réponse de 74.125.230.248 : octets=32 temps=14 ms TTL=51
Réponse de 74.125.230.248 : octets=32 temps=14 ms TTL=51

Statistiques Ping pour 74.125.230.248:
Paquets : envoyés = 4, recus = 4, perdus = 0 (perte 0%),
Durée approximative des boucles en millisecondes :
Minimum = 14ms, Maximum = 14ms, Moyenne = 14ms
	\end{lstlisting}
	\subsubsection{}
	\subsection{La commande \texttt{traceroute}}
\begin{lstlisting}[language=bash, basicstyle=\scriptsize\ttfamily]
C:\Users\21003495.MIG.000>tracert 130.120.8.170
Détermination de l'itinéraire vers 206-08.mig.snap.fsi [130.120.8.170] avec un maximum de 30
sauts :
1 <1 ms <1 ms <1 ms 206-08.mig.snap.fsi [130.120.8.170] 
#Le chemin est direct car les deux pc doivent être connecté au même switch, 
#ainsi le paquet ne passe pas par un routeur intérmédiaire
Itinéraire déterminé.

C:\Users\21003495.MIG.000>tracert www.univ-tlse3.fr
Détermination de l'itinéraire vers servcms.ups-tlse.fr [195.220.59.77] avec un maximum de 30 sauts :
1 <1 ms <1 ms <1 ms 130.120.8.1 
2 <1 ms <1 ms <1 ms rtr-cict.cict.fr [130.120.124.15]
3 <1 ms <1 ms <1 ms servcms.ups-tlse.fr [195.220.59.77]
Itinéraire déterminé.

C:\Users\21003495.MIG.000>tracert google.fr
Détermination de l'itinéraire vers google.fr [74.125.230.247] avec un maximum de 30 sauts :
1 <1 ms <1 ms <1 ms 130.120.8.1
2 <1 ms <1 ms <1 ms 194.167.94.98
3 <1 ms <1 ms <1 ms 194.199.11.2
4 <1 ms <1 ms <1 ms te1-3-toulouse-rtr-021.noc.renater.fr [193.51.181.178]
5 14 ms 14 ms 14 ms te1-3-montpellier-rtr-021.noc.renater.fr [193.51.189.33]
6 17 ms 17 ms 17 ms te1-2-marseille1-rtr-021.noc.renater.fr [193.51. 189.21]
7 18 ms 19 ms 19 ms te0-0-0-0-lyon1-rtr-001.noc.renater.fr [193.51.189.17]
8 14 ms 14 ms 14 ms te1-1-lyon2-rtr-021.noc.renater.fr [193.51.179.117]
9 45 ms 48 ms 45 ms te0-0-0-1-paris2-rtr-001.noc.renater.fr [193.51. 189.2]
10 18 ms 18 ms 18 ms te1-1-paris2-rtr-021.noc.renater.fr [193.51.189. 9]
11 25 ms 18 ms 17 ms 193.51.182.197
12 15 ms 14 ms 14 ms 72.14.238.228
13 18 ms 18 ms 18 ms 209.85.242.51
14 15 ms 15 ms 15 ms par08s10-in-f23.1e100.net [74.125.230.247]
Itinéraire déterminé.

\end{lstlisting}
	\section{Protocole \ftp{} et \http{}}
	\subsection{Connexion \texttt{telnet}}
\begin{lstlisting}[language=bash, basicstyle=\scriptsize\ttfamily]
login: 21003495
Password:
Last login: Tue Nov 13 13:44:20 from 130.120.240.36
-bash-3.00$\$$ hostname
ouvea.edu.ups-tlse.fr
\end{lstlisting}
	\subsection{\http}
	\subsubsection{Récupérer une ressource}
	IL faut utiliser la commande \http{}  \texttt{GET}, celle-ci nous renverra le fichier ou la page web demandée, avec le code de retour 200, pour dire que tout
	s'est bien passé, dans les autres cas, un code d'erreur différent de 200 est renvoyé pour que nous sachions pour quelle raison le fichier demandée nous ai
	pas délivré.
	\subsubsection{Telnet \http}
\begin{lstlisting}[language=bash, basicstyle=\scriptsize\ttfamily]
telnet http://www.crie-mig-iup.ups-tlse.fr 80
GET /u3.html HTTP/1.0
Host: www.crie-mig-iup.ups-tlse.fr
Referer: http://www.crie-mig-iup.ups-tlse.fr/
User-Agent: CERN-LineMode/2.15 libwww/2.17b3
\end{lstlisting}	
\begin{lstlisting}[caption=Réponse du serveur, language=bash, basicstyle=\scriptsize\ttfamily]
HTTP/1.1 200 OK
Date: Tue, 27 Nov 2012 07:52:40 GMT
Server: Apache
Last-Modified: Tue, 09 Oct 2012 13:24:38 GMT
ETag: ``10615f2-14e4-4cba042841180''
Accept-Ranges: bytes
Content-Length: 5348
Connection: close
Content-Type: text/html
#suivi du code HTML de la page
\end{lstlisting}
	\subsection{\ftp}
\begin{lstlisting}[language=bash, basicstyle=\scriptsize\ttfamily]
-bash-3.00$\$$ telnet ftp.cict.fr 21
Trying 195.220.59.23...
Connected to aurore.cict.fr.
Escape character is '^2]'.
220 aurore.cict.fr FTP server (Version wu-2.6.2(1) Wed Oct 29 11:07:38 MET 2003) ready.
USER anonymous
331 Guest login ok, send your complete e-mail address as password.
PASS adresse@mail.com
230 Guest login ok, access restrictions apply.
SYST
215 UNIX Type: L8
PWD
257 ``/'' is current directory.
CWD /pub/doc
250 CWD command successful.
PASV
227 Entering Passive Mode (195,220,59,23,153,153)
\end{lstlisting}
	\subsubsection{Connexion de contrôle}
	Numéro port = $153\times 256 + 153 = 39321$
\begin{lstlisting}[caption=connexion de contrôle, language=bash, basicstyle=\scriptsize\ttfamily]
PASV
227 Entering Passive Mode (195,220,59,23,153,153)
LIST
150 Opening ASCII mode data connection for /bin/ls.
226 Transfer complete.
\end{lstlisting}
	\subsubsection{Connexion de données}
\begin{lstlisting}[caption=connexion de contrôle, language=bash, basicstyle=\scriptsize\ttfamily]
-bash-3.00$\$$ telnet 195.220.59.23 39321
Trying 195.220.59.23\ldots
Connected to 195.220.59.23.
Escape character is '^]'.
total 3048
drwxrwxr-x   4 ftpadm   cict        4096 Mar  6  2012 .
dr-xrwxr-x   4 ftp      cict        4096 Oct  5 14:17 ..
-rw-rw-r--   1 ftpadm   cict       63040 Jan 29  1993 RFCs.ps
-rw-rw-r--   1 ftpadm   cict       26695 Jan 29  1993 WAIS.txt
-rw-rw-r--   1 ftpadm   cict      492528 Jan 29  1993 Zen.intro.ps
-rw-rw-r--   1 ftpadm   cict      624664 Jun 21  1995 apprends_latex.ps
-rw-rw-r--   1 ftpadm   cict       16998 Jan 29  1993 biblis.internet.txt
-rw-rw-r--   1 ftpadm   cict       62766 Jan 29  1993 ftp.anonymous.ps
-rw-rw-r--   1 ftpadm   cict      217127 Jan 21  1994 micro-reseau.ps.210194
drwxrwxr-x   3 ftpadm   cict        4096 Apr 28  1995 rezo
drwxrwxr-x   2 ftpadm   cict        4096 Oct 10  1994 xlispstat
Connection to 195.220.59.23 closed by foreign host.
-bash-3.00$\$$
\end{lstlisting}
\end{document}


