\documentclass[a4paper, 11pt]{article}

\usepackage{lmodern}
\usepackage{xcolor}
\usepackage[utf8]{inputenc}
\usepackage[T1]{fontenc}
\usepackage[francais]{babel}
\usepackage[top=1.7cm, bottom=1.7cm, left=2.5cm, right=2.5cm]{geometry}
\usepackage{verbatim}
\usepackage{tikz} %Vectoriel
\usepackage{listings}
\usepackage{fancyhdr}
\usepackage{multido}
\usepackage{amssymb}
\usepackage{multicol}
\usepackage{float}

\newcommand{\titre}{Réseau}
\newcommand{\numero}{2}

\newcommand{\typeDoc}{TDM}

\newcommand{\module}{Réseau}
\newcommand{\sigle}{reseau}

\newcommand{\semestre}{3}


\usepackage{ifthen}
\date{\today}

\chead{Antoine de \bsc{Roquemaurel}}
\rhead{TP\no\typeDoc}
\lhead{\titre}
%\makeindex

\lfoot{Université Toulouse III -- Paul Sabatier}
\rfoot{\sigle\semestre}
%\rfoot{}
\cfoot{--~~\thepage~~--}

\makeglossary
\makeatletter
\def\clap#1{\hbox to 0pt{\hss #1\hss}}%

\def\haut#1#2#3{%
	\hbox to \hsize{%
		\rlap{\vtop{\raggedright #1}
	}%
	\hss
	\clap{\vtop{\centering #2}
}%
\hss
\llap{\vtop{\raggedleft #3}}}}%
\def\bas#1#2#3{%
	\hbox to \hsize{%
		\rlap{\vbox{
			\raggedright #1
		}
	}%
	\hss \clap{\vbox{\centering #2}}%
	\hss
	\llap{\vbox{\raggedleft #3}}}
}%
\def\maketitle{%
	\thispagestyle{empty}{%
		\haut{}{\@blurb}{}
		%	
		%\vfill

		\begin{center}
			\vspace{-2.0cm}
			\usefont{OT1}{ptm}{m}{n}
			\huge \@type \@title
		\end{center}
		\par
		\hrule height 1pt
		\par
		\vspace{1cm}
		\bas{}{}{}
}%
}
\def\date#1{\def\@date{#1}}
\def\author#1{\def\@author{#1}}
\def\type#1{\def\@type{#1}}
\def\title#1{\def\@title{#1}}
\def\location#1{\def\@location{#1}}
\def\blurb#1{\def\@blurb{#1}}
\date{\today}
\newboolean{monBool}
\setboolean{monBool}{true}
\author{}
\title{}
\ifthenelse{\equal{\typeDoc}{}}{
\numeroTD{}
}
{
	\type{\typeDoc~--- }
}
\location{Amiens}\blurb{}
%\makeatother
\title{\titre}
\author{%Semestre \semestre
}

\location{Toulouse}
\blurb{%
\vspace{-35px}
\begin{flushleft}
	Université Toulouse III -- Paul Sabatier\\
	L2 Informatique\\
\end{flushleft}
\begin{flushright}
	\vspace{-45px}
	\Large \textbf \module \\
	\normalsize \textit \today\\
	Semestre \semestre
	\vspace{30px}
\end{flushright}
Antoine de \bsc{Roquemaurel}
}%



%\title{Cours \\ \titre}
%\date{\today\\ Semestre \semestre}

%\lhead{Cours: \titre}
%\chead{}
%\rhead{\thepage}

%\lfoot{Université Paul Sabatier Toulouse III}
%\cfoot{\thepage}
%\rfoot{\sigle\semestre}

\pagestyle{fancy}

\input{/home/aroquemaurel/cours/includesLaTeX/listings.tex} %prise en charge du langage C 




%----------------------------------------------------------------------------------------
%	DEFINITION OF COLORED BOXES
%----------------------------------------------------------------------------------------

\RequirePackage[framemethod=default]{mdframed} % Required for creating the theorem, definition, exercise and corollary boxes

% Theorem box
\newmdenv[skipabove=7pt,
skipbelow=7pt,
backgroundcolor=black!5,
linecolor=ocre,
innerleftmargin=5pt,
innerrightmargin=5pt,
innertopmargin=5pt,
leftmargin=0cm,
rightmargin=0cm,
innerbottommargin=5pt]{tBox}

% Exercise box	  
\newmdenv[skipabove=7pt,
skipbelow=7pt,
rightline=false,
leftline=true,
topline=false,
bottomline=false,
backgroundcolor=ocre!10,
linecolor=ocre,
innerleftmargin=5pt,
innerrightmargin=5pt,
innertopmargin=5pt,
innerbottommargin=5pt,
leftmargin=0cm,
rightmargin=0cm,
linewidth=4pt]{eBox}	

% Definition box
\newmdenv[skipabove=10pt,
skipbelow=10pt,
rightline=false,
leftline=true,
topline=false,
bottomline=false,
linecolor=ocre,
innerleftmargin=5pt,
innerrightmargin=5pt,
innertopmargin=0pt,
leftmargin=0cm,
rightmargin=0cm,
linewidth=4pt,
innerbottommargin=0pt]{dBox}	

% Corollary box
\newmdenv[skipabove=7pt,
skipbelow=7pt,
rightline=false,
leftline=true,
topline=false,
bottomline=false,
linecolor=gray,
backgroundcolor=black!5,
innerleftmargin=5pt,
innerrightmargin=5pt,
innertopmargin=5pt,
leftmargin=0cm,
rightmargin=0cm,
linewidth=4pt,
innerbottommargin=5pt]{cBox}		

% Corollary box
\newmdenv[skipabove=7pt,
skipbelow=7pt,
rightline=true,
leftline=false,
topline=false,
bottomline=true,
linecolor=gray,
backgroundcolor=black!5,
innerleftmargin=5pt,
innerrightmargin=5pt,
innertopmargin=5pt,
leftmargin=0cm,
rightmargin=0cm,
linewidth=1pt,
innerbottommargin=5pt]{rBox}				  
		  

% Creates an environment for each type of theorem and assigns it a theorem text style from the "Theorem Styles" section above and a colored box from above
\newenvironment{theorem}{\begin{tBox}\begin{theoremeT}}{\end{theoremeT}\end{tBox}}
\newenvironment{example}{\begin{exampleT}}{\hfill{\tiny\ensuremath{\blacksquare}}\end{exampleT}}
\newenvironment{definition}{\begin{dBox}\begin{definitionT}}{\end{definitionT}\end{dBox}}
\newenvironment{attention}{\begin{eBox}\small}{\end{eBox}}				  	
\newenvironment{exemple}{\begin{cBox}\small}{\end{cBox}}	

%----------------------------------------------------------------------------------------
%	REMARK ENVIRONMENT
%----------------------------------------------------------------------------------------

\newenvironment{remarque}{\par\vskip10pt\small
\begin{rBox}
\begin{list}{}{
\leftmargin=35pt % Indentation on the left
\rightmargin=25pt}\item\ignorespaces % Indentation on the right
\makebox[-2.5pt]{\begin{tikzpicture}[overlay]
\node[draw=ocre!60,line width=1pt,circle,fill=ocre!25,font=\sffamily\bfseries,inner sep=2pt,outer sep=0pt] at (-15pt,0pt){\textcolor{ocre}{R}};\end{tikzpicture}} % Orange R in a circle
\advance\baselineskip -1pt}
{\end{list}\vskip1mm\end{rBox}\vskip5pt} % Tighter line spacing and white space after remark



\input{/home/aroquemaurel/cours/includesLaTeX/polices.tex}
\input{/home/aroquemaurel/cours/includesLaTeX/affichageChapitre.tex}

\makeatother
\newcommand{\http}{\bsc{http}}
\newcommand{\osi}{\bsc{osi}}
\newcommand{\dns}{\bsc{dns}}
\newcommand{\udp}{\bsc{udp}}
\newcommand{\ip}{\bsc{ip}}
\newcommand{\tcp}{\bsc{tcp}}
\newcommand{\ftp}{\bsc{ftp}}
\newcommand{\mac}{\bsc{mac}}
\newcommand{\arp}{\bsc{ftp}}
\begin{document}
	\maketitle
	\section{Concept d'adresse}
	\subsection{Adresses physiques}
	
	\begin{description}
		\item[Protocole] La couche liaison de données utilise le \textbf{protocole \udp}.
		\item[Adresse physique source] \texttt{00:1b:fc:23:f0:94}
		\item[Adresse physique de destination] \texttt{00:07:cb:3e:fd:73}
	\end{description}
	Un adresse \mac{} est codé sur 6 octets, ainsi, il existe $2^{48}$ adresses \mac{} différentes.

	\subsection{Adresses logiques}
	\begin{description}
		\item[Protocole] La couche liaison de données utilise le \textbf{protocole IP}.
		\item[Adresse réseau source] \texttt{192.168.0.10}
		\item[Adresse réseau destination] \texttt{212.27.40.240}
	\end{description}
	Une adresse IP v4 est codé sur 4 octets ainsi il existe $2^{32}$ adresse différentes, cependant ces IP sont arrivés à leur fin, ainsi IPv6 à été
	développé afin de palier à ce problème, ces adresse là sont codés en hexadécimal sur 16 octets, soit $2^128$ adresses différentes.

	\subsection{}
		L'adresse mac de destination est une adresse physique qui est connue dans le réseau local, c'est en l'occurence l'adresse du routeur qui se
		chargera de transmettre les paquets au routeur suivant jusqu'à arriver à l'adresse IP de destination.

		\subsection{Adresse de broadcast} Une adresse de \textit{broadcast} est une adresse de diffusion, lorsque l'on envoie un paquet à cette adresse, tous les clients du
	 réseau recoivent le paquet. En l'occurence, cette requête de \textit{broadcast} est une requête \arp{}, elle sert à identifier les postes sur le réseau.
	\section{Exemples de formats d'adresses liaison et réseau}
	\subsection{Adresse \mac}
	L'adresse \mac{} de la carte réseau est \texttt{00:0f:fe:d3:b5:f6}.
	\subsection{Adresse IP}
	L'adresse IP de l'ordinateur est \texttt{130.120.8.168} avec un masque de sous-réseau de \texttt{255.255.252.0}.

	\begin{itemize}
		\item Les deux premiers octets de cette adresse logique sont pour la partie réseau. 
		\item Le troisième octet possède 6 bits réseau et 3 bits equipements.
		\item Le dernier octet est pour l'equipement
	\end{itemize}

	\section{Gestion des adresses IP officielles}
	\subsection{Regional Internet Registry}
	Il existe aujourd'hui cinq RIR\footnote{Regional Internet Registry}. Par ordre de création, ce sont :
	\begin{itemize}
		\item RIPE-NCC\footnote{Réseaux IP Européens} pour l'Europe et le Moyen-Orient ;
		\item ARIN \footnote{American Registry for Internet Numbers} pour l'Amérique du Nord ;
		\item APNIC \footnote{Asia Pacific Network Information Center} pour l'Asie et le Pacifique ;
		\item LACNIC \footnote{Latin American and Caribbean IP address Regional Registry} pour l'Amérique latine et les îles des Caraïbes ;
		\item AfriNIC \footnote{African Network Information Center} pour l'Afrique.
	\end{itemize}

	\subsection{Registre d'allocations}
	\subsubsection{\texttt{192.48.251.195}}
		\begin{description}
			\item[Préfixe] Le préfixe 193 appartient à RIPE.
			\item[Propriétaire] Cette adresse IP appartient à Jacques \bsc{Landru}, Martine \bsc{Sion} et Tovoherizo \bsc{Rakotonavalona} pour la cité scientifique de Lille1.
		\end{description}

	
		\subsubsection{\texttt{64.248.129.75}}
		\begin{description}
			\item[Préfixe] Le préfixe appartient à ??? 
			\item[Propriétaire] Cette adresse IP n'est pas reconnue.
		\end{description}

		\subsubsection{\texttt{202.56.176.26}}
		\begin{description}
			\item[Préfixe] Le préfixe 202 appartient à APNIC.
			\item[Propriétaire] Cette adresse IP appartient à Nasir \bsc{Abdul Rahimy} et Stanislav \bsc{Kolodzinsky} pour CeReTechs Main network. 
		\end{description}
		
		\subsection{Adresse de l'université}
		Le préfixe 130 appartient à RIPE.

		L'adresse IP \texttt{130.120.84.5} appartient à l'université Paul Sabatier, l'université possède une plage d'IP de classe B, ainsi toutes les IP
		de \texttt{160.120.0.0} à \texttt{130.120.255.255} appartiennent à l'UPS\footnote{Université Paul Sabatier}.

\end{document}


