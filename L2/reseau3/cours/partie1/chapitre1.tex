\chapter{Histoire de la communication}
	Les humains ont toujours voulu \textbf{communiquer plus vite et plus loin}, ceci en utilisant des codes, alphabets, langages, \ldots
	
	\exemple{Les Gaulois, écrit Jules César dans ``La guerre des Gaules'', avec la voix de champ en champ pouvaient transmettre une nouvelle à 240km de distance en une journée.

		Les Grecs, en utilisant des flambeaux disposés de façon à indiquer les lettres de l'alphabet communiquaient, au temps d'Alexandre, de l'Inde à la Grèce en 5 jours.
		}	

		Le concept de la communication n'a pas changé de nos jours, nous avons toujours un système de codage afin que l'émetteur et le destinataire puisse communiquer. Cependant les supports de la communication ont changé afin de gagner en rapidité (ondes radio, fibre optique\ldots)\footnote{Les supports de communication présentent tous des caractéristiques techniques.}

		\begin{description}
			\item[1464] Poste Royale (Louis XI)\\
				L'inconvénient principal était le temps de transmission.
			\item[1794] Télégraphe optique (Chappe)\\
		Les inconvénients du télégraphe optique sont la visibilité et l'atténuation\ldots Cependant, nous procédons de la même façon, nous utilisons un système de relais: c'est un fondamental.
			\item[1832] Télégraphe \'Electrique (Shilling)
			\item[1837] Code Télégraphique (Morse) et création de l'administration du Télégraphe
			\item[1854] 1$^{er}$ projet de téléphone (Bourseul)
			\item[1860] Lois de l'électromagnétisme (Maxwell)
			\item[1876] Brevet du Téléphone (Bell)
			\item[1887] \'Etude sur les ondes Radioélectriques (Hertz)
			\item[1889] Nationalisation de la société Français de Téléphone
			\item[1892] \'Etude sur la Radiodiffusion (Crooker)
			\item[1896] Liaison de TSF (Marconi)
			\item[1897] \'Emission Radio au Panthéon
			\item[1901] Monopole d'état sur la radiodiffusion
			\item[1915] Téléphone automatique
			\item[1917] Télégraphe de Baudot
			\item[1943] Premier calculateur électronique. Début de l'ère du traitement électronique de l'information: \textbf{Informatique}, 
				suivit de la volonté d'obtenir un moyen de télécommunications entre les équipements Informatique: \textbf{Réseaux Informatiques}.\\ Ainsi, un support de communication, nécessitait un réseau différent (Son $\Rightarrow$ Radio, Image $\Rightarrow$ TV, Texte $\Rightarrow$ Télégraphe, \ldots). Une fusion va se produire.
		\end{description}

		Composants $\Rightarrow$ Signal $\Rightarrow$ Équipements $\Rightarrow$ Protocoles $\Rightarrow$ Architectures $\Rightarrow$ Services.

