\documentclass[10pt,a4paper,openany]{book}

\usepackage{lmodern}
\usepackage{xcolor}
\input{/home/aroquemaurel/cours/includesLaTeX/couleurs.tex}

\usepackage[utf8]{inputenc}
\usepackage[T1]{fontenc}
\usepackage[francais]{babel}
\usepackage[top=1.7cm, bottom=1.7cm, left=1.7cm, right=1.7cm]{geometry}
\usepackage{verbatim}
\usepackage[urlbordercolor={1 1 1}, linkbordercolor={1 1 1}, linkcolor=vert1, urlcolor=bleu, colorlinks=true]{hyperref}
\usepackage{tikz} %Vectoriel
\usepackage{listings}
\usepackage{fancyhdr}
\usepackage{multido}
\usepackage{amssymb}
\usepackage{float}
\usepackage{wrapfig}
\usepackage{pdfpages}

\newcommand{\titre}{Les réseaux informatiques}
\newcommand{\sigle}{reseau}
\newcommand{\semestre}{3}

\input{/home/aroquemaurel/cours/includesLaTeX/listings.tex} %prise en charge du langage C 
\input{/home/satenske/cours/includesLaTeX/entete-l2-cours.tex}




%----------------------------------------------------------------------------------------
%	DEFINITION OF COLORED BOXES
%----------------------------------------------------------------------------------------

\RequirePackage[framemethod=default]{mdframed} % Required for creating the theorem, definition, exercise and corollary boxes

% Theorem box
\newmdenv[skipabove=7pt,
skipbelow=7pt,
backgroundcolor=black!5,
linecolor=ocre,
innerleftmargin=5pt,
innerrightmargin=5pt,
innertopmargin=5pt,
leftmargin=0cm,
rightmargin=0cm,
innerbottommargin=5pt]{tBox}

% Exercise box	  
\newmdenv[skipabove=7pt,
skipbelow=7pt,
rightline=false,
leftline=true,
topline=false,
bottomline=false,
backgroundcolor=ocre!10,
linecolor=ocre,
innerleftmargin=5pt,
innerrightmargin=5pt,
innertopmargin=5pt,
innerbottommargin=5pt,
leftmargin=0cm,
rightmargin=0cm,
linewidth=4pt]{eBox}	

% Definition box
\newmdenv[skipabove=10pt,
skipbelow=10pt,
rightline=false,
leftline=true,
topline=false,
bottomline=false,
linecolor=ocre,
innerleftmargin=5pt,
innerrightmargin=5pt,
innertopmargin=0pt,
leftmargin=0cm,
rightmargin=0cm,
linewidth=4pt,
innerbottommargin=0pt]{dBox}	

% Corollary box
\newmdenv[skipabove=7pt,
skipbelow=7pt,
rightline=false,
leftline=true,
topline=false,
bottomline=false,
linecolor=gray,
backgroundcolor=black!5,
innerleftmargin=5pt,
innerrightmargin=5pt,
innertopmargin=5pt,
leftmargin=0cm,
rightmargin=0cm,
linewidth=4pt,
innerbottommargin=5pt]{cBox}		

% Corollary box
\newmdenv[skipabove=7pt,
skipbelow=7pt,
rightline=true,
leftline=false,
topline=false,
bottomline=true,
linecolor=gray,
backgroundcolor=black!5,
innerleftmargin=5pt,
innerrightmargin=5pt,
innertopmargin=5pt,
leftmargin=0cm,
rightmargin=0cm,
linewidth=1pt,
innerbottommargin=5pt]{rBox}				  
		  

% Creates an environment for each type of theorem and assigns it a theorem text style from the "Theorem Styles" section above and a colored box from above
\newenvironment{theorem}{\begin{tBox}\begin{theoremeT}}{\end{theoremeT}\end{tBox}}
\newenvironment{example}{\begin{exampleT}}{\hfill{\tiny\ensuremath{\blacksquare}}\end{exampleT}}
\newenvironment{definition}{\begin{dBox}\begin{definitionT}}{\end{definitionT}\end{dBox}}
\newenvironment{attention}{\begin{eBox}\small}{\end{eBox}}				  	
\newenvironment{exemple}{\begin{cBox}\small}{\end{cBox}}	

%----------------------------------------------------------------------------------------
%	REMARK ENVIRONMENT
%----------------------------------------------------------------------------------------

\newenvironment{remarque}{\par\vskip10pt\small
\begin{rBox}
\begin{list}{}{
\leftmargin=35pt % Indentation on the left
\rightmargin=25pt}\item\ignorespaces % Indentation on the right
\makebox[-2.5pt]{\begin{tikzpicture}[overlay]
\node[draw=ocre!60,line width=1pt,circle,fill=ocre!25,font=\sffamily\bfseries,inner sep=2pt,outer sep=0pt] at (-15pt,0pt){\textcolor{ocre}{R}};\end{tikzpicture}} % Orange R in a circle
\advance\baselineskip -1pt}
{\end{list}\vskip1mm\end{rBox}\vskip5pt} % Tighter line spacing and white space after remark



\input{/home/aroquemaurel/cours/includesLaTeX/polices.tex}
\input{/home/aroquemaurel/cours/includesLaTeX/affichageChapitre.tex}
\begin{document}
	\setcounter{tocdepth}{2}
	\setcounter{secnumdepth}{3}
	\maketitle
	\tableofcontents
	\part{Introduction aux réseaux informatiques}
		\chapter{Projet et Équipe de management}
	\section{Présentation des créateurs}
		\begin{itemize}
			\item \Bonte{}, Ingénieur en \gHabitat{}
			\item \Ben{}, Ingénieur en \gHabitat{}
			\item \Drm{}, Développeur Web
			\item \Soum{}, Développeur \texttt{\glo{C++}{C++}{4e langage de programmation
				le plus utilisé au monde. Il est compilé, permettant de produire un programme
				s'éxecutant le plus rapidement possible.}/\glo{Qt}{Qt}{Bibliothèque programmée
				en C++ permettant de créer des interfaces graphiques.}}
			\item \Clem{}, Développeur \texttt{C++/Qt}, Administrateur Système et Réseau
		\end{itemize}
		
		\subsection{Formations}
			\Bonte{} et \Ben{} ont obtenu un Master Pro \gHabitat{} à l'INSA de Toulouse et
			font actuellement une thèse.

			\Drm{}, \Soum{} et \Clem{} sont en deuxième année de DUT Informatique 
			à l'IUT Paul Sabatier de Toulouse. Ces derniers sont
			également autodidactes et ont pu acquérir de nombreuses conpétences lors de projets personnels.
	
		\subsection{Expériences professionnelles}
			\Bonte{} a passé un an dans un bureau d'étude à réaliser des bilans thermique. C'est pendant cette periode 
			que lui est venue l'idée de notre projet après avoir constaté le manque dramatique d'affordance des solutions disponibles.
			Le reste de l'équipe s'est constitué autour de ce constat global.
	
	\section{Atouts}
		%atouts qui font qu'on a des facilités a créer l'entreprise

		%facultés particulières
		Ayant déja eu une expérience professionnelle, \bonte{} et \ben{} connaissent les difficultés et les besoins des TPE et PME du \gHabitat{}.
		%contacts
		Ainsi, nous disposons déja de contacts dans ce secteur,
		nottamment au sein d'établissements universitaires et de bureaux d'étude.
		Ce premier carnet d'adresse peut être facilement étoffé car notre clientelle est,
		par sa taille humaine, facilement abordable et particulièrement à l'écoute pour trouver des solutions gratuites et efficaces.
		%connaissances pratiques théo
		Grâce aux cours généraux enseignés à l'université et en DUT, 
		tout les associés ont des connaissances en Comptabilité, Gestion et Droit des Entreprises,
		ce qui permet de faciliter les échanges avec les professionnels (comptables, avocats...) que nous ne manquerons pas de contacter.
		%part à des orga assoc
		\bonte, actuellement thésard, interviens dans des promotions de \gHabitat{} et a, 
		de ce fait, la possibilité de présenter des produits de \K{} aux étudiants. \\
		\clem{} est quant à lui impliqué dans diveres associations et a de ce fait rencontré plusieurs personnes ayant fondé ou travaillant dans une \glo{SCOP}{SCOP}{Société soumise à l’impératif de rentabilité comme toute entreprise.
Ses salariés-coopérateurs y sont en effet associés (ou « co-entrepreneurs ») majoritaires et détiennent au moins 51\% du capital et 65\% des droits de vote. Par ailleurs, quelle que soit la quantité du capital détenu, chaque coopérateur ne dispose que d'une seule voix lors de l'assemblée générale de l'entreprise.
}\footnotesouvenir{scop}{\textbf{S}ociété \textbf{CO}opérative et \textbf{P}articipative}. Cela nous permet d'avoir des réponses rapides et un premier contact avec le réseau des SCOPs\footnoterappel{scop}, qui permet aux jeunes entreprises de bénéficier d'avantages divers afin de se développer.
		%aide famille...


	\section{L'idée}
		% societe de dev de logiciel et de prestation de services informatiques dans le génie de l'habitat
		\K{} est une Société de Développement de Logiciels et de Prestation de Services Informatiques dans le secteur du \gHabitat{}.

		% Comment est venue l'idée
		% Secteur du génie de l'habitat
		L'idée de ce projet est née à travers diverses expériences dans le domaine du génie climatique.
		À l'heure actuelle, les professionnels n'ont à leurs disposition que peu d'outils : 
		\begin{itemize}
			\item Tableurs Excel réalisés en interne, aux résultats approximatifs dans un contexte de maîtrise de l'énergie
				et dans lesquels la saisie des données est peu aisée.
			\item Logiciels réglementaires coûtant plusieurs milliers d'euros, à l'ergonomie souvent douteuse et peu adaptés a de petites structures telles que les PME\footnote{\textbf{P}etites et \textbf{M}oyennes \textbf{E}ntreprises} et les TPE\footnote{\textbf{T}rès \textbf{P}etite \textbf{E}ntreprise}
		\end{itemize}
		
		% C'est une création
		Nous souhaitons donc créer une société a l'écoute des besoins de ces petites structures, afin de leurs permettre d'économiser leurs ressources lors de leurs projets grâce à des outils adaptés à leurs échelle.
		% Sur Toulouse
		Notre équipe s'étant formée à Toulouse, et le secteur du \gHabitat{} y étant largement développé\footnote{Voir Chapitre \ref{marché}. Marché}, c'est donc dans cette ville que nous implanterons notre société.

	\section{Objectifs du projet}
		%quel objectif ? expansion, retabilité ? autonomie ?
		%prépondérent
		%d'autres ?
		La société \K{} repose sur des valeurs et des principes communautaires
		où le seul objectif est de rendre accessible au plus grand nombre
		l'accès a des outils ergonomiques, intuitifs et performants
		afin de fournir les résultats les plus précis possible 
		dans une optique de maîtrise de l'énergie et de développement durable.

	\section{Taille de l'entreprise}
	%dimension (effectif, CA, capitaux 10890, parts de marché
	%taille max ? min ?
	L'entreprise \K{} souhaite rester une entreprise à taille Humaine, ainsi elle sera composé d'un maximum de 20 personnes afin que tout le monde soit impliqué dans l'entreprise.

	Le capital de départ sera de 10090\euro{} et pourra évoluer selon les activités de l'entreprise. 

	Notre but serai d'avoir un chiffre d'affaire de $104\;400$\euro{} au bout de deux ans d'activités et atteindre les 2~220~000\euro{} dans les 4 ans après la création de notre entreprise. 

	Dans un premier temps, \K{} favorisera l'évolution au sein de Midi-Pyrénnés, si celle-ci fonctionne
	convenablement, elle s'étendra au reste de la France dans un second temps.

		\chapter{\'Evolution des réseaux}
		De la même manière que la téléphonie et le télégraphe, nous sommes passé d'une phase expérimentale à une phase d'utilisation. Ainsi l'Informatique à beaucoup évolué. Cette évolution à été progressive, il y a eu plusieurs étapes qui ont marqués les réseaux de communication.
		\paragraph{Coûts des équipements Informatiques / Coûts de la Communication} À l'origine seul les grands comptes étaient capable d'avoir des équipements informatiques. Ainsi les SSI\footnote{Société de Service en Informatiques} sont nées.
		\paragraph{Système de Télétraitement} Ces systèmes ont été destiné aux entreprise, afin qu'a distance elles puissent utiliser la puissance d'un calculateur qui était géographiquement loin. Une première structure de réseau Informatique fut créée.
		\remarque{Nous sommes en train de revenir à cette solution créée 40 ans auparavant: Le cloud computing}

		\section{Les équipements créés}
		Afin de construire ces structures de réseaux de communication nous avons mis en place des équipements :
		\begin{description}
			\item[Processeur Frontal de Communication\footnote{FEP: Front End Processor}]
			\item[Multiplexeurs et concentrateurs] Équipement de partage du support de communication, permettent d'avoir des nœuds de communication.
			\item[Liaisons Spécialisées] Nous avions besoin d'un réseau spécialisé afin d'interconnecter les appareils, pour les connections point à point.
			\item[Modem] Pour les trafics de grande ligne, il fut choisir d'utiliser un réseau déjà existant, le téléphone.
		Cependant, le signal à transmettre doit être adapté au support de transmission, on va donc utiliser un adaptateur qui permettra de faire 
		passer le signal sur le réseau téléphonique : le modem.
			\item[Commutateurs] Pour avoir une connexion la plus rapide possible, nous avions besoin d'un algorithme de routage afin de passer par un chemin en fonction du trafic présent sur la ligne: le routeur.
			\item[Protocole de communication] Permet de faire dialoguer deux machines entre elles, elle doivent utiliser le même protocole afin de se comprendre syntaxiquement et sémantiquement.
		\end{description}
		
		\section{Démocratisation de l'Informatique}
		\begin{description}
			\item[1970] La genèse des protocoles de communication date des années 1970. En réseau, rien n'a été inventé de nouveau, cela à surtout été des progrès technologiques : rapidité, miniaturisation, coûts et donc démocratisation. Les premiers mini-calculateurs.
			\item[1980] Début de l'informatique personnelle et mise en \oe{}uvre des réseaux locaux.
			\item[1990] Applications de l'Internet, premiers mobiles et satellites. 
		\end{description}


		\chapter{Complexité d'algorithmes définis par récurrence}
	\section{Exemple introductif : Tri fusion}
\'Etant donné un tableau T, on note T[i:j] le sous tableau de T qui va de la case i à la case j. L'algorithme de tri fusion utilise une procédure
\texttt{fusion(T,i,j,k)}. On suppose que les deux sous tableaux T[i:j] et T[j+1:k] sont déjà triés. En temps $\Theta(n)$, où $n=k-i+1$, la procédure
fusion produit le sous tableau T[i:k] trié à partir de la fusion de ces deux tableaux.

\lstinputlisting[language=algo, caption=Algorithme du tri fusion]{triFusion.algo}
	\section{Méthode naïve d'analyse de complexité}
	Soit un temps maximal d'exécution de tri fusion sur un tableau de longueur $n$.

	D'après l'algorithme, on a $$U_n = U_{\frac{n}{2}} + U_{\frac{n}{2}} \Theta(n)$$
	et $u_1 = 0$

	Pour simplifier la récurrence on suppose que $n$ est pair, et donc $U_n = 2U_{\frac{n}{2}} + \Theta(n)$

	La méthode naïve consiste à deviner la solution, ici on devine $U_n \leq c n \log_2 n$. On suppose $U_{\frac{n}{2}} \leq C \frac{n}{2} \log_2
	\frac{n}{2}$ et on essaye d'en déduire $U_n \leq c n \log_2 n$
	\begin{eqnarray*}
		U_n = 2U_{\frac{n}{2}} + cn &\leq& 2c \frac{n}{2} \log_2 \frac{n}{2} + cn \\&&= cn(\log_2 n -1) + cn = cn\log_2 n
	\end{eqnarray*}

	Puisque $u_1=0 \leq c 1 \log_2 1$, on en déduit $\forall n, i_n \leq cn \log_2 n$

	\subsection{Résumé de la méthode naïve}
	Pour une équation récurrente $u_n = f_n(U_{n-1}, \cdots, u_1)$ où f est une fonction monotone croissante
	\begin{enumerate}
		\item On devine une fonction $g$
		\item On suppose que $\forall n < 1$ on a $U_n \leq g(m)$
		\item On montre $U_n = f_n(U_{n-1},\cdots,u_1 \leq f_n(g(n-1), \cdots, g(1)) \leq g(n)$
		\item On conclut par récurrence que $\forall n$ on a $U_n \leq g(n)$
	\end{enumerate}
	\subsection{Exemples d'application}
	On commence par une \textbf{mauvaise} utilisation. Soit l'équation $U_n = 2 U_{\frac{n}{2}}$. L'intuition $U_n \leq kn$ n'est pas correcte.

	En effet, en remplaçant on obtient : 
	\begin{eqnarray*}
		n_n &=& 2U_{\frac{n}{2}}+1\\
		&=& 2k \frac{n}{2} + 1\\
		&=& kn + 1
	\end{eqnarray*}
	
	La bonne intuition est $u_n \leq kn - b$. En remplaçant on obtient : 
	\begin{eqnarray*}
		u_n = 2U_{\frac{n}{2}} + 1\\
		&\leq& 2(k\frac{n}{2} - b) + 1= kn - 2b + 1\\
		&\leq& kn -b\textrm{ Si } b \geq 1
	\end{eqnarray*}

	\subsection{Réduction à des formes simples}
	Lors de l'analyse d'algorithmes récursifs, on rencontre souvent des équations récurrentes de la forme $$u_n = aU_{\frac{n}{2}}+b,$$ où a et b sont des
	constantes. Par exemple le tri fusion.

	Pour convertir ce type de récurrence en une forme affine $u'_n = a'u'_{n-1}+b'$, on pose
	$$v_k = U_{2^k}$$
	Autrement dit, on étudiera la suite $\{u_n\}_{n \geq 0}$ uniquement sur les puissances de 2.
	\\
	Par exemple, pour le tri fusion, en remplaçant $n$ par $2^k$, 
	\begin{eqnarray*}
		U_{2^k} &=& 2U_{\frac{2^k}{2}} + C2^k\\
		\textrm{donc }V_k &=& 2v_{k-1} + c2^k
	\end{eqnarray*}

	\section{Équation récurrentes linéaires}
	\paragraph{Définition} Une équation récurrente linéaire à coefficients constants d'ordre $k$ est une équation de la forme 
	\begin{displaymath}
		\left\{ \begin{array}{llll}
			u_1 &=& C_i (O \leq i \leq k-1) & \textsc{Conditions initiales (CI)}\\
			u_n &=&  \sum^k_i=1 a_i u_{n-i} + g(n) & \textsc{Equation générale}
		\end{array} \right.
	\end{displaymath}

	Une équation est \textbf{homogène} si $\forall n g(n) = 0$. La solution générale est une suite satisfaisant uniquement l'équation générale. Une
	solution particulière est une solution générale satisfaisant aussi des conditions initiales.

	\subsection{Équations récurrentes linéaires homogènes d'ordre 1}
	\paragraph{Proposition} La solution particulière de l'équation : 
	\begin{displaymath}
		\left\{ \begin{array}{lll}
			u_0 &=& c\\
			u_n &=&  a u_{n-1}
		\end{array} \right.
	\end{displaymath}
	est $u_n=C a^n$ (c'est une suite géométrique)

	\subsection{Équations récurrentes linéaires non-homogènes d'ordre 1}
	On ne sait traiter facilement que les équations dans lesquelles le second membre $g(n)$ est un polynôme ou une exponentielle. Pour cela, on
	<<dérive>> l'équation pour faire baisser le degré du polynôme jusqu'à arriver à 0.

	\exemple{\textbf{Le tri fusion}\\
	On a une équation qui n'est pas homogène: $$V_n = 2V_{n-1} + C 2^n$$
	Donc, au rang $n+1$, on a aussi $$V_{n+1} = 2 V_n + C \times 2 ^{n+1}$$

	Pour éliminer la partie non-homogène, on enlève 2 fois la première équation à la seconde.
	\begin{eqnarray*}
		V_{n+1} -2V_n &=& 2V_n - 4 V_{n-1}\\
		V_{n+1} &=& 4V_n - 4V_{n-1}
	\end{eqnarray*}
	}

	\subsection{Recherche d'une solution générale pour les équations récurrentes linéaires homogènes d'ordre 2}
	Une équation récurrente homogène d'ordre 2 est de la forme 
	\begin{displaymath}
		\left\{ \begin{array}{lll}
			u_0 &=&  C_0\\
			u_1 &=&  C_1\\
			u_n &=& a_1 u_{n-1} + a_2 u_{n-2}
		\end{array} \right.
	\end{displaymath}
	On peut obtenir ce type d'équation indirectement lorsque l'on a réduit une équation d'ordre 1 à une équation homogène d'ordre 2.
	\exemple{ L'équation récurrente linéaire homogène d'ordre 2 de Fibonacci
	\begin{displaymath}
		\left\{ \begin{array}{lll}
			U_0 &=&  1\\
			U_1 &=&  1\\
			U_n &=&  U_{n-1} + U_{n-2}
		\end{array} \right.
	\end{displaymath}
	On résout ces équations d'ordre 2 comme des équations d'ordre 1 : On cherche une solution générale de la forme $\lambda r^n$. Une telle solution
	vérifie, pour le cas de la suite de Fibonacci : $\forall n \geq 2$, $\lambda r^n = \lambda r^{n-1} + \lambda r^{n-2}$

	Soit, en divisant par $\lambda r^{n-2}$ $$r^2 = r + 1$$

	Autrement dit, $r$ est une racine du polynôme $P(x) = x^3 -x - 1$.
	}
	\paragraph{Définition}
	Le polynôme caractéristique d'une équation récurrente homogène d'ordre $k$
	$$V_{n+k} + a_1 V_{n+k+1} + \cdots + a_kV_n = 0$$ est le polynôme $P(x) = x^k + a_1x^{k-1}+\ldots+a_{k-1}+a_k$

	\paragraph{Théorème} Si $r$ est une racine du polynôme caractéristique d'une équation récurrente linéaire homogène, alors pour toute constante
	$\lambda$, toute suite de la forme $\{\lambda r^n\}_{n \geq 0}$ est une solution générale de cette équation.

	Dans le cas de la suite de Fibonacci, on calcule le discriminant $\Delta=5$ et on trouve les deux racines $r_1 = \frac{1-\sqrt{5}}{2}$ et 
	$r_2=\frac{1+\sqrt{5}}{2}$

	\paragraph{Cas des racines doubles} Si le discriminant $\Delta = 0$, alors le polynôme n'a qu'une seule racine (de multiplicité 2). En remarquant
	que $r$ racine double de P(x) implique que $r$ est aussi une racine de $P'(x)$ on peut démontrer que $\{n\lambda r^n\}_{n \geq 0}$ est aussi une
	solution de l'équation récurrente.

	\paragraph{Théorème} Les solutions générales d'une équation récurrente linéaire homogène d'ordre 2 dont le polynôme caractéristique de deux racines
	$r_1$ et $r_2$ sont :
	\begin{itemize}
		\item Si $r_1 \neq r_2$ : $\{ \lambda_1 r_1^n + \lambda_2 r_2^n\}_{n \geq 0}$ pour toute constantes $\lambda_1$, $\lambda_2$
		\item Si $r_1 = r_2$ :  $\{(\lambda_1 + \lambda_2 \times n)r_1^n\}_{n \geq 0}$ pour toute constantes $\lambda_1$, $\lambda_2$
	\end{itemize}

	\paragraph{Preuve dans le cas d'une racine double}
	Soit l'équation $u_{n+2} + a U_{n+1} + bu_n = 0$ et soit $r$ une racine double du polynôme caractéristique.\\  
	$P(x) = x^2 + ax + b$, donc $r$ est aussi une racine de $P'(x)= 2x+a$. La suite $\{nr^n\}_{n\geq 0}$ est une solution de l'équation car
	\begin{eqnarray*}
	(n+2)r^{n+2} + a(n+1)r^{n+1} + bnr^n &=& n(r^{n+2} + ar^{n+1} + br^n) + 2r^{n+2} + ar^{n+1}\\
	&=& r^n[n(r^2 + ar + b) + r(2r+a)]\\
	&=& 0
	\end{eqnarray*}
	\subsection{Recherche de solutions particulières pour les équations récurrentes linéaires homogènes d'ordre 2}
	Dans le cas de la suite de Fibonacci, on cherche une solution particulières satisfaisant les conditions initiales et qui est de la forme
	$\lambda_1r_1^n + \lambda_2r_2^n$ où $r_1 = \frac{1-\sqrt{5}}{2}$ et $r_2 = \frac{1+\sqrt{5}}{2}$

	Donc on cherche $\lambda_1$, $\lambda_2$ tels que
	\begin{eqnarray*}
		u_0 &=&  1 = \lambda_1 r^0_1 + \lambda_2r_2^0 = \lambda_1 + \lambda_2\\
		u_1&=&  1 = \lambda_1 r^1_1 + \lambda r^1_2 = \frac{\lambda_1 + \lambda_2}{2} + \frac{\lambda_2 - \lambda _ 1}{2} \times \sqrt{5}\\
	\end{eqnarray*}
	\begin{displaymath}
		\Rightarrow 
		\left\{ \begin{array}{lll}
			1 &=&  \lambda_1 + \lambda_2\\
			\frac{1}{2} &=& \frac{\lambda_2 - \lambda_1}{2}\sqrt{5}
		\end{array} \right.
		\Rightarrow 
		\left\{ \begin{array}{lll}
			1 &=&  \lambda_1 + \lambda_2\\
			\frac{1}{\sqrt{5}} &=& \lambda_2 - \lambda_1
		\end{array} \right.
		\Rightarrow 
		\left\{ \begin{array}{lll}
			\lambda_2 &=&  \frac{1+\frac{1}{\sqrt{5}}}{2}\\
			\lambda_1 &=&  \frac{1-\frac{1}{\sqrt{5}}}{2}
		\end{array} \right.
	\end{displaymath}
	Au final, on trouve la solution particulière : 
	$$U_n = \frac{\sqrt{5}-1}{2\sqrt{5}}(\frac{1-\sqrt{5}}{2}) + \frac{\sqrt{5}+1}{2\sqrt{5}}(\frac{1+\sqrt{5}}{2})^n$$
	
	\subsubsection{Résumé de la méthode pour les équations homogènes d'ordre 2}
	Pour résoudre l'équation $u_n = aU_{n-1} + bu_{n-2}$
	\begin{enumerate}
		\item On calcule le polynôme caractéristique $P(x) = x^2 - ax -b$
		\item On calcul les racines (éventuellement complexes)
			$r_1$ et $r_2$ de $P$
		\item On cherche les coefficients $\lambda_1$ $\lambda_2$ tels que $\lambda_1r_1^n + \lambda_2r_2^n$ satisfaisant les CI
	\end{enumerate}

	\subsection{Équations récurrentes d'ordre $k$}
	Pour les équations récurrentes homogènes d'ordre k, les considérations sur le polynôme caractéristique et ses racines restent valables. La difficulté
	est calculatoire car il faut trouver les racines d'un polynôme de degré k. Mais lorsque l'équation a été obtenu en éliminant  la partie
	non-homogène, les coefficients utilisés sont des solutions. 
	\exemple{Pour l'algorithme de Strasser, on a obtenu l'équation en faisant
	$$E_n-GE_{n-1}$$ où $E_n$ désigne l'équation de rang $n$

	$\Rightarrow$ 4 est une racine du polynôme caractéristique.
	}

	En cas de racine d'ordre $m$, on peut montrer par récurrence que $\{n^j\alpha^n\}_{n\geq 0}$ est une solution de l'équation récurrente homogène pour
	tout $j=0,\cdots,m-1$. Ceci nous permet d'avoir $k$ variables dans le système d'équation linéaires dérivées des CI.

	Le théorème suivant généralise le théorème 2 au cas de récurrences homogènes d'ordre $k > 2$ et prend en compte directement le second membre.
	\paragraph{Théorème 3}
	Supposons que le polynôme caractéristique de la récurrence homogène $u_n = au_{n-1} + \cdots + a_ku_{n-k}$ admet $p$ racines $ri(i=1,\cdots,p)$ de
	multiplicité $mi(i=1,\cdots,p)$. Alors la solution de la récurrence 
	$$u_n = a_1u_{n-1}+\cdots+a_ku_{n-k} + \sum^t_{i=1} b_i^n P_i(n)$$
	où $p_i$ est un polynôme de degré $d_i$) est donnée par $$\sum^t_{i=1}b_i^n Q_i(n)\footnote{Partie de la solution qui prend en compte le second membre} + 
	\sum_{i \in \{1,\ldots,p\}}r_i^n R_i(n)\footnote{Solution pour la récurrence homogène}$$
	tel que $r_i \not\in \{b_1,\cdots,b_t\}$

	Où 
	\begin{displaymath}
		\textrm{deg}(Q_i) = \left\{ \begin{array}{lll}
			d_i & si & b_i \not\in \{r_i,\cdots,r_p \}\\
			d_i+m_j & si& b_i = r_j
		\end{array} \right.
	\end{displaymath}

	Et $\textrm{deg}(R_i) = m_i-1$

	On obtient les polynômes $Q_i$ et $R_i$ à partir des CI et par identification des coefficients des termes $b_i^n n^j$ dans la récurrence.

	\remarque{Dans le théorème 2, il n'y avait de second membre (t=0) et les polynômes $R_i$ étaient de la forme $\lambda_1$ ou $\lambda_1 = \lambda_2 n$
	}
	\begin{displaymath}
		\left. \begin{array}{lll}
			u_n &=&  u_{n-1} + 1n^3\\
			u_{n-1} &=&  u_{n-2} + 1
		\end{array}
		\right\}
		u_n - u_{n-1} = u_{n-1} - u_{n-2}
	\end{displaymath}
	\exemple{
	$$T(n) = 2T(\frac{n}{2}) + n ; T(1) = 1$$
	Après changement de variable $n=2^k$, $u_k = T(n)$, nous avons $u_k = 2u_{k-1}+2^k$.

	Ici le second membre $$\sum^t_{i=1} b_i^k Pi(k) = 2^k$$

	Donc $T=1$, $P_i(k)=1$, $b_i=2$

	Le polynôme caractéristique $P(x)=x-2$. La seule racine est $r_i = 2$.
	Donc la solution particulière  est de la forme $2^k(q_0 + q_i^k)$ car $\textrm{deg}(Q_i) = \textrm{deg}{P_i}  +\textrm{multiplicité}  = 0 + 1$, et
	cette solution satisfait la CI et la récurrence $1 = T(1) = u_0$
	$$2^k(q_0+q_1k) = 2 \times 2^{k-1}(q_0+q_i(k-1))$$	
			D'où $q_0 = 1$, $q_1 = 1$ donc $u_n = 2^k(1+k)$ et $T(n) = u_k = n(1+\log_2n)$
	}

	\subsection{Théorème pour les récurrences par divisions}
	Le théorème suivant nous donne directement l'ordre de grandeur de la solution en fonction des coefficients de l'équation récurrente.

	\paragraph{Théorème 4} Soient $a \geq 1$, $b > 1$ deux constantes, $f(n)$ une fonction, et $\{t(n\}_{n\geq 0}$ une suite vérifiant l'équation 
	$T(n) = aT(\frac{n}{b}) + f(n)$

	On a pour $\epsilon > 0$
\begin{itemize}
	\item Si $f(n) = 0(n^{\log_b a - \epsilon}$ alors $T(n) = \Theta(n^{\log_b a})$
	\item Si $f(n) = \Theta(n^{\log_b a})$ alors $T(n) = \Theta(n^{\log_b a . \log n})$
	\item Si $f(n) = \Omega(n^{\log_b a + \epsilon}$ et $af(\frac{n}{b}) \leq cf(n)$ pour une constante $c >1$, alors $T(n) = \Theta(f(n))$
\end{itemize}

		
	\chapter{Spécification d'un programme}
	\remarque{Durant ce chapitre, nous parlerons de programme, cependant cela est valable également pour les sous-programme}
	Un programme est spécifié par un triplet : 
	\begin{itemize}
		\item Prédicat d'entrée P(E) ou précondition
		\item action (E, S)
		\item Prédicat de sortie P(S) ou postcondition
	\end{itemize}

		Les prédicats sont écrits en utilisant le formalisme de la logique des prédicats et de sopérations booléeenes.
		\section{Mots clés à utiliser dans les prédicats}
		Les mots clés pouvant être utilisés: 
		\begin{itemize}
			\item Les quantificateurs logiques : $\forall$(quelque soit), $\exists$(il existe), $\nu$(nombre de)
			\item Les connecteurs logiques : $\wedge$(et), $\vee$(ou), $\rightarrow$(implique), $\leftrightarrow$(equivalence), $\lnot$(not)
		\end{itemize}
		\section{\'Ecriture de la spécification}
		C'es une traduction de l'énoncé et de l'analyse faite dans l'étape 1 de la méthodologie : c'est un \textbf{triplet} logique.
		La démarche pour écrire la spécification est la suivante.
			\begin{itemize}
				\item Identifier les propriétés des données d'entrée et les exprimer sous forme logique
				\item Identifier les propriétés sur les données en sortie et les exprimer sous forme logique. 
			\end{itemize}
		\exemple{
			\'Ecrire un programme qui trie un tableau T de N éléments.\\
			\begin{itemize}
				\item $N > 1$
				\item \texttt{trier (T, N, t);}
				\item $(\forall I : 0 \leq I < N-1 \longrightarrow T[I] \leq T[I+1]) \wedge$\\$
					(\forall I : 0 \leq I < N \longrightarrow $\\$(\nu J : 0 \leq J < N \wedge t[I] = t[J]) = (\nu J : o \leq J < N \wedge t[I] = T[J]))$ 
			\end{itemize}
		}
	

				\chapter{Vérification formelle de programmes}\label{pfp}
	Vérifier que le programme est correct revient à démontrer l'implication suivante :\\ PE $\rightarrow$ pfp(\texttt{(action(D,r, PS);}

	Avec pfp étant la plus faible précondition\footnote{wp weakest precondition}.
	\begin{description}
		\item[pfp]	C'est un système de réécriture permetant de transformer une formule logique en une autre selon le programme qui doit s'éxécuter.
			C'est donc une réécriture syntaxique du prédicat de sortie en fonction des actions du programme.  \item[Tableau de situation] Système de réécriture des données en entrée (mémoireà vers les données en sortie (mémoire)) en fonction des actions du programme.  \end{description}
	\section{Système de réécriture Pfp}
		Ensemble de règles de réécriture permettant de transformer une formule logique en fonction des structures de base des langages de programmation.

		Les langages impératifs:
			\begin{itemize}
				\item affectation
				\item Séquence
				\item Sélection
				\item répétition
			\end{itemize}

		Règle de réécriture : \texttt{pfp(structure, formule) = formule}

	\section{Calcul de pfp d'une affectation}
	$$pfp("x=e", Q) = Q^e_n$$

	Avec la formule Q dans laquelle toutes les occurences de ``x'' sont remplacées par ``e'' (remplacement textuel)
	\exemple {
		Soit le programme suivant :
		\begin{itemize}
			\item $x > 0$
			\item $x = x-1$
			\item $x \geq 0$
		\end{itemize}
		On doit se poser la question \texttt{PE} $\rightarrow$ \texttt{pfp(programme, PS)} ?\\
		\begin{eqnarray*}
			(x > 0) &\rightarrow& \texttt{pfp}("x=x-1", x \geq 0)\\
			(x>0)&\rightarrow&(x-1 \geq 0)\\
			(x>0) &\rightarrow& (x > 0)
		\end{eqnarray*}
	}
	\section{Calcul du pfp d'une séquence}
	$$\texttt{pfp}("a1;a2;a3", Q) = \texttt{pfp}("a1;a2;", \texttt{pfp}("a3;", Q));$$
	\exemple{
	\begin{itemize}
		\item \texttt{/* f = i! */}
		\item \texttt{i = i + 1;\\f = f*i}
		\item \texttt{/* f = i! */}
	\end{itemize}
\begin{eqnarray*}
	 f = i! &\rightarrow& \texttt{pfp}("i = i + 1 ; f = f \times i;", f = i!)\\
	 f = i! &\rightarrow& \texttt{pfp}("i = i +1", \texttt{pfp}("f = f \times i", f = i !)\footnote{\'Evaluation de la "règle" la plus profonde}\\
	 f = i! &\rightarrow& \texttt{pfp}("i = i+1", f \times i = i!)\\
	 f * i! &\rightarrow& f\times (i+1) = (i+1)!\\
	 &&i! \times(i+1) = (i+1)!\ \textmd{Par définition de}\ i!
\end{eqnarray*}
}
	\section{Calcul du pfp de la séléction} \begin{center} \texttt{if(B)\{$a_1$\} [else\{$A_2$\}]}
	 \end{center}
\begin{eqnarray*}
	\texttt{pfp}("\texttt{if}(B)\{A_1\}\texttt{else}\{1_2\}", PS) &=&
	B \rightarrow \texttt{pfp}(A_1, PS) \wedge \neg B \rightarrow \texttt{pfp} (A_2, PS)\\
	\texttt{pfp}("\texttt{if}(B)\{A_1\}\texttt{else}\{A_2\}", PS) &=& B \rightarrow \texttt{pfp}(A_1, PS) \wedge \neg B \rightarrow PS 
\end{eqnarray*}

Un exemple est disponible annexe \ref{pfpSequence} page \pageref{pfpSequence}.

\section{Calcul du pfp d'une répétition.}\label{pfpBoucle}
Modélisation formelle de la répétition.
\begin{lstlisting}[language=C]
/* PE */
initialisation;
/* INVARIANT */
while(c) {
	/* $c \wedge \textsc{INVARIANT}$ */
	corps de boucle;
	/* $\textsc{INVARIANT}$ */
}
/* $\neg c \wedge \textsc{INVARIANT}$ */
\end{lstlisting}

Il y a cinq étapes pour la vérification de la boucle.
\begin{enumerate}
	\item $PE \rightarrow \pfp(\texttt{initialisation}, \texttt{INVARIANT})$
	\item $c \wedge \texttt{INVARIANT} \rightarrow \pfp (\texttt{corps}, \texttt{INVARIANT})$\label{etape2boucle}
	\item $\neg c \wedge \texttt{INVARIANT} \rightarrow PS$ \label{etape3boucle}
	\item $ \texttt{INVARIANT} \wedge c \rightarrow f > 0$\footnote{f: fonction définie positive : f est appelée ``variante''}
	\item $F = f \wedge \texttt{INVARIANT} \wedge c \rightarrow \pfp (\texttt{corps}, F > f)$\footnote{f est décroissante}
\end{enumerate}

\appendix

		\chapter{Architecture de communication}


	\section{Définition d'une architecture de communication}
Quand on parle d'Architecture, on se réfère à une structure d'éléments définissant un système complexe. Dans le langage courant, l'architecture est ``l'art de concevoir et de construire un bâtiment selon des règles techniques'' (Le Petit Larousse). Pour un informaticien, il est fait souvent référence à l'architecture du calculateur, ensemble structuré d'éléments électroniques et logiques.

L'Architecture de Communication définit l'ensemble des entités nécessaires à la Communication ainsi que les règles régissant les échanges entre elles. On parle aussi d'Architecture de Réseau.

Pour bien comprendre les notions sous-jacentes à l'architecture de communication, prenons un exemple d'une communication entre individus par l'intermédiaire du réseau postal:

Le responsable d'une entreprise française (FR) négocie un marché avec le responsable d'une entreprise brésilienne (BR). Pour cela, un échange de documents en langue anglaise (langue commune) entre les deux responsables est réalisé. Le processus d'échange peut-être décrit de la façon suivante (on supposera que pour chaque fonction bien identifiée, un service est requis):
\begin{enumerate}
\item FR rédige le document explicitant les conditions du marché; FR confie ce document au service de traduction pour effectuer la traduction et se charger de l'envoi;
\item Le traducteur effectue la traduction et confie le document au service secrétariat pour envoi;
\item Le secrétaire référence le document et demande au service courrier de s'occuper de l'envoi;
\item Le service courrier en fonction de la qualité de service requise pour cet envoi choisit le mode d'acheminement (courrier postal, fax, messager,\ldots)  le plus approprié, précise l'adresse complète du destinataire final et expédie le document;
\item L'acheminement se fera à travers différents réseaux des différents pays en utilisant l'adresse du site destination ainsi que les informations de trafic;
\item Sur chaque liaison traversée, des mécanismes de contrôle sont mises en oeuvre pour s'assurer de la non altération du document transporté;
\item Selon le service support utilisé, une interface spécifique et une représentation physique de l'information est mise en oeuvre;
\end{enumerate}

Des fonctions similaires seront mises en oeuvre du coté destinataire en remontant les différentes couches.

\section{Les services offerts par une architecture de communication}
Les services offerts par une architecture de communication couvrnt tous les aspects de la transmission physique jusqu'à la synchronisation des processus applicatifs :
\begin{description}
	\item[Transmission physique] Correspond au supports, au type d'encodage, aux liaisons, toute l'architecture physique.
	\item[Contrôle d'erreurs]  Vérifier que les paquets sont bien arrivés.
	\item[Contrôle de flux]  S'assurer que l'émetteur n'aille pas trop vite par rapport au récepteur
	\item[Routage]  en cas de nœud de communication, choisir le chemin le plus rapide ceci en fonction du trafic. 
	\item[Régulation de flux (congestion)] Réguler le flux pour éviter la congestion\footnote{Peut être assimiler aux bouchons}, il préfère la prévention afin d'éviter les bouchons. 
	\item[Séquencement]  Les fichiers sont découpés en plusieurs paquets, en effet un paquet à une taille maximum, le séquencement réassemble les paquets afin de reconstituer le fichier grâce à une numérotation des paquets..
	\item[Contrôle de bout en bout] Vérifier que le fichier à bien été reconstitué. 
	\item[Gestion du dialogue] 
	\item[Reprise sur incidents]Cela permet aussi de gérer un arrêt de la connexion réseau afin de reprendre le transfert à l'endroit où il s'était arrêté
	\item[Transformation de l'information] Codage de l'information (avec le code ASCII par exemple), compression (codecs), sécurité de l'information (cryptage) 
	\item[Synchronisation des processus] Sémantique de l'application, c'est-à-dire quelle opération au niveau applicatifs (Renommer un fichier, créer un répertoire, \ldots)
\end{description}

\section{Pourquoi normaliser l'architecture ?}
Les opérateurs de Télécommunications, réunis au sein du CCITT (ITU actuellement), ont défini des architectures de communications permettant l'échange d'informations. Ainsi, leurs réseaux étaient interopérables ce qui a permis la constitution de réseaux internationaux.

Le monde Informatique n'a pas réagi de la même façon. Les intérêts n'étaient pas les mêmes. Au début de l'ère informatique, les constructeurs ont défini des Architectures de Communication permettant l'échange de données entre leurs équipements informatiques. Ainsi, IBM a défini SNA (Systems Network Architecture), DEC a défini DNA (Digital Network Architecture)\ldots Ces Architectures avaient l'inconvénient majeur d'être trop souvent liées à des équipements spécifiques : ce sont des Architectures Constructeurs ou des Architectures Propriétaires. L'aberration de cette situation se répercuta sur les utilisateurs : par exemple une agence de voyages devait se munir d'autant de terminaux que de Systèmes Informatiques différents auxquels elle devait accéder. Des îlots de réseaux de constructeurs s'étaient formés.

Face à cette situation, en 1977, l'organisation ISO a constitué des comités pour le développement d'une architecture commune permettant la connexion des équipements et l'échange de données entre eux. Ainsi, au sein du Comité Technique (TC : Technical Committee) TC97, deux Sous Comités (SC : SubCommittee) SC6 et SC21 s'occupèrent de la normalisation dans le domaine des Télécommunications et de l'Interconnexion de Systèmes. Le premier modèle a été achevé en 1979. En 1984, ISO publia le document ISO 7498 relatif au modèle de référence pour l'Interconnexion de Systèmes Ouverts OSI (Open Systems Interconnection). Le modèle OSI est référencé au CCITT sous la norme X.200. 

	Des \textbf{architectures normalisées} ont été mis en place par les opérateurs de Télécommunications (X21, X25, ISDN,\ldots).\\
	Des \textbf{architectures propriétaires} ont également être mise en place par les constructeurs informatiques (SNA, DNA, DSA).\\
	\begin{description}
		\item[1977] \textbf{ISO} constitue un comité pour la normalisation dans le domaine des télécommunications et de l'interconnexion des systèmes.\\
		\item[1984] ISO 7948 référence CCITT X.200 (ITU)
		\item[OSI] Cadre fonctionnel -- Le modèle de référence. Les objectifs du modèle OSI sont les suivants : 
			\begin{itemize}
				\item Décomposer (décomposition fonctionnelle)
				\item Structurer
				\item Assurer l'indépendance vis à vis du matériel et du logiciel.
			\end{itemize}
	\end{description}
\section{Le modèle OSI}
\subsection{Utilité et objectifs}
On appelle Système Ouvert Réel un système réel dont la communication avec un autre système réel se fait conformément au modèle OSI.

Le modèle OSI définit un cadre fonctionnel pour l'élaboration de normes d'interconnexion de systèmes. En aucun cas, OSI ne décrit comment ces systèmes fonctionnent en interne ou comment les normes doivent être implantées. OSI est un modèle et non une pile de protocoles.

Les objectifs du modèle OSI sont :
\begin{itemize}
	\item Décomposer et structurer le système de communication en éléments directement réalisables (Décomposition fonctionnelle) ;
	\item Assurer le maximum d'indépendance vis à vis du matériel et du logiciel ;
\end{itemize}

SI regroupe les entités en 7 couches. Chaque couche correspond à un niveau logique de fonctions. On distingue :
\begin{itemize}
	\item Les couches basses (1-4) relatives au transfert de l'information ;
	\item Les couches hautes (5-7) relatives au traitement réparti de l'information ;
\end{itemize}

\newpage
\subsection{Les couches}
\begin{verbatim}
      +-------------------+ Exemple  
7 N+4 |   Application     | HTTP   
      +-------------------+     
6 N+3 |   Présentation    | HTML, MPEG     
      +-------------------+              SERVICES APPLICATIFS (Couches hautes)
5 N+2 |   Session         |   
      +-------------------+ ===========================================================
4 N+1 |   Transport       | TCP 
      +-------------------+    
3 N   |   Réseau          | IP    
      +-------------------+                 SERVICES TRANSPORTS(Couches basses)
2 N-1 |   Liaison         | Wifi    
      +-------------------+    
1 N-2 |   Physique        | RJ45  
      +-------------------+  
\end{verbatim}
\subsubsection{Couche 1}
1. Couche Physique : Elle s'occupe de la transmission des bits de façon brute sur un canal de communication. Cette couche doit garantir la parfaite transmission des données (un bit 1 envoyé doit bien être reçu comme bit valant 1). Concrètement, cette couche doit normaliser les caractéristiques électriques (un bit 1 doit être représenté par une tension de 5 V, par exemple), les caractéristiques mécaniques (forme des connecteurs, de la topologie\ldots), les caractéristiques fonctionnelles des circuits de données et les procédures d'établissement, de maintien et de libération du circuit de données. ;
\subsubsection{Couche 2}
2. Couche Liaison de Données : Son rôle est un rôle de ``liant'' : elle va transformer la couche physique en une liaison a priori exempte d'erreurs de transmission pour la couche réseau. Elle fractionne les données d'entrée de l'émetteur en trames, transmet ces trames en séquence et gère les trames d'acquittement renvoyées par le récepteur. La couche liaison de données doit être capable de renvoyer une trame lorsqu'il y a eu un problème sur la ligne de transmission. De manière générale, un rôle important de cette couche est la détection et la correction d'erreurs intervenues sur la couche physique. Cette couche intègre également une fonction de contrôle de flux pour éviter l'engorgement du récepteur.
\subsubsection{Couche 3}
3. Couche Réseau : C'est la couche qui permet de gérer le sous-réseau, i.e. le routage des paquets sur ce sous-réseau et l'interconnexion des différents sous-réseaux entre eux. Au moment de sa conception, il faut bien déterminer le mécanisme de routage et de calcul des tables de routage (tables statiques ou dynamiques\ldots). La couche réseau contrôle également l'engorgement du sous-réseau. On peut également y intégrer des fonctions de comptabilité pour la facturation au volume, mais cela peut être délicat.
L'unité d'information de la couche réseau est le paquet.
\subsubsection{Couche 4}
4. Couche Transport : Si la couche réseau rend le service de transfert d'informations de terminal réseau à terminal réseau, la couche transport contrôle le transfert de bout en bout (d'utilisateur final à utilisateur final). Le rôle principal de la couche transport est de prendre les messages de la couche session, de les découper s'il le faut en unités plus petites et de les passer à la couche réseau, tout en s'assurant que les morceaux arrivent correctement de l'autre côté. Cette couche effectue donc aussi le réassemblage du message à la réception des morceaux.Cette couche est également responsable du type de service à fournir à la couche session, et finalement aux utilisateurs du réseau : service en mode connecté ou non, avec ou sans garantie d'ordre de délivrance, diffusion du message à plusieurs destinataires à la fois\ldots Cette couche est donc également responsable de l'établissement et du relâchement des connexions sur le réseau. Un des tous derniers rôles à évoquer est le contrôle de flux.
\subsubsection{Couche 5}
5. Couche Session : La session de transfert d'informations peut subir divers incidents. Un service de reprise sur incidents peut être nécessaire. D'autre part, des outils nécessaires à la gestion du dialogue peuvent être utilisés. Cette couche organise et synchronise les échanges entre tâches distantes.
\subsubsection{Couche 6}
6. Couche Présentation : Il ne suffit pas de transférer les données. Il faut aussi les interpréter en vue d'une bonne coopération. La syntaxe des données échangées entre entités applicatives est définie à ce niveau. Typiquement, cette couche peut convertir les données, les reformater, les crypter et les compresser.
\subsubsection{Couche 7}
7. Couche Application : Elle comprend les programmes d'applications ainsi que des fonctions applicatives génériques permettant le développement d'applications distribuées.

\subsection{Intéraction entre entités}
Nous utilisons la notation '(N)' pour signifier `` de rang N''.

La technique de structuration de base du Modèle OSI est la structure en couches : chaque système est logiquement composé d'un ensemble ordonné de sous-systèmes représentés verticalement. Voilà quelques définitions.
\subsubsection{Définitions}
\begin{description}
	\item[Sous-système (N)]\'Elément de rang N d'une division hiérarchique d'un système n'ayant d'interactions qu'avec les éléments des niveaux immédiatement supérieur et inférieur de cette division.
	\item[Couche (N)] Subdivision de l'architecture OSI, constituée de sous-systèmes de rang N. On dit qu'une couche fournit un service ou qu'elle est prestataire de services.
	\item[Entité (N)] Élément actif d'un sous-système (N).
	\item[Service] Capacité que possède la couche (N) -et les couches inférieures à celle-ci, fournie aux entités (N+1), à la frontière entre la couche (N) et la couche (N+1). Ces services sont invoqués par des primitives, spécifiques du service.
	\item[Facilité (N)]Élément d'un service (N).
	\item[Point d'accès à des services\footnote{Service Acces point ou SAP}] Point où les services (N) sont fournis par une entité (N) à une entité (N+1).
	\item[Protocole (N)] Ensemble de règles et de formats (sémantiques et syntaxiques) déterminant les caractéristiques de communication des entités (N) lorsqu'elle effectuent les fonctions (N).
\end{description}

Chaque couche (N) fournit des services (N) aux entités (N+1) de la couche (N+1). Les services d'une couche (N) sont fournis à la couche (N+1) grâce aux fonctions effectuées à l'intérieur de la couche (N) et, au besoin, avec l'aide des services offerts par la couche (N-1).
La coopération entre entités (N) est régie par un ou plusieurs protocoles (N).

\begin{figure}[H]
	\centering
	\includegraphics[width=15cm]{partie1/interaction.jpg}
\end{figure}
\begin{figure}[H]
	\centering
	\includegraphics[width=12cm]{partie1/interaction1.jpg}
\end{figure}
\begin{figure}[H]
	\centering
	\includegraphics[width=12cm]{partie1/interaction2.jpg}
\end{figure}
\subsection{Transfert des données utilisateurs}
Les données des utilisateurs traversent toutes les couches du modèle OSI jusqu'au niveau physique qui génère le signal transmis sur le média. Chaque couche rajoute des informations de contrôle du protocole (PCI : Protocole Control Information) aux données qui lui sont passées par l'entité supérieure qui demande le service (SDU : Service Data Unit). C'est l'encapsulation des données. Cet ajout détériore les performances de débit mais sont nécessaires pour assurer les services des diféfrentes couches: adressage, contrôle d'erreurs, contrôle de flux\ldots

Cette structure est aussi appelée aussi ``structure en pelure d'oignon''.

Les données sont échangées entre les entités de même niveau qui savent les interpréter. C'est le protocole de communication qui définit le format des unités de données échangées (PDU: Protocol Data Unit) et les règles de communication.

\begin{figure}[H]
	\centering
	\includegraphics[width=15cm]{partie1/encapsulation.jpg}
\end{figure}

\paragraph{Segmentation/Réassemblage} Lorsque le service fourni par la couche $(N)$ fixe une limite de taille sur les données trop petites par rapport au service de la couche $(N+1)$, la couche $(N+1)$ découpe les $(N+1)-SDU$ en plusieurs fragments correspondant chacun à un $(N+1)-PDU$ avant envoi. À la réception, la couche $(N+1)$ concatène les fragments pour retrouver le $(N+1)-SDU$ initial.

\subsection{Intéractions entre entités}
Une couche donnée fournit un ensemble de services au niveau supérieur, les services sont invoqués par des primitives. Dans le Modèle OSI, les primitives sont désignées par un nom précédé de la première lettre du nom anglais de la couche. Par exemple, \texttt{$T\_$CONNECT} est une primitive de la couche Transport, \texttt{N$\_$DATA} une primitive de la couche Réseau.

Il y a 4 types de primitives de service :
\begin{itemize}
	\item Requête: une entité sollicite un service pour faire   une activité;
	\item Indication: Informe d’un évènement;
	\item Réponse: réponse à l’évènement;
	\item Confirmation: informe de la demande de service;
\end{itemize}


%		\chapter{Ecriture d'un programme à partir de sa preuves}
Il s'agit d'écrire un programme à partir de ses spécifications..
\begin{lstlisting}[language=C, numbers=none]
/* P */
action(N,r);
/* Q */
\end{lstlisting}
Il permet de définir un modèle de solution : 
\begin{itemize}
	\item Séquence, boucle, recherche linéaire, recherche dichotomique
	\item Déduit de l'analyse de Q et de P
\end{itemize}

\section{Modèle de boucle}
	Il faut déterminer l'invariant et la condition de boucle afin de construire complètement le programme. Nous allons ainsi utiliser la propriété suivante : 

	$$ INV \wedge \neg C \rightarrow Q $$

	On pose $INV \wedge \neg C = Q$, si cette propriété est vérifiée, alors l'implication sera vérifiée.

	\subsection{1$^{ère}$ approche: Preuve avec invariant trivial}
	$Q = \top \wedge Q$ Q étant $\neg$condition 
	\begin{enumerate}
		\item $p \rightarrow \pfp (init , \top) = p \rightarrow \top$ Toujours vrai
		\item $\top \wedge Q \rightarrow \pfp(corps, \top) = \top \wedge Q \rightarrow \top$
		\item $\top \wedge Q \rightarrow Q$ Toujours vrai\footnote{$Q\rightarrow Q$} 
	\end{enumerate}
	Toute la difficulté est de prouver la terminaison.
	\subsubsection{Exemple}
	\begin{lstlisting}[language=C]
/* (x = A)$\wedge$(y=B)$\wedge$(z=C) */
init
/* $\top$ */
while ((x >= y) || (y >= z)) {
	/* $\top \wedge (x \geq y) \wedge (y \geq z)$ */
	if(x >= y) 
		echange(&x, &y);

	if(y >= z) 
		echange(&y, &z);
	/* \top */
}
/* $\top \wedge$ (x < y) $\wedge$ (y < z) */
/* (x < y) $\wedge$ (y < z)
	\end{lstlisting}
	Nous pouvons prendre $x-z + |A+B+C|$ comme variante

	\attention{Prendre un invariant trivial complique la preuve de terminaison et cela réduit l'écriture du programme à la recherche de la variante.}
	
	\subsection{2$^{nd}$ approche : Élimination de conjoint} 
	$Q : A \wedge B \wedge C \wedge \cdots$\footnote{A, B et C sont des conjoints}
	
	Si on peut trouver une séquence d'initialisation qui permet de vérifier les conjoints de façon simple; ces conjoints forment l'invariant.

	$Q = A \wedge B \wedge E$ on peut écrire $INV \wedge E$.

	\subsubsection{Exemple}
\begin{lstlisting}[language=C]
/* N > 0 */
a = N;
/* $a^2 \leq N$ */
while ( a*a > N) {
	/* $a^2 \leq N \wedge cond$ */
	a = a - 1;
	/* $a^2 \leq N$ */
}

/* 
 * $a^2 \leq N \leq (a+1)^2$ peut aussi être écrit $a^2 \leq N \wedge N \leq (a+1)^2$ 
 * $a^2 \leq N$ : INV
 * $N \leq (a+1)^2$ : $\neg C$
 */
\end{lstlisting}
\remarque{Cette solution fonctionne, cependant le programme à une complexité linéaire (N), celui-ci peut être résolu avec une complexité logarithmique.}


\subsection{3$^{ème}$ approche : introduction d'une variable dans Q} 
	En général, $Q$ s'écrit $Q(N)$. On va le réécrit en introduisant une variable : $Q(i) \wedge (i = N)$

	Ainsi $Q(i)$ devient l'invariant et $i = N$ la condition de boucle ($\neg$condition).

	\subsubsection{Exemple}

\begin{lstlisting}[language=C]
/* N > 0 */
f = 1;
i = 1;
/* f = i! */
while (i != N) {
	++i;	
	f *= i;
}

/* 
 * f = N ! 
 * On le réécrit f = i! $\wedge$ i = N
 */
\end{lstlisting}

%		\chapter{Transformation d'une spécification récursive en un programme itératif}
	Une approche récursive est équivalente en temps là une approche itérative, cependant la consommation mémoire sera beaucoup plus importante que
l'itérative, ceci étant dût à la pile système.

\remarque{Certains compilateurs peuvent transformer une approche récursive terminale en itératif\\
Notamment le compilateur Ocaml, gcc le fait également pour les types primitifs (scalaires), pour les structures de données, il n'en est pas capable.}
\section{Récursivité terminale}
	\begin{displaymath}
		G(x)
		\left\{ \begin{array}{ll}
			\textrm{si }&h(x)\textrm{ alors } a\\
			\textrm{sinon }&f(x) \oplus G(t(x)) 
		\end{array} \right.
	\end{displaymath}
	\remarque{le $\oplus$ est l'opérateur de combinaison intermédiaire}

	SI nous avons : 
	\begin{itemize}
		\item $h(x)$ fonction booléenne
		\item $\oplus$ associatif avec un élément neutre $e$ (à gauche)
	\end{itemize}
	Alors le programme suivant est correct.

\begin{lstlisting}[language=C]
/* PE: $\top$ */
x = X;
r = e;
/* INV: G(X) = r $\oplus$ G(x) */
while (!(h(x))) {
	r = r $\oplus$  f(x);
	x = t(x);
}
r = r $\oplus$ a;
/* PS: G(X) = r */
\end{lstlisting}

\exemple{
Écrire un programme qui calcule $Y=x^N$ avec $N$ entier et $N \geq 0$. X et Y des réels. 

\begin{enumerate}
	\item Spécification récursive
	\item Programme
\end{enumerate}
\begin{displaymath}
	\texttt{puissance}(X,N) = \left\{\begin{array}{ll}
		\textrm{si }  N = 0& 1\\
		\textrm{sinon} & x \times \texttt{puissance}(X,N-1)\\
	\end{array}\right.
\end{displaymath}
\begin{itemize}
	\item $\oplus$ : *réels
	\item $e$ : $1.0$
	\item $h(X, N)$: $N == 0$
	\item $f(X, N)$: $X$
	\item $t(X,N)$: $(X,N-1)$
\end{itemize}
\lstinputlisting[language=C, numbers=none]{annexes/exo7.c}
}

	\part{Transmission des données}
		\chapter{Ecriture d'un programme à partir de sa preuves}
Il s'agit d'écrire un programme à partir de ses spécifications..
\begin{lstlisting}[language=C, numbers=none]
/* P */
action(N,r);
/* Q */
\end{lstlisting}
Il permet de définir un modèle de solution : 
\begin{itemize}
	\item Séquence, boucle, recherche linéaire, recherche dichotomique
	\item Déduit de l'analyse de Q et de P
\end{itemize}

\section{Modèle de boucle}
	Il faut déterminer l'invariant et la condition de boucle afin de construire complètement le programme. Nous allons ainsi utiliser la propriété suivante : 

	$$ INV \wedge \neg C \rightarrow Q $$

	On pose $INV \wedge \neg C = Q$, si cette propriété est vérifiée, alors l'implication sera vérifiée.

	\subsection{1$^{ère}$ approche: Preuve avec invariant trivial}
	$Q = \top \wedge Q$ Q étant $\neg$condition 
	\begin{enumerate}
		\item $p \rightarrow \pfp (init , \top) = p \rightarrow \top$ Toujours vrai
		\item $\top \wedge Q \rightarrow \pfp(corps, \top) = \top \wedge Q \rightarrow \top$
		\item $\top \wedge Q \rightarrow Q$ Toujours vrai\footnote{$Q\rightarrow Q$} 
	\end{enumerate}
	Toute la difficulté est de prouver la terminaison.
	\subsubsection{Exemple}
	\begin{lstlisting}[language=C]
/* (x = A)$\wedge$(y=B)$\wedge$(z=C) */
init
/* $\top$ */
while ((x >= y) || (y >= z)) {
	/* $\top \wedge (x \geq y) \wedge (y \geq z)$ */
	if(x >= y) 
		echange(&x, &y);

	if(y >= z) 
		echange(&y, &z);
	/* \top */
}
/* $\top \wedge$ (x < y) $\wedge$ (y < z) */
/* (x < y) $\wedge$ (y < z)
	\end{lstlisting}
	Nous pouvons prendre $x-z + |A+B+C|$ comme variante

	\attention{Prendre un invariant trivial complique la preuve de terminaison et cela réduit l'écriture du programme à la recherche de la variante.}
	
	\subsection{2$^{nd}$ approche : Élimination de conjoint} 
	$Q : A \wedge B \wedge C \wedge \cdots$\footnote{A, B et C sont des conjoints}
	
	Si on peut trouver une séquence d'initialisation qui permet de vérifier les conjoints de façon simple; ces conjoints forment l'invariant.

	$Q = A \wedge B \wedge E$ on peut écrire $INV \wedge E$.

	\subsubsection{Exemple}
\begin{lstlisting}[language=C]
/* N > 0 */
a = N;
/* $a^2 \leq N$ */
while ( a*a > N) {
	/* $a^2 \leq N \wedge cond$ */
	a = a - 1;
	/* $a^2 \leq N$ */
}

/* 
 * $a^2 \leq N \leq (a+1)^2$ peut aussi être écrit $a^2 \leq N \wedge N \leq (a+1)^2$ 
 * $a^2 \leq N$ : INV
 * $N \leq (a+1)^2$ : $\neg C$
 */
\end{lstlisting}
\remarque{Cette solution fonctionne, cependant le programme à une complexité linéaire (N), celui-ci peut être résolu avec une complexité logarithmique.}


\subsection{3$^{ème}$ approche : introduction d'une variable dans Q} 
	En général, $Q$ s'écrit $Q(N)$. On va le réécrit en introduisant une variable : $Q(i) \wedge (i = N)$

	Ainsi $Q(i)$ devient l'invariant et $i = N$ la condition de boucle ($\neg$condition).

	\subsubsection{Exemple}

\begin{lstlisting}[language=C]
/* N > 0 */
f = 1;
i = 1;
/* f = i! */
while (i != N) {
	++i;	
	f *= i;
}

/* 
 * f = N ! 
 * On le réécrit f = i! $\wedge$ i = N
 */
\end{lstlisting}

		\chapter{Transformation d'une spécification récursive en un programme itératif}
	Une approche récursive est équivalente en temps là une approche itérative, cependant la consommation mémoire sera beaucoup plus importante que
l'itérative, ceci étant dût à la pile système.

\remarque{Certains compilateurs peuvent transformer une approche récursive terminale en itératif\\
Notamment le compilateur Ocaml, gcc le fait également pour les types primitifs (scalaires), pour les structures de données, il n'en est pas capable.}
\section{Récursivité terminale}
	\begin{displaymath}
		G(x)
		\left\{ \begin{array}{ll}
			\textrm{si }&h(x)\textrm{ alors } a\\
			\textrm{sinon }&f(x) \oplus G(t(x)) 
		\end{array} \right.
	\end{displaymath}
	\remarque{le $\oplus$ est l'opérateur de combinaison intermédiaire}

	SI nous avons : 
	\begin{itemize}
		\item $h(x)$ fonction booléenne
		\item $\oplus$ associatif avec un élément neutre $e$ (à gauche)
	\end{itemize}
	Alors le programme suivant est correct.

\begin{lstlisting}[language=C]
/* PE: $\top$ */
x = X;
r = e;
/* INV: G(X) = r $\oplus$ G(x) */
while (!(h(x))) {
	r = r $\oplus$  f(x);
	x = t(x);
}
r = r $\oplus$ a;
/* PS: G(X) = r */
\end{lstlisting}

\exemple{
Écrire un programme qui calcule $Y=x^N$ avec $N$ entier et $N \geq 0$. X et Y des réels. 

\begin{enumerate}
	\item Spécification récursive
	\item Programme
\end{enumerate}
\begin{displaymath}
	\texttt{puissance}(X,N) = \left\{\begin{array}{ll}
		\textrm{si }  N = 0& 1\\
		\textrm{sinon} & x \times \texttt{puissance}(X,N-1)\\
	\end{array}\right.
\end{displaymath}
\begin{itemize}
	\item $\oplus$ : *réels
	\item $e$ : $1.0$
	\item $h(X, N)$: $N == 0$
	\item $f(X, N)$: $X$
	\item $t(X,N)$: $(X,N-1)$
\end{itemize}
\lstinputlisting[language=C, numbers=none]{annexes/exo7.c}
}

		\chapter{Analyse spectrale}
\section{Quelles sont les différentes formes physiques utilisées pour le transport de l'information ?}
L'information est véhiculée grâce à un signal physique. Ce signal peut être soit de nature
analogique soit de nature digital (numérique).

Le signal analogique est un signal continu dans le temps. C'est à dire que son amplitude varie
d'une façon continue. Les signaux de notre environnement (voix, image\ldots) sont de nature
analogiques.

\begin{figure}[H]
	\centering
	\includegraphics[width=8cm]{partie2/signalanalog.jpg}
\end{figure}

Le signal numérique est un signal discret dans le temps. C'est à dire que son amplitude varie
d'une façon discrète (nombre fini de valeurs). Les signaux générés par certains équipements
(générateurs de tensions électriques ou pulsions) sont de nature numériques.

\begin{figure}[H]
	\centering
	\includegraphics[width=8cm]{partie2/signaldigit.jpg}
\end{figure}
Il ne faut pas confondre l'information avec le signal généré. En effet, l'information utilisée
élémentaire utilisée en informatique est discrète puisque l'ensemble des valeurs est fini
(bit=0 ou 1) tandis que le signal physique le représentant peut être soit numérique soit
analogique. De la même façon qu'on peut véhiculer l'information analogique (par ex.
information vocale) par un signal de nature analogique ou de nature numérique.

\section{Quelles sont les caractéristiques d'un signal analogique élémentaire ?}
Le signal analogique élémentaire est le signal sinusoïdal. C'est un signal périodique
caractérisé par trois paramètres:

\begin{itemize}
	\item La fréquence
	\item L'amplitude
	\item La phase
\end{itemize}
Le signal peut être représenté dans le domaine temporel (amplitude en fonction du temps):
$y(t)=A \sin(2 pft+f)$

Le signal peut être représenté dans le domaine fréquentiel (spectre d'énergie en fonction de
la fréquence). Dans le cas d'un signal sinusoïdal parfait, l'énergie est concentrée sur la
fréquence du signal. Mais le signal durant sa propagation subit des distorsions en fréquence
(comme on le verra plus tard) ce qui conduit à un étalement du spectre autour de la fréquence
théorique.

\section{Comment peut-on caractériser un signal quelconque ?}
Un signal périodique quelconque peut être décomposé en une somme de signaux sinusoïdaux
(Analyse de Fourier).
\begin{figure}[H]
	\centering
	\includegraphics[width=5cm]{partie2/signalaudio.jpg}
\end{figure}

Nous avons alors un spectre de raies. Comme il y a des fluctuations autour de chaque fréquence
contenue dans le signal, nous obtenons un spectre continu. Ainsi, un signal quelconque est
caractérisé par sa bande de fréquences. Plus particulièrement, au niveau quantitatif, nous
nous intéresserons à la largeur de la bande de fréquences.
\begin{figure}[H]
	\centering
	\includegraphics[width=8cm]{partie2/signalfourrier.jpg}
\end{figure}

\section{Quelles sont les déformations affectant un signal ?}
Un signal durant sa transmission subit différentes déformations dues aux caractéristiques du
support de transmission et de l'environnement.

\paragraph{L'affaiblissement} L'affaiblissement d'un signal est du à des caractéristiques du support
(résistance, dispersion de l'onde hertzienne\ldots).

Pour compenser cet affaiblissemnt, on utilise des amplificateurs. L'atténuation (perte) et
l'amplification (gain)s'expriment en décibels (dB): N=10*log10(PS/PE) - le décibel est la
représentation algorithmique d'un rapport de puissances mais on peut l'exprimer avec des
tensions: N=20*log10(TS/TE) avec P=V2/R
PS (resp. TS) et PE (resp. TE) représentent la puissance en sortie (resp. tension en sortie)et
la puissance en entrée (resp. tension en entrée). Pour une perte N est négatif et pour un gain
N est positif.
Le choix de l'échelle logarithmique est du à deux raisons essentielles: la variation de
l'énergie du signal est logarithmique et le fait que des gains et des pertes en cascades se
calculeront par des additions et des soustractions.
On exprime aussi la puissance en dBW: P(dBW)=10*log10P(W) et la tension en dBmV:
T(dBmV)=20*log10T(mV).

\paragraph{Les distorsions} Les distorsions subies par le signal affectent ses paramètres (amplitude,
phase, fréquence). 
La distorsion en amplitude est du au fait que le support a une impédance et pas uniquement une
résistance: l'affaiblissement du signal varie en fonction de la fréquence. Dans ce cas, il
faut utiliser un égalisateur qui amplifie en fonction de la fréquence.
La distorsion en phase est du au fait que la vitesse de propagation varie en fonction de la
fréquence du signal. Dans ce cas, le problème est un problème de synchronisation et nous avons
besoin de solutions de rephasage ou de resynchronisation.
La distorsion en fréquence est du au fait que le support de transmission ainsi que les
équipements d'extrémité filtrent certaines fréquences. Ils sont caractérisés par la bande de
fréquences qu'ils laissent passer ou Bande Passante. Si la Bande Passante n'est pas suffisante
pour la Bande de Fréquences du signal alors il est nécessaire d'effectuer une transmission par
transposition de fréquences.

\paragraph{Les bruits} Un bruit est un signal parasite additif au signal utile. Ce parasitage peut
amener à des erreurs d'interprétation du signal.

Pour limiter au maximum les bruits, différentes techniques sont utilisées agissant sur le type
du support utilisé (coaxial, blindé, torsadé, optique\ldots). L'influence du bruit est mesurée
par le rapport Signal sur Bruit en dB: (S/B)dB = 10*log10(S/B).


\section{Quelle est la capacité d'un support de transmission ?}
La capacité d'un support de transmission est la quantité maximum d'informations qu'il peut
véhiculer par unité de temps. En informatique, cette capacité est exprimée en bits par seconde
(ou ses multiples). Par conséquence, la capacité représente le débit maximum possible ur le
support. Il ne faut pas confondre capacité d'un support et débit d'un équipement. D'après les
travaux de Shannon nous avons: C = W*log2(S/B+1) où C est la capacité en bps, W la largeur de
la bande passante en Hertz et S/B le rapport puissance du signal sur puissance du bruit.




		\include{partie2/chapitre10}
		\include{partie2/chapitre11}
	\appendix
	\includepdf[page=1-]{TD1.pdf}
	\includepdf[page=1-]{TD2.pdf}
\end{document}



