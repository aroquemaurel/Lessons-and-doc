\documentclass[12pt,a4paper,openany]{book}

\usepackage{lmodern}
\usepackage{xcolor}
\input{/home/aroquemaurel/cours/includesLaTeX/couleurs.tex}

\usepackage[utf8]{inputenc}
\usepackage[T1]{fontenc}
\usepackage[francais]{babel}
\usepackage[top=1.7cm, bottom=1.7cm, left=1.7cm, right=1.7cm]{geometry}
\usepackage{verbatim}
\usepackage[urlbordercolor={1 1 1}, linkbordercolor={1 1 1}, linkcolor=vert1, urlcolor=bleu, colorlinks=true]{hyperref}
\usepackage{tikz} %Vectoriel
\usepackage{listings}
\usepackage{fancyhdr}
\usepackage{multido}
\usepackage{amssymb}

\newcommand{\titre}{Les réseaux informatiques}
\newcommand{\sigle}{reseau}
\newcommand{\semestre}{3}

\input{/home/aroquemaurel/cours/includesLaTeX/listings.tex} %prise en charge du langage C 
\input{/home/satenske/cours/includesLaTeX/entete-l2-cours.tex}




%----------------------------------------------------------------------------------------
%	DEFINITION OF COLORED BOXES
%----------------------------------------------------------------------------------------

\RequirePackage[framemethod=default]{mdframed} % Required for creating the theorem, definition, exercise and corollary boxes

% Theorem box
\newmdenv[skipabove=7pt,
skipbelow=7pt,
backgroundcolor=black!5,
linecolor=ocre,
innerleftmargin=5pt,
innerrightmargin=5pt,
innertopmargin=5pt,
leftmargin=0cm,
rightmargin=0cm,
innerbottommargin=5pt]{tBox}

% Exercise box	  
\newmdenv[skipabove=7pt,
skipbelow=7pt,
rightline=false,
leftline=true,
topline=false,
bottomline=false,
backgroundcolor=ocre!10,
linecolor=ocre,
innerleftmargin=5pt,
innerrightmargin=5pt,
innertopmargin=5pt,
innerbottommargin=5pt,
leftmargin=0cm,
rightmargin=0cm,
linewidth=4pt]{eBox}	

% Definition box
\newmdenv[skipabove=10pt,
skipbelow=10pt,
rightline=false,
leftline=true,
topline=false,
bottomline=false,
linecolor=ocre,
innerleftmargin=5pt,
innerrightmargin=5pt,
innertopmargin=0pt,
leftmargin=0cm,
rightmargin=0cm,
linewidth=4pt,
innerbottommargin=0pt]{dBox}	

% Corollary box
\newmdenv[skipabove=7pt,
skipbelow=7pt,
rightline=false,
leftline=true,
topline=false,
bottomline=false,
linecolor=gray,
backgroundcolor=black!5,
innerleftmargin=5pt,
innerrightmargin=5pt,
innertopmargin=5pt,
leftmargin=0cm,
rightmargin=0cm,
linewidth=4pt,
innerbottommargin=5pt]{cBox}		

% Corollary box
\newmdenv[skipabove=7pt,
skipbelow=7pt,
rightline=true,
leftline=false,
topline=false,
bottomline=true,
linecolor=gray,
backgroundcolor=black!5,
innerleftmargin=5pt,
innerrightmargin=5pt,
innertopmargin=5pt,
leftmargin=0cm,
rightmargin=0cm,
linewidth=1pt,
innerbottommargin=5pt]{rBox}				  
		  

% Creates an environment for each type of theorem and assigns it a theorem text style from the "Theorem Styles" section above and a colored box from above
\newenvironment{theorem}{\begin{tBox}\begin{theoremeT}}{\end{theoremeT}\end{tBox}}
\newenvironment{example}{\begin{exampleT}}{\hfill{\tiny\ensuremath{\blacksquare}}\end{exampleT}}
\newenvironment{definition}{\begin{dBox}\begin{definitionT}}{\end{definitionT}\end{dBox}}
\newenvironment{attention}{\begin{eBox}\small}{\end{eBox}}				  	
\newenvironment{exemple}{\begin{cBox}\small}{\end{cBox}}	

%----------------------------------------------------------------------------------------
%	REMARK ENVIRONMENT
%----------------------------------------------------------------------------------------

\newenvironment{remarque}{\par\vskip10pt\small
\begin{rBox}
\begin{list}{}{
\leftmargin=35pt % Indentation on the left
\rightmargin=25pt}\item\ignorespaces % Indentation on the right
\makebox[-2.5pt]{\begin{tikzpicture}[overlay]
\node[draw=ocre!60,line width=1pt,circle,fill=ocre!25,font=\sffamily\bfseries,inner sep=2pt,outer sep=0pt] at (-15pt,0pt){\textcolor{ocre}{R}};\end{tikzpicture}} % Orange R in a circle
\advance\baselineskip -1pt}
{\end{list}\vskip1mm\end{rBox}\vskip5pt} % Tighter line spacing and white space after remark



\input{/home/aroquemaurel/cours/includesLaTeX/polices.tex}
\input{/home/aroquemaurel/cours/includesLaTeX/affichageChapitre.tex}
\begin{document}
	\setcounter{tocdepth}{2}
	\setcounter{secnumdepth}{3}
	\maketitle
	\chapter{Histoire de la communication}
		Les humains ont toujours voulu \textbf{communiquer plus vite et plus loin}, ceci en utilisant des codes, alphabets, langages, \ldots
		
		\exemple{Les Gaulois, écrit Jules César dans ``La guerre des Gaules'', avec la voix de champ en champ pouvaient transmettre une nouvelle à 240km de distance en une journée.

			Les Grecs, en utilisant des flambeaux disposés de façon à indiquer les lettres de l'alphabet communiquaient, au temps d'Alexandre, de l'Inde à la Grèce en 5 jours.
			}	

			Le concept de la communication n'a pas changé de nos jours, nous avons toujours un système de codage afin que l'émetteur et le destinataire puisse communiquer. Cependant les supports de la communication ont changé afin de gagner en rapidité (ondes radio, fibre optique\ldots)\footnote{Les supports de communication présentent tous des caractéristiques techniques.}

			\begin{description}
				\item[1464] Poste Royale (Louis XI)\\
					L'inconvénient principal était le temps de transmission.
				\item[1794] Télégraphe optique (Chappe)\\
			Les inconvénients du télégraphe optique sont la visibilité et l'atténuation\ldots Cependant, nous procédons de la même façon, nous utilisons un système de relais: c'est un fondamental.
				\item[1832] Télégraphe \'Electrique (Shilling)
				\item[1837] Code Télégraphique (Morse) et création de l'administration du Télégraphe
				\item[1854] 1$^{er}$ projet de téléphone (Bourseul)
				\item[1860] Lois de l'électromagnétisme (Maxwell)
				\item[1876] Brevet du Téléphone (Bell)
				\item[1887] \'Etude sur les ondes Radioélectriques (Hertz)
				\item[1889] Nationalisation de la société Français de Téléphone
				\item[1892] \'Etude sur la Radiodiffusion (Crooker)
				\item[1896] Liaison de TSF (Marconi)
				\item[1897] \'Emission Radio au Panthéon
				\item[1901] Monopole d'état sur la radiodiffusion
				\item[1915] Téléphone automatique
				\item[1917] Télégraphe de Baudot
				\item[1943] Premier calculateur électronique. Début de l'ère du traitement électronique de l'information: \textbf{Informatique}, 
					suivit de la volonté d'obtenir un moyen de télécommunications entre les équipements Informatique: \textbf{Réseaux Informatiques}.\\ Ainsi, un support de communication, nécessitait un réseau différent (Son $\Rightarrow$ Radio, Image $\Rightarrow$ TV, Texte $\Rightarrow$ Télégraphe, \ldots). Une fusion va se produire.
			\end{description}

			Composants $\Rightarrow$ Signal $\Rightarrow$ Équipements $\Rightarrow$ Protocoles $\Rightarrow$ Architectures $\Rightarrow$ Services.

	\chapter{\'Evolution des réseaux}
			De la même manière que la téléphonie et le télégraphe, nous sommes passé d'une phase expérimentale à une phase d'utilisation. Ainsi l'Informatique à beaucoup évolué. Cette évolution à été progressive, il y a eu plusieurs étapes qui ont marqués les réseaux de communication.
			\paragraph{Coûts des équipements Informatiques / Coûts de la Communication} À l'origine seul les grands comptes étaient capable d'avoir des équipements informatiques. Ainsi les SSI\footnote{Société de Service en Informatiques} sont nées.
			\paragraph{Système de Télétraitement} Ces systèmes ont été destiné aux entreprise, afin qu'a distance elles puissent utiliser la puissance d'un calculateur qui était géographiquement loin. Une première structure de réseau Informatique fut créée.
			\remarque{Nous sommes en train de revenir à cette solution créée 40 ans auparavant: Le cloud computing}

			\section{Les équipements créés}
			Afin de construire ces structures de réseaux de communication nous avons mis en place des équipements :
			\begin{description}
				\item[Processeur Frontal de Communication\footnote{FEP: Front End Processor}]
				\item[Multiplexeurs et concentrateurs] Équipement de partage du support de communication, permettent d'avoir des nœuds de communication.
				\item[Liaisons Spécialisées] Nous avions besoin d'un réseau spécialisé afin d'interconnecter les appareils, pour les connections point à point.
				\item[Modem] Pour les trafics de grande ligne, il fut choisir d'utiliser un réseau déjà existant, le téléphone.
			Cependant, le signal à transmettre doit être adapté au support de transmission, on va donc utiliser un adaptateur qui permettra de faire 
			passer le signal sur le réseau téléphonique : le modem.
				\item[Commutateurs] Pour avoir une connexion la plus rapide possible, nous avions besoin d'un algorithme de routage afin de passer par un chemin en fonction du trafic présent sur la ligne: le routeur.
				\item[Protocole de communication] Permet de faire dialoguer deux machines entre elles, elle doivent utiliser le même protocole afin de se comprendre syntaxiquement et sémantiquement.
			\end{description}
			
			\section{Démocratisation de l'Informatique}
			\begin{description}
				\item[1970] La genèse des protocoles de communication date des années 1970. En réseau, rien n'a été inventé de nouveau, cela à surtout été des progrès technologiques : rapidité, miniaturisation, coûts et donc démocratisation. Les premiers mini-calculateurs.
				\item[1980] Début de l'informatique personnelle et mise en \oe{}uvre des réseaux locaux.
				\item[1990] Applications de l'Internet, premiers mobiles et satellites. 
			\end{description}

	\chapter{Classification}
		Le critère de classification est la distance entre les entités communicantes.
		\begin{enumerate}
			\item Architectures des calculateurs / Architecture de communication
			\item LAN\footnote{Local Area Network} ou RLE\footnote{Réseau Local d'Entreprise}
			\item WAN\footnote{Wide Area Network} ou RLD\footnote{Réseau Longue Distance}
			\item DAN\footnote{Departmental Area Network}, MAN\footnote{Metropolitain Area Network}\ldots
		\end{enumerate}
		\begin{tabular}{| c c c |}
			\hline
			Distance & & Exemple\\
			1m & Square meter & PAN\\
			\hline
			10m & Room & LAN\\
			\hline
			100m & Building & LAN\\
			\hline
			1km & Campus & LAN, DAN\\
			\hline
			100 km & City & WAN\\
			\hline
			1000km & Continent & WAN\\
			\hline
			10 000km & Planet & WAN, The Internet\\
			\hline
		\end{tabular}

		Les autres critères de classifications : 
		\begin{itemize}
			\item Débit
			\item Architecture (OSI,TCP/IP)
			\item \ldots
		\end{itemize}

		Les critères de classifications pour un LAN: 
		\begin{itemize}
			\item PABX
			\item Bureautique
			\item Industriel
			\item Large bande
			\item \ldots
		\end{itemize}
	\chapter{Topologie}
	\chapter{Normalisation}
	\chapter{Architecture de communication}
	\chapter{Modèle OSI}
	\chapter{Approche métiers}

\end{document}






