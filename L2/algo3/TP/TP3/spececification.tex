\documentclass{article}


\usepackage{lmodern}
\usepackage{xcolor}
\usepackage[utf8]{inputenc}
\usepackage[T1]{fontenc}
\usepackage[francais]{babel}
\usepackage[top=1.7cm, bottom=1.7cm, left=1.7cm, right=1.7cm]{geometry}
\usepackage{verbatim}
\usepackage{tikz} %Vectoriel
\usepackage{listings}
\usepackage{fancyhdr}
\usepackage{multido}
\usepackage{amssymb}

\newcommand{\titre}{Algorithmique en langage C}
\newcommand{\numTD}{3}

\newcommand{\module}{algo}
\newcommand{\sigle}{algo}

\newcommand{\semestre}{3}


\usepackage{ifthen}
\date{\today}

\chead{Antoine de \bsc{Roquemaurel}}
\rhead{TP\no\numTD}
\lhead{\titre}
%\makeindex

\lfoot{Université Toulouse III -- Paul Sabatier}
\rfoot{\sigle\semestre}
%\rfoot{}
\cfoot{--~~\thepage~~--}

\makeglossary
\makeatletter
\def\clap#1{\hbox to 0pt{\hss #1\hss}}%

\def\haut#1#2#3{%
	\hbox to \hsize{%
		\rlap{\vtop{\raggedright #1}
	}%
	\hss
	\clap{\vtop{\centering #2}
}%
\hss
\llap{\vtop{\raggedleft #3}}}}%
\def\bas#1#2#3{%
	\hbox to \hsize{%
		\rlap{\vbox{
			\raggedright #1
		}
	}%
	\hss \clap{\vbox{\centering #2}}%
	\hss
	\llap{\vbox{\raggedleft #3}}}
}%
\def\maketitle{%
	\thispagestyle{empty}{%
		\haut{}{\@blurb}{}
		%	
		%\vfill

		\begin{center}
			\vspace{-2.0cm}
			\usefont{OT1}{ptm}{m}{n}
			\huge \@numeroTD \@title
		\end{center}
		\par
		\hrule height 1pt
		\par
		\vspace{1cm}
		\bas{}{}{}
}%
}
\def\date#1{\def\@date{#1}}
\def\author#1{\def\@author{#1}}
\def\numeroTD#1{\def\@numeroTD{#1}}
\def\title#1{\def\@title{#1}}
\def\location#1{\def\@location{#1}}
\def\blurb#1{\def\@blurb{#1}}
\date{\today}
\newboolean{monBool}
\setboolean{monBool}{true}
\author{}
\title{}
\ifthenelse{\equal{\numTD}{}}{
\numeroTD{}
}
{
	\numeroTD{TP \no\numTD~--- }
}
\location{Amiens}\blurb{}
%\makeatother
\title{\titre}
\author{%Semestre \semestre
}

\location{Toulouse}
\blurb{%
\vspace{-35px}
\begin{flushleft}
	Université Toulouse III -- Paul Sabatier\\
	L2 Informatique\\
\end{flushleft}
\begin{flushright}
	\vspace{-45px}
	\Large \textbf \module \\
	\normalsize \textit \today\\
	Semestre \semestre
	\vspace{30px}
\end{flushright}
Antoine de \bsc{Roquemaurel}
}%



%\title{Cours \\ \titre}
%\date{\today\\ Semestre \semestre}

%\lhead{Cours: \titre}
%\chead{}
%\rhead{\thepage}

%\lfoot{Université Paul Sabatier Toulouse III}
%\cfoot{\thepage}
%\rfoot{\sigle\semestre}

\pagestyle{fancy}

\newcommand{\mod}{\textrm{mod}}

\begin{document}
	\maketitle
	Dans ce fichier, sont disponibles tous les cartouches des exercices, ceci afin d'avoir une meilleure présentation du langage logique.
	\section{Exercice 1}
	~\\
	Exo1: Modification d'un vecteur de 10 entiers\\
	Données : int T[10]\\
	Après l'étape 1, Résultat1 : int t[10].
	\begin{eqnarray*}
		PS1 &=& \forall i : ( ( 0 \leq i \leq 9 \wedge T[i]\ \mod\ 2 = 1) \rightarrow t[i]= 0) \wedge \\&&\forall i : ( ( 0 \leq i \leq 9 \wedge T[i]\ \mod\ 2 = 0) \rightarrow t[i] = T[i])\\
		PS2 &=& \forall i : ( ( 0 \leq i \leq 9 \wedge T[i]\ \mod\ 2 = 1 \vee i\ \mod\ 2 = 1) \rightarrow t[i] = 0) \wedge \\&&\forall i : ( ( 0 \leq i \leq 9 \wedge T[i]\ \mod\ 2 = 0) \rightarrow t[i] = T[i])
	\end{eqnarray*}
	\section{Exercice 2}
	~\\
	Exo2: Détermination du caractère palindrome d'un tableau\\
	Données : int N, int T[10]\\
	Résultat : \texttt{bool} \texttt{est\_palin}, $\texttt{est\_palin} =  \forall i : 0 \leq i < N \rightarrow T[i] = T[N - 1 -i]$
	\section{Exercice 3}
	~\\
	Exo2: Change l'ordre des éléments d'un tableau \\
	Données : int N, int T[10]\\
	Résultat : int t[10], $\forall i : 0 \leq i < N \rightarrow t[i]=T[N-1-i]$

	\section{Exercice 4}
	~\\
	Exo4: Tri pair impair\\
	Données : int N, int T[N]\\
	Résultat : int t[N], \\
	
	$t=(\forall i : 0 \leq i < (\nu j : 0 \leq j \leq N \wedge T[j] \mod 2 = 0) \wedge T[i] \mod\ 2 = 0) \wedge\\
	(
		\forall i : ((\nu j : 0 \leq j < N \wedge T[j] \mod 2 = 0) < i < N) \wedge T[i] \mod\ 2 = 1
	)$ 
\end{document}






