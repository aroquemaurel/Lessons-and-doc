\documentclass{article}


\usepackage{lmodern}
\usepackage{xcolor}
\usepackage[utf8]{inputenc}
\usepackage[T1]{fontenc}
\usepackage[francais]{babel}
\usepackage[top=1.7cm, bottom=1.7cm, left=1.7cm, right=1.7cm]{geometry}
\usepackage{verbatim}
\usepackage{tikz} %Vectoriel
\usepackage{listings}
\usepackage{fancyhdr}
\usepackage{multido}
\usepackage{amssymb}

\newcommand{\titre}{Algorithmique en langage C}
\newcommand{\numTD}{3}

\newcommand{\module}{algo}
\newcommand{\sigle}{algo}

\newcommand{\semestre}{3}

\input{/home/aroquemaurel/cours/includesLaTeX/entete-l2-tp.tex}
\newcommand{\mod}{\textrm{mod}}

\begin{document}
	\maketitle
	Dans ce fichier, sont disponibles tous les cartouches des exercices, ceci afin d'avoir une meilleure présentation du langage logique.
	\section{Exercice 1}
	~\\
	Exo1: Modification d'un vecteur de 10 entiers\\
	Données : int T[10]\\
	Après l'étape 1, Résultat1 : int t[10].
	\begin{eqnarray*}
		PS1 &=& \forall i : ( ( 0 \leq i \leq 9 \wedge T[i]\ \mod\ 2 = 1) \rightarrow t[i]= 0) \wedge \\&&\forall i : ( ( 0 \leq i \leq 9 \wedge T[i]\ \mod\ 2 = 0) \rightarrow t[i] = T[i])\\
		PS2 &=& \forall i : ( ( 0 \leq i \leq 9 \wedge T[i]\ \mod\ 2 = 1 \vee i\ \mod\ 2 = 1) \rightarrow t[i] = 0) \wedge \\&&\forall i : ( ( 0 \leq i \leq 9 \wedge T[i]\ \mod\ 2 = 0) \rightarrow t[i] = T[i])
	\end{eqnarray*}
	\section{Exercice 2}
	~\\
	Exo2: Détermination du caractère palindrome d'un tableau\\
	Données : int N, int T[10]\\
	Résultat : \texttt{bool} \texttt{est\_palin}, $\texttt{est\_palin} =  \forall i : 0 \leq i < N \rightarrow T[i] = T[N - 1 -i]$
	\section{Exercice 3}
	~\\
	Exo2: Change l'ordre des éléments d'un tableau \\
	Données : int N, int T[10]\\
	Résultat : int t[10], $\forall i : 0 \leq i < N \rightarrow t[i]=T[N-1-i]$

	\section{Exercice 4}
	~\\
	Exo4: Tri pair impair\\
	Données : int N, int T[N]\\
	Résultat : int t[N], \\
	
	$t=(\forall i : 0 \leq i < (\nu j : 0 \leq j \leq N \wedge T[j] \mod 2 = 0) \wedge T[i] \mod\ 2 = 0) \wedge\\
	(
		\forall i : ((\nu j : 0 \leq j < N \wedge T[j] \mod 2 = 0) < i < N) \wedge T[i] \mod\ 2 = 1
	)$ 
\end{document}






