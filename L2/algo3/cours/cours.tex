\documentclass[12pt,a4paper,openany]{article}

\usepackage{lmodern}
\usepackage{xcolor}
\input{/home/aroquemaurel/cours/includesLaTeX/couleurs.tex}

\usepackage[utf8]{inputenc}
\usepackage[T1]{fontenc}
\usepackage[francais]{babel}
\usepackage[top=1.7cm, bottom=1.7cm, left=1.7cm, right=1.7cm]{geometry}
\usepackage{verbatim}
\usepackage[urlbordercolor={1 1 1}, linkbordercolor={1 1 1}, linkcolor=vert1, urlcolor=bleu, colorlinks=true]{hyperref}
\usepackage{tikz} %Vectoriel
\usepackage{listings}
\usepackage{fancyhdr}
\usepackage{multido}
\usepackage{amssymb}

\newcommand{\titre}{Algorithmie en langage C}

\newcommand{\pole}{}
\newcommand{\sigle}{}

\newcommand{\semestre}{3}

\input{/home/aroquemaurel/cours/includesLaTeX/listings.tex} %prise en charge du langage algo
\input{/home/aroquemaurel/cours/includesLaTeX/entete_iut-cours.tex}




%----------------------------------------------------------------------------------------
%	DEFINITION OF COLORED BOXES
%----------------------------------------------------------------------------------------

\RequirePackage[framemethod=default]{mdframed} % Required for creating the theorem, definition, exercise and corollary boxes

% Theorem box
\newmdenv[skipabove=7pt,
skipbelow=7pt,
backgroundcolor=black!5,
linecolor=ocre,
innerleftmargin=5pt,
innerrightmargin=5pt,
innertopmargin=5pt,
leftmargin=0cm,
rightmargin=0cm,
innerbottommargin=5pt]{tBox}

% Exercise box	  
\newmdenv[skipabove=7pt,
skipbelow=7pt,
rightline=false,
leftline=true,
topline=false,
bottomline=false,
backgroundcolor=ocre!10,
linecolor=ocre,
innerleftmargin=5pt,
innerrightmargin=5pt,
innertopmargin=5pt,
innerbottommargin=5pt,
leftmargin=0cm,
rightmargin=0cm,
linewidth=4pt]{eBox}	

% Definition box
\newmdenv[skipabove=10pt,
skipbelow=10pt,
rightline=false,
leftline=true,
topline=false,
bottomline=false,
linecolor=ocre,
innerleftmargin=5pt,
innerrightmargin=5pt,
innertopmargin=0pt,
leftmargin=0cm,
rightmargin=0cm,
linewidth=4pt,
innerbottommargin=0pt]{dBox}	

% Corollary box
\newmdenv[skipabove=7pt,
skipbelow=7pt,
rightline=false,
leftline=true,
topline=false,
bottomline=false,
linecolor=gray,
backgroundcolor=black!5,
innerleftmargin=5pt,
innerrightmargin=5pt,
innertopmargin=5pt,
leftmargin=0cm,
rightmargin=0cm,
linewidth=4pt,
innerbottommargin=5pt]{cBox}		

% Corollary box
\newmdenv[skipabove=7pt,
skipbelow=7pt,
rightline=true,
leftline=false,
topline=false,
bottomline=true,
linecolor=gray,
backgroundcolor=black!5,
innerleftmargin=5pt,
innerrightmargin=5pt,
innertopmargin=5pt,
leftmargin=0cm,
rightmargin=0cm,
linewidth=1pt,
innerbottommargin=5pt]{rBox}				  
		  

% Creates an environment for each type of theorem and assigns it a theorem text style from the "Theorem Styles" section above and a colored box from above
\newenvironment{theorem}{\begin{tBox}\begin{theoremeT}}{\end{theoremeT}\end{tBox}}
\newenvironment{example}{\begin{exampleT}}{\hfill{\tiny\ensuremath{\blacksquare}}\end{exampleT}}
\newenvironment{definition}{\begin{dBox}\begin{definitionT}}{\end{definitionT}\end{dBox}}
\newenvironment{attention}{\begin{eBox}\small}{\end{eBox}}				  	
\newenvironment{exemple}{\begin{cBox}\small}{\end{cBox}}	

%----------------------------------------------------------------------------------------
%	REMARK ENVIRONMENT
%----------------------------------------------------------------------------------------

\newenvironment{remarque}{\par\vskip10pt\small
\begin{rBox}
\begin{list}{}{
\leftmargin=35pt % Indentation on the left
\rightmargin=25pt}\item\ignorespaces % Indentation on the right
\makebox[-2.5pt]{\begin{tikzpicture}[overlay]
\node[draw=ocre!60,line width=1pt,circle,fill=ocre!25,font=\sffamily\bfseries,inner sep=2pt,outer sep=0pt] at (-15pt,0pt){\textcolor{ocre}{R}};\end{tikzpicture}} % Orange R in a circle
\advance\baselineskip -1pt}
{\end{list}\vskip1mm\end{rBox}\vskip5pt} % Tighter line spacing and white space after remark



\input{/home/aroquemaurel/cours/includesLaTeX/polices.tex}
\input{/home/aroquemaurel/cours/includesLaTeX/affichageChapitre.tex}

\begin{document}
	\setcounter{tocdepth}{2}
	\setcounter{secnumdepth}{3}
	\maketitle
	\section{Paradigmes de programmation}
		Un paradigme est une manière de programmer, il en existe plusieurs: 
		\subsection{Programmation fonctionnelles}
		\begin{description}
		\item[Type de langage] \footnote{Traduction du langage source vers le langage cible(compilation) + une édition de liens, qui est une instanciation sur la machine d'exécution (Recherche d'adresse, mémoire, résolution de fonctions) Elle peut être statique ou dynamique. Ex: C, Adda} ou interprétés\footnote{
		Le langage source est traduit en langage cible à la volée par un interpréteur. Il est ainsi possible de modifier le programme pendant le fonctionnement du programme.}. Ce paradigme
		\item[Entité de base] Appel de fonction
		\item[Structure de contrôle] Approche récursive. 


		Elle est utilisée pour des systèmes critiques\footnote{Besoin d'une sureté de fonctionnement}. Elle à une approche très mathématiques, ce qui 
		permet d'avoir des outils de preuves générique.

		Elle possède une abstraction de l'environnement d'exécution, approche détachée de la machine, pas de notion de mémoire.

		\exemple{Le Caml est un langage de programmation fonctionnelle}
	\end{description}
		\subsection{Programmation déclarative}
		\begin{description}
			\item[Type de langage] Interprété
			\item[Entité de base] Règles de déduction logique.
			\item[Structure de contrôle  ]
			Possède une abstraction de la machine cible.
			\exemple{Le prolog est un langage de programmation déclarative}
		\end{description}
		\subsection{Programmation Impérative}
			La programmation est directement liée à la machine d'exécution.
			\begin{description}	
				\item[Type de langage] Compilé ou Interprété
				\item[Entité de base] Affectation d'une valeur à une variable, qui est une place en mémoire.
				\item[Structure de contrôle] Séquence, sélection, répétition.
			\exemple{C, Python, Ada \ldots}
			\end{description}

	\section{Programmation impérative en C}
	Énormément de langage sont fondés sur la syntaxe du langage C.

	Il a été développé dans les années 1960 par Dennis Ritchie. 

	On trouvera toujours une partie description de l'organisation des données en mémoire\footnote{C'est un grand tableau découpé en cases mémoire.},
	nous aurons donc une déclaration de variables et un type de données.
	\begin{lstlisting}[caption=Syntaxe de déclaration de variable]
type nomVariable;		
	\end{lstlisting}
	\subsection{Description de l'organisation des données en mémoire}\label{types}
	Le C possède différents type de données: 
	\begin{description}
		\item[int] Entiers signés
		\item[unsigned int] Entiers non signés 
		\item[float] Nombre réel sur 32bits. 
		\item[double] Nombre réel sur 64bits.
		\item[char] Entier signé sur 8bits.
		\item[pointeur] type* ptr; La case mémoire contient une adresse.
	\end{description}
	\subsection{Code syntaxe}
	\subsubsection{Blocs}
\begin{lstlisting}[language=C, caption=Syntaxe d'un bloc]
bloc { // début du bloc
} //fin du bloc
\end{lstlisting}
	Toute variable est visible dans son bloc de déclaration et ses blocs imbriqués.

	Un bloc transforme une séquence en action.

	\subsubsection{Séquence}
\begin{lstlisting}[language=C, caption=Syntaxe des actions]
action 1;
action 2;
action 3;
\end{lstlisting}
	\subsubsection{Séléction}
\begin{lstlisting}[language=C, caption=Syntaxe d'une structure de contrôle]
if(conditon) {
	action 1;
} else {
	action 2;
}
\end{lstlisting}
Condition est une expression booléenne\footnote{Expression renvoyant vrai($!= 0$ ou faux($=0$)} %% TODO

\subsubsection{Répétition}
\begin{lstlisting}[language=C, caption=Syntaxe de répétition]
while(condition) {
	action;
}
\end{lstlisting}
Condition est une expression booléenne, tant que la condition est vrai, les actions se répètent.

\subsubsection{Affectation}
\begin{lstlisting}[language=C, caption=Syntaxe d'une affectation ]
variable = expression;
\end{itemize}
\end{lstlisting}
\subsubsection{Opérateurs de base sur les types}
\begin{description}
	\item[=] Affectation
	\item[+, -, /, *] Opérateurs arithmétiques.
	\item[\&\&, ||, !] Opérateurs logiques 
	\item[==, !=, <, >, <=, >=] Opérateurs booléens
	\item[++i, i++, --i, i--] Opérateur unaires d'incrémentation.
\end{description}
\subsubsection{Opérateurs d'entrées / sorties}
\paragraph{Ecriture}~\\
\begin{lstlisting}[language=C, caption=Syntaxe de l'appel de printf]
printf('format', var1, var2);
\end{lstlisting}
La chaine format peut contenir une chaine de caractères, avec des caractères spéciaux : 
\begin{description}
	\item[\texttt{'\%d'}] Entier sous forme décimale
	\item[\texttt{'\%ox'}] Entier sous forme hexadécimale
	\item[\texttt{'\%f'}] Flottant 
	\item[\texttt{'\%c'}] Caractère 
	\item[\texttt{'\%s'}] Chaine de caractères 
	\item[\texttt{'\\n'}] Vide le buffer et fait un retour chariot 
	\item[\texttt{'\\t'}] Tabulation 
	\item[\texttt{'\\r'}] Revient en début de ligne.
	\item['\ldots'] RTFM	
\end{description}
Les différents formats doivent être dans l'ordre des variables passés en paramètres.
\paragraph{Lecture}~\newline
\begin{lstlisting}[language=C, caption=Syntaxe de l'appel de scanf]
scanf('format', &var1); // & représente l'adresse de la variable dans laquelle écrire.
\end{lstlisting}
\attention{L'utilisation de cette fonction est risquée. En effet, un utilisateur malveillant peut écrire à des cases mémoires où il n'est pas autorisé.}

\subsubsection{Tableaux}
Un tableau est une collection d'éléments de même type.
\begin{lstlisting}[language=C, caption=Syntaxe de déclaration d'un tableau]
// avec tableau le nom de la variable et N la taille du tableau. 
type tableau[N], i; 
i = tableau[P]; //i recoit la valeur de la case P du tableau
\end{lstlisting}
\attention{Un tableau commence toujours à 0 et finit à N-1, ainsi, il faut faire très attention au dépassement de la taille d'un tableau.}
\subsubsection{Les sous-programmes}
Un sous-programme est un sous-ensemble du programme dans sa hiérarchie fonctionnelle. En C, il correspond toujours à une fonction ou une procédure.
\begin{lstlisting}[language=C, caption=Syntaxe d'un sous programme]
typeRetour nomFonction (typeArg1 nomArg1, typeArg2 nomArg2) {
	/*
     *	code
	 */
	 [return (valeur)];
}
\end{lstlisting}
\texttt{typeRetour} peut posséder comme valeur les même types qu'une variable, voir \ref{types} page \pageref{types}. Celui-ci peut également être \texttt{void}, 
cela signifie que que la fonction ne renvoie rien, c'est donc une procédure.
\subsection{Structure d'une programme en C}
\begin{tabular}{p{8cm}|p{8cm}}
	\textbf{Interface} & \textbf{Implantation}\\
	\begin{minipage}{0.8\textwidth}
		\begin{itemize}
			\item Déclaration des fonctions (prototype)
			\item Constantes, types
			\item Comment utiliser le programme\\~~~~ $\Rightarrow$ \'Ecrire dans un fichier .h (header)
			\item Préprocesseur (define, macros, \ldots)
		\end{itemize}
	\end{minipage}&
	\begin{minipage}{0.8\textwidth}
		\begin{itemize}
			\item Définitions des fonctions: le code\\~~~~ $\Rightarrow$ \'Ecrire dans un fichier .c
		\end{itemize}
	\end{minipage}
\end{tabular}
Un programme en C ne possède qu'une seul point d'entrée : une instruction est exécuté, c'est la fonction \texttt{main}.
\begin{lstlisting}[language=C, caption=Point d'entrée du programme: le main]
int main (int argc, char **argv); 
//le programme renvoie un entier. C'est le profil d'une fonction.
\end{lstlisting}

\subsection{La compilation}
% TODO Schéma
\subsubsection{\'Etape 1 : Le préprocesseur}
Le pré processeur sont les instructions situés en dehors d'un programme, ceux-ci sont préfixé par un dièse (\#).
\begin{description}
	\item[Entrée] fichier.c
	\item[Sortie] fichier obtenu une fois les modifications effectués.
\end{description}~
\begin{lstlisting}[language=C, caption=Exemple d'instructions pré-processeurs]
#include // remplace par le continu du fichier inclus
#define Arg1 Arg2 // remplace syntaxique de Arg1 par Arg2
\end{lstlisting}
\newpage
\subsubsection{Etape 2 : La compilation}
\begin{description}
	\item[Entrée] fichier.c, fichier.h
	\item[Sortie] fichier.o 
\end{description}
\begin{lstlisting}[language=bash]
gcc -c fic1.c fic2.c fic3.c # Créé les fichiers .c
gcc *.o nomExe #Créer l'executable.
\end{lstlisting}
\remarque{La compilation sera étudiée en détails lors des cours de L3 et M1}
\section{Exercice}
\'Ecrire un programme qui lit une série de 10 valeurs et affiche la position du minimum et du maximum de la série.
\subsection{\'Etape 1: Analyser le problème}
\begin{enumerate}
	\item Lire les valeurs
	\item calculer les min et max
	\item afficher le résultat
\end{enumerate}
\subsection{\'Etape 2: Spécifier les sous-problèmes}
Identifier les entrée, les sorties et leurs propriétés.
\subsubsection{\texttt{LireLesValeurs}}
\begin{description}
	\item[Entrée] Nombre, les valeurs à lire
	\item[Sortie] Tableau contenant les valeurs lues
\end{description}
\subsubsection{\texttt{CalculerMinEtMax}}
\begin{description}
	\item[Entrée] Le tableau des valeurs et le nombre de valeur 
	\item[Sortie] Position, min et max. 
\end{description}
\newpage
\subsection{\'Etape 3: Le code}
\lstinputlisting[language=C, caption=Code du programme]{exo1.c}
% Glossaire
	% Compilation
	% Interpretation
	% Edition de liens
	% Intérpreteur
\end{document}






