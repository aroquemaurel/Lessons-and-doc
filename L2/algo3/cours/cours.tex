\documentclass[12pt,a4paper,openany]{article}

\usepackage{lmodern}
\usepackage{xcolor}
\input{/home/satenske/cours/includesLaTeX/couleurs.tex}

\usepackage[utf8]{inputenc}
\usepackage[T1]{fontenc}
\usepackage[francais]{babel}
\usepackage[top=1.7cm, bottom=1.7cm, left=1.7cm, right=1.7cm]{geometry}
\usepackage{verbatim}
\usepackage[urlbordercolor={1 1 1}, linkbordercolor={1 1 1}, linkcolor=vert1, urlcolor=bleu, colorlinks=true]{hyperref}
\usepackage{tikz} %Vectoriel
\usepackage{listings}
\usepackage{fancyhdr}
\usepackage{multido}
\usepackage{amssymb}

\newcommand{\titre}{Algorithmie en langage C}

\newcommand{\pole}{}
\newcommand{\sigle}{}

\newcommand{\semestre}{3}

\input{/home/satenske/cours/listings.tex} %prise en charge du langage algo
\input{/home/satenske/cours/entete_iut-cours.tex}




%----------------------------------------------------------------------------------------
%	DEFINITION OF COLORED BOXES
%----------------------------------------------------------------------------------------

\RequirePackage[framemethod=default]{mdframed} % Required for creating the theorem, definition, exercise and corollary boxes

% Theorem box
\newmdenv[skipabove=7pt,
skipbelow=7pt,
backgroundcolor=black!5,
linecolor=ocre,
innerleftmargin=5pt,
innerrightmargin=5pt,
innertopmargin=5pt,
leftmargin=0cm,
rightmargin=0cm,
innerbottommargin=5pt]{tBox}

% Exercise box	  
\newmdenv[skipabove=7pt,
skipbelow=7pt,
rightline=false,
leftline=true,
topline=false,
bottomline=false,
backgroundcolor=ocre!10,
linecolor=ocre,
innerleftmargin=5pt,
innerrightmargin=5pt,
innertopmargin=5pt,
innerbottommargin=5pt,
leftmargin=0cm,
rightmargin=0cm,
linewidth=4pt]{eBox}	

% Definition box
\newmdenv[skipabove=10pt,
skipbelow=10pt,
rightline=false,
leftline=true,
topline=false,
bottomline=false,
linecolor=ocre,
innerleftmargin=5pt,
innerrightmargin=5pt,
innertopmargin=0pt,
leftmargin=0cm,
rightmargin=0cm,
linewidth=4pt,
innerbottommargin=0pt]{dBox}	

% Corollary box
\newmdenv[skipabove=7pt,
skipbelow=7pt,
rightline=false,
leftline=true,
topline=false,
bottomline=false,
linecolor=gray,
backgroundcolor=black!5,
innerleftmargin=5pt,
innerrightmargin=5pt,
innertopmargin=5pt,
leftmargin=0cm,
rightmargin=0cm,
linewidth=4pt,
innerbottommargin=5pt]{cBox}		

% Corollary box
\newmdenv[skipabove=7pt,
skipbelow=7pt,
rightline=true,
leftline=false,
topline=false,
bottomline=true,
linecolor=gray,
backgroundcolor=black!5,
innerleftmargin=5pt,
innerrightmargin=5pt,
innertopmargin=5pt,
leftmargin=0cm,
rightmargin=0cm,
linewidth=1pt,
innerbottommargin=5pt]{rBox}				  
		  

% Creates an environment for each type of theorem and assigns it a theorem text style from the "Theorem Styles" section above and a colored box from above
\newenvironment{theorem}{\begin{tBox}\begin{theoremeT}}{\end{theoremeT}\end{tBox}}
\newenvironment{example}{\begin{exampleT}}{\hfill{\tiny\ensuremath{\blacksquare}}\end{exampleT}}
\newenvironment{definition}{\begin{dBox}\begin{definitionT}}{\end{definitionT}\end{dBox}}
\newenvironment{attention}{\begin{eBox}\small}{\end{eBox}}				  	
\newenvironment{exemple}{\begin{cBox}\small}{\end{cBox}}	

%----------------------------------------------------------------------------------------
%	REMARK ENVIRONMENT
%----------------------------------------------------------------------------------------

\newenvironment{remarque}{\par\vskip10pt\small
\begin{rBox}
\begin{list}{}{
\leftmargin=35pt % Indentation on the left
\rightmargin=25pt}\item\ignorespaces % Indentation on the right
\makebox[-2.5pt]{\begin{tikzpicture}[overlay]
\node[draw=ocre!60,line width=1pt,circle,fill=ocre!25,font=\sffamily\bfseries,inner sep=2pt,outer sep=0pt] at (-15pt,0pt){\textcolor{ocre}{R}};\end{tikzpicture}} % Orange R in a circle
\advance\baselineskip -1pt}
{\end{list}\vskip1mm\end{rBox}\vskip5pt} % Tighter line spacing and white space after remark



\input{/home/satenske/cours/includesLaTeX/polices.tex}
\input{/home/satenske/cours/includesLaTeX/affichageChapitre.tex}

\begin{document}
	\setcounter{tocdepth}{2}
	\setcounter{secnumdepth}{3}
	\maketitle
	\section{Paradigmes de programmation}
		Un paradigme est une manière de programmer, il en existe plusieurs: 
		\subsection{Programmation fonctionnelles}
		\begin{description}
		\item[Type de langage] \footnote{Traduction du langage source vers le langage cible(compilation) + une édition de liens, qui est une instanciation sur la machine d'execution (Recherche d'adresse, mémoire, résolution de fonctions) Elle peut être statique ou dynamique. Ex: C, Adda} ou interprétés\footnote{
		Le langage source est traduit en langage cible à la volée par un intérpreteur. Il est ainsi possible de modifier le programme pendant le fonctionnement du programme.}. Ce paragradigme
		\item[Entité de base] Appel de fonction
		\item[Structure de contrôle] Approche récursive. 


		Elle est utilisée pour des systèmes critiques\footnote{Besoin d'une sureté de fonctionnement}. Elle à une approche très mathématiques, ce qui 
		permet d'avoir des outils de preuves générique.

		Elle possède une abstraction de l'environnement d'execution, approche détachée de la machine, pas de notion de mémoire.

		\exemple{Le Caml est un langage de programmation fonctionnelle}
	\end{description}
		\subsection{Programmation déclarative}
		\begin{description}
			\item[Type de langage] Intérprété
			\item[Entité de base] Règles de déduction logique.
			\item[Structure de contrôle  ]
			Possède une abstraction de la machine cible.
			\exemple{Le prolog est un langage de programmation déclarative}
		\end{description}
		\subsection{Programmation Impérative}
			La programmation est directement liée à la machine d'execution.
			\begin{description}	
				\item[Type de langage] Compilé ou Intérprété
				\item[Entité de base] Affectation d'une valeure à une variable, qui est une place en mémoire.
				\item[Structure de contrôle] Séquence, séléction, répétition.
			\exemple{C, Python, Ada \ldots}
			\end{description}

	\section{Programmation impérative en C}
	\'Enormement de langage sont fondés sur la syntaxe du langage C.

	Il a été développé dans les années 1960 par Denis Ritchie. %% Otho? 

	On trouvera toujours une partie description de l'organisation des données en mémoire\footnote{C'est un grand tableau découpé en cases mémoire.},
	nous aurons donc une déclaration de variables et un type de données.
	\begin{lstlisting}[caption=Syntaxe de déclaration de variable]
type nomVariable;		
	\end{lstlisting}
	\subsection{Description de l'organisation des données en mémoire}
	Le C possède différents type de données: 
	\begin{description}
		\item[int] Entiers signés
		\item[unsigned int] Entiers non signés 
		\item[float] Nombre réél sur 32bits. 
		\item[double] Nombre réél sur 64bits.
		\item[char] Entier signé sur 8bits.
		\item[pointeur] type* ptr; La case mémoire contient une adresse.
	\end{description}
	\subsection{Code syntaxe}
	\subsubsection{Blocs}
\begin{lstlisting}[language=C, caption=Syntaxe de déclaration de variable]
bloc { // début du bloc
} //fin du bloc
\end{lstlisting}
	Toute variable est visible dans son bloc de déclaration et ses blocs imbriqués.

	Un bloc transforme une séquence en action.

	\subsubsection{Séquence}
\begin{lstlisting}[language=C, caption=Syntaxe de déclaration de variable]
action 1;
action 2;
action 3;
\end{lstlisting}
	\subsubsection{Séléction}
\begin{lstlisting}[language=C, caption=Syntaxe de déclaration de variable]
if(conditon) {
	action 1;
} else {
	action 2;
}
\end{lstlisting}
Condition est une expression booléenne\footnote{Expression renvoyant vrai($!= 0$ ou faux($=0$)} %% TODO

\subsubsection{Répétition}
\begin{lstlisting}[language=C, caption=Syntaxe de déclaration de variable]
while(condition) {
	action;
}
\end{lstlisting}
Condition est une expression booléenne, tant que la condition est vrai, les actions se repettent.

\subsubsection{Affectation}
\begin{lstlisting}[language=C, caption=Syntaxe de déclaration de variable]
variable = expression;
\end{itemize}
\end{lstlisting}
\subsubsection{Opérateurs de base sur les types}
\begin{description}
	\item[=] Affectation
	\item[+, -, /, *] Opérateurs arithmétiques.
	\item[\&\&, ||, !] Opérateurs logiques 
	\item[==, !=, <, >, <=, >=] Opérateurs booléens
	\item[++i, i++, --i, i--] Opérateur unaires d'incrémentation.
\end{description}
\subsubsection{Opérateurs d'entrées / sorties}
\paragraph{Ecriture}
\begin{lstlisting}[language=C, caption=Syntaxe de déclaration de variable]
	printf('format', var1, var2);
\end{lstlisting}
La chaine format peut contenir une chaine de caractères, avec des caractères spéciaux : 
\begin{description}
	\item['\%d'] Entier sous forme décimale
	\item['\%ox'] Entier sous forme héxadécimale
	\item['\%f'] Flottant 
	\item['\%c'] Caractère 
	\item['\%s'] Chaine de caractères 
	\item['\\n'] Vide le buffer et fait un retour chariot 
	\item['\\t'] Tabulation 
	\item['\\r'] Revient en début de ligne.
	\item['\ldots'] RTFM	
\end{description}
Les différents formats doivent être dans l'ordre des variables passés en paramètres.
\paragraph{Lecture}
\begin{lstlisting}[language=C, caption=Syntaxe de déclaration de variable]
	scanf('format', &var1);
\end{lstlisting}
\attention{Ne jamais utiliser scanf}


% Glossaire
	% Compilation
	% Interpretation
	% Edition de liens
	% Intérpreteur
\end{document}






