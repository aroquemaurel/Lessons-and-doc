
	\chapter{Paradigmes de programmation}
		Un paradigme est une manière de programmer, il en existe plusieurs: 
		\begin{itemize}
			\item La programmation fonctionnelles (cf. \ref{fonctionnelles})
			\item La programmation déclarative (cf. \ref{declarative})
			\item La programmation impérative (cf. \ref{imperative}
		\end{itemize}
		\section{Programmation fonctionnelles} \label{fonctionnelles}
		\begin{description}
		\item[Type de langage] Compilés \footnote{Traduction du langage source vers le langage cible(compilation) + une édition de liens, qui est une instanciation sur la machine d'exécution (Recherche d'adresse, mémoire, résolution de fonctions) Elle peut être statique ou dynamique. Ex: C, Adda} ou interprétés\footnote{
		Le langage source est traduit en langage cible à la volée par un interpréteur. Il est ainsi possible de modifier le programme pendant le fonctionnement du programme.}. 
		\item[Entité de base] Appel de fonction
		\item[Structure de contrôle] Approche récursive. 


		Elle est utilisée pour des systèmes critiques\footnote{Besoin d'une sureté de fonctionnement}. Elle à une approche très mathématiques, ce qui 
		permet d'avoir des outils de preuves générique.

		Elle possède une abstraction de l'environnement d'exécution, approche détachée de la machine, pas de notion de mémoire.

		\exemple{Le Caml est un langage de programmation fonctionnelle}
	\end{description}
	\section{Programmation déclarative} \label{declarative}
		\begin{description}
			\item[Type de langage] Interprété
			\item[Entité de base] Règles de déduction logique.
			\item[Structure de contrôle  ]
			Possède une abstraction de la machine cible.
			\exemple{Le prolog est un langage de programmation déclarative}
		\end{description}
		\newpage
		\section{Programmation Impérative}\label{imperative}
			La programmation est directement liée à la machine d'exécution.
			\begin{description}	
				\item[Type de langage] Compilé ou Interprété
				\item[Entité de base] Affectation d'une valeur à une variable, qui est une place en mémoire.
				\item[Structure de contrôle] Séquence, sélection, répétition.
			\exemple{C, Python, Ada \ldots}
			\end{description}

