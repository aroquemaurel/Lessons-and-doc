
				\chapter{Le tableau de situation}\label{tableauSituation}
		Le tableau de situation sert à tester le programme, la vérification n'est pas exhaustive. Il s'agit de vérifier que l'état des variables est cohérent.
			\begin{description}
				\item[Données en entrée] Programme ``instrumenté'' : code source + point d'arrêt: 
					localisation dans l'espace du programme d'une opération de photographie de l'état de l'ordinateur.
				\item[Opération de transformation] Dénuder le programme et prendre les photos. 
				\item[Donnée en sortie] Liste de ``photos'' qui déçoit l'exécution de la même mémoire au cours de l'exécution.
			\end{description}
		\end{itemize}
		\exemple{
\lstinputlisting[language=C, caption=Exercice 3, numbers=none]{annexes/exo2.c}
Nous choisis le jeu de données $n = 0$ et $n = 4$ afin de passer dans tous les cas possibles.
\begin{center}
\begin{tabular}{c  |  c  c  c  c  }
	&\textbf{\'Echelle} & \textbf{n} & \textbf{f} & \textbf{point d'arrêt}\\
n=0	&1 & 0 & indéfini & 1\\
	\hline
	&/ & / & / & /\\
	\hline
	\hline
	&indéfini & 4 & 4 & 2\\
	\hline
	&indéfini &  3 & 12 & 3\\	
	\hline
	&indéfini & 2 & 24 & 3\\
	\hline
	&indéfini & 1 & 24 & 3\\
	\hline
	&24 & 6 & 24 & 4\\
	\hline
	&/ & / & / & /\\
	\hline
\end{tabular}
\end{center}
		}
