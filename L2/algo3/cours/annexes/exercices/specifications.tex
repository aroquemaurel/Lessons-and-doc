\section{Spécification}
\subsection{Exercice 7}
	\'Ecrire la spécification d'un programme qui dans un tableacu T de N entiers calcul le nombre n de nombre positifs dans le tableau.

	\begin{itemize}
		\item $N > 0$
		\item \texttt{calculeNbPos(T,N,n)}
		\item $(o \leq n \leq N) \wedge (n = \nu I : o \leq I < N T[I] \geq 0) $
	\end{itemize}
\subsubsection{Exercice 8}
	Soit T un tableau croissant (non strict) de N entier et X un entier.\\
	Spécifier un programme qui calcule la position de la dernière occurrence de T inférieure ou égale à X avec $T[0] \leq X < T[N-1]$

	\begin{itemize}
	%	\item $T[0] \leq X < T[N-1]$, $T[I] \leq T[I+1]$, $N > 0$
		\item $(N > 1) \wedge (T[0] \leq X) \wedge (X < T[N-1]) \wedge (\forall I : 0 \leq I < N - 1 \rightarrow T[I] \leq T[I+1])$
		\item \texttt{searchPosition(T, N, X, p));} 
		\item $(0 < p < N - 1) \wedge (T[p] \leq X) \wedge (T[p+1] > X)$
	\end{itemize}
	\remarque{Dans la suite du cours, nous pourrons utiliser un raccourcie afin de savoir si un tableau est trié par ordre croissant : $(T,N,\leq)$

	Celle-ci pourra être utilisée dans la copie à condition qu'elle soit définie au préalable.}

\subsection{Exercice 8}
	Soit un tableau T non vide de N entiers.  \'Ecrire la spécifications du qui programme qui calculent :
	\begin{itemize}
		\item La première position de la valeur max de T
		\item La dernière position de la valeur max de T
	\end{itemize}
	\subsubsection{Calcule de la première position}
	\begin{itemize}
		\item $N > 0$ 
		\item \texttt{searchFirstPosition(T, N, f);} 
		\item $(\forall I :0 \leq I < f \rightarrow T[I] < T[f]) \wedge (\forall I (f \leq I < N) \rightarrow (T[I] \leq T[f]))$ 
	\end{itemize}
	\subsubsection{Calcule de la dernière position}
	\begin{itemize}
		\item $N > 0$ 
		\item \texttt{searchLastPosition(T, N, l);} 
		\item $(\forall I :0 \leq I < l \rightarrow T[I] \leq T[l]) \wedge 
			(\forall I (l < I < N) \rightarrow (T[I] < T[l]))$ 
	\end{itemize}

\subsection{Exercice 9}
	\'Ecrire la spécification d'un programme qui, dans un tableau T de N entiers tous différents cherche la position d'une valeur X si elle existe ou retourne N si elle n'existe pas.

	\begin{itemize}
		\item $(N\geq0) \wedge (\forall I : 0 \leq I < N \rightarrow (\forall (I, J) : 0 \leq I < N \wedge (0 \leq J < N) \rightarrow T[I] = T[J] \leftrightarrow (I = J)$\footnote{Cela peut aussi s'écrire $N\leq 0) \wedge (\forall I (O \leq I < N) \rightarrow \forall J : J \neq I \wedge 0 \leq J < N \rightarrow T[I] \neq T[J]$}
		\item \texttt{search(T, N, x)}
		\item $(0 \leq p < N \wedge T[p] = X) \vee (p=N) \leftrightarrow \forall I (o\leq I < N) \rightarrow T[I] \neq X))$ 
	\end{itemize}

	\subsection{Exercice 10}
	Spécifier un programme qui, dans un tableau T de N éléments trié par ordre croissant non strict retourne la longueur du plus grand plateau\footnote{Un plateau est quand il y a plusieurs fois le même caractère}.

\begin{itemize}
	\item $(T, N, \leq) \wedge N > 0$
	\item \texttt{longueurPlusGrandPlateau(T,N,l);}
	\item $(1 \leq l \leq N) \wedge (\exists I : 0 \leq I < N-l)\wedge (T[I] = T[I+l-1])$ 
\end{itemize}
	\subsection{Exercice 11}
		Spécifier un programme qui, dans un tableau de N entiers calcule le nombre de doublons : un doublon est une succession de 2 nombres identiques.

		\begin{itemize}
			\item $N > 0$
			\item \texttt{calculeDoublons(T, N, n);}
			\item $n = \nu I : 0 \leq I < N \wedge T[I] = T[I+1]$
		\end{itemize}
		\remarque{
			Dans le cas ou deux doublons ne sont pas forcément côte à côte, le prédicat de sortie deviendrait : 
			$$n = \sum^{N-1}_{I=0} \nu J : I < J < N \wedge T[J] = T[I]$$
		}

