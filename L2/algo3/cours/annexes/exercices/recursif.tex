\section{Transformation d'une spécification récursive en un programme itératif}
\subsection{$N^2$}
Ecrire un programme qui calcule $N^2$ pour N entier. 
$$N^2 = \sum^N_{i=1}(2\times i)-1$$
\begin{displaymath}
	\texttt{carre(N)}
	\left\{
	\begin{array}{ll}
		\textrm{Si } \overbrace{N == 0}^{h(N)} & \overbrace{0}^{a}\\
		\textrm{Sinon } & \underbrace{(2*N)-1)}_{f(N)}+\underbrace{\texttt{carre}(N-1)}_{t(N)}
	\end{array}
	\right.
\end{displaymath}


\begin{lstlisting}[language=C]
x= N;
r = 0;
while (N != 0) {
	r += (N << 1) - 1;
	N = N - 1;
}
\end{lstlisting}

\subsection{Palindrome}
Écrire un programme qui détermine si un texte de $N$ caractères est un palindrome.

\begin{displaymath}
	\texttt{palindrome(T,D,F)}
	\left\{
	\begin{array}{ll}
		\left.
		\begin{array}{ll}
		\textrm{Si } F - D == 0 & Vrai \\
		\textrm{Si } F - D == 1 & T[D] == T[F] \\
		\end{array}
		\right) F-D \leq 1 T[D] ==  T[F]
		\\ &\\
		\textrm{Sinon } \texttt{palindrome}(T,D+1,F-1) \underbrace{\wedge}_{\oplus} T[D] == T[F]&
	\end{array}
	\right.
\end{displaymath}
\begin{itemize}
	\item e : true
	\item h : $F - D \leq 1$
	\item f : T[D] ==  T[F]
	\item t : (T, D+1, F-1)
\end{itemize}

 ~\\
\begin{lstlisting}[language=C]
/* x = (T, D, F) */
r= true;
while (!(F-D <= 1) && r) {
	r = r && T[D] == T[F];
	++D;
	--F;
}
r = r && (T[D] = T[F]);
\end{lstlisting}
