\section{Plus faible précondition}
\subsection{Séquence}\label{exoPfpSequence}
\subsubsection{Exercice 1}
\begin{lstlisting}[language=C]
	/* f = i! */
	f = f * (i + 1);
	i = i + 1;
	/* f = i! */
\end{lstlisting}
\begin{eqnarray*}
	f = i! &\rightarrow& \texttt{pfp}("f = f\times (i+1); i = i +1;", f= i!)\\
	f=i!&\rightarrow&\texttt{pfp}("f=f\times(i+1);",\texttt{pfp}("i=i+1",f=i!)\\
	f = i! &\rightarrow& \texttt{pfp}("f = f \times (i + 1);" f = (i + 1)!)
\end{eqnarray*}
\newpage
\subsubsection{Exercice 2}
\begin{lstlisting}[language=C]
	/* (x = A) $\wedge$ (y = B) $\wedge$ (z = C) */
	x = x + y + z;
	z = x - y - z;
	y = x - y - z;
	x = x - y - z;
	/* (x = B) $\wedge$ (y = C) $\wedge$ (z = A) */
\end{lstlisting}
\begin{eqnarray*}
	PE &\rightarrow& \texttt{pfp}("x=x+y+z;z=x-y-z;y=x-y-z;x=x-y-z;", \\&&(x=B)\wedge(y=C)\wedge(z=A)\\
	PE &\rightarrow& \texttt{pfp}("x=x+y+z, z-x-y-z;y=x-y-z;", \\&&\texttt{pfp}("x=x-y-z;(x=B)\wedge(y=C)\wedge(z=A))\\
	PE &\rightarrow& \texttt{pfp}("x=x+y+z, z-x-y-z;y=x-y-z;", x-y-z=B)\wedge y=C \wedge z = A\\
	PE &\rightarrow& \texttt{pfp}("x=x+y+z,z=x-y-z", (y=B)\wedge (x-y-z=C)\wedge z = A\\
	PE &\rightarrow& \texttt{pfp}("x=x+y+z", (y=B)\wedge (z=C) \wedge (x-y-z=A)\\
	PE &\rightarrow& (y=B) \wedge (z=C)\wedge(x=A) \textmd{Vrai parceque }p\rightarrow p = \textmd{vrai}
\end{eqnarray*}

\subsection{Sélection} \label{exoPfpSelect}
\subsubsection{Exercice 1}
\begin{lstlisting}[language=C]
/* x = A $\wedge$ y = B */
if (A < B) {
	x = A;
	y = B;
} else {
	x = B;
	y = A;
}
/* x $\leq$ y */ 
\end{lstlisting}
\begin{eqnarray*}
PE &\rightarrow& \pfp ("\ifp (A < B) \{ x = A; y = B\}\elsep \{ x = B; y =A\}", x \leq y)\\
PE&\rightarrow&((A<B)\rightarrow\pfp("x=A;y=B",x\leq y))\wedge((A\geq B)\rightarrow\pfp("x=B;y=A",x\leq y))\\
PE &\rightarrow& ((( A < B) \rightarrow (A \leq B)) \wedge (A \geq B ) \rightarrow (B \leq A))
\end{eqnarray*}
Vrai par définition. A et toujours inférieur à B. $(A \geq B) \rightarrow (B \leq A)$ est une Tautologie. 
\subsubsection{Exercice 2}
$$\pfp ("\ifp (x \geq y )\{z=x;\}\elsep\{z=y\}",z=\texttt{max}(x,y))$$
\begin{eqnarray*}
x \geq y  &\rightarrow&	\pfp ("\ifp (x \geq y )\{z=x;\}\elsep\{z=y\}",z=\texttt{max}(x,y))\\
x \geq y &\rightarrow& \pfp ("z = x", z = \texttt{max}(x,y))\\
x < y &\rightarrow& \pfp ("z = y", z = \texttt{max}(x,y))\\
x \geq y &\rightarrow& x = \texttt{max}(x,y)\\
x < y &\rightarrow& y = \texttt{max}(x,y)
\end{eqnarray*}
C'est une tautologie par définition de \texttt{max}.

\subsubsection{Exercice 3}
$$\pfp ("\ifp (x > y) \{ if(x \%2 ==  0) \{ x = x - 2\}\} \elsep \{y=y-1;\}", y-2 < x);$$
\begin{eqnarray*}
(x > y) &\rightarrow& \pfp("if(x\%2 == 0) { x = x-2; } ", y-2 < x)) \wedge ( ( x \leq y) \rightarrow \pfp ("y = y-1;", y - 2 < x) )\\
(x > y) &\rightarrow& ( ( (x \%2 = =0) \rightarrow \pfp ("x = x-2", y-2 < x) \wedge \\
& &(x \% 2 != 0) \rightarrow(y-2 < x)) \wedge ( (x \leq y) \rightarrow (y-z < x))\\ 
(x > y) &\rightarrow& ( (x\%2 = = 0)\rightarrow ( (y-2) < (x - 2) ) \wedge ( ( x\%2 != 0) \rightarrow (y-2 < x))) \wedge (x \leq y \rightarrow y-z < x) \\
(x > y) &\rightarrow& (x \leq y \rightarrow y - z < x)
\end{eqnarray*}
\subsection{Boucles}\label{exoPfpBoucles}
\subsubsection{Exercice 1}
\begin{lstlisting}[language=C]
/* $N \geq 0$ */

/* P Tableau de polynôme
* N Degré du polynôme
* X Point ou je veux évoluer le polynôme
* r Résultat du polynôme
*/
calculPolynome(P,N,X,r);
/* $r = \sum^N_{k=0} A[I]X^k $ */
\end{lstlisting}
\begin{lstlisting}[language=C]
/* $N \geq 0$ */

/*	INV = $(o \leq i \leq N) \wedge (r = \sum^i_{k=0} A[k]x^k) \wedge (y = x^i)$*/
while(c) {
/* c \wedge INV */
corps;
/* INV */
}
/* $\neg c \wedge INV$ */
/* $p = \sum^N_{k=0} A[k] x^k$ */
\end{lstlisting}
\paragraph{INVARIANT}
$$(o \leq i \leq N) \wedge (r = \sum^i_{k=0} A[k]x^k) \wedge (y = x^i)$$
\paragraph{Intitialisation}
\begin{enumerate} 
	\item INIT 0 \begin{lstlisting}[language=C,numbers=none]
i = 0; 
p = 0; 
y = 1;
\end{lstlisting}~
\begin{eqnarray*}
PE &\rightarrow& \pfp (\texttt{init}\ 0, \texttt{INV})\\
N \geq 0 &\rightarrow& \pfp ("i=0;p=0", p=\sum^i_k=0 A[k]x^k \wedge 1 = x^i);\\
N \geq 0 &\rightarrow& \pfp ("i=0", 0 = \sum^i_{k=0} A[k]x^k \wedge 1 = x^i\\
N \geq 0 &\rightarrow& 0 = \sum^0_{k=0} A[k]x^k \wedge 1 = x^0\\
(N \geq 0 &\rightarrow& 0 = A[0] \wedge 1 = 1) \Rightarrow \textrm{Faux et Vrai, donc le programme est Faux} \end{eqnarray*}
	\item INIT 1 \begin{lstlisting}[language=C,numbers=none]
i = 0; 
p = A[0]; 
y = 1;
\end{lstlisting}
\begin{eqnarray*}
PE &\rightarrow& \pfp (\texttt{init}\ 1, \texttt{INV})\\
N \geq 0 &\rightarrow& \pfp ("i=0;p=A[0]", p=\sum^i_k=0 A[k]x^k \wedge 1 = x^i);\\
N \geq 0 &\rightarrow& \pfp ("i=0", A[0] = \sum^i_{k=0} A[k]x^k \wedge 1 = x^i\\
N \geq 0 &\rightarrow& A[0] = \sum^0_{k=0} A[k]x^k \wedge 1 = x^0\\
(N \geq 0 &\rightarrow& A[0] = A[0] \wedge 1 = 1) \Rightarrow \textrm{Vrai et Vrai, donc le programme est correct}
\end{eqnarray*}
\end{enumerate}
L'initialisation nécessaire est donc init 1.
\paragraph{Boucles} 
\begin{enumerate} 
	\item \begin{lstlisting}[language=C,numbers=none]
while(i < N) { 
++i; 
p = p + A[i-1] * y; 
y = y * X; 
}
	\end{lstlisting}~
	\'Etape numéro \ref{etape3boucle} (Cf section \ref{pfpBoucle} page \pageref{etape3boucle})
	\begin{eqnarray*}
		\neg C \wedge INV &\rightarrow& PS\\
		(i > N) \wedge (p = \sum^i_{k=0} A[k] x^k) \wedge(y=x^i) &\rightarrow& p = \sum^N_{k=0}A[k]x^k\\
		(i > N) \wedge (p = \sum^i_{k=0} A[k] x^k) \wedge(y=x^i) &\rightarrow& p = \sum^N_{k=0}A[k]x^k\wedge (i=N)\\
		(i > N) &\rightarrow& (i = N) \Rightarrow \textrm{C'est donc faux.}\\
	\end{eqnarray*}
	\remarque{Nous sommes partis de la conclusion et avons essayé de faire apparaitre notre hypothèse en partie droite.}
	\item \begin{lstlisting}[language=C,numbers=none]
while( i != N) { 
++i; 
p = p + a[i-1] * y; 
y = y * X 
} 
\end{lstlisting}~
	\'Etape numéro \ref{etape3boucle} (Cf section \ref{pfpBoucle} page \pageref{etape3boucle})
	\begin{eqnarray*}
		\neg C \wedge INV &\rightarrow& PS\\
		(i \neq N) \wedge (p = \sum^i_{k=0} A[k] x^k) \wedge(y=x^i) &\rightarrow& p = \sum^N_{k=0}A[k]x^k\\
		(i \neq N) \wedge (p = \sum^i_{k=0} A[k] x^k) \wedge(y=x^i) &\rightarrow& p = \sum^N_{k=0}A[k]x^k\wedge (i=N)\\
		(i \neq N) &\rightarrow& (i = N) \Rightarrow \textrm{C'est donc vrai}. 
	\end{eqnarray*}
	\'Etape numéro \ref{etape2boucle} (Cf section \ref{pfpBoucle} page \pageref{etape3boucle})
	\begin{eqnarray*}
		C \wedge INV &\rightarrow& \pfp(corps, INV)\\
		C \wedge INV &\rightarrow& \pfp ("i++;p=p+A[i-1]*y;", (p=\sum^i_{k=0}A[k]x^k)\\&&\wedge(y*x=x^i)\wedge(0\leq i\leq n )\\
		C \wedge INV &\rightarrow& \pfp ("i++;", (p+A[i-1]*y = \sum^i_{k=0}A[k]x^k)\\&&\wedge(y*x=x^i)\wedge(0\leq i\leq n )\\
		(i \neq N) \wedge (0\leq i \leq N) \wedge\\ (p = \sum^i_{k=0} A[k] x^k) \wedge (y=X^i) &\rightarrow& (p+A[i] * i = \sum^{i+1}_{k=0}A[k]x^k) \wedge (y \times x = x^{i+1}) \wedge (0 \leq i+1 \leq N)\\
		(i \neq N) \wedge (0\leq i \leq N) \wedge \\(p = \sum^i_{k=0} A[k] x^k) \wedge (y=X^i) &\rightarrow& A[i] \times x^i = p+A[i+1]x^{i+1} \wedge x^i \times = x^{i+1} \wedge Tautologie \\
		(i \neq N) \wedge (0\leq i \leq N) \wedge \\(p = \sum^i_{k=0} A[k] x^k) \wedge (y=X^i) &\rightarrow& Faux \wedge Tautologie \wedge Tautologie\Rightarrow \textrm{Le programme est donc faux} \\
	\end{eqnarray*}
\item \begin{lstlisting}[language=C,numbers=none]
while(i == N) { 
++i;	
p = p +(A[i] * y * X); 
y = y * X; 
}
\end{lstlisting}
Ce programme effectue la correction de l'erreur détectée plus haut. Il est correct.
\end{enumerate}

\subsubsection{Exercice 2 : Suite de Fibonacci}
\remarque{
Le rang $n$ de la suite de Fibonacci est défini comme suit: \\
$F_n = 1$ si $n=0$ ou $n=1$\\
$F_n = F_n-2 + F_n-1$ si i$n > 1$}

\begin{lstlisting}[language=C,numbers=none]
/* $N \geq 0$ */
i = 0; 
a = 1; 
b = 1;
/* $INV = (F_i = a) \wedge (F_n-1 = b) \wedge (o \leq i \leq N)$ */
while(c) {
/* $c \neg INV$ */
i++;
b += a;
a = b - a;
/* $INV$ */
}
/* $\neg c \wedge INV$ */
/* $F_n = \{\cdots\}$ */
\end{lstlisting}

\remarque{Il est conseillé de commencer par l'étape la plus facile, en effet, une étape et nous n'avons pas à effectuer les autres}
\paragraph{Etape 1}
\begin{eqnarray*}
PE &\rightarrow& \pfp("init", INV)\\
N \geq 0 &\rightarrow& (F_{i+1} = 1) \wedge (F_i = 1) \wedge (i=0)\\
N \geq 0 &\rightarrow& (F_1 = 1) \wedge (F_0 = 1)\Rightarrow \textrm{Tautologie par définition}
\end{eqnarray*}
\paragraph{Etape 3}
\begin{eqnarray*}
(i = N) \wedge (F_i = a) \wedge (F_{i+1} = b) \wedge (0 \leq i \leq N) &\rightarrow& F_N = a\\
(F_n = a) \wedge (F_{N+1}) \wedge (0 \leq N) &\rightarrow& F_N = a \Rightarrow \textrm{Tautologie}
\end{eqnarray*}

\paragraph{Etape 2}
\begin{eqnarray*}
(i \neq N) \wedge (F_i = a) \wedge (F_{i+1} = b) \wedge (0 \leq i \leq N) &\rightarrow& \pfp("\cdots", F_i=b-a \wedge F_{i+1} = b \wedge \cdots)\\
&\rightarrow& \pfp ("\cdots", F_i = b+a - a \wedge F_{i+1} = b+a\\
&\rightarrow& F_i+1 = b \wedge F_{i+2} = b + a \wedge i+1 \leq N\\
&\rightarrow& T \wedge T \textrm{par définition} \wedge T 
\end{eqnarray*}

\paragraph{Etape 4}
Tester une existance de $f > 0$ avant l'execution du corps : $c \wedge INV \rightarrow f > 0$.
\begin{eqnarray*}
(i \neq  N) \wedge (F_i = a) \wedge (F_n-1 = b) \wedge (o \leq i \leq N) &\rightarrow& f > 0\\
 f = N-i \Rightarrow N \geq 0\textrm{ donc Vrai.}
\end{eqnarray*}

\paragraph{\'Etape 5}
\begin{eqnarray*}
f = F \wedge c \wedge INV &\rightarrow& \pfp("corps", f < F)\\
N-i = F \wedge c \wedge INV &\rightarrow& \pfp("corps", N-i < F)\\
N - i = F \wedge c \wedge INV &\rightarrow& N - (i+1) < F\\
&\rightarrow& N -i-1 < F \\
&\rightarrow& F-1 < F \Rightarrow \textrm{Tautologie}
\end{eqnarray*}

\subsubsection{Exercice 3 : plus grand plateau}
\remarque{$(B(N), \geq)$ signifique le tableau est ordonné dans l'ordre croissant non strict}

\subsubsection{Exercice 3 : Définition variante}
\begin{eqnarray*}
	C \wedge INV &\rightarrow& f > 0\\
	J \neq N \wedge INV &\rightarrow& (f = N-j) >0 \Rightarrow \textrm{ Vrai }
\end{eqnarray*}

\begin{eqnarray*}
	N-j  =F \wedge C \wedge INV &\rightarrow& \pfp(corps, N-j < F)\\
	N-j  =F \wedge C \wedge INV &\rightarrow& \pfp(\ifp(\cdots))\cdots,\pfp(j++,N-j<F)\\
	N-j  =F \wedge C \wedge INV &\rightarrow& B|j-p] = B[j] \rightarrow N - j -1 < F \wedge B[j-p] \neq B[j] \rightarrow N-j-1 < F\\
	N-j = F \wedge C \wedge INV \rightarrow N-j-1 < F \wedge F-1 < F
\end{eqnarray*}

\subsubsection{Exercice 4}
\begin{lstlisting}[language=C]
	/* $A \leq 0 \wedge B \geq 0 */
	x = A;
	y = B;
	z = 1;

	/* (z = x^y = A^B) \wedge y \geq 0$ */
	while(y != 0) {
		/* $z *x^y = A^B \wedge y \geq 0$ */
		while(y \% 2 == 0) {
			y = y / 2 ;
			x = x*x;
		}
		/* $z = x^y = A^B \wedge y > 0 \wedge y \%2 \neq 0 $*/
		y--;
		z = z * x;
		/* z = x^y = A^B \wedge y \geq 0 */
	}
	PS : /* z = A^b */

\end{lstlisting}
