\section{Tableau de situation}	
\subsection{Exercice 2}
\lstinputlisting[language=C, caption=Exercice 3]{annexes/exo3.c}
\begin{center}
\begin{tabular}{c  |  c  c  c  c  }
\textbf{Point d'arrêt }& \textbf{T} & \textbf{j} & \textbf{i} &\textbf{n}\\
\hline
1 & 0,0,0,0,0 & 3 & 5 & 1\\
\hline
2 & 0,0,0,1,0 & 4 & 3 & @\\
\hline
3 & 5,9,1,1,1&4&5&\\
\hline
\end{tabular}
\end{center}
\newpage
\subsection{Effets de bords}
\subsubsection{Exercice 4}
\lstinputlisting[language=C, caption=Exercice 4]{annexes/exo4.c}
\begin{center}
\begin{tabular}{c |  c  c  c  c  c  c  }
	\textbf{Point d'arrêt} & \textbf{y} & \textbf{x} & \textbf{t} & \textbf{i} & \textbf{z} & \textbf{f}\\
	\hline
	2 & 10 & / & / & 1 && \\
	\hline
	1 & 11 & 1 & 2 & 1 & / & 13\\
	\hline
	3 & 11 & / & / & 1 & 24 & /\\
	\hline
\end{tabular}
\end{center}
\newpage
\subsubsection{Exercice 5}
\lstinputlisting[language=C, caption=Exercice 5]{annexes/exo5.c}
\begin{center}
\begin{tabular}{c |  c  c  c  c  c  }
	\textbf{Point d'arrêt} & \textbf{i} & \textbf{j} & \textbf{x} & \textbf{y} & \textbf{z}\\
	\hline
	3 & 3 & 7 &/ &/&/\\
	\hline
	1 & &&@i&@i&@j\\
	\hline
	2 & 6 & 12 & @i & @i & @j\\
	\hline
	4 & 6 & 12 & / & / & /\\
	\hline
	5 & 3 & 7 & / & / & / \\
	\hline
	1 & 3&10 &@j&@j&@i\\
	\hline
	2 & 20 & 10 & @j&@j&@i\\
	\hline
\end{tabular}
\end{center}
\newpage
\subsubsection{Exercice 6}
\lstinputlisting[language=C, caption=Exercice 6]{annexes/exo6.c} 
\begin{tabular}{c |  c  c  c  c  c  c  }
	\textbf{Point d'arrêt} & \textbf{i} & \textbf{j} & \textbf{x} & \textbf{a} & \textbf{a} & \textbf{f}\\
	\hline
	2&10&40&/&/&/&/\\
	\hline
	1 & 20 & 40 & / & 10 & 40 & 50\\
	\hline
	3 & 20 & 40 & 50 & / & / & /\\
	\hline
\end{tabular}

