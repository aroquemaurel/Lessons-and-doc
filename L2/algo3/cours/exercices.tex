\chapter{Exercices}
\section{Initiation}
	\subsection{Exercice 1}
\'Ecrire un programme qui lit une série de 10 valeurs et affiche la position du minimum et du maximum de la série.
\subsubsection{\'Etape 1: Analyser le problème}
\begin{enumerate}
	\item Lire les valeurs
	\item calculer les min et max
	\item afficher le résultat
\end{enumerate}
\subsubsection{\'Etape 2: Spécifier les sous-problèmes}
Identifier les entrée, les sorties et leurs propriétés.
\paragraph{\texttt{LireLesValeurs}}
\begin{description}
	\item[Entrée] Nombre, les valeurs à lire
	\item[Sortie] Tableau contenant les valeurs lues
\end{description}
\paragraph{\texttt{CalculerMinEtMax}}
\begin{description}
	\item[Entrée] Le tableau des valeurs et le nombre de valeur 
	\item[Sortie] Position, min et max. 
\end{description}
\newpage
\subsubsection{\'Etape 3: Le code}
\lstinputlisting[language=C, caption=Exercice 1 -- Code du programme]{exo1.c}
\newpage
	\section{Tableau de situation}	
	\subsection{Exercice 2}
\subsection{Exercice 3}

\lstinputlisting[language=C, caption=Exercice 3]{exo3.c}
\begin{center}
\begin{tabular}{c  |  c  c  c  c  }
	\textbf{Point d'arrêt }& \textbf{T} & \textbf{j} & \textbf{i} &\textbf{n}\\
	\hline
	1 & 0,0,0,0,0 & 3 & 5 & 1\\
	\hline
	2 & 0,0,0,1,0 & 4 & 3 & @\\
	\hline
	3 & 5,9,1,1,1&4&5&\\
	\hline
\end{tabular}
\end{center}
\newpage
\section{Effets de bords}
\subsection{Exercice 4}
\lstinputlisting[language=C, caption=Exercice 4]{exo4.c}
\begin{center}
\begin{tabular}{c |  c  c  c  c  c  c  }
	\textbf{Point d'arrêt} & \textbf{y} & \textbf{x} & \textbf{t} & \textbf{i} & \textbf{z} & \textbf{f}\\
	\hline
	2 & 10 & / & / & 1 && \\
	\hline
	1 & 11 & 1 & 2 & 1 & / & 13\\
	\hline
	3 & 11 & / & / & 1 & 24 & /\\
	\hline
\end{tabular}
\end{center}
\newpage
\subsection{Exercice 5}
\lstinputlisting[language=C, caption=Exercice 5]{exo5.c}
\begin{center}
\begin{tabular}{c |  c  c  c  c  c  }
	\textbf{Point d'arrêt} & \textbf{i} & \textbf{j} & \textbf{x} & \textbf{y} & \textbf{z}\\
	\hline
	3 & 3 & 7 &/ &/&/\\
	\hline
	1 & &&@i&@i&@j\\
	\hline
	2 & 6 & 12 & @i & @i & @j\\
	\hline
	4 & 6 & 12 & / & / & /\\
	\hline
	5 & 3 & 7 & / & / & / \\
	\hline
	1 & 3&10 &@j&@j&@i\\
	\hline
	2 & 20 & 10 & @j&@j&@i\\
	\hline
\end{tabular}
\end{center}
\newpage
\subsection{Exercice 6}
\lstinputlisting[language=C, caption=Exercice 6]{exo6.c} 
\begin{tabular}{c |  c  c  c  c  c  c  }
	\textbf{Point d'arrêt} & \textbf{i} & \textbf{j} & \textbf{x} & \textbf{a} & \textbf{a} & \textbf{f}\\
	\hline
	2&10&40&/&/&/&/\\
	\hline
	1 & 20 & 40 & / & 10 & 40 & 50\\
	\hline
	3 & 20 & 40 & 50 & / & / & /\\
	\hline
\end{tabular}

\section{Spécification}
\subsection{Exercice 7}
	\'Ecrire la spécification d'un programme qui dans un tableacu T de N entiers calcul le nombre n de nombre positifs dans le tableau.

	\begin{itemize}
		\item $N > 0$
		\item \texttt{calculeNbPos(T,N,n)}
		\item $(o \leq n \leq N) \wedge (n = \nu I : o \leq I < N T[I] \geq 0) $
	\end{itemize}
\subsubsection{Exercice 8}
	Soit T un tableau croissant (non strict) de N entier et X un entier.\\
	Spécifier un programme qui calcule la position de la dernière occurrence de T inférieure ou égale à X avec $T[0] \leq X < T[N-1]$

	\begin{itemize}
	%	\item $T[0] \leq X < T[N-1]$, $T[I] \leq T[I+1]$, $N > 0$
		\item $(N > 1) \wedge (T[0] \leq X) \wedge (X < T[N-1]) \wedge (\forall I : 0 \leq I < N - 1 \rightarrow T[I] \leq T[I+1])$
		\item \texttt{searchPosition(T, N, X, p));} 
		\item $(0 < p < N - 1) \wedge (T[p] \leq X) \wedge (T[p+1] > X)$
	\end{itemize}
	\remarque{Dans la suite du cours, nous pourrons utiliser un raccourcie afin de savoir si un tableau est trié par ordre croissant : $(T,N,\leq)$

	Celle-ci pourra être utilisée dans la copie à condition qu'elle soit définie au préalable.}

\subsection{Exercice 8}
	Soit un tableau T non vide de N entiers.  \'Ecrire la spécifications du qui programme qui calculent :
	\begin{itemize}
		\item La première position de la valeur max de T
		\item La dernière position de la valeur max de T
	\end{itemize}
	\subsubsection{Calcule de la première position}
	\begin{itemize}
		\item $N > 0$ 
		\item \texttt{searchFirstPosition(T, N, f);} 
		\item $(\forall I :0 \leq I < f \rightarrow T[I] < T[f]) \wedge (\forall I (f \leq I < N) \rightarrow (T[I] \leq T[f]))$ 
	\end{itemize}
	\subsubsection{Calcule de la dernière position}
	\begin{itemize}
		\item $N > 0$ 
		\item \texttt{searchLastPosition(T, N, l);} 
		\item $(\forall I :0 \leq I < l \rightarrow T[I] \leq T[l]) \wedge 
			(\forall I (l < I < N) \rightarrow (T[I] < T[l]))$ 
	\end{itemize}

\subsection{Exercice 9}
	\'Ecrire la spécification d'un programme qui, dans un tableau T de N entiers tous différents cherche la position d'une valeur X si elle existe ou retourne N si elle n'existe pas.

	\begin{itemize}
		\item $(N\geq0) \wedge (\forall I : 0 \leq I < N \rightarrow (\forall (I, J) : 0 \leq I < N \wedge (0 \leq J < N) \rightarrow T[I] = T[J] \leftrightarrow (I = J)$\footnote{Cela peut aussi s'écrire $N\leq 0) \wedge (\forall I (O \leq I < N) \rightarrow \forall J : J \neq I \wedge 0 \leq J < N \rightarrow T[I] \neq T[J]$}
		\item \texttt{search(T, N, x)}
		\item $(0 \leq p < N \wedge T[p] = X) \vee (p=N) \leftrightarrow \forall I (o\leq I < N) \rightarrow T[I] \neq X))$ 
	\end{itemize}

	\subsection{Exercice 10}
	Spécifier un programme qui, dans un tableau T de N éléments trié par ordre croissant non strict retourne la longueur du plus grand plateau\footnote{Un plateau est quand il y a plusieurs fois le même caractère}.

\begin{itemize}
	\item $(T, N, \leq) \wedge N > 0$
	\item \texttt{longueurPlusGrandPlateau(T,N,l);}
	\item $(1 \leq l \leq N) \wedge (\exists I : 0 \leq I < N-l)\wedge (T[I] = T[I+l-1])$ 
\end{itemize}
	\subsection{Exercice 11}
		Spécifier un programme qui, dans un tableau de N entiers calcule le nombre de doublons : un doublon est une succession de 2 nombres identiques.

		\begin{itemize}
			\item $N > 0$
			\item \texttt{calculeDoublons(T, N, n);}
			\item $n = \nu I : 0 \leq I < N \wedge T[I] = T[I+1]$
		\end{itemize}
		\remarque{
			Dans le cas ou deux doublons ne sont pas forcément côte à côte, le prédicat de sortie deviendrait : 
			$$n = \sum^{N-1}_{I=0} \nu J : I < J < N \wedge T[J] = T[I]$$
		}

	\section{Preuves de programmes}
	\subsection{Séquence}
	\begin{itemize}
		\item \texttt{/* f = i! */}
		\item \texttt{f = f * (i + 1);\\i = i + 1;}
		\item \texttt{/* f = i! */}
	\end{itemize}
	\begin{eqnarray*}
		f = i! &\rightarrow& \texttt{pfp}("f = f\times (i+1); i = i +1;", f= i!)\\
		f=i!&\rightarrow&\texttt{pfp}("f=f\times(i+1);",\texttt{pfp}("i=i+1",f=i!)\\
		%f = i! &\rightarrow& \textt{pfp}("f = f \times (i + 1);" f = (i + 1)!)
	\end{eqnarray*}
	\begin{itemize}
		\item \texttt{/* (x = A) $\wedge$ (y = B) $\wedge$ (z = C)}
		\item \texttt{x = x + y +z;\\z = x - y - z;\\y = x - y -z;\\x = x - y - z;}
		\item \texttt{/* (x = B) $\wedge$ (y = C) $\wedge$ (z = A) */}
	\end{itemize}
	\begin{eqnarray*}
		PE &\rightarrow& \texttt{pfp}("x=x+y+z;z=x-y-z;y=x-y-z;x=x-y-z;", (x=B)\wedge(y=C)\wedge(z=A)\\
		PE &\rightarrow& \texttt{pfp}("x=x+y+z, z-x-y-z;y=x-y-z;", \\&&\texttt{pfp}("x=x-y-z;(x=B)\wedge(y=C)\wedge(z=A)\\
		PE &\rightarrow& \texttt{pfp}("x=x+y+z, z-x-y-z;y=x-y-z;", x-y-z=B)\wedge y=C \wedge z = A\\
		PE &\rightarrow& \texttt{pfp}("x=x+y+z,z=x-y-z", (y=B)\wedge (x-y-z=C)\wedge z = A\\
		PE &\rightarrow& \texttt{pfp}("x=x+y+z", (y=B)\wedge (z=C) \wedge (x-y-z=A)\\
		PE &\rightarrow& (y=B) \wedge (z=C)\wedge(x=A) \textmd{Vrai parceque }p\rightarrow p = \textmd{vrai}
	\end{eqnarray*}
	
	\subsection{Sélection} \label{pfpSequence}
	\subsubsection{Exercice 1}
\begin{lstlisting}[language=C]
/* x = A */
if(x < 0)
	x = -x;

/* x = |A| */
\end{lstlisting}

\begin{eqnarray*}
	/*x = A*/ &\rightarrow& \pfp ("if(x < 0)\{ x = -x \}", x = |A|)\\
	/*x = A*/ &\rightarrow& (((x<0)\rightarrow \pfp ("x=-x;", x = |A| ) \wedge (x >= 0 \rightarrow x = |A|)))\\
	/*x = A*/ &\rightarrow& (((x < 0) \rightarrow (-x = |A|)) \wedge (( x >= 0) \rightarrow (x = |A|)))\\
\end{eqnarray*}
$( A ( A < 0) \rightarrow (-A = |A|) ) \wedge (A >= 0 \rightarrow (A=|A|))$
Définition de la valeur absolue $|.|$.

\subsubsection{Exercice 2}
\begin{lstlisting}[language=C]
/* x = A $\wedge$ y = B */
if (A < B) {
	x = A;
	y = B;
} else {
	x = B;
	y = A;
}
/* x $\leq$ y */ 
\end{lstlisting}
\begin{eqnarray*}
	PE &\rightarrow& \pfp ("\ifp (A < B) \{ x = A; y = B\}\elsep \{ x = B; y =A\}", x \leq y)\\
PE&\rightarrow&((A<B)\rightarrow\pfp("x=A;y=B",x\leq y))\wedge((A\geq B)\rightarrow\pfp("x=B;y=A",x\leq y))\\
	PE &\rightarrow& ((( A < B) \rightarrow (A \leq B)) \wedge (A \geq B ) \rightarrow (B \leq A))
\end{eqnarray*}
Vrai par définition. A et toujours inférieur à B. $(A \geq B) \rightarrow (B \leq A)$ est une Tautologie. 
\subsubsection{Exercice 3}
$$\pfp ("\ifp (x \geq y )\{z=x;\}\elsep\{z=y\}",z=\texttt{max}(x,y))$$
\begin{eqnarray*}
x \geq y  &\rightarrow&	\pfp ("\ifp (x \geq y )\{z=x;\}\elsep\{z=y\}",z=\texttt{max}(x,y))\\
x \geq y &\rightarrow& \pfp ("z = x", z = \texttt{max}(x,y))\\
x < y &\rightarrow& \pfp ("z = y", z = \texttt{max}(x,y))\\
x \geq y &\rightarrow& x = \texttt{max}(x,y)\\
x < y &\rightarrow& y = \texttt{max}(x,y)
\end{eqnarray*}
C'est une tautologie par définition de \texttt{max}.

\subsubsection{Exercice 4}
$$\pfp ("\ifp (x > y) \{ if(x \%2 ==  0) \{ x = x - 2\}\} \elsep \{y=y-1;\}", y-2 < x);$$
\begin{eqnarray*}
	(x > y) &\rightarrow& \pfp("if(x\%2 == 0) { x = x-2; } ", y-2 < x)) \wedge ( ( x \leq y) \rightarrow \pfp ("y = y-1;", y - 2 < x) )\\
	(x > y) &\rightarrow& ( ( (x \%2 = =0) \rightarrow \pfp ("x = x-2", y-2 < x) \wedge \\
	& &(x \% 2 != 0) \rightarrow(y-2 < x)) \wedge ( (x \leq y) \rightarrow (y-z < x))\\ 
	(x > y) &\rightarrow& ( (x\%2 = = 0)\rightarrow ( (y-2) < (x - 2) ) \wedge ( ( x\%2 != 0) \rightarrow (y-2 < x))) \wedge (x \leq y \rightarrow y-z < x) \\
	(x > y) &\rightarrow& (x \leq y \rightarrow y - z < x)
\end{eqnarray*}

\subsubsection{Exercice 5}
\begin{lstlisting}[language=C]
/* $N \geq 0$ */

/* P Tableau de polynôme
 * N Degré du polynôme
 * X Point ou je veux évoluer le polynôme
 * r Résultat du polynôme
 */
calculPolynome(P,N,X,r);
/* $r = \sum^N_{I=0} p[I]X^I $ */
\end{lstlisting}
	\paragraph{INVARIANT}
		$$(o \leq i \leq N) \wedge (r = \sum^i_{k=0} A[k]x^k) \wedge (y = x^i)$$
	\paragraph{Intitialisation}
		\begin{enumerate} 
			\item \begin{lstlisting}[language=C,numbers=none]
i = 0; 
r = 0; 
y = 1;
\end{lstlisting}~
		\item \begin{lstlisting}[language=C,numbers=none]
i = 0; 
r = A[0]; 
y = 1;
\end{lstlisting}
\end{enumerate}

\paragraph{Boucles} 
	\begin{enumerate} 
		\item \begin{lstlisting}[language=C,numbers=none]
while(i < N) { 
	++i; 
	r = r + P[i-1] * y; 
	y = y * X; 
}
		\end{lstlisting}~
		\item \begin{lstlisting}[language=C,numbers=none]
while( i != N) { 
	++i; 
	p = p + a[i-1] * y; 
	y = y * X 
} 
\end{lstlisting}~
\item \begin{lstlisting}[language=C,numbers=none]
while(i == N) { 
	++i;	
	p = p +(A[i] * y * X); 
	y = y * X; 
}
\end{lstlisting}
	\end{enumerate}

