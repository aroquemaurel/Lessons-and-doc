\chapter{Spécification d'un programme}
\minitoc
\remarque{Durant ce chapitre, nous parlerons de programme, cependant cela est valable également pour les sous-programme}
Un programme est spécifié par un triplet : 
\begin{itemize}
	\item Prédicat d'entrée P(E) ou précondition
	\item action (E, S)
	\item Prédicat de sortie P(S) ou postcondition
\end{itemize}

	Les prédicats sont écrits en utilisant le formalisme de la logique des prédicats et de sopérations booléeenes.
	\section{Mots clés à utiliser dans les prédicats}
	Les mots clés pouvant être utilisés: 
	\begin{itemize}
		\item Les quantificateurs logiques : $\forall$(quelque soit), $\exists$(il existe), $\nu$(nombre de)
		\item Les connecteurs logiques : $\wedge$(et), $\vee$(ou), $\rightarrow$(implique), $\leftrightarrow$(equivalence), $\lnot$(not)
	\end{itemize}
	\section{\'Ecriture de la spécification}
	C'es une traduction de l'énoncé et de l'analyse faite dans l'étape 1 de la méthodologie : c'est un \textbf{triplet} logique.
	La démarche pour écrire la spécification est la suivante.
		\begin{itemize}
			\item Identifier les propriétés des données d'entrée et les exprimer sous forme logique
			\item Identifier les propriétés sur les données en sortie et les exprimer sous forme logique. 
		\end{itemize}
	\exemple{
		\'Ecrire un programme qui trie un tableau T de N éléments.\\
		\begin{itemize}
			\item $N > 1$
			\item \texttt{trier (T, N, t);}
			\item $(\forall I : 0 \leq I < N-1 \longrightarrow T[I] \leq T[I+1]) \wedge$\\$
				(\forall I : 0 \leq I < N \longrightarrow $\\$(\nu J : 0 \leq J < N \wedge t[I] = t[J]) = (\nu J : o \leq J < N \wedge t[I] = T[J]))$ 
		\end{itemize}
	}

