\documentclass[12pt,a4paper,openany]{book}
\usepackage{lmodern}
\usepackage{xcolor}
\input{/home/aroquemaurel/cours/includesLaTeX/couleurs.tex}

\usepackage[utf8]{inputenc} \usepackage[T1]{fontenc}
\usepackage[francais]{babel}
\usepackage[top=1.7cm, bottom=1.7cm, left=1.7cm, right=1.7cm]{geometry}
\usepackage{verbatim}
\usepackage[urlbordercolor={1 1 1}, linkbordercolor={1 1 1}, linkcolor=vert1, urlcolor=bleu, colorlinks=true]{hyperref}
\usepackage{tikz} %Vectoriel
\usepackage{listings}
\usepackage{fancyhdr}
\usepackage{multido}
\usepackage{amssymb}
\usepackage{float}
\usepackage[francais]{minitoc}
\usepackage[final]{pdfpages} 

\newcommand{\titre}{Structures de données}

\newcommand{\pole}{}
\newcommand{\sigle}{sd}

\newcommand{\semestre}{4}

\input{/home/aroquemaurel/cours/includesLaTeX/listings.tex}
\documentclass[12pt,a4paper,openany]{book}
\usepackage{lmodern}
\usepackage{xcolor}
\input{/home/aroquemaurel/cours/includesLaTeX/couleurs.tex}

\usepackage[utf8]{inputenc}
\usepackage[T1]{fontenc}
\usepackage[francais]{babel}
\usepackage[top=1.7cm, bottom=1.7cm, left=1.7cm, right=1.7cm]{geometry}
\usepackage{verbatim}
\usepackage[urlbordercolor={1 1 1}, linkbordercolor={1 1 1}, linkcolor=vert1, urlcolor=bleu, colorlinks=true]{hyperref}
\usepackage{tikz} %Vectoriel
\usepackage{listings}
\usepackage{fancyhdr}
\usepackage{multido}
\usepackage{amssymb}
\usepackage{float}
\usepackage[francais]{minitoc}

\newcommand{\titre}{Complexité des algorithmes}

\newcommand{\pole}{}
\newcommand{\sigle}{complexite}

\newcommand{\semestre}{3}

\input{/home/aroquemaurel/cours/includesLaTeX/listings.tex} %prise en charge du langage C 
\input{/home/aroquemaurel/cours/includesLaTeX/entete-l2-cours.tex}




%----------------------------------------------------------------------------------------
%	DEFINITION OF COLORED BOXES
%----------------------------------------------------------------------------------------

\RequirePackage[framemethod=default]{mdframed} % Required for creating the theorem, definition, exercise and corollary boxes

% Theorem box
\newmdenv[skipabove=7pt,
skipbelow=7pt,
backgroundcolor=black!5,
linecolor=ocre,
innerleftmargin=5pt,
innerrightmargin=5pt,
innertopmargin=5pt,
leftmargin=0cm,
rightmargin=0cm,
innerbottommargin=5pt]{tBox}

% Exercise box	  
\newmdenv[skipabove=7pt,
skipbelow=7pt,
rightline=false,
leftline=true,
topline=false,
bottomline=false,
backgroundcolor=ocre!10,
linecolor=ocre,
innerleftmargin=5pt,
innerrightmargin=5pt,
innertopmargin=5pt,
innerbottommargin=5pt,
leftmargin=0cm,
rightmargin=0cm,
linewidth=4pt]{eBox}	

% Definition box
\newmdenv[skipabove=10pt,
skipbelow=10pt,
rightline=false,
leftline=true,
topline=false,
bottomline=false,
linecolor=ocre,
innerleftmargin=5pt,
innerrightmargin=5pt,
innertopmargin=0pt,
leftmargin=0cm,
rightmargin=0cm,
linewidth=4pt,
innerbottommargin=0pt]{dBox}	

% Corollary box
\newmdenv[skipabove=7pt,
skipbelow=7pt,
rightline=false,
leftline=true,
topline=false,
bottomline=false,
linecolor=gray,
backgroundcolor=black!5,
innerleftmargin=5pt,
innerrightmargin=5pt,
innertopmargin=5pt,
leftmargin=0cm,
rightmargin=0cm,
linewidth=4pt,
innerbottommargin=5pt]{cBox}		

% Corollary box
\newmdenv[skipabove=7pt,
skipbelow=7pt,
rightline=true,
leftline=false,
topline=false,
bottomline=true,
linecolor=gray,
backgroundcolor=black!5,
innerleftmargin=5pt,
innerrightmargin=5pt,
innertopmargin=5pt,
leftmargin=0cm,
rightmargin=0cm,
linewidth=1pt,
innerbottommargin=5pt]{rBox}				  
		  

% Creates an environment for each type of theorem and assigns it a theorem text style from the "Theorem Styles" section above and a colored box from above
\newenvironment{theorem}{\begin{tBox}\begin{theoremeT}}{\end{theoremeT}\end{tBox}}
\newenvironment{example}{\begin{exampleT}}{\hfill{\tiny\ensuremath{\blacksquare}}\end{exampleT}}
\newenvironment{definition}{\begin{dBox}\begin{definitionT}}{\end{definitionT}\end{dBox}}
\newenvironment{attention}{\begin{eBox}\small}{\end{eBox}}				  	
\newenvironment{exemple}{\begin{cBox}\small}{\end{cBox}}	

%----------------------------------------------------------------------------------------
%	REMARK ENVIRONMENT
%----------------------------------------------------------------------------------------

\newenvironment{remarque}{\par\vskip10pt\small
\begin{rBox}
\begin{list}{}{
\leftmargin=35pt % Indentation on the left
\rightmargin=25pt}\item\ignorespaces % Indentation on the right
\makebox[-2.5pt]{\begin{tikzpicture}[overlay]
\node[draw=ocre!60,line width=1pt,circle,fill=ocre!25,font=\sffamily\bfseries,inner sep=2pt,outer sep=0pt] at (-15pt,0pt){\textcolor{ocre}{R}};\end{tikzpicture}} % Orange R in a circle
\advance\baselineskip -1pt}
{\end{list}\vskip1mm\end{rBox}\vskip5pt} % Tighter line spacing and white space after remark



\input{/home/aroquemaurel/cours/includesLaTeX/polices.tex}
\input{/home/aroquemaurel/cours/includesLaTeX/affichageChapitre.tex}
\newcommand{\pfp}{\texttt{pfp}}

\newcommand{\ifp}{\texttt{if}}
\newcommand{\moy}{\textrm{moy}}
\newcommand{\prob}{\textrm{prob}}
\newcommand{\elsep}{\texttt{else}}
\newcommand{\perm}{\texttt{perm}}
\newcommand{\random}{\texttt{perm}}

\makeatother
\includeonly{
chapitre1,
chapitre2,
chapitre3,
%chapitre4,
annexes,
}
\begin{document}
	\frontmatter
	\setcounter{tocdepth}{1}
	\setcounter{secnumdepth}{3}
	\setcounter{minitocdepth}{1}
	\dominitoc
	\maketitle
	\chapter*{Modalité de Contrôle de connaissance}
	\begin{enumerate}
		\item Devoir Maison à rendre avant 10h le 21 décembre. (10\%)
		\item Contrôle de continue sous forme de QCM lundi 3 décembre de 8h15 à 9h45 (20\%)
		\item Contrôle terminal vendredi 21 décembre de 10h à 12h (70\%)
	\end{enumerate}
	\tableofcontents
	\mainmatter
	\chapter{Projet et Équipe de management}
	\section{Présentation des créateurs}
		\begin{itemize}
			\item \Bonte{}, Ingénieur en \gHabitat{}
			\item \Ben{}, Ingénieur en \gHabitat{}
			\item \Drm{}, Développeur Web
			\item \Soum{}, Développeur \texttt{\glo{C++}{C++}{4e langage de programmation
				le plus utilisé au monde. Il est compilé, permettant de produire un programme
				s'éxecutant le plus rapidement possible.}/\glo{Qt}{Qt}{Bibliothèque programmée
				en C++ permettant de créer des interfaces graphiques.}}
			\item \Clem{}, Développeur \texttt{C++/Qt}, Administrateur Système et Réseau
		\end{itemize}
		
		\subsection{Formations}
			\Bonte{} et \Ben{} ont obtenu un Master Pro \gHabitat{} à l'INSA de Toulouse et
			font actuellement une thèse.

			\Drm{}, \Soum{} et \Clem{} sont en deuxième année de DUT Informatique 
			à l'IUT Paul Sabatier de Toulouse. Ces derniers sont
			également autodidactes et ont pu acquérir de nombreuses conpétences lors de projets personnels.
	
		\subsection{Expériences professionnelles}
			\Bonte{} a passé un an dans un bureau d'étude à réaliser des bilans thermique. C'est pendant cette periode 
			que lui est venue l'idée de notre projet après avoir constaté le manque dramatique d'affordance des solutions disponibles.
			Le reste de l'équipe s'est constitué autour de ce constat global.
	
	\section{Atouts}
		%atouts qui font qu'on a des facilités a créer l'entreprise

		%facultés particulières
		Ayant déja eu une expérience professionnelle, \bonte{} et \ben{} connaissent les difficultés et les besoins des TPE et PME du \gHabitat{}.
		%contacts
		Ainsi, nous disposons déja de contacts dans ce secteur,
		nottamment au sein d'établissements universitaires et de bureaux d'étude.
		Ce premier carnet d'adresse peut être facilement étoffé car notre clientelle est,
		par sa taille humaine, facilement abordable et particulièrement à l'écoute pour trouver des solutions gratuites et efficaces.
		%connaissances pratiques théo
		Grâce aux cours généraux enseignés à l'université et en DUT, 
		tout les associés ont des connaissances en Comptabilité, Gestion et Droit des Entreprises,
		ce qui permet de faciliter les échanges avec les professionnels (comptables, avocats...) que nous ne manquerons pas de contacter.
		%part à des orga assoc
		\bonte, actuellement thésard, interviens dans des promotions de \gHabitat{} et a, 
		de ce fait, la possibilité de présenter des produits de \K{} aux étudiants. \\
		\clem{} est quant à lui impliqué dans diveres associations et a de ce fait rencontré plusieurs personnes ayant fondé ou travaillant dans une \glo{SCOP}{SCOP}{Société soumise à l’impératif de rentabilité comme toute entreprise.
Ses salariés-coopérateurs y sont en effet associés (ou « co-entrepreneurs ») majoritaires et détiennent au moins 51\% du capital et 65\% des droits de vote. Par ailleurs, quelle que soit la quantité du capital détenu, chaque coopérateur ne dispose que d'une seule voix lors de l'assemblée générale de l'entreprise.
}\footnotesouvenir{scop}{\textbf{S}ociété \textbf{CO}opérative et \textbf{P}articipative}. Cela nous permet d'avoir des réponses rapides et un premier contact avec le réseau des SCOPs\footnoterappel{scop}, qui permet aux jeunes entreprises de bénéficier d'avantages divers afin de se développer.
		%aide famille...


	\section{L'idée}
		% societe de dev de logiciel et de prestation de services informatiques dans le génie de l'habitat
		\K{} est une Société de Développement de Logiciels et de Prestation de Services Informatiques dans le secteur du \gHabitat{}.

		% Comment est venue l'idée
		% Secteur du génie de l'habitat
		L'idée de ce projet est née à travers diverses expériences dans le domaine du génie climatique.
		À l'heure actuelle, les professionnels n'ont à leurs disposition que peu d'outils : 
		\begin{itemize}
			\item Tableurs Excel réalisés en interne, aux résultats approximatifs dans un contexte de maîtrise de l'énergie
				et dans lesquels la saisie des données est peu aisée.
			\item Logiciels réglementaires coûtant plusieurs milliers d'euros, à l'ergonomie souvent douteuse et peu adaptés a de petites structures telles que les PME\footnote{\textbf{P}etites et \textbf{M}oyennes \textbf{E}ntreprises} et les TPE\footnote{\textbf{T}rès \textbf{P}etite \textbf{E}ntreprise}
		\end{itemize}
		
		% C'est une création
		Nous souhaitons donc créer une société a l'écoute des besoins de ces petites structures, afin de leurs permettre d'économiser leurs ressources lors de leurs projets grâce à des outils adaptés à leurs échelle.
		% Sur Toulouse
		Notre équipe s'étant formée à Toulouse, et le secteur du \gHabitat{} y étant largement développé\footnote{Voir Chapitre \ref{marché}. Marché}, c'est donc dans cette ville que nous implanterons notre société.

	\section{Objectifs du projet}
		%quel objectif ? expansion, retabilité ? autonomie ?
		%prépondérent
		%d'autres ?
		La société \K{} repose sur des valeurs et des principes communautaires
		où le seul objectif est de rendre accessible au plus grand nombre
		l'accès a des outils ergonomiques, intuitifs et performants
		afin de fournir les résultats les plus précis possible 
		dans une optique de maîtrise de l'énergie et de développement durable.

	\section{Taille de l'entreprise}
	%dimension (effectif, CA, capitaux 10890, parts de marché
	%taille max ? min ?
	L'entreprise \K{} souhaite rester une entreprise à taille Humaine, ainsi elle sera composé d'un maximum de 20 personnes afin que tout le monde soit impliqué dans l'entreprise.

	Le capital de départ sera de 10090\euro{} et pourra évoluer selon les activités de l'entreprise. 

	Notre but serai d'avoir un chiffre d'affaire de $104\;400$\euro{} au bout de deux ans d'activités et atteindre les 2~220~000\euro{} dans les 4 ans après la création de notre entreprise. 

	Dans un premier temps, \K{} favorisera l'évolution au sein de Midi-Pyrénnés, si celle-ci fonctionne
	convenablement, elle s'étendra au reste de la France dans un second temps.

	\chapter{\'Evolution des réseaux}
		De la même manière que la téléphonie et le télégraphe, nous sommes passé d'une phase expérimentale à une phase d'utilisation. Ainsi l'Informatique à beaucoup évolué. Cette évolution à été progressive, il y a eu plusieurs étapes qui ont marqués les réseaux de communication.
		\paragraph{Coûts des équipements Informatiques / Coûts de la Communication} À l'origine seul les grands comptes étaient capable d'avoir des équipements informatiques. Ainsi les SSI\footnote{Société de Service en Informatiques} sont nées.
		\paragraph{Système de Télétraitement} Ces systèmes ont été destiné aux entreprise, afin qu'a distance elles puissent utiliser la puissance d'un calculateur qui était géographiquement loin. Une première structure de réseau Informatique fut créée.
		\remarque{Nous sommes en train de revenir à cette solution créée 40 ans auparavant: Le cloud computing}

		\section{Les équipements créés}
		Afin de construire ces structures de réseaux de communication nous avons mis en place des équipements :
		\begin{description}
			\item[Processeur Frontal de Communication\footnote{FEP: Front End Processor}]
			\item[Multiplexeurs et concentrateurs] Équipement de partage du support de communication, permettent d'avoir des nœuds de communication.
			\item[Liaisons Spécialisées] Nous avions besoin d'un réseau spécialisé afin d'interconnecter les appareils, pour les connections point à point.
			\item[Modem] Pour les trafics de grande ligne, il fut choisir d'utiliser un réseau déjà existant, le téléphone.
		Cependant, le signal à transmettre doit être adapté au support de transmission, on va donc utiliser un adaptateur qui permettra de faire 
		passer le signal sur le réseau téléphonique : le modem.
			\item[Commutateurs] Pour avoir une connexion la plus rapide possible, nous avions besoin d'un algorithme de routage afin de passer par un chemin en fonction du trafic présent sur la ligne: le routeur.
			\item[Protocole de communication] Permet de faire dialoguer deux machines entre elles, elle doivent utiliser le même protocole afin de se comprendre syntaxiquement et sémantiquement.
		\end{description}
		
		\section{Démocratisation de l'Informatique}
		\begin{description}
			\item[1970] La genèse des protocoles de communication date des années 1970. En réseau, rien n'a été inventé de nouveau, cela à surtout été des progrès technologiques : rapidité, miniaturisation, coûts et donc démocratisation. Les premiers mini-calculateurs.
			\item[1980] Début de l'informatique personnelle et mise en \oe{}uvre des réseaux locaux.
			\item[1990] Applications de l'Internet, premiers mobiles et satellites. 
		\end{description}


	\chapter{Complexité d'algorithmes définis par récurrence}
	\section{Exemple introductif : Tri fusion}
\'Etant donné un tableau T, on note T[i:j] le sous tableau de T qui va de la case i à la case j. L'algorithme de tri fusion utilise une procédure
\texttt{fusion(T,i,j,k)}. On suppose que les deux sous tableaux T[i:j] et T[j+1:k] sont déjà triés. En temps $\Theta(n)$, où $n=k-i+1$, la procédure
fusion produit le sous tableau T[i:k] trié à partir de la fusion de ces deux tableaux.

\lstinputlisting[language=algo, caption=Algorithme du tri fusion]{triFusion.algo}
	\section{Méthode naïve d'analyse de complexité}
	Soit un temps maximal d'exécution de tri fusion sur un tableau de longueur $n$.

	D'après l'algorithme, on a $$U_n = U_{\frac{n}{2}} + U_{\frac{n}{2}} \Theta(n)$$
	et $u_1 = 0$

	Pour simplifier la récurrence on suppose que $n$ est pair, et donc $U_n = 2U_{\frac{n}{2}} + \Theta(n)$

	La méthode naïve consiste à deviner la solution, ici on devine $U_n \leq c n \log_2 n$. On suppose $U_{\frac{n}{2}} \leq C \frac{n}{2} \log_2
	\frac{n}{2}$ et on essaye d'en déduire $U_n \leq c n \log_2 n$
	\begin{eqnarray*}
		U_n = 2U_{\frac{n}{2}} + cn &\leq& 2c \frac{n}{2} \log_2 \frac{n}{2} + cn \\&&= cn(\log_2 n -1) + cn = cn\log_2 n
	\end{eqnarray*}

	Puisque $u_1=0 \leq c 1 \log_2 1$, on en déduit $\forall n, i_n \leq cn \log_2 n$

	\subsection{Résumé de la méthode naïve}
	Pour une équation récurrente $u_n = f_n(U_{n-1}, \cdots, u_1)$ où f est une fonction monotone croissante
	\begin{enumerate}
		\item On devine une fonction $g$
		\item On suppose que $\forall n < 1$ on a $U_n \leq g(m)$
		\item On montre $U_n = f_n(U_{n-1},\cdots,u_1 \leq f_n(g(n-1), \cdots, g(1)) \leq g(n)$
		\item On conclut par récurrence que $\forall n$ on a $U_n \leq g(n)$
	\end{enumerate}
	\subsection{Exemples d'application}
	On commence par une \textbf{mauvaise} utilisation. Soit l'équation $U_n = 2 U_{\frac{n}{2}}$. L'intuition $U_n \leq kn$ n'est pas correcte.

	En effet, en remplaçant on obtient : 
	\begin{eqnarray*}
		n_n &=& 2U_{\frac{n}{2}}+1\\
		&=& 2k \frac{n}{2} + 1\\
		&=& kn + 1
	\end{eqnarray*}
	
	La bonne intuition est $u_n \leq kn - b$. En remplaçant on obtient : 
	\begin{eqnarray*}
		u_n = 2U_{\frac{n}{2}} + 1\\
		&\leq& 2(k\frac{n}{2} - b) + 1= kn - 2b + 1\\
		&\leq& kn -b\textrm{ Si } b \geq 1
	\end{eqnarray*}

	\subsection{Réduction à des formes simples}
	Lors de l'analyse d'algorithmes récursifs, on rencontre souvent des équations récurrentes de la forme $$u_n = aU_{\frac{n}{2}}+b,$$ où a et b sont des
	constantes. Par exemple le tri fusion.

	Pour convertir ce type de récurrence en une forme affine $u'_n = a'u'_{n-1}+b'$, on pose
	$$v_k = U_{2^k}$$
	Autrement dit, on étudiera la suite $\{u_n\}_{n \geq 0}$ uniquement sur les puissances de 2.
	\\
	Par exemple, pour le tri fusion, en remplaçant $n$ par $2^k$, 
	\begin{eqnarray*}
		U_{2^k} &=& 2U_{\frac{2^k}{2}} + C2^k\\
		\textrm{donc }V_k &=& 2v_{k-1} + c2^k
	\end{eqnarray*}

	\section{Équation récurrentes linéaires}
	\paragraph{Définition} Une équation récurrente linéaire à coefficients constants d'ordre $k$ est une équation de la forme 
	\begin{displaymath}
		\left\{ \begin{array}{llll}
			u_1 &=& C_i (O \leq i \leq k-1) & \textsc{Conditions initiales (CI)}\\
			u_n &=&  \sum^k_i=1 a_i u_{n-i} + g(n) & \textsc{Equation générale}
		\end{array} \right.
	\end{displaymath}

	Une équation est \textbf{homogène} si $\forall n g(n) = 0$. La solution générale est une suite satisfaisant uniquement l'équation générale. Une
	solution particulière est une solution générale satisfaisant aussi des conditions initiales.

	\subsection{Équations récurrentes linéaires homogènes d'ordre 1}
	\paragraph{Proposition} La solution particulière de l'équation : 
	\begin{displaymath}
		\left\{ \begin{array}{lll}
			u_0 &=& c\\
			u_n &=&  a u_{n-1}
		\end{array} \right.
	\end{displaymath}
	est $u_n=C a^n$ (c'est une suite géométrique)

	\subsection{Équations récurrentes linéaires non-homogènes d'ordre 1}
	On ne sait traiter facilement que les équations dans lesquelles le second membre $g(n)$ est un polynôme ou une exponentielle. Pour cela, on
	<<dérive>> l'équation pour faire baisser le degré du polynôme jusqu'à arriver à 0.

	\exemple{\textbf{Le tri fusion}\\
	On a une équation qui n'est pas homogène: $$V_n = 2V_{n-1} + C 2^n$$
	Donc, au rang $n+1$, on a aussi $$V_{n+1} = 2 V_n + C \times 2 ^{n+1}$$

	Pour éliminer la partie non-homogène, on enlève 2 fois la première équation à la seconde.
	\begin{eqnarray*}
		V_{n+1} -2V_n &=& 2V_n - 4 V_{n-1}\\
		V_{n+1} &=& 4V_n - 4V_{n-1}
	\end{eqnarray*}
	}

	\subsection{Recherche d'une solution générale pour les équations récurrentes linéaires homogènes d'ordre 2}
	Une équation récurrente homogène d'ordre 2 est de la forme 
	\begin{displaymath}
		\left\{ \begin{array}{lll}
			u_0 &=&  C_0\\
			u_1 &=&  C_1\\
			u_n &=& a_1 u_{n-1} + a_2 u_{n-2}
		\end{array} \right.
	\end{displaymath}
	On peut obtenir ce type d'équation indirectement lorsque l'on a réduit une équation d'ordre 1 à une équation homogène d'ordre 2.
	\exemple{ L'équation récurrente linéaire homogène d'ordre 2 de Fibonacci
	\begin{displaymath}
		\left\{ \begin{array}{lll}
			U_0 &=&  1\\
			U_1 &=&  1\\
			U_n &=&  U_{n-1} + U_{n-2}
		\end{array} \right.
	\end{displaymath}
	On résout ces équations d'ordre 2 comme des équations d'ordre 1 : On cherche une solution générale de la forme $\lambda r^n$. Une telle solution
	vérifie, pour le cas de la suite de Fibonacci : $\forall n \geq 2$, $\lambda r^n = \lambda r^{n-1} + \lambda r^{n-2}$

	Soit, en divisant par $\lambda r^{n-2}$ $$r^2 = r + 1$$

	Autrement dit, $r$ est une racine du polynôme $P(x) = x^3 -x - 1$.
	}
	\paragraph{Définition}
	Le polynôme caractéristique d'une équation récurrente homogène d'ordre $k$
	$$V_{n+k} + a_1 V_{n+k+1} + \cdots + a_kV_n = 0$$ est le polynôme $P(x) = x^k + a_1x^{k-1}+\ldots+a_{k-1}+a_k$

	\paragraph{Théorème} Si $r$ est une racine du polynôme caractéristique d'une équation récurrente linéaire homogène, alors pour toute constante
	$\lambda$, toute suite de la forme $\{\lambda r^n\}_{n \geq 0}$ est une solution générale de cette équation.

	Dans le cas de la suite de Fibonacci, on calcule le discriminant $\Delta=5$ et on trouve les deux racines $r_1 = \frac{1-\sqrt{5}}{2}$ et 
	$r_2=\frac{1+\sqrt{5}}{2}$

	\paragraph{Cas des racines doubles} Si le discriminant $\Delta = 0$, alors le polynôme n'a qu'une seule racine (de multiplicité 2). En remarquant
	que $r$ racine double de P(x) implique que $r$ est aussi une racine de $P'(x)$ on peut démontrer que $\{n\lambda r^n\}_{n \geq 0}$ est aussi une
	solution de l'équation récurrente.

	\paragraph{Théorème} Les solutions générales d'une équation récurrente linéaire homogène d'ordre 2 dont le polynôme caractéristique de deux racines
	$r_1$ et $r_2$ sont :
	\begin{itemize}
		\item Si $r_1 \neq r_2$ : $\{ \lambda_1 r_1^n + \lambda_2 r_2^n\}_{n \geq 0}$ pour toute constantes $\lambda_1$, $\lambda_2$
		\item Si $r_1 = r_2$ :  $\{(\lambda_1 + \lambda_2 \times n)r_1^n\}_{n \geq 0}$ pour toute constantes $\lambda_1$, $\lambda_2$
	\end{itemize}

	\paragraph{Preuve dans le cas d'une racine double}
	Soit l'équation $u_{n+2} + a U_{n+1} + bu_n = 0$ et soit $r$ une racine double du polynôme caractéristique.\\  
	$P(x) = x^2 + ax + b$, donc $r$ est aussi une racine de $P'(x)= 2x+a$. La suite $\{nr^n\}_{n\geq 0}$ est une solution de l'équation car
	\begin{eqnarray*}
	(n+2)r^{n+2} + a(n+1)r^{n+1} + bnr^n &=& n(r^{n+2} + ar^{n+1} + br^n) + 2r^{n+2} + ar^{n+1}\\
	&=& r^n[n(r^2 + ar + b) + r(2r+a)]\\
	&=& 0
	\end{eqnarray*}
	\subsection{Recherche de solutions particulières pour les équations récurrentes linéaires homogènes d'ordre 2}
	Dans le cas de la suite de Fibonacci, on cherche une solution particulières satisfaisant les conditions initiales et qui est de la forme
	$\lambda_1r_1^n + \lambda_2r_2^n$ où $r_1 = \frac{1-\sqrt{5}}{2}$ et $r_2 = \frac{1+\sqrt{5}}{2}$

	Donc on cherche $\lambda_1$, $\lambda_2$ tels que
	\begin{eqnarray*}
		u_0 &=&  1 = \lambda_1 r^0_1 + \lambda_2r_2^0 = \lambda_1 + \lambda_2\\
		u_1&=&  1 = \lambda_1 r^1_1 + \lambda r^1_2 = \frac{\lambda_1 + \lambda_2}{2} + \frac{\lambda_2 - \lambda _ 1}{2} \times \sqrt{5}\\
	\end{eqnarray*}
	\begin{displaymath}
		\Rightarrow 
		\left\{ \begin{array}{lll}
			1 &=&  \lambda_1 + \lambda_2\\
			\frac{1}{2} &=& \frac{\lambda_2 - \lambda_1}{2}\sqrt{5}
		\end{array} \right.
		\Rightarrow 
		\left\{ \begin{array}{lll}
			1 &=&  \lambda_1 + \lambda_2\\
			\frac{1}{\sqrt{5}} &=& \lambda_2 - \lambda_1
		\end{array} \right.
		\Rightarrow 
		\left\{ \begin{array}{lll}
			\lambda_2 &=&  \frac{1+\frac{1}{\sqrt{5}}}{2}\\
			\lambda_1 &=&  \frac{1-\frac{1}{\sqrt{5}}}{2}
		\end{array} \right.
	\end{displaymath}
	Au final, on trouve la solution particulière : 
	$$U_n = \frac{\sqrt{5}-1}{2\sqrt{5}}(\frac{1-\sqrt{5}}{2}) + \frac{\sqrt{5}+1}{2\sqrt{5}}(\frac{1+\sqrt{5}}{2})^n$$
	
	\subsubsection{Résumé de la méthode pour les équations homogènes d'ordre 2}
	Pour résoudre l'équation $u_n = aU_{n-1} + bu_{n-2}$
	\begin{enumerate}
		\item On calcule le polynôme caractéristique $P(x) = x^2 - ax -b$
		\item On calcul les racines (éventuellement complexes)
			$r_1$ et $r_2$ de $P$
		\item On cherche les coefficients $\lambda_1$ $\lambda_2$ tels que $\lambda_1r_1^n + \lambda_2r_2^n$ satisfaisant les CI
	\end{enumerate}

	\subsection{Équations récurrentes d'ordre $k$}
	Pour les équations récurrentes homogènes d'ordre k, les considérations sur le polynôme caractéristique et ses racines restent valables. La difficulté
	est calculatoire car il faut trouver les racines d'un polynôme de degré k. Mais lorsque l'équation a été obtenu en éliminant  la partie
	non-homogène, les coefficients utilisés sont des solutions. 
	\exemple{Pour l'algorithme de Strasser, on a obtenu l'équation en faisant
	$$E_n-GE_{n-1}$$ où $E_n$ désigne l'équation de rang $n$

	$\Rightarrow$ 4 est une racine du polynôme caractéristique.
	}

	En cas de racine d'ordre $m$, on peut montrer par récurrence que $\{n^j\alpha^n\}_{n\geq 0}$ est une solution de l'équation récurrente homogène pour
	tout $j=0,\cdots,m-1$. Ceci nous permet d'avoir $k$ variables dans le système d'équation linéaires dérivées des CI.

	Le théorème suivant généralise le théorème 2 au cas de récurrences homogènes d'ordre $k > 2$ et prend en compte directement le second membre.
	\paragraph{Théorème 3}
	Supposons que le polynôme caractéristique de la récurrence homogène $u_n = au_{n-1} + \cdots + a_ku_{n-k}$ admet $p$ racines $ri(i=1,\cdots,p)$ de
	multiplicité $mi(i=1,\cdots,p)$. Alors la solution de la récurrence 
	$$u_n = a_1u_{n-1}+\cdots+a_ku_{n-k} + \sum^t_{i=1} b_i^n P_i(n)$$
	où $p_i$ est un polynôme de degré $d_i$) est donnée par $$\sum^t_{i=1}b_i^n Q_i(n)\footnote{Partie de la solution qui prend en compte le second membre} + 
	\sum_{i \in \{1,\ldots,p\}}r_i^n R_i(n)\footnote{Solution pour la récurrence homogène}$$
	tel que $r_i \not\in \{b_1,\cdots,b_t\}$

	Où 
	\begin{displaymath}
		\textrm{deg}(Q_i) = \left\{ \begin{array}{lll}
			d_i & si & b_i \not\in \{r_i,\cdots,r_p \}\\
			d_i+m_j & si& b_i = r_j
		\end{array} \right.
	\end{displaymath}

	Et $\textrm{deg}(R_i) = m_i-1$

	On obtient les polynômes $Q_i$ et $R_i$ à partir des CI et par identification des coefficients des termes $b_i^n n^j$ dans la récurrence.

	\remarque{Dans le théorème 2, il n'y avait de second membre (t=0) et les polynômes $R_i$ étaient de la forme $\lambda_1$ ou $\lambda_1 = \lambda_2 n$
	}
	\begin{displaymath}
		\left. \begin{array}{lll}
			u_n &=&  u_{n-1} + 1n^3\\
			u_{n-1} &=&  u_{n-2} + 1
		\end{array}
		\right\}
		u_n - u_{n-1} = u_{n-1} - u_{n-2}
	\end{displaymath}
	\exemple{
	$$T(n) = 2T(\frac{n}{2}) + n ; T(1) = 1$$
	Après changement de variable $n=2^k$, $u_k = T(n)$, nous avons $u_k = 2u_{k-1}+2^k$.

	Ici le second membre $$\sum^t_{i=1} b_i^k Pi(k) = 2^k$$

	Donc $T=1$, $P_i(k)=1$, $b_i=2$

	Le polynôme caractéristique $P(x)=x-2$. La seule racine est $r_i = 2$.
	Donc la solution particulière  est de la forme $2^k(q_0 + q_i^k)$ car $\textrm{deg}(Q_i) = \textrm{deg}{P_i}  +\textrm{multiplicité}  = 0 + 1$, et
	cette solution satisfait la CI et la récurrence $1 = T(1) = u_0$
	$$2^k(q_0+q_1k) = 2 \times 2^{k-1}(q_0+q_i(k-1))$$	
			D'où $q_0 = 1$, $q_1 = 1$ donc $u_n = 2^k(1+k)$ et $T(n) = u_k = n(1+\log_2n)$
	}

	\subsection{Théorème pour les récurrences par divisions}
	Le théorème suivant nous donne directement l'ordre de grandeur de la solution en fonction des coefficients de l'équation récurrente.

	\paragraph{Théorème 4} Soient $a \geq 1$, $b > 1$ deux constantes, $f(n)$ une fonction, et $\{t(n\}_{n\geq 0}$ une suite vérifiant l'équation 
	$T(n) = aT(\frac{n}{b}) + f(n)$

	On a pour $\epsilon > 0$
\begin{itemize}
	\item Si $f(n) = 0(n^{\log_b a - \epsilon}$ alors $T(n) = \Theta(n^{\log_b a})$
	\item Si $f(n) = \Theta(n^{\log_b a})$ alors $T(n) = \Theta(n^{\log_b a . \log n})$
	\item Si $f(n) = \Omega(n^{\log_b a + \epsilon}$ et $af(\frac{n}{b}) \leq cf(n)$ pour une constante $c >1$, alors $T(n) = \Theta(f(n))$
\end{itemize}

	
	\chapter{Spécification d'un programme}
	\remarque{Durant ce chapitre, nous parlerons de programme, cependant cela est valable également pour les sous-programme}
	Un programme est spécifié par un triplet : 
	\begin{itemize}
		\item Prédicat d'entrée P(E) ou précondition
		\item action (E, S)
		\item Prédicat de sortie P(S) ou postcondition
	\end{itemize}

		Les prédicats sont écrits en utilisant le formalisme de la logique des prédicats et de sopérations booléeenes.
		\section{Mots clés à utiliser dans les prédicats}
		Les mots clés pouvant être utilisés: 
		\begin{itemize}
			\item Les quantificateurs logiques : $\forall$(quelque soit), $\exists$(il existe), $\nu$(nombre de)
			\item Les connecteurs logiques : $\wedge$(et), $\vee$(ou), $\rightarrow$(implique), $\leftrightarrow$(equivalence), $\lnot$(not)
		\end{itemize}
		\section{\'Ecriture de la spécification}
		C'es une traduction de l'énoncé et de l'analyse faite dans l'étape 1 de la méthodologie : c'est un \textbf{triplet} logique.
		La démarche pour écrire la spécification est la suivante.
			\begin{itemize}
				\item Identifier les propriétés des données d'entrée et les exprimer sous forme logique
				\item Identifier les propriétés sur les données en sortie et les exprimer sous forme logique. 
			\end{itemize}
		\exemple{
			\'Ecrire un programme qui trie un tableau T de N éléments.\\
			\begin{itemize}
				\item $N > 1$
				\item \texttt{trier (T, N, t);}
				\item $(\forall I : 0 \leq I < N-1 \longrightarrow T[I] \leq T[I+1]) \wedge$\\$
					(\forall I : 0 \leq I < N \longrightarrow $\\$(\nu J : 0 \leq J < N \wedge t[I] = t[J]) = (\nu J : o \leq J < N \wedge t[I] = T[J]))$ 
			\end{itemize}
		}
	

	\appendix
	\chapter{Exercices}
\section{TD 1}
\subsection{Lesquelles des affirmations suivantes sont vraies ?}
\begin{enumerate}
	\item $n^2.5 = \Theta(n^3)$ : Faux 
	\item $n^2.5 = O(n^3)$ : Vrai 
	\item $n^2.5 = \Omega(n^3)$ : Faux
	\item $log_2(2n) = \Theta(\log n)$ : Vrai
	\item Vrai
	\item Faux
\end{enumerate}
\subsection{Une seule des afirmations suivantes est vraie. Laquelle ?}
Réponse D
\subsection{Une seule des affirmations suivantes est vraie. Laquelle ?}
Réponse C
$n + n\log_2 n \leq 2 n\log_2 n= \Theta(n \log n)$
\remarque{On ne s'occupe pas des facteurs constants}
\subsection{}
Réponse D
\subsection{Laquelle des affirmations suivantes sont vraies}
\begin{enumerate}
	\item $\max{(f(n),g(n))} = \Theta(f(n)+g(n))$ Vrai : $\max (f(n), g(n)) \leq f(n) + n(n) \leq 2 \max(f(n),g(n))$
	\item Vrai : $\frac{1}{c}f(n) \leq g(n)$ et $g(n) \leq \frac{1}{2}f(n)$
	\item Vrai : $\forall n \geq n_0 : f(n) \leq c g(n)$
	\item Faux
	\item Vrai
	\item Faux \\
		$$g(n) = 2n, f(n) = n\\ g(n) = O(f(n))\\ 2^{g(n)} = 2^{2n} = (2^n)^2$$
\end{enumerate}
\subsection{Lesquelles des affirmations suivantes sont vraies ?}
Réponse D. 
\begin{eqnarray*}
	f(n) \leq c_1 g(n)&,& g(n) \leq c_2 f(n)\\
	\frac{1}{c_2} \leq \frac{f(n)}{g(n)} \leq c_1.1 &\Rightarrow& \frac{f(n)}{g(n)} = \Theta(1)
\end{eqnarray*}
\subsection{Simplifiez les expressions suivantes}
\begin{enumerate}
	\item $O(4n^2+3n^2+7\log_2(n^n)) = O(n^3)$ 
	\item $\Theta(n\log_2 n + 17n + 2n^3 = \Theta (n^2)$ 
	\item $\Omega(4n^2 + 3n^3) = \Omega(n^3)$ 
	\item $O(2^{n\log_3}n + 3\log_2 n!) = O(n^2)$
	\item $O(2\log_3 n + 3\log_2 n + 6) = O(\log n)$
\end{enumerate}
\subsection{Classez les fonctions suivantes dans l'ordre croissant d'ordre de grandeur}
\begin{enumerate}
\item $4n \log_2 n + 4n$
\item $2n \log_2 n + 4n$
\item $n^2 \log_e n$
\end{enumerate}

\subsection{}
\begin{eqnarray*}
	\Theta(\frac{1}{1-p})\\
	p = 1 - (\frac{1}{6})^{n-1}\\
	\Theta(\frac{1}{1-(1-\frac{1}{6^{n-1}})}) = \Theta(\frac{1}{\frac{1}{6^{n-1}}}) = \Theta(6^{n-1}) = \Theta(6^n)
\end{eqnarray*}
Donc réponse D.
\subsection{}
\begin{enumerate}
	\item[a]$\Theta(1)$
	\item[b] $\Theta(1)$
	\item[c] $\Theta(\log n)$
	\item[d] $\Theta(n\log n)$
	\item[e] $\Theta(n^3)$
	\item[f] $\Theta(n^4)$
\end{enumerate}

\subsection{}
\subsection{}

\end{document}






%----------------------------------------------------------------------------------------
%	DEFINITION OF COLORED BOXES
%----------------------------------------------------------------------------------------

\RequirePackage[framemethod=default]{mdframed} % Required for creating the theorem, definition, exercise and corollary boxes

% Theorem box
\newmdenv[skipabove=7pt,
skipbelow=7pt,
backgroundcolor=black!5,
linecolor=ocre,
innerleftmargin=5pt,
innerrightmargin=5pt,
innertopmargin=5pt,
leftmargin=0cm,
rightmargin=0cm,
innerbottommargin=5pt]{tBox}

% Exercise box	  
\newmdenv[skipabove=7pt,
skipbelow=7pt,
rightline=false,
leftline=true,
topline=false,
bottomline=false,
backgroundcolor=ocre!10,
linecolor=ocre,
innerleftmargin=5pt,
innerrightmargin=5pt,
innertopmargin=5pt,
innerbottommargin=5pt,
leftmargin=0cm,
rightmargin=0cm,
linewidth=4pt]{eBox}	

% Definition box
\newmdenv[skipabove=10pt,
skipbelow=10pt,
rightline=false,
leftline=true,
topline=false,
bottomline=false,
linecolor=ocre,
innerleftmargin=5pt,
innerrightmargin=5pt,
innertopmargin=0pt,
leftmargin=0cm,
rightmargin=0cm,
linewidth=4pt,
innerbottommargin=0pt]{dBox}	

% Corollary box
\newmdenv[skipabove=7pt,
skipbelow=7pt,
rightline=false,
leftline=true,
topline=false,
bottomline=false,
linecolor=gray,
backgroundcolor=black!5,
innerleftmargin=5pt,
innerrightmargin=5pt,
innertopmargin=5pt,
leftmargin=0cm,
rightmargin=0cm,
linewidth=4pt,
innerbottommargin=5pt]{cBox}		

% Corollary box
\newmdenv[skipabove=7pt,
skipbelow=7pt,
rightline=true,
leftline=false,
topline=false,
bottomline=true,
linecolor=gray,
backgroundcolor=black!5,
innerleftmargin=5pt,
innerrightmargin=5pt,
innertopmargin=5pt,
leftmargin=0cm,
rightmargin=0cm,
linewidth=1pt,
innerbottommargin=5pt]{rBox}				  
		  

% Creates an environment for each type of theorem and assigns it a theorem text style from the "Theorem Styles" section above and a colored box from above
\newenvironment{theorem}{\begin{tBox}\begin{theoremeT}}{\end{theoremeT}\end{tBox}}
\newenvironment{example}{\begin{exampleT}}{\hfill{\tiny\ensuremath{\blacksquare}}\end{exampleT}}
\newenvironment{definition}{\begin{dBox}\begin{definitionT}}{\end{definitionT}\end{dBox}}
\newenvironment{attention}{\begin{eBox}\small}{\end{eBox}}				  	
\newenvironment{exemple}{\begin{cBox}\small}{\end{cBox}}	

%----------------------------------------------------------------------------------------
%	REMARK ENVIRONMENT
%----------------------------------------------------------------------------------------

\newenvironment{remarque}{\par\vskip10pt\small
\begin{rBox}
\begin{list}{}{
\leftmargin=35pt % Indentation on the left
\rightmargin=25pt}\item\ignorespaces % Indentation on the right
\makebox[-2.5pt]{\begin{tikzpicture}[overlay]
\node[draw=ocre!60,line width=1pt,circle,fill=ocre!25,font=\sffamily\bfseries,inner sep=2pt,outer sep=0pt] at (-15pt,0pt){\textcolor{ocre}{R}};\end{tikzpicture}} % Orange R in a circle
\advance\baselineskip -1pt}
{\end{list}\vskip1mm\end{rBox}\vskip5pt} % Tighter line spacing and white space after remark



\input{/home/aroquemaurel/cours/includesLaTeX/polices.tex}
\input{/home/aroquemaurel/cours/includesLaTeX/affichageChapitre.tex}
\newcommand{\pfp}{\texttt{pfp}}

\newcommand{\ifp}{\texttt{if}}
\newcommand{\elsep}{\texttt{else}}

\makeatother
\includeonly {
%	content/chapitre1,
%	content/chapitre2,
%	content/chapitre3,
	content/chapitre4,
%	annexes/annexes,
%	annexes/exercices/indexExercices,
%	annexes/pointeurs,
%	annexes/sdDynamique,
}
\begin{document}
	\dominitoc
	\setcounter{tocdepth}{1}
	\setcounter{secnumdepth}{3}
	\setcounter{minitocdepth}{1}
	\maketitle
	\chapter*{Avant-propos}
	Suite du module d'algorithmique et programmation, accent sur les structures de données
	\begin{itemize}
		\item Pile
		\item File
		\item Arbre
		\item \ldots
	\end{itemize}
	\section*{Heures}
	\begin{itemize}
		\item 24h de CTDI
		\item 26 de TDM
	\end{itemize}
	\section*{Notation}
	\begin{description}
		\item[Contrôle intermédiaire] 30\%
		\item[Contrôle terminal] 50\%
		\item[TP] 20\%
			\begin{description}
				\item[TP Noté] 50\%
				\item[Devoir ecrit] 25\%
				\item[Devoir TP] 25 \%
			\end{description}
	\end{description}
	\tableofcontents
	\chapter{Projet et Équipe de management}
	\section{Présentation des créateurs}
		\begin{itemize}
			\item \Bonte{}, Ingénieur en \gHabitat{}
			\item \Ben{}, Ingénieur en \gHabitat{}
			\item \Drm{}, Développeur Web
			\item \Soum{}, Développeur \texttt{\glo{C++}{C++}{4e langage de programmation
				le plus utilisé au monde. Il est compilé, permettant de produire un programme
				s'éxecutant le plus rapidement possible.}/\glo{Qt}{Qt}{Bibliothèque programmée
				en C++ permettant de créer des interfaces graphiques.}}
			\item \Clem{}, Développeur \texttt{C++/Qt}, Administrateur Système et Réseau
		\end{itemize}
		
		\subsection{Formations}
			\Bonte{} et \Ben{} ont obtenu un Master Pro \gHabitat{} à l'INSA de Toulouse et
			font actuellement une thèse.

			\Drm{}, \Soum{} et \Clem{} sont en deuxième année de DUT Informatique 
			à l'IUT Paul Sabatier de Toulouse. Ces derniers sont
			également autodidactes et ont pu acquérir de nombreuses conpétences lors de projets personnels.
	
		\subsection{Expériences professionnelles}
			\Bonte{} a passé un an dans un bureau d'étude à réaliser des bilans thermique. C'est pendant cette periode 
			que lui est venue l'idée de notre projet après avoir constaté le manque dramatique d'affordance des solutions disponibles.
			Le reste de l'équipe s'est constitué autour de ce constat global.
	
	\section{Atouts}
		%atouts qui font qu'on a des facilités a créer l'entreprise

		%facultés particulières
		Ayant déja eu une expérience professionnelle, \bonte{} et \ben{} connaissent les difficultés et les besoins des TPE et PME du \gHabitat{}.
		%contacts
		Ainsi, nous disposons déja de contacts dans ce secteur,
		nottamment au sein d'établissements universitaires et de bureaux d'étude.
		Ce premier carnet d'adresse peut être facilement étoffé car notre clientelle est,
		par sa taille humaine, facilement abordable et particulièrement à l'écoute pour trouver des solutions gratuites et efficaces.
		%connaissances pratiques théo
		Grâce aux cours généraux enseignés à l'université et en DUT, 
		tout les associés ont des connaissances en Comptabilité, Gestion et Droit des Entreprises,
		ce qui permet de faciliter les échanges avec les professionnels (comptables, avocats...) que nous ne manquerons pas de contacter.
		%part à des orga assoc
		\bonte, actuellement thésard, interviens dans des promotions de \gHabitat{} et a, 
		de ce fait, la possibilité de présenter des produits de \K{} aux étudiants. \\
		\clem{} est quant à lui impliqué dans diveres associations et a de ce fait rencontré plusieurs personnes ayant fondé ou travaillant dans une \glo{SCOP}{SCOP}{Société soumise à l’impératif de rentabilité comme toute entreprise.
Ses salariés-coopérateurs y sont en effet associés (ou « co-entrepreneurs ») majoritaires et détiennent au moins 51\% du capital et 65\% des droits de vote. Par ailleurs, quelle que soit la quantité du capital détenu, chaque coopérateur ne dispose que d'une seule voix lors de l'assemblée générale de l'entreprise.
}\footnotesouvenir{scop}{\textbf{S}ociété \textbf{CO}opérative et \textbf{P}articipative}. Cela nous permet d'avoir des réponses rapides et un premier contact avec le réseau des SCOPs\footnoterappel{scop}, qui permet aux jeunes entreprises de bénéficier d'avantages divers afin de se développer.
		%aide famille...


	\section{L'idée}
		% societe de dev de logiciel et de prestation de services informatiques dans le génie de l'habitat
		\K{} est une Société de Développement de Logiciels et de Prestation de Services Informatiques dans le secteur du \gHabitat{}.

		% Comment est venue l'idée
		% Secteur du génie de l'habitat
		L'idée de ce projet est née à travers diverses expériences dans le domaine du génie climatique.
		À l'heure actuelle, les professionnels n'ont à leurs disposition que peu d'outils : 
		\begin{itemize}
			\item Tableurs Excel réalisés en interne, aux résultats approximatifs dans un contexte de maîtrise de l'énergie
				et dans lesquels la saisie des données est peu aisée.
			\item Logiciels réglementaires coûtant plusieurs milliers d'euros, à l'ergonomie souvent douteuse et peu adaptés a de petites structures telles que les PME\footnote{\textbf{P}etites et \textbf{M}oyennes \textbf{E}ntreprises} et les TPE\footnote{\textbf{T}rès \textbf{P}etite \textbf{E}ntreprise}
		\end{itemize}
		
		% C'est une création
		Nous souhaitons donc créer une société a l'écoute des besoins de ces petites structures, afin de leurs permettre d'économiser leurs ressources lors de leurs projets grâce à des outils adaptés à leurs échelle.
		% Sur Toulouse
		Notre équipe s'étant formée à Toulouse, et le secteur du \gHabitat{} y étant largement développé\footnote{Voir Chapitre \ref{marché}. Marché}, c'est donc dans cette ville que nous implanterons notre société.

	\section{Objectifs du projet}
		%quel objectif ? expansion, retabilité ? autonomie ?
		%prépondérent
		%d'autres ?
		La société \K{} repose sur des valeurs et des principes communautaires
		où le seul objectif est de rendre accessible au plus grand nombre
		l'accès a des outils ergonomiques, intuitifs et performants
		afin de fournir les résultats les plus précis possible 
		dans une optique de maîtrise de l'énergie et de développement durable.

	\section{Taille de l'entreprise}
	%dimension (effectif, CA, capitaux 10890, parts de marché
	%taille max ? min ?
	L'entreprise \K{} souhaite rester une entreprise à taille Humaine, ainsi elle sera composé d'un maximum de 20 personnes afin que tout le monde soit impliqué dans l'entreprise.

	Le capital de départ sera de 10090\euro{} et pourra évoluer selon les activités de l'entreprise. 

	Notre but serai d'avoir un chiffre d'affaire de $104\;400$\euro{} au bout de deux ans d'activités et atteindre les 2~220~000\euro{} dans les 4 ans après la création de notre entreprise. 

	Dans un premier temps, \K{} favorisera l'évolution au sein de Midi-Pyrénnés, si celle-ci fonctionne
	convenablement, elle s'étendra au reste de la France dans un second temps.
	
	\chapter{\'Evolution des réseaux}
		De la même manière que la téléphonie et le télégraphe, nous sommes passé d'une phase expérimentale à une phase d'utilisation. Ainsi l'Informatique à beaucoup évolué. Cette évolution à été progressive, il y a eu plusieurs étapes qui ont marqués les réseaux de communication.
		\paragraph{Coûts des équipements Informatiques / Coûts de la Communication} À l'origine seul les grands comptes étaient capable d'avoir des équipements informatiques. Ainsi les SSI\footnote{Société de Service en Informatiques} sont nées.
		\paragraph{Système de Télétraitement} Ces systèmes ont été destiné aux entreprise, afin qu'a distance elles puissent utiliser la puissance d'un calculateur qui était géographiquement loin. Une première structure de réseau Informatique fut créée.
		\remarque{Nous sommes en train de revenir à cette solution créée 40 ans auparavant: Le cloud computing}

		\section{Les équipements créés}
		Afin de construire ces structures de réseaux de communication nous avons mis en place des équipements :
		\begin{description}
			\item[Processeur Frontal de Communication\footnote{FEP: Front End Processor}]
			\item[Multiplexeurs et concentrateurs] Équipement de partage du support de communication, permettent d'avoir des nœuds de communication.
			\item[Liaisons Spécialisées] Nous avions besoin d'un réseau spécialisé afin d'interconnecter les appareils, pour les connections point à point.
			\item[Modem] Pour les trafics de grande ligne, il fut choisir d'utiliser un réseau déjà existant, le téléphone.
		Cependant, le signal à transmettre doit être adapté au support de transmission, on va donc utiliser un adaptateur qui permettra de faire 
		passer le signal sur le réseau téléphonique : le modem.
			\item[Commutateurs] Pour avoir une connexion la plus rapide possible, nous avions besoin d'un algorithme de routage afin de passer par un chemin en fonction du trafic présent sur la ligne: le routeur.
			\item[Protocole de communication] Permet de faire dialoguer deux machines entre elles, elle doivent utiliser le même protocole afin de se comprendre syntaxiquement et sémantiquement.
		\end{description}
		
		\section{Démocratisation de l'Informatique}
		\begin{description}
			\item[1970] La genèse des protocoles de communication date des années 1970. En réseau, rien n'a été inventé de nouveau, cela à surtout été des progrès technologiques : rapidité, miniaturisation, coûts et donc démocratisation. Les premiers mini-calculateurs.
			\item[1980] Début de l'informatique personnelle et mise en \oe{}uvre des réseaux locaux.
			\item[1990] Applications de l'Internet, premiers mobiles et satellites. 
		\end{description}

	
	\chapter{Complexité d'algorithmes définis par récurrence}
	\section{Exemple introductif : Tri fusion}
\'Etant donné un tableau T, on note T[i:j] le sous tableau de T qui va de la case i à la case j. L'algorithme de tri fusion utilise une procédure
\texttt{fusion(T,i,j,k)}. On suppose que les deux sous tableaux T[i:j] et T[j+1:k] sont déjà triés. En temps $\Theta(n)$, où $n=k-i+1$, la procédure
fusion produit le sous tableau T[i:k] trié à partir de la fusion de ces deux tableaux.

\lstinputlisting[language=algo, caption=Algorithme du tri fusion]{triFusion.algo}
	\section{Méthode naïve d'analyse de complexité}
	Soit un temps maximal d'exécution de tri fusion sur un tableau de longueur $n$.

	D'après l'algorithme, on a $$U_n = U_{\frac{n}{2}} + U_{\frac{n}{2}} \Theta(n)$$
	et $u_1 = 0$

	Pour simplifier la récurrence on suppose que $n$ est pair, et donc $U_n = 2U_{\frac{n}{2}} + \Theta(n)$

	La méthode naïve consiste à deviner la solution, ici on devine $U_n \leq c n \log_2 n$. On suppose $U_{\frac{n}{2}} \leq C \frac{n}{2} \log_2
	\frac{n}{2}$ et on essaye d'en déduire $U_n \leq c n \log_2 n$
	\begin{eqnarray*}
		U_n = 2U_{\frac{n}{2}} + cn &\leq& 2c \frac{n}{2} \log_2 \frac{n}{2} + cn \\&&= cn(\log_2 n -1) + cn = cn\log_2 n
	\end{eqnarray*}

	Puisque $u_1=0 \leq c 1 \log_2 1$, on en déduit $\forall n, i_n \leq cn \log_2 n$

	\subsection{Résumé de la méthode naïve}
	Pour une équation récurrente $u_n = f_n(U_{n-1}, \cdots, u_1)$ où f est une fonction monotone croissante
	\begin{enumerate}
		\item On devine une fonction $g$
		\item On suppose que $\forall n < 1$ on a $U_n \leq g(m)$
		\item On montre $U_n = f_n(U_{n-1},\cdots,u_1 \leq f_n(g(n-1), \cdots, g(1)) \leq g(n)$
		\item On conclut par récurrence que $\forall n$ on a $U_n \leq g(n)$
	\end{enumerate}
	\subsection{Exemples d'application}
	On commence par une \textbf{mauvaise} utilisation. Soit l'équation $U_n = 2 U_{\frac{n}{2}}$. L'intuition $U_n \leq kn$ n'est pas correcte.

	En effet, en remplaçant on obtient : 
	\begin{eqnarray*}
		n_n &=& 2U_{\frac{n}{2}}+1\\
		&=& 2k \frac{n}{2} + 1\\
		&=& kn + 1
	\end{eqnarray*}
	
	La bonne intuition est $u_n \leq kn - b$. En remplaçant on obtient : 
	\begin{eqnarray*}
		u_n = 2U_{\frac{n}{2}} + 1\\
		&\leq& 2(k\frac{n}{2} - b) + 1= kn - 2b + 1\\
		&\leq& kn -b\textrm{ Si } b \geq 1
	\end{eqnarray*}

	\subsection{Réduction à des formes simples}
	Lors de l'analyse d'algorithmes récursifs, on rencontre souvent des équations récurrentes de la forme $$u_n = aU_{\frac{n}{2}}+b,$$ où a et b sont des
	constantes. Par exemple le tri fusion.

	Pour convertir ce type de récurrence en une forme affine $u'_n = a'u'_{n-1}+b'$, on pose
	$$v_k = U_{2^k}$$
	Autrement dit, on étudiera la suite $\{u_n\}_{n \geq 0}$ uniquement sur les puissances de 2.
	\\
	Par exemple, pour le tri fusion, en remplaçant $n$ par $2^k$, 
	\begin{eqnarray*}
		U_{2^k} &=& 2U_{\frac{2^k}{2}} + C2^k\\
		\textrm{donc }V_k &=& 2v_{k-1} + c2^k
	\end{eqnarray*}

	\section{Équation récurrentes linéaires}
	\paragraph{Définition} Une équation récurrente linéaire à coefficients constants d'ordre $k$ est une équation de la forme 
	\begin{displaymath}
		\left\{ \begin{array}{llll}
			u_1 &=& C_i (O \leq i \leq k-1) & \textsc{Conditions initiales (CI)}\\
			u_n &=&  \sum^k_i=1 a_i u_{n-i} + g(n) & \textsc{Equation générale}
		\end{array} \right.
	\end{displaymath}

	Une équation est \textbf{homogène} si $\forall n g(n) = 0$. La solution générale est une suite satisfaisant uniquement l'équation générale. Une
	solution particulière est une solution générale satisfaisant aussi des conditions initiales.

	\subsection{Équations récurrentes linéaires homogènes d'ordre 1}
	\paragraph{Proposition} La solution particulière de l'équation : 
	\begin{displaymath}
		\left\{ \begin{array}{lll}
			u_0 &=& c\\
			u_n &=&  a u_{n-1}
		\end{array} \right.
	\end{displaymath}
	est $u_n=C a^n$ (c'est une suite géométrique)

	\subsection{Équations récurrentes linéaires non-homogènes d'ordre 1}
	On ne sait traiter facilement que les équations dans lesquelles le second membre $g(n)$ est un polynôme ou une exponentielle. Pour cela, on
	<<dérive>> l'équation pour faire baisser le degré du polynôme jusqu'à arriver à 0.

	\exemple{\textbf{Le tri fusion}\\
	On a une équation qui n'est pas homogène: $$V_n = 2V_{n-1} + C 2^n$$
	Donc, au rang $n+1$, on a aussi $$V_{n+1} = 2 V_n + C \times 2 ^{n+1}$$

	Pour éliminer la partie non-homogène, on enlève 2 fois la première équation à la seconde.
	\begin{eqnarray*}
		V_{n+1} -2V_n &=& 2V_n - 4 V_{n-1}\\
		V_{n+1} &=& 4V_n - 4V_{n-1}
	\end{eqnarray*}
	}

	\subsection{Recherche d'une solution générale pour les équations récurrentes linéaires homogènes d'ordre 2}
	Une équation récurrente homogène d'ordre 2 est de la forme 
	\begin{displaymath}
		\left\{ \begin{array}{lll}
			u_0 &=&  C_0\\
			u_1 &=&  C_1\\
			u_n &=& a_1 u_{n-1} + a_2 u_{n-2}
		\end{array} \right.
	\end{displaymath}
	On peut obtenir ce type d'équation indirectement lorsque l'on a réduit une équation d'ordre 1 à une équation homogène d'ordre 2.
	\exemple{ L'équation récurrente linéaire homogène d'ordre 2 de Fibonacci
	\begin{displaymath}
		\left\{ \begin{array}{lll}
			U_0 &=&  1\\
			U_1 &=&  1\\
			U_n &=&  U_{n-1} + U_{n-2}
		\end{array} \right.
	\end{displaymath}
	On résout ces équations d'ordre 2 comme des équations d'ordre 1 : On cherche une solution générale de la forme $\lambda r^n$. Une telle solution
	vérifie, pour le cas de la suite de Fibonacci : $\forall n \geq 2$, $\lambda r^n = \lambda r^{n-1} + \lambda r^{n-2}$

	Soit, en divisant par $\lambda r^{n-2}$ $$r^2 = r + 1$$

	Autrement dit, $r$ est une racine du polynôme $P(x) = x^3 -x - 1$.
	}
	\paragraph{Définition}
	Le polynôme caractéristique d'une équation récurrente homogène d'ordre $k$
	$$V_{n+k} + a_1 V_{n+k+1} + \cdots + a_kV_n = 0$$ est le polynôme $P(x) = x^k + a_1x^{k-1}+\ldots+a_{k-1}+a_k$

	\paragraph{Théorème} Si $r$ est une racine du polynôme caractéristique d'une équation récurrente linéaire homogène, alors pour toute constante
	$\lambda$, toute suite de la forme $\{\lambda r^n\}_{n \geq 0}$ est une solution générale de cette équation.

	Dans le cas de la suite de Fibonacci, on calcule le discriminant $\Delta=5$ et on trouve les deux racines $r_1 = \frac{1-\sqrt{5}}{2}$ et 
	$r_2=\frac{1+\sqrt{5}}{2}$

	\paragraph{Cas des racines doubles} Si le discriminant $\Delta = 0$, alors le polynôme n'a qu'une seule racine (de multiplicité 2). En remarquant
	que $r$ racine double de P(x) implique que $r$ est aussi une racine de $P'(x)$ on peut démontrer que $\{n\lambda r^n\}_{n \geq 0}$ est aussi une
	solution de l'équation récurrente.

	\paragraph{Théorème} Les solutions générales d'une équation récurrente linéaire homogène d'ordre 2 dont le polynôme caractéristique de deux racines
	$r_1$ et $r_2$ sont :
	\begin{itemize}
		\item Si $r_1 \neq r_2$ : $\{ \lambda_1 r_1^n + \lambda_2 r_2^n\}_{n \geq 0}$ pour toute constantes $\lambda_1$, $\lambda_2$
		\item Si $r_1 = r_2$ :  $\{(\lambda_1 + \lambda_2 \times n)r_1^n\}_{n \geq 0}$ pour toute constantes $\lambda_1$, $\lambda_2$
	\end{itemize}

	\paragraph{Preuve dans le cas d'une racine double}
	Soit l'équation $u_{n+2} + a U_{n+1} + bu_n = 0$ et soit $r$ une racine double du polynôme caractéristique.\\  
	$P(x) = x^2 + ax + b$, donc $r$ est aussi une racine de $P'(x)= 2x+a$. La suite $\{nr^n\}_{n\geq 0}$ est une solution de l'équation car
	\begin{eqnarray*}
	(n+2)r^{n+2} + a(n+1)r^{n+1} + bnr^n &=& n(r^{n+2} + ar^{n+1} + br^n) + 2r^{n+2} + ar^{n+1}\\
	&=& r^n[n(r^2 + ar + b) + r(2r+a)]\\
	&=& 0
	\end{eqnarray*}
	\subsection{Recherche de solutions particulières pour les équations récurrentes linéaires homogènes d'ordre 2}
	Dans le cas de la suite de Fibonacci, on cherche une solution particulières satisfaisant les conditions initiales et qui est de la forme
	$\lambda_1r_1^n + \lambda_2r_2^n$ où $r_1 = \frac{1-\sqrt{5}}{2}$ et $r_2 = \frac{1+\sqrt{5}}{2}$

	Donc on cherche $\lambda_1$, $\lambda_2$ tels que
	\begin{eqnarray*}
		u_0 &=&  1 = \lambda_1 r^0_1 + \lambda_2r_2^0 = \lambda_1 + \lambda_2\\
		u_1&=&  1 = \lambda_1 r^1_1 + \lambda r^1_2 = \frac{\lambda_1 + \lambda_2}{2} + \frac{\lambda_2 - \lambda _ 1}{2} \times \sqrt{5}\\
	\end{eqnarray*}
	\begin{displaymath}
		\Rightarrow 
		\left\{ \begin{array}{lll}
			1 &=&  \lambda_1 + \lambda_2\\
			\frac{1}{2} &=& \frac{\lambda_2 - \lambda_1}{2}\sqrt{5}
		\end{array} \right.
		\Rightarrow 
		\left\{ \begin{array}{lll}
			1 &=&  \lambda_1 + \lambda_2\\
			\frac{1}{\sqrt{5}} &=& \lambda_2 - \lambda_1
		\end{array} \right.
		\Rightarrow 
		\left\{ \begin{array}{lll}
			\lambda_2 &=&  \frac{1+\frac{1}{\sqrt{5}}}{2}\\
			\lambda_1 &=&  \frac{1-\frac{1}{\sqrt{5}}}{2}
		\end{array} \right.
	\end{displaymath}
	Au final, on trouve la solution particulière : 
	$$U_n = \frac{\sqrt{5}-1}{2\sqrt{5}}(\frac{1-\sqrt{5}}{2}) + \frac{\sqrt{5}+1}{2\sqrt{5}}(\frac{1+\sqrt{5}}{2})^n$$
	
	\subsubsection{Résumé de la méthode pour les équations homogènes d'ordre 2}
	Pour résoudre l'équation $u_n = aU_{n-1} + bu_{n-2}$
	\begin{enumerate}
		\item On calcule le polynôme caractéristique $P(x) = x^2 - ax -b$
		\item On calcul les racines (éventuellement complexes)
			$r_1$ et $r_2$ de $P$
		\item On cherche les coefficients $\lambda_1$ $\lambda_2$ tels que $\lambda_1r_1^n + \lambda_2r_2^n$ satisfaisant les CI
	\end{enumerate}

	\subsection{Équations récurrentes d'ordre $k$}
	Pour les équations récurrentes homogènes d'ordre k, les considérations sur le polynôme caractéristique et ses racines restent valables. La difficulté
	est calculatoire car il faut trouver les racines d'un polynôme de degré k. Mais lorsque l'équation a été obtenu en éliminant  la partie
	non-homogène, les coefficients utilisés sont des solutions. 
	\exemple{Pour l'algorithme de Strasser, on a obtenu l'équation en faisant
	$$E_n-GE_{n-1}$$ où $E_n$ désigne l'équation de rang $n$

	$\Rightarrow$ 4 est une racine du polynôme caractéristique.
	}

	En cas de racine d'ordre $m$, on peut montrer par récurrence que $\{n^j\alpha^n\}_{n\geq 0}$ est une solution de l'équation récurrente homogène pour
	tout $j=0,\cdots,m-1$. Ceci nous permet d'avoir $k$ variables dans le système d'équation linéaires dérivées des CI.

	Le théorème suivant généralise le théorème 2 au cas de récurrences homogènes d'ordre $k > 2$ et prend en compte directement le second membre.
	\paragraph{Théorème 3}
	Supposons que le polynôme caractéristique de la récurrence homogène $u_n = au_{n-1} + \cdots + a_ku_{n-k}$ admet $p$ racines $ri(i=1,\cdots,p)$ de
	multiplicité $mi(i=1,\cdots,p)$. Alors la solution de la récurrence 
	$$u_n = a_1u_{n-1}+\cdots+a_ku_{n-k} + \sum^t_{i=1} b_i^n P_i(n)$$
	où $p_i$ est un polynôme de degré $d_i$) est donnée par $$\sum^t_{i=1}b_i^n Q_i(n)\footnote{Partie de la solution qui prend en compte le second membre} + 
	\sum_{i \in \{1,\ldots,p\}}r_i^n R_i(n)\footnote{Solution pour la récurrence homogène}$$
	tel que $r_i \not\in \{b_1,\cdots,b_t\}$

	Où 
	\begin{displaymath}
		\textrm{deg}(Q_i) = \left\{ \begin{array}{lll}
			d_i & si & b_i \not\in \{r_i,\cdots,r_p \}\\
			d_i+m_j & si& b_i = r_j
		\end{array} \right.
	\end{displaymath}

	Et $\textrm{deg}(R_i) = m_i-1$

	On obtient les polynômes $Q_i$ et $R_i$ à partir des CI et par identification des coefficients des termes $b_i^n n^j$ dans la récurrence.

	\remarque{Dans le théorème 2, il n'y avait de second membre (t=0) et les polynômes $R_i$ étaient de la forme $\lambda_1$ ou $\lambda_1 = \lambda_2 n$
	}
	\begin{displaymath}
		\left. \begin{array}{lll}
			u_n &=&  u_{n-1} + 1n^3\\
			u_{n-1} &=&  u_{n-2} + 1
		\end{array}
		\right\}
		u_n - u_{n-1} = u_{n-1} - u_{n-2}
	\end{displaymath}
	\exemple{
	$$T(n) = 2T(\frac{n}{2}) + n ; T(1) = 1$$
	Après changement de variable $n=2^k$, $u_k = T(n)$, nous avons $u_k = 2u_{k-1}+2^k$.

	Ici le second membre $$\sum^t_{i=1} b_i^k Pi(k) = 2^k$$

	Donc $T=1$, $P_i(k)=1$, $b_i=2$

	Le polynôme caractéristique $P(x)=x-2$. La seule racine est $r_i = 2$.
	Donc la solution particulière  est de la forme $2^k(q_0 + q_i^k)$ car $\textrm{deg}(Q_i) = \textrm{deg}{P_i}  +\textrm{multiplicité}  = 0 + 1$, et
	cette solution satisfait la CI et la récurrence $1 = T(1) = u_0$
	$$2^k(q_0+q_1k) = 2 \times 2^{k-1}(q_0+q_i(k-1))$$	
			D'où $q_0 = 1$, $q_1 = 1$ donc $u_n = 2^k(1+k)$ et $T(n) = u_k = n(1+\log_2n)$
	}

	\subsection{Théorème pour les récurrences par divisions}
	Le théorème suivant nous donne directement l'ordre de grandeur de la solution en fonction des coefficients de l'équation récurrente.

	\paragraph{Théorème 4} Soient $a \geq 1$, $b > 1$ deux constantes, $f(n)$ une fonction, et $\{t(n\}_{n\geq 0}$ une suite vérifiant l'équation 
	$T(n) = aT(\frac{n}{b}) + f(n)$

	On a pour $\epsilon > 0$
\begin{itemize}
	\item Si $f(n) = 0(n^{\log_b a - \epsilon}$ alors $T(n) = \Theta(n^{\log_b a})$
	\item Si $f(n) = \Theta(n^{\log_b a})$ alors $T(n) = \Theta(n^{\log_b a . \log n})$
	\item Si $f(n) = \Omega(n^{\log_b a + \epsilon}$ et $af(\frac{n}{b}) \leq cf(n)$ pour une constante $c >1$, alors $T(n) = \Theta(f(n))$
\end{itemize}
	
	
	\chapter{Spécification d'un programme}
	\remarque{Durant ce chapitre, nous parlerons de programme, cependant cela est valable également pour les sous-programme}
	Un programme est spécifié par un triplet : 
	\begin{itemize}
		\item Prédicat d'entrée P(E) ou précondition
		\item action (E, S)
		\item Prédicat de sortie P(S) ou postcondition
	\end{itemize}

		Les prédicats sont écrits en utilisant le formalisme de la logique des prédicats et de sopérations booléeenes.
		\section{Mots clés à utiliser dans les prédicats}
		Les mots clés pouvant être utilisés: 
		\begin{itemize}
			\item Les quantificateurs logiques : $\forall$(quelque soit), $\exists$(il existe), $\nu$(nombre de)
			\item Les connecteurs logiques : $\wedge$(et), $\vee$(ou), $\rightarrow$(implique), $\leftrightarrow$(equivalence), $\lnot$(not)
		\end{itemize}
		\section{\'Ecriture de la spécification}
		C'es une traduction de l'énoncé et de l'analyse faite dans l'étape 1 de la méthodologie : c'est un \textbf{triplet} logique.
		La démarche pour écrire la spécification est la suivante.
			\begin{itemize}
				\item Identifier les propriétés des données d'entrée et les exprimer sous forme logique
				\item Identifier les propriétés sur les données en sortie et les exprimer sous forme logique. 
			\end{itemize}
		\exemple{
			\'Ecrire un programme qui trie un tableau T de N éléments.\\
			\begin{itemize}
				\item $N > 1$
				\item \texttt{trier (T, N, t);}
				\item $(\forall I : 0 \leq I < N-1 \longrightarrow T[I] \leq T[I+1]) \wedge$\\$
					(\forall I : 0 \leq I < N \longrightarrow $\\$(\nu J : 0 \leq J < N \wedge t[I] = t[J]) = (\nu J : o \leq J < N \wedge t[I] = T[J]))$ 
			\end{itemize}
		}
	
	
	\appendix
	\chapter{Cours sur les pointeurs en C}\label{pointeurs}
Déjà vu par le passages de paramètres.
\section{Syntaxe}
	\subsection{Déclaration}
\begin{lstlisting}[language=C, numbers=none,frame=none]
typePointé* nomPointeur
\end{lstlisting}
\lstinputlisting[caption=Exemple de déclaration, numbers=none,language=C]{annexes/ptr1.c}
	\subsection{Utilisation}
\begin{lstlisting}[language=C, numbers=none,frame=none]
nomPointeur // manipule l'adresse
*nomPointeur //manipule la zone pointée
\end{lstlisting}
\begin{lstlisting}[language=C, numbers=none,frame=none]
pe=&n; //opérateur d'adressage
\end{lstlisting}

\subsection{Constante}
	\texttt{NULL} représente une adresse inexistante.
	\begin{lstlisting}[language=C, numbers=none,frame=none]
pe = NULL;
*pe; // Erreur à l'exécution
	\end{lstlisting}

\section{Opérateur autorisés sur les pointeurs}
\subsection{L'affectation}
\begin{verbatim}
nomPointeur =  expression correspondant à une adresse ou à NULL
\end{verbatim}

\subsection{Addition et la soustraction entre un pointeur et un entier}
\begin{verbatim}
nomPointeur = nomPointeur + 10; 
nomPointeur = nomPointeur - 15; 
\end{verbatim}
On obtient une expression correspondant à une adresse
\begin{verbatim}
pe = pe+10; //pe contient l'adresse du 10e entier après la valeu initalie de pe.
\end{verbatim}
\attention{À utiliser que si pe pointe sur un tableau}

\subsection{Soustraction de deux pointeurs}
Renvoi un entier donnant le nombre d'éléments pointés entre les deux pointeurs
\attention{Uniquement si les deux pointeurs sont sur le même tableau}

\subsection{Comparaison sur des pointeurs}
Ce sont les opérateurs de comparaison classique : \texttt{= =} et \texttt{!=}

\subsection{Allocation dynamique de mémoire}
\begin{lstlisting}[language=C, numbers=none,frame=none]
nomPointeur = (typePointeur) malloc(sizeof(typePointé));
nomPointeur = (typePointé*) malloc(n*sizeof(typePointé));
\end{lstlisting}
\begin{lstlisting}[language=C, numbers=none,frame=none]
	int *e;
	pe = (int*) malloc(sizeof(int));
\end{lstlisting}
\begin{enumerate}
	\item Le programme demande au gestionnaire mémoire à avoir une place de la taille \texttt{sizeof(int)}
	\item Si la place est disponible retourne l'adresse demandée ou la première case du <<tableau>> dynamique
	\item Sinon retourne \texttt{NULL}
\end{enumerate}

\subsection{Libération dynamique de mémoire}
\begin{lstlisting}[language=C, numbers=none,frame=none]
free(nomPointeur);
\end{lstlisting}
\begin{enumerate}
	\item Le programme contact le gestionnaire mémoire 
	\item Le gestionnaire mémoire <<libère>> la place
\end{enumerate}

Cela veut dire que la place n'est plus réservé au programme, elle pourra être alloué à un autre programme.
\attention{Le gestionnaire de mémoire ne met pas à jour la case mémoire, celle-ci contient toujours la valeur, si personne ne récupère la case, il sera toujours
possible d'accéder à la donnée. C'est donc aléatoire, c'est une source d'erreurs.}


	\lstlistoflistings
	\listoffigures
	\chapter{Exercices}
\section{Initiation à la programmation}
	\subsection{Exercice 1}
\'Ecrire un programme qui lit une série de 10 valeurs et affiche la position du minimum et du maximum de la série.
\subsubsection{\'Etape 1: Analyser le problème}
\begin{enumerate}
	\item Lire les valeurs
	\item calculer les min et max
	\item afficher le résultat
\end{enumerate}
\subsubsection{\'Etape 2: Spécifier les sous-problèmes}
Identifier les entrée, les sorties et leurs propriétés.
\paragraph{\texttt{LireLesValeurs}}
\begin{description}
	\item[Entrée] Nombre, les valeurs à lire
	\item[Sortie] Tableau contenant les valeurs lues
\end{description}
\paragraph{\texttt{CalculerMinEtMax}}
\begin{description}
	\item[Entrée] Le tableau des valeurs et le nombre de valeur 
	\item[Sortie] Position, min et max. 
\end{description}
\newpage
\subsubsection{\'Etape 3: Le code}
\lstinputlisting[language=C, caption=Exercice 1 -- Code du programme]{annexes/exo1.c}
\newpage

\section{Spécification}
\subsection{Exercice 7}
	\'Ecrire la spécification d'un programme qui dans un tableacu T de N entiers calcul le nombre n de nombre positifs dans le tableau.

	\begin{itemize}
		\item $N > 0$
		\item \texttt{calculeNbPos(T,N,n)}
		\item $(o \leq n \leq N) \wedge (n = \nu I : o \leq I < N T[I] \geq 0) $
	\end{itemize}
\subsubsection{Exercice 8}
	Soit T un tableau croissant (non strict) de N entier et X un entier.\\
	Spécifier un programme qui calcule la position de la dernière occurrence de T inférieure ou égale à X avec $T[0] \leq X < T[N-1]$

	\begin{itemize}
	%	\item $T[0] \leq X < T[N-1]$, $T[I] \leq T[I+1]$, $N > 0$
		\item $(N > 1) \wedge (T[0] \leq X) \wedge (X < T[N-1]) \wedge (\forall I : 0 \leq I < N - 1 \rightarrow T[I] \leq T[I+1])$
		\item \texttt{searchPosition(T, N, X, p));} 
		\item $(0 < p < N - 1) \wedge (T[p] \leq X) \wedge (T[p+1] > X)$
	\end{itemize}
	\remarque{Dans la suite du cours, nous pourrons utiliser un raccourcie afin de savoir si un tableau est trié par ordre croissant : $(T,N,\leq)$

	Celle-ci pourra être utilisée dans la copie à condition qu'elle soit définie au préalable.}

\subsection{Exercice 8}
	Soit un tableau T non vide de N entiers.  \'Ecrire la spécifications du qui programme qui calculent :
	\begin{itemize}
		\item La première position de la valeur max de T
		\item La dernière position de la valeur max de T
	\end{itemize}
	\subsubsection{Calcule de la première position}
	\begin{itemize}
		\item $N > 0$ 
		\item \texttt{searchFirstPosition(T, N, f);} 
		\item $(\forall I :0 \leq I < f \rightarrow T[I] < T[f]) \wedge (\forall I (f \leq I < N) \rightarrow (T[I] \leq T[f]))$ 
	\end{itemize}
	\subsubsection{Calcule de la dernière position}
	\begin{itemize}
		\item $N > 0$ 
		\item \texttt{searchLastPosition(T, N, l);} 
		\item $(\forall I :0 \leq I < l \rightarrow T[I] \leq T[l]) \wedge 
			(\forall I (l < I < N) \rightarrow (T[I] < T[l]))$ 
	\end{itemize}

\subsection{Exercice 9}
	\'Ecrire la spécification d'un programme qui, dans un tableau T de N entiers tous différents cherche la position d'une valeur X si elle existe ou retourne N si elle n'existe pas.

	\begin{itemize}
		\item $(N\geq0) \wedge (\forall I : 0 \leq I < N \rightarrow (\forall (I, J) : 0 \leq I < N \wedge (0 \leq J < N) \rightarrow T[I] = T[J] \leftrightarrow (I = J)$\footnote{Cela peut aussi s'écrire $N\leq 0) \wedge (\forall I (O \leq I < N) \rightarrow \forall J : J \neq I \wedge 0 \leq J < N \rightarrow T[I] \neq T[J]$}
		\item \texttt{search(T, N, x)}
		\item $(0 \leq p < N \wedge T[p] = X) \vee (p=N) \leftrightarrow \forall I (o\leq I < N) \rightarrow T[I] \neq X))$ 
	\end{itemize}

	\subsection{Exercice 10}
	Spécifier un programme qui, dans un tableau T de N éléments trié par ordre croissant non strict retourne la longueur du plus grand plateau\footnote{Un plateau est quand il y a plusieurs fois le même caractère}.

\begin{itemize}
	\item $(T, N, \leq) \wedge N > 0$
	\item \texttt{longueurPlusGrandPlateau(T,N,l);}
	\item $(1 \leq l \leq N) \wedge (\exists I : 0 \leq I < N-l)\wedge (T[I] = T[I+l-1])$ 
\end{itemize}
	\subsection{Exercice 11}
		Spécifier un programme qui, dans un tableau de N entiers calcule le nombre de doublons : un doublon est une succession de 2 nombres identiques.

		\begin{itemize}
			\item $N > 0$
			\item \texttt{calculeDoublons(T, N, n);}
			\item $n = \nu I : 0 \leq I < N \wedge T[I] = T[I+1]$
		\end{itemize}
		\remarque{
			Dans le cas ou deux doublons ne sont pas forcément côte à côte, le prédicat de sortie deviendrait : 
			$$n = \sum^{N-1}_{I=0} \nu J : I < J < N \wedge T[J] = T[I]$$
		}


\section{Tableau de situation}	
\subsection{Exercice 2}
\lstinputlisting[language=C, caption=Exercice 3]{annexes/exo3.c}
\begin{center}
\begin{tabular}{c  |  c  c  c  c  }
\textbf{Point d'arrêt }& \textbf{T} & \textbf{j} & \textbf{i} &\textbf{n}\\
\hline
1 & 0,0,0,0,0 & 3 & 5 & 1\\
\hline
2 & 0,0,0,1,0 & 4 & 3 & @\\
\hline
3 & 5,9,1,1,1&4&5&\\
\hline
\end{tabular}
\end{center}
\newpage
\subsection{Effets de bords}
\subsubsection{Exercice 4}
\lstinputlisting[language=C, caption=Exercice 4]{annexes/exo4.c}
\begin{center}
\begin{tabular}{c |  c  c  c  c  c  c  }
	\textbf{Point d'arrêt} & \textbf{y} & \textbf{x} & \textbf{t} & \textbf{i} & \textbf{z} & \textbf{f}\\
	\hline
	2 & 10 & / & / & 1 && \\
	\hline
	1 & 11 & 1 & 2 & 1 & / & 13\\
	\hline
	3 & 11 & / & / & 1 & 24 & /\\
	\hline
\end{tabular}
\end{center}
\newpage
\subsubsection{Exercice 5}
\lstinputlisting[language=C, caption=Exercice 5]{annexes/exo5.c}
\begin{center}
\begin{tabular}{c |  c  c  c  c  c  }
	\textbf{Point d'arrêt} & \textbf{i} & \textbf{j} & \textbf{x} & \textbf{y} & \textbf{z}\\
	\hline
	3 & 3 & 7 &/ &/&/\\
	\hline
	1 & &&@i&@i&@j\\
	\hline
	2 & 6 & 12 & @i & @i & @j\\
	\hline
	4 & 6 & 12 & / & / & /\\
	\hline
	5 & 3 & 7 & / & / & / \\
	\hline
	1 & 3&10 &@j&@j&@i\\
	\hline
	2 & 20 & 10 & @j&@j&@i\\
	\hline
\end{tabular}
\end{center}
\newpage
\subsubsection{Exercice 6}
\lstinputlisting[language=C, caption=Exercice 6]{annexes/exo6.c} 
\begin{tabular}{c |  c  c  c  c  c  c  }
	\textbf{Point d'arrêt} & \textbf{i} & \textbf{j} & \textbf{x} & \textbf{a} & \textbf{a} & \textbf{f}\\
	\hline
	2&10&40&/&/&/&/\\
	\hline
	1 & 20 & 40 & / & 10 & 40 & 50\\
	\hline
	3 & 20 & 40 & 50 & / & / & /\\
	\hline
\end{tabular}


\section{Plus faible précondition}
\subsection{Séquence}\label{exoPfpSequence}
\subsubsection{Exercice 1}
\begin{lstlisting}[language=C]
	/* f = i! */
	f = f * (i + 1);
	i = i + 1;
	/* f = i! */
\end{lstlisting}
\begin{eqnarray*}
	f = i! &\rightarrow& \texttt{pfp}("f = f\times (i+1); i = i +1;", f= i!)\\
	f=i!&\rightarrow&\texttt{pfp}("f=f\times(i+1);",\texttt{pfp}("i=i+1",f=i!)\\
	f = i! &\rightarrow& \texttt{pfp}("f = f \times (i + 1);" f = (i + 1)!)
\end{eqnarray*}
\newpage
\subsubsection{Exercice 2}
\begin{lstlisting}[language=C]
	/* (x = A) $\wedge$ (y = B) $\wedge$ (z = C) */
	x = x + y + z;
	z = x - y - z;
	y = x - y - z;
	x = x - y - z;
	/* (x = B) $\wedge$ (y = C) $\wedge$ (z = A) */
\end{lstlisting}
\begin{eqnarray*}
	PE &\rightarrow& \texttt{pfp}("x=x+y+z;z=x-y-z;y=x-y-z;x=x-y-z;", \\&&(x=B)\wedge(y=C)\wedge(z=A)\\
	PE &\rightarrow& \texttt{pfp}("x=x+y+z, z-x-y-z;y=x-y-z;", \\&&\texttt{pfp}("x=x-y-z;(x=B)\wedge(y=C)\wedge(z=A))\\
	PE &\rightarrow& \texttt{pfp}("x=x+y+z, z-x-y-z;y=x-y-z;", x-y-z=B)\wedge y=C \wedge z = A\\
	PE &\rightarrow& \texttt{pfp}("x=x+y+z,z=x-y-z", (y=B)\wedge (x-y-z=C)\wedge z = A\\
	PE &\rightarrow& \texttt{pfp}("x=x+y+z", (y=B)\wedge (z=C) \wedge (x-y-z=A)\\
	PE &\rightarrow& (y=B) \wedge (z=C)\wedge(x=A) \textmd{Vrai parceque }p\rightarrow p = \textmd{vrai}
\end{eqnarray*}

\subsection{Sélection} \label{exoPfpSelect}
\subsubsection{Exercice 1}
\begin{lstlisting}[language=C]
/* x = A $\wedge$ y = B */
if (A < B) {
	x = A;
	y = B;
} else {
	x = B;
	y = A;
}
/* x $\leq$ y */ 
\end{lstlisting}
\begin{eqnarray*}
PE &\rightarrow& \pfp ("\ifp (A < B) \{ x = A; y = B\}\elsep \{ x = B; y =A\}", x \leq y)\\
PE&\rightarrow&((A<B)\rightarrow\pfp("x=A;y=B",x\leq y))\wedge((A\geq B)\rightarrow\pfp("x=B;y=A",x\leq y))\\
PE &\rightarrow& ((( A < B) \rightarrow (A \leq B)) \wedge (A \geq B ) \rightarrow (B \leq A))
\end{eqnarray*}
Vrai par définition. A et toujours inférieur à B. $(A \geq B) \rightarrow (B \leq A)$ est une Tautologie. 
\subsubsection{Exercice 2}
$$\pfp ("\ifp (x \geq y )\{z=x;\}\elsep\{z=y\}",z=\texttt{max}(x,y))$$
\begin{eqnarray*}
x \geq y  &\rightarrow&	\pfp ("\ifp (x \geq y )\{z=x;\}\elsep\{z=y\}",z=\texttt{max}(x,y))\\
x \geq y &\rightarrow& \pfp ("z = x", z = \texttt{max}(x,y))\\
x < y &\rightarrow& \pfp ("z = y", z = \texttt{max}(x,y))\\
x \geq y &\rightarrow& x = \texttt{max}(x,y)\\
x < y &\rightarrow& y = \texttt{max}(x,y)
\end{eqnarray*}
C'est une tautologie par définition de \texttt{max}.

\subsubsection{Exercice 3}
$$\pfp ("\ifp (x > y) \{ if(x \%2 ==  0) \{ x = x - 2\}\} \elsep \{y=y-1;\}", y-2 < x);$$
\begin{eqnarray*}
(x > y) &\rightarrow& \pfp("if(x\%2 == 0) { x = x-2; } ", y-2 < x)) \wedge ( ( x \leq y) \rightarrow \pfp ("y = y-1;", y - 2 < x) )\\
(x > y) &\rightarrow& ( ( (x \%2 = =0) \rightarrow \pfp ("x = x-2", y-2 < x) \wedge \\
& &(x \% 2 != 0) \rightarrow(y-2 < x)) \wedge ( (x \leq y) \rightarrow (y-z < x))\\ 
(x > y) &\rightarrow& ( (x\%2 = = 0)\rightarrow ( (y-2) < (x - 2) ) \wedge ( ( x\%2 != 0) \rightarrow (y-2 < x))) \wedge (x \leq y \rightarrow y-z < x) \\
(x > y) &\rightarrow& (x \leq y \rightarrow y - z < x)
\end{eqnarray*}
\subsection{Boucles}\label{exoPfpBoucles}
\subsubsection{Exercice 1}
\begin{lstlisting}[language=C]
/* $N \geq 0$ */

/* P Tableau de polynôme
* N Degré du polynôme
* X Point ou je veux évoluer le polynôme
* r Résultat du polynôme
*/
calculPolynome(P,N,X,r);
/* $r = \sum^N_{k=0} A[I]X^k $ */
\end{lstlisting}
\begin{lstlisting}[language=C]
/* $N \geq 0$ */

/*	INV = $(o \leq i \leq N) \wedge (r = \sum^i_{k=0} A[k]x^k) \wedge (y = x^i)$*/
while(c) {
/* c \wedge INV */
corps;
/* INV */
}
/* $\neg c \wedge INV$ */
/* $p = \sum^N_{k=0} A[k] x^k$ */
\end{lstlisting}
\paragraph{INVARIANT}
$$(o \leq i \leq N) \wedge (r = \sum^i_{k=0} A[k]x^k) \wedge (y = x^i)$$
\paragraph{Intitialisation}
\begin{enumerate} 
	\item INIT 0 \begin{lstlisting}[language=C,numbers=none]
i = 0; 
p = 0; 
y = 1;
\end{lstlisting}~
\begin{eqnarray*}
PE &\rightarrow& \pfp (\texttt{init}\ 0, \texttt{INV})\\
N \geq 0 &\rightarrow& \pfp ("i=0;p=0", p=\sum^i_k=0 A[k]x^k \wedge 1 = x^i);\\
N \geq 0 &\rightarrow& \pfp ("i=0", 0 = \sum^i_{k=0} A[k]x^k \wedge 1 = x^i\\
N \geq 0 &\rightarrow& 0 = \sum^0_{k=0} A[k]x^k \wedge 1 = x^0\\
(N \geq 0 &\rightarrow& 0 = A[0] \wedge 1 = 1) \Rightarrow \textrm{Faux et Vrai, donc le programme est Faux} \end{eqnarray*}
	\item INIT 1 \begin{lstlisting}[language=C,numbers=none]
i = 0; 
p = A[0]; 
y = 1;
\end{lstlisting}
\begin{eqnarray*}
PE &\rightarrow& \pfp (\texttt{init}\ 1, \texttt{INV})\\
N \geq 0 &\rightarrow& \pfp ("i=0;p=A[0]", p=\sum^i_k=0 A[k]x^k \wedge 1 = x^i);\\
N \geq 0 &\rightarrow& \pfp ("i=0", A[0] = \sum^i_{k=0} A[k]x^k \wedge 1 = x^i\\
N \geq 0 &\rightarrow& A[0] = \sum^0_{k=0} A[k]x^k \wedge 1 = x^0\\
(N \geq 0 &\rightarrow& A[0] = A[0] \wedge 1 = 1) \Rightarrow \textrm{Vrai et Vrai, donc le programme est correct}
\end{eqnarray*}
\end{enumerate}
L'initialisation nécessaire est donc init 1.
\paragraph{Boucles} 
\begin{enumerate} 
	\item \begin{lstlisting}[language=C,numbers=none]
while(i < N) { 
++i; 
p = p + A[i-1] * y; 
y = y * X; 
}
	\end{lstlisting}~
	\'Etape numéro \ref{etape3boucle} (Cf section \ref{pfpBoucle} page \pageref{etape3boucle})
	\begin{eqnarray*}
		\neg C \wedge INV &\rightarrow& PS\\
		(i > N) \wedge (p = \sum^i_{k=0} A[k] x^k) \wedge(y=x^i) &\rightarrow& p = \sum^N_{k=0}A[k]x^k\\
		(i > N) \wedge (p = \sum^i_{k=0} A[k] x^k) \wedge(y=x^i) &\rightarrow& p = \sum^N_{k=0}A[k]x^k\wedge (i=N)\\
		(i > N) &\rightarrow& (i = N) \Rightarrow \textrm{C'est donc faux.}\\
	\end{eqnarray*}
	\remarque{Nous sommes partis de la conclusion et avons essayé de faire apparaitre notre hypothèse en partie droite.}
	\item \begin{lstlisting}[language=C,numbers=none]
while( i != N) { 
++i; 
p = p + a[i-1] * y; 
y = y * X 
} 
\end{lstlisting}~
	\'Etape numéro \ref{etape3boucle} (Cf section \ref{pfpBoucle} page \pageref{etape3boucle})
	\begin{eqnarray*}
		\neg C \wedge INV &\rightarrow& PS\\
		(i \neq N) \wedge (p = \sum^i_{k=0} A[k] x^k) \wedge(y=x^i) &\rightarrow& p = \sum^N_{k=0}A[k]x^k\\
		(i \neq N) \wedge (p = \sum^i_{k=0} A[k] x^k) \wedge(y=x^i) &\rightarrow& p = \sum^N_{k=0}A[k]x^k\wedge (i=N)\\
		(i \neq N) &\rightarrow& (i = N) \Rightarrow \textrm{C'est donc vrai}. 
	\end{eqnarray*}
	\'Etape numéro \ref{etape2boucle} (Cf section \ref{pfpBoucle} page \pageref{etape3boucle})
	\begin{eqnarray*}
		C \wedge INV &\rightarrow& \pfp(corps, INV)\\
		C \wedge INV &\rightarrow& \pfp ("i++;p=p+A[i-1]*y;", (p=\sum^i_{k=0}A[k]x^k)\\&&\wedge(y*x=x^i)\wedge(0\leq i\leq n )\\
		C \wedge INV &\rightarrow& \pfp ("i++;", (p+A[i-1]*y = \sum^i_{k=0}A[k]x^k)\\&&\wedge(y*x=x^i)\wedge(0\leq i\leq n )\\
		(i \neq N) \wedge (0\leq i \leq N) \wedge\\ (p = \sum^i_{k=0} A[k] x^k) \wedge (y=X^i) &\rightarrow& (p+A[i] * i = \sum^{i+1}_{k=0}A[k]x^k) \wedge (y \times x = x^{i+1}) \wedge (0 \leq i+1 \leq N)\\
		(i \neq N) \wedge (0\leq i \leq N) \wedge \\(p = \sum^i_{k=0} A[k] x^k) \wedge (y=X^i) &\rightarrow& A[i] \times x^i = p+A[i+1]x^{i+1} \wedge x^i \times = x^{i+1} \wedge Tautologie \\
		(i \neq N) \wedge (0\leq i \leq N) \wedge \\(p = \sum^i_{k=0} A[k] x^k) \wedge (y=X^i) &\rightarrow& Faux \wedge Tautologie \wedge Tautologie\Rightarrow \textrm{Le programme est donc faux} \\
	\end{eqnarray*}
\item \begin{lstlisting}[language=C,numbers=none]
while(i == N) { 
++i;	
p = p +(A[i] * y * X); 
y = y * X; 
}
\end{lstlisting}
Ce programme effectue la correction de l'erreur détectée plus haut. Il est correct.
\end{enumerate}

\subsubsection{Exercice 2 : Suite de Fibonacci}
\remarque{
Le rang $n$ de la suite de Fibonacci est défini comme suit: \\
$F_n = 1$ si $n=0$ ou $n=1$\\
$F_n = F_n-2 + F_n-1$ si i$n > 1$}

\begin{lstlisting}[language=C,numbers=none]
/* $N \geq 0$ */
i = 0; 
a = 1; 
b = 1;
/* $INV = (F_i = a) \wedge (F_n-1 = b) \wedge (o \leq i \leq N)$ */
while(c) {
/* $c \neg INV$ */
i++;
b += a;
a = b - a;
/* $INV$ */
}
/* $\neg c \wedge INV$ */
/* $F_n = \{\cdots\}$ */
\end{lstlisting}

\remarque{Il est conseillé de commencer par l'étape la plus facile, en effet, une étape et nous n'avons pas à effectuer les autres}
\paragraph{Etape 1}
\begin{eqnarray*}
PE &\rightarrow& \pfp("init", INV)\\
N \geq 0 &\rightarrow& (F_{i+1} = 1) \wedge (F_i = 1) \wedge (i=0)\\
N \geq 0 &\rightarrow& (F_1 = 1) \wedge (F_0 = 1)\Rightarrow \textrm{Tautologie par définition}
\end{eqnarray*}
\paragraph{Etape 3}
\begin{eqnarray*}
(i = N) \wedge (F_i = a) \wedge (F_{i+1} = b) \wedge (0 \leq i \leq N) &\rightarrow& F_N = a\\
(F_n = a) \wedge (F_{N+1}) \wedge (0 \leq N) &\rightarrow& F_N = a \Rightarrow \textrm{Tautologie}
\end{eqnarray*}

\paragraph{Etape 2}
\begin{eqnarray*}
(i \neq N) \wedge (F_i = a) \wedge (F_{i+1} = b) \wedge (0 \leq i \leq N) &\rightarrow& \pfp("\cdots", F_i=b-a \wedge F_{i+1} = b \wedge \cdots)\\
&\rightarrow& \pfp ("\cdots", F_i = b+a - a \wedge F_{i+1} = b+a\\
&\rightarrow& F_i+1 = b \wedge F_{i+2} = b + a \wedge i+1 \leq N\\
&\rightarrow& T \wedge T \textrm{par définition} \wedge T 
\end{eqnarray*}

\paragraph{Etape 4}
Tester une existance de $f > 0$ avant l'execution du corps : $c \wedge INV \rightarrow f > 0$.
\begin{eqnarray*}
(i \neq  N) \wedge (F_i = a) \wedge (F_n-1 = b) \wedge (o \leq i \leq N) &\rightarrow& f > 0\\
 f = N-i \Rightarrow N \geq 0\textrm{ donc Vrai.}
\end{eqnarray*}

\paragraph{\'Etape 5}
\begin{eqnarray*}
f = F \wedge c \wedge INV &\rightarrow& \pfp("corps", f < F)\\
N-i = F \wedge c \wedge INV &\rightarrow& \pfp("corps", N-i < F)\\
N - i = F \wedge c \wedge INV &\rightarrow& N - (i+1) < F\\
&\rightarrow& N -i-1 < F \\
&\rightarrow& F-1 < F \Rightarrow \textrm{Tautologie}
\end{eqnarray*}

\subsubsection{Exercice 3 : plus grand plateau}
\remarque{$(B(N), \geq)$ signifique le tableau est ordonné dans l'ordre croissant non strict}

\subsubsection{Exercice 3 : Définition variante}
\begin{eqnarray*}
	C \wedge INV &\rightarrow& f > 0\\
	J \neq N \wedge INV &\rightarrow& (f = N-j) >0 \Rightarrow \textrm{ Vrai }
\end{eqnarray*}

\begin{eqnarray*}
	N-j  =F \wedge C \wedge INV &\rightarrow& \pfp(corps, N-j < F)\\
	N-j  =F \wedge C \wedge INV &\rightarrow& \pfp(\ifp(\cdots))\cdots,\pfp(j++,N-j<F)\\
	N-j  =F \wedge C \wedge INV &\rightarrow& B|j-p] = B[j] \rightarrow N - j -1 < F \wedge B[j-p] \neq B[j] \rightarrow N-j-1 < F\\
	N-j = F \wedge C \wedge INV \rightarrow N-j-1 < F \wedge F-1 < F
\end{eqnarray*}

\subsubsection{Exercice 4}
\begin{lstlisting}[language=C]
	/* $A \leq 0 \wedge B \geq 0 */
	x = A;
	y = B;
	z = 1;

	/* (z = x^y = A^B) \wedge y \geq 0$ */
	while(y != 0) {
		/* $z *x^y = A^B \wedge y \geq 0$ */
		while(y \% 2 == 0) {
			y = y / 2 ;
			x = x*x;
		}
		/* $z = x^y = A^B \wedge y > 0 \wedge y \%2 \neq 0 $*/
		y--;
		z = z * x;
		/* z = x^y = A^B \wedge y \geq 0 */
	}
	PS : /* z = A^b */

\end{lstlisting}

\section{\'Ecriture d'un programme à partir de sa preuve}
\subsection{Exercice 1 : Dichotomie}
\begin{lstlisting}[language=C]
/* N > 0 */
a = 1;
b = N;
while(b != a + 1) {
	m = (a+b)/2;
	if(0 <= N-m*m) {
		a = m;
	} else {
		b = m;
	}
}
/*
 * Réécriture : $a^2 \leq N \wedge N \leq b^2 \wedge (b=a+1)$
 * $a^2 \leq N \wedge N \leq (A+1)^2$  
 */
\end{lstlisting}
Variante : $b-a$ : taille de l'intervalle. Fonction décroissante.
\subsection{Exercice 2}

\begin{lstlisting}[language=C]
/* N $\leq$ 2 $\wedge$ B(0..N-1, $\leq$) $\wedge$ B[0] $\leq$ X $\wedge$ B[N-1] > X */
cherchep(B,N,X,p);
p = 0; 
/* 0 $\leq$ p $\leq$ i-2 $\wedge$ B(p) $\leq$ x $\wedge$ B[p+1] > x */ 
while (B[p+1] <= X) {
	++p;
}
/* 
 * 0 $\leq$ p $\leq$ N-2 $\wedge$ B[p] $\leq$ x < B[p+1] 
 */
\end{lstlisting}
Variante : $N - i$


	\addtocounter{chapter}{1}
	\addcontentsline{toc}{chapter}{\protect\numberline{\thechapter}Cours obtenu sur Moodle}
	\addcontentsline{toc}{section}{\protect\numberline{\thesection} Les Types Abstraits de Données}
	\addtocounter{section}{1}
	\includepdf[pages={1-}]{annexes/TAD1.pdf}
	\addcontentsline{toc}{section}{\protect\numberline{\thesection} Le TAD \texttt{File} et ses implémentations}
	\addtocounter{section}{1}
	\includepdf[pages={1-}]{annexes/TAD2-sh.pdf}
	\addcontentsline{toc}{section}{\protect\numberline{\thesection} Liste doublement chaînée en dynamique}
	\addtocounter{section}{1}
	\includepdf[pages={1-}]{annexes/TAD3-sh.pdf}
	\addcontentsline{toc}{section}{\protect\numberline{\thesection} Itérateur pour Liste doublement chainée en dynamique}
	\addtocounter{section}{1}
	\includepdf[pages={1-}]{annexes/TAD4-sh.pdf}
	\addcontentsline{toc}{section}{\protect\numberline{\thesection} Arbre binaire d'entiers en dynamique}
	\addtocounter{section}{1}
	\includepdf[pages={1-}]{annexes/TAD5-sh.pdf}
	\addcontentsline{toc}{section}{\protect\numberline{\thesection} Arbres rouges et noirs}
	\addtocounter{section}{1}
	\includepdf[pages={1-}]{annexes/TAD7-sh.pdf}
\end{document}
