\documentclass[a4paper, 11pt]{article}

\usepackage{xcolor}
\input{/home/aroquemaurel/cours/includesLaTeX/couleurs.tex}
\usepackage{lmodern}
\usepackage[utf8]{inputenc}
\usepackage[T1]{fontenc}
\usepackage[francais]{babel}
\usepackage[top=1.7cm, bottom=1.7cm, left=2.5cm, right=2.5cm]{geometry}
\usepackage{verbatim}
\usepackage{tikz} %Vectoriel
\usepackage{listings}
\usepackage{fancyhdr}
\usepackage{multido}
\usepackage{amssymb}
\usepackage{multicol}
\usepackage{float}
\usepackage[urlbordercolor={1 1 1}, linkbordercolor={1 1 1}, linkcolor=vert1, urlcolor=bleu, colorlinks=true]{hyperref}

\newcommand{\titre}{La \texttt{PileMixte}}
\newcommand{\numero}{1}
\newcommand{\typeDoc}{DM}
\newcommand{\module}{Structures de données}
\newcommand{\sigle}{sd}
\newcommand{\semestre}{4}
\newcommand{\pfp}{\texttt{pfp}}
	
\newcommand{\ifp}{\texttt{if}}
	
\newcommand{\elsep}{\texttt{else}}



\usepackage{ifthen}
\date{\today}

\chead{Antoine de \bsc{Roquemaurel}}
\rhead{TP\no\typeDoc}
\lhead{\titre}
%\makeindex

\lfoot{Université Toulouse III -- Paul Sabatier}
\rfoot{\sigle\semestre}
%\rfoot{}
\cfoot{--~~\thepage~~--}

\makeglossary
\makeatletter
\def\clap#1{\hbox to 0pt{\hss #1\hss}}%

\def\haut#1#2#3{%
	\hbox to \hsize{%
		\rlap{\vtop{\raggedright #1}
	}%
	\hss
	\clap{\vtop{\centering #2}
}%
\hss
\llap{\vtop{\raggedleft #3}}}}%
\def\bas#1#2#3{%
	\hbox to \hsize{%
		\rlap{\vbox{
			\raggedright #1
		}
	}%
	\hss \clap{\vbox{\centering #2}}%
	\hss
	\llap{\vbox{\raggedleft #3}}}
}%
\def\maketitle{%
	\thispagestyle{empty}{%
		\haut{}{\@blurb}{}
		%	
		%\vfill

		\begin{center}
			\vspace{-2.0cm}
			\usefont{OT1}{ptm}{m}{n}
			\huge \@type \@title
		\end{center}
		\par
		\hrule height 1pt
		\par
		\vspace{1cm}
		\bas{}{}{}
}%
}
\def\date#1{\def\@date{#1}}
\def\author#1{\def\@author{#1}}
\def\type#1{\def\@type{#1}}
\def\title#1{\def\@title{#1}}
\def\location#1{\def\@location{#1}}
\def\blurb#1{\def\@blurb{#1}}
\date{\today}
\newboolean{monBool}
\setboolean{monBool}{true}
\author{}
\title{}
\ifthenelse{\equal{\typeDoc}{}}{
\numeroTD{}
}
{
	\type{\typeDoc~--- }
}
\location{Amiens}\blurb{}
%\makeatother
\title{\titre}
\author{%Semestre \semestre
}

\location{Toulouse}
\blurb{%
\vspace{-35px}
\begin{flushleft}
	Université Toulouse III -- Paul Sabatier\\
	L2 Informatique\\
\end{flushleft}
\begin{flushright}
	\vspace{-45px}
	\Large \textbf \module \\
	\normalsize \textit \today\\
	Semestre \semestre
	\vspace{30px}
\end{flushright}
Antoine de \bsc{Roquemaurel}
}%



%\title{Cours \\ \titre}
%\date{\today\\ Semestre \semestre}

%\lhead{Cours: \titre}
%\chead{}
%\rhead{\thepage}

%\lfoot{Université Paul Sabatier Toulouse III}
%\cfoot{\thepage}
%\rfoot{\sigle\semestre}

\pagestyle{fancy}

\input{/home/aroquemaurel/cours/includesLaTeX/listings.tex} %prise en charge du langage C 




%----------------------------------------------------------------------------------------
%	DEFINITION OF COLORED BOXES
%----------------------------------------------------------------------------------------

\RequirePackage[framemethod=default]{mdframed} % Required for creating the theorem, definition, exercise and corollary boxes

% Theorem box
\newmdenv[skipabove=7pt,
skipbelow=7pt,
backgroundcolor=black!5,
linecolor=ocre,
innerleftmargin=5pt,
innerrightmargin=5pt,
innertopmargin=5pt,
leftmargin=0cm,
rightmargin=0cm,
innerbottommargin=5pt]{tBox}

% Exercise box	  
\newmdenv[skipabove=7pt,
skipbelow=7pt,
rightline=false,
leftline=true,
topline=false,
bottomline=false,
backgroundcolor=ocre!10,
linecolor=ocre,
innerleftmargin=5pt,
innerrightmargin=5pt,
innertopmargin=5pt,
innerbottommargin=5pt,
leftmargin=0cm,
rightmargin=0cm,
linewidth=4pt]{eBox}	

% Definition box
\newmdenv[skipabove=10pt,
skipbelow=10pt,
rightline=false,
leftline=true,
topline=false,
bottomline=false,
linecolor=ocre,
innerleftmargin=5pt,
innerrightmargin=5pt,
innertopmargin=0pt,
leftmargin=0cm,
rightmargin=0cm,
linewidth=4pt,
innerbottommargin=0pt]{dBox}	

% Corollary box
\newmdenv[skipabove=7pt,
skipbelow=7pt,
rightline=false,
leftline=true,
topline=false,
bottomline=false,
linecolor=gray,
backgroundcolor=black!5,
innerleftmargin=5pt,
innerrightmargin=5pt,
innertopmargin=5pt,
leftmargin=0cm,
rightmargin=0cm,
linewidth=4pt,
innerbottommargin=5pt]{cBox}		

% Corollary box
\newmdenv[skipabove=7pt,
skipbelow=7pt,
rightline=true,
leftline=false,
topline=false,
bottomline=true,
linecolor=gray,
backgroundcolor=black!5,
innerleftmargin=5pt,
innerrightmargin=5pt,
innertopmargin=5pt,
leftmargin=0cm,
rightmargin=0cm,
linewidth=1pt,
innerbottommargin=5pt]{rBox}				  
		  

% Creates an environment for each type of theorem and assigns it a theorem text style from the "Theorem Styles" section above and a colored box from above
\newenvironment{theorem}{\begin{tBox}\begin{theoremeT}}{\end{theoremeT}\end{tBox}}
\newenvironment{example}{\begin{exampleT}}{\hfill{\tiny\ensuremath{\blacksquare}}\end{exampleT}}
\newenvironment{definition}{\begin{dBox}\begin{definitionT}}{\end{definitionT}\end{dBox}}
\newenvironment{attention}{\begin{eBox}\small}{\end{eBox}}				  	
\newenvironment{exemple}{\begin{cBox}\small}{\end{cBox}}	

%----------------------------------------------------------------------------------------
%	REMARK ENVIRONMENT
%----------------------------------------------------------------------------------------

\newenvironment{remarque}{\par\vskip10pt\small
\begin{rBox}
\begin{list}{}{
\leftmargin=35pt % Indentation on the left
\rightmargin=25pt}\item\ignorespaces % Indentation on the right
\makebox[-2.5pt]{\begin{tikzpicture}[overlay]
\node[draw=ocre!60,line width=1pt,circle,fill=ocre!25,font=\sffamily\bfseries,inner sep=2pt,outer sep=0pt] at (-15pt,0pt){\textcolor{ocre}{R}};\end{tikzpicture}} % Orange R in a circle
\advance\baselineskip -1pt}
{\end{list}\vskip1mm\end{rBox}\vskip5pt} % Tighter line spacing and white space after remark



\input{/home/aroquemaurel/cours/includesLaTeX/polices.tex}
\input{/home/aroquemaurel/cours/includesLaTeX/affichageChapitre.tex}
\makeatother
\begin{document}
	\maketitle
	\section{Compilation}
	\begin{itemize}
		\item \texttt{make} Compile les différents fichiers et produit l'exécutable de tests
		\item \texttt{make clean} supprime les fichiers binaires (.o)
		\item \texttt{make all} Recompile le projet après avoir supprimés tous les fichiers binaires 
	\end{itemize}

	Les différents tests sont situés dans le fichier \texttt{./tests.c}dans 3 fonctions différentes. 
	\begin{itemize}
		\item \texttt{testPileStatique}
		\item \texttt{testPileMixte}
		\item \texttt{testPileDynamique}
	\end{itemize}
	Pour lancer les tests, il faut lancer le fichier executables \texttt{./tests}, ceux-ci afficheront <<Passed>> 
	si le test à réussis et <<Failed>> si celui-ci à échoué. (Si le test ne retourne pas la valeur attendue.). 

	Certains tests sont en commentaire, c'est parce que ce sont des tests qui fonctionnent dans un état anormal et doivent couper le programme à l'aide d'un
	\texttt{assert}. Afin d'éviter les problèmes dans le programme de tests, ceux-ci sont commentés mais peuvent être exécutés afin de vérifier que le programme s'arrête
	bien.

	Les tests effectués sont les même pour les trois piles, afin de pouvoir remarquer que la pile s'utilise de la même manière indépendamment de l'implémentation
	de la structure de donnée.

	\section{Petite modification au niveau de la spécification}
	Afin de pouvoir effectuer ce DM, j'ai dut modifier la spécification qui nous étais donné pour \texttt{empiler} et \texttt{depiler}, cette modification est présente dans les
	trois modules afin de garder une cohérence, bien que celle-ci n'était pas utile pour le module \texttt{pileStatique}. 

	En effet, la spécification demandée était la suivante:
\begin{lstlisting}[language=C, numbers=none]
void pileDynamiqueEmpiler(PileDynamique pPile, Element pElement);
Element pileDynamiqueDepiler(PileDynamique pPile);
\end{lstlisting}

	Cependant, lorsque l'ont empile ou dépile, le pointeur de début de pile est modifié, ainsi en gardant la spécification demandée, la pile ne pouvait pas etre
	modifiée lorsque l'on empilait ou dépilait.

	Deux solutions se présentaient à moi pour palier à ce problème :
	\begin{itemize}
		\item Changer le prototype afin de retourner une nouvelle Pile
		\item Changer le premier paramètre afin d'y faire passer la pile via un pointeur
	\end{itemize}

	Afin de garder une cohérence avec les autres fonctions du module, j'ai choisis la seconde solution, ainsi les prototypes des fonction \texttt{empiler} et
	\texttt{depiler} sont devenus ceux-ci : 
\begin{lstlisting}[language=C, numbers=none]
void pileDynamiqueEmpiler(PileDynamique* pPile, Element pElement);
Element pileDynamiqueDepiler(PileDynamique* pPile);
\end{lstlisting}
\section{État incohérent du programme}
	Afin d'être sûr que le programme n'entre pas dans un état incohérent, en cas de problème le programme s'arrête à l'aide de assert. En effet, je pars du
	principe que l'utilisateur de la pile connait les pré-conditions et ne doit pas essayer de dépiler une pile vide ou d'empiler si la pile est pleine. 
	Si cela se produit, le programme s'arrêtera et il se rencontrera de lui même qu'il à effectué une erreur.

	De plus, la fonction \texttt{depiler} doit renvoyer l'élément courant, cependant ci la pile est vide, renvoyer la valeur est impossible.

\section{Exemple d'exécution des tests}
\begin{lstlisting}[language=sh, basicstyle=\scriptsize\ttfamily, caption=Exemple d'execution des tests]
aroquemaurel@Garp:~/cours/L2/sd4/DM
$\$$ ./tests
[Passed] PileStatique  Création de la pile
[Passed] PileStatique  Sommet
[Passed] PileStatique  Empiler un élément
[Passed] PileStatique  Depiler un élément
[Passed] PileStatique  La pile est vide
[Passed] PileStatique  La pile est pleine
[Passed] PileStatique  On n'empile pas si la pile est pleine
[Passed] PileStatique  La pile est vide

[Passed] PileMixte Création de la pile
[Passed] PileMixte Sommet
[Passed] PileMixte Empiler un élément
[Passed] PileMixte Depiler un élément
[Passed] PileMixte La pile est vide
[Passed] PileMixte Pile est pleine
[Passed] PileMixte On n'empile pas si la pile est pleine
[Passed] PileMixte La pile est vide
[Passed] PileMixte On ne dépile pas si la pile est vide

[Passed] PileDynamique Création de la pile
[Passed] PileDynamique Sommet
[Passed] PileDynamique Empiler un élément
[Passed] PileDynamique Depiler un élément
[Passed] PileDynamique La pile est vide
[Passed] PileDynamique Pile est pleine
[Passed] PileDynamique On n'empile pas si la pile est pleine
[Passed] PileDynamique La pile est vide
[Passed] PileDynamique On ne dépile pas si la pile est vide
\end{lstlisting}

\end{document}
