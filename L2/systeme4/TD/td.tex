\documentclass[12pt,a4paper,openany]{book}


\usepackage{lmodern}
\usepackage{xcolor}
\input{/home/aroquemaurel/cours/includesLaTeX/couleurs.tex}

\usepackage[utf8]{inputenc}
\usepackage[T1]{fontenc}
\usepackage[francais]{babel}
\usepackage[top=1.7cm, bottom=1.7cm, left=1.7cm, right=1.7cm]{geometry}
\usepackage{verbatim}
\usepackage[urlbordercolor={1 1 1}, linkbordercolor={1 1 1}, linkcolor=vert1, urlcolor=bleu, colorlinks=true]{hyperref}
\usepackage{tikz} %Vectoriel
\usepackage{listings}
\usepackage{fancyhdr}
\usepackage{multido}
\usepackage{amssymb}

\newcommand{\titre}{Systèmes 2 --- TD}

\newcommand{\pole}{}
\newcommand{\sigle}{systèmes}

\newcommand{\semestre}{4}

\input{/home/aroquemaurel/cours/includesLaTeX/listings.tex} %prise en charge du langage algo
\input{/home/aroquemaurel/cours/includesLaTeX/entete-l2-cours.tex}




%----------------------------------------------------------------------------------------
%	DEFINITION OF COLORED BOXES
%----------------------------------------------------------------------------------------

\RequirePackage[framemethod=default]{mdframed} % Required for creating the theorem, definition, exercise and corollary boxes

% Theorem box
\newmdenv[skipabove=7pt,
skipbelow=7pt,
backgroundcolor=black!5,
linecolor=ocre,
innerleftmargin=5pt,
innerrightmargin=5pt,
innertopmargin=5pt,
leftmargin=0cm,
rightmargin=0cm,
innerbottommargin=5pt]{tBox}

% Exercise box	  
\newmdenv[skipabove=7pt,
skipbelow=7pt,
rightline=false,
leftline=true,
topline=false,
bottomline=false,
backgroundcolor=ocre!10,
linecolor=ocre,
innerleftmargin=5pt,
innerrightmargin=5pt,
innertopmargin=5pt,
innerbottommargin=5pt,
leftmargin=0cm,
rightmargin=0cm,
linewidth=4pt]{eBox}	

% Definition box
\newmdenv[skipabove=10pt,
skipbelow=10pt,
rightline=false,
leftline=true,
topline=false,
bottomline=false,
linecolor=ocre,
innerleftmargin=5pt,
innerrightmargin=5pt,
innertopmargin=0pt,
leftmargin=0cm,
rightmargin=0cm,
linewidth=4pt,
innerbottommargin=0pt]{dBox}	

% Corollary box
\newmdenv[skipabove=7pt,
skipbelow=7pt,
rightline=false,
leftline=true,
topline=false,
bottomline=false,
linecolor=gray,
backgroundcolor=black!5,
innerleftmargin=5pt,
innerrightmargin=5pt,
innertopmargin=5pt,
leftmargin=0cm,
rightmargin=0cm,
linewidth=4pt,
innerbottommargin=5pt]{cBox}		

% Corollary box
\newmdenv[skipabove=7pt,
skipbelow=7pt,
rightline=true,
leftline=false,
topline=false,
bottomline=true,
linecolor=gray,
backgroundcolor=black!5,
innerleftmargin=5pt,
innerrightmargin=5pt,
innertopmargin=5pt,
leftmargin=0cm,
rightmargin=0cm,
linewidth=1pt,
innerbottommargin=5pt]{rBox}				  
		  

% Creates an environment for each type of theorem and assigns it a theorem text style from the "Theorem Styles" section above and a colored box from above
\newenvironment{theorem}{\begin{tBox}\begin{theoremeT}}{\end{theoremeT}\end{tBox}}
\newenvironment{example}{\begin{exampleT}}{\hfill{\tiny\ensuremath{\blacksquare}}\end{exampleT}}
\newenvironment{definition}{\begin{dBox}\begin{definitionT}}{\end{definitionT}\end{dBox}}
\newenvironment{attention}{\begin{eBox}\small}{\end{eBox}}				  	
\newenvironment{exemple}{\begin{cBox}\small}{\end{cBox}}	

%----------------------------------------------------------------------------------------
%	REMARK ENVIRONMENT
%----------------------------------------------------------------------------------------

\newenvironment{remarque}{\par\vskip10pt\small
\begin{rBox}
\begin{list}{}{
\leftmargin=35pt % Indentation on the left
\rightmargin=25pt}\item\ignorespaces % Indentation on the right
\makebox[-2.5pt]{\begin{tikzpicture}[overlay]
\node[draw=ocre!60,line width=1pt,circle,fill=ocre!25,font=\sffamily\bfseries,inner sep=2pt,outer sep=0pt] at (-15pt,0pt){\textcolor{ocre}{R}};\end{tikzpicture}} % Orange R in a circle
\advance\baselineskip -1pt}
{\end{list}\vskip1mm\end{rBox}\vskip5pt} % Tighter line spacing and white space after remark



\input{/home/aroquemaurel/cours/includesLaTeX/polices.tex}
\input{/home/aroquemaurel/cours/includesLaTeX/affichageChapitre.tex}

\begin{document}
	\setcounter{tocdepth}{2}
	\setcounter{secnumdepth}{3}
	\maketitle
	\tableofcontents
	\chapter{Introduction}
		\section{Exercice 1}
		\lstinputlisting[language=C, caption=Exercice 1 -- Version portable]{codes/1.c}
		\lstinputlisting[language=C, caption=Exercice 1 -- Version Unix]{codes/1-2.c}
		\section{Exercice 2}
		\lstinputlisting[language=C, caption=Exercice 2]{codes/2-2.c}
		\lstinputlisting[language=C, caption=Exercice 2 -- Correction]{codes/2.c}
	\chapter{Processus}
		\section{Exercice 3}
		\lstinputlisting[language=C, caption=Exercice 3]{codes/3.c}
		\section{Exercice 4}
		\lstinputlisting[language=C, caption=Exercice 4]{codes/4.c}
		\section{Exercice 5}
\begin{lstlisting}[language=C, numbers=none]
execlp("date", "date", NULL);
\end{lstlisting}
\begin{lstlisting}[language=C, numbers=none]
	Maintenant: 
	vendredi 8 février 2013, 15:13:33 (UTC+0100)
\end{lstlisting}
		\section{Exercice 6}
		Il va afficher deux fois le pid du programme courant.
		\section{Exercice 7}
		\lstinputlisting[language=C, caption=Exercice 7]{codes/7.c}
	\chapter{Gestion de la mémoire: mémoire virtuelle et allocation non contiguë}
		\section{Exercice 8}
		\subsection{}
		\begin{itemize}
			\item \textbf{Pages 2 et 3} $[5 \times 1024 ; 6 \times 1024 -1] = [2048 ; 4095]$ 
			\item \textbf{Page 5} $[7 \times 1024 ; 8 \times 1024 -1] = [5120 ; 6143]$
			\item \textbf{Page 7} $[2 \times 1024 ; 4 \times 1024 -1] = [7168 ; 8191]$
		\end{itemize}
		\subsection{}
		\remarque{Numéro de cadre $\times$ 1024 + décalage}
		\begin{tabular}{c|c|c|c}
			\textbf{Adresse virtuelle} & \textbf{Page virtuelle} & \textbf{Page physique} & \textbf{Adresse physique}\\
			\hline
			0 &  0 & 3 & $3\times 1024=3072$\\
			3728 & 3 & défaut de page & défaut de page\\
			1023 & 0 & 3 & $3 \times 1024 + 1023 = 4095$\\ 
			1024 & 1 & 1 & 1024\\
			1025 & 1 & 1 & 1025\\
			7800 & 7 & défaut de page & défaut de page\\
			4096 & 4 & 2 & 2048
		\end{tabular}
		\section{Exercice 9}
		d: taux de défaut de page.

		\subsection{Temps d'accès moyen}
		$$(1-d) \times 2 \times 0.5 + d \times (20 + 2 \times 0.5) = (1-d)+21d = 1+20d$$
		\subsection{Taux maximal de défaut de page}
		\begin{eqnarray*}
			1 + 20d &<& 1.2\\
			d &<& \frac{0.2}{20}\\
			d &<& 0.01
		\end{eqnarray*}
		\section{Exercice 10}
		\subsection{}
			Chaque processus provoque 5 défauts de page correspondants aux changements initiaux des 5 pages.

			Soit $n$ le nombre de page référencancés par un processus. Le taux de défaut de page est $\frac{5}{n}$.
	\chapter{Structure interne du système de fichier d'Unix}
	\chapter{Primitives Unix (POSIX.1) de manipulation des fichiers}
		\lstinputlisting[language=C, caption=Exercie 16]{codes/16.c}
\end{document}


