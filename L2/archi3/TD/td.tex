\documentclass[12pt,a4paper,openany]{book}


\usepackage{lmodern}
\usepackage{xcolor}
\input{/home/satenske/cours/includesLaTeX/couleurs.tex}

\usepackage[utf8]{inputenc}
\usepackage[T1]{fontenc}
\usepackage[francais]{babel}
\usepackage[top=1.7cm, bottom=1.7cm, left=1.7cm, right=1.7cm]{geometry}
\usepackage{verbatim}
\usepackage[urlbordercolor={1 1 1}, linkbordercolor={1 1 1}, linkcolor=vert1, urlcolor=bleu, colorlinks=true]{hyperref}
\usepackage{tikz} %Vectoriel
\usepackage{listings}
\usepackage{fancyhdr}
\usepackage{multido}
\usepackage{amssymb}

\newcommand{\titre}{Architecture des systèmes Informatiques --- TD}

\newcommand{\pole}{}
\newcommand{\sigle}{archi}

\newcommand{\semestre}{3}

\input{/home/satenske/cours/includesLaTeX/listings.tex} %prise en charge du langage algo
\input{/home/aroquemaurel/cours/includesLaTeX/entete-l2-cours.tex}




%----------------------------------------------------------------------------------------
%	DEFINITION OF COLORED BOXES
%----------------------------------------------------------------------------------------

\RequirePackage[framemethod=default]{mdframed} % Required for creating the theorem, definition, exercise and corollary boxes

% Theorem box
\newmdenv[skipabove=7pt,
skipbelow=7pt,
backgroundcolor=black!5,
linecolor=ocre,
innerleftmargin=5pt,
innerrightmargin=5pt,
innertopmargin=5pt,
leftmargin=0cm,
rightmargin=0cm,
innerbottommargin=5pt]{tBox}

% Exercise box	  
\newmdenv[skipabove=7pt,
skipbelow=7pt,
rightline=false,
leftline=true,
topline=false,
bottomline=false,
backgroundcolor=ocre!10,
linecolor=ocre,
innerleftmargin=5pt,
innerrightmargin=5pt,
innertopmargin=5pt,
innerbottommargin=5pt,
leftmargin=0cm,
rightmargin=0cm,
linewidth=4pt]{eBox}	

% Definition box
\newmdenv[skipabove=10pt,
skipbelow=10pt,
rightline=false,
leftline=true,
topline=false,
bottomline=false,
linecolor=ocre,
innerleftmargin=5pt,
innerrightmargin=5pt,
innertopmargin=0pt,
leftmargin=0cm,
rightmargin=0cm,
linewidth=4pt,
innerbottommargin=0pt]{dBox}	

% Corollary box
\newmdenv[skipabove=7pt,
skipbelow=7pt,
rightline=false,
leftline=true,
topline=false,
bottomline=false,
linecolor=gray,
backgroundcolor=black!5,
innerleftmargin=5pt,
innerrightmargin=5pt,
innertopmargin=5pt,
leftmargin=0cm,
rightmargin=0cm,
linewidth=4pt,
innerbottommargin=5pt]{cBox}		

% Corollary box
\newmdenv[skipabove=7pt,
skipbelow=7pt,
rightline=true,
leftline=false,
topline=false,
bottomline=true,
linecolor=gray,
backgroundcolor=black!5,
innerleftmargin=5pt,
innerrightmargin=5pt,
innertopmargin=5pt,
leftmargin=0cm,
rightmargin=0cm,
linewidth=1pt,
innerbottommargin=5pt]{rBox}				  
		  

% Creates an environment for each type of theorem and assigns it a theorem text style from the "Theorem Styles" section above and a colored box from above
\newenvironment{theorem}{\begin{tBox}\begin{theoremeT}}{\end{theoremeT}\end{tBox}}
\newenvironment{example}{\begin{exampleT}}{\hfill{\tiny\ensuremath{\blacksquare}}\end{exampleT}}
\newenvironment{definition}{\begin{dBox}\begin{definitionT}}{\end{definitionT}\end{dBox}}
\newenvironment{attention}{\begin{eBox}\small}{\end{eBox}}				  	
\newenvironment{exemple}{\begin{cBox}\small}{\end{cBox}}	

%----------------------------------------------------------------------------------------
%	REMARK ENVIRONMENT
%----------------------------------------------------------------------------------------

\newenvironment{remarque}{\par\vskip10pt\small
\begin{rBox}
\begin{list}{}{
\leftmargin=35pt % Indentation on the left
\rightmargin=25pt}\item\ignorespaces % Indentation on the right
\makebox[-2.5pt]{\begin{tikzpicture}[overlay]
\node[draw=ocre!60,line width=1pt,circle,fill=ocre!25,font=\sffamily\bfseries,inner sep=2pt,outer sep=0pt] at (-15pt,0pt){\textcolor{ocre}{R}};\end{tikzpicture}} % Orange R in a circle
\advance\baselineskip -1pt}
{\end{list}\vskip1mm\end{rBox}\vskip5pt} % Tighter line spacing and white space after remark



\input{/home/satenske/cours/includesLaTeX/polices.tex}
\input{/home/satenske/cours/includesLaTeX/affichageChapitre.tex}

\begin{document}
	\setcounter{tocdepth}{2}
	\setcounter{secnumdepth}{3}
	\maketitle
	\tableofcontents
	\chapter{Numérotation et codage\\ TD optionnel}
		\section{Numérotation}
		\subsection{Réaliser l'opération suivante en binaire : $(1101011-10110)\times 11001$}
		\begin{eqnarray*}
			(110\;1011)_2 &=& (107)_{10}\\
			(1\;0110)_2&=&(22)_{10}\\
			(1\;1001)_2&=&(22)_{10}
		\end{eqnarray*}
		\begin{center}
		\begin{tabular}{cccccccc}
			&1&1&0&1&0&1&1\\
			$-$&$$&&1&0&1&1&0\\
			\hline 
			&1&0&1&0&1&0&1\\
		\end{tabular} Ou 
		\begin{tabular}{cccccccc}
			&1&1&0&1&0&1&1\\
			+&1&1&1&0&1&0&1\\
			\hline
			&1&0&1&0&1&0&1
		\end{tabular}
		\begin{eqnarray*}
			(1010101)_2&=&(85)_{10}
		\end{eqnarray*}
		\begin{tabular}{cccccccccccccc}
			&&&&&&1&0&1&0&1&0&1&\\
			&&&&&$\times$&&1&1&0&0&1&\\
			\hline
			&&&&&&1&0&1&0&1&0&1&\\
			&&&1&0&1&0&1&0&1&.&.&.&\\
			&&1&0&1&0&1&0&1&.&.&.&.&\\
			\hline
			1&0&0&0&0&0&1&0&0&1&1&0&1&
		\end{tabular}
	\end{center}
		\subsection{Réaliser les opérations suivantes en hexadécimal : $(389A+7293)-EB2$}
		\begin{center}
			
		\begin{tabular}{ccccc}
			&3&8&9&A\\
			+&7&2&9&3\\
			\hline
			&A&B&2&B\\
			$-$&E&B&2&B\\
			\hline
			&9&C&7&B
		\end{tabular}
		\begin{eqnarray*}
			(389A)_{16} &=& (0011\; 1000\; 1001\; 1010)_2\\
			(7293)_{16} &=& (0111\; 0010\; 1001\; 0011)_2\\
			(AB2B)_{16} &=& (1010\;1011\;0010\;1101)_2\\
			(EB2B)_{16} &=& (0000\;1110\;1011\;0010)_2\\
			(9C7B)_{16} &=& (1001\;1100\;0111\;1011)_2
		\end{eqnarray*}
		\end{center}

		\subsection{Effectuer les conversions ci-dessous}
			\subsubsection{$(1447.140625)_{10} = (??)_2 = (??)_{16}$}
			\begin{eqnarray*}
				1447\div 16 &=& 90\; R=7\\
				90 \div 16 &=&~5\; R=A\\
				5 \div 16 &=&  ~0 \; R=5\\ 
				(1447)_{10} &=& (5A7)_{16}\\\\
				0.140625 \times 16 &=& 2.25\\
				0.25 \times 16 &=& 4.00\\
				(0.140625)_{10} &=& (0.24)_{16}\\\\
				(1447.140625)_{10} = (5A7.24)_{16} &=& (0101\;1010\;0111.0010\;0100)_2
			\end{eqnarray*}
			\subsubsection{$(1111100101.01011)_{2} = (??)_{10} = (??)_{16}$}
			\begin{eqnarray*}
				(11\;1110\;0101.0\;1011)_2 &=& (3E5;58)_{16}\\
				3E5 &=& (3\times 16^2 + 14*16 + 5 + 5 \times 16^{-1} + 8 \times 16 ^{-2}\\
				&=& (997,34375)_{10}
			\end{eqnarray*}
		\section{Codage}
		\subsection{Codage de nombres entiers relatifs}
		On veut coder des entiers relatifs sur 16 chifffres binaires (deux octets).

		\begin{tabular}{| p{1.8cm}|p{4cm}|p{2cm}|p{3.7cm}|p{3.7cm}|}
			\hline
			& \multicolumn{2}{ c|}{\textbf{Valeur absolue}}  & \multicolumn{2}{ c |}{\textbf{Valeur relative}}\\
			\hline
			\textbf{Base 10} & \textbf{Base 2} & \textbf{Base 16} & \textbf{Valeur absolue + signe} & \textbf{Complément à 2}\\
			\hline
			$\ \ 35671$&$10000\;010\;0101\;0111$& 75B8 & Hors intervalle & Hors intervalle\\
			\hline
			$-32768$ & $1000\; 0000\; 0000\; 0000$ & $8000$ & Hors intervalle & $1000\; 0000\; 0000$\\
			\hline
			$\ \ 46443$& $1011\; 0101\; 0110\; 1011$ & B56B & Hors intervalle & Hors intervalle\\
			\hline
			$-19536$& $0100\; 1100\; 0110\;0100$ & 4C64 & $1100\; 1100\; 0110\; 0100$ & $1011\; 0011\; 1001\; 1100$\\
			\hline
			$-19040$ & $0100\; 1010\; 0110\; 0000$ & 4A60 & $1100\; 1010\; 0110\; 0000$ & $1011\; 0101\; 1010\; 0000$\\
			\hline
		\end{tabular}
		\subsubsection{Calcul de la valeur absolue en base 2}
		\begin{eqnarray*}
			3567 \div 16 &=& 229 \textrm{ et reste 7}\\
			2229 \div 16 &=& 139 \textrm{ et reste 5}\\
			139 \div 16 &=& 8 \textrm{ et reste B}\\
			8 \div 16 &=&0 \textrm{ et reste 8}\\
		\end{eqnarray*}
		$$\Rightarrow (35671)_{10} = (8B57)_{16} = (1000\; 1010\; 0101\; 0111)_2$$
		\remarque{
		Afin de pouvoir représenter un nombre celui-ci ne doit pas dépasser un certain intervalle : 

\begin{description}
	\item[Entier naturel] $[0;2^{16-1}] = [0;65535]$
	\item[Valeur absolue + signe] $[-2^{15-1};2^{15-1}] = [-16384 ; 16384]$
	\item[Complément à deux] $[-2^{15};2^{15}] = [-32768 ; 32768]$
\end{description}
		}
		\subsection{Convertir un nombre flottant en décimal}
		\begin{center}
			\begin{tabular}{|c|c|c|}
				\hline
				$0$ & $1000\; 0001$ & $1110\; 0000\; 0000\; 0000\; 0000\; 0000$\\
				\hline
			\end{tabular}
		\end{center}
	$$C = E + biais$$
	En simple précision biais = 127.

	\begin{eqnarray*}
		S &=& 0 \rightarrow \textrm{ positif}\\
		C &=& 129\\
		E &=& C - 127 = 2\\
		1.M &=& 1.111 \Rightarrow 1.111 \times 2^2 = 111.1 \times 2^0 = (7.5)_{10}\\
	\end{eqnarray*}
	\newpage	
	\subsection{Convertir un nombre décimal en flottant}
	\begin{eqnarray*}
		(35.5)_{10} &=& ? \\
		100011.1 &=& 10000111 \times 2^5\\
		\textrm{Nombre positif donc } S &=& 0\\
		E &=& 5\\
		C &=& E + 127 = 132 = 128+4 = (10000100)_2\\
		1.M &=& 1.00011\\
		M &=& 0011\\
	\end{eqnarray*}
	
		\begin{center}
			\begin{tabular}{|c|c|c|}
				\hline
				$0$ & $1000\; 0001$ & $0100\; 0011\; 0000\; 0000\; 0000\; 0000$\\
				\hline
			\end{tabular}
		\end{center}
		

		\chapter{Algèbre de Bool}
	\section{Table de vérité des opérateurs classiques}
	\subsection{Exercice 1}
		\subsubsection{A -- Démontrer que les opérateurs NAND et NOR sont des opérateurs complets}
		\begin{eqnarray*}
			\overline A &=& A|A\\
			A.B &=& \overline{\overline{A.B}} = \overline{\overline A + \overline B} = \overline{A/B} = (A | B) | (A | B)\\
			A+B &=& \overline{\overline{A+B}} = \overline{\overline A + \overline B} = (A|B) |(B |B)
		\end{eqnarray*}
		\begin{eqnarray*}
			\overline A = \overline{A+A} = A \downarrow A\\
			A . B = \overline{\overline{A.B}} = \overline{\overline A+ \overline B} = A \downarrow B \downarrow (\downarrow B \downarrow B)
		\end{eqnarray*}
		\subsubsection{B --}
		\begin{enumerate}
			\item $f(A,B,C,D)=\overline A B \overline D + B\overline C + A\overline C D= \overline{\overline{\overline A B \overline D + B \overline C + A \overline C D}} 
				= \overline{(\overline{\overline B\overline D).(\overline{A\overline C D)}}} =\\ ( (A|A)|B( (D|D))) | (B|(C|C)) | (A|(C|C)|D)$
			\item $f(A,B,C,D)=(A+B)(\overline C + \overline D)(\overline A + \overline B + \overline C) = \overline{\overline{(A+B)(\overline C + \overline B)(\overline A + \overline B + \overline C)}}=\\
				\overline{\overline A \overline B) + CD + ABC} = \overline{\overline A \overline B} . \overline{CD} . \overline{ABC} =\\
				((A|A)(B|B)|(C|D)|(A|B|C))|((A|A)(B|B)|(C|D)|(A|B|C)) =\\
				((A\downarrow B)\downarrow (C \downarrow C) \downarrow (D \downarrow D) \downarrow (A \downarrow A)\downarrow (B\downarrow B) \downarrow (C \downarrow C))$
		\end{enumerate}
	\subsection{Exercice 3}
	\subsubsection{1}
	$f(w,x,y,z) = \sum n(1,3,4,7,9,11,12,15) = yz+\overline x z + x\overline y \overline z$
	\\
	\begin{tabular}{c|c|c|c|c|}
		& 00&01&11&10\\
		\hline
		00&&1&1&\\
		\hline
		01&1&&1&\\
		\hline
		11&1&&1&\\
		\hline
		10&&1&1&\\
		\hline
	\end{tabular}
	\subsubsection{2}
	$f(w,x,y,z) = \sum m(0,1,3,6,9,13,15) = \overline w \overline x \overline y + \overline w \overline x z + wxz + w\overline y z+w\overline x y \overline z$\\
	\begin{tabular}{c|c|c|c|c|}
		& 00&01&11&10\\
		\hline
		00&1&1&1&\\
		\hline
		01&&&&1\\
		\hline
		11&&1&1&\\
		\hline
		10&&1&&\\
		\hline
	\end{tabular}
	\subsubsection{3}
	$f(w,x,y,z) = \sum m (0,1,5,7,8,10,14,15) = \overline w \overline x \overline y +  \overline w x z + wxy + w\overline x \overline  z$\\
	\begin{tabular}{c|c|c|c|c|}
		& 00&01&11&10\\
		\hline
		00&1&1&&\\
		\hline
		01&&1&1&\\
		\hline
		11&&&1&1\\
		\hline
		10&1&&&1\\
		\hline
	\end{tabular}

	\subsection{4}
	\begin{tabular}{c|c|c|c|c|}
		& 00&01&11&10\\
		\hline
	\end{tabular}
	\subsection{5}
	\begin{tabular}{c|c|c|c|c|}
		& 00&01&11&10\\
		\hline
	\end{tabular}
	\subsection{9}
	$f(w,x,y,z) = \Pi M(1,3,4,9,11,14) CI(w,x,y,z) = (x+\overline z)(\overline x + z)$\\
	\begin{tabular}{c|c|c|c|c|}
		& 00&01&11&10\\
		\hline
		00&&0&0&\\
		\hline
		01&0&&&*\\
		\hline
		11&*&*&&0\\
		\hline
		10&&0&0&\\
		\hline
	\end{tabular}
	\subsection{11}
	$f(w,x,y,z) = \sum m(0,1,3,5,6,7,11,13,14,15) CI(w,x,y,z) = 4\\
	f(w,x,y,z) = \overline y \overline w + yz + xy +xz$\\
	\begin{tabular}{c|c|c|c|c|}
		& 00&01&11&10\\
		\hline
		00&1&1&1&\\
		\hline
		01&*&1&1&1\\
		\hline
		11&&1&1&1\\
		\hline
		10&&&1&\\
		\hline
	\end{tabular}

	\chapter{Fonctions logiques}
	\section{Exercice 1 : Simplifications algébriques}
	\subsection{}
	\begin{tabular}{ccc||c}
		x&y&z & $F_1$\\
		0&0&0&	0\\
		0&0&1& 	1\\
		0&1&0& 	1\\
		0&1&1& 	1\\
		1&0&0& 	1\\
		1&0&1& 	1\\
		1&1&0& 	1\\
		1&1&1& 	1\\
	\end{tabular}
	\begin{eqnarray*}
		F_1 &=& (x+y+z).(\overline x + \overline y + z + xy + \overline x \overline y)\\
		&=& (x+y+z).(\overline{\overline x} + \overline{\overline y} + z + xy + \overline{\overline x \overline y})\\
		&=& (x+y+z)(\overline x + \overline y (1 + \overline x) + z +xy)\\
		&=& (x+y+z)(\overline x + \overline y + z + xy)\\
		&=& (x+y+z)(\overline{xy} + xy + z)\\
		&=& (x+y+z)(1+z)\\
		F_1 &=& x+y+z
	\end{eqnarray*}
	\subsection{}
	\begin{eqnarray*}
		F_2 &=& \sum m(0,4,6,7,14,15)\\
		&=& \overline x \overline y \overline z \overline w + \overline x y \overline z \overline w + \overline x y z \overline w + \overline x yzw + xyz\overline w + xyzw\\
		&=& \overline x \overline z \overline w(y+\overline y) +\overline x yz (w+\overline w) + xyz(w + \overline w)\\
		&=& \overline x \overline z \overline w + \overline x y z + xyz\\
		&=& \overline x \overline z \overline w + xy(x + \overline x)\\
		F_2 &=& \overline x \overline z \overline w + xy
	\end{eqnarray*}
	\section{Exercice 2 : Formes canoniques}
	\subsection{}
	$G_1 = I_1 + I_2$ avec $I_1 = \sum m (0,4,6)$ et $I_2 = \Pi M(1,4,5)$
	\begin{eqnarray*}
		G_1 &=& \sum m(0,4,6) + \Pi M(1,4,5)\\
		&=& \sum m(0,4,6) + \sum m (0,2,3,6,7)\\
		G_1 &=& \sum m (0,2,3,4,6,7) \Rightarrow \textrm{ Forme canonique Disjonctive}\\
		G_1 &=& \Pi M (1,5) \Rightarrow \textrm{ Forme Canonique Conjonctive}
	\end{eqnarray*}
	
	\chapter{Les circuits combinatoires}
	\section{Exercice 1}
	\subsection{Encodeur de priorité}
	\begin{tabular}{cccc||cc}
		$E_3$ & $E_2$ & $E_1$ & $E_0$ & $S_1$ & $S_0$\\
		\hline
		0&0&0&1&0&0\\
		0&0&1&*&0&1\\
		0&1&*&*&0&1\\
		1&*&*&*&1&1\\
	\end{tabular}\\
	\begin{eqnarray*}
	S_1&=& E_3E_2+E_3=E_3+E_2\\
	S_0 &=& E_3E_2E_1+E_3 = E_3+\overline{E_2} E_1
	\end{eqnarray*}
	\remarque{Théorème d'absorption}
	\subsection{Comparateur 4 bits}
	\subsubsection{Comparateur 1 bit}
		\begin{tabular}{cc||ccc}
			A & B & S & I & E\\
			\hline
			0&0&0&0&1\\
			0&1&0&1&0\\
			1&0&1&0&0\\
			1&1&0&0&1
		\end{tabular}
		\begin{eqnarray*}
			S&=& a\overline b\\
			I&=& \overline a b\\
			E&=& \overline a \overline b + ab = a \odot b
		\end{eqnarray*}
		\remarque{ $$a \overline \oplus b \Leftrightarrow a \odot b$$ }
	\subsubsection{Comparateur 2 bits}% a_1a_0=>A b_1b_0=>B
	$$ \underbrace{a_1a_0}_A + \underbrace{b_1b_0}_B$$
	\begin{eqnarray*}
		E &=&  (a_1 \odot b_1)(a_0 \odot b_O)\\
		S &=&  a_1 \overline{b_1} + a_0\overline{b_0}(a_1 \odot b_1)\\
		I &=& \overline{a_1}b_1 + \overline{a_0}b_0(a_1 \odot b_1)
	\end{eqnarray*}
	\subsubsection{Comparateur 4 bits -- Généralisation}
	\begin{eqnarray*}
		E &=&  (a_3 \odot b_3)(a_2 \odot b_2)(a_1 \odot b_1)(a_0 \odot b_0)\\
		S &=& a_3\overline{b_3}+a_2\overline{b_2}(a_3 \odot b_3) + a_1 \overline{b_1} (a_3 \odot b_3)(a_2 \odot b_2) + a_0
		\overline{b_0}(a_3\odot b_3)(a_2\odot b_2)(a_1 \odot b_1)\\
		I &=&  \overline{a_3}b_3 + \overline{a_2}b_2(a_3 \odot b_3) + \overline{a_1} b_1 (a_3 \odot b_3)(a_2 \odot b_3) +
		\overline{a_1}b_1 (a_3 \odot b_3)(a_2 \odot b_2) +\\&& \overline{a_0}b_0 (a_3 \odot b_3)(a_2 \odot b_2)(a_1 \odot b_1)
	\end{eqnarray*}

	\section{Exercice 2}
	\begin{tabular}{cc||cccc}
		$C_1$&$C_0$&$S_3$&$S_2$&$S_1$&$S_0$\\
		\hline
		0&0&$E_3$&$E_2$&$E_1$&$E_0$\\
		0&1&0&$E_3$&$E_2$&$E_1$\\
		1&0&0&0&$E_3$&$E_2$\\
		1&1&0&0&0&$E_3$
	\end{tabular}
	\begin{eqnarray*}
		S_3 &=&  \overline{C_1}\overline{C_0}E_3\\
		S_2 &=&  \overline C_1 \overline C_0 E_2 + \overline{C_1} C_0 E_3\\
		S_1 &=&  \overline{C_1} \overline{C_0}E_2 + C_1\overline{C_0}E_3\\
		S_0 &=& \overline{C_1} \overline{C_0}E_0 + C_1 \overline{C_0}E_1+C_1\overline{C_0}E_2 + C_1C_0E_3
	\end{eqnarray*}
\end{document}






