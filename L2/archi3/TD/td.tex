\documentclass[12pt,a4paper,openany]{book}


\usepackage{lmodern}
\usepackage{xcolor}
\input{/home/satenske/cours/includesLaTeX/couleurs.tex}

\usepackage[utf8]{inputenc}
\usepackage[T1]{fontenc}
\usepackage[francais]{babel}
\usepackage[top=1.7cm, bottom=1.7cm, left=1.7cm, right=1.7cm]{geometry}
\usepackage{verbatim}
\usepackage[urlbordercolor={1 1 1}, linkbordercolor={1 1 1}, linkcolor=vert1, urlcolor=bleu, colorlinks=true]{hyperref}
\usepackage{tikz} %Vectoriel
\usepackage{listings}
\usepackage{fancyhdr}
\usepackage{multido}
\usepackage{amssymb}

\newcommand{\titre}{Architecture des systèmes Informatiques --- TD}

\newcommand{\pole}{}
\newcommand{\sigle}{archi}

\newcommand{\semestre}{3}

\input{/home/satenske/cours/includesLaTeX/listings.tex} %prise en charge du langage algo
\input{/home/aroquemaurel/cours/includesLaTeX/entete-l2-cours.tex}




%----------------------------------------------------------------------------------------
%	DEFINITION OF COLORED BOXES
%----------------------------------------------------------------------------------------

\RequirePackage[framemethod=default]{mdframed} % Required for creating the theorem, definition, exercise and corollary boxes

% Theorem box
\newmdenv[skipabove=7pt,
skipbelow=7pt,
backgroundcolor=black!5,
linecolor=ocre,
innerleftmargin=5pt,
innerrightmargin=5pt,
innertopmargin=5pt,
leftmargin=0cm,
rightmargin=0cm,
innerbottommargin=5pt]{tBox}

% Exercise box	  
\newmdenv[skipabove=7pt,
skipbelow=7pt,
rightline=false,
leftline=true,
topline=false,
bottomline=false,
backgroundcolor=ocre!10,
linecolor=ocre,
innerleftmargin=5pt,
innerrightmargin=5pt,
innertopmargin=5pt,
innerbottommargin=5pt,
leftmargin=0cm,
rightmargin=0cm,
linewidth=4pt]{eBox}	

% Definition box
\newmdenv[skipabove=10pt,
skipbelow=10pt,
rightline=false,
leftline=true,
topline=false,
bottomline=false,
linecolor=ocre,
innerleftmargin=5pt,
innerrightmargin=5pt,
innertopmargin=0pt,
leftmargin=0cm,
rightmargin=0cm,
linewidth=4pt,
innerbottommargin=0pt]{dBox}	

% Corollary box
\newmdenv[skipabove=7pt,
skipbelow=7pt,
rightline=false,
leftline=true,
topline=false,
bottomline=false,
linecolor=gray,
backgroundcolor=black!5,
innerleftmargin=5pt,
innerrightmargin=5pt,
innertopmargin=5pt,
leftmargin=0cm,
rightmargin=0cm,
linewidth=4pt,
innerbottommargin=5pt]{cBox}		

% Corollary box
\newmdenv[skipabove=7pt,
skipbelow=7pt,
rightline=true,
leftline=false,
topline=false,
bottomline=true,
linecolor=gray,
backgroundcolor=black!5,
innerleftmargin=5pt,
innerrightmargin=5pt,
innertopmargin=5pt,
leftmargin=0cm,
rightmargin=0cm,
linewidth=1pt,
innerbottommargin=5pt]{rBox}				  
		  

% Creates an environment for each type of theorem and assigns it a theorem text style from the "Theorem Styles" section above and a colored box from above
\newenvironment{theorem}{\begin{tBox}\begin{theoremeT}}{\end{theoremeT}\end{tBox}}
\newenvironment{example}{\begin{exampleT}}{\hfill{\tiny\ensuremath{\blacksquare}}\end{exampleT}}
\newenvironment{definition}{\begin{dBox}\begin{definitionT}}{\end{definitionT}\end{dBox}}
\newenvironment{attention}{\begin{eBox}\small}{\end{eBox}}				  	
\newenvironment{exemple}{\begin{cBox}\small}{\end{cBox}}	

%----------------------------------------------------------------------------------------
%	REMARK ENVIRONMENT
%----------------------------------------------------------------------------------------

\newenvironment{remarque}{\par\vskip10pt\small
\begin{rBox}
\begin{list}{}{
\leftmargin=35pt % Indentation on the left
\rightmargin=25pt}\item\ignorespaces % Indentation on the right
\makebox[-2.5pt]{\begin{tikzpicture}[overlay]
\node[draw=ocre!60,line width=1pt,circle,fill=ocre!25,font=\sffamily\bfseries,inner sep=2pt,outer sep=0pt] at (-15pt,0pt){\textcolor{ocre}{R}};\end{tikzpicture}} % Orange R in a circle
\advance\baselineskip -1pt}
{\end{list}\vskip1mm\end{rBox}\vskip5pt} % Tighter line spacing and white space after remark



\input{/home/satenske/cours/includesLaTeX/polices.tex}
\input{/home/satenske/cours/includesLaTeX/affichageChapitre.tex}

\begin{document}
	\setcounter{tocdepth}{2}
	\setcounter{secnumdepth}{3}
	\maketitle
	\chapter{Numérotation et codage\\ TD optionnel}
		\section{Numérotation}
		\subsection{Réaliser l'opération suivante en binaire : $(1101011-10110)\times 11001$}
		\begin{eqnarray*}
			(110\;1011)_2 &=& (107)_{10}\\
			(1\;0110)_2&=&(22)_{10}\\
			(1\;1001)_2&=&(22)_{10}
		\end{eqnarray*}
		\begin{center}
		\begin{tabular}{cccccccc}
			&1&1&0&1&0&1&1\\
			$-$&$$&&1&0&1&1&0\\
			\hline 
			&1&0&1&0&1&0&1\\
		\end{tabular} Ou 
		\begin{tabular}{cccccccc}
			&1&1&0&1&0&1&1\\
			+&1&1&1&0&1&0&1\\
			\hline
			&1&0&1&0&1&0&1
		\end{tabular}
		\begin{eqnarray*}
			(1010101)_2&=&(85)_{10}
		\end{eqnarray*}
		\begin{tabular}{cccccccccccccc}
			&&&&&&1&0&1&0&1&0&1&\\
			&&&&&$\times$&&1&1&0&0&1&\\
			\hline
			&&&&&&1&0&1&0&1&0&1&\\
			&&&1&0&1&0&1&0&1&.&.&.&\\
			&&1&0&1&0&1&0&1&.&.&.&.&\\
			\hline
			1&0&0&0&0&0&1&0&0&1&1&0&1&
		\end{tabular}
	\end{center}
		\subsection{Réaliser les opérations suivantes en hexadécimal : $(389A+7293)-EB2$}
		\begin{center}
			
		\begin{tabular}{ccccc}
			&3&8&9&A\\
			+&7&2&9&3\\
			\hline
			&A&B&2&B\\
			$-$&E&B&2&B\\
			\hline
			&9&C&7&B
		\end{tabular}
		\begin{eqnarray*}
			(389A)_{16} &=& (0011\; 1000\; 1001\; 1010)_2\\
			(7293)_{16} &=& (0111\; 0010\; 1001\; 0011)_2\\
			(AB2B)_{16} &=& (1010\;1011\;0010\;1101)_2\\
			(EB2B)_{16} &=& (0000\;1110\;1011\;0010)_2\\
			(9C7B)_{16} &=& (1001\;1100\;0111\;1011)_2
		\end{eqnarray*}
		\end{center}

		\subsection{Effectuer les conversions ci-dessous}
			\subsubsection{$(1447.140625)_{10} = (??)_2 = (??)_{16}$}
			\begin{eqnarray*}
				1447\div 16 &=& 90\; R=7\\
				90 \div 16 &=&~5\; R=A\\
				5 \div 16 &=&  ~0 \; R=5\\ 
				(1447)_{10} &=& (5A7)_{16}\\\\
				0.140625 \times 16 &=& 2.25\\
				0.25 \times 16 &=& 4.00\\
				(0.140625)_{10} &=& (0.24)_{16}\\\\
				(1447.140625)_{10} = (5A7.24)_{16} &=& (0101\;1010\;0111.0010\;0100)_2
			\end{eqnarray*}
			\subsubsection{$(1111100101.01011)_{2} = (??)_{10} = (??)_{16}$}
			\begin{eqnarray*}
				(11\;1110\;0101.0\;1011)_2 &=& (3E5;58)_{16}\\
				3E5 &=& (3\times 16^2 + 14*16 + 5 + 5 \times 16^{-1} + 8 \times 16 ^{-2}\\
				&=& (997,34375)_{10}
			\end{eqnarray*}
		\section{Codage}
		\subsection{Codage de nombres entiers relatifs}
		On veut coder des entiers relatifs sur 16 chifffres binaires (deux octets).

		\begin{tabular}{| p{1.8cm}|p{4cm}|p{2cm}|p{3.7cm}|p{3.7cm}|}
			\hline
			& \multicolumn{2}{ c|}{\textbf{Valeur absolue}}  & \multicolumn{2}{ c |}{\textbf{Valeur relative}}\\
			\hline
			\textbf{Base 10} & \textbf{Base 2} & \textbf{Base 16} & \textbf{Valeur absolue + signe} & \textbf{Complément à 2}\\
			\hline
			$\ \ 35671$&$10000\;010\;0101\;0111$& 75B8 & Hors intervalle & Hors intervalle\\
			\hline
			$-32768$ & $1000\; 0000\; 0000\; 0000$ & $8000$ & Hors intervalle & $1000\; 0000\; 0000$\\
			\hline
			$\ \ 46443$& $1011\; 0101\; 0110\; 1011$ & B56B & Hors intervalle & Hors intervalle\\
			\hline
			$-19536$& $0100\; 1100\; 0110\;0100$ & 4C64 & $1100\; 1100\; 0110\; 0100$ & $1011\; 0011\; 1001\; 1100$\\
			\hline
			$-19040$ & $0100\; 1010\; 0110\; 0000$ & 4A60 & $1100\; 1010\; 0110\; 0000$ & $1011\; 0101\; 1010\; 0000$\\
			\hline
		\end{tabular}
		\subsubsection{Calcul de la valeur absolue en base 2}
		\begin{eqnarray*}
			3567 \div 16 &=& 229 \textrm{ et reste 7}\\
			2229 \div 16 &=& 139 \textrm{ et reste 5}\\
			139 \div 16 &=& 8 \textrm{ et reste B}\\
			8 \div 16 &=&0 \textrm{ et reste 8}\\
		\end{eqnarray*}
		$$\Rightarrow (35671)_{10} = (8B57)_{16} = (1000\; 1010\; 0101\; 0111)_2$$
		\remarque{
		Afin de pouvoir représenter un nombre celui-ci ne doit pas dépasser un certain intervalle : 

\begin{description}
	\item[Entier naturel] $[0;2^{16-1}] = [0;65535]$
	\item[Valeur absolue + signe] $[-2^{15-1};2^{15-1}] = [-16384 ; 16384]$
	\item[Complément à deux] $[-2^{15};2^{15}] = [-32768 ; 32768]$
\end{description}
		}
		\subsection{Convertir un nombre flottant en décimal}
		\begin{center}
			\begin{tabular}{|c|c|c|}
				\hline
				$0$ & $1000\; 0001$ & $1110\; 0000\; 0000\; 0000\; 0000\; 0000$\\
				\hline
			\end{tabular}
		\end{center}
	$$C = E + biais$$
	En simple précision biais = 127.

	\begin{eqnarray*}
		S &=& 0 \rightarrow \textrm{ positif}\\
		C &=& 129\\
		E &=& C - 127 = 2\\
		1.M &=& 1.111 \Rightarrow 1.111 \times 2^2 = 111.1 \times 2^0 = (7.5)_{10}\\
	\end{eqnarray*}
	\newpage	
	\subsection{Convertir un nombre décimal en flottant}
	\begin{eqnarray*}
		(35.5)_{10} &=& ? \\
		100011.1 &=& 10000111 \times 2^5\\
		\textrm{Nombre positif donc } S &=& 0\\
		E &=& 5\\
		C &=& E + 127 = 132 = 128+4 = (10000100)_2\\
		1.M &=& 1.00011\\
		M &=& 0011\\
	\end{eqnarray*}
	
		\begin{center}
			\begin{tabular}{|c|c|c|}
				\hline
				$0$ & $1000\; 0001$ & $0100\; 0011\; 0000\; 0000\; 0000\; 0000$\\
				\hline
			\end{tabular}
		\end{center}
		

		\chapter{Algèbre de Bool}
	\section{Table de vérité des opérateurs classiques}
	\subsection{Exercice 1}
		\subsubsection{A -- Démontrer que les opérateurs NAND et NOR sont des opérateurs complets}
		\begin{eqnarray*}
			\bar A &=& A|A\\
			A.B &=& \overline{\overline{A.B}} = \overline{\bar A + \bar B} = \overline{A/B} = (A | B) | (A | B)\\
			A+B &=& \overline{\overline{A+B}} = \overline{\bar A + \bar B} = (A|B) |(B |B)
		\end{eqnarray*}
		\begin{eqnarray*}
			\bar A = \overline{A+A} = A \downarrow A\\
			A . B = \overline{\overline{A.B}} = \overline{\bar A+ \bar B} = A \downarrow B \downarrow (\downarrow B \downarrow B)
		\end{eqnarray*}
		\subsubsection{B --}
		\begin{enumerate}
			\item $f(A,B,C,D)=\bar A B D + B\bar C + A\bar C D= \overline{\overline{\bar A B \bar D + B \bar C + A \bar C D}} 
				= \overline{(\overline{\bar B\bar D).(\overline{A\bar C D)}}} = ( (A|A)|B( (D|D))) | (B|(C|C)) | (A|(C|C)|D)
				$

			
		\end{enumerate}<++>
\end{document}






