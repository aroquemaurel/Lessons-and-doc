\documentclass[12pt,a4paper,openany]{report}

\usepackage{lmodern}
\usepackage{xcolor}
\usepackage[utf8]{inputenc}
\usepackage[T1]{fontenc}
\usepackage[francais]{babel}
\usepackage[top=1.7cm, bottom=1.7cm, left=1.7cm, right=1.7cm]{geometry}
\usepackage{verbatim}
\usepackage{tikz} %Vectoriel
\usepackage{listings}
\usepackage{fancyhdr}
\usepackage{multido}
\usepackage{amssymb}

\newcommand{\titre}{Systèmes répartis et réseaux locaux}

\newcommand{\module}{}
\newcommand{\sigle}{srrl}

\newcommand{\semestre}{4}

\input{/home/satenske/cours/listings.tex} %prise en charge du langage algo
\input{/home/satenske/cours/entete_iut-cours.tex}

\begin{document}
	\maketitle
	Clef moodle $=$ \$asr4
	\chapter{Réseaux et protocoles}
	\section{La famille des protocoles TCP/IP}
		\subsection{Internet et TCP-IP}
		\begin{itemize}
			\item Le protocole TCP-IP date des années 70-75. Ce n'est pas un protocole nouveau. Il à été créer par le 
			département de la défense américaine DoD. Protection de la sécurité du pays.
			\item \bsc{HAYES} pour les modem
			\item \bsc{SMTP} pour les mail
			\item \bsc{MOSAIC} (client)
		\end{itemize}

		\subsubsection{Raison de son succés}
		\begin{itemize}
			\item Protocole du réseau Internet
			\item Tous les Systèmes d'exploitation 
			\item Différentes couches
				\begin{itemize}
					\item Couche transport (TCP ou UDP)
					\item Couche réseau (IP)
				\end{itemize}
			\item Routable
			\item Protocole ouvert
			\item Indépendant du support physique
		\end{itemize}
		\subsubsection{Les différentes couches}
		\paragraph{Couches d'accés au réseau}
		LLC\footnote{Logical Link Control} et MAC\footnote{Medium Acces Control}\\
		
		

\end{document}



