\documentclass[12pt,a4paper,openany]{book}

\usepackage{lmodern}
\usepackage{xcolor}
\usepackage[utf8]{inputenc}
\usepackage[T1]{fontenc}
\usepackage[francais]{babel}
\usepackage[top=1.7cm, bottom=1.7cm, left=1.7cm, right=1.7cm]{geometry}
%\usepackage[frenchb]{babel}
%\usepackage{layout}
%\usepackage{setspace}
%\usepackage{soul}
%\usepackage{ulem}
%\usepackage{eurosym}
%\usepackage{bookman}
%\usepackage{charter}
%\usepackage{newcent}
%\usepackage{lmodern}
%\usepackage{mathpazo}
%\usepackage{mathptmx}
%\usepackage{url}
%\usepackage{verbatim}
%\usepackage{moreverb}
%\usepackage{wrapfig}
%\usepackage{amsmath}
%\usepackage{mathrsfs}
%\usepackage{asmthm}
%\usepackage{makeidx}
%\usepackage{tikz} %Vectoriel
\usepackage{listings}
\usepackage{fancyhdr}
%\usepackage{multido}
%\usepackage{amssymb}

%\definecolor{gris1}{gray}{0.40}
\definecolor{gris2}{gray}{0.55}
\definecolor{gris3}{gray}{0.65}
\definecolor{gris4}{gray}{0.50}
\definecolor{vert}{rgb}{0,0.4,0}
\definecolor{violet}{rgb}{0.65, 0.2, 0.65}
\definecolor{bleu1}{rgb}{0,0,0.8}
\definecolor{bleu2}{rgb}{0,0.2,0.6}
\definecolor{bleu3}{rgb}{0,0.2,0.2}
\definecolor{rouge}{HTML}{F93928}


\lstdefinelanguage{algo}{%
   morekeywords={%
    %%% couleur 1
		importer, programme, glossaire, fonction, procedure, constante, type, 
	%%% IMPORT & Co.
		si, sinon, alors, fin, tantque, debut, faire, lorsque, fin lorsque, 
		declenche, declencher, enregistrement, tableau, retourne, retourner, =, pour, a,
		/=, <, >, traite,exception, 
	%%% types 
		Entier, Reel, Booleen, Caractere, Réél, Booléen, Caractère,
	%%% types 
		entree, maj, sortie,entrée,
	%%% types 
		et, ou, non,
	},
  sensitive=true,
  morecomment=[l]{--},
  morestring=[b]',
}

\lstset{language=algo,
    %%% BOUCLE, TEST & Co.
      emph={importer, programme, glossaire, fonction, procedure, constante, type},
      emphstyle=\color{bleu2},
    %%% IMPORT & Co.  
	emph={[2]
		si, sinon, alors, fin , tantque, debut, faire, lorsque, fin lorsque, 
		declencher, retourner, et, ou, non,enregistrement, retourner, retourne, 
		tableau, /=, <, =, >, traite,exception, pour, a
	},
      emphstyle=[2]\color{bleu1},
    %%% FONCTIONS NUMERIQUES
      emph={[3]Entier, Reel, Booleen, Caractere, Booléen, Réél, Caractère},
      emphstyle=[3]\color{gris1},
    %%% FONCTIONS NUMERIQUES
      emph={[4]entree, maj, sortie, entrée},	
      emphstyle=[4]\color{gris1},
}
\lstdefinelanguage{wl}{%
   morekeywords={%
    %%% couleur 1
		importer, programme, glossaire, fonction, procedure, constante, type, 
	%%% IMPORT & Co.
		si, sinon, alors, fin, TANTQUE, tantque, FIN, PROCEDURE, debut, faire, lorsque, 
		fin lorsque, declenche, declencher, enregistrement, tableau, retourne, retourner, =, 
		/=, <, >, traite,exception, 
	%%% types 
		Entier, Reel, Booleen, Caractere, Réél, Booléen, Caractère,
	%%% types 
		entree, maj, sortie,entrée,
	%%% types 
		et, ou, non,
	},
  sensitive=true,
  morecomment=[l]{//},
  morestring=[b]',
}

\lstset{language=wl,
    %%% BOUCLE, TEST & Co.
      emph={importer, programme, glossaire, fonction, procedure, constante, type},
      emphstyle=\color{bleu2},
    %%% IMPORT & Co.  
	emph={[2]
		si, sinon, alors, fin , tantque, debut, faire, lorsque, fin lorsque, 
		declencher, retourner, et, ou, non,enregistrement, retourner, retourne, 
		tableau, /=, <, =, >, traite,exception
	},
      emphstyle=[2]\color{bleu1},
    %%% FONCTIONS NUMERIQUES
      emph={[3]Entier, Reel, Booleen, Caractere, Booléen, Réél, Caractère},
      emphstyle=[3]\color{gris1},
    %%% FONCTIONS NUMERIQUES
      emph={[4]entree, maj, sortie, entrée},	
      emphstyle=[4]\color{gris1},
}
\lstdefinelanguage{css}{%
   morekeywords={%
    %%% couleur 1
		background, image, repeat, position, index, color, border, font, 
		size, url, family, style, variant, weight, letter, spacing, line, 
		height, text, decoration, align, indent, transform, shadow, 
		background, image, repeat, position, index, color, border, font, 
		size, url, family, style, variant, weight, letter, spacing, line, 
		height, text, decoration, align, indent, transform, shadow, 
		vertical, align, white, space, word, spacing,attachment, width, 
		max, min, margin, padding, clip, direction, display, overflow,
		visibility, clear, float, top, right, bottom, left, list, type, 
		collapse, side, empty, cells, table, layout, cursor, marks, page, break,
		before, after, inside, orphans, windows, azimuth, after, before, cue, 
		elevation, pause, play, during, pitch, range, richness, spek, header, 
		numeral, punctuation, rate, stress, voice, volume,
	%%% types 
		left, right, bottom, top, none, center, solid, black, blue, red, green,
	},
  sensitive=true,
  sensitive=true,
  morecomment=[s]{/*}{*/},
  morestring=[b]',
}
\lstset{language=css,
    %%% BOUCLE, TEST & Co.
      emph={
		background, image, repeat, position, index, color, border, font, 
		size, url, family, style, variant, weight, letter, spacing, line, 
		height, text, decoration, align, indent, transform, shadow, 
		background, image, repeat, position, index, color, border, font, 
		size, url, family, style, variant, weight, letter, spacing, line, 
		height, text, decoration, align, indent, transform, shadow, 
		vertical, align, white, space, word, spacing,attachment, width, 
		max, min, margin, padding, clip, direction, display, overflow,
		visibility, clear, float, top, right, bottom, left, list, type, 
		collapse, side, empty, cells, table, layout, cursor, marks, page, break,
		before, after, inside, orphans, windows, azimuth, after, before, cue, 
		elevation, pause, play, during, pitch, range, richness, spek, header, 
		numeral, punctuation, rate, stress, voice, volume,
	  },
      emphstyle=\color{bleu2},
    %%% FONCTIONS NUMERIQUES
      emph={[3]
		left, right, bottom, top,none, solid, black, blue, green,
		  },
      emphstyle=[3]\color{bleu3},
    %%% FONCTIONS NUMERIQUES
}

\lstset{language=SQL,
    %%% BOUCLE, TEST & Co.
      emph={INSERT, UPDATE, DELETE, WHERE, SET, GROUP, BY, ORDER, REFERENCES},
      emphstyle=\color{bleu2},
    %%% IMPORT & Co.  
	emph={[2]
		if, end, begin, then, for, each, else, after, of, on, to
	},
      emphstyle=[2]\color{bleu1},
    %%% FONCTIONS NUMERIQUES
      emph={[3]Entier, Reel, Booleen, Caractere, Booléen, Réél, Caractère},
      emphstyle=[3]\color{gris1},
    %%% FONCTIONS NUMERIQUES
      emph={[4]entree, maj, sortie, entrée},	
      emphstyle=[4]\color{gris1},
}
\lstdefinelanguage{ARM}{%
   morekeywords={%
   ADD, SUB, MOV, MUL, RSB,CMP, BLS, BLE, B,BHI,LDR,
   BGE, RSBLT, BGT, BEQ, BNE,BLT,BHS,STR,STRB
	},
  sensitive=true,
  morecomment=[l]{@},
  morestring=[b]',
}

\lstset{ % general style for listings 
   numbers=left 
   , literate={é}{{\'e}}1 {è}{{\`e}}1 {à}{{\`a}}1 {ê}{{\^e}}1 {É}{{\'E}}1 {ô}{{\^o}}1 {€}{{\euro}}1{°}{{$^{\circ}$}}1 {ç}{ {c}}1 {ù}{u}1
	, extendedchars=\true
   , tabsize=2 
   , frame=l
   , framerule=1.1pt
   , linewidth=520px
   , breaklines=true 
   , basicstyle=\footnotesize\ttfamily 
   , numberstyle=\tiny\ttfamily 
   , framexleftmargin=0mm 
   , xleftmargin=0mm 
   , captionpos=b 
	, keywordstyle=\color{bleu2}
	, commentstyle=\color{vert}
	, stringstyle=\color{rouge}
	, showstringspaces=false
	, extendedchars=true
	, mathescape=true
} 
%	\lstlistoflistings
%	\addcontentsline{toc}{part}{List of code examples}


\title{Cours\\ Adminsitration système}
\date{ADS\\ Semestre 2}

\lhead{Cours: Administration système}
\chead{}
\rhead{\thepage}

\lfoot{Université paul sabatier Toulouse III}
\cfoot{\thepage}
\rfoot{ads2}

\pagestyle{fancy}
\begin{document}
	\maketitle
	L'administration système est l'ensemble des tâches faites par des informaticiens.
	Elle sert à maintenir un logiciel dans un parc informatiques, les systèmes les 
	plus utilisés sont Linux et Windows.

	\section{Tâches d'administration systèmes}
		\paragraph{}
		Il faut d'abord \textbf{installer et configurer} les systèmes d'exploitation
			de la machine.
	\paragraph{}
	Mettre en place les moyens et les procédures pour garantir les performances
	et la disponibilités des systèmes. 
	\begin{itemize}
		\item Gérer les comptes utilisateurs
		\item installer et mettre à jour les nouveaux logiciels
		\item organiser et mainteir les systèmes de fichiers.
		\item organiser et mainteir les systèmes d'impression.
		\item concevoir et développer les outils logiciels pour automatiser
			certaines tâches d'administration
	\end{itemize}
	\paragraph{}
		il faut administrer, maintenir et faire évoluer le réseau local et très
		souvent l'accès au réseau public.
	\paragraph{}
		Également mettre en places les moyens et les procédures pour garantir la 
		sécurité des systèmes et des réseaux.
	\paragraph{}
		Résoudre les incidents ponctuels (mot de pass oublié....)
	\paragraph{}
		Conseiller, assister et former les utilisateurs
	\paragraph{}
		Animer et coordonner l'activié des techniciens informatiques
	\paragraph{}
		assurer la veille technologique
	\paragraph{}
		planifier l'évolution matérielleet logiciels en fonction des besoin des utilisateurs
	\paragraph{}
		négocier avec les fournisseurs d'équipements informatiques
	\section{Les différents niveaux d'administration}
		\paragraph{}
			Depuis les années 80, les progrès technologiques ont entrainé un dévelopement 
			considérable des réseaux locaux et publics d'ordinateurs.
		\paragraph{}
			La plupart des machines sont connectées en réseau, par conséquent les 
			tâches d'administartion réseau on pris une importance considérable.
		\paragraph{}
			C'est pourquoi on décompose l'administation système en deux niveaux:
			\begin{itemize}
				\item le niveau système
				\item le niveau réseau
			\end{itemize}
	\chapter{Installation des systèmes Linux}
		Grâce à la démocratisation de Linux, les différentes distribution ont amélioré
		l'interface d'installation.\\
		Il faut connaître l'architecture exact de la machine.\\
		Si la machine possède déjà un autre OS, il est nécessaire de comprendre ce qui
		va se passer durant le processus d'installation.\\
		Pour procéder à l'installation d'un système il faut disposer d'une image de la
		distribution du système sur un support \textit{bootable} (CD, DVD, clef USB...)

		\section{Notion de systèmes de fichier et de partionnement}
		\paragraph{}
			Le système de Gestion de Fichier (SGF) est la composante de l'OS
			qui fournit aux utilisateurs un modèle logique abstrait de stockage et 
			de gestion des informations persistantes et de leur organisation des 
			des fichiers en faisant abstraction du support matériel sur lequel
			sont stockées ces informations 
		\paragraph{}
			La structure de données la plus couramment utilisée est une organisation
			arborescente basé sur le concept de répertoire (apellé aussi catalogue
					ou dossier)
			\begin{itemize}
				\item Les systèmes UNIX ou Linux utilisent une arborescence unique
				\item Les systèmes Windows utilisent une arborescence particulière
				dans laquelle est le noeuds du premier niveau dont des unités 
				logiques (A, C, D, ...) contenant chacune une arborescence unique.
			\end{itemize}	
			L'implémentation interne de la structure de donnée du SGF est réalisée
			par des structures de données interners spécifiques appelées 
			\textit{file system} en anglais et traduit par système de fichier.
			\paragraph{}
			Un systeme de fichier définit l'organisation interne d'un ensemble
			de fichiers enregistrés sur tout ou partie d'un média de type
			\begin{itemize}
				\item Unités de disque magnétiques
				\item Unités de disques optiques
				\item Unités de disque SSD
				\item Autre....
			\end{itemize}
			Il existe de très nombreux système de fichiers qui pour la plupart
			sont spécifiques à un système d'exploitation, ou à un média.
			\paragraph{}
			Les caractéristiques du SGF des systèmes UNIX ou Linux et de leur
			système de fichiers et leur administration seront présentées plus
			en détail dans le chapitre 3 du cours.
			\paragraph{}
			Dans la plupart des cas, plusieurs système de fichiers seront nécessaire
			pour implémenter le SGF. \\
			Pour le système Linux par exemple la partie résidente est en ext3, les disques
			en FAT32
			\paragraph{}
			L'espace de stockage de certains médias comme les disques durs ou les 
			disques SSD peut être découpé en plusieurs parties contiguës de 
			différentes tailles appelées partition. Une partition est gérée par 
			le système comme un disque logique dans lequel il est possible 
			d'implémenter un système de fichier. 
			\paragraph{Opération de formatage}
			L'implémentation d'un système de fichier sur une partition d'un média 
			ou sur la totalité de l'espace de stockage de certains média, à ne 
			pas confondre avec le formatage de bas niveau qui consiste à créer 
			les secteurs physiques.
			\subsection{Pourquoi partitionner?}
				La création d'une ou plusieurs partitions sur un ou plusieurs 
				disques durs correspond généralement à la mise en œuvre d'une
				politique de gestion de la part de l'administration et/ou à une
				obligation imposée par le gestionnaire d'E/S des disques durs 
				(le BIOS des PC par exemple)\\
				Les principales raison qui conduisent l'administrateur à créer
				 plusieurs partitions: 
			 \paragraph{La limitation des risques}
				La création de plusieurs partitions limite la perte de données à 
				la seule partition qui est endommagée. Sur une machine serveur,
				mettre dans des partitions différentes les données qui ne varient
				pas (noyau...) et celles qui varient en permanence (home, logs, 
						fichiers temporaires...) est une saine pratique qui 
				permet de conserver un système qui fonctionne car les partitions 
				qui ont le plus de risque d'être endommagées sont celle où on 
				supprime et ou on créer des fichiers. 
			\paragraph{Le multiboot}
				Il faut une partition pour OS (système de fichier différents) 
			\paragraph{Le contournement de la limitation de la taille d'une
				partition}
				La taille des partitions est limitée, cette limite varie selon 
				le type de système de fichier, mais aussi selon les contraintes
				matérielles de la machine. Chaque type de système de fichiers 
				disposes de tables d'allocations qui répertorient les blocs 
				occupés et libres de la partition qui le supporte. 
				Ces tables ont un nombre d'entrées limité qui définit le nombre
				de blocs logiques accessible et donc la taille maximal de la 
				partition. Ces limites sont par exemple de 2Go en FAT16, de
				2 To en FAT32, de 16G fois Go en NTFS.\\
				Aujourd'hui cela ne constituent pas une réel limitation par 
				rapport à la taille des disques durs, exceptée le FAT16.\
				Ces limitations sont théoriques, car une limitation réel
				plus restrictive est imposée par les sous programmes du BIOS
				qui réalisent les accès au secteurs physiques des disques durs.
			\subsection{Les limitations du partitionnement}
				Les contraintes du BIOS ne sont pas les seuls à influer sur
				le partitionnement. Le contexte de gestion de l'espace disque
				peut limiter la taille des partitions mais aussi leur nombre, 
				leur position relatives suivant le type (partitions bootable 
						ou pas)
				\paragraph{Les contraintes du MBR}
					Pour gérer le nombre et la taille des partitions une table des 
					partitions appelée MBR (Master Boot Record) est enregistrée
					dans le premier secteur de 512 octets du disque (les disques 
							durs pour PC sont formatés avec des secteurs de 512 octets)

			\subsubsection{Les frontières de cylindres}
				Les frontières des partitions sont enregistrées en unités
				de secteurs. Selon les systèmes et aussi les utilitaires
				de partitionnement, l'unité de découpage n'est pas le secteur mais 
				le cylindre \\
				Un cylindre est constitués de toutes les pistes de même rayon
				situées sur les faces des différents plateaux. Il faut donc
				connaître le nombre de secteurs contenus dans un cylindre et
				créer des partitions dont la taille en secteurs est un multiple
				de celle des cylindres.\\
				Heureusement la plupart des utilitaires de partitionnement arrondissent 
				automatiquement les limites de partitions fournies en nombre de secteurs
				en multiple de cylindres. Certains même ne demandent que la taille 
				approximative en Mo ou Go et font les calculs de limites 
				automatiquement.
		 \subsection{L'ordre des partitions}
			L'ordre n'a apriori l'ordre des partitions ne devrait pas avoir d'importance
			puisque chaque partition est indépendante des autres. En réalité certains
			systèmes ne reconnaissent pas le type des partitions des autres systèmes.
	\section{Préparation de l'installation}
			\subsection{Inventaire de la machine}
			Certains composants peuvent poser des problèmes si leur pilotes ne sont
			pas reconnus automatiquement lors de l'installation. Il est donc 
			nécessaire de connaitre la configuration matérielle de la machine afin
			de pouvoir répondre aux différentes questions que pourrait poser le 
			logiciel d'installation.
			\paragraph{Le type de disque dur et éventuellement sa taille}
			\paragraph{la taille de la mémoire centrale}
				Pour gérer l'exécution simultanée de plusieurs programmes 
				les systèmes multi tâches ont besoin d'un espace disque spécial 
				qu'ils utilisent comme une extension de la mémoire centrale.
				Cette extension s'appelle swap.
			\paragraph{Le contrôleur réseau}
				Ils posent de moins en moins problèmes. Il peut être nécessaire
				de connaître les références du contrôleur afin de paramétrer correctement
				le réseau, notamment si l'installation se fait par le réseau. 
			\paragraph{Le contrôleur graphique}
				Le contrôleur graphique peut ne pas être reconnu automatiquement.
				Il faut connaitre ses références pour paramétrer correctement 
				l'affichage. Il n'y a pas de problème tant que l'installation se
				déroule en mode graphique basique (mode VGA, géré par n'importe
						quel contrôleur graphique actuel)
			\paragraph{Le type de souris}
				La plupart des souris sont maintenant compatible PS/2.
			\paragraph{Le chipset}
			un problème peut survenir si on installe une ancienne distribution
			sur une machine récente. Le chipset reconnu ne correspond pas à celui
			de la carte mère. Mais comme dans une même gamme les chipset récents 
			couvrent les fonctionnalités des plus anciens.
		\subsection{Différences entre poste de travail et serveur}
			Contrairement aux systèmes Windows qui sont différents pour les postes
			de travail et les serveurs le système Linux comme UNIX est fondamentalement
			un système pour serveurs.\\ \\
			À l'installation, les différences entre un poste de travail et un 
			serveur vont porter principalement sur:
			\begin{itemize}
				\item le rationnement
			\end{itemize}
		\subsection{Le plan de partitionnement}
		Avant de commencer il faut définir le plan de actionnement du ou des disques
		 durs de la machine en fonction de son utilisation et aussi des évolutions futures.
		 \paragraph{Définir la carte de partition}
			On a vu aux chapitres précédents, la façon de procéder	
	\section{Les types d'installations}
		\subsection{Créer des CD d'installation}
		On peut trouver des CD ou des DVD d'installations dans des magazines, 
		mais on peut aussi récupérer les images de distributions Linux sur 
		le site de l'éditeur ou sur des miroirs.

		\subsection{Installation par le réseau}
		On lance un CD minimal pour démarrer l'installation à partir du réseau.
		Le BIOS va se connecter ensuite au serveur, tous les programmes seront 
		téléchargés par le réseau.

	\chapter{Gestion des systèmes de fichiers}
	\section{Les types de fichiers}
	Un fichier est un ensemble de données archivées en mémoire auxiliaire. Dans certains
	OS ces données sont structurées en enregistrement suivant une organisation 
	déterminée.(séquentielle, calculée, séquentielle..)\\
		\subsection{Fichiers ordinaires}
		Deux types:
		\begin{itemize}
			\item fichiers texte
			\item fichiers binaires
		\end{itemize}
		\subsection{Fichiers spécial}
		Un organe périphérique réel ou virtuel est considéré du point de vue du 
		système comme un fichier spécial dans lequel on peut lire et/ou écrire
		des donnée avec les même opérations que celles utilisées pour les fichiers
		ordinaires.\\
		Deux types de fichiers spéciaux: 
		\begin{itemize}
			\item blocs opérations d'E/S par blocs d'octets
			\item caractères opérations d'E/S octets par octets
		\end{itemize}
		\subsection{Autres types de fichiers}
		\subsection{Répertoire}
			\subsubsection{Répertoires standards}
	\section{}
			\subsection{Structure internet}
			Le SGF est enregistré en mémoire auxiliaires sur un ou plusieurs
			média de type
			\begin{itemize}
				\item Unités de disque magnétiques (disquettes, disque durs)
				\item unités de disque optiques (CD-ROM, DVD-ROM)
				\item Cartes mémoires (clé USB, carte memory stick, carte secure digital)
			\end{itemize}
			Après formatage l'espace de stockage d'un disque magnétique ou d'une carte
			mémoire est constitué d'un ensemble de N blocs consécutifs de taille fixe
			(512, 1024, 2048 ... Octets)
			\paragraph{} Les disques durs peuvent être découper en plusieurs partitions,
			partitions pouvant disposer de système de fichiers différents, le 
			nombre de partition varie en fonction de la taille du disque et de 
			l'architecture voulu par l'administrateur. 
			\paragraph{}
			Une partition est constituée par un intervalle de blocs consécutifs 
			et est gérée comme un disque logique comprenant une suite de M blocs 
			numérotés de 0 à M - 1.
			\paragraph{Unix}
			gère une arborescence unique qui peut être
			enregistrée sur une ou plusieurs partitions structurées en système
			de fichiers.
			\paragraph{Windows}
			Chaque partition d'un organe périphérique est vue par le système
			comme une unité logique, supportant un seul système de fichier.
			\paragraph{}
		\subsection{Les différents types de systèmes de fichiers}
			\begin{itemize}
				\item minix: débuts Linux
				\item ext2: standard de Linux
				\item ext3: évolution de ext2
				\item ext4: évolution de ext3
				\item sysv: ancien, fichiers Unix, System V
				\item ffs: Unix et BSD
				\item ufs: Extension du ffs Unix et BSD
				\item msdos: Système Fat16 pour MS DOS
				\item vfat: Extension de MS DOS
				\item ntfs: Partitions Windows NT/2000/XP
				\item hpfs: Partitions hpfs 
				\item iso9660: CD-ROM
				\item nfs: accessible par réseau développé par SUN Mycrosystem
			\end{itemize}
		\subsection{Structure du système de fichiers sysv d'Unix, System V de 
			AT\&T}
			\begin{verbatim}
	    +-------------------------------+
	0   |        bloc de "boot"         |
	    +-------------------------------+
	1   |         "super bloc"          |
	    +-------------------------------+
	2   |                               |
	    |      table des inodes         |	
	    |                               |
	k   |                               |
	    +-------------------------------+
	k+1 |                               |
	    |       bloc de données         |
	    |       et blocs libres	        |
	    |                               |
	M-1 +-------------------------------+

			\end{verbatim}

			\paragraph{Bloc 0} Contient l'identification du disque logique
			\paragraph{Bloc 1} contient les attributs du système de fichier
			\paragraph{Bloc 2 à k} Contiennent la table d'allocation des fichiers
			\paragraph{Blocs k + 1 à M - 1} Contiennent les données des fichiers
				et répertoires de l'arborescence supportée par le système de fichiers.

		\subsection{Structure du système de fichier ext2 de Linux}
		Le système de fichier ext2 à été développé pour pallier les inconvénients
		des systèmes de fichiers originels minix et ensuite ext2 e particulier le 
		besoin de supporter des partitions de grande taille.
		\subsection{Liens physiques et liens symboliques}
		\paragraph{Lien symboliques}
			
		\paragraph{Lien physiques}
			\begin{verbatim}
				+------+
				|      |  
				|      |
				+------+
			\end{verbatim}
	\section{Commandes d'administration des systèmes de fichiers}
		\subsection{Fichiers Spéciaux}
			Sous Linux les fichiers spéciaux qui sont les interfaces avec les pilotes
			de périphériques (device driver) qui gèrent les opérations d'E/S dans les
			partitions des unités de disques. 
			\begin{itemize}
				\item /dev/hda1 partition 1 du premier disque IDE
				\item /dev/hda2 partition 2 du premier disque IDE
				\item /dev/hdan partition n du premier disque IDE

				\item /dev/sda1 partition 1 du premier disque S-ATA
				\item /dev/sda2 partition 2 du premier disque S-ATA
				\item /dev/sdan partition n du premier disque S-ATA
				\item /dev/cdrom CD-Rom
			\end{itemize}		
		\subsectin{Création d'un système de fichier}
		Sous Linux la commande man fs permet de connaitre les types de systèmes
		de fichiers.
		$$mkfs~options~filesys$$
		\subsection{Montage et démontage d'un système de fichiers}

 \end{document}
