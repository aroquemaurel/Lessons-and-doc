\documentclass[12pt,a4paper,openany]{article}

\usepackage{lmodern}
\usepackage{xcolor}
\usepackage[utf8]{inputenc}
\usepackage[T1]{fontenc}
\usepackage[francais]{babel}
\usepackage[top=1.7cm, bottom=1.7cm, left=1.7cm, right=1.7cm]{geometry}
%\usepackage[frenchb]{babel}
%\usepackage{layout}
%\usepackage{setspace}
%\usepackage{soul}
%\usepackage{ulem}
%\usepackage{eurosym}
%\usepackage{bookman}
%\usepackage{charter}
%\usepackage{newcent}
%\usepackage{lmodern}
%\usepackage{mathpazo}
%\usepackage{mathptmx}
%\usepackage{url}
%\usepackage{verbatim}
%\usepackage{moreverb}
%\usepackage{wrapfig}
%\usepackage{amsmath}
%\usepackage{mathrsfs}
%\usepackage{asmthm}
%\usepackage{makeidx}
%\usepackage{tikz} %Vectoriel
\usepackage{listings}
%\usepackage{fancyhdr}
%\usepackage{multido}
%\usepackage{amssymb}


%\lstset{ % general style for listings 
 %  numbers=left 
%	, extendedchars=\true
  % , tabsize=2 
 %  , frame=single 
%   , breaklines=true 
   %, basicstyle=\ttfamily 
  % , numberstyle=\tiny\ttfamily 
 %  , framexleftmargin=13mm 
%  , xleftmargin=12mm 
 % , captionpos=b 
%	, language=html
%	, keywordstyle=\color{blue}
%	, commentstyle=\color{vert}
%	, showstringspaces=false
%	, extendedchars=true
%	, mathescape=true
%} 

\title{Cours\\ Programmation Internet}
\date{PGI\\ Semestre 2}

\pagestyle{fancy}
\begin{document}
	\maketitle
	\chapter{HTML}
		\begin{itemize}
			\item Balises
			\item Doctype
			\item Formulaires
		\end{itemize}
%		\lstinputlisting[language=html]{test.html}	
		\paragraph{CMS} Content Managment System. Créer des sites sans utiliser 
		du code HTML, automatisé, souvent logiciels libre, rapide et plus facile.
		\section{Formulaires}
		\paragraph{} Permet d'entrer des données par l'utilisateurs, lié à une 
		page dynamique, forum; chat etc...\\
		%% Graphic for TeX using PGF
% Title: /usr/home/satenske/Diagram1.dia
% Creator: Dia v0.97.1
% CreationDate: Tue Feb  1 08:26:44 2011
% For: satenske
% \usepackage{tikz}
% The following commands are not supported in PSTricks at present
% We define them conditionally, so when they are implemented,
% this pgf file will use them.
\ifx\du\undefined
  \newlength{\du}
\fi
\setlength{\du}{15\unitlength}
\begin{tikzpicture}
\pgftransformxscale{1.000000}
\pgftransformyscale{-1.000000}
\definecolor{dialinecolor}{rgb}{0.000000, 0.000000, 0.000000}
\pgfsetstrokecolor{dialinecolor}
\definecolor{dialinecolor}{rgb}{1.000000, 1.000000, 1.000000}
\pgfsetfillcolor{dialinecolor}
\pgfsetlinewidth{0.100000\du}
\pgfsetdash{}{0pt}
\pgfsetdash{}{0pt}
\pgfsetbuttcap
\pgfsetmiterjoin
\pgfsetlinewidth{0.100000\du}
\pgfsetbuttcap
\pgfsetmiterjoin
\pgfsetdash{}{0pt}
\definecolor{dialinecolor}{rgb}{1.000000, 1.000000, 1.000000}
\pgfsetfillcolor{dialinecolor}
\fill (1.382258\du,9.600000\du)--(1.382258\du,11.600000\du)--(3.317742\du,11.600000\du)--(3.317742\du,9.600000\du)--cycle;
\definecolor{dialinecolor}{rgb}{0.000000, 0.000000, 0.000000}
\pgfsetstrokecolor{dialinecolor}
\draw (1.382258\du,9.600000\du)--(1.382258\du,11.600000\du)--(3.317742\du,11.600000\du)--(3.317742\du,9.600000\du)--cycle;
\pgfsetbuttcap
\pgfsetmiterjoin
\pgfsetdash{}{0pt}
\definecolor{dialinecolor}{rgb}{0.000000, 0.000000, 0.000000}
\pgfsetstrokecolor{dialinecolor}
\draw (1.382258\du,9.600000\du)--(1.382258\du,11.600000\du)--(3.317742\du,11.600000\du)--(3.317742\du,9.600000\du)--cycle;
\pgfsetlinewidth{0.100000\du}
\pgfsetdash{}{0pt}
\pgfsetdash{}{0pt}
\pgfsetbuttcap
\pgfsetmiterjoin
\pgfsetlinewidth{0.100000\du}
\pgfsetbuttcap
\pgfsetmiterjoin
\pgfsetdash{}{0pt}
\definecolor{dialinecolor}{rgb}{1.000000, 1.000000, 1.000000}
\pgfsetfillcolor{dialinecolor}
\fill (6.032258\du,9.600000\du)--(6.032258\du,11.600000\du)--(7.967742\du,11.600000\du)--(7.967742\du,9.600000\du)--cycle;
\definecolor{dialinecolor}{rgb}{0.000000, 0.000000, 0.000000}
\pgfsetstrokecolor{dialinecolor}
\draw (6.032258\du,9.600000\du)--(6.032258\du,11.600000\du)--(7.967742\du,11.600000\du)--(7.967742\du,9.600000\du)--cycle;
\pgfsetbuttcap
\pgfsetmiterjoin
\pgfsetdash{}{0pt}
\definecolor{dialinecolor}{rgb}{0.000000, 0.000000, 0.000000}
\pgfsetstrokecolor{dialinecolor}
\draw (6.032258\du,9.600000\du)--(6.032258\du,11.600000\du)--(7.967742\du,11.600000\du)--(7.967742\du,9.600000\du)--cycle;
\pgfsetlinewidth{0.100000\du}
\pgfsetdash{}{0pt}
\pgfsetdash{}{0pt}
\pgfsetbuttcap
{
\definecolor{dialinecolor}{rgb}{0.000000, 0.000000, 0.000000}
\pgfsetfillcolor{dialinecolor}
% was here!!!
\pgfsetarrowsend{stealth}
\definecolor{dialinecolor}{rgb}{0.000000, 0.000000, 0.000000}
\pgfsetstrokecolor{dialinecolor}
\draw (2.867688\du,10.692395\du)--(5.900000\du,10.750000\du);
}
\pgfsetlinewidth{0.100000\du}
\pgfsetdash{}{0pt}
\pgfsetdash{}{0pt}
\pgfsetbuttcap
\pgfsetmiterjoin
\pgfsetlinewidth{0.100000\du}
\pgfsetbuttcap
\pgfsetmiterjoin
\pgfsetdash{}{0pt}
\definecolor{dialinecolor}{rgb}{1.000000, 1.000000, 1.000000}
\pgfsetfillcolor{dialinecolor}
\fill (7.953011\du,9.605000\du)--(7.953011\du,11.605000\du)--(9.888495\du,11.605000\du)--(9.888495\du,9.605000\du)--cycle;
\definecolor{dialinecolor}{rgb}{0.000000, 0.000000, 0.000000}
\pgfsetstrokecolor{dialinecolor}
\draw (7.953011\du,9.605000\du)--(7.953011\du,11.605000\du)--(9.888495\du,11.605000\du)--(9.888495\du,9.605000\du)--cycle;
\pgfsetbuttcap
\pgfsetmiterjoin
\pgfsetdash{}{0pt}
\definecolor{dialinecolor}{rgb}{0.000000, 0.000000, 0.000000}
\pgfsetstrokecolor{dialinecolor}
\draw (7.953011\du,9.605000\du)--(7.953011\du,11.605000\du)--(9.888495\du,11.605000\du)--(9.888495\du,9.605000\du)--cycle;
\pgfsetlinewidth{0.100000\du}
\pgfsetdash{}{0pt}
\pgfsetdash{}{0pt}
\pgfsetbuttcap
{
\definecolor{dialinecolor}{rgb}{0.000000, 0.000000, 0.000000}
\pgfsetfillcolor{dialinecolor}
% was here!!!
\pgfsetarrowsend{stealth}
\definecolor{dialinecolor}{rgb}{0.000000, 0.000000, 0.000000}
\pgfsetstrokecolor{dialinecolor}
\draw (9.288661\du,10.645714\du)--(12.475000\du,10.588125\du);
}
% setfont left to latex
\definecolor{dialinecolor}{rgb}{0.000000, 0.000000, 0.000000}
\pgfsetstrokecolor{dialinecolor}
\node[anchor=west] at (1.950000\du,10.700000\du){};
% setfont left to latex
\definecolor{dialinecolor}{rgb}{0.000000, 0.000000, 0.000000}
\pgfsetstrokecolor{dialinecolor}
\node[anchor=west] at (2.350000\du,10.600000\du){};
% setfont left to latex
\definecolor{dialinecolor}{rgb}{0.000000, 0.000000, 0.000000}
\pgfsetstrokecolor{dialinecolor}
\node[anchor=west] at (6.900000\du,10.750000\du){e1};
% setfont left to latex
\definecolor{dialinecolor}{rgb}{0.000000, 0.000000, 0.000000}
\pgfsetstrokecolor{dialinecolor}
\node[anchor=west] at (2.350000\du,10.600000\du){};
\pgfsetlinewidth{0.100000\du}
\pgfsetdash{}{0pt}
\pgfsetdash{}{0pt}
\pgfsetbuttcap
\pgfsetmiterjoin
\pgfsetlinewidth{0.100000\du}
\pgfsetbuttcap
\pgfsetmiterjoin
\pgfsetdash{}{0pt}
\definecolor{dialinecolor}{rgb}{1.000000, 1.000000, 1.000000}
\pgfsetfillcolor{dialinecolor}
\fill (12.425000\du,9.793125\du)--(12.425000\du,11.793125\du)--(14.360484\du,11.793125\du)--(14.360484\du,9.793125\du)--cycle;
\definecolor{dialinecolor}{rgb}{0.000000, 0.000000, 0.000000}
\pgfsetstrokecolor{dialinecolor}
\draw (12.425000\du,9.793125\du)--(12.425000\du,11.793125\du)--(14.360484\du,11.793125\du)--(14.360484\du,9.793125\du)--cycle;
\pgfsetbuttcap
\pgfsetmiterjoin
\pgfsetdash{}{0pt}
\definecolor{dialinecolor}{rgb}{0.000000, 0.000000, 0.000000}
\pgfsetstrokecolor{dialinecolor}
\draw (12.425000\du,9.793125\du)--(12.425000\du,11.793125\du)--(14.360484\du,11.793125\du)--(14.360484\du,9.793125\du)--cycle;
\pgfsetlinewidth{0.100000\du}
\pgfsetdash{}{0pt}
\pgfsetdash{}{0pt}
\pgfsetbuttcap
\pgfsetmiterjoin
\pgfsetlinewidth{0.100000\du}
\pgfsetbuttcap
\pgfsetmiterjoin
\pgfsetdash{}{0pt}
\definecolor{dialinecolor}{rgb}{1.000000, 1.000000, 1.000000}
\pgfsetfillcolor{dialinecolor}
\fill (14.345753\du,9.798125\du)--(14.345753\du,11.798125\du)--(16.281237\du,11.798125\du)--(16.281237\du,9.798125\du)--cycle;
\definecolor{dialinecolor}{rgb}{0.000000, 0.000000, 0.000000}
\pgfsetstrokecolor{dialinecolor}
\draw (14.345753\du,9.798125\du)--(14.345753\du,11.798125\du)--(16.281237\du,11.798125\du)--(16.281237\du,9.798125\du)--cycle;
\pgfsetbuttcap
\pgfsetmiterjoin
\pgfsetdash{}{0pt}
\definecolor{dialinecolor}{rgb}{0.000000, 0.000000, 0.000000}
\pgfsetstrokecolor{dialinecolor}
\draw (14.345753\du,9.798125\du)--(14.345753\du,11.798125\du)--(16.281237\du,11.798125\du)--(16.281237\du,9.798125\du)--cycle;
\pgfsetlinewidth{0.100000\du}
\pgfsetdash{}{0pt}
\pgfsetdash{}{0pt}
\pgfsetbuttcap
{
\definecolor{dialinecolor}{rgb}{0.000000, 0.000000, 0.000000}
\pgfsetfillcolor{dialinecolor}
% was here!!!
\pgfsetarrowsend{stealth}
\definecolor{dialinecolor}{rgb}{0.000000, 0.000000, 0.000000}
\pgfsetstrokecolor{dialinecolor}
\draw (15.681403\du,10.838839\du)--(18.867742\du,10.781250\du);
}
% setfont left to latex
\definecolor{dialinecolor}{rgb}{0.000000, 0.000000, 0.000000}
\pgfsetstrokecolor{dialinecolor}
\node[anchor=west] at (13.292742\du,10.943125\du){e1};
\pgfsetlinewidth{0.100000\du}
\pgfsetdash{}{0pt}
\pgfsetdash{}{0pt}
\pgfsetbuttcap
\pgfsetmiterjoin
\pgfsetlinewidth{0.100000\du}
\pgfsetbuttcap
\pgfsetmiterjoin
\pgfsetdash{}{0pt}
\definecolor{dialinecolor}{rgb}{1.000000, 1.000000, 1.000000}
\pgfsetfillcolor{dialinecolor}
\fill (18.375000\du,9.743125\du)--(18.375000\du,11.743125\du)--(20.310484\du,11.743125\du)--(20.310484\du,9.743125\du)--cycle;
\definecolor{dialinecolor}{rgb}{0.000000, 0.000000, 0.000000}
\pgfsetstrokecolor{dialinecolor}
\draw (18.375000\du,9.743125\du)--(18.375000\du,11.743125\du)--(20.310484\du,11.743125\du)--(20.310484\du,9.743125\du)--cycle;
\pgfsetbuttcap
\pgfsetmiterjoin
\pgfsetdash{}{0pt}
\definecolor{dialinecolor}{rgb}{0.000000, 0.000000, 0.000000}
\pgfsetstrokecolor{dialinecolor}
\draw (18.375000\du,9.743125\du)--(18.375000\du,11.743125\du)--(20.310484\du,11.743125\du)--(20.310484\du,9.743125\du)--cycle;
\pgfsetlinewidth{0.100000\du}
\pgfsetdash{}{0pt}
\pgfsetdash{}{0pt}
\pgfsetbuttcap
\pgfsetmiterjoin
\pgfsetlinewidth{0.100000\du}
\pgfsetbuttcap
\pgfsetmiterjoin
\pgfsetdash{}{0pt}
\definecolor{dialinecolor}{rgb}{1.000000, 1.000000, 1.000000}
\pgfsetfillcolor{dialinecolor}
\fill (20.295753\du,9.748125\du)--(20.295753\du,11.748125\du)--(22.231237\du,11.748125\du)--(22.231237\du,9.748125\du)--cycle;
\definecolor{dialinecolor}{rgb}{0.000000, 0.000000, 0.000000}
\pgfsetstrokecolor{dialinecolor}
\draw (20.295753\du,9.748125\du)--(20.295753\du,11.748125\du)--(22.231237\du,11.748125\du)--(22.231237\du,9.748125\du)--cycle;
\pgfsetbuttcap
\pgfsetmiterjoin
\pgfsetdash{}{0pt}
\definecolor{dialinecolor}{rgb}{0.000000, 0.000000, 0.000000}
\pgfsetstrokecolor{dialinecolor}
\draw (20.295753\du,9.748125\du)--(20.295753\du,11.748125\du)--(22.231237\du,11.748125\du)--(22.231237\du,9.748125\du)--cycle;
% setfont left to latex
\definecolor{dialinecolor}{rgb}{0.000000, 0.000000, 0.000000}
\pgfsetstrokecolor{dialinecolor}
\node[anchor=west] at (19.242742\du,10.893125\du){e1};
\pgfsetlinewidth{0.100000\du}
\pgfsetdash{}{0pt}
\pgfsetdash{}{0pt}
\pgfsetmiterjoin
\pgfsetbuttcap
{
\definecolor{dialinecolor}{rgb}{0.000000, 0.000000, 0.000000}
\pgfsetfillcolor{dialinecolor}
% was here!!!
\pgfsetarrowsend{stealth}
{\pgfsetcornersarced{\pgfpoint{0.000000\du}{0.000000\du}}\definecolor{dialinecolor}{rgb}{0.000000, 0.000000, 0.000000}
\pgfsetstrokecolor{dialinecolor}
\draw (21.625000\du,10.813125\du)--(21.275000\du,10.813125\du)--(21.275000\du,18.113125\du)--(17.025000\du,18.113125\du);
}}
\pgfsetlinewidth{0.100000\du}
\pgfsetdash{}{0pt}
\pgfsetdash{}{0pt}
\pgfsetbuttcap
\pgfsetmiterjoin
\pgfsetlinewidth{0.100000\du}
\pgfsetbuttcap
\pgfsetmiterjoin
\pgfsetdash{}{0pt}
\definecolor{dialinecolor}{rgb}{1.000000, 1.000000, 1.000000}
\pgfsetfillcolor{dialinecolor}
\fill (10.482258\du,17.018125\du)--(10.482258\du,19.018125\du)--(12.417742\du,19.018125\du)--(12.417742\du,17.018125\du)--cycle;
\definecolor{dialinecolor}{rgb}{0.000000, 0.000000, 0.000000}
\pgfsetstrokecolor{dialinecolor}
\draw (10.482258\du,17.018125\du)--(10.482258\du,19.018125\du)--(12.417742\du,19.018125\du)--(12.417742\du,17.018125\du)--cycle;
\pgfsetbuttcap
\pgfsetmiterjoin
\pgfsetdash{}{0pt}
\definecolor{dialinecolor}{rgb}{0.000000, 0.000000, 0.000000}
\pgfsetstrokecolor{dialinecolor}
\draw (10.482258\du,17.018125\du)--(10.482258\du,19.018125\du)--(12.417742\du,19.018125\du)--(12.417742\du,17.018125\du)--cycle;
\pgfsetlinewidth{0.100000\du}
\pgfsetdash{}{0pt}
\pgfsetdash{}{0pt}
\pgfsetbuttcap
\pgfsetmiterjoin
\pgfsetlinewidth{0.100000\du}
\pgfsetbuttcap
\pgfsetmiterjoin
\pgfsetdash{}{0pt}
\definecolor{dialinecolor}{rgb}{1.000000, 1.000000, 1.000000}
\pgfsetfillcolor{dialinecolor}
\fill (12.422366\du,17.063125\du)--(12.422366\du,19.023125\du)--(14.319140\du,19.023125\du)--(14.319140\du,17.063125\du)--cycle;
\definecolor{dialinecolor}{rgb}{0.000000, 0.000000, 0.000000}
\pgfsetstrokecolor{dialinecolor}
\draw (12.422366\du,17.063125\du)--(12.422366\du,19.023125\du)--(14.319140\du,19.023125\du)--(14.319140\du,17.063125\du)--cycle;
\pgfsetbuttcap
\pgfsetmiterjoin
\pgfsetdash{}{0pt}
\definecolor{dialinecolor}{rgb}{0.000000, 0.000000, 0.000000}
\pgfsetstrokecolor{dialinecolor}
\draw (12.422366\du,17.063125\du)--(12.422366\du,19.023125\du)--(14.319140\du,19.023125\du)--(14.319140\du,17.063125\du)--cycle;
% setfont left to latex
\definecolor{dialinecolor}{rgb}{0.000000, 0.000000, 0.000000}
\pgfsetstrokecolor{dialinecolor}
\node[anchor=west] at (10.950000\du,18.368125\du){ei};
% setfont left to latex
\definecolor{dialinecolor}{rgb}{0.000000, 0.000000, 0.000000}
\pgfsetstrokecolor{dialinecolor}
\node[anchor=west] at (15.882258\du,18.213125\du){.....};
\pgfsetlinewidth{0.100000\du}
\pgfsetdash{}{0pt}
\pgfsetdash{}{0pt}
\pgfsetmiterjoin
\pgfsetbuttcap
{
\definecolor{dialinecolor}{rgb}{0.000000, 0.000000, 0.000000}
\pgfsetfillcolor{dialinecolor}
% was here!!!
\pgfsetarrowsend{stealth}
{\pgfsetcornersarced{\pgfpoint{0.000000\du}{0.000000\du}}\definecolor{dialinecolor}{rgb}{0.000000, 0.000000, 0.000000}
\pgfsetstrokecolor{dialinecolor}
\draw (13.338124\du,19.072678\du)--(13.232258\du,22.413125\du);
}}
\pgfsetlinewidth{0.100000\du}
\pgfsetdash{}{0pt}
\pgfsetdash{}{0pt}
\pgfsetbuttcap
\pgfsetmiterjoin
\pgfsetlinewidth{0.100000\du}
\pgfsetbuttcap
\pgfsetmiterjoin
\pgfsetdash{}{0pt}
\definecolor{dialinecolor}{rgb}{1.000000, 1.000000, 1.000000}
\pgfsetfillcolor{dialinecolor}
\fill (10.482258\du,22.068125\du)--(10.482258\du,24.068125\du)--(12.417742\du,24.068125\du)--(12.417742\du,22.068125\du)--cycle;
\definecolor{dialinecolor}{rgb}{0.000000, 0.000000, 0.000000}
\pgfsetstrokecolor{dialinecolor}
\draw (10.482258\du,22.068125\du)--(10.482258\du,24.068125\du)--(12.417742\du,24.068125\du)--(12.417742\du,22.068125\du)--cycle;
\pgfsetbuttcap
\pgfsetmiterjoin
\pgfsetdash{}{0pt}
\definecolor{dialinecolor}{rgb}{0.000000, 0.000000, 0.000000}
\pgfsetstrokecolor{dialinecolor}
\draw (10.482258\du,22.068125\du)--(10.482258\du,24.068125\du)--(12.417742\du,24.068125\du)--(12.417742\du,22.068125\du)--cycle;
\pgfsetlinewidth{0.100000\du}
\pgfsetdash{}{0pt}
\pgfsetdash{}{0pt}
\pgfsetbuttcap
\pgfsetmiterjoin
\pgfsetlinewidth{0.100000\du}
\pgfsetbuttcap
\pgfsetmiterjoin
\pgfsetdash{}{0pt}
\definecolor{dialinecolor}{rgb}{1.000000, 1.000000, 1.000000}
\pgfsetfillcolor{dialinecolor}
\fill (12.422366\du,22.113125\du)--(12.422366\du,24.073125\du)--(14.319140\du,24.073125\du)--(14.319140\du,22.113125\du)--cycle;
\definecolor{dialinecolor}{rgb}{0.000000, 0.000000, 0.000000}
\pgfsetstrokecolor{dialinecolor}
\draw (12.422366\du,22.113125\du)--(12.422366\du,24.073125\du)--(14.319140\du,24.073125\du)--(14.319140\du,22.113125\du)--cycle;
\pgfsetbuttcap
\pgfsetmiterjoin
\pgfsetdash{}{0pt}
\definecolor{dialinecolor}{rgb}{0.000000, 0.000000, 0.000000}
\pgfsetstrokecolor{dialinecolor}
\draw (12.422366\du,22.113125\du)--(12.422366\du,24.073125\du)--(14.319140\du,24.073125\du)--(14.319140\du,22.113125\du)--cycle;
% setfont left to latex
\definecolor{dialinecolor}{rgb}{0.000000, 0.000000, 0.000000}
\pgfsetstrokecolor{dialinecolor}
\node[anchor=west] at (10.950000\du,23.418125\du){en};
% setfont left to latex
\definecolor{dialinecolor}{rgb}{0.000000, 0.000000, 0.000000}
\pgfsetstrokecolor{dialinecolor}
\node[anchor=west] at (12.570753\du,23.343125\du){NULL};
\pgfsetlinewidth{0.100000\du}
\pgfsetdash{}{0pt}
\pgfsetdash{}{0pt}
\pgfsetbuttcap
{
\definecolor{dialinecolor}{rgb}{0.000000, 0.000000, 0.000000}
\pgfsetfillcolor{dialinecolor}
% was here!!!
\pgfsetarrowsend{stealth}
\definecolor{dialinecolor}{rgb}{0.000000, 0.000000, 0.000000}
\pgfsetstrokecolor{dialinecolor}
\draw (15.632258\du,18.213125\du)--(14.319140\du,18.043125\du);
}
% setfont left to latex
\definecolor{dialinecolor}{rgb}{0.000000, 0.000000, 0.000000}
\pgfsetstrokecolor{dialinecolor}
\node[anchor=west] at (2.232258\du,12.513125\du){l};
\end{tikzpicture}

		La saisie de formulaire se traite grâce au PHP, cependant leurs création
		reste du HTML.\\
		deux méthodes, GET et POST.\\
		\subparagraph{GET} Accessible par l'URL. L'inconveignant étant la taille
		maximum de caractère (buffer clavier)
		\subparagraph{POST} Accessible grâce aux formulaires. L'inconveignant 
		étant de resaisir les informations en cas d'erreur (sauf si utilisation 
		de PHP ou JavaScript) \\
		Les champs name sont toujours associer à un nom de variable en PHP,
		toujours le renseigner, donner des noms claires.
		
		\subsection{Champ de texte monolignes}
		\lstinputlisting[language=html]{1.html}	
		\subsection{Champ de texte multilignes}
		Textarea, permet d'insérer plusieurs lignes de code, avec éventuellement 
		une scrollbar. (exemple: commentaire, message forum...)
		\lstinputlisting[language=html]{2.html}	
		\subsection{Champ de fichiers}
		Permet de poster un fichier sur le serveur, possibilité de limiter le 
		poids de fichier.\\
		ATTENTIION: Toujours vérifier l'extension de fichier et la taille du fichier!
		\lstinputlisting[language=html]{3.html}	
		\subsection{Champ caché}
		Permet de faire passer un champ que ne voit pas l'utilisateur, récupération 
		de la donnée passé grâce au PHP
		\lstinputlisting[language=html]{4.html}	
		\subsection{Bouton radio}
		Le bouton radio permet à l'utilisateur de ne choisir qu'un choix parmis 
		une liste de choix (exemple Monsieur/Madame)
		\lstinputlisting[language=html]{5.html}	
		\subsection{Bouton simple}
		\lstinputlisting[language=html]{6.html}	
		\subsection{Liste}
		\lstinputlisting[language=html]{7.html}	

	\section{En-tête de document (head)}
	La valise <head> contient de nombreux renseignement sur la page, sur l'auteur 
	ainsi que la plupart des scripts.
	\subsection{Titre}
		\lstinputlisting[language=html]{8.html}	
	\subsection{Insértion JavaScript}
		\lstinputlisting[language=html]{9.html}	
	\subsection{Style CSS}	
		\lstinputlisting[language=html]{10.html}	
	\subsection{Méta-données}
		La plupart sont utilisées pour le référencement.
		\lstinputlisting[language=html, caption="Encodage de caractère]{11.html}	
		\begin{itemize}
		\item meta description: description de la page
		\item meta keyword: mots clefs de la page
		\item meta rating: public visé
		\item meta robots: pours les bots de référencement
		\end{itemize}
		\lstinputlisting[language=html]{12.html}	
	\chapter{CSS}
		CSS =  pour compléter pour le HTML: permet de mettre en forme la page.
		Le HTML ne devrait être utilisé que pour le contenu et le html que pour
		le fond.\\
		\section{Insertion CSS}	
		Trois façon de l'insérer comme dit précédemment. 
		\subsection{FIchier à part}
		\lstinputlisting[language=html]{13.html}	
		\subsection{Dans l'en-tête}
		\lstinputlisting[language=html]{14.html}	
		\subsection{Dans le HTML}
		\lstinputlisting[language=html]{15.html}	
		oijoij
		\section{Syntaxe}
			Le CSS possède sa propre << syntaxe >>.\\
			Dans un CSS, on trouve trois élements différents: 
		%	\begin{itemize}
		%		\item Noms de balise
		%		\item Des propriétés CSS
		%		\item Des valeurs
		%	\end{itemize}
			\subsection{une Feuille de style}
			ijlkjlkj
%		\lstinputlisting[language=HTML]{1.html}	
%		\lstinputlisting[language=HTML]{2.html}	

\end{document}


