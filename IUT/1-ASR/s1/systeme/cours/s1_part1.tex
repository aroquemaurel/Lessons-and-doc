\part{Les commandes de bases}
	\chapter{Notion de commande}
		Une commande est une suite de mots séparés par au moins un espace.\\
		Le premier est le nom de la commande, suivi par une liste facultative d'options et d'arguments.
		\begin{verbatim}	
			commande [-options] [arg1 ... argn]
		\end{verbatim}
		Les majuscules et les minuscules  sont differenciées dans le système Unix.\\ \\

		Il est possible d'écrire plusieurs commandes sur la même ligne. Le séparateur de commandes est le ";"

		\section{Quelques exemples de commandes: }
			\begin{itemize}
				\item ls
				\item ls -l
				\item ls -l -a
				\item ls -la
				\item ls -al
				\item ls bidule
				\item ls -l bidule
			\end{itemize}
		\textbf{Remarques:} sur la présentation de la syntaxe des commandes, tout texte entre [ ] (crochets) est optionnel.
	\chapter{Obtenir de l'aide sous Unix}
		Il existe des informations  en ligne disponibles sur Unix.\\
		Une manière simple d'obtenir ces informations c'est d'utiliser la commande man\\
		Pour obtenir la correspondance des codes ASCII en octal et hexadecimal, tapez la commande : man ascii \\ \\

		Pour avoir la syntaxe de la commande man, tapez : man man 
	\chapter{La commande $man$}
		\section{Définition}
			Cette commande donne accès au manuel en ligne du système en vue d'obtenir de la documentation.
		\section{Syntaxe}
			\begin{verbatim}	
				man [section] nom de la commande
			\end{verbatim}
	\chapter{Les commandes de gestion de fichiers}
		Dans ce chapitre, nous allons aborder la notion de système de fichiers, pour cela nous donnerons la définition d'un fichier, les différents types de fichiers existant sous Unix ainsi que les chemins d'accès.\\
		Nous détaillerons aussi les commandes de gestion de fichiers et de catalogues.\\ \\
		Le chapitre se divise en :\\
		\begin{itemize}
			\item la définition d'un système de fichiers, 
			\item les principales commandes de manipulation de fichiers, 
			\item les principales commandes de manipulation de répertoires. 
		\end{itemize}
		\section{Le système de fichier}
			\subsection{Qu'est ce qu'un fichier?}
				Un fichier UNIX est une suite de caractères non structurée. UNIX n'a pas la notion d'organisation de fichier (indexée, relative, etc ...). 
				A tout fichier est attribué un bloc d'informations appelé noeud d'index ou i-node. Cet i-node contient des informations générales concernant le fichier: 
				\begin{itemize}
					\item sa taille (en octets),
					\item l'adresse des blocs utilisés par le fichier sur le disque,
					\item l'identification du propriétaire du fichier, 
					\item les droits d'accès des différents groupes d'utilisateurs, 
					\item le type du fichier, 
					\item un compteur de liens, 
					\item les dates des principales opérations (création, mise à jour, consultation).
				\end{itemize}
				\textbf{Remarque:} le i-node ne contient pas le nom du fichier.
			\subsection{Les types de fichiers}
				Dans le système UNIX il existe 3 types de fichiers:
				Les \textbf{fichiers ordinaires} peuvent être :
				\begin{itemize}
					\item des programmes exécutables (compilateurs, éditeurs, tableurs, ...) 
					\item des fichiers texte 
					\item des fichiers de données
				\end{itemize}\\					
				Il n'y a pas de format à respecter pour le nom des fichiers UNIX (jusqu'à 256 caractères).\\
				Les \textbf{fichiers spéciaux} : 
				Ce sont des fichiers associés à un dispositif d'entrée/sortie (E/S) physique. Ils sont traités par le système comme des fichiers ordinaires mais la 	lecture et l'écriture sur ces fichiers activent les mécanismes physiques associés (drivers).
				Il existe 2 types de fichiers spéciaux: 
				\begin{itemize}				
					\item mode caractére : E/S réalisées caractère par caractère (terminaux, imprimantes, lignes de communication, ...)
					\item mode bloc : E/S réalisées par blocs de caractères (disques, bandes).
				\end{itemize}
				Les \textbf{répertoires} : 
				Contiennent les couples (i-node, nom de fichier). On ne peut créer, effacer, lire ou écrire dans des répertoires qu'au moyen de primitives systèmes 	spécifiques.\\ 
				Les répertoires sont aussi appelés catalogues ou directories.
			\subsection{Conventions de nommage des répertoires}				
				. (point) désigne le répertoire courant.\\
				.. (point point) désigne le répertoire père du répertoire courant.\\
				/ (slash) désigne la racine de l'arborescence des fichiers.\\ \\

				Dans la désignation d'un chemin, c'est un séparateur de catalogue.\\
				~ (tilde) désigne le "home directory"  de l'utilisateur.
			\subsection{Chemin d'accès à un fichier ou à un répertoire}							
				Le chemin d'accès à un fichier (ou à un catalogue) est la description qui permet d'identifier le fichier (ou le catalogue) dans la hiérarchie du 	système.\\
				Le chemin d'accès correspond en une suite de noms de répertoires séparés par des caractères / (slash) et terminé par le nom du fichier ou du répertoire.\\ \\

				 Ainsi le chemin d'accès suivant : /users/fudmip/prof/.login \\
				représente le fichier .login qui se trouve dans le catalogue profcatalogue de connexion lui-même placé sous le catalogue fudmip, lui-même contenu dans le catalogue users qui se trouve sous la racine /.\\
				 \textbf{Remarque} : Le caractère / marque la séparation entre catalogues lorsqu'on décrit le "chemin d'accès" à un fichier ou un catalogue. 

		\section{Manipulation de fichiers}
			\subsection{Afficher la liste des fichiers d'un répértoire: $ls$}
				La commande $ls$ affiche la liste des fichiers du répertoire courant :\\
				\begin{verbatim}	
					$: ls
					boite fic fic2 Fich.c
				\end{verbatim}

			
			\subsection{Afficher le contenu d'un fichier: $cat$}
				\subsubsection{Définition}
					Cette commande permet d'afficher le contenu d'un fichier dont le nom est passé en paramètre.\\
					En réalité la commande cat concatène le contenu de tous les fichiers passés en arguments et envoi le résultat sur l'écran  \\ \\
	
					Elle permet aussi de créer un fichier en utilisant la redirection d'E/S (>).

				 \subsubsection{Syntaxe}
				 	\begin{verbatim}
				 		cat essai.c
				 		cat > truc
					\end{verbatim}
			\subsection{Afficher le contenu du fichier avec arrêt à chaque page: $more$}			
				\subsubsection{Définition}
					Cette commande permet d'afficher le contenu d'un fichier avec arrêt à chaque page.
					On peut alors utiliser : 
					la touche ESPACE pour passer à la page suivante,
					la touche h pour avoir l'aide, 
					la touche q ou \^D pour sortir de more.
					Remarque : la commande more fait partie de la famille des pagers , il en existe d'autres pg, less, ...
				\subsubsection{syntaxe}
					\begin{verbatim}
				 		more monFichier
					\end{verbatim}
			\subsection{Affichage partiel du contenu d'un fichier}			
				Les commandes suivantes permettent d'afficher à l'écran  une partie du contenu d'un fichier :
				\begin{itemize}
					\item les premières lignes : $head$
					\item les dernières lignes : $tail$
					\item certaines lignes : $grep$
					\item des parties de lignes : $cut$
				\end{itemize}
				\subsubsection{Les premières lignes d'un fichier : $head$}
					\paragraph{Définition}			
						Cette commande affiche les premières lignes d'un fichier.\\
						Des options permettent de modifier le nombre de lignes à afficher.
					\paragraph{syntaxe}					
						\begin{verbatim}
					 		head [-n] [fichier1] [fichier2 ...] 
						\end{verbatim}
			
				\subsubsection{Les dernières lignes d'un fichier : $tail$}
					\paragraph{Définition}			
					Cette commande affiche les dernières lignes   d'un fichier.\\
					Des options permettent d'en modifier le nombre par défaut.
					\paragraph{syntaxe}					
						\begin{verbatim}
					 		tail [+/-n] [fichier]
						\end{verbatim}
							
				\subsubsection{Certaines lignes d'un fichier : $grep$}
					\paragraph{Définition}			
						La commande grep affiche toutes les lignes d'un fichier contenant la chaîne de caractères spécifiée en argument.
						Il est possible d'utiliser des métacaractères  pour définir la chaîne à rechercher.
					\paragraph{syntaxe}					
						\begin{verbatim}
					 		grep  chaîne fichier
						\end{verbatim}
							
				\subsubsection{Une partie des lignes d'un fichier : $cut$}
					\paragraph{Définition}			
						La commande grep affiche toutes les lignes d'un fichier contenant la chaîne de caractères spécifiée en argument.
						Il est possible d'utiliser des métacaractères  pour définir la chaîne à rechercher.
					\paragraph{syntaxe}					
						\begin{verbatim}
					 		grep  chaîne fichier
						\end{verbatim}			
			\subsection{Copier un fichier : $cp$}
				\subsection{Définition}			
					Cette commande permet de copier des fichiers.\\
					L'option -R autorise la copie de catalogue.
				\subsection{syntaxe}					
					\begin{verbatim}
						cp   fic-source    fic-cible
						cp   fic-source   ktal-cible
						cp -R  ktal-source   ktal-cible
					\end{verbatim}						
	
			\subsection{Copier un fichier : $rm$}
				\subsubsection{Définition}			
					Cette commande permet de détruire les fichiers passés en paramètres.\\
					De nombreuses options de cette commande sont pratiques mais dangereuses !\\
					Par défaut, la commande rm ne demande aucune confirmation : les fichiers sont donc irrémédiablement perdus.
				\subsubsection{syntaxe}					
					\begin{verbatim}
						rm  -[Rfi] mon-fichier
						rm  -i mon-fichier
						rm  -f mon-fichier
						rm  -R mon-fichier
					\end{verbatim}								
			\subsection{Renommer ou déplacer un fichier : $mv$}
					\subsubsection{Définition}			
						Cette commande permet de déplacer ou de renommer un fichier ou un répertoire.
					\subsubsection{syntaxe}					
						\begin{verbatim}
							mv mon-fichier nouv-fichier
						\end{verbatim}								
				
			\subsection{Comparaison de 2 fichiers}
				UNIX met a disposition 2 commandes pour comparer le contenu de fichiers :
				\begin{itemize}
					\item cmp indique si les contenus des fichiers sont les mêmes.
					\item diff affiche les modifications à apporter pour les rendre identiques.
				\end{itemize}
			
					\subsection{Comparaison du contenu de 2 fichiers : $cmp$}			
						\subsubsection{Définition}			
							Cette commande permet de comparer le contenu de 2 fichiers.
							Elle affiche le numéro de ligne et de caractère de la première différence rencontrée.
						\subsubsection{syntaxe}					
							\begin{verbatim}
						 		cmp fichier1 fichier2
							\end{verbatim}		
					\subsection{Différence entre 2 fichiers : $diff$}			
						\subsubsection{Définition}			
							Cette commande permet d'afficher les différence entre deux fichiers.
						\subsubsection{syntaxe}					
							\begin{verbatim}
						 		diff fichier1 fichier2
							\end{verbatim}	
			\subsection{Trier des fichiers : sort}
					\subsection{Définition}		
						sort trie, regroupe ou compare toutes les lignes des fichiers indiqués. Si aucun fichier n'est fourni, ou si le nom `-' est mentionné, la lecture se fera depuis l'entrée standard.
					\subsection{syntaxe}					
						\begin{verbatim}
							sort [-cmus] [-t séparateur] [-o fichier_de_sortie] [-T répertoire_temporaire] [-bdfiMnr] [+POS1 [-POS2]] [-k POS1[,POS2]] [fichier...]  
						\end{verbatim}					
			\subsection{Compter les caractères, mots , lignes d'un fichier : $wc$}
					\subsection{Définition}	
						La commande wc compte les mots, les lignes et/ou caractères d'un fichier.
					\subsection{syntaxe}					
						\begin{verbatim}
							wc [-lcw] fichier
						\end{verbatim}					
			\subsection{Numéroter des lignes: $nl$}
					\subsection{Définition}	
						La commande nl numéro les lignes d'un fichier et l'affiche à l'écran
					\subsection{syntaxe}					
						\begin{verbatim}
							nl fichier
						\end{verbatim}														
		\section{Manipulation de répertoires}
		Les répertoires servent à ranger des fichiers et/ou catalogues.\\ \\
		L'espace de travail de l'utilisateur est une arborescence de répertoires et de fichiers.\\ \\
		Des commandes UNIX permettent d'organiser et de gérer cette hiérarchie ( créer, effacer, parcourir, ...)\\ \\
		Il y a un catalogue particulier à chaque usager: le catalogue personnel.\\ 
		Ce catalogue est le sommet de l'arborescence de l'espace de travail de l'utilisateur, c'est le catalogue dans lequel il est placé à la connexion (home directory).\\ \\

		Le catalogue de connexion est repéré par la variable d'environnement HOME, mais aussi par le caractère ~.\\
		Généralement le nom de votre répertoire personnel est identique à votre nom d'utilisateur.\\
		Les principales commandes sur les répertoires:\\
		\begin{itemize}
			\item Afficher le répertoire courant : $pwd$
			\item Se déplacer dans l'arborescence : $cd$
			\item Créer un répertoire : $mkdir$
			\item Effacer un répertoire : $rmdir$
			\item Copier un répertoire : $cp -R$
		\end{itemize}
			\subsection{Afficher le répertoire courant : $pwd$}
			Le répertoire courant est le catalogue dans lequel vous êtes en train de travailler.\\
			Initialement le répertoire courant est le catalogue de connexion.\\
			Il est nécessaire de connaître sa position dans l'arborescence du système à tout instant.
				\subsubsection{Définition}
					La commande pwd affiche à l'écran le chemin d'accès au catalogue courant.
				\subsubsection{Syntaxe}
					\begin{verbatim}				
					pwd
					\end{verbatim}
				\subsubsection{Remarque}
					Certains systèmes maintiennent une variable PWD qui contient le chemin d'accès au catalogue courant.\\
					En shell csh (tcsh, ...) la variable cwd contient aussi le catalogue courant.
			\subsection{Se Déplacer dans un répertoire : $cd$}
				\subsubsection{Définition}
					Cette commande permet de se déplacer dans l'arborescence des catalogues existants sur le système.
				\subsubsection{Syntaxe}
					\begin{verbatim}
						cd
						cd [ chemin relatif ] 
						cd [ chemin absolu ] 
					\end{verbatim}
			\subsection{Créer un répertoire : $mkdir$}
				\subsubsection{Définition}
					Cette commande permet de créer des répertoires, il faut bien sûr pouvoir le faire, c'est-à-dire être dans son espace de travail.
				\subsubsection{Syntaxe}
					\begin{verbatim}
						mkdir [-p] nom-répertoire
					\end{verbatim}					

			\subsection{Détruire un répertoire : $rmdir$}
				\subsubsection{Définition}
					La commande rmdir permet de détruire des catalogues vides.
				\subsubsection{Syntaxe}
					\begin{verbatim}
						rmdir [-f |-i] [-p] répertoire
						rm -R   répertoire
					\end{verbatim}				
			\subsection{Copier un répertoire : $cp$}
				\subsubsection{Définition}
					Il est possible de dupliquer le contenu d'un répertoire en utilisant la commande cp (copy) et l'option -r (récursive).\\
					De cette façon, tous les fichiers contenus dans tous les sous-répertoires du répertoire copié seront copiés également.
				\subsubsection{Syntaxe}
					\begin{verbatim}
						cp -r répertoire-a-copier nouveau-répertoire
					\end{verbatim}		
	\chapter{Notion d'utilisateur}
		Tout utilisateur est enregistré dans deux fichiers systèmes : \\
		/etc/passwd\\ 
		Ce fichier contient : 
		\begin{itemize}
			\item nom de login
			\item mot de passe chiffré
			\item numéro unique d'utilisateur (UID)
			\item numéro unique de groupe (GID)
			\item nom complet de l'utilisateur
			\item répertoire initial
			\item interpréteur de commande  
		\end{itemize}		\\ \\
		/etc/group\\
		Ce fichier contient :
		\begin{itemize}		 
			\item nom de groupe
			\item numéro unique de groupe (GID)
			\item liste des utilisateurs du groupe
		\end{itemize}
		\textbf{Remarque} : un utilisateur peut faire partie de plusieurs groupes.\\ \\
		Les notions d'utilisateurs et de groupes sont fondamentales pour l'attribution de droits d'accès aux fichiers.
		\section{Protection: Droits d'accès}
			\subsection{Permissions et contrôle d'accès}
				Un ensemble de permissions d'accès est associé à tout fichier; ces permissions déterminent qui peut accéder au fichier et comment:\\ \\
				\begin{tabular}{|c|c|c|c|}
					\hline
						\multicolumn{2}{|c}{Fichier} & \multicolumn{2}{|c|}{Répertoire} \\
					\hline
						r & Accès en lecture & r & Accès en lecture \\
					\hline
						w & Accès en écriture & w & Accès en création, modification et destruction\\
					\hline 
						x & Accès en exection & x & Accès au nom \\
					\hline 
				\end{tabular}\\ \\
				
				Pour accéder à une feuille ou à un noeud dans une arborescence de fichiers, il faut avoir la permission en x sur tous les répertoires de niveau supérieur (chemin d'accès).\\
				Il existe trois classes d'utilisateurs: \\ \\
				\begin{tabular}{|c|c|}
					\hline 
						u & Propriétaire \\
					\hline 
						g & groupe du propriétaire \\
					\hline 
						o & Les autres \\		
					\hline 
				\end{tabular}\\ \\
				Il existe quatre principaux types de fichiers :\\ \\
				\begin{tabular}{|c|c|}
					\hline 
						- & Fichier ordinaire \\
					\hline 
						d & Répertoire (directory) \\
					\hline 
						c & Fichier mode caractère \\		
					\hline 
						b & Fichier mode bloc \\		
					\hline 					
				\end{tabular}\\
				\subsubsection{Modification des droits d'accès : $chmod$}
					Les permissions d'accès peuvent être modifiées par la commande chmod\\
					Seul le propriétaire du fichier (ou répertoire) a la possibilité de modifier les droits d'accès sur son fichier.\\
					\paragraph{syntaxe}
						\begin{verbatim}
							chmod mode fichier [ou répertoire]
						\end{verbatim}						
				\subsubsection{Changement de propriétaire : $chown$}					
					Le propriétaire d'un fichier ou d'un répertoire peut être changé par la commande chown \\
					Seul le propriétaire du fichier (ou répertoire) et le SU peuvent effectuer ce changement.
					\paragraph{syntaxe}
						\begin{verbatim}
							chown NouveauPropriétaire fichier [ou répertoire]
						\end{verbatim}	

				\subsubsection{Changement de groupe : $chgrp$}					
					LeVous pouvez changer le groupe auquel s'applique les protections de niveau groupe par la commande chgrp \\
					Seul le propriétaire du fichier (ou répertoire) et le SU peuvent effectuer ce changement.
					\paragraph{syntaxe}
						\begin{verbatim}
							chgrp NouveauGroupe fichier [ou répertoire]
						\end{verbatim}	
						
				\subsubsection{Notion de masque : $umask$}					
					Tout fichier créé par le système d'exploitation a des droits d'accès par défaut.\\
					Par exemple sous HP-VUE un fichier texte a comme droits d'accès 666, un fichier exécutable ou un répertoire a pour droits d'accès 777.\\
					Pour éviter toute intrusion sur ces fichiers le fichier .login ou .profile (selon le Shell utilisé) contient souvent la commande umask 022 qui change les droits d'accès par défaut.\\ \\

					Ces nouveaux droits sont : \\
					\begin{itemize}
						\item pour un fichier texte, 644,
						\item pour un fichier exécutable ou un répertoire, 755.
						\item Pour les répertoires et les fichiers exécutables, les bits à 1 du masque invalident les droits correspondant.
						\item Inversement, les bits à 0 du masque autorisent les droits correspondant.
						\item Pour les fichiers texte, même application en enlevant en plus le droit en exécution.
					\end{itemize}
					Pour obtenir la valeur courante du masque taper la commande umask\\
					La valeur retournée est donnée en octal.\\ \\

					Bien entendu, la commande chmod ignore les masques
				\subsubsection{Notion de superutilisateur}
					Les restrictions d'accès s'appliquent aux utilisateurs.\\
					Un seul utilisateur est exempt de contrôles d'accès:\\
					le Super Utilisateur ==> login: root\\
	\chapter{Notion de procéssus}
		Un processus est un programme en cours d'exécution.\\ \\

		Un programme, produit par un éditeur de liens, est un fichier binaire exécutable mémorisé sur disque. \\
		Pour l'exécuter le système le charge en mémoire, il devient alors un processus.\\ \\

		Un processus est identifié au sein du système par un numéro unique, le PID.\\
		Les commandes de gestion de processus:\\
		\begin{itemize}
			\item Etat des processus actifs : $ps$
			\item Arreter un processus actif : $kill$
			\item Différentes façons de lancer un processus ($nohup$, $at time$, $batch$, $nice$).
		\end{itemize}
		\section{La commande $ps$}
			\subsection{Définition}
				Cette commande affiche la liste des processus actifs sur le système. \
				Il existe 2 types de processus : les processus systèmes qui accomplissent des services généraux et les processus utilisateurs. \\
				Par défaut, la commande ps n'affiche que les processus utilisateurs.\\
			\subsection{syntaxe}
				\begin{verbatim}
					ps [-u utilisateur] [-e] [-f]
				\end{verbatim}	

		\section{La commande $kill$}
			\subsection{Définition}
				Cette commande interrompt un processus en cours d'exécution.\\
				En réalité kill envoie un signal au processus spécifié.
			\subsection{syntaxe}
				\begin{verbatim}
					kill [-signal] PID 
				\end{verbatim}		

		\section{Différentes façon de lancer un processus}
			\subsection{Continuité d'exécution d'un processus : $nohup$}
				Lors de la déconnexion, les processus lancés par l'utilisateur qui s'exécutent encore sont tués automatiquement. Toutefois il est possible de 	prolonger l'exécution d'un processus en utilisant cette commande.

				\begin{verbatim}
					nohup commande [parametres]
				\end{verbatim}		

		\subsection{Exécution d'un processus en différé : $at time$, $batch$}
			at : permet de spécifier le moment de l'exécution. \\
				batch : la commande est mise en attente et est exécutée quand le système n'est pas surchargé
				\begin{verbatim}
					at time commande [parametres]
					batch commande [parametres]
				\end{verbatim}		

		\subsection{Exécution d'un processus avec priorité basse : $nice$}
			La commande nice permet de donner des priorités plus ou moins élevées pour l'exécution de processus selon l'importance de la tâche qu'ils remplissent.

				\begin{verbatim}
					nice [+/-nombre] commande [parametres]
	
					avec 1 (fort) <= nombre <= 19 (faible). 
					Par défaut, nombre est égal à 10.
				\end{verbatim}		
				\textbf{Remarque}: Seul le SU (Super Utilisateur) peut exécuter un processus en augmentant sa priorité. Un utilisateur normal ne peut que la diminuer.



		
			


						

