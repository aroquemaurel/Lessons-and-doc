\documentclass[12pt,a4paper,openany]{report}


\usepackage{lmodern}
\usepackage{xcolor}
\usepackage[utf8]{inputenc}
\usepackage[T1]{fontenc}
\usepackage[francais]{babel}
\usepackage[top=1.7cm, bottom=1.7cm, left=1.7cm, right=1.7cm]{geometry}
\usepackage{verbatim}
\usepackage{tikz} %Vectoriel
\usepackage{listings}
\usepackage{fancyhdr}
\usepackage{multido}
\usepackage{amssymb}

\newcommand{\titre}{Gestion des processus}

\newcommand{\module}{Gestion des processus}
\newcommand{\sigle}{gpr}

\newcommand{\semestre}{3}

\input{/home/satenske/cours/listings.tex} %prise en charge du langage algo
\input{/home/satenske/cours/entete_iut-cours.tex}

\begin{document}
	\maketitle
	\chapter{Contrôle des processus}
	\section{Caractéristiques d'un processus}
		\subsection{Mode de déroulement d'un processus}
		Un processus se déroule alternativement dans deux modes:
		\begin{itemize}
			\item Mode utilisateur
			\item Mode superviseur(ou mode privilégié)
		\end{itemize}
		Le passage du mode utilisateur au mode superviseur est provoqué par l'occurence d'un événement de type suivant
		\paragraph{événement internet synchrone} tel qu'un appel système
		\paragraph{éénement externe asynchrone} tel q'une interruption matérielle
		\subsection{Structure du programme objet}
\end{document}




