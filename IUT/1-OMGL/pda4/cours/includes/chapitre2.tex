\chapter{Scrum} \label{scrum}
\nouveauChapitre
	\section{Rôles}
		La méthode Scrum dispose de plusieurs rôles, chacun des rôles est important, en effet, comme toutes les méthodes Agiles l'humain est au centre du projet.
		\paragraph{Le Product Owner} C'est le client, il est extrememnt important, l'équipe de projet doit le voir régulièrement pour qu'il valide chacun des sprints\footnote{
		Les sprints sont définis section \ref{sprints} page \pageref{sprints}} afin de voir si ses besoins sont respectés. 
		\paragraph{Le Scrum master} Il aide l'équipe il va essayer d'aider au mieux le déroulement du projet et veiller au respect de Scrum, en cas de problème, c'est lui qu'il faut contacter.
		\paragraph{L'équipe} Ce sont toutes les personnes développant le projet. Elle ne doit pas trop nombreuse pour que les développeurs se sente impliqués dans le projet, et qu'ils puissent avoir
		une vision globale du projet.
	\section{Timeboxes}
		\paragraph{Timeboxes} 
		Une boîte de temps ou ``timebox'' est une période fixe pendant laquelle on cherche à exécuter le plus efficacement possible une ou plusieurs tâches.
	\section{Story}
		Une story est exprimée sous la forme suivante: \\
		En tant que \ldots\\
		Je désire \ldots\\
		Quand \ldots\\
		Afin de \ldots\\
		Cette story va avoir un certain nombre d'état. 
		\subsection{Éléments composants la story}
			\begin{itemize}
				\item Un nom
				\item Une description
				\item Un Type 
					\exemple{User, Default, Technique\ldots}
				\item Un État
					\remarque{Les états peuvent être les suivants, une story doit passer par tous les états.
						\begin{enumerate}
							\item Créé (Tout le monde)
							\item Acceptée (Product Owner)
							\item Estimée (Équipe)
							\item Planifiée (Équipe)
							\item En cours 
							\item Finie (Prend en compte les tests)
						\end{enumerate}
					}
				\item Un Poids 
					\exemple{1, 3, 5, 8, 13 \ldots \newline
						Le poids est basé sur la suite de Fibonacci}
				\item Un BVP\footnote{Business Value Point}, cela sert à connaître la priorités des stories
					\remarque{
						\begin{itemize}
							\item (900--1000) Must (indispensable)
							\item (700--800) Should 
							\item (400--600) Could
							\item (100--300) Would 
						\end{itemize}
						La priorité est donnée par le client.
					}
			\end{itemize}

	\section{Sprints}\label{sprints}
		Un sprint correspond à un incrément du projet. En effet, avec la méthode Scrum nous travaillons par incrément, le but étant d'avoir un programme fonctionnelle à la fin de chaque sprint. Ainsi, nous 
		devons pouvoir montrer une démonstration au Product Owner, chaque sprint devra être une surcouche du précédent et ainsi améliorer le logiciel, tout en restant totalement fonctionnel.

		Un sprint contient plusieurs stories qui devront être finies à la fin du sprint.
		\remarque{Une tâche est finie que lorsque celle-ci à été correctement testée, que celle-ci est fonctionnelle, et peut donc être montrée au client.}
		\newpage	
	\section{Revue}
		Une revue permet un feedback concret sur le produit (c'est mieux que sur de la documentation).

		Le but de la revue est de montrer ce qui a été réalisé pendant le sprint afin d'en tirer les conséquences pour la suite du projet.
		\subsection{Étapes}
			Voici les différentes étapes de la revue de sprint.
			\begin{itemize}
				\item Sprint basé sur une démo 
				\item Calculer la vélocité de l'équipe.
				\item Release 
				\item Ajouter les autres sprints et la release
			\end{itemize}
			À la fin des plans de sprint et des plans de release, on calcule: 
			\begin{itemize}
				\item La vélocité
				\item La capacité
			\end{itemize}
	\section{Artefacts}
		\subsection{Backlog produits}
			Les équipes agiles ne produisent pas une documentation faite au début du projet, qui décrit en détail toutes les spécifications fonctionnelles. 
			Elles collectent les fonctions essentielles (les features) et les raffinent progressivement. Il n'y a pas un gros document de spécification, l'outil de collecte s'appelle le backlog de produit.
		\subsubsection{Objectif}
			Le backlog de produit est la liste des fonctionnalités attendues d'un produit. Plus exactement, au-delà de cet aspect fonctionnel, 
			il contient tous les éléments qui vont nécessiter du travail pour l'équipe. Les éléments y sont classés par priorité ce qui permet de définir l'ordre de réalisation.
		\subsubsection{Rôle}
			Le backlog de produit est sous la responsabilité du product owner. Chacun peut contribuer à collecter des éléments, mais c'est le product owner qui les accepte finalement et c'est lui qui définit les priorités.
		\subsection{Plan de sprint}
			Les plans de sprints correspondent à ce qui est prévu pour le sprint. C'est-à-dire les différentes stories que nous souhaitons développer pour l'increment. 
			Ces stories doivent être regouprés en fonction de leurs poids et de leurs BVP.
		\subsection{Plan de releases}
			Le plan de release correspond au plan de la globalité du projet. C'est-à-dire les plans de chaque sprints.
		\subsection{Burndown}
			L'équipe affiche en grand format, sur un des murs de son local, un graphe représentant la quantité de travail restant à effectuer (sur l'axe vertical) rapportée au temps (sur l'axe horizontal). 
			C'est un "radiateur d'information".

			Ce graphe peut concerner l'itération en cours (``iteration burndown'') ou plus couramment l'ensemble du projet (``product burndown'').

