\chapter{Sprint 3}
Avant le commencement de sprint, nous avons choisis de ré effectuer un planning poker,
afin de confirmer les poids des différents \USs{}. Nous avons donc élevé le poids 
de la \story{Surveiller}.

Nous avons prévu 7 User Stories, contenant 1 must, en effet, une \US à été basculé sur le sprint
courant\footnote{cf section \ref{USsurveiller} \pageref{USsurveiller}}, cependant elle en été à 60\%
d'avancement. 

Également, nous avions les deux nouvelles \USs{} demandés par le product owner durant le sprint 2, 
celle-ci était composé d'un Must et d'un Should, nous devions donc faire attention pour le Must.

Heureusement, 3 \US{} se ressemblait beaucoup, et était donc assez rapide à faire une fois
qu'une \US{} était effectuée, nous avions également la \story{Initialisation des enseignants} qui
avait un poids de 40, que nous avons choisis de ne pas descendre la story étant compliquée.
Celle-ci étant Could, nous nous sommes donc mis d'accord en début de sprint de d'abord effectuer les
\USs{} importantes à faire, et d'avoir un logiciel stable, étant donné que cette \USs{} aller donner
une revue de release. Si nous avions du temps, nous nous serions occupés de cette story, nous n'avons
malheureusement pas eu le temps d'effectuer cette dernière.

\section{\USs{} prévues}
\subsection{Quota surveillance}		
Cette \US{} à été effectué lors du sprint 1, cf section \ref{usQuotaSurveillance} page
\pageref{usQuotaSurveillance}.

\subsection{Surveiller}
Cette \US à été basculé sur le sprint courant étant donné que nous n'avons pus la finir précédemment. Cf
section \ref{USsurveiller}.

\begin{tabular}{ll}
	\textbf{En tant que}	&	Enseignant \\
	\textbf{Je désire}&	M'ajouter à la liste des personnes d'un contrôle\\
	\textbf{Afin de}	& Surveiller\\
	&\\
	\textbf{BVP} & Should\\
	\textbf{Poids} & 5\\
\end{tabular}

\subsection{Initialisation des enseignants}		
Cette \US{} n'a pus être effectuée par manque de temps, son poids étant important.

\begin{tabular}{ll}
	\textbf{En tant que}	&	Responsable des plannings\\
	\textbf{Je désire}& récupérer la liste des enseignants du département à partir d'un fichier Excel\textregistered\\	
	\textbf{Afin de}	& Initialiser une partie de la base de donnée\\	
	&\\
	\textbf{BVP} & Could\\
	\textbf{Poids} & 40\\
\end{tabular}
\subsection{Imprimer la liste des contrôles}
Celle-ci à un poids de 20 car cela demandera de la documentation, une fois que nous saurons imprimer
les autres \USs{} serons beaucoup plus simple, elles ont donc un poids de 8.

\begin{tabular}{ll}
	\textbf{En tant que}	&	Tout le monde \\
	\textbf{Je désire}&	imprimer la liste des contrôles de l'année universitaire en cours\\
	\textbf{Afin de}	& l'afficher\\
	&\\
	\textbf{BVP} & Would\\
	\textbf{Poids} & 20\\
\end{tabular}

\subsection{Imprimer liste de mes contrôles surveillés}
\begin{tabular}{ll}
	\textbf{En tant que}	&	Enseignant \\
	\textbf{Je désire}&	Imprimer la liste effective des contrôles surveillés avec le détail des heures\\
	&de surveillance et le cumul\\
	\textbf{Afin de}	&	pouvoir archiver à un instant donné mes surveillances\\
	&\\
	\textbf{BVP} & Must\\
	\textbf{Poids} & 8\\
\end{tabular}
\subsection{Imprimer liste effective des contrôles de tous les enseignants}
\begin{tabular}{ll}
	\textbf{En tant que}	&	Responsable de planning\\
	\textbf{Je désire}&	imprimer la liste effective des contrôles surveillés avec le détail des heures de\\
	& surveillance et de cumul de tous les enseignants triés soit par ordre alphabétique\\
	& soit par cumul croissant\\
	\textbf{Afin de}	& Archiver à un instant donné toutes les surveillances\\
	&\\
	\textbf{BVP} & Should\\
	\textbf{Poids} & 8\\
\end{tabular}



\section{Difficultés du sprint}
La difficulté du sprint était surtout le nombre de \USs{} à effectuer. Pour palier à ce problème, nous en 
avons mis une de côté, afin d'avoir un logiciel fonctionnel, avec une \US{} en moins plutôt qu'avoir
un logiciel ayant toutes les \USs{} mais ayant des bugs, ce que le product owner ne souhaite surtout pas.

\section{Bilan du sprint}
Nous avons donc pus effectuer toutes les \US{} que nous voulions au début du sprint. L'impression qui
pouvait poser des problème n'a posé aucun problème, celui-ci étant de se documenter, nous avons pus
trouver comment imprimer. 

À la fin de ce sprint, nous avons effectués une revue de release qui s'est passé sans encombre.

