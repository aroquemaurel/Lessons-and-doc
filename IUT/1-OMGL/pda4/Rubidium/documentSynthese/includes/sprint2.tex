\chapter{Sprint 2}
Avant le commencement de sprint, nous avons choisis de ré effectuer un planning poker,
afin de confirmer les poids des différents \USs{}.

Nous avons prévu 5 User Stories, contenant 2 must, il fallait donc faire attention, 
cependant, l'équipe s'étant habitués aux technologies, cela semblé faisable. 

Nous avons donc effectués les must sans soucis, mais nous avions sous-estimé la 
\story{surveiller}\footnote{cf section \ref{USsurveiller} page \pageref{USsurveiller}}, 
ainsi, nous n'avons pus la finir et l'avons basculé sur le sprint 3.

\section{\USs{} prévues}
\subsection{Quota surveillance}		
Cette \US{} à été effectué lors du sprint 1, cf section \ref{usQuotaSurveillance} page
\pageref{usQuotaSurveillance}.

\begin{tabular}{ll}
	\textbf{En tant que}	&	Enseignant \\
	\textbf{Je désire}&	Connaitre le nombre d'heures de surveillance effective\\
	\textbf{Afin de}	&	savoir si mon quota est atteint\\
	&\\
	\textbf{BVP} & Should\\
	\textbf{Poids} & 2\\
\end{tabular}
\subsection{Mise à jour des contrôles}		
\begin{tabular}{ll}
	\textbf{En tant que}	&	Responsable de matière\\
	\textbf{Je désire}& Modifier les informations d'un contrôle\\	
	\textbf{Afin de}	& Le mettre à jour\\	
	&\\
	\textbf{BVP} & Could\\
	\textbf{Poids} & 8\\
\end{tabular}
\subsection{Initialiser contrôles}
\begin{tabular}{ll}
	\textbf{En tant que}	&	Responsable de matière\\
	\textbf{Je désire}&	saisir en début d'année l'ensemble des contrôles\\
	\textbf{Afin de}	& planifier le contrôle continu	\\
	&\\
	\textbf{BVP} & Must\\
	\textbf{Poids} & 3\\
\end{tabular}

\subsection{Consulter contrôle par semestre}
\begin{tabular}{ll}
	\textbf{En tant que}	&	Enseignant \\
	\textbf{Je désire}&	Consulter la liste des contrôles par semestre\\
	\textbf{Afin de}	&	M'assurer de la bonne planification\\
	&\\
	\textbf{BVP} & Must\\
	\textbf{Poids} & 8\\
\end{tabular}

\subsection{Surveiller}\label{USsurveiller}
Cette \US{} n'a pus être finis durant ce sprint, en effet, nous avions sous estimé son poids, et nous
n'avons donc pas réussis à la finir avant la revue de sprint.

Nous l'avons donc basculé sur le sprint suivant.

\begin{tabular}{ll}
	\textbf{En tant que}	&	Enseignant \\
	\textbf{Je désire}&	M'ajouter à la liste des personnes d'un contrôle\\
	\textbf{Afin de}	& Surveiller\\
	&\\
	\textbf{BVP} & Should\\
	\textbf{Poids} & 5\\
\end{tabular}

\section{Difficultés du sprint}
La principale difficulté à été de ne pas avoir bien estimé le poids des \USs{}, ainsi l'équipe été
soumise à pression afin de pouvoir finir toutes les \USs{} à temps, cela n'ayant pas été possible, il 
était plus judicieux de décaler une \USs{} d'un sprint. 

Également, en cours de sprint le product owner nous à donné deux nouvelles \USs{}, nous avons donc dû les 
planifier, étant donné que nous avions déjà beaucoup de travail, nous les avons planifiés pour le 
troisième sprint.
\section{Bilan du sprint}
Ce sprint à permis d'avoir un logiciel répondant à la plupart des must exigés par le product owner.

Également, l'équipe est devenue plus soudée.

