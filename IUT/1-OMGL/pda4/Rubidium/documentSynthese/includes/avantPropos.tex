\chapter*{Avant-propos}
\addcontentsline{toc}{chapter}{Avant-propos}
L'enseignement Agile a été mis en \oe{}uvre à travers la création d'une application de gestion de 
surveillance d'examens, ce logiciel fut baptisé ``\textit{Rubidium}''.

Ce logiciel doit permettre aux enseignants de s'auto-affecter à la surveillance de partiels\footnote{
En effet, ils ont un quota d'heure de surveillance à respecter, ce logiciel permettra de les aider à savoir où ils en sont dans ce quota}, aux responsables de matières de créer et d'éditer des partiels et aux administrateurs de gérer en intégralité l'organisation de ces derniers.

Nous avons décidé de développer cette application en C++ grâce à la bibliothèque Qt(Logo figure \ref{fig:qt}) pour des raisons pratiques\footnote{La flexibilité du code notamment, mais également une rapidité d'exécution, et enfin, le fait qu'elle soit multi-plateforme.} et du fait de l'expérience de certains membres du groupe dans cette bibliothèque.
\begin{figure}[H]
	\begin{center}
		\includegraphics[width=5cm]{images/qt.png}
	\end{center}
	\caption{Logo de Qt}
	\label{fig:qt}
\end{figure}

Dans ce document, nous détaillons nos avancés du projet, sprint par sprint, avec nos différents outils 
respectant la méthode Scrum.

Ce projet à commencé le 09 Février 2012, et la première revue de release, clôturant notre enseignement
Agile à été effectué vendre 30 Mars 2012. Ainsi en deux mois, nous avons pus présenter une version
du logiciel pouvant être utilisé.

