\documentclass{article}


\usepackage{lmodern}
\usepackage{xcolor}
\usepackage[utf8]{inputenc}
\usepackage[T1]{fontenc}
\usepackage[francais]{babel}
\usepackage[top=1.7cm, bottom=1.7cm, left=1.7cm, right=1.7cm]{geometry}
\usepackage{verbatim}
\usepackage{tikz} %Vectoriel
\usepackage{listings}
\usepackage{fancyhdr}
\usepackage{multido}
\usepackage{amssymb}

\newcommand{\titre}{Langage de Mise-à-jour des données}
\newcommand{\numTD}{2}

\newcommand{\module}{Base de données}
\newcommand{\sigle}{bdd}

\newcommand{\semestre}{3}

\input{/home/satenske/cours/listings.tex} %prise en charge du langage algo

\usepackage{ifthen}
\date{\today}

\chead{}
\rhead{TD\no\numTD}
\lhead{\titre}
%\makeindex

\lfoot{Université Paul Sabatier Toulouse III}
\rfoot{\sigle\semestre}
%\rfoot{}
\cfoot{--~~\thepage~~--}

\makeglossary
\makeatletter
\def\clap#1{\hbox to 0pt{\hss #1\hss}}%

\def\haut#1#2#3{%
	\hbox to \hsize{%
		\rlap{\vtop{\raggedright #1}
	}%
	\hss
	\clap{\vtop{\centering #2}
}%
\hss
\llap{\vtop{\raggedleft #3}}}}%
\def\bas#1#2#3{%
	\hbox to \hsize{%
		\rlap{\vbox{
			\raggedright #1
		}
	}%
	\hss \clap{\vbox{\centering #2}}%
	\hss
	\llap{\vbox{\raggedleft #3}}}
}%
\def\maketitle{%
	\thispagestyle{empty}{%
		\haut{}{\@blurb}{}
		%	
		%\vfill

		\begin{center}
			\vspace{-1.5cm}
			\usefont{OT1}{ptm}{m}{n}
			\huge \@numeroTD \@title
		\end{center}
		\par
		\hrule height 1pt
		\par
		\vspace{1cm}
		\bas{}{}{}
}%
}
\def\date#1{\def\@date{#1}}
\def\author#1{\def\@author{#1}}
\def\numeroTD#1{\def\@numeroTD{#1}}
\def\title#1{\def\@title{#1}}
\def\location#1{\def\@location{#1}}
\def\blurb#1{\def\@blurb{#1}}
\date{\today}
\newboolean{monBool}
\setboolean{monBool}{true}
\author{}
\title{}
\ifthenelse{\equal{\numTD}{}}{
\numeroTD{}
}
{
	\numeroTD{TD \no\numTD~--- }
}
\location{Amiens}\blurb{}
%\makeatother
\title{\titre}
\author{%Semestre \semestre
}

\location{Toulouse}
\blurb{%
\vspace{-35px}
\begin{flushleft}
	Université Paul Sabatier -- Toulouse III\\
	IUT A - Toulouse Rangueil\\
\end{flushleft}
\begin{flushright}
	\vspace{-45px}
	\Large \textbf \module \\
	\normalsize \textit \today\\
	Semestre \semestre
	\vspace{30px}
\end{flushright}
}%



%\title{Cours \\ \titre}
%\date{\today\\ Semestre \semestre}

%\lhead{Cours: \titre}
%\chead{}
%\rhead{\thepage}

%\lfoot{Université Paul Sabatier Toulouse III}
%\cfoot{\thepage}
%\rfoot{\sigle\semestre}

\pagestyle{fancy}


\begin{document}
	\maketitle
	\section{}
	\lstinputlisting[language=SQL, caption=Ajout le vin V12 à la carte du restaut R30 au prix de 12]{1.sql}	
	\section{}
	\lstinputlisting[language=SQL, caption=Supprimer de la table plat tous les plans recommandés avec le vin V12]{2.sql}	
	\section{}
	\lstinputlisting[language=SQL, caption=Prendre en compte le dessert P55 peche Melba recommandés avec le vin V35]{3.sql}	
	spec prendra la valeur N car il y a une valeure par default
	
	\section{}
	\subsection{}
	\lstinputlisting[language=SQL, caption=Augmenter les prix de tous les plats de 1\% le prix restant inferieur à 50euros]{4-1.sql}	
	\subsection{}
	\lstinputlisting[language=SQL, caption=Diminuer les prix des spécialitées de 0.5\%]{4-2.sql}	

	\section{}
	\lstinputlisting[language=SQL, caption=Prendre en compte le nouveau vin recommandé pour les plats P12 P15 P20: v33]{5.sql}	
	\section{}
	\lstinputlisting[language=SQL, caption=Remplacer le vin recommandé V30 par le vin V40 pour tous les autres plats que les dessert]{6.sql}	
	\section{}
	\lstinputlisting[language=SQL, caption=Prendre en compte que le restaurant R10 propose le vin V12 au prix de 15euros]{7.sql}	
	\section{ }
	\lstinputlisting[language=SQL, caption=Avancer de 15 jours la date d'ouverture du restaurant R40 et regarder de 15 jours sa fermeture]{8.sql}	
	\section{}
	\lstinputlisting[language=SQL, caption=Rajouter des colonnes puis faire la mise à jour de ces colonnes pour P10]{9.sql}	
	
\end{document}






