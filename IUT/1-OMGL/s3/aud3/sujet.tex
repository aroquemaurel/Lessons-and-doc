\documentclass[12pt,a4paper,openany]{report}


\usepackage{lmodern}
\usepackage{xcolor}
\usepackage[utf8]{inputenc}
\usepackage[T1]{fontenc}
\usepackage[francais]{babel}
\usepackage[top=1.7cm, bottom=1.7cm, left=1.7cm, right=1.7cm]{geometry}
\usepackage{verbatim}
\usepackage{tikz} %Vectoriel
\usepackage{listings}
\usepackage{fancyhdr}
\usepackage{multido}
\usepackage{amssymb}

\newcommand{\titre}{Ligue National Syldave de Rugby}

\newcommand{\module}{Audit}
\newcommand{\sigle}{audit}

\newcommand{\semestre}{3}

\input{/home/satenske/cours/listings.tex} %prise en charge du langage algo
\input{/home/satenske/cours/entete_iut-cours.tex}

\begin{document}
	\maketitle
	
	La LNSR organise tous les ans un championnat semi professionel, grande couverture médiatique.
	Divisé en deux poles appellé Elit1 et Elit 2.

	\paragraph{Elit1} Deux poules de 2 équipes en match aller retour, 4premiers pour les phases
	final en éliminatoire.
	Les deux derniers descendent et les deux premiers monte
	Victoire = 3 points, match nul 2 points, defaite 1 point. en cas d'étalité, tirage au sort. 
	\paragraph{ORganisation interne}
	\subsection{Commissions(acteur)}
	\begin{itemize}
		\item Commission des clubs pour la gestion des licences. Joueurs professionel, demande de
			licence doit être joint au contrat. 
			Periode des demandes de licence va de la fin du championnat (soir de la final) et
			s'arrête un mois avant la reprise du championnat.
			à la cloture.
			Passer cette méthode de mutation seul les joueurs blesseure peuvent être demandé
	\end{itemize}
	Calendrier du championnat
		match intégrés au calendrier
		Tous les matchs de elit1 ou elit2 se déroulent le WE (prévoir rattrapage, 6 WEk, cas d'antéperie)

	\subsection{Coupe international des clubs}
		4 clubs demi finaliste saison précédente, retenue pour cette compétition 
		(ERC European Ruby Club fournit le calendrier)
	En général les championnats elit1 et elit2 commencent début septembre et finissent fin mai, calendrier
	comminuqé au plus tard fin mai.
	Demis douzaine de partenaire retenue sur une année, 

	\subsection{Arbitrage, commission arbitrage}
	Réunion tous les mardis matin, à partir des différents compte rendu des feuilles de match 
	récupéré de la journée précédente. (problème avec les arbitres)
	Un arbitre ne peu pas arbitré un match opposant un club de sa région, arbitre n'intervient pas plus
	de 4 fois pour un même club. 
	Après réunion mardi matin, courrier vers arbitre avec toutes les pièes nécessaire (rapport, feuilles de match et tickets transport)
	Récupération des résultat
		Faxé et envoyé par courrier, une copie et toujours gardée par les arbitres. 
		à la reception des différents documents les informations sont saisies, afin de faire ressortir
		toutes les anomalies possible. 
	
		\subsection{Commission disipline Sanctions}
		Carton rouge (expulsion) ainsi que les éventuels fait hors match incluant un joueur et mentionné
		dans le rapport des arbitres.
		Donc besoin pièces qui supporte toutes ses informations là. 
		En fonction de l'historique la commission donne une durée ou un nb de matchs de suspension,
		Les médias sont eux aussi informés des sanctions prononcés.

		\subsection{Commissin contrôle de gestion}
		Avoir certaine équité entre les clubs.
		Elle se réunie autant de fois que nécessaire à la demande de la commission de gestion des licences
		et elle donne son aval à la signature des contrôle professionnel, donc obligatoire pour licence
		professionel
		Avis sans appel. 
	
	à la tête de la NSR un conseil d'administration et un président. 

	
	LNSR fictive
	
	
\end{document}
