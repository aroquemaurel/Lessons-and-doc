\documentclass{article}


\usepackage{lmodern}
\usepackage{xcolor}
\usepackage[utf8]{inputenc}
\usepackage[T1]{fontenc}
\usepackage[francais]{babel}
\usepackage[top=1.7cm, bottom=1.7cm, left=1.7cm, right=1.7cm]{geometry}
\usepackage{verbatim}
\usepackage{tikz} %Vectoriel
\usepackage{listings}
\usepackage{fancyhdr}
\usepackage{multido}
\usepackage{amssymb}

\newcommand{\titre}{Atelier de création multimédia}
\newcommand{\numTD}{4}

\newcommand{\module}{production de logiciel}
\newcommand{\sigle}{prl}

\newcommand{\semestre}{3}

\input{/home/satenske/cours/listings.tex} %prise en charge du langage algo

\usepackage{ifthen}
\date{\today}

\chead{}
\rhead{TD\no\numTD}
\lhead{\titre}
%\makeindex

\lfoot{Université Paul Sabatier Toulouse III}
\rfoot{\sigle\semestre}
%\rfoot{}
\cfoot{--~~\thepage~~--}

\makeglossary
\makeatletter
\def\clap#1{\hbox to 0pt{\hss #1\hss}}%

\def\haut#1#2#3{%
	\hbox to \hsize{%
		\rlap{\vtop{\raggedright #1}
	}%
	\hss
	\clap{\vtop{\centering #2}
}%
\hss
\llap{\vtop{\raggedleft #3}}}}%
\def\bas#1#2#3{%
	\hbox to \hsize{%
		\rlap{\vbox{
			\raggedright #1
		}
	}%
	\hss \clap{\vbox{\centering #2}}%
	\hss
	\llap{\vbox{\raggedleft #3}}}
}%
\def\maketitle{%
	\thispagestyle{empty}{%
		\haut{}{\@blurb}{}
		%	
		%\vfill

		\begin{center}
			\vspace{-1.5cm}
			\usefont{OT1}{ptm}{m}{n}
			\huge \@numeroTD \@title
		\end{center}
		\par
		\hrule height 1pt
		\par
		\vspace{1cm}
		\bas{}{}{}
}%
}
\def\date#1{\def\@date{#1}}
\def\author#1{\def\@author{#1}}
\def\numeroTD#1{\def\@numeroTD{#1}}
\def\title#1{\def\@title{#1}}
\def\location#1{\def\@location{#1}}
\def\blurb#1{\def\@blurb{#1}}
\date{\today}
\newboolean{monBool}
\setboolean{monBool}{true}
\author{}
\title{}
\ifthenelse{\equal{\numTD}{}}{
\numeroTD{}
}
{
	\numeroTD{TD \no\numTD~--- }
}
\location{Amiens}\blurb{}
%\makeatother
\title{\titre}
\author{%Semestre \semestre
}

\location{Toulouse}
\blurb{%
\vspace{-35px}
\begin{flushleft}
	Université Paul Sabatier -- Toulouse III\\
	IUT A - Toulouse Rangueil\\
\end{flushleft}
\begin{flushright}
	\vspace{-45px}
	\Large \textbf \module \\
	\normalsize \textit \today\\
	Semestre \semestre
	\vspace{30px}
\end{flushright}
}%



%\title{Cours \\ \titre}
%\date{\today\\ Semestre \semestre}

%\lhead{Cours: \titre}
%\chead{}
%\rhead{\thepage}

%\lfoot{Université Paul Sabatier Toulouse III}
%\cfoot{\thepage}
%\rfoot{\sigle\semestre}

\pagestyle{fancy}


\begin{document}
	\maketitle
	\section{Les différents acteurs}
		\subsection{Acteurs principaux}
		\begin{itemize}
			\item Élève 
		\end{itemize}
		\subsection{Acteurs secondaires}
		\begin{itemize}
			\item Animateur
			\item Agents Municipaux
		\end{itemize}
		\subsection{Système externe}
		\begin{itemize}
			\item Fonds documentaire musée 
			\item Fonds bibliothèque
		\end{itemize}
		\subsection{Matériel externe}
		\begin{itemize}
			\item Lecteur de carte
			\item Lecteur CD-ROM 
		\end{itemize}

		\newpage
	\section{Les besoins du client et le diagramme d'utilisation}
	\begin{center}% Graphic for TeX using PGF
% Title: /home/satenske/cours/OMGL/PRL/TD3/diag1.dia
% Creator: Dia v0.97.1
% CreationDate: Wed Sep 28 08:48:05 2011
% For: satenske
% \usepackage{tikz}
% The following commands are not supported in PSTricks at present
% We define them conditionally, so when they are implemented,
% this pgf file will use them.
\ifx\du\undefined
  \newlength{\du}
\fi
\setlength{\du}{15\unitlength}
\begin{tikzpicture}
\pgftransformxscale{1.000000}
\pgftransformyscale{-1.000000}
\definecolor{dialinecolor}{rgb}{0.000000, 0.000000, 0.000000}
\pgfsetstrokecolor{dialinecolor}
\definecolor{dialinecolor}{rgb}{1.000000, 1.000000, 1.000000}
\pgfsetfillcolor{dialinecolor}
\pgfsetlinewidth{0.000000\du}
\pgfsetdash{}{0pt}
\definecolor{dialinecolor}{rgb}{1.000000, 1.000000, 1.000000}
\pgfsetfillcolor{dialinecolor}
\pgfpathellipse{\pgfpoint{7.600000\du}{2.800000\du}}{\pgfpoint{0.300000\du}{0\du}}{\pgfpoint{0\du}{0.300000\du}}
\pgfusepath{fill}
\definecolor{dialinecolor}{rgb}{0.000000, 0.000000, 0.000000}
\pgfsetstrokecolor{dialinecolor}
\pgfpathellipse{\pgfpoint{7.600000\du}{2.800000\du}}{\pgfpoint{0.300000\du}{0\du}}{\pgfpoint{0\du}{0.300000\du}}
\pgfusepath{stroke}
\definecolor{dialinecolor}{rgb}{0.000000, 0.000000, 0.000000}
\pgfsetstrokecolor{dialinecolor}
\draw (6.400000\du,3.400000\du)--(8.800000\du,3.400000\du);
\definecolor{dialinecolor}{rgb}{0.000000, 0.000000, 0.000000}
\pgfsetstrokecolor{dialinecolor}
\draw (7.600000\du,3.100000\du)--(7.600000\du,4.600000\du);
\definecolor{dialinecolor}{rgb}{0.000000, 0.000000, 0.000000}
\pgfsetstrokecolor{dialinecolor}
\draw (7.600000\du,4.600000\du)--(6.400000\du,5.900000\du);
\definecolor{dialinecolor}{rgb}{0.000000, 0.000000, 0.000000}
\pgfsetstrokecolor{dialinecolor}
\draw (7.600000\du,4.600000\du)--(8.800000\du,5.900000\du);
% setfont left to latex
\definecolor{dialinecolor}{rgb}{0.000000, 0.000000, 0.000000}
\pgfsetstrokecolor{dialinecolor}
\node at (7.600000\du,7.095000\du){Chronométreur};
\pgfsetlinewidth{0.100000\du}
\pgfsetdash{{\pgflinewidth}{0.200000\du}}{0cm}
\pgfsetdash{{\pgflinewidth}{0.200000\du}}{0cm}
\pgfsetbuttcap
{
\definecolor{dialinecolor}{rgb}{0.000000, 0.000000, 0.000000}
\pgfsetfillcolor{dialinecolor}
% was here!!!
\definecolor{dialinecolor}{rgb}{0.000000, 0.000000, 0.000000}
\pgfsetstrokecolor{dialinecolor}
\draw (7.600000\du,4.600000\du)--(7.600000\du,4.600000\du);
}
\pgfsetlinewidth{0.020000\du}
\pgfsetdash{{\pgflinewidth}{0.200000\du}}{0cm}
\pgfsetdash{{\pgflinewidth}{0.200000\du}}{0cm}
\pgfsetbuttcap
{
\definecolor{dialinecolor}{rgb}{0.000000, 0.000000, 0.000000}
\pgfsetfillcolor{dialinecolor}
% was here!!!
\definecolor{dialinecolor}{rgb}{0.000000, 0.000000, 0.000000}
\pgfsetstrokecolor{dialinecolor}
\draw (7.600000\du,7.350256\du)--(7.600000\du,27.800000\du);
}
\pgfsetlinewidth{0.000000\du}
\pgfsetdash{}{0pt}
\definecolor{dialinecolor}{rgb}{1.000000, 1.000000, 1.000000}
\pgfsetfillcolor{dialinecolor}
\fill (19.300000\du,3.750000\du)--(20.755000\du,3.750000\du)--(21.355000\du,4.350000\du)--(21.355000\du,5.450000\du)--(19.300000\du,5.450000\du)--cycle;
\definecolor{dialinecolor}{rgb}{0.000000, 0.000000, 0.000000}
\pgfsetstrokecolor{dialinecolor}
\draw (19.300000\du,3.750000\du)--(20.755000\du,3.750000\du)--(21.355000\du,4.350000\du)--(21.355000\du,5.450000\du)--(19.300000\du,5.450000\du)--cycle;
\pgfsetlinewidth{0.000000\du}
\definecolor{dialinecolor}{rgb}{0.000000, 0.000000, 0.000000}
\pgfsetstrokecolor{dialinecolor}
\draw (20.755000\du,3.750000\du)--(20.755000\du,4.350000\du)--(21.355000\du,4.350000\du);
% setfont left to latex
\definecolor{dialinecolor}{rgb}{0.000000, 0.000000, 0.000000}
\pgfsetstrokecolor{dialinecolor}
\node[anchor=west] at (19.600000\du,4.945000\du){SAE};
\pgfsetlinewidth{0.000000\du}
\pgfsetdash{}{0pt}
\definecolor{dialinecolor}{rgb}{1.000000, 1.000000, 1.000000}
\pgfsetfillcolor{dialinecolor}
\fill (32.305000\du,3.715000\du)--(39.150000\du,3.715000\du)--(39.750000\du,4.315000\du)--(39.750000\du,5.415000\du)--(32.305000\du,5.415000\du)--cycle;
\definecolor{dialinecolor}{rgb}{0.000000, 0.000000, 0.000000}
\pgfsetstrokecolor{dialinecolor}
\draw (32.305000\du,3.715000\du)--(39.150000\du,3.715000\du)--(39.750000\du,4.315000\du)--(39.750000\du,5.415000\du)--(32.305000\du,5.415000\du)--cycle;
\pgfsetlinewidth{0.000000\du}
\definecolor{dialinecolor}{rgb}{0.000000, 0.000000, 0.000000}
\pgfsetstrokecolor{dialinecolor}
\draw (39.150000\du,3.715000\du)--(39.150000\du,4.315000\du)--(39.750000\du,4.315000\du);
% setfont left to latex
\definecolor{dialinecolor}{rgb}{0.000000, 0.000000, 0.000000}
\pgfsetstrokecolor{dialinecolor}
\node[anchor=west] at (32.605000\du,4.910000\du){:IntGestionCourse};
\pgfsetlinewidth{0.020000\du}
\pgfsetdash{{\pgflinewidth}{0.200000\du}}{0cm}
\pgfsetdash{{\pgflinewidth}{0.200000\du}}{0cm}
\pgfsetbuttcap
{
\definecolor{dialinecolor}{rgb}{0.000000, 0.000000, 0.000000}
\pgfsetfillcolor{dialinecolor}
% was here!!!
\definecolor{dialinecolor}{rgb}{0.000000, 0.000000, 0.000000}
\pgfsetstrokecolor{dialinecolor}
\draw (20.165000\du,6.875000\du)--(20.165000\du,27.324744\du);
}
\pgfsetlinewidth{0.020000\du}
\pgfsetdash{{\pgflinewidth}{0.200000\du}}{0cm}
\pgfsetdash{{\pgflinewidth}{0.200000\du}}{0cm}
\pgfsetbuttcap
{
\definecolor{dialinecolor}{rgb}{0.000000, 0.000000, 0.000000}
\pgfsetfillcolor{dialinecolor}
% was here!!!
\definecolor{dialinecolor}{rgb}{0.000000, 0.000000, 0.000000}
\pgfsetstrokecolor{dialinecolor}
\draw (35.320000\du,6.440000\du)--(35.320000\du,26.889744\du);
}
\definecolor{dialinecolor}{rgb}{1.000000, 1.000000, 1.000000}
\pgfsetfillcolor{dialinecolor}
\fill (19.350000\du,7.312500\du)--(19.350000\du,18.762500\du)--(20.950000\du,18.762500\du)--(20.950000\du,7.312500\du)--cycle;
\pgfsetlinewidth{0.020000\du}
\pgfsetdash{}{0pt}
\pgfsetdash{}{0pt}
\pgfsetmiterjoin
\definecolor{dialinecolor}{rgb}{0.000000, 0.000000, 0.000000}
\pgfsetstrokecolor{dialinecolor}
\draw (19.350000\du,7.312500\du)--(19.350000\du,18.762500\du)--(20.950000\du,18.762500\du)--(20.950000\du,7.312500\du)--cycle;
% setfont left to latex
\definecolor{dialinecolor}{rgb}{0.000000, 0.000000, 0.000000}
\pgfsetstrokecolor{dialinecolor}
\node at (20.150000\du,13.232500\du){};
\pgfsetlinewidth{0.020000\du}
\pgfsetdash{}{0pt}
\pgfsetdash{}{0pt}
\pgfsetbuttcap
{
\definecolor{dialinecolor}{rgb}{0.000000, 0.000000, 0.000000}
\pgfsetfillcolor{dialinecolor}
% was here!!!
\pgfsetarrowsend{stealth}
\definecolor{dialinecolor}{rgb}{0.000000, 0.000000, 0.000000}
\pgfsetstrokecolor{dialinecolor}
\pgfpathmoveto{\pgfpoint{21.199923\du}{8.612513\du}}
\pgfpatharc{81}{-157}{0.955437\du and 0.955437\du}
\pgfusepath{stroke}
}
% setfont left to latex
\definecolor{dialinecolor}{rgb}{0.000000, 0.000000, 0.000000}
\pgfsetstrokecolor{dialinecolor}
\node[anchor=west] at (22.100000\du,6.212500\du){prendreEnCompteResultat};
\definecolor{dialinecolor}{rgb}{1.000000, 1.000000, 1.000000}
\pgfsetfillcolor{dialinecolor}
\fill (34.165000\du,11.337500\du)--(34.165000\du,13.157500\du)--(35.765000\du,13.157500\du)--(35.765000\du,11.337500\du)--cycle;
\pgfsetlinewidth{0.020000\du}
\pgfsetdash{}{0pt}
\pgfsetdash{}{0pt}
\pgfsetmiterjoin
\definecolor{dialinecolor}{rgb}{0.000000, 0.000000, 0.000000}
\pgfsetstrokecolor{dialinecolor}
\draw (34.165000\du,11.337500\du)--(34.165000\du,13.157500\du)--(35.765000\du,13.157500\du)--(35.765000\du,11.337500\du)--cycle;
% setfont left to latex
\definecolor{dialinecolor}{rgb}{0.000000, 0.000000, 0.000000}
\pgfsetstrokecolor{dialinecolor}
\node at (34.965000\du,12.442500\du){};
\pgfsetlinewidth{0.000000\du}
\pgfsetdash{}{0pt}
\pgfsetdash{}{0pt}
\pgfsetbuttcap
{
\definecolor{dialinecolor}{rgb}{0.000000, 0.000000, 0.000000}
\pgfsetfillcolor{dialinecolor}
% was here!!!
\pgfsetarrowsend{to}
\definecolor{dialinecolor}{rgb}{0.000000, 0.000000, 0.000000}
\pgfsetstrokecolor{dialinecolor}
\draw (7.550000\du,11.412500\du)--(19.300000\du,11.412500\du);
}
\pgfsetlinewidth{0.000000\du}
\pgfsetdash{}{0pt}
\pgfsetdash{}{0pt}
\pgfsetbuttcap
{
\definecolor{dialinecolor}{rgb}{0.000000, 0.000000, 0.000000}
\pgfsetfillcolor{dialinecolor}
% was here!!!
\pgfsetarrowsend{to}
\definecolor{dialinecolor}{rgb}{0.000000, 0.000000, 0.000000}
\pgfsetstrokecolor{dialinecolor}
\draw (7.605000\du,14.377500\du)--(19.355000\du,14.377500\du);
}
\pgfsetlinewidth{0.000000\du}
\pgfsetdash{}{0pt}
\pgfsetdash{}{0pt}
\pgfsetbuttcap
{
\definecolor{dialinecolor}{rgb}{0.000000, 0.000000, 0.000000}
\pgfsetfillcolor{dialinecolor}
% was here!!!
\pgfsetarrowsend{to}
\definecolor{dialinecolor}{rgb}{0.000000, 0.000000, 0.000000}
\pgfsetstrokecolor{dialinecolor}
\draw (21.010000\du,11.342500\du)--(34.165000\du,11.337500\du);
}
\pgfsetlinewidth{0.000000\du}
\pgfsetdash{}{0pt}
\pgfsetdash{}{0pt}
\pgfsetbuttcap
{
\definecolor{dialinecolor}{rgb}{0.000000, 0.000000, 0.000000}
\pgfsetfillcolor{dialinecolor}
% was here!!!
\pgfsetarrowsend{to}
\definecolor{dialinecolor}{rgb}{0.000000, 0.000000, 0.000000}
\pgfsetstrokecolor{dialinecolor}
\draw (7.815000\du,17.257500\du)--(19.565000\du,17.257500\du);
}
\definecolor{dialinecolor}{rgb}{1.000000, 1.000000, 1.000000}
\pgfsetfillcolor{dialinecolor}
\fill (34.210000\du,14.277310\du)--(34.210000\du,16.097310\du)--(35.810000\du,16.097310\du)--(35.810000\du,14.277310\du)--cycle;
\pgfsetlinewidth{0.020000\du}
\pgfsetdash{}{0pt}
\pgfsetdash{}{0pt}
\pgfsetmiterjoin
\definecolor{dialinecolor}{rgb}{0.000000, 0.000000, 0.000000}
\pgfsetstrokecolor{dialinecolor}
\draw (34.210000\du,14.277310\du)--(34.210000\du,16.097310\du)--(35.810000\du,16.097310\du)--(35.810000\du,14.277310\du)--cycle;
% setfont left to latex
\definecolor{dialinecolor}{rgb}{0.000000, 0.000000, 0.000000}
\pgfsetstrokecolor{dialinecolor}
\node at (35.010000\du,15.382310\du){};
\pgfsetlinewidth{0.000000\du}
\pgfsetdash{}{0pt}
\pgfsetdash{}{0pt}
\pgfsetbuttcap
{
\definecolor{dialinecolor}{rgb}{0.000000, 0.000000, 0.000000}
\pgfsetfillcolor{dialinecolor}
% was here!!!
\pgfsetarrowsend{to}
\definecolor{dialinecolor}{rgb}{0.000000, 0.000000, 0.000000}
\pgfsetstrokecolor{dialinecolor}
\draw (21.055000\du,14.282310\du)--(34.210000\du,14.277310\du);
}
\definecolor{dialinecolor}{rgb}{1.000000, 1.000000, 1.000000}
\pgfsetfillcolor{dialinecolor}
\fill (34.310000\du,17.377310\du)--(34.310000\du,19.197310\du)--(35.910000\du,19.197310\du)--(35.910000\du,17.377310\du)--cycle;
\pgfsetlinewidth{0.020000\du}
\pgfsetdash{}{0pt}
\pgfsetdash{}{0pt}
\pgfsetmiterjoin
\definecolor{dialinecolor}{rgb}{0.000000, 0.000000, 0.000000}
\pgfsetstrokecolor{dialinecolor}
\draw (34.310000\du,17.377310\du)--(34.310000\du,19.197310\du)--(35.910000\du,19.197310\du)--(35.910000\du,17.377310\du)--cycle;
% setfont left to latex
\definecolor{dialinecolor}{rgb}{0.000000, 0.000000, 0.000000}
\pgfsetstrokecolor{dialinecolor}
\node at (35.110000\du,18.482310\du){};
\pgfsetlinewidth{0.000000\du}
\pgfsetdash{}{0pt}
\pgfsetdash{}{0pt}
\pgfsetbuttcap
{
\definecolor{dialinecolor}{rgb}{0.000000, 0.000000, 0.000000}
\pgfsetfillcolor{dialinecolor}
% was here!!!
\pgfsetarrowsend{to}
\definecolor{dialinecolor}{rgb}{0.000000, 0.000000, 0.000000}
\pgfsetstrokecolor{dialinecolor}
\draw (21.155000\du,17.382310\du)--(34.310000\du,17.377310\du);
}
% setfont left to latex
\definecolor{dialinecolor}{rgb}{0.000000, 0.000000, 0.000000}
\pgfsetstrokecolor{dialinecolor}
\node[anchor=west] at (10.000000\du,10.412500\du){saisirInfoArrivant};
% setfont left to latex
\definecolor{dialinecolor}{rgb}{0.000000, 0.000000, 0.000000}
\pgfsetstrokecolor{dialinecolor}
\node[anchor=west] at (10.705000\du,13.672500\du){saisirInfoArrivant};
% setfont left to latex
\definecolor{dialinecolor}{rgb}{0.000000, 0.000000, 0.000000}
\pgfsetstrokecolor{dialinecolor}
\node[anchor=west] at (10.410000\du,16.337500\du){saisirInfoArrivant};
% setfont left to latex
\definecolor{dialinecolor}{rgb}{0.000000, 0.000000, 0.000000}
\pgfsetstrokecolor{dialinecolor}
\node[anchor=west] at (23.765000\du,10.652500\du){saisirInfoArrivant};
% setfont left to latex
\definecolor{dialinecolor}{rgb}{0.000000, 0.000000, 0.000000}
\pgfsetstrokecolor{dialinecolor}
\node[anchor=west] at (24.420000\du,13.367500\du){saisirInfoArrivant};
% setfont left to latex
\definecolor{dialinecolor}{rgb}{0.000000, 0.000000, 0.000000}
\pgfsetstrokecolor{dialinecolor}
\node[anchor=west] at (23.775000\du,16.282500\du){saisirInfoArrivant};
\end{tikzpicture}
\end{center}
	\section{cas d'utilisation}
		\subsection{S'identifier}
		\paragraph{Rôle}
			Ce cas d'utilisation regroupe les différents scénarios et les différents fonctionnalités
			permettant aux agents municipaux et aux animateurs de se connecter au système une fois 
			identifié. \textit{Ce cas d'utilisation est déclenché une seule fois lorsque l'animateur
			désire réaliser une activité. Tant que l'animateur ou l'agent municipal reste connecté
			il n'a plus à s'identifier pour enchaîner les activités}.
		\paragraph{Début}
			Le cas d'utilisation se produit quand un animateur ou un agent municipal désire utiliser
			le système pour:
			\begin{itemize}
				\item Connecter élèves
				\item Insérer de nouveaux éléments
				\item Enregistrer de nouveaux élèves
				\item Valider des documents d'élèves
				\item Connecter ou déconnecter des fonds
				\item Faire des statistiques
			\end{itemize}

		\paragraph{Fin}
			Le cas d'utilisation est terminé lorsque l'autorisation de se connecter au système se 
			produit, que l'animateur/l'agent municipal n'est pas reconnu ou qu'il annule. 
		\paragraph{Pré-condition}
			Le lecteur de carte est opérationnel.	
		\paragraph{Post-condition}
		\begin{itemize}
			\item L'animateur ou l'agent municipal est reconnu $\Rightarrow$ le système est débloqué.	
			\item L'animateur ou l'agent municipal n'est pas reconnu $\Rightarrow$ Le système n'est pas débloqué.
		\end{itemize}
		\paragraph{Données en entrée}
			La carte de l'utilisateur.
		\paragraph{Données en sortie}
			Ok / Pas Ok
		\paragraph{Exception}
			La carte introduite n'est pas valide. 
	\subsection{GérerConnexion}
		\paragraph{Rôle}
			Ce cas d'utilisation regroupe les différents scénarios et les différents fonctionnalités 
			permettant aux animateurs d'autoriser des éléves à se connecter au site et aux éléves
			de se connecter. Ce cas d'utilisation est déclenché chaque fois que l'animateur autorise
			des élèves à se connecter, que l'animateur supprimel'autorisation à des élèves, qu'un élève
			essaie de se connecter ou de se déconnecter. Tantque l'élève reste connecté et que l'animateur
			l'autorise à travailler, il n'a plus à se reconnecter pour enchainer les activités.
		\paragraph{Début}
			Le cas d'utilisation se produit quand un animateur désire autoriser ou interdire aux élèves
			de se connecter ou qu'un élève désire utiliser le système pour créer des documents ou qu'il
			désire se déconnecter. 
		\paragraph{Fin}
			Le cas d'utilisation est terminée lorsque le ou les élèves sont effectivement autorisés
			à se connecter ou qu'un élève est autorisé ou que l'animateur à supprimé l'autorisation
			à un ou plusieurs élèves de se connecter\paragraph{fin}
			Le cas d'utilisation est terminée lorsque le ou les élèves sont effectivement autorisés 
			à se connecter ou qu'un élève s'est déconnécté. 
		\paragraph{Pré-condition}
		L'animateur est identifié et les élèves qui veulent se connecter sont enregistrés dans le 
		système.
		\paragraph{Post-condition}
		L'animateur autorise des utilisateurs à se connecter $\Rightarrow$ La liste des éléves
		autorisé contient les nouveaux élèves autorisés. 
		\paragraph{Données en entrée}
		La liste des éléves enregistrés dans le système.
		\paragraph{Données en sortie}
		\begin{itemize}
			\item L'animateur supprime l'autorisation de connexion à des élèves $\Rightarrow$ 
				les élèves qui n'ont plus l'autorisation de se connecter sont suprimés de la liste 
				des élèves autorisés.
			\item L'élève se déconnecte $\Rightarrow$ l'élève qui se déconnecte et supprimé
				de la liste des élèves
			\item L'élève se connecte $\Rightarrow$ la liste des éléves connectés contient le nouvel 
				élève connecté.
		\end{itemize}
		\subsection{Remarque}
		La décomposition présente ici est loin d'être unique toute fois elle doit respecter les deux fondamentaux suivant: 
		\begin{itemize}
			\item Chaque cas d'utilisation permet d'atteindre un but précis
			\item CHaque cas d'utilisation materialise bien les interaction d'un acteur actif avec le système
		\end{itemize}
		\section{}
		\subsection{Lecteur de Carte}
		\begin{itemize}
			\item Le système consomme: ok pas ok
			\item Le système fournit: / 
		\end{itemize}
		\subsection{Lecteur CD-Rom}
		\begin{itemize}
			\item Le système consomme: données présente sur le CD
			\item Le système fournit: / 
		\end{itemize}
		\subsection{Le fond documentaire du musée}
		\begin{itemize}
			\item Le système consomme: Des documents du musée
			\item Le système fournit: ? 
		\end{itemize}
		\subsection{Le fond documentaire de la bibliothèque}
		\begin{itemize}
			\item Le système consomme: Documents de la bibliothèque
			\item Le système fournit: Les reqûetes 
		\end{itemize}
		\section{}
		% Graphic for TeX using PGF
% Title: /home/satenske/cours/OMGL/prl3/TD4/Diag2.dia
% Creator: Dia v0.97.1
% CreationDate: Fri Oct  7 10:19:37 2011
% For: satenske
% \usepackage{tikz}
% The following commands are not supported in PSTricks at present
% We define them conditionally, so when they are implemented,
% this pgf file will use them.
\ifx\du\undefined
  \newlength{\du}
\fi
\setlength{\du}{15\unitlength}
\begin{tikzpicture}
\pgftransformxscale{1.000000}
\pgftransformyscale{-1.000000}
\definecolor{dialinecolor}{rgb}{0.000000, 0.000000, 0.000000}
\pgfsetstrokecolor{dialinecolor}
\definecolor{dialinecolor}{rgb}{1.000000, 1.000000, 1.000000}
\pgfsetfillcolor{dialinecolor}
% setfont left to latex
\definecolor{dialinecolor}{rgb}{0.000000, 0.000000, 0.000000}
\pgfsetstrokecolor{dialinecolor}
\node[anchor=west] at (12.200000\du,-28.200000\du){};
\pgfsetlinewidth{0.050000\du}
\pgfsetdash{}{0pt}
\pgfsetdash{}{0pt}
\pgfsetmiterjoin
\definecolor{dialinecolor}{rgb}{1.000000, 1.000000, 1.000000}
\pgfsetfillcolor{dialinecolor}
\fill (8.537500\du,-32.475000\du)--(8.537500\du,-31.145000\du)--(12.550100\du,-31.145000\du)--(12.550100\du,-32.475000\du)--cycle;
\definecolor{dialinecolor}{rgb}{0.000000, 0.000000, 0.000000}
\pgfsetstrokecolor{dialinecolor}
\draw (8.537500\du,-32.475000\du)--(8.537500\du,-31.145000\du)--(12.550100\du,-31.145000\du)--(12.550100\du,-32.475000\du)--cycle;
% setfont left to latex
\definecolor{dialinecolor}{rgb}{0.000000, 0.000000, 0.000000}
\pgfsetstrokecolor{dialinecolor}
\node[anchor=west] at (8.737500\du,-32.075000\du){<passif>};
% setfont left to latex
\definecolor{dialinecolor}{rgb}{0.000000, 0.000000, 0.000000}
\pgfsetstrokecolor{dialinecolor}
\node[anchor=west] at (8.737500\du,-31.722222\du){:Lecteur cam};
% setfont left to latex
\definecolor{dialinecolor}{rgb}{0.000000, 0.000000, 0.000000}
\pgfsetstrokecolor{dialinecolor}
\node[anchor=west] at (9.300100\du,-32.370000\du){};
\pgfsetlinewidth{0.050000\du}
\pgfsetdash{}{0pt}
\pgfsetdash{}{0pt}
\pgfsetmiterjoin
\definecolor{dialinecolor}{rgb}{1.000000, 1.000000, 1.000000}
\pgfsetfillcolor{dialinecolor}
\fill (15.587600\du,-32.445000\du)--(15.587600\du,-31.170000\du)--(17.587600\du,-31.170000\du)--(17.587600\du,-32.445000\du)--cycle;
\definecolor{dialinecolor}{rgb}{0.000000, 0.000000, 0.000000}
\pgfsetstrokecolor{dialinecolor}
\draw (15.587600\du,-32.445000\du)--(15.587600\du,-31.170000\du)--(17.587600\du,-31.170000\du)--(17.587600\du,-32.445000\du)--cycle;
% setfont left to latex
\definecolor{dialinecolor}{rgb}{0.000000, 0.000000, 0.000000}
\pgfsetstrokecolor{dialinecolor}
\node[anchor=west] at (16.037600\du,-31.895000\du){:SAE};
% setfont left to latex
\definecolor{dialinecolor}{rgb}{0.000000, 0.000000, 0.000000}
\pgfsetstrokecolor{dialinecolor}
\node[anchor=west] at (16.325200\du,-32.890000\du){};
\pgfsetlinewidth{0.050000\du}
\pgfsetdash{}{0pt}
\pgfsetdash{}{0pt}
\pgfsetmiterjoin
\definecolor{dialinecolor}{rgb}{1.000000, 1.000000, 1.000000}
\pgfsetfillcolor{dialinecolor}
\fill (21.637600\du,-32.495000\du)--(21.637600\du,-31.070000\du)--(26.775100\du,-31.070000\du)--(26.775100\du,-32.495000\du)--cycle;
\definecolor{dialinecolor}{rgb}{0.000000, 0.000000, 0.000000}
\pgfsetstrokecolor{dialinecolor}
\draw (21.637600\du,-32.495000\du)--(21.637600\du,-31.070000\du)--(26.775100\du,-31.070000\du)--(26.775100\du,-32.495000\du)--cycle;
% setfont left to latex
\definecolor{dialinecolor}{rgb}{0.000000, 0.000000, 0.000000}
\pgfsetstrokecolor{dialinecolor}
\node[anchor=west] at (21.962600\du,-32.070000\du){<passif>};
% setfont left to latex
\definecolor{dialinecolor}{rgb}{0.000000, 0.000000, 0.000000}
\pgfsetstrokecolor{dialinecolor}
\node[anchor=west] at (21.962600\du,-31.717222\du){:FD bibliothèque};
% setfont left to latex
\definecolor{dialinecolor}{rgb}{0.000000, 0.000000, 0.000000}
\pgfsetstrokecolor{dialinecolor}
\node[anchor=west] at (21.150200\du,-32.640000\du){};
\pgfsetlinewidth{0.050000\du}
\pgfsetdash{}{0pt}
\pgfsetdash{}{0pt}
\pgfsetmiterjoin
\definecolor{dialinecolor}{rgb}{1.000000, 1.000000, 1.000000}
\pgfsetfillcolor{dialinecolor}
\fill (29.537600\du,-32.445000\du)--(29.537600\du,-31.145000\du)--(33.825100\du,-31.145000\du)--(33.825100\du,-32.445000\du)--cycle;
\definecolor{dialinecolor}{rgb}{0.000000, 0.000000, 0.000000}
\pgfsetstrokecolor{dialinecolor}
\draw (29.537600\du,-32.445000\du)--(29.537600\du,-31.145000\du)--(33.825100\du,-31.145000\du)--(33.825100\du,-32.445000\du)--cycle;
% setfont left to latex
\definecolor{dialinecolor}{rgb}{0.000000, 0.000000, 0.000000}
\pgfsetstrokecolor{dialinecolor}
\node[anchor=west] at (29.737600\du,-32.045000\du){<passif>};
% setfont left to latex
\definecolor{dialinecolor}{rgb}{0.000000, 0.000000, 0.000000}
\pgfsetstrokecolor{dialinecolor}
\node[anchor=west] at (29.737600\du,-31.692222\du){:FD musée};
% setfont left to latex
\definecolor{dialinecolor}{rgb}{0.000000, 0.000000, 0.000000}
\pgfsetstrokecolor{dialinecolor}
\node[anchor=west] at (26.275200\du,-32.665000\du){};
\pgfsetlinewidth{0.050000\du}
\pgfsetdash{{\pgflinewidth}{0.200000\du}}{0cm}
\pgfsetdash{{\pgflinewidth}{0.200000\du}}{0cm}
\pgfsetbuttcap
{
\definecolor{dialinecolor}{rgb}{0.000000, 0.000000, 0.000000}
\pgfsetfillcolor{dialinecolor}
% was here!!!
\definecolor{dialinecolor}{rgb}{0.000000, 0.000000, 0.000000}
\pgfsetstrokecolor{dialinecolor}
\draw (10.550100\du,-31.245000\du)--(10.525100\du,-21.470000\du);
}
\pgfsetlinewidth{0.050000\du}
\pgfsetdash{{\pgflinewidth}{0.200000\du}}{0cm}
\pgfsetdash{{\pgflinewidth}{0.200000\du}}{0cm}
\pgfsetbuttcap
{
\definecolor{dialinecolor}{rgb}{0.000000, 0.000000, 0.000000}
\pgfsetfillcolor{dialinecolor}
% was here!!!
\definecolor{dialinecolor}{rgb}{0.000000, 0.000000, 0.000000}
\pgfsetstrokecolor{dialinecolor}
\draw (16.587600\du,-31.170000\du)--(16.600100\du,-16.920000\du);
}
\pgfsetlinewidth{0.050000\du}
\pgfsetdash{{\pgflinewidth}{0.200000\du}}{0cm}
\pgfsetdash{{\pgflinewidth}{0.200000\du}}{0cm}
\pgfsetbuttcap
{
\definecolor{dialinecolor}{rgb}{0.000000, 0.000000, 0.000000}
\pgfsetfillcolor{dialinecolor}
% was here!!!
\definecolor{dialinecolor}{rgb}{0.000000, 0.000000, 0.000000}
\pgfsetstrokecolor{dialinecolor}
\draw (24.137600\du,-31.134900\du)--(24.125200\du,-16.820000\du);
}
\pgfsetlinewidth{0.050000\du}
\pgfsetdash{{\pgflinewidth}{0.200000\du}}{0cm}
\pgfsetdash{{\pgflinewidth}{0.200000\du}}{0cm}
\pgfsetbuttcap
{
\definecolor{dialinecolor}{rgb}{0.000000, 0.000000, 0.000000}
\pgfsetfillcolor{dialinecolor}
% was here!!!
\definecolor{dialinecolor}{rgb}{0.000000, 0.000000, 0.000000}
\pgfsetstrokecolor{dialinecolor}
\draw (31.812600\du,-31.284900\du)--(31.787700\du,-21.474900\du);
}
\pgfsetlinewidth{0.050000\du}
\pgfsetdash{}{0pt}
\pgfsetdash{}{0pt}
\pgfsetmiterjoin
\definecolor{dialinecolor}{rgb}{1.000000, 1.000000, 1.000000}
\pgfsetfillcolor{dialinecolor}
\fill (16.575100\du,-28.470000\du)--(16.575100\du,-26.370000\du)--(17.050100\du,-26.370000\du)--(17.050100\du,-28.470000\du)--cycle;
\definecolor{dialinecolor}{rgb}{0.000000, 0.000000, 0.000000}
\pgfsetstrokecolor{dialinecolor}
\draw (16.575100\du,-28.470000\du)--(16.575100\du,-26.370000\du)--(17.050100\du,-26.370000\du)--(17.050100\du,-28.470000\du)--cycle;
\pgfsetlinewidth{0.050000\du}
\pgfsetdash{}{0pt}
\pgfsetdash{}{0pt}
\pgfsetmiterjoin
\definecolor{dialinecolor}{rgb}{1.000000, 1.000000, 1.000000}
\pgfsetfillcolor{dialinecolor}
\fill (16.612600\du,-24.810000\du)--(16.612600\du,-18.820000\du)--(17.087600\du,-18.820000\du)--(17.087600\du,-24.810000\du)--cycle;
\definecolor{dialinecolor}{rgb}{0.000000, 0.000000, 0.000000}
\pgfsetstrokecolor{dialinecolor}
\draw (16.612600\du,-24.810000\du)--(16.612600\du,-18.820000\du)--(17.087600\du,-18.820000\du)--(17.087600\du,-24.810000\du)--cycle;
\pgfsetlinewidth{0.050000\du}
\pgfsetdash{}{0pt}
\pgfsetdash{}{0pt}
\pgfsetbuttcap
{
\definecolor{dialinecolor}{rgb}{0.000000, 0.000000, 0.000000}
\pgfsetfillcolor{dialinecolor}
% was here!!!
\pgfsetarrowsend{to}
\definecolor{dialinecolor}{rgb}{0.000000, 0.000000, 0.000000}
\pgfsetstrokecolor{dialinecolor}
\pgfpathmoveto{\pgfpoint{17.175071\du}{-28.019986\du}}
\pgfpatharc{65}{-135}{0.373634\du and 0.373634\du}
\pgfusepath{stroke}
}
\pgfsetlinewidth{0.050000\du}
\pgfsetdash{}{0pt}
\pgfsetdash{}{0pt}
\pgfsetbuttcap
{
\definecolor{dialinecolor}{rgb}{0.000000, 0.000000, 0.000000}
\pgfsetfillcolor{dialinecolor}
% was here!!!
\pgfsetarrowsend{to}
\definecolor{dialinecolor}{rgb}{0.000000, 0.000000, 0.000000}
\pgfsetstrokecolor{dialinecolor}
\pgfpathmoveto{\pgfpoint{17.297470\du}{-24.369888\du}}
\pgfpatharc{69}{-138}{0.377916\du and 0.377916\du}
\pgfusepath{stroke}
}
\pgfsetlinewidth{0.050000\du}
\pgfsetdash{}{0pt}
\pgfsetdash{}{0pt}
\pgfsetbuttcap
{
\definecolor{dialinecolor}{rgb}{0.000000, 0.000000, 0.000000}
\pgfsetfillcolor{dialinecolor}
% was here!!!
\pgfsetarrowsend{to}
\definecolor{dialinecolor}{rgb}{0.000000, 0.000000, 0.000000}
\pgfsetstrokecolor{dialinecolor}
\draw (10.618900\du,-28.045000\du)--(16.493900\du,-28.020000\du);
}
% setfont left to latex
\definecolor{dialinecolor}{rgb}{0.000000, 0.000000, 0.000000}
\pgfsetstrokecolor{dialinecolor}
\node[anchor=west] at (12.818900\du,-28.370000\du){ok};
\pgfsetlinewidth{0.050000\du}
\pgfsetdash{}{0pt}
\pgfsetdash{}{0pt}
\pgfsetbuttcap
{
\definecolor{dialinecolor}{rgb}{0.000000, 0.000000, 0.000000}
\pgfsetfillcolor{dialinecolor}
% was here!!!
\pgfsetarrowsend{to}
\definecolor{dialinecolor}{rgb}{0.000000, 0.000000, 0.000000}
\pgfsetstrokecolor{dialinecolor}
\draw (31.525200\du,-23.770000\du)--(17.150200\du,-23.770000\du);
}
% setfont left to latex
\definecolor{dialinecolor}{rgb}{0.000000, 0.000000, 0.000000}
\pgfsetstrokecolor{dialinecolor}
\node[anchor=west] at (21.600200\du,-23.920000\du){ListeDeDocuments};
\pgfsetlinewidth{0.050000\du}
\pgfsetdash{}{0pt}
\pgfsetdash{}{0pt}
\pgfsetbuttcap
{
\definecolor{dialinecolor}{rgb}{0.000000, 0.000000, 0.000000}
\pgfsetfillcolor{dialinecolor}
% was here!!!
\pgfsetarrowsend{to}
\definecolor{dialinecolor}{rgb}{0.000000, 0.000000, 0.000000}
\pgfsetstrokecolor{dialinecolor}
\draw (17.325200\du,-22.645000\du)--(24.100200\du,-22.645000\du);
}
\pgfsetlinewidth{0.050000\du}
\pgfsetdash{}{0pt}
\pgfsetdash{}{0pt}
\pgfsetbuttcap
{
\definecolor{dialinecolor}{rgb}{0.000000, 0.000000, 0.000000}
\pgfsetfillcolor{dialinecolor}
% was here!!!
\pgfsetarrowsend{to}
\definecolor{dialinecolor}{rgb}{0.000000, 0.000000, 0.000000}
\pgfsetstrokecolor{dialinecolor}
\draw (24.100200\du,-21.545000\du)--(17.275200\du,-21.545000\du);
}
\pgfsetlinewidth{0.050000\du}
\pgfsetdash{}{0pt}
\pgfsetdash{}{0pt}
\pgfsetbuttcap
{
\definecolor{dialinecolor}{rgb}{0.000000, 0.000000, 0.000000}
\pgfsetfillcolor{dialinecolor}
% was here!!!
\pgfsetarrowsend{to}
\definecolor{dialinecolor}{rgb}{0.000000, 0.000000, 0.000000}
\pgfsetstrokecolor{dialinecolor}
\draw (24.056100\du,-19.259500\du)--(17.231100\du,-19.259500\du);
}
\pgfsetlinewidth{0.050000\du}
\pgfsetdash{}{0pt}
\pgfsetdash{}{0pt}
\pgfsetbuttcap
{
\definecolor{dialinecolor}{rgb}{0.000000, 0.000000, 0.000000}
\pgfsetfillcolor{dialinecolor}
% was here!!!
\pgfsetarrowsend{to}
\definecolor{dialinecolor}{rgb}{0.000000, 0.000000, 0.000000}
\pgfsetstrokecolor{dialinecolor}
\draw (23.943600\du,-20.424500\du)--(17.118600\du,-20.424500\du);
}
% setfont left to latex
\definecolor{dialinecolor}{rgb}{0.000000, 0.000000, 0.000000}
\pgfsetstrokecolor{dialinecolor}
\node[anchor=west] at (19.375200\du,-23.020000\du){Requete};
% setfont left to latex
\definecolor{dialinecolor}{rgb}{0.000000, 0.000000, 0.000000}
\pgfsetstrokecolor{dialinecolor}
\node[anchor=west] at (19.275200\du,-21.845000\du){Documents};
% setfont left to latex
\definecolor{dialinecolor}{rgb}{0.000000, 0.000000, 0.000000}
\pgfsetstrokecolor{dialinecolor}
\node[anchor=west] at (19.162700\du,-20.772500\du){Documents};
% setfont left to latex
\definecolor{dialinecolor}{rgb}{0.000000, 0.000000, 0.000000}
\pgfsetstrokecolor{dialinecolor}
\node[anchor=west] at (19.287700\du,-19.697500\du){Documents};
% setfont left to latex
\definecolor{dialinecolor}{rgb}{0.000000, 0.000000, 0.000000}
\pgfsetstrokecolor{dialinecolor}
\node[anchor=west] at (18.012500\du,-24.975000\du){creerDocument};
% setfont left to latex
\definecolor{dialinecolor}{rgb}{0.000000, 0.000000, 0.000000}
\pgfsetstrokecolor{dialinecolor}
\node[anchor=west] at (17.750000\du,-28.195000\du){s'identifier};
\end{tikzpicture}


\end{document}

