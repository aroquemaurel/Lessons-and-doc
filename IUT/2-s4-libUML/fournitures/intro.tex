\chapter*{Avant-propos}
Ce document fait partis d'un groupe de documents permettant de montrer notre approche pour développer une bibliothèque d'objets graphiques 
UML\footnote{Unified Modelling Language}. UML est un langage permettant de modéliser une application, de la concevoir. 

\section*{Présentation du groupe}
	Notre groupe projet est composé de quatre étudiants de deuxième année de DUT Informatique à l'IUT\footnote{Institut Universitaire de Technologie} A de Toulouse, voici la composition de l'équipe: 
	\begin{itemize}
		\item Antoine de \bsc{Roquemaurel} 
		\item Mathieu \bsc{Soum} 
		\item Geoffroy \bsc{Subias}
		\item Marie-Ly \bsc{Tang} 
	\end{itemize}
	\vspace{20px}
	\paragraph{}
	Nous avons monté ce groupe, car nos compétences sont complémentaires et que nous savons déjà comment chacun travaille.
	
	Antoine de \bsc{Roquemaurel} et Mathieu \bsc{Soum} sont spécialisés en programmation par objet, Geoffroy \bsc{Subias} est le plus compétent lorsqu'il s'agit de modélisation UML\footnote{Unified Modelling Language} et Marie-Ly \bsc{Tang} s'occupera principalement de l'organisation et de la gestion de projet. 

	Nos compétences sont différentes mais sont complémentaires pour mener à bien notre projet.
\section*{Fonctionnement du document}
Ce document est un document expliquant notre approche pour développer une bibliothèque d'objets graphiques UML.\\

Dans ce dossier, vous pourrez repérer diverses notations, cette introduction a pour but de vous expliquer les notations afin
que vous puissiez lire en toute sérénité.
\subsection*{Le glossaire}
Un mot dans le glossaire a une police particulière, vous pourrez savoir qu'un mot est dans le glossaire lorsque vous repérerez un mot avec la police suivante: 
\policeGlossaire{leMotDansLeGlossiare}. Si vous voyez cette police, vous pouvez donc vous référez à l'annexe \ref{glossaire} page \pageref{glossaire}.
\subsection*{Les noms de méthode, d'attribut ou de classe}
Les mots se référant à un nom présent dans du code ont une police particulière, une police type ``machine à écrire'', si vous voyez la police suivante, c'est que c'est un nom 
de méthode, d'attribut ou de classe: \texttt{uneFonction}.
\subsection*{Les noms de paquetage}
Les noms de paquetage utiliseront une police particulière, afin que l'on puisse les différencier d'une classe ou d'une méthode, vous les trouverez 
comme suit : \policePackage{unPaquetage}.
\subsection*{Les notes de bas de page}
Nous utilisons régulièrement des notes de bas de pages, pour donner un acronyme, pour expliquer plus en détail une notion, ces notes de bas de pages sont un numéro
en exposant, vous trouverez la note correspondante en bas de la page courante, comme ceci\footnote{Ceci est une note de bas de page}.
\subsection*{Les liens hypertextes}
Dans le document, nous pouvons faire référence à un lien d'un site web, tous les liens seront donc symbolisés par une petite puce, et une police particulière comme ceci:\\
	$\rhd$ \url{http://monLien.fr/index.html}\\

