\documentclass[12pt,a4paper,openany]{report}

\usepackage{lmodern}
\usepackage{xcolor}
\usepackage[utf8]{inputenc}
\usepackage[T1]{fontenc}
\usepackage[francais]{babel}
\usepackage[top=1.7cm, bottom=1.7cm, left=1.7cm, right=1.7cm]{geometry}
\usepackage{verbatim}
\usepackage{tikz} %Vectoriel
\usepackage{listings}
\usepackage{fancyhdr}
\usepackage{multido}
\usepackage{amssymb}

\newcommand{\titre}{Création d'entreprise}

\newcommand{\module}{}
\newcommand{\sigle}{cre}

\newcommand{\semestre}{3}

\definecolor{gris1}{gray}{0.40}
\definecolor{gris2}{gray}{0.55}
\definecolor{gris3}{gray}{0.65}
\definecolor{gris4}{gray}{0.50}
\definecolor{vert}{rgb}{0,0.4,0}
\definecolor{violet}{rgb}{0.65, 0.2, 0.65}
\definecolor{bleu1}{rgb}{0,0,0.8}
\definecolor{bleu2}{rgb}{0,0.2,0.6}
\definecolor{bleu3}{rgb}{0,0.2,0.2}
\definecolor{rouge}{HTML}{F93928}


\lstdefinelanguage{algo}{%
   morekeywords={%
    %%% couleur 1
		importer, programme, glossaire, fonction, procedure, constante, type, 
	%%% IMPORT & Co.
		si, sinon, alors, fin, tantque, debut, faire, lorsque, fin lorsque, 
		declenche, declencher, enregistrement, tableau, retourne, retourner, =, pour, a,
		/=, <, >, traite,exception, 
	%%% types 
		Entier, Reel, Booleen, Caractere, Réél, Booléen, Caractère,
	%%% types 
		entree, maj, sortie,entrée,
	%%% types 
		et, ou, non,
	},
  sensitive=true,
  morecomment=[l]{--},
  morestring=[b]',
}

\lstset{language=algo,
    %%% BOUCLE, TEST & Co.
      emph={importer, programme, glossaire, fonction, procedure, constante, type},
      emphstyle=\color{bleu2},
    %%% IMPORT & Co.  
	emph={[2]
		si, sinon, alors, fin , tantque, debut, faire, lorsque, fin lorsque, 
		declencher, retourner, et, ou, non,enregistrement, retourner, retourne, 
		tableau, /=, <, =, >, traite,exception, pour, a
	},
      emphstyle=[2]\color{bleu1},
    %%% FONCTIONS NUMERIQUES
      emph={[3]Entier, Reel, Booleen, Caractere, Booléen, Réél, Caractère},
      emphstyle=[3]\color{gris1},
    %%% FONCTIONS NUMERIQUES
      emph={[4]entree, maj, sortie, entrée},	
      emphstyle=[4]\color{gris1},
}
\lstdefinelanguage{wl}{%
   morekeywords={%
    %%% couleur 1
		importer, programme, glossaire, fonction, procedure, constante, type, 
	%%% IMPORT & Co.
		si, sinon, alors, fin, TANTQUE, tantque, FIN, PROCEDURE, debut, faire, lorsque, 
		fin lorsque, declenche, declencher, enregistrement, tableau, retourne, retourner, =, 
		/=, <, >, traite,exception, 
	%%% types 
		Entier, Reel, Booleen, Caractere, Réél, Booléen, Caractère,
	%%% types 
		entree, maj, sortie,entrée,
	%%% types 
		et, ou, non,
	},
  sensitive=true,
  morecomment=[l]{//},
  morestring=[b]',
}

\lstset{language=wl,
    %%% BOUCLE, TEST & Co.
      emph={importer, programme, glossaire, fonction, procedure, constante, type},
      emphstyle=\color{bleu2},
    %%% IMPORT & Co.  
	emph={[2]
		si, sinon, alors, fin , tantque, debut, faire, lorsque, fin lorsque, 
		declencher, retourner, et, ou, non,enregistrement, retourner, retourne, 
		tableau, /=, <, =, >, traite,exception
	},
      emphstyle=[2]\color{bleu1},
    %%% FONCTIONS NUMERIQUES
      emph={[3]Entier, Reel, Booleen, Caractere, Booléen, Réél, Caractère},
      emphstyle=[3]\color{gris1},
    %%% FONCTIONS NUMERIQUES
      emph={[4]entree, maj, sortie, entrée},	
      emphstyle=[4]\color{gris1},
}
\lstdefinelanguage{css}{%
   morekeywords={%
    %%% couleur 1
		background, image, repeat, position, index, color, border, font, 
		size, url, family, style, variant, weight, letter, spacing, line, 
		height, text, decoration, align, indent, transform, shadow, 
		background, image, repeat, position, index, color, border, font, 
		size, url, family, style, variant, weight, letter, spacing, line, 
		height, text, decoration, align, indent, transform, shadow, 
		vertical, align, white, space, word, spacing,attachment, width, 
		max, min, margin, padding, clip, direction, display, overflow,
		visibility, clear, float, top, right, bottom, left, list, type, 
		collapse, side, empty, cells, table, layout, cursor, marks, page, break,
		before, after, inside, orphans, windows, azimuth, after, before, cue, 
		elevation, pause, play, during, pitch, range, richness, spek, header, 
		numeral, punctuation, rate, stress, voice, volume,
	%%% types 
		left, right, bottom, top, none, center, solid, black, blue, red, green,
	},
  sensitive=true,
  sensitive=true,
  morecomment=[s]{/*}{*/},
  morestring=[b]',
}
\lstset{language=css,
    %%% BOUCLE, TEST & Co.
      emph={
		background, image, repeat, position, index, color, border, font, 
		size, url, family, style, variant, weight, letter, spacing, line, 
		height, text, decoration, align, indent, transform, shadow, 
		background, image, repeat, position, index, color, border, font, 
		size, url, family, style, variant, weight, letter, spacing, line, 
		height, text, decoration, align, indent, transform, shadow, 
		vertical, align, white, space, word, spacing,attachment, width, 
		max, min, margin, padding, clip, direction, display, overflow,
		visibility, clear, float, top, right, bottom, left, list, type, 
		collapse, side, empty, cells, table, layout, cursor, marks, page, break,
		before, after, inside, orphans, windows, azimuth, after, before, cue, 
		elevation, pause, play, during, pitch, range, richness, spek, header, 
		numeral, punctuation, rate, stress, voice, volume,
	  },
      emphstyle=\color{bleu2},
    %%% FONCTIONS NUMERIQUES
      emph={[3]
		left, right, bottom, top,none, solid, black, blue, green,
		  },
      emphstyle=[3]\color{bleu3},
    %%% FONCTIONS NUMERIQUES
}

\lstset{language=SQL,
    %%% BOUCLE, TEST & Co.
      emph={INSERT, UPDATE, DELETE, WHERE, SET, GROUP, BY, ORDER, REFERENCES},
      emphstyle=\color{bleu2},
    %%% IMPORT & Co.  
	emph={[2]
		if, end, begin, then, for, each, else, after, of, on, to
	},
      emphstyle=[2]\color{bleu1},
    %%% FONCTIONS NUMERIQUES
      emph={[3]Entier, Reel, Booleen, Caractere, Booléen, Réél, Caractère},
      emphstyle=[3]\color{gris1},
    %%% FONCTIONS NUMERIQUES
      emph={[4]entree, maj, sortie, entrée},	
      emphstyle=[4]\color{gris1},
}
\lstdefinelanguage{ARM}{%
   morekeywords={%
   ADD, SUB, MOV, MUL, RSB,CMP, BLS, BLE, B,BHI,LDR,
   BGE, RSBLT, BGT, BEQ, BNE,BLT,BHS,STR,STRB
	},
  sensitive=true,
  morecomment=[l]{@},
  morestring=[b]',
}

\lstset{ % general style for listings 
   numbers=left 
   , literate={é}{{\'e}}1 {è}{{\`e}}1 {à}{{\`a}}1 {ê}{{\^e}}1 {É}{{\'E}}1 {ô}{{\^o}}1 {€}{{\euro}}1{°}{{$^{\circ}$}}1 {ç}{ {c}}1 {ù}{u}1
	, extendedchars=\true
   , tabsize=2 
   , frame=l
   , framerule=1.1pt
   , linewidth=520px
   , breaklines=true 
   , basicstyle=\footnotesize\ttfamily 
   , numberstyle=\tiny\ttfamily 
   , framexleftmargin=0mm 
   , xleftmargin=0mm 
   , captionpos=b 
	, keywordstyle=\color{bleu2}
	, commentstyle=\color{vert}
	, stringstyle=\color{rouge}
	, showstringspaces=false
	, extendedchars=true
	, mathescape=true
} 
%	\lstlistoflistings
%	\addcontentsline{toc}{part}{List of code examples}
 %prise en charge du langage algo

\date{\today}

\chead{}
\rhead{--~ \thepage ~--}
\lhead{\titre}
\makeindex
\lfoot{Université Paul Sabatier Toulouse III}
\rfoot{\sigle\semestre}
%\rfoot{}
\cfoot{}
\makeglossary
\makeatletter
\def\clap#1{\hbox to 0pt{\hss #1\hss}}%
\def\ligne#1{%
\hbox to \hsize{%
\vbox{\centering #1}}}%
\def\haut#1#2#3{%
\hbox to \hsize{%
\rlap{\vtop{\raggedright #1}}%
\hss
\clap{\vtop{\centering #2}}%
\hss
\llap{\vtop{\raggedleft #3}}}}%
\def\bas#1#2#3{%
\hbox to \hsize{%
\rlap{\vbox{\raggedright #1}}%
\hss \clap{\vbox{\centering #2}}%
\hss
\llap{\vbox{\raggedleft #3}}}}%
\def\maketitle{%
\thispagestyle{empty}\vbox to \vsize{%
\haut{}{\@blurb}{}

\vfill
\vspace{1cm}
\begin{flushleft}
\usefont{OT1}{ptm}{m}{n}
\huge \@title
\end{flushleft}
\par
\hrule height 4pt
\par
\begin{flushright}
\usefont{OT1}{phv}{m}{n}
\Large \@author
\par
\end{flushright}
\vspace{1cm}
\vfill
\vfill
\bas{}{\@location, le \@date}{}
}%
\cleardoublepage
}
\def\date#1{\def\@date{#1}}
\def\author#1{\def\@author{#1}}
\def\title#1{\def\@title{#1}}
\def\location#1{\def\@location{#1}}
\def\blurb#1{\def\@blurb{#1}}
\date{\today}
\author{}
\title{}
\location{Amiens}\blurb{}
\makeatother
\title{\titre}
\author{Semestre \semestre}

\location{Toulouse}
\blurb{%
Université Paul Sabatier -- Toulouse III\\
IUT A - Toulouse Rangueil\\
}%



%\title{Cours \\ \titre}
%\date{\today\\ Semestre \semestre}

%\lhead{Cours: \titre}
%\chead{}
%\rhead{\thepage}

%\lfoot{Université Paul Sabatier Toulouse III}
%\cfoot{\thepage}
%\rfoot{\sigle\semestre}

\pagestyle{fancy}


\begin{document}
	\maketitle
	\chapter{Avant la création}
	\section{Survie de l'entreprise}
	\begin{itemize}
		\item Avec des salariés, le taux de survie augmente significativement! 	
		\item Reprendre une entreprise à plus de chance de survivre (passer le cap des trois ans)
			\begin{itemize}
				\item Charges patronales et salariales tombent toutes au bout de trois ans, c'est donc ce cap qui est difficile à franchir
			\end{itemize}
		\item Meilleure chance de survie avec un capital superieur ou égale à 7500 euros
	\end{itemize}
	\section{La volonté d'entreprendre}
	Il faut connaître les avantages et les inconvénients de créer son entreprise pour  
	\subsection{Avantages}
	\subsection{Inconvénients}
	Être pret à faire des sacrifices, financiers par exemple: peu de salaire la première année, mais on peut avoir l'ACRE \footnote{Aide à la création de l'entreprise
	(état)}. 
	\section{L'idée de départ}
	\subsection{Idée non novatrice}
	On peut donc prévoir le comportement du marché, savoir comment ça marche, et récupérer une certaine clientèle, mais est-ce pertinent de s'installer dans 
	un marché déjà établis
	\subsection{Idée d'un autre}
	Peut être innovante, mais cela peut être protégé.
	\subsection{Innocation pure}
	Penser à protéger son idée, la prise de risque est importante, mais cela peut très bien marcher. 
	\chapter{Les étapes de la création}
	\section{L'élaboration du plan d'affaire}
	Il permet de répondre à plusieurs question:
		\begin{itemize}
			\item D'où l'entreprise va démarrer
			\item Quelle seront ses moyens pour parvenir à ses objectifs ?
			\item Comme va elle progresser dans le temps
		\end{itemize}
	\subsection{La partie économique}
		Évaluer le marché dans lequel va évoluer l'entreprise. 
		\begin{itemize}
			\item\textbf{ L'étude de marché }
				\begin{itemize}
					\item Analyse de la demande : Les clients
					\item Analyse de l'offre : les concurrents (nombre, caractéristiques etc\ldots)
					\item Le marché : Déterminer le lieu où on va s'implanter
				\end{itemize}
			\item \textbf{La stratégie commerciale }
				\begin{itemize}
					\item Stratégie de concentration : Une cible précise et peu concurentielle\footnote{Exemple Google}
					\item Stratégie de différenciation : Mettre en avant le caractère unique du produit\footnote{En général le produit sera haut de gamme 
						-- Exemple la mini}
					\item Stratégie de domination par les coûts : Cout faible ou haut de gamme et élevés\footnote{Les personnes qui payent du haut de gamme
						s'attendent à avoir un prix qui soit en cohérence -- Exemple mercedes}
				\end{itemize}
				Indispensable pour demander un pret)
		\end{itemize}

		\subsection{Le plan d'affaire : La partie financière}
		\begin{itemize}
			\item \textbf{La faisabilité du plan de marche} : Combien de client pour que l'activité soit perenne
			\item Compte de résultat prévisionnel(indispensable pour demander des prets)
			\item Le besoin en fonds de roulement
		\end{itemize}
	\section{Le choix du statut juridique}
			\subsubsection{L'entreprise indiduelle}
			\textit{``Vous formez une seule personne juridique avec votre entreprise''}
			\newpage
			\subsection{Avantage}
				\begin{itemize}
					\item vous êtes le seul maitre dans l'entreprise
					\item Vous pouvez recruter des salariés
				\end{itemize}
			\subsection{Inconvénient}
				\begin{itemize}
					\item Les biens personnels et de l'entreprise son confondus
				\end{itemize}
			\subsection{La société}
			\textit{``Vous donnez naissance à une personne morale, dinstincte de vous juridiquement''}
			\subsubsection{Avantages}
				\begin{itemize}
					\item Les biens personnels et de l'entreprises son séparés
					\item On peut avoir un collaborateur
				\end{itemize}
			\subsubsection{Inconvénients}
				\ldots
	\section{Les dérnières étapes}
		\subsection{Les formalités de création}
		Le centre de formalités des entreprises
		\subsection{Les aides}
	\begin{itemize}
		\item Aux conseils et à la réalisation
			\begin{itemize}
				\item Pôle emploi, CCI, CMA, APCE
			\end{itemize}
		\item Aux financements
			\begin{itemize}
				\item Prêt à taux zéro, versement des allocations chômage sous forme de capital, prêts d'honneur, concours, PRCE\ldots
			\end{itemize}
	\end{itemize}
	\subsection{L'expert comptable}
		Conseils comptable et juridique, il coute de l'argent, mais l'expert comptable peut nous allerter sur des soucis de gestion dans l'entreprise. \\
		Il évite des problèmes vis-à-vis de l'état en cas de mauvais documents. 
	\chapter{Aprés la création}
	\section{Les 3 pilliers de l'entreprise}
\begin{itemize}
	\item Le produit
	\item La force commerciale
	\item Le suivi comptable
\end{itemize}
\section{Les erreurs à ne pas commetre}
\begin{itemize}
	\item Ne pas faire suffisamment de prospection
	\item Manquer de sérieux et perdre ses clients
	\item Manquer de rigueur dans le suivi comptable
	\item Mal évaluer le temps de travail dans un projet
	\item Mal évaluer le marché et donc le positionnement tarifaire
	\item Ne pas demander d'accomptes
	\item Se verser un salaire trop important
	\item Vendre des produits obsolètes
	\item Négliger son réseau de connaissance
\end{itemize}
\section{Quelques marchés porteurs et d'avenir}
\begin{itemize}
	\item L'E-commerce et le M-commerce :Sur écran nmade, tablettes etc\ldots -- exemple Ebay mobile
	\item Le marché des CRM
	\item Le secteur de l'internet mobile
\end{itemize}

\end{document}






