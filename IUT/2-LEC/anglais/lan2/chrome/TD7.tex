\documentclass{article}

\usepackage{lmodern}
\usepackage{xcolor}
\usepackage[utf8]{inputenc}
\usepackage[T1]{fontenc}
\usepackage[francais]{babel}
\usepackage[top=1.7cm, bottom=1.7cm, left=1.7cm, right=1.7cm]{geometry}
%\usepackage[frenchb]{babel}
%\usepackage{layout}
%\usepackage{setspace}
%\usepackage{soul}
%\usepackage{ulem}
%\usepackage{eurosym}
%\usepackage{bookman}
%\usepackage{charter}
%\usepackage{newcent}
%\usepackage{lmodern}
%\usepackage{mathpazo}
%\usepackage{mathptmx}
%\usepackage{url}
%\usepackage{verbatim}
%\usepackage{moreverb}
%\usepackage{wrapfig}
%\usepackage{amsmath}
%\usepackage{mathrsfs}
%\usepackage{asmthm}
%\usepackage{makeidx}
%\usepackage{tikz} %Vectoriel
\usepackage{listings}
\usepackage{fancyhdr}
\usepackage{multido}
\usepackage{amssymb}
\usepackage{graphicx}
\usepackage{multicol}


\input{/home/satenske/cours/listings.tex} %prise en charge du langage algo

\title{Google Chrome Operating System}
\date{Antoine de Roquemaurel \\Group F2}

\lhead{Google Chrome Operating System}
\rhead{Antoine de Roquemaurel}
\chead{\thepage}

\lfoot{Université Paul Sabatier Toulouse III}
\cfoot{\thepage}
\rfoot{lan2}

\pagestyle{fancy}
\begin{document}
	\maketitle
	\begin{center}
	\includegraphics[width=15cm]{logo.jpg}
	\end{center}
		\newpage
	\section{Summary}
	\textbf{Google Chrome Operating system}(chrome OS) is a Linux-based operating system. It must work 
	exclusively with \textbf{web} application. In fact, Google has develop much application in web such 
	as Google doc, Gmail, Google books etc.\\ 
	Chrome OS is an operating system will be very \textbf{speed}, because there is the minimum installed. 
	You can write or open documents with Google Doc, you can surf on the web with Google Chrome,
	you can read or send e-mail with Gmail, you can use an calculator... \\ \\
	Nothing is saved on the PC, every documents are saved on Google server, it's <<\textbf{cloud computing}>>, so if
	you forget, break or lose your pc, you data are always on the server!\\
	But many people think saved the \textbf{data} on Google server is not an good idea, because Google can use
	the information, and is it really secure?\\ \\
	The marketing of Google Chrome OS will be on \textbf{netbook}, the only way for use chrome OS is buy an 
	netbook with chrome OS (e.g some netbook Asus or Acer).\\ \\
	For help the development of Chrome OS, Google has leave an version of chrome OS who the \textbf{software} is
	free: \textbf{chromium OS}, many people can modify the code source, and distribute the software. \\
	if informations is interesting, Google can use the idea in Google chrome OS
	\section{Keywords}
	\begin{multicols}{3}
		\begin{itemize}
			\item Google
			\item Chrome OS
			\item Chromium OS
			\item Speed
			\item Operating system
			\item Web
			\item Cloud computing
			\item Data
			\item Netbook
			\item Software
		\end{itemize}
	\end{multicols}
	\newpage
	\section{Bibliography}	
		\subsection{Wikipédia}
			\subsubsection{Introduction}
			Google Chrome OS is a Linux-based operating system designed by Google to work exclusively with web applications. Google announced the operating system on July 7, 2009 and made it an open source project, called Chromium OS, that November.[2][3]\\
			Unlike Chromium OS, which can be compiled from the downloaded source code, Chrome OS will only ship on specific hardware from Google's manufacturing partners.[4] The user interface takes a minimalist approach, resembling that of the Chrome web browser. Since Google Chrome OS is aimed at users who spend most of their computer time on the Internet, the only application on the device will be a browser incorporating a media player.[2][5][6][7][8]\\
			The expected launch date for retail hardware featuring Chrome OS slipped since Google first announced the operating system: from late 2010 to June 15, 2011, when "Chromebooks" from Acer and Samsung are expected to ship.\\
			(...)	
			\subsubsection{Reception}

			Ahead of the commercial launch of Chrome OS devices, industry observers have evaluated the operating system in terms of its potential success, advantages and limitations.\\
			Early on, Chrome OS was viewed as a competitor to Microsoft, both directly to Microsoft Windows and indirectly the company's word processing and spreadsheet applications—the latter through Chrome OS's reliance on cloud computing.[44][45] But Chrome OS engineering director Matthew Papakipos argued that the two operating systems would not fully overlap in functionality because Chrome OS hosted is intended for netbooks, which lack the computational power to run a resource-intensive program like Photoshop.[5]\\
			Some observers claimed that other operating systems already fill the niche that Chrome OS is aiming for, with the added advantage of supporting native applications in addition to a browser. Tony Bradley of PC World wrote in November 2009: "We can already do most, if not all, of what Chrome OS promises to deliver. Using a Windows 7 or Linux-based netbook, users can simply not install anything but a web browser and connect to the vast array of Google products and other web-based services and applications. Netbooks have been successful at capturing the low-end PC market, and they provide a web-centric computing experience today. I am not sure why we should get excited that a year from now we'll be able to do the same thing, but locked into doing it from the fourth-place web browser."[46]\\
			A year later, Ryan Paul of Ars Technica came to similar conclusions. He wrote that Google's Cr-48 prototype "met the basic requirements for Web surfing, gaming, and personal productivity, but falls short for more intensive tasks." He praised Google's approach to security, but wondered whether mainstream computer users would accept an operating system whose only application is a browser. "In its current form, I think that the operating system could appeal to some niche audiences, like regular consumers users who really just need browsing or office productivity workers at companies that have gone Google or only use intranet apps. It's decidedly not a full-fledged alternative to the general purpose computing environments that currently ship on netbooks." Paul wrote that most of Chrome OS's advantages "can be found in other software environments without having to sacrifice native applications."[29]\\
			In reviewing the Cr-48 on December 29, 2010, Kurt Bakke of Conceivably Tech said: "in my household the Chromebook has turned into a family appliance and the most frequented computer in our household. Its 15 second startup time and dedicated Google user accounts made it the go-to device for quick searches, email as well as YouTube and Facebook activities. It has not turned into a device that can rival the appeal of any of our other notebooks – we have one gaming laptop, two mainstream notebooks and two netbooks in our household with five kids. The biggest complaint I heard was its lack of performance in Flash applications."[47]\\
			In ongoing testing, Wolfgang Gruener, also writing in Conceivably Tech, said that cloud computing at cellular data speeds is unacceptable and that the lack of offline capability turns the Cr-48 turns "into a useless brick" when not connected.[48] "It's difficult to use the Chromebook as an everyday device and give up what you are used to on a Mac/Windows PC, while you surely enjoy the dedicated cloud computing capabilities occasionally."[49] He praised a March 2011 update that included new trackpad control features, scrolling support, power improvements and a large number of bug fixes.[50]\\(...)
	\begin{verbatim}http://en.wikipedia.org/wiki/Google_Chrome_OS\end{verbatim}

	\subsection{Google's Blog}
	\subsubsection{Introducing the Google Chrome OS}
	\textit{7/07/2009 09:37:00 PM}\\
	It's been an exciting nine months since we launched the Google Chrome browser. Already, over 30 million people use it regularly. We designed Google Chrome for people who live on the web — searching for information, checking email, catching up on the news, shopping or just staying in touch with friends. However, the operating systems that browsers run on were designed in an era where there was no web. So today, we're announcing a new project that's a natural extension of Google Chrome — the Google Chrome Operating System. It's our attempt to re-think what operating systems should be.\\ 

	Google Chrome OS is an open source, lightweight operating system that will initially be targeted at netbooks. Later this year we will open-source its code, and netbooks running Google Chrome OS will be available for consumers in the second half of 2010. Because we're already talking to partners about the project, and we'll soon be working with the open source community, we wanted to share our vision now so everyone understands what we are trying to achieve.\\ 

	Speed, simplicity and security are the key aspects of Google Chrome OS. We're designing the OS to be fast and lightweight, to start up and get you onto the web in a few seconds. The user interface is minimal to stay out of your way, and most of the user experience takes place on the web. And as we did for the Google Chrome browser, we are going back to the basics and completely redesigning the underlying security architecture of the OS so that users don't have to deal with viruses, malware and security updates. It should just work.\\ 

	Google Chrome OS will run on both x86 as well as ARM chips and we are working with multiple OEMs to bring a number of netbooks to market next year. The software architecture is simple — Google Chrome running within a new windowing system on top of a Linux kernel. For application developers, the web is the platform. All web-based applications will automatically work and new applications can be written using your favorite web technologies. And of course, these apps will run not only on Google Chrome OS, but on any standards-based browser on Windows, Mac and Linux thereby giving developers the largest user base of any platform. \\ 

	Google Chrome OS is a new project, separate from Android. Android was designed from the beginning to work across a variety of devices from phones to set-top boxes to netbooks. Google Chrome OS is being created for people who spend most of their time on the web, and is being designed to power computers ranging from small netbooks to full-size desktop systems. While there are areas where Google Chrome OS and Android overlap, we believe choice will drive innovation for the benefit of everyone, including Google. \\ 

	We hear a lot from our users and their message is clear — computers need to get better. People want to get to their email instantly, without wasting time waiting for their computers to boot and browsers to start up. They want their computers to always run as fast as when they first bought them. They want their data to be accessible to them wherever they are and not have to worry about losing their computer or forgetting to back up files. Even more importantly, they don't want to spend hours configuring their computers to work with every new piece of hardware, or have to worry about constant software updates. And any time our users have a better computing experience, Google benefits as well by having happier users who are more likely to spend time on the Internet. \\ 

	We have a lot of work to do, and we're definitely going to need a lot of help from the open source community to accomplish this vision. We're excited for what's to come and we hope you are too. Stay tuned for more updates in the fall and have a great summer. \\ 

	\textit{Update on 7/8/2009: We have posted an FAQ on the Google Chrome Blog.}\\

	Posted by Sundar Pichai, VP Product Management and Linus Upson, Engineering Director
	\begin{verbatim}http://googleblog.blogspot.com/2009/07/introducing-google-chrome-os.html\end{verbatim}
	\newpage
	\subsection{Giz modo}
	\subsubsection{Everything You Need To Know About Chrome OS}
	  John Herrman —  Until today, Google's Chrome OS has been little more than a wordy concept. Now, finally, we truly know what it is, what it looks like, and how it works. Here's the breakdown:\\

	  Google went to great pains to emphasize that today's event wasn't a launch—that'll come a year from now, apparently, with a public beta still well over the horizon. This is all about a seeing the OS for the first time; understanding in real terms how it's different from what's out there; figuring out why you might actually want to use it; etc. So! Here's what we knew going in:\\

	  Google Chrome OS is an open source, lightweight operating system that will initially be targeted at netbooks" and "most of the user experience takes place on the web." That is, it's "Google Chrome running within a new windowing system on top of a Linux kernel" with the web as the platform. It runs on x86 processors (like your standard Core 2 Duo) and ARM processors (like inside every mobile smartphone). Underneath lies security architecture that's completely redesigned to be virus-resistant and easy to update.\\

	  Like I said, there were plenty of questions. Onwards:\\

	 \subsubsection{What It Is}
	  It's basically just a browser: meaning that it'll be based around preexisting web services like Gmail, Google Docs, and so on. There are going to be no conventional applications, just web applications—nothing gets installed, updated, or whatever. Seriously.\\

	  It only runs web apps: It's going to integrate web apps into the operating system deeper than we've ever seen before, meaning that a) they'll seem more like native apps than web apps and b) they'll be able to tap into local resources more than a typical web app in Firefox, for example. They're web apps in name, but they'll have native powers.\\

	  How, exactly?: With HTML 5. This is the next version of HTML, which gives the browser more access to local resources like location info, offline storage—the kinds of things you'd normally associate with native apps. More on that here.\\

	  Chrome is Chrome: The user's experience with Chrome OS will basically be synonymous with their experience on Chrome Browser. Technically speaking, Chrome OS is a Linux-based OS, but you won't be installing Linux binaries like you might on Ubuntu or some other Linux distribution. Any "apps" you have will be used within the browser. Chrome OS is effectively a new version of Chrome, that you can't leave. There are a few reasons Google's pushing this, which we'll get to in a bit.\\

	  And as you've probably guessed, it's super-light. It starts up in a matter of seconds, and boot straight into the browser. Likewise, the Chrome browser is apparently very, very optimized for Chrome OS, so it should be faster than we've ever seen it.\\

	  It won't support hard drives, just solid state storage. I mean, hard drives are dying, sure, but this is pretty bold. Hardware support sounds like it'll be pretty slim, because:\\

	   You'll have to buy a Chrome OS device: You might be able to hack this thing onto your current machine, but you won't just be able to install it to replace Windows, or opt for it on your next laptop, for example. You'll have to buy hardware that Google approved, either component by component, or in a whole package. They're already working on reference designs.\\

	  For now, it's for netbooks. It's not intended for desktops, to the point that Google is saying that the first generation of Chrome hardware will be secondary machines.\\

	 \subsubsection{How It Looks}
	  It looks like Chrome browser—specifically, like the leaked shots we saw before. As in a browser, you have tabs—these have to serve as a taskbar as well. To the left of the tabs, you have a sort of start menu, which opens up a panel full of shortcuts. These are your favorites. These are your apps. (Get used to this weird feeling, btw. That Google whole point here.\\

			  You can peg smaller windows, like chat windows or music players, to sit above your tabs at all times. This feature looks a lot like the Gchat feature in Gmail, which is to say, it's a box in the corner.\\

			  Along with tabs, it's got its own version of virtual desktops. This means you can have multiple "windows" of Chrome OS to switch between, each of which is a different set of tabs. Think one desktop for work, one for play, on for porn, etc etc etc. It's a bit like using Spaces on Mac, except only with the browser.\\

			 \subsubsection{When, and How, It's Coming}
			  Google's staying specifics on the exact release date—it'll be sometime next year—but the source code for the project is published now. That doesn't mean it's ready, really, but rather that they're just planning on developing it in the open from here on out. Expect builds to start showing up online, which'll probably work wonderfully in a virtual machine app like VirtualBox.\\

			  The code is available as part of the Chromium OS (the Chromium/Chrome distinction should be familiar to anyone who's wrestled with the open source Mac version of Chrome) project, posted here.\\

			 \subsubsection{Why It Matters}
			  With Chrome OS, Google is taking (or in a way, forcing) the operating system to go totally online. As Google's freshly designated evangelists are eager to tell you, the browser is already the center of most people's computing experience. The idea here is to make the browser powerful enough to render the rest of the operating system, and its native apps, moot.\\

			  It's more pure than a lot of people expected: When Google said that Chrome OS would be centered around the web, I think most people just assumed it would be a lightweight Linux distribution with deep integration for Google web services. It's not that. It's a browser.\\

			  But it's a browser that runs different processes for each tab, that will have access to local OS resources, will to some extent work offline. In other words, it's not really a browser in the sense that we use the word, and the web apps that we'll be using won't be like the ones we're used to now, either. The idea, here, it seems, is to replicate most, if not all, of the functionality in a native OS, while keeping the lightweight, ultra-secure framework of a thin client. In other words, Google's not asking much of its users in terms of changing how they do stuff; they're trying to change the way the operating system lets you do those things, transparently.\\

	  Think of it this way: now, the buttons in your taskbar or dock are now tabs; your email client now runs within your browser, but stores stuff offline just like Mail or Outlook; your documents will still open with a few clicks, but they'll be stored remotely (and locally only if you choose). It's all the same stuff, given to you in a different way.\\
	 \textit{Update: you can download it here. [Chrome on Giz]}
		\begin{verbatim}http://gizmodo.com/5408504/everything-you-need-to-know-about-chrome-os\end{verbatim}
		\newpage
		\subsection{Techland}
		\subsubsection{Why Chrome OS Is Still a Big Lie}
		Google likes to say that Chrome OS is an operating system that lets you do everything on the web. Don't believe it.\\

		Until now, I've been pretty optimistic about Google's web-based operating system. The idea of computing entirely over the Internet is a tantalizing taste of the future, and I've certainly enjoyed watching the Chrome Web Store become a discovery point for awesome web services.\\

		But at the Google IO conference today, the search giant made a rather disappointing announcement: After months of feedback from beta testers, Chrome OS would finally get a file browser for locally-stored data.\\

		Google should never have listened to those people.\\

		Here's the problem: Chrome OS's file browser isn't meant for significant amounts of data, because Chromebooks will run on solid state drives with little room for local storage. The idea is that you'll store all your data online, so it's never tied to a single piece of hardware.\\

		That's a great idea, but Chrome OS doesn't provide a web-based alternative to the file browser. Instead, users are on their own to choose from third-party services like Box.net and Dropbox. Those services are okay, but they're not native file browsers. To get your data onto a service like Box.net, you'll have to download it to Chrome OS, then upload it back to the storage service. That's a messy solution.\\

		What Chrome OS really needs is a web-based file manager that's fully integrated with the operating system, so although it looks like you're storing files locally, what you're really doing is putting them on Google's servers. You grab a picture from a friend's Facebook page, or a sample MP3 from an indie band's website, and those files stay online. The act of "downloading" would really be a file transfer from one location on the web to another.\\

		Instead, Chrome OS expects users to store all their data in individual services. Your photos go to Picasa. Your spreadsheets go to Google Docs. Your music goes to Google Music.\\

		I'm not convinced people are comfortable having their files tied up in specific services. That might work for casual computing devices, like iPads, but it won't fly for serious work on a laptop. Users need a central repository for all their precious data so it can be easily transferred to any number of web services. Chrome OS doesn't provide this service. Until that changes, you won't be able to do everything on the web.
		\begin{verbatim}http://techland.time.com/2011/05/12/why-chrome-os-is-still-a-big-lie/#ixzz1OKkSakmD \end{verbatim}
\end{document}

