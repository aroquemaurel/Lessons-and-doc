\documentclass{article}

\usepackage[utf8]{inputenc}
\usepackage[T1]{fontenc}
\usepackage[francais]{babel}
\usepackage[top=2cm, bottom=2cm, left=2cm, right=2cm]{geometry} %marges
\usepackage{fancyhdr}
\usepackage{multido}
\usepackage{amssymb}

\title{Questionnaire}
\date{Sécurité \& réseaux}
\newcommand{\Pointilles}[1][3]
{
	\multido{}{#1}{\makebox[\linewidth]{\dotfill}\\[\parskip]}
}
\linespread{1.1}

\lhead{Réseaux \& sécurité}
\chead{}
\rhead{\thepage}

\lfoot{Université paul sabatier}
\cfoot{\thepage}
\rfoot{Toulouse III}

\pagestyle{fancy}
\begin{document}
	\maketitle 
	\begin{center}
	\LARGE{\textit{\\\ \\En quoi l'informatique peut-elle apporter une solution aux problèmes d'insécurité?\\\ \\\ \\\ \\}}
	\end{center}
	Vous avez accepté de répondre à ce questionnaire dans le cadre de notre projet professionnel et nous vous en remercions.\\
	Ce dernier nous permettra de recueillir des informations utiles pour notre équipe composée de trois personnes. \\\\
	Si au cours de ce questionnaire vous ne souhaitez pas répondre à l'une des questions proposé veuillez nous l'indiquez.(en barrant le numéro de la question).
	\newpage
	\section{Conditions de travail}
		\subsection{Quel est en moyenne le nombre d'heure par semaine?}
			\noindent\Pointilles[4]
		\subsection{Vos horaires sont-elles flexibles?}
			\noindent\Pointilles[9]
		\subsection{Vous arrive-t-il de prolonger votre temps de travail pour terminer ou avancer un projet? (travailler le week-end)}
			\noindent\Pointilles[9]
		\subsection{Travaillez-vous en équipe?}
			\noindent\Pointilles[7]
		\subsection{Dans quelle fourchette de salaire mensuel net vous situez vous?}
			\noindent
			$\square$ Inférieur à 1000 $\square$ entre 1000 et 1500 $\square$ entre 1500 et 2000 $\square$ entre 2000 et 2500 $\square$ Superieur à 2500
		\subsection{Quels sont les avantages de votre profession?}
			\noindent\Pointilles[8]
		\subsection{Les inconvénients?} 
			\noindent\Pointilles[8]
		\subsection{Vous arrive-t-il de travailler à domicile?}
			\noindent\Pointilles[8]
		\subsection{Êtes vous au contact du client?}
			\noindent\Pointilles[8]
		\newpage
	\section{Travail}
		\subsection{Quel type de données protégez-vous?}
			\noindent \Pointilles[7]
		\subsection{Avez vous déjà restauré un de vos sytèmes suite à une attaque?}
			\noindent \Pointilles[7]
		\subsection{Essayez vous de retrouver la source des attaques ou vous contentez vous de protéger les sytèmes?}
			\noindent \Pointilles[6]
		\subsection{Quelle type de demande est la plus récurrente?}
			\noindent \Pointilles[6]
		\subsection{Est-il nécessaire de s'informer sur les nouvelles failles et leurs évolutions (conférence, meeting...)}
			\noindent \Pointilles[6]
		\subsection{Quelles études avez vous fait?}
			\noindent \Pointilles[6]
		\subsection{Quelles sont les compétences requises?}
			\noindent \Pointilles[6]
		\subsection{Quel est votre secteur d'activité (particulié/professionnel/...)}
			\noindent \Pointilles[6]
		\subsection{Est-ce qu'il y a des évolutions de carrière?}
			\noindent \Pointilles[6]
		\subsection{Quelles sont vos principales missions?}
			\noindent \Pointilles[7]
		\subsection{Quels peuvent être les menaces?}
			\noindent\Pointilles[7]
		\subsection{Comment évoluent les systèmes de sécurité? Renouvellez vous le système d'un client gratuitement si une faille est détéctée?}
			\noindent\Pointilles[7]
		\subsection{Comment se déroule la formation continue après les études?}
			\noindent\Pointilles[7]
		\subsection{Travaillez vous sur une environnement logiciel ou materiel?}
			\noindent\Pointilles[7]
	\newpage
	\section{Opinions et vie privée}
		\subsection{Avez vous une anecdote sur vos études, votre métier?}
			\noindent\Pointilles[15]
		\subsection{Que pensez vous de la loi Hadopi?}
		\noindent\Pointilles[15]
\end{document}

