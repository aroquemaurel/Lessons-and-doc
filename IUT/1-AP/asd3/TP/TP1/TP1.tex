\documentclass{article}


\usepackage{lmodern}
\usepackage{xcolor}
\usepackage[utf8]{inputenc}
\usepackage[T1]{fontenc}
\usepackage[francais]{babel}
\usepackage[top=1.7cm, bottom=1.7cm, left=1.7cm, right=1.7cm]{geometry}
\usepackage{verbatim}
\usepackage{tikz} %Vectoriel
\usepackage{listings}
\usepackage{fancyhdr}
\usepackage{multido}
\usepackage{amssymb}

\newcommand{\titre}{TAD Couleur}
\newcommand{\numTP}{1}

\newcommand{\module}{Algorithmique et Structure de données}
\newcommand{\sigle}{asd}

\newcommand{\semestre}{3}

\input{/home/satenske/cours/listings.tex} %prise en charge du langage algo
\input{/home/satenske/cours/entete_iut-tp.tex}

\begin{document}
	\maketitle
	\section{Schéma du TAD Test}
		\begin{verbatim}
	 _________________________________________________________
	|     class TestUnitaire                                  |
	|_________________________________________________________|
	|- typedef bool (*test)()                                 |
	|- test functionOfTest                                    |
	|- string erreurTest                                      |
	+---------------------------------------------------------+
	|Constructeur :                                           |
	|+ TestUnitaire(test ptrFunction, string message)         |
	|Accsseurs :                                              |
	|+ test getFunctionOfTest()                               |
	|+ string getErrorOfTest()                                |
	|_________________________________________________________|

	
	 _______________________________________		     _______________________________________ 
	|     vector<TestUnitaire> lesTests     |		    |     vector<bool> resultat             |
	+--------+------------------------------+		    +--------+------------------------------+
|   1    |         TestUnitaire         |			    |   1    |             bool             |
	|--------+------------------------------+		    |--------+------------------------------+
|   2    |              //              |			    |   2    |              //              |
	|--------+------------------------------+		    |--------+------------------------------+
|   3    |              //              |			    |   3    |              //              |
	|--------+------------------------------+		    |--------+------------------------------+
|   4    |              //              |			    |   4    |              //              |
	|--------+------------------------------+     |--------+------------------------------+
|   5    |              //              |			    |   5    |              //              |
	|--------+------------------------------+		    |--------+------------------------------+
|   6    |              //              |			    |   6    |              //              |
	|--------+------------------------------+		    |--------+------------------------------+
	|________|______________________________|		    |________|______________________________|

		\end{verbatim}
		\section{A quoi sert le classe \textit{TestUnitaire}?}
	La classe \textit{TestUnitaire} fournit des outils pour tester chaque fonction et repérer l'erreur correspondante.\\
	La classe associe un pointeur sur la fonction défaillante et un message d'erreur approprié.\\
		\begin{itemize}
			\item  \textit{functionOfTest} désigne un pointeur vers la fonction qui pose problème (accesseur : \textit{getFunctionOfTest()})
			\item  \textit{errorTest} contient le message de l'erreur à afficher (accesseur : \textit{getErrorTest()})
		\end{itemize}
		\section{Comment fonctionne la procédure \textit{runAllTests}?}
		La procédure \textit{runAllTest} initialise les \textit{TestUnitaires}, les effectuent et nous signale une erreur le cas échéant.\\
		Elle initialise le vecteur \textit{lesTest} avec la fonction \textit{initializeAllTests} et rempli le vecteur résultat de 'false'.\\
		Ensuite, pour chaque \textit{TestUnitaire}, ont vérifie que les fonctions ne posent pas de problème et modifie le vecteur résultat en conséquence.
		\begin{itemize}
			\item true $\Rightarrow$ pas de soucis
			\item false $\Rightarrow$ erreur à traiter
		\end{itemize}
		Elle vérifie que tous les tests ont réussis (\textit{testsAllOkay(resultat)})\\
		Si les tests ont réussi on affiche "Tous les tests sont Okay",
		dans le cas contraire on affiche le message d'erreur correspondant aux différentes erreurs 
		rencontrées à l'aide de la fonction \textit{getErrorTest()}

		\section{Comment ajouter un nouveau test? }
		\begin{enumerate}
			\item  Ajouter un fonction de test dans le main.cpp
			\item  Ajouter une entrée dans cette fonction :
\begin{lstlisting}[language=C++, caption=Initialisation du test]
void initializeAllTests(vector<TestUnitaire>& lesTests) {
	// ... Tests précédents 
	lesTests.push_back(TestUnitaire(&nomDeMaFonctionDeTest, 
		string ("Message message d'erreur pour déterminer le problème"));
}
\end{lstlisting}
	\item  Recompiler et tester sa fonction de test en faisant une erreur volontaire et vérifier que le test échoue, puis corriger l'erreur et vérifier que le test réussi.
	\item  En cas d'erreur non volontaire, corriger cette erreur dans notre fonction de test.
	\end{enumerate}

	\section{Que garder pour écrire les tests d'une nouvelle classe?}
	On garde le TAD étudié en 1.: les deux vecteurs (\textit{vector<bool> resultat} et \textit{vector<TestUnitaire> lesTests}). De même pour la classe \textit{TestUnitaire}.\\
\begin{lstlisting}[language=C++, caption=Classe TestUnitaire  vector resultat et vector lesTests]
vector<bool> resultat;
vector<TestUnitaire> lesTests;
class TestUnitaire{ 
   private :  
      // déclaration d'un pointeur de fonction
      // la fonction ne prend pas de paramètres et renvoie un booléen
      typedef bool (*test)();  
      test functionOfTest;
      string errorTest;
   
   public :
      // constructeur de test
      TestUnitaire(test ptrFunction, string message) 
      {
         this->functionOfTest = ptrFunction;
         this->errorTest = message;
      }
      // les accesseurs en lecture
      test getFunctionOfTest() const { return this->functionOfTest; }
      string getErrorTest() const { return this->errorTest; }
};
\end{lstlisting}
	Puis toutes les fonctions d'automatisation des tests :
\begin{lstlisting}[language=C++, caption=Prototype des fonctions à garder]
void initializeAllTests(vector<TestUnitaire>& lesTests)
bool testsAllOkay(vector<bool>& resultats)
void runAllTests()
int main(int argc, char *argv[])
\end{lstlisting}

	Pour ajouter un test, cf. Question 4
		
			

\end{document}




