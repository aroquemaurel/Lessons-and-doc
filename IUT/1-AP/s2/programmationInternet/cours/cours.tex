\documentclass[12pt,a4paper,openany]{article}

\usepackage{lmodern}
\usepackage{xcolor}
\usepackage[utf8]{inputenc}
\usepackage[T1]{fontenc}
\usepackage[francais]{babel}
\usepackage[top=1.7cm, bottom=1.7cm, left=1.7cm, right=1.7cm]{geometry}
%\usepackage[frenchb]{babel}
%\usepackage{layout}
%\usepackage{setspace}
%\usepackage{soul}
%\usepackage{ulem}
%\usepackage{eurosym}
%\usepackage{bookman}
%\usepackage{charter}
%\usepackage{newcent}
%\usepackage{lmodern}
%\usepackage{mathpazo}
%\usepackage{mathptmx}
%\usepackage{url}
%\usepackage{verbatim}
%\usepackage{moreverb}
%\usepackage{wrapfig}
%\usepackage{amsmath}
%\usepackage{mathrsfs}
%\usepackage{asmthm}
%\usepackage{makeidx}
%\usepackage{tikz} %Vectoriel
\usepackage{listings}
\usepackage{fancyhdr}
%\usepackage{multido}
%\usepackage{amssymb}

\definecolor{gris1}{gray}{0.40}
\definecolor{gris2}{gray}{0.55}
\definecolor{gris3}{gray}{0.65}
\definecolor{gris4}{gray}{0.50}
\definecolor{vert}{rgb}{0,0.4,0}
\definecolor{violet}{rgb}{0.65, 0.2, 0.65}
\definecolor{bleu1}{rgb}{0,0,0.8}
\definecolor{bleu2}{rgb}{0,0.2,0.6}
\definecolor{bleu3}{rgb}{0,0.2,0.2}
\definecolor{rouge}{HTML}{F93928}


\lstdefinelanguage{algo}{%
   morekeywords={%
    %%% couleur 1
		importer, programme, glossaire, fonction, procedure, constante, type, 
	%%% IMPORT & Co.
		si, sinon, alors, fin, tantque, debut, faire, lorsque, fin lorsque, 
		declenche, declencher, enregistrement, tableau, retourne, retourner, =, pour, a,
		/=, <, >, traite,exception, 
	%%% types 
		Entier, Reel, Booleen, Caractere, Réél, Booléen, Caractère,
	%%% types 
		entree, maj, sortie,entrée,
	%%% types 
		et, ou, non,
	},
  sensitive=true,
  morecomment=[l]{--},
  morestring=[b]',
}

\lstset{language=algo,
    %%% BOUCLE, TEST & Co.
      emph={importer, programme, glossaire, fonction, procedure, constante, type},
      emphstyle=\color{bleu2},
    %%% IMPORT & Co.  
	emph={[2]
		si, sinon, alors, fin , tantque, debut, faire, lorsque, fin lorsque, 
		declencher, retourner, et, ou, non,enregistrement, retourner, retourne, 
		tableau, /=, <, =, >, traite,exception, pour, a
	},
      emphstyle=[2]\color{bleu1},
    %%% FONCTIONS NUMERIQUES
      emph={[3]Entier, Reel, Booleen, Caractere, Booléen, Réél, Caractère},
      emphstyle=[3]\color{gris1},
    %%% FONCTIONS NUMERIQUES
      emph={[4]entree, maj, sortie, entrée},	
      emphstyle=[4]\color{gris1},
}
\lstdefinelanguage{wl}{%
   morekeywords={%
    %%% couleur 1
		importer, programme, glossaire, fonction, procedure, constante, type, 
	%%% IMPORT & Co.
		si, sinon, alors, fin, TANTQUE, tantque, FIN, PROCEDURE, debut, faire, lorsque, 
		fin lorsque, declenche, declencher, enregistrement, tableau, retourne, retourner, =, 
		/=, <, >, traite,exception, 
	%%% types 
		Entier, Reel, Booleen, Caractere, Réél, Booléen, Caractère,
	%%% types 
		entree, maj, sortie,entrée,
	%%% types 
		et, ou, non,
	},
  sensitive=true,
  morecomment=[l]{//},
  morestring=[b]',
}

\lstset{language=wl,
    %%% BOUCLE, TEST & Co.
      emph={importer, programme, glossaire, fonction, procedure, constante, type},
      emphstyle=\color{bleu2},
    %%% IMPORT & Co.  
	emph={[2]
		si, sinon, alors, fin , tantque, debut, faire, lorsque, fin lorsque, 
		declencher, retourner, et, ou, non,enregistrement, retourner, retourne, 
		tableau, /=, <, =, >, traite,exception
	},
      emphstyle=[2]\color{bleu1},
    %%% FONCTIONS NUMERIQUES
      emph={[3]Entier, Reel, Booleen, Caractere, Booléen, Réél, Caractère},
      emphstyle=[3]\color{gris1},
    %%% FONCTIONS NUMERIQUES
      emph={[4]entree, maj, sortie, entrée},	
      emphstyle=[4]\color{gris1},
}
\lstdefinelanguage{css}{%
   morekeywords={%
    %%% couleur 1
		background, image, repeat, position, index, color, border, font, 
		size, url, family, style, variant, weight, letter, spacing, line, 
		height, text, decoration, align, indent, transform, shadow, 
		background, image, repeat, position, index, color, border, font, 
		size, url, family, style, variant, weight, letter, spacing, line, 
		height, text, decoration, align, indent, transform, shadow, 
		vertical, align, white, space, word, spacing,attachment, width, 
		max, min, margin, padding, clip, direction, display, overflow,
		visibility, clear, float, top, right, bottom, left, list, type, 
		collapse, side, empty, cells, table, layout, cursor, marks, page, break,
		before, after, inside, orphans, windows, azimuth, after, before, cue, 
		elevation, pause, play, during, pitch, range, richness, spek, header, 
		numeral, punctuation, rate, stress, voice, volume,
	%%% types 
		left, right, bottom, top, none, center, solid, black, blue, red, green,
	},
  sensitive=true,
  sensitive=true,
  morecomment=[s]{/*}{*/},
  morestring=[b]',
}
\lstset{language=css,
    %%% BOUCLE, TEST & Co.
      emph={
		background, image, repeat, position, index, color, border, font, 
		size, url, family, style, variant, weight, letter, spacing, line, 
		height, text, decoration, align, indent, transform, shadow, 
		background, image, repeat, position, index, color, border, font, 
		size, url, family, style, variant, weight, letter, spacing, line, 
		height, text, decoration, align, indent, transform, shadow, 
		vertical, align, white, space, word, spacing,attachment, width, 
		max, min, margin, padding, clip, direction, display, overflow,
		visibility, clear, float, top, right, bottom, left, list, type, 
		collapse, side, empty, cells, table, layout, cursor, marks, page, break,
		before, after, inside, orphans, windows, azimuth, after, before, cue, 
		elevation, pause, play, during, pitch, range, richness, spek, header, 
		numeral, punctuation, rate, stress, voice, volume,
	  },
      emphstyle=\color{bleu2},
    %%% FONCTIONS NUMERIQUES
      emph={[3]
		left, right, bottom, top,none, solid, black, blue, green,
		  },
      emphstyle=[3]\color{bleu3},
    %%% FONCTIONS NUMERIQUES
}

\lstset{language=SQL,
    %%% BOUCLE, TEST & Co.
      emph={INSERT, UPDATE, DELETE, WHERE, SET, GROUP, BY, ORDER, REFERENCES},
      emphstyle=\color{bleu2},
    %%% IMPORT & Co.  
	emph={[2]
		if, end, begin, then, for, each, else, after, of, on, to
	},
      emphstyle=[2]\color{bleu1},
    %%% FONCTIONS NUMERIQUES
      emph={[3]Entier, Reel, Booleen, Caractere, Booléen, Réél, Caractère},
      emphstyle=[3]\color{gris1},
    %%% FONCTIONS NUMERIQUES
      emph={[4]entree, maj, sortie, entrée},	
      emphstyle=[4]\color{gris1},
}
\lstdefinelanguage{ARM}{%
   morekeywords={%
   ADD, SUB, MOV, MUL, RSB,CMP, BLS, BLE, B,BHI,LDR,
   BGE, RSBLT, BGT, BEQ, BNE,BLT,BHS,STR,STRB
	},
  sensitive=true,
  morecomment=[l]{@},
  morestring=[b]',
}

\lstset{ % general style for listings 
   numbers=left 
   , literate={é}{{\'e}}1 {è}{{\`e}}1 {à}{{\`a}}1 {ê}{{\^e}}1 {É}{{\'E}}1 {ô}{{\^o}}1 {€}{{\euro}}1{°}{{$^{\circ}$}}1 {ç}{ {c}}1 {ù}{u}1
	, extendedchars=\true
   , tabsize=2 
   , frame=l
   , framerule=1.1pt
   , linewidth=520px
   , breaklines=true 
   , basicstyle=\footnotesize\ttfamily 
   , numberstyle=\tiny\ttfamily 
   , framexleftmargin=0mm 
   , xleftmargin=0mm 
   , captionpos=b 
	, keywordstyle=\color{bleu2}
	, commentstyle=\color{vert}
	, stringstyle=\color{rouge}
	, showstringspaces=false
	, extendedchars=true
	, mathescape=true
} 
%	\lstlistoflistings
%	\addcontentsline{toc}{part}{List of code examples}


\title{Cours\\ Programmation Internet}
\date{PGI\\ Semestre 2}

\lhead{Cours: Programmation Internet}
\chead{}
\rhead{\thepage}

\lfoot{Université paul sabatier Toulouse III}
\cfoot{\thepage}
\rfoot{pgi2}

\pagestyle{fancy}
\begin{document}
	\maketitle
	\chapter{HTML}
		\begin{itemize}
			\item Balises
			\item Doctype
			\item Formulaires
		\end{itemize}
%		\lstinputlisting[language=html]{test.html}	
		\paragraph{CMS} Content Managment System. Créer des sites sans utiliser 
		du code HTML, automatisé, souvent logiciels libre, rapide et plus facile.
		\section{Formulaires}
		\paragraph{} Permet d'entrer des données par l'utilisateurs, lié à une 
		page dynamique, forum; chat etc...\\
		%% Graphic for TeX using PGF
% Title: /usr/home/satenske/Diagram1.dia
% Creator: Dia v0.97.1
% CreationDate: Wed Mar 30 09:12:35 2011
% For: satenske
% \usepackage{tikz}
% The following commands are not supported in PSTricks at present
% We define them conditionally, so when they are implemented,
% this pgf file will use them.
\ifx\du\undefined
  \newlength{\du}
\fi
\setlength{\du}{15\unitlength}
\begin{tikzpicture}
\pgftransformxscale{1.000000}
\pgftransformyscale{-1.000000}
\definecolor{dialinecolor}{rgb}{0.000000, 0.000000, 0.000000}
\pgfsetstrokecolor{dialinecolor}
\definecolor{dialinecolor}{rgb}{1.000000, 1.000000, 1.000000}
\pgfsetfillcolor{dialinecolor}
\pgfsetlinewidth{0.100000\du}
\pgfsetdash{}{0pt}
\pgfsetdash{}{0pt}
\pgfsetbuttcap
\pgfsetmiterjoin
\pgfsetlinewidth{0.001000\du}
\pgfsetbuttcap
\pgfsetmiterjoin
\pgfsetdash{}{0pt}
\definecolor{dialinecolor}{rgb}{0.717647, 0.717647, 0.615686}
\pgfsetfillcolor{dialinecolor}
\pgfpathmoveto{\pgfpoint{9.544836\du}{9.940680\du}}
\pgfpathlineto{\pgfpoint{11.543262\du}{9.940680\du}}
\pgfpathlineto{\pgfpoint{11.543262\du}{10.310013\du}}
\pgfpathlineto{\pgfpoint{9.544836\du}{10.310013\du}}
\pgfpathlineto{\pgfpoint{9.544836\du}{9.940680\du}}
\pgfusepath{fill}
\pgfsetbuttcap
\pgfsetmiterjoin
\pgfsetdash{}{0pt}
\definecolor{dialinecolor}{rgb}{0.286275, 0.286275, 0.211765}
\pgfsetstrokecolor{dialinecolor}
\pgfpathmoveto{\pgfpoint{9.544836\du}{9.940680\du}}
\pgfpathlineto{\pgfpoint{11.543262\du}{9.940680\du}}
\pgfpathlineto{\pgfpoint{11.543262\du}{10.310013\du}}
\pgfpathlineto{\pgfpoint{9.544836\du}{10.310013\du}}
\pgfpathlineto{\pgfpoint{9.544836\du}{9.940680\du}}
\pgfusepath{stroke}
\pgfsetbuttcap
\pgfsetmiterjoin
\pgfsetdash{}{0pt}
\definecolor{dialinecolor}{rgb}{0.788235, 0.788235, 0.713726}
\pgfsetfillcolor{dialinecolor}
\pgfpathmoveto{\pgfpoint{9.544836\du}{9.940680\du}}
\pgfpathlineto{\pgfpoint{9.756738\du}{9.739798\du}}
\pgfpathlineto{\pgfpoint{11.755164\du}{9.739798\du}}
\pgfpathlineto{\pgfpoint{11.543262\du}{9.940680\du}}
\pgfpathlineto{\pgfpoint{9.544836\du}{9.940680\du}}
\pgfusepath{fill}
\pgfsetbuttcap
\pgfsetmiterjoin
\pgfsetdash{}{0pt}
\definecolor{dialinecolor}{rgb}{0.286275, 0.286275, 0.211765}
\pgfsetstrokecolor{dialinecolor}
\pgfpathmoveto{\pgfpoint{9.544836\du}{9.940680\du}}
\pgfpathlineto{\pgfpoint{9.756738\du}{9.739798\du}}
\pgfpathlineto{\pgfpoint{11.755164\du}{9.739798\du}}
\pgfpathlineto{\pgfpoint{11.543262\du}{9.940680\du}}
\pgfpathlineto{\pgfpoint{9.544836\du}{9.940680\du}}
\pgfusepath{stroke}
\pgfsetlinewidth{0.106000\du}
\pgfsetbuttcap
\pgfsetmiterjoin
\pgfsetdash{}{0pt}
\definecolor{dialinecolor}{rgb}{0.000000, 0.000000, 0.000000}
\pgfsetstrokecolor{dialinecolor}
\pgfpathmoveto{\pgfpoint{11.431486\du}{10.108816\du}}
\pgfpathlineto{\pgfpoint{10.951952\du}{10.108816\du}}
\pgfusepath{stroke}
\pgfsetlinewidth{0.001000\du}
\pgfsetbuttcap
\pgfsetmiterjoin
\pgfsetdash{}{0pt}
\definecolor{dialinecolor}{rgb}{0.478431, 0.478431, 0.352941}
\pgfsetfillcolor{dialinecolor}
\pgfpathmoveto{\pgfpoint{11.543262\du}{10.310013\du}}
\pgfpathlineto{\pgfpoint{11.755164\du}{10.097481\du}}
\pgfpathlineto{\pgfpoint{11.755164\du}{9.739798\du}}
\pgfpathlineto{\pgfpoint{11.543262\du}{9.940680\du}}
\pgfpathlineto{\pgfpoint{11.543262\du}{10.310013\du}}
\pgfusepath{fill}
\pgfsetbuttcap
\pgfsetmiterjoin
\pgfsetdash{}{0pt}
\definecolor{dialinecolor}{rgb}{0.286275, 0.286275, 0.211765}
\pgfsetstrokecolor{dialinecolor}
\pgfpathmoveto{\pgfpoint{11.543262\du}{10.310013\du}}
\pgfpathlineto{\pgfpoint{11.755164\du}{10.097481\du}}
\pgfpathlineto{\pgfpoint{11.755164\du}{9.739798\du}}
\pgfpathlineto{\pgfpoint{11.543262\du}{9.940680\du}}
\pgfpathlineto{\pgfpoint{11.543262\du}{10.310013\du}}
\pgfusepath{stroke}
\pgfsetbuttcap
\pgfsetmiterjoin
\pgfsetdash{}{0pt}
\definecolor{dialinecolor}{rgb}{0.788235, 0.788235, 0.713726}
\pgfsetfillcolor{dialinecolor}
\pgfpathmoveto{\pgfpoint{9.556171\du}{10.544270\du}}
\pgfpathlineto{\pgfpoint{9.779093\du}{10.264987\du}}
\pgfpathlineto{\pgfpoint{11.320025\du}{10.264987\du}}
\pgfpathlineto{\pgfpoint{11.097103\du}{10.544270\du}}
\pgfpathlineto{\pgfpoint{9.556171\du}{10.544270\du}}
\pgfusepath{fill}
\pgfsetbuttcap
\pgfsetmiterjoin
\pgfsetdash{}{0pt}
\definecolor{dialinecolor}{rgb}{0.286275, 0.286275, 0.211765}
\pgfsetstrokecolor{dialinecolor}
\pgfpathmoveto{\pgfpoint{9.556171\du}{10.544270\du}}
\pgfpathlineto{\pgfpoint{9.779093\du}{10.264987\du}}
\pgfpathlineto{\pgfpoint{11.320025\du}{10.264987\du}}
\pgfpathlineto{\pgfpoint{11.097103\du}{10.544270\du}}
\pgfpathlineto{\pgfpoint{9.556171\du}{10.544270\du}}
\pgfusepath{stroke}
\pgfsetbuttcap
\pgfsetmiterjoin
\pgfsetdash{}{0pt}
\definecolor{dialinecolor}{rgb}{0.478431, 0.478431, 0.352941}
\pgfsetfillcolor{dialinecolor}
\pgfpathmoveto{\pgfpoint{11.097103\du}{10.600000\du}}
\pgfpathlineto{\pgfpoint{11.320025\du}{10.365743\du}}
\pgfpathlineto{\pgfpoint{11.320025\du}{10.264987\du}}
\pgfpathlineto{\pgfpoint{11.097103\du}{10.544270\du}}
\pgfpathlineto{\pgfpoint{11.097103\du}{10.600000\du}}
\pgfusepath{fill}
\pgfsetbuttcap
\pgfsetmiterjoin
\pgfsetdash{}{0pt}
\definecolor{dialinecolor}{rgb}{0.286275, 0.286275, 0.211765}
\pgfsetstrokecolor{dialinecolor}
\pgfpathmoveto{\pgfpoint{11.097103\du}{10.600000\du}}
\pgfpathlineto{\pgfpoint{11.320025\du}{10.365743\du}}
\pgfpathlineto{\pgfpoint{11.320025\du}{10.264987\du}}
\pgfpathlineto{\pgfpoint{11.097103\du}{10.544270\du}}
\pgfpathlineto{\pgfpoint{11.097103\du}{10.600000\du}}
\pgfusepath{stroke}
\pgfsetbuttcap
\pgfsetmiterjoin
\pgfsetdash{}{0pt}
\definecolor{dialinecolor}{rgb}{0.717647, 0.717647, 0.615686}
\pgfsetfillcolor{dialinecolor}
\pgfpathmoveto{\pgfpoint{9.556171\du}{10.544270\du}}
\pgfpathlineto{\pgfpoint{11.097103\du}{10.544270\du}}
\pgfpathlineto{\pgfpoint{11.097103\du}{10.600000\du}}
\pgfpathlineto{\pgfpoint{9.556171\du}{10.600000\du}}
\pgfpathlineto{\pgfpoint{9.556171\du}{10.544270\du}}
\pgfusepath{fill}
\pgfsetbuttcap
\pgfsetmiterjoin
\pgfsetdash{}{0pt}
\definecolor{dialinecolor}{rgb}{0.286275, 0.286275, 0.211765}
\pgfsetstrokecolor{dialinecolor}
\pgfpathmoveto{\pgfpoint{9.556171\du}{10.544270\du}}
\pgfpathlineto{\pgfpoint{11.097103\du}{10.544270\du}}
\pgfpathlineto{\pgfpoint{11.097103\du}{10.600000\du}}
\pgfpathlineto{\pgfpoint{9.556171\du}{10.600000\du}}
\pgfpathlineto{\pgfpoint{9.556171\du}{10.544270\du}}
\pgfusepath{stroke}
\pgfsetbuttcap
\pgfsetmiterjoin
\pgfsetdash{}{0pt}
\definecolor{dialinecolor}{rgb}{0.000000, 0.000000, 0.000000}
\pgfsetfillcolor{dialinecolor}
\pgfpathmoveto{\pgfpoint{9.846159\du}{9.896285\du}}
\pgfpathlineto{\pgfpoint{10.013980\du}{9.739798\du}}
\pgfpathlineto{\pgfpoint{11.431486\du}{9.739798\du}}
\pgfpathlineto{\pgfpoint{11.275630\du}{9.896285\du}}
\pgfpathlineto{\pgfpoint{9.846159\du}{9.896285\du}}
\pgfusepath{fill}
\pgfsetbuttcap
\pgfsetmiterjoin
\pgfsetdash{}{0pt}
\definecolor{dialinecolor}{rgb}{0.000000, 0.000000, 0.000000}
\pgfsetstrokecolor{dialinecolor}
\pgfpathmoveto{\pgfpoint{9.846159\du}{9.896285\du}}
\pgfpathlineto{\pgfpoint{10.013980\du}{9.739798\du}}
\pgfpathlineto{\pgfpoint{11.431486\du}{9.739798\du}}
\pgfpathlineto{\pgfpoint{11.275630\du}{9.896285\du}}
\pgfpathlineto{\pgfpoint{9.846159\du}{9.896285\du}}
\pgfusepath{stroke}
\pgfsetbuttcap
\pgfsetmiterjoin
\pgfsetdash{}{0pt}
\definecolor{dialinecolor}{rgb}{0.788235, 0.788235, 0.713726}
\pgfsetfillcolor{dialinecolor}
\pgfpathmoveto{\pgfpoint{9.834824\du}{8.745151\du}}
\pgfpathlineto{\pgfpoint{9.991625\du}{8.600000\du}}
\pgfpathlineto{\pgfpoint{11.409761\du}{8.600000\du}}
\pgfpathlineto{\pgfpoint{11.252960\du}{8.745151\du}}
\pgfpathlineto{\pgfpoint{9.834824\du}{8.745151\du}}
\pgfusepath{fill}
\pgfsetbuttcap
\pgfsetmiterjoin
\pgfsetdash{}{0pt}
\definecolor{dialinecolor}{rgb}{0.286275, 0.286275, 0.211765}
\pgfsetstrokecolor{dialinecolor}
\pgfpathmoveto{\pgfpoint{9.834824\du}{8.745151\du}}
\pgfpathlineto{\pgfpoint{9.991625\du}{8.600000\du}}
\pgfpathlineto{\pgfpoint{11.409761\du}{8.600000\du}}
\pgfpathlineto{\pgfpoint{11.252960\du}{8.745151\du}}
\pgfpathlineto{\pgfpoint{9.834824\du}{8.745151\du}}
\pgfusepath{stroke}
\pgfsetbuttcap
\pgfsetmiterjoin
\pgfsetdash{}{0pt}
\definecolor{dialinecolor}{rgb}{0.717647, 0.717647, 0.615686}
\pgfsetfillcolor{dialinecolor}
\pgfpathmoveto{\pgfpoint{9.834824\du}{8.745151\du}}
\pgfpathlineto{\pgfpoint{11.264295\du}{8.745151\du}}
\pgfpathlineto{\pgfpoint{11.264295\du}{9.873615\du}}
\pgfpathlineto{\pgfpoint{9.834824\du}{9.873615\du}}
\pgfpathlineto{\pgfpoint{9.834824\du}{8.745151\du}}
\pgfusepath{fill}
\pgfsetbuttcap
\pgfsetmiterjoin
\pgfsetdash{}{0pt}
\definecolor{dialinecolor}{rgb}{0.286275, 0.286275, 0.211765}
\pgfsetstrokecolor{dialinecolor}
\pgfpathmoveto{\pgfpoint{9.834824\du}{8.745151\du}}
\pgfpathlineto{\pgfpoint{11.263665\du}{8.745151\du}}
\pgfpathlineto{\pgfpoint{11.263665\du}{9.873300\du}}
\pgfpathlineto{\pgfpoint{9.834824\du}{9.873300\du}}
\pgfpathlineto{\pgfpoint{9.834824\du}{8.745151\du}}
\pgfusepath{stroke}
\pgfsetbuttcap
\pgfsetmiterjoin
\pgfsetdash{}{0pt}
\definecolor{dialinecolor}{rgb}{1.000000, 1.000000, 1.000000}
\pgfsetfillcolor{dialinecolor}
\pgfpathmoveto{\pgfpoint{9.957620\du}{8.889987\du}}
\pgfpathlineto{\pgfpoint{11.141184\du}{8.889987\du}}
\pgfpathlineto{\pgfpoint{11.141184\du}{9.761839\du}}
\pgfpathlineto{\pgfpoint{9.957620\du}{9.761839\du}}
\pgfpathlineto{\pgfpoint{9.957620\du}{8.889987\du}}
\pgfusepath{fill}
\pgfsetbuttcap
\pgfsetmiterjoin
\pgfsetdash{}{0pt}
\definecolor{dialinecolor}{rgb}{0.286275, 0.286275, 0.211765}
\pgfsetstrokecolor{dialinecolor}
\pgfpathmoveto{\pgfpoint{9.957620\du}{8.889987\du}}
\pgfpathlineto{\pgfpoint{11.141184\du}{8.889987\du}}
\pgfpathlineto{\pgfpoint{11.141184\du}{9.761524\du}}
\pgfpathlineto{\pgfpoint{9.957620\du}{9.761524\du}}
\pgfpathlineto{\pgfpoint{9.957620\du}{8.889987\du}}
\pgfusepath{stroke}
\pgfsetbuttcap
\pgfsetmiterjoin
\pgfsetdash{}{0pt}
\definecolor{dialinecolor}{rgb}{0.478431, 0.478431, 0.352941}
\pgfsetfillcolor{dialinecolor}
\pgfpathmoveto{\pgfpoint{11.252960\du}{9.862909\du}}
\pgfpathlineto{\pgfpoint{11.409761\du}{9.706423\du}}
\pgfpathlineto{\pgfpoint{11.409761\du}{8.600000\du}}
\pgfpathlineto{\pgfpoint{11.252960\du}{8.745151\du}}
\pgfpathlineto{\pgfpoint{11.252960\du}{9.862909\du}}
\pgfusepath{fill}
\pgfsetbuttcap
\pgfsetmiterjoin
\pgfsetdash{}{0pt}
\definecolor{dialinecolor}{rgb}{0.286275, 0.286275, 0.211765}
\pgfsetstrokecolor{dialinecolor}
\pgfpathmoveto{\pgfpoint{11.252960\du}{9.862909\du}}
\pgfpathlineto{\pgfpoint{11.409761\du}{9.706423\du}}
\pgfpathlineto{\pgfpoint{11.409761\du}{8.600000\du}}
\pgfpathlineto{\pgfpoint{11.252960\du}{8.745151\du}}
\pgfpathlineto{\pgfpoint{11.252960\du}{9.862909\du}}
\pgfusepath{stroke}
\pgfsetlinewidth{0.100000\du}
\pgfsetdash{}{0pt}
\pgfsetdash{}{0pt}
\pgfsetbuttcap
\pgfsetmiterjoin
\pgfsetlinewidth{0.001000\du}
\pgfsetbuttcap
\pgfsetmiterjoin
\pgfsetdash{}{0pt}
\definecolor{dialinecolor}{rgb}{0.717647, 0.717647, 0.615686}
\pgfsetfillcolor{dialinecolor}
\pgfpathmoveto{\pgfpoint{20.628989\du}{8.998912\du}}
\pgfpathlineto{\pgfpoint{20.628989\du}{10.850000\du}}
\pgfpathlineto{\pgfpoint{21.723081\du}{10.850000\du}}
\pgfpathlineto{\pgfpoint{21.723081\du}{8.998912\du}}
\pgfpathlineto{\pgfpoint{20.628989\du}{8.998912\du}}
\pgfusepath{fill}
\pgfsetbuttcap
\pgfsetmiterjoin
\pgfsetdash{}{0pt}
\definecolor{dialinecolor}{rgb}{0.286275, 0.286275, 0.211765}
\pgfsetstrokecolor{dialinecolor}
\pgfpathmoveto{\pgfpoint{20.628989\du}{8.998912\du}}
\pgfpathlineto{\pgfpoint{20.628989\du}{10.850000\du}}
\pgfpathlineto{\pgfpoint{21.723081\du}{10.850000\du}}
\pgfpathlineto{\pgfpoint{21.723081\du}{8.998912\du}}
\pgfpathlineto{\pgfpoint{20.628989\du}{8.998912\du}}
\pgfusepath{stroke}
\pgfsetbuttcap
\pgfsetmiterjoin
\pgfsetdash{}{0pt}
\definecolor{dialinecolor}{rgb}{0.788235, 0.788235, 0.713726}
\pgfsetfillcolor{dialinecolor}
\pgfpathmoveto{\pgfpoint{20.628989\du}{8.998912\du}}
\pgfpathlineto{\pgfpoint{20.777246\du}{8.850000\du}}
\pgfpathlineto{\pgfpoint{21.871011\du}{8.850000\du}}
\pgfpathlineto{\pgfpoint{21.723081\du}{8.998912\du}}
\pgfpathlineto{\pgfpoint{20.628989\du}{8.998912\du}}
\pgfusepath{fill}
\pgfsetbuttcap
\pgfsetmiterjoin
\pgfsetdash{}{0pt}
\definecolor{dialinecolor}{rgb}{0.286275, 0.286275, 0.211765}
\pgfsetstrokecolor{dialinecolor}
\pgfpathmoveto{\pgfpoint{20.628989\du}{8.998912\du}}
\pgfpathlineto{\pgfpoint{20.777246\du}{8.850000\du}}
\pgfpathlineto{\pgfpoint{21.863811\du}{8.850000\du}}
\pgfusepath{stroke}
\pgfsetbuttcap
\pgfsetmiterjoin
\pgfsetdash{}{0pt}
\definecolor{dialinecolor}{rgb}{0.286275, 0.286275, 0.211765}
\pgfsetstrokecolor{dialinecolor}
\pgfpathmoveto{\pgfpoint{21.863811\du}{8.857527\du}}
\pgfpathlineto{\pgfpoint{21.723081\du}{8.998912\du}}
\pgfpathlineto{\pgfpoint{20.628989\du}{8.998912\du}}
\pgfusepath{stroke}
\pgfsetbuttcap
\pgfsetmiterjoin
\pgfsetdash{}{0pt}
\definecolor{dialinecolor}{rgb}{0.788235, 0.788235, 0.713726}
\pgfsetfillcolor{dialinecolor}
\pgfpathmoveto{\pgfpoint{20.696408\du}{9.106586\du}}
\pgfpathlineto{\pgfpoint{21.196163\du}{9.106586\du}}
\pgfpathlineto{\pgfpoint{21.196163\du}{9.349427\du}}
\pgfpathlineto{\pgfpoint{20.696408\du}{9.349427\du}}
\pgfpathlineto{\pgfpoint{20.696408\du}{9.106586\du}}
\pgfusepath{fill}
\pgfsetbuttcap
\pgfsetmiterjoin
\pgfsetdash{}{0pt}
\definecolor{dialinecolor}{rgb}{0.384314, 0.384314, 0.282353}
\pgfsetstrokecolor{dialinecolor}
\pgfpathmoveto{\pgfpoint{20.696408\du}{9.106586\du}}
\pgfpathlineto{\pgfpoint{21.195835\du}{9.106586\du}}
\pgfpathlineto{\pgfpoint{21.195835\du}{9.349100\du}}
\pgfpathlineto{\pgfpoint{20.696408\du}{9.349100\du}}
\pgfpathlineto{\pgfpoint{20.696408\du}{9.106586\du}}
\pgfusepath{stroke}
\pgfsetlinewidth{0.030000\du}
\pgfsetbuttcap
\pgfsetmiterjoin
\pgfsetdash{}{0pt}
\definecolor{dialinecolor}{rgb}{0.925490, 0.925490, 0.905882}
\pgfsetstrokecolor{dialinecolor}
\pgfpathmoveto{\pgfpoint{20.763828\du}{9.228334\du}}
\pgfpathlineto{\pgfpoint{21.114343\du}{9.228334\du}}
\pgfusepath{stroke}
\pgfsetlinewidth{0.001000\du}
\pgfsetbuttcap
\pgfsetmiterjoin
\pgfsetdash{}{0pt}
\definecolor{dialinecolor}{rgb}{0.478431, 0.478431, 0.352941}
\pgfsetfillcolor{dialinecolor}
\pgfpathmoveto{\pgfpoint{21.723081\du}{10.850000\du}}
\pgfpathlineto{\pgfpoint{21.871011\du}{10.700761\du}}
\pgfpathlineto{\pgfpoint{21.871011\du}{8.850000\du}}
\pgfpathlineto{\pgfpoint{21.723081\du}{8.998912\du}}
\pgfpathlineto{\pgfpoint{21.723081\du}{10.850000\du}}
\pgfusepath{fill}
\pgfsetbuttcap
\pgfsetmiterjoin
\pgfsetdash{}{0pt}
\definecolor{dialinecolor}{rgb}{0.286275, 0.286275, 0.211765}
\pgfsetstrokecolor{dialinecolor}
\pgfpathmoveto{\pgfpoint{21.723081\du}{10.850000\du}}
\pgfpathlineto{\pgfpoint{21.863811\du}{10.708288\du}}
\pgfusepath{stroke}
\pgfsetbuttcap
\pgfsetmiterjoin
\pgfsetdash{}{0pt}
\definecolor{dialinecolor}{rgb}{0.286275, 0.286275, 0.211765}
\pgfsetstrokecolor{dialinecolor}
\pgfpathmoveto{\pgfpoint{21.863811\du}{8.857527\du}}
\pgfpathlineto{\pgfpoint{21.723081\du}{8.998912\du}}
\pgfpathlineto{\pgfpoint{21.723081\du}{10.850000\du}}
\pgfusepath{stroke}
\pgfsetlinewidth{0.030000\du}
\pgfsetbuttcap
\pgfsetmiterjoin
\pgfsetdash{}{0pt}
\definecolor{dialinecolor}{rgb}{0.925490, 0.925490, 0.905882}
\pgfsetstrokecolor{dialinecolor}
\pgfpathmoveto{\pgfpoint{20.642734\du}{10.727925\du}}
\pgfpathlineto{\pgfpoint{21.722754\du}{10.727925\du}}
\pgfusepath{stroke}
\pgfsetbuttcap
\pgfsetmiterjoin
\pgfsetdash{}{0pt}
\definecolor{dialinecolor}{rgb}{0.000000, 0.000000, 0.000000}
\pgfsetstrokecolor{dialinecolor}
\pgfpathmoveto{\pgfpoint{20.642734\du}{9.741834\du}}
\pgfpathlineto{\pgfpoint{21.722754\du}{9.741834\du}}
\pgfusepath{stroke}
\pgfsetbuttcap
\pgfsetmiterjoin
\pgfsetdash{}{0pt}
\definecolor{dialinecolor}{rgb}{0.286275, 0.286275, 0.211765}
\pgfsetstrokecolor{dialinecolor}
\pgfpathmoveto{\pgfpoint{20.628989\du}{10.714507\du}}
\pgfpathlineto{\pgfpoint{21.721772\du}{10.714507\du}}
\pgfusepath{stroke}
\pgfsetbuttcap
\pgfsetmiterjoin
\pgfsetdash{}{0pt}
\definecolor{dialinecolor}{rgb}{0.000000, 0.000000, 0.000000}
\pgfsetstrokecolor{dialinecolor}
\pgfpathmoveto{\pgfpoint{20.628989\du}{9.728089\du}}
\pgfpathlineto{\pgfpoint{21.721772\du}{9.728089\du}}
\pgfusepath{stroke}
\pgfsetlinewidth{0.001000\du}
\pgfsetbuttcap
\pgfsetmiterjoin
\pgfsetdash{}{0pt}
\definecolor{dialinecolor}{rgb}{0.925490, 0.925490, 0.905882}
\pgfsetstrokecolor{dialinecolor}
\pgfpathmoveto{\pgfpoint{20.696408\du}{9.336336\du}}
\pgfpathlineto{\pgfpoint{20.696408\du}{9.106586\du}}
\pgfpathlineto{\pgfpoint{21.182417\du}{9.106586\du}}
\pgfusepath{stroke}
\pgfsetlinewidth{0.100000\du}
\pgfsetdash{}{0pt}
\pgfsetdash{}{0pt}
\pgfsetbuttcap
\pgfsetmiterjoin
\pgfsetlinewidth{0.001000\du}
\pgfsetbuttcap
\pgfsetmiterjoin
\pgfsetdash{}{0pt}
\definecolor{dialinecolor}{rgb}{0.788235, 0.788235, 0.713726}
\pgfsetfillcolor{dialinecolor}
\pgfpathmoveto{\pgfpoint{22.752053\du}{13.227519\du}}
\pgfpathlineto{\pgfpoint{22.999199\du}{13.000000\du}}
\pgfpathlineto{\pgfpoint{25.247947\du}{13.000000\du}}
\pgfpathlineto{\pgfpoint{25.000401\du}{13.227519\du}}
\pgfpathlineto{\pgfpoint{22.752053\du}{13.227519\du}}
\pgfusepath{fill}
\pgfsetbuttcap
\pgfsetmiterjoin
\pgfsetdash{}{0pt}
\definecolor{dialinecolor}{rgb}{0.286275, 0.286275, 0.211765}
\pgfsetstrokecolor{dialinecolor}
\pgfpathmoveto{\pgfpoint{22.763669\du}{13.217104\du}}
\pgfpathlineto{\pgfpoint{22.989986\du}{13.008412\du}}
\pgfpathlineto{\pgfpoint{25.239135\du}{13.008412\du}}
\pgfpathlineto{\pgfpoint{25.000401\du}{13.227519\du}}
\pgfpathlineto{\pgfpoint{22.763669\du}{13.227519\du}}
\pgfpathlineto{\pgfpoint{22.763669\du}{13.217104\du}}
\pgfusepath{stroke}
\pgfsetbuttcap
\pgfsetmiterjoin
\pgfsetdash{}{0pt}
\definecolor{dialinecolor}{rgb}{0.717647, 0.717647, 0.615686}
\pgfsetfillcolor{dialinecolor}
\pgfpathmoveto{\pgfpoint{22.752053\du}{13.227519\du}}
\pgfpathlineto{\pgfpoint{25.018426\du}{13.227519\du}}
\pgfpathlineto{\pgfpoint{25.018426\du}{15.000000\du}}
\pgfpathlineto{\pgfpoint{22.752053\du}{15.000000\du}}
\pgfpathlineto{\pgfpoint{22.752053\du}{13.227519\du}}
\pgfusepath{fill}
\pgfsetbuttcap
\pgfsetmiterjoin
\pgfsetdash{}{0pt}
\definecolor{dialinecolor}{rgb}{0.286275, 0.286275, 0.211765}
\pgfsetstrokecolor{dialinecolor}
\pgfpathmoveto{\pgfpoint{22.763669\du}{13.227519\du}}
\pgfpathlineto{\pgfpoint{25.017625\du}{13.227519\du}}
\pgfpathlineto{\pgfpoint{25.017625\du}{14.999199\du}}
\pgfpathlineto{\pgfpoint{22.763669\du}{14.999199\du}}
\pgfpathlineto{\pgfpoint{22.763669\du}{13.227519\du}}
\pgfusepath{stroke}
\pgfsetbuttcap
\pgfsetmiterjoin
\pgfsetdash{}{0pt}
\definecolor{dialinecolor}{rgb}{0.478431, 0.478431, 0.352941}
\pgfsetfillcolor{dialinecolor}
\pgfpathmoveto{\pgfpoint{25.000401\du}{14.982776\du}}
\pgfpathlineto{\pgfpoint{25.247947\du}{14.736832\du}}
\pgfpathlineto{\pgfpoint{25.247947\du}{13.000000\du}}
\pgfpathlineto{\pgfpoint{25.000401\du}{13.227519\du}}
\pgfpathlineto{\pgfpoint{25.000401\du}{14.982776\du}}
\pgfusepath{fill}
\pgfsetbuttcap
\pgfsetmiterjoin
\pgfsetdash{}{0pt}
\definecolor{dialinecolor}{rgb}{0.286275, 0.286275, 0.211765}
\pgfsetstrokecolor{dialinecolor}
\pgfpathmoveto{\pgfpoint{25.000401\du}{14.982776\du}}
\pgfpathlineto{\pgfpoint{25.247947\du}{14.736832\du}}
\pgfpathlineto{\pgfpoint{25.247947\du}{13.008412\du}}
\pgfpathlineto{\pgfpoint{25.239135\du}{13.008412\du}}
\pgfpathlineto{\pgfpoint{25.000401\du}{13.227519\du}}
\pgfpathlineto{\pgfpoint{25.000401\du}{14.982776\du}}
\pgfpathlineto{\pgfpoint{25.000401\du}{14.982776\du}}
\pgfusepath{stroke}
\pgfsetlinewidth{0.050000\du}
\pgfsetdash{}{0pt}
\pgfsetdash{}{0pt}
\pgfsetbuttcap
{
\definecolor{dialinecolor}{rgb}{0.000000, 0.000000, 0.000000}
\pgfsetfillcolor{dialinecolor}
% was here!!!
\pgfsetarrowsend{to}
\definecolor{dialinecolor}{rgb}{0.000000, 0.000000, 0.000000}
\pgfsetstrokecolor{dialinecolor}
\draw (11.755164\du,9.918640\du)--(20.628989\du,9.924456\du);
}
\pgfsetlinewidth{0.100000\du}
\pgfsetdash{}{0pt}
\pgfsetdash{}{0pt}
\pgfsetbuttcap
{
\definecolor{dialinecolor}{rgb}{0.000000, 0.000000, 0.000000}
\pgfsetfillcolor{dialinecolor}
% was here!!!
\pgfsetarrowsend{to}
\definecolor{dialinecolor}{rgb}{0.000000, 0.000000, 0.000000}
\pgfsetstrokecolor{dialinecolor}
\draw (21.870751\du,10.650831\du)--(23.500000\du,12.750000\du);
}
\pgfsetlinewidth{0.050000\du}
\pgfsetdash{}{0pt}
\pgfsetdash{}{0pt}
\pgfsetbuttcap
{
\definecolor{dialinecolor}{rgb}{0.000000, 0.000000, 0.000000}
\pgfsetfillcolor{dialinecolor}
% was here!!!
\pgfsetarrowsend{to}
\definecolor{dialinecolor}{rgb}{0.000000, 0.000000, 0.000000}
\pgfsetstrokecolor{dialinecolor}
\draw (22.757095\du,13.994376\du)--(17.400000\du,14.000000\du);
}
\pgfsetlinewidth{0.050000\du}
\pgfsetdash{}{0pt}
\pgfsetdash{}{0pt}
\pgfsetbuttcap
\pgfsetmiterjoin
\pgfsetlinewidth{0.050000\du}
\pgfsetbuttcap
\pgfsetmiterjoin
\pgfsetdash{}{0pt}
\definecolor{dialinecolor}{rgb}{1.000000, 1.000000, 1.000000}
\pgfsetfillcolor{dialinecolor}
\fill (14.682258\du,12.900000\du)--(14.682258\du,15.605000\du)--(17.300000\du,15.605000\du)--(17.300000\du,12.900000\du)--cycle;
\definecolor{dialinecolor}{rgb}{0.000000, 0.000000, 0.000000}
\pgfsetstrokecolor{dialinecolor}
\draw (14.682258\du,12.900000\du)--(14.682258\du,15.605000\du)--(17.300000\du,15.605000\du)--(17.300000\du,12.900000\du)--cycle;
\pgfsetbuttcap
\pgfsetmiterjoin
\pgfsetdash{}{0pt}
\definecolor{dialinecolor}{rgb}{0.000000, 0.000000, 0.000000}
\pgfsetstrokecolor{dialinecolor}
\draw (14.682258\du,12.900000\du)--(14.682258\du,15.605000\du)--(17.300000\du,15.605000\du)--(17.300000\du,12.900000\du)--cycle;
% setfont left to latex
\definecolor{dialinecolor}{rgb}{0.000000, 0.000000, 0.000000}
\pgfsetstrokecolor{dialinecolor}
\node[anchor=west] at (15.239919\du,14.659583\du){html};
\pgfsetlinewidth{0.050000\du}
\pgfsetdash{}{0pt}
\pgfsetdash{}{0pt}
\pgfsetbuttcap
{
\definecolor{dialinecolor}{rgb}{0.000000, 0.000000, 0.000000}
\pgfsetfillcolor{dialinecolor}
% was here!!!
\pgfsetarrowsend{to}
\definecolor{dialinecolor}{rgb}{0.000000, 0.000000, 0.000000}
\pgfsetstrokecolor{dialinecolor}
\draw (14.682258\du,14.252500\du)--(10.500000\du,10.750000\du);
}
% setfont left to latex
\definecolor{dialinecolor}{rgb}{0.000000, 0.000000, 0.000000}
\pgfsetstrokecolor{dialinecolor}
\node[anchor=west] at (8.950000\du,14.350000\du){Navigateur};
% setfont left to latex
\definecolor{dialinecolor}{rgb}{0.000000, 0.000000, 0.000000}
\pgfsetstrokecolor{dialinecolor}
\node[anchor=west] at (10.800000\du,7.600000\du){Client};
% setfont left to latex
\definecolor{dialinecolor}{rgb}{0.000000, 0.000000, 0.000000}
\pgfsetstrokecolor{dialinecolor}
\node[anchor=west] at (20.250000\du,7.800000\du){Serveur};
% setfont left to latex
\definecolor{dialinecolor}{rgb}{0.000000, 0.000000, 0.000000}
\pgfsetstrokecolor{dialinecolor}
\node[anchor=west] at (22.950000\du,15.500000\du){Apache};
% setfont left to latex
\definecolor{dialinecolor}{rgb}{0.000000, 0.000000, 0.000000}
\pgfsetstrokecolor{dialinecolor}
\node[anchor=west] at (19.400000\du,15.200000\du){PHP};
\end{tikzpicture}

		La saisie de formulaire se traite grâce au PHP, cependant leurs création
		reste du HTML.\\
		deux méthodes, GET et POST.\\
		\subparagraph{GET} Accessible par l'URL. L'inconveignant étant la taille
		maximum de caractère (buffer clavier)
		\subparagraph{POST} Accessible grâce aux formulaires. L'inconveignant 
		étant de resaisir les informations en cas d'erreur (sauf si utilisation 
		de PHP ou JavaScript) \\
		Les champs name sont toujours associer à un nom de variable en PHP,
		toujours le renseigner, donner des noms claires.
		
		\subsection{Champ de texte monolignes}
		\lstinputlisting[language=html]{1.html}	
		\subsection{Champ de texte multilignes}
		Textarea, permet d'insérer plusieurs lignes de code, avec éventuellement 
		une scrollbar. (exemple: commentaire, message forum...)
		\lstinputlisting[language=html]{2.html}	
		\subsection{Champ de fichiers}
		Permet de poster un fichier sur le serveur, possibilité de limiter le 
		poids de fichier.\\
		ATTENTIION: Toujours vérifier l'extension de fichier et la taille du fichier!
		\lstinputlisting[language=html]{3.html}	
		\subsection{Champ caché}
		Permet de faire passer un champ que ne voit pas l'utilisateur, récupération 
		de la donnée passé grâce au PHP
		\lstinputlisting[language=html]{4.html}	
		\subsection{Bouton radio}
		Le bouton radio permet à l'utilisateur de ne choisir qu'un choix parmis 
		une liste de choix (exemple Monsieur/Madame)
		\lstinputlisting[language=html]{5.html}	
		\subsection{Bouton simple}
		\lstinputlisting[language=html]{6.html}	
		\subsection{Liste}
		\lstinputlisting[language=html]{7.html}	

	\section{En-tête de document (head)}
	La valise <head> contient de nombreux renseignement sur la page, sur l'auteur 
	ainsi que la plupart des scripts.
	\subsection{Titre}
		\lstinputlisting[language=html]{8.html}	
	\subsection{Insértion JavaScript}
		\lstinputlisting[language=html]{9.html}	
	\subsection{Style CSS}	
		\lstinputlisting[language=html]{10.html}	
	\subsection{Méta-données}
		La plupart sont utilisées pour le référencement.
		\lstinputlisting[language=html, caption="Encodage de caractère]{11.html}	
		\begin{itemize}
		\item meta description: description de la page
		\item meta keyword: mots clefs de la page
		\item meta rating: public visé
		\item meta robots: pours les bots de référencement
		\end{itemize}
		\lstinputlisting[language=html]{12.html}	
	\chapter{CSS}
		CSS =  pour compléter pour le HTML: permet de mettre en forme la page.
		Le HTML ne devrait être utilisé que pour le contenu et le html que pour
		le fond.\\
		\section{Insertion CSS}	
		Trois façon de l'insérer comme dit précédemment. 
		\subsection{FIchier à part}
		\lstinputlisting[language=html]{13.html}	
		\subsection{Dans l'en-tête}
		\lstinputlisting[language=html]{14.html}	
		\subsection{Dans le HTML}
		\lstinputlisting[language=html]{15.html}	
		\section{Syntaxe}
			Le CSS possède sa propre << syntaxe >>.\\
			Dans un CSS, on trouve trois élements différents: 
			\begin{itemize}
				\item Noms de balise
				\item Des propriétés CSS
				\item Des valeurs
			\end{itemize}
			\subsection{une Feuille de style}
		\lstinputlisting[language=css]{1.css}	
		\lstinputlisting[language=css]{2.css}	
		\subsection{Exemple de propriétés}
		\begin{itemize}
			\item Font-family
			\item Font-size
			\item Font-style
			\item Font-variangt
			\item Color
		\end{itemize}	
		\subsection{Commentaires}
			Les commentaires en css se font grâce à /* et */
		\subsection{Attributs}
			Les attributs s'utilisent pour donner un style à une ou plusieurs 
			balise. On donne un nom a la ou les balises.
			\subsubsection{id}
				id ne peut s'utiliser qu'une seule fois dans le site.  
		\lstinputlisting[language=html]{16.html}	
		\lstinputlisting[language=css]{3.css}	
			\subsubsection{class}
				Class peut s'utiliser autant de fois que l'on veut dans le site.
		\lstinputlisting[language=html]{17.html}	
		\lstinputlisting[language=css]{4.css}	
			\subsection{Balises universel}
				Elles n'ont aucune utilité dans le html, elle ne servent qu'a 
				donner un style css à des blocs.
				\subsubsection{span}
				span est une balise de type inline, elle est utilisée pour mettre
				en forme du texte, elle doit être imbriquée dans une balise de type
				block.
				\subsubsection{div}
				div est une balise de type block, retour à la ligne automatique.\\
				Elle à été créer pour remplacer les tableaux, la plus utilisée.
				Elle sert à mettre en forme le site web, avec on peut mettre en 
				forme les blocs, les situer à x pixels du bord, centrer, etc...
				\newpage
				\subsection{Exemple}
		\lstinputlisting[language=html, caption="Fichier HTML"]{18.html}	
		\lstinputlisting[language=css, caption="Fichier CSS correspondant"]{5.css}	
				\subsection{Découpage}
%				% Graphic for TeX using PGF
% Title: /usr/home/satenske/cours/AP/pgi2/cours/decoupe.dia
% Creator: Dia v0.97.1
% CreationDate: Wed Apr  6 09:25:52 2011
% For: satenske
% \usepackage{tikz}
% The following commands are not supported in PSTricks at present
% We define them conditionally, so when they are implemented,
% this pgf file will use them.
\ifx\du\undefined
  \newlength{\du}
\fi
\setlength{\du}{15\unitlength}
\begin{tikzpicture}
\pgftransformxscale{1.000000}
\pgftransformyscale{-1.000000}
\definecolor{dialinecolor}{rgb}{0.000000, 0.000000, 0.000000}
\pgfsetstrokecolor{dialinecolor}
\definecolor{dialinecolor}{rgb}{1.000000, 1.000000, 1.000000}
\pgfsetfillcolor{dialinecolor}
\definecolor{dialinecolor}{rgb}{1.000000, 1.000000, 1.000000}
\pgfsetfillcolor{dialinecolor}
\fill (6.950000\du,4.150000\du)--(6.950000\du,27.900000\du)--(21.700000\du,27.900000\du)--(21.700000\du,4.150000\du)--cycle;
\pgfsetlinewidth{0.050000\du}
\pgfsetdash{}{0pt}
\pgfsetdash{}{0pt}
\pgfsetmiterjoin
\definecolor{dialinecolor}{rgb}{0.000000, 0.000000, 0.000000}
\pgfsetstrokecolor{dialinecolor}
\draw (6.950000\du,4.150000\du)--(6.950000\du,27.900000\du)--(21.700000\du,27.900000\du)--(21.700000\du,4.150000\du)--cycle;
% setfont left to latex
\definecolor{dialinecolor}{rgb}{0.000000, 0.000000, 0.000000}
\pgfsetstrokecolor{dialinecolor}
\node at (14.325000\du,16.220000\du){Content};
\definecolor{dialinecolor}{rgb}{1.000000, 1.000000, 1.000000}
\pgfsetfillcolor{dialinecolor}
\fill (6.950000\du,4.200000\du)--(6.950000\du,8.650000\du)--(21.650000\du,8.650000\du)--(21.650000\du,4.200000\du)--cycle;
\pgfsetlinewidth{0.050000\du}
\pgfsetdash{}{0pt}
\pgfsetdash{}{0pt}
\pgfsetmiterjoin
\definecolor{dialinecolor}{rgb}{0.000000, 0.000000, 0.000000}
\pgfsetstrokecolor{dialinecolor}
\draw (6.950000\du,4.200000\du)--(6.950000\du,8.650000\du)--(21.650000\du,8.650000\du)--(21.650000\du,4.200000\du)--cycle;
% setfont left to latex
\definecolor{dialinecolor}{rgb}{0.000000, 0.000000, 0.000000}
\pgfsetstrokecolor{dialinecolor}
\node at (14.300000\du,6.620000\du){header};
\definecolor{dialinecolor}{rgb}{1.000000, 1.000000, 1.000000}
\pgfsetfillcolor{dialinecolor}
\fill (6.900000\du,25.200000\du)--(6.900000\du,28.000000\du)--(21.700000\du,28.000000\du)--(21.700000\du,25.200000\du)--cycle;
\pgfsetlinewidth{0.050000\du}
\pgfsetdash{}{0pt}
\pgfsetdash{}{0pt}
\pgfsetmiterjoin
\definecolor{dialinecolor}{rgb}{0.000000, 0.000000, 0.000000}
\pgfsetstrokecolor{dialinecolor}
\draw (6.900000\du,25.200000\du)--(6.900000\du,28.000000\du)--(21.700000\du,28.000000\du)--(21.700000\du,25.200000\du)--cycle;
% setfont left to latex
\definecolor{dialinecolor}{rgb}{0.000000, 0.000000, 0.000000}
\pgfsetstrokecolor{dialinecolor}
\node at (14.300000\du,26.795000\du){footer};
\definecolor{dialinecolor}{rgb}{1.000000, 1.000000, 1.000000}
\pgfsetfillcolor{dialinecolor}
\fill (6.950000\du,8.750000\du)--(6.950000\du,25.200000\du)--(10.750000\du,25.200000\du)--(10.750000\du,8.750000\du)--cycle;
\pgfsetlinewidth{0.050000\du}
\pgfsetdash{}{0pt}
\pgfsetdash{}{0pt}
\pgfsetmiterjoin
\definecolor{dialinecolor}{rgb}{0.000000, 0.000000, 0.000000}
\pgfsetstrokecolor{dialinecolor}
\draw (6.950000\du,8.750000\du)--(6.950000\du,25.200000\du)--(10.750000\du,25.200000\du)--(10.750000\du,8.750000\du)--cycle;
% setfont left to latex
\definecolor{dialinecolor}{rgb}{0.000000, 0.000000, 0.000000}
\pgfsetstrokecolor{dialinecolor}
\node at (8.850000\du,17.170000\du){leftbar};
\definecolor{dialinecolor}{rgb}{1.000000, 1.000000, 1.000000}
\pgfsetfillcolor{dialinecolor}
\fill (18.200000\du,8.800000\du)--(18.200000\du,14.950000\du)--(21.650000\du,14.950000\du)--(21.650000\du,8.800000\du)--cycle;
\pgfsetlinewidth{0.050000\du}
\pgfsetdash{}{0pt}
\pgfsetdash{}{0pt}
\pgfsetmiterjoin
\definecolor{dialinecolor}{rgb}{0.000000, 0.000000, 0.000000}
\pgfsetstrokecolor{dialinecolor}
\draw (18.200000\du,8.800000\du)--(18.200000\du,14.950000\du)--(21.650000\du,14.950000\du)--(21.650000\du,8.800000\du)--cycle;
% setfont left to latex
\definecolor{dialinecolor}{rgb}{0.000000, 0.000000, 0.000000}
\pgfsetstrokecolor{dialinecolor}
\node at (19.925000\du,12.070000\du){News};
\end{tikzpicture}

				\subsection{Liste des propriétés}
					\begin{tabular}{|c|c|}
						\hline
							\textbf{Propriété} & \textbf{Valeurs possibles} \\
						\hline
							background-color & couleur (hexa, rgb, nom anglais)\\
						\hline
							background-image & url('urldelimage');\\
						\hline
							background-repeat & repeat-x, repeat-y, no-repeat\\
						\hline
							background & Super Propriété\\
						\hline		
					\end{tabular}\\ \\

					\begin{tabular}{|c|c|}
						\hline
							\textbf{Propriété} & \textbf{Valeurs possibles} \\
						\hline
							text-decoration & none, underline, overline, ...\\
						\hline
					\end{tabular}\\ \\
	\section{PHP \& MySQL}
		PHP est un langage serveur qui est interprété
		servant à dynamiser une page web. \\
		En effet, avec uniquement (X)HTML, les sites sont statiques, le visiteur
		ne peut pas interagire avec la page. \\
		Cela devient possible grâce à PHP (Ex: page protégé par mot de passe)\\
		c'est encore plus puissant si on couple PHP avec une base de données 
		(exemple MySQL), on peut ainsi faire des forums, 
		chat, news automatisés...\\
		\paragraph{Exemple de concurrents PHP} PHP peut être remplacé par ASP, 
		Ruby, JEE, Python ... \\
		\paragraph{Exemple de concurrents MySQL} MySQL peut être remplacé 
			par Oracle, PortgreSQL...
%		% Graphic for TeX using PGF
% Title: /usr/home/satenske/cours/AP/pgi2/cours/schemaCS.dia
% Creator: Dia v0.97.1
% CreationDate: Wed Apr  6 09:49:30 2011
% For: satenske
% \usepackage{tikz}
% The following commands are not supported in PSTricks at present
% We define them conditionally, so when they are implemented,
% this pgf file will use them.
\ifx\du\undefined
  \newlength{\du}
\fi
\setlength{\du}{15\unitlength}
\begin{tikzpicture}
\pgftransformxscale{1.000000}
\pgftransformyscale{-1.000000}
\definecolor{dialinecolor}{rgb}{0.000000, 0.000000, 0.000000}
\pgfsetstrokecolor{dialinecolor}
\definecolor{dialinecolor}{rgb}{1.000000, 1.000000, 1.000000}
\pgfsetfillcolor{dialinecolor}
\pgfsetlinewidth{0.100000\du}
\pgfsetdash{}{0pt}
\pgfsetdash{}{0pt}
\pgfsetbuttcap
\pgfsetmiterjoin
\pgfsetlinewidth{0.001000\du}
\pgfsetbuttcap
\pgfsetmiterjoin
\pgfsetdash{}{0pt}
\definecolor{dialinecolor}{rgb}{0.717647, 0.717647, 0.615686}
\pgfsetfillcolor{dialinecolor}
\pgfpathmoveto{\pgfpoint{4.644836\du}{13.519602\du}}
\pgfpathlineto{\pgfpoint{7.804327\du}{13.519602\du}}
\pgfpathlineto{\pgfpoint{7.804327\du}{14.103513\du}}
\pgfpathlineto{\pgfpoint{4.644836\du}{14.103513\du}}
\pgfpathlineto{\pgfpoint{4.644836\du}{13.519602\du}}
\pgfusepath{fill}
\pgfsetbuttcap
\pgfsetmiterjoin
\pgfsetdash{}{0pt}
\definecolor{dialinecolor}{rgb}{0.286275, 0.286275, 0.211765}
\pgfsetstrokecolor{dialinecolor}
\pgfpathmoveto{\pgfpoint{4.644836\du}{13.519602\du}}
\pgfpathlineto{\pgfpoint{7.804327\du}{13.519602\du}}
\pgfpathlineto{\pgfpoint{7.804327\du}{14.103513\du}}
\pgfpathlineto{\pgfpoint{4.644836\du}{14.103513\du}}
\pgfpathlineto{\pgfpoint{4.644836\du}{13.519602\du}}
\pgfusepath{stroke}
\pgfsetbuttcap
\pgfsetmiterjoin
\pgfsetdash{}{0pt}
\definecolor{dialinecolor}{rgb}{0.788235, 0.788235, 0.713726}
\pgfsetfillcolor{dialinecolor}
\pgfpathmoveto{\pgfpoint{4.644836\du}{13.519602\du}}
\pgfpathlineto{\pgfpoint{4.979851\du}{13.202010\du}}
\pgfpathlineto{\pgfpoint{8.139342\du}{13.202010\du}}
\pgfpathlineto{\pgfpoint{7.804327\du}{13.519602\du}}
\pgfpathlineto{\pgfpoint{4.644836\du}{13.519602\du}}
\pgfusepath{fill}
\pgfsetbuttcap
\pgfsetmiterjoin
\pgfsetdash{}{0pt}
\definecolor{dialinecolor}{rgb}{0.286275, 0.286275, 0.211765}
\pgfsetstrokecolor{dialinecolor}
\pgfpathmoveto{\pgfpoint{4.644836\du}{13.519602\du}}
\pgfpathlineto{\pgfpoint{4.979851\du}{13.202010\du}}
\pgfpathlineto{\pgfpoint{8.139342\du}{13.202010\du}}
\pgfpathlineto{\pgfpoint{7.804327\du}{13.519602\du}}
\pgfpathlineto{\pgfpoint{4.644836\du}{13.519602\du}}
\pgfusepath{stroke}
\pgfsetlinewidth{0.106000\du}
\pgfsetbuttcap
\pgfsetmiterjoin
\pgfsetdash{}{0pt}
\definecolor{dialinecolor}{rgb}{0.000000, 0.000000, 0.000000}
\pgfsetstrokecolor{dialinecolor}
\pgfpathmoveto{\pgfpoint{7.627611\du}{13.785423\du}}
\pgfpathlineto{\pgfpoint{6.869472\du}{13.785423\du}}
\pgfusepath{stroke}
\pgfsetlinewidth{0.001000\du}
\pgfsetbuttcap
\pgfsetmiterjoin
\pgfsetdash{}{0pt}
\definecolor{dialinecolor}{rgb}{0.478431, 0.478431, 0.352941}
\pgfsetfillcolor{dialinecolor}
\pgfpathmoveto{\pgfpoint{7.804327\du}{14.103513\du}}
\pgfpathlineto{\pgfpoint{8.139342\du}{13.767503\du}}
\pgfpathlineto{\pgfpoint{8.139342\du}{13.202010\du}}
\pgfpathlineto{\pgfpoint{7.804327\du}{13.519602\du}}
\pgfpathlineto{\pgfpoint{7.804327\du}{14.103513\du}}
\pgfusepath{fill}
\pgfsetbuttcap
\pgfsetmiterjoin
\pgfsetdash{}{0pt}
\definecolor{dialinecolor}{rgb}{0.286275, 0.286275, 0.211765}
\pgfsetstrokecolor{dialinecolor}
\pgfpathmoveto{\pgfpoint{7.804327\du}{14.103513\du}}
\pgfpathlineto{\pgfpoint{8.139342\du}{13.767503\du}}
\pgfpathlineto{\pgfpoint{8.139342\du}{13.202010\du}}
\pgfpathlineto{\pgfpoint{7.804327\du}{13.519602\du}}
\pgfpathlineto{\pgfpoint{7.804327\du}{14.103513\du}}
\pgfusepath{stroke}
\pgfsetbuttcap
\pgfsetmiterjoin
\pgfsetdash{}{0pt}
\definecolor{dialinecolor}{rgb}{0.788235, 0.788235, 0.713726}
\pgfsetfillcolor{dialinecolor}
\pgfpathmoveto{\pgfpoint{4.662757\du}{14.473871\du}}
\pgfpathlineto{\pgfpoint{5.015194\du}{14.032328\du}}
\pgfpathlineto{\pgfpoint{7.451392\du}{14.032328\du}}
\pgfpathlineto{\pgfpoint{7.098955\du}{14.473871\du}}
\pgfpathlineto{\pgfpoint{4.662757\du}{14.473871\du}}
\pgfusepath{fill}
\pgfsetbuttcap
\pgfsetmiterjoin
\pgfsetdash{}{0pt}
\definecolor{dialinecolor}{rgb}{0.286275, 0.286275, 0.211765}
\pgfsetstrokecolor{dialinecolor}
\pgfpathmoveto{\pgfpoint{4.662757\du}{14.473871\du}}
\pgfpathlineto{\pgfpoint{5.015194\du}{14.032328\du}}
\pgfpathlineto{\pgfpoint{7.451392\du}{14.032328\du}}
\pgfpathlineto{\pgfpoint{7.098955\du}{14.473871\du}}
\pgfpathlineto{\pgfpoint{4.662757\du}{14.473871\du}}
\pgfusepath{stroke}
\pgfsetbuttcap
\pgfsetmiterjoin
\pgfsetdash{}{0pt}
\definecolor{dialinecolor}{rgb}{0.478431, 0.478431, 0.352941}
\pgfsetfillcolor{dialinecolor}
\pgfpathmoveto{\pgfpoint{7.098955\du}{14.561980\du}}
\pgfpathlineto{\pgfpoint{7.451392\du}{14.191622\du}}
\pgfpathlineto{\pgfpoint{7.451392\du}{14.032328\du}}
\pgfpathlineto{\pgfpoint{7.098955\du}{14.473871\du}}
\pgfpathlineto{\pgfpoint{7.098955\du}{14.561980\du}}
\pgfusepath{fill}
\pgfsetbuttcap
\pgfsetmiterjoin
\pgfsetdash{}{0pt}
\definecolor{dialinecolor}{rgb}{0.286275, 0.286275, 0.211765}
\pgfsetstrokecolor{dialinecolor}
\pgfpathmoveto{\pgfpoint{7.098955\du}{14.561980\du}}
\pgfpathlineto{\pgfpoint{7.451392\du}{14.191622\du}}
\pgfpathlineto{\pgfpoint{7.451392\du}{14.032328\du}}
\pgfpathlineto{\pgfpoint{7.098955\du}{14.473871\du}}
\pgfpathlineto{\pgfpoint{7.098955\du}{14.561980\du}}
\pgfusepath{stroke}
\pgfsetbuttcap
\pgfsetmiterjoin
\pgfsetdash{}{0pt}
\definecolor{dialinecolor}{rgb}{0.717647, 0.717647, 0.615686}
\pgfsetfillcolor{dialinecolor}
\pgfpathmoveto{\pgfpoint{4.662757\du}{14.473871\du}}
\pgfpathlineto{\pgfpoint{7.098955\du}{14.473871\du}}
\pgfpathlineto{\pgfpoint{7.098955\du}{14.561980\du}}
\pgfpathlineto{\pgfpoint{4.662757\du}{14.561980\du}}
\pgfpathlineto{\pgfpoint{4.662757\du}{14.473871\du}}
\pgfusepath{fill}
\pgfsetbuttcap
\pgfsetmiterjoin
\pgfsetdash{}{0pt}
\definecolor{dialinecolor}{rgb}{0.286275, 0.286275, 0.211765}
\pgfsetstrokecolor{dialinecolor}
\pgfpathmoveto{\pgfpoint{4.662757\du}{14.473871\du}}
\pgfpathlineto{\pgfpoint{7.098955\du}{14.473871\du}}
\pgfpathlineto{\pgfpoint{7.098955\du}{14.561980\du}}
\pgfpathlineto{\pgfpoint{4.662757\du}{14.561980\du}}
\pgfpathlineto{\pgfpoint{4.662757\du}{14.473871\du}}
\pgfusepath{stroke}
\pgfsetbuttcap
\pgfsetmiterjoin
\pgfsetdash{}{0pt}
\definecolor{dialinecolor}{rgb}{0.000000, 0.000000, 0.000000}
\pgfsetfillcolor{dialinecolor}
\pgfpathmoveto{\pgfpoint{5.121224\du}{13.449413\du}}
\pgfpathlineto{\pgfpoint{5.386548\du}{13.202010\du}}
\pgfpathlineto{\pgfpoint{7.627611\du}{13.202010\du}}
\pgfpathlineto{\pgfpoint{7.381203\du}{13.449413\du}}
\pgfpathlineto{\pgfpoint{5.121224\du}{13.449413\du}}
\pgfusepath{fill}
\pgfsetbuttcap
\pgfsetmiterjoin
\pgfsetdash{}{0pt}
\definecolor{dialinecolor}{rgb}{0.000000, 0.000000, 0.000000}
\pgfsetstrokecolor{dialinecolor}
\pgfpathmoveto{\pgfpoint{5.121224\du}{13.449413\du}}
\pgfpathlineto{\pgfpoint{5.386548\du}{13.202010\du}}
\pgfpathlineto{\pgfpoint{7.627611\du}{13.202010\du}}
\pgfpathlineto{\pgfpoint{7.381203\du}{13.449413\du}}
\pgfpathlineto{\pgfpoint{5.121224\du}{13.449413\du}}
\pgfusepath{stroke}
\pgfsetbuttcap
\pgfsetmiterjoin
\pgfsetdash{}{0pt}
\definecolor{dialinecolor}{rgb}{0.788235, 0.788235, 0.713726}
\pgfsetfillcolor{dialinecolor}
\pgfpathmoveto{\pgfpoint{5.103303\du}{11.629482\du}}
\pgfpathlineto{\pgfpoint{5.351204\du}{11.400000\du}}
\pgfpathlineto{\pgfpoint{7.593263\du}{11.400000\du}}
\pgfpathlineto{\pgfpoint{7.345362\du}{11.629482\du}}
\pgfpathlineto{\pgfpoint{5.103303\du}{11.629482\du}}
\pgfusepath{fill}
\pgfsetbuttcap
\pgfsetmiterjoin
\pgfsetdash{}{0pt}
\definecolor{dialinecolor}{rgb}{0.286275, 0.286275, 0.211765}
\pgfsetstrokecolor{dialinecolor}
\pgfpathmoveto{\pgfpoint{5.103303\du}{11.629482\du}}
\pgfpathlineto{\pgfpoint{5.351204\du}{11.400000\du}}
\pgfpathlineto{\pgfpoint{7.593263\du}{11.400000\du}}
\pgfpathlineto{\pgfpoint{7.345362\du}{11.629482\du}}
\pgfpathlineto{\pgfpoint{5.103303\du}{11.629482\du}}
\pgfusepath{stroke}
\pgfsetbuttcap
\pgfsetmiterjoin
\pgfsetdash{}{0pt}
\definecolor{dialinecolor}{rgb}{0.717647, 0.717647, 0.615686}
\pgfsetfillcolor{dialinecolor}
\pgfpathmoveto{\pgfpoint{5.103303\du}{11.629482\du}}
\pgfpathlineto{\pgfpoint{7.363283\du}{11.629482\du}}
\pgfpathlineto{\pgfpoint{7.363283\du}{13.413572\du}}
\pgfpathlineto{\pgfpoint{5.103303\du}{13.413572\du}}
\pgfpathlineto{\pgfpoint{5.103303\du}{11.629482\du}}
\pgfusepath{fill}
\pgfsetbuttcap
\pgfsetmiterjoin
\pgfsetdash{}{0pt}
\definecolor{dialinecolor}{rgb}{0.286275, 0.286275, 0.211765}
\pgfsetstrokecolor{dialinecolor}
\pgfpathmoveto{\pgfpoint{5.103303\du}{11.629482\du}}
\pgfpathlineto{\pgfpoint{7.362287\du}{11.629482\du}}
\pgfpathlineto{\pgfpoint{7.362287\du}{13.413074\du}}
\pgfpathlineto{\pgfpoint{5.103303\du}{13.413074\du}}
\pgfpathlineto{\pgfpoint{5.103303\du}{11.629482\du}}
\pgfusepath{stroke}
\pgfsetbuttcap
\pgfsetmiterjoin
\pgfsetdash{}{0pt}
\definecolor{dialinecolor}{rgb}{1.000000, 1.000000, 1.000000}
\pgfsetfillcolor{dialinecolor}
\pgfpathmoveto{\pgfpoint{5.297443\du}{11.858467\du}}
\pgfpathlineto{\pgfpoint{7.168646\du}{11.858467\du}}
\pgfpathlineto{\pgfpoint{7.168646\du}{13.236856\du}}
\pgfpathlineto{\pgfpoint{5.297443\du}{13.236856\du}}
\pgfpathlineto{\pgfpoint{5.297443\du}{11.858467\du}}
\pgfusepath{fill}
\pgfsetbuttcap
\pgfsetmiterjoin
\pgfsetdash{}{0pt}
\definecolor{dialinecolor}{rgb}{0.286275, 0.286275, 0.211765}
\pgfsetstrokecolor{dialinecolor}
\pgfpathmoveto{\pgfpoint{5.297443\du}{11.858467\du}}
\pgfpathlineto{\pgfpoint{7.168646\du}{11.858467\du}}
\pgfpathlineto{\pgfpoint{7.168646\du}{13.236358\du}}
\pgfpathlineto{\pgfpoint{5.297443\du}{13.236358\du}}
\pgfpathlineto{\pgfpoint{5.297443\du}{11.858467\du}}
\pgfusepath{stroke}
\pgfsetbuttcap
\pgfsetmiterjoin
\pgfsetdash{}{0pt}
\definecolor{dialinecolor}{rgb}{0.478431, 0.478431, 0.352941}
\pgfsetfillcolor{dialinecolor}
\pgfpathmoveto{\pgfpoint{7.345362\du}{13.396647\du}}
\pgfpathlineto{\pgfpoint{7.593263\du}{13.149244\du}}
\pgfpathlineto{\pgfpoint{7.593263\du}{11.400000\du}}
\pgfpathlineto{\pgfpoint{7.345362\du}{11.629482\du}}
\pgfpathlineto{\pgfpoint{7.345362\du}{13.396647\du}}
\pgfusepath{fill}
\pgfsetbuttcap
\pgfsetmiterjoin
\pgfsetdash{}{0pt}
\definecolor{dialinecolor}{rgb}{0.286275, 0.286275, 0.211765}
\pgfsetstrokecolor{dialinecolor}
\pgfpathmoveto{\pgfpoint{7.345362\du}{13.396647\du}}
\pgfpathlineto{\pgfpoint{7.593263\du}{13.149244\du}}
\pgfpathlineto{\pgfpoint{7.593263\du}{11.400000\du}}
\pgfpathlineto{\pgfpoint{7.345362\du}{11.629482\du}}
\pgfpathlineto{\pgfpoint{7.345362\du}{13.396647\du}}
\pgfusepath{stroke}
\pgfsetlinewidth{0.050000\du}
\pgfsetdash{}{0pt}
\pgfsetdash{}{0pt}
\pgfsetbuttcap
{
\definecolor{dialinecolor}{rgb}{0.000000, 0.000000, 0.000000}
\pgfsetfillcolor{dialinecolor}
% was here!!!
\pgfsetarrowsend{to}
\definecolor{dialinecolor}{rgb}{0.000000, 0.000000, 0.000000}
\pgfsetstrokecolor{dialinecolor}
\draw (7.950000\du,11.950000\du)--(18.500000\du,11.950000\du);
}
% setfont left to latex
\definecolor{dialinecolor}{rgb}{0.000000, 0.000000, 0.000000}
\pgfsetstrokecolor{dialinecolor}
\node[anchor=west] at (5.200000\du,10.500000\du){Client};
% setfont left to latex
\definecolor{dialinecolor}{rgb}{0.000000, 0.000000, 0.000000}
\pgfsetstrokecolor{dialinecolor}
\node[anchor=west] at (19.350000\du,9.250000\du){Serveur};
\pgfsetlinewidth{0.100000\du}
\pgfsetdash{}{0pt}
\pgfsetdash{}{0pt}
\pgfsetbuttcap
\pgfsetmiterjoin
\pgfsetlinewidth{0.001000\du}
\pgfsetbuttcap
\pgfsetmiterjoin
\pgfsetdash{}{0pt}
\definecolor{dialinecolor}{rgb}{0.717647, 0.717647, 0.615686}
\pgfsetfillcolor{dialinecolor}
\pgfpathmoveto{\pgfpoint{19.528989\du}{10.939463\du}}
\pgfpathlineto{\pgfpoint{19.528989\du}{13.916166\du}}
\pgfpathlineto{\pgfpoint{21.288380\du}{13.916166\du}}
\pgfpathlineto{\pgfpoint{21.288380\du}{10.939463\du}}
\pgfpathlineto{\pgfpoint{19.528989\du}{10.939463\du}}
\pgfusepath{fill}
\pgfsetbuttcap
\pgfsetmiterjoin
\pgfsetdash{}{0pt}
\definecolor{dialinecolor}{rgb}{0.286275, 0.286275, 0.211765}
\pgfsetstrokecolor{dialinecolor}
\pgfpathmoveto{\pgfpoint{19.528989\du}{10.939463\du}}
\pgfpathlineto{\pgfpoint{19.528989\du}{13.916166\du}}
\pgfpathlineto{\pgfpoint{21.288380\du}{13.916166\du}}
\pgfpathlineto{\pgfpoint{21.288380\du}{10.939463\du}}
\pgfpathlineto{\pgfpoint{19.528989\du}{10.939463\du}}
\pgfusepath{stroke}
\pgfsetbuttcap
\pgfsetmiterjoin
\pgfsetdash{}{0pt}
\definecolor{dialinecolor}{rgb}{0.788235, 0.788235, 0.713726}
\pgfsetfillcolor{dialinecolor}
\pgfpathmoveto{\pgfpoint{19.528989\du}{10.939463\du}}
\pgfpathlineto{\pgfpoint{19.767399\du}{10.700000\du}}
\pgfpathlineto{\pgfpoint{21.526264\du}{10.700000\du}}
\pgfpathlineto{\pgfpoint{21.288380\du}{10.939463\du}}
\pgfpathlineto{\pgfpoint{19.528989\du}{10.939463\du}}
\pgfusepath{fill}
\pgfsetbuttcap
\pgfsetmiterjoin
\pgfsetdash{}{0pt}
\definecolor{dialinecolor}{rgb}{0.286275, 0.286275, 0.211765}
\pgfsetstrokecolor{dialinecolor}
\pgfpathmoveto{\pgfpoint{19.528989\du}{10.939463\du}}
\pgfpathlineto{\pgfpoint{19.767399\du}{10.700000\du}}
\pgfpathlineto{\pgfpoint{21.514686\du}{10.700000\du}}
\pgfusepath{stroke}
\pgfsetbuttcap
\pgfsetmiterjoin
\pgfsetdash{}{0pt}
\definecolor{dialinecolor}{rgb}{0.286275, 0.286275, 0.211765}
\pgfsetstrokecolor{dialinecolor}
\pgfpathmoveto{\pgfpoint{21.514686\du}{10.712105\du}}
\pgfpathlineto{\pgfpoint{21.288380\du}{10.939463\du}}
\pgfpathlineto{\pgfpoint{19.528989\du}{10.939463\du}}
\pgfusepath{stroke}
\pgfsetbuttcap
\pgfsetmiterjoin
\pgfsetdash{}{0pt}
\definecolor{dialinecolor}{rgb}{0.788235, 0.788235, 0.713726}
\pgfsetfillcolor{dialinecolor}
\pgfpathmoveto{\pgfpoint{19.637405\du}{11.112612\du}}
\pgfpathlineto{\pgfpoint{20.441051\du}{11.112612\du}}
\pgfpathlineto{\pgfpoint{20.441051\du}{11.503120\du}}
\pgfpathlineto{\pgfpoint{19.637405\du}{11.503120\du}}
\pgfpathlineto{\pgfpoint{19.637405\du}{11.112612\du}}
\pgfusepath{fill}
\pgfsetbuttcap
\pgfsetmiterjoin
\pgfsetdash{}{0pt}
\definecolor{dialinecolor}{rgb}{0.384314, 0.384314, 0.282353}
\pgfsetstrokecolor{dialinecolor}
\pgfpathmoveto{\pgfpoint{19.637405\du}{11.112612\du}}
\pgfpathlineto{\pgfpoint{20.440525\du}{11.112612\du}}
\pgfpathlineto{\pgfpoint{20.440525\du}{11.502594\du}}
\pgfpathlineto{\pgfpoint{19.637405\du}{11.502594\du}}
\pgfpathlineto{\pgfpoint{19.637405\du}{11.112612\du}}
\pgfusepath{stroke}
\pgfsetlinewidth{0.030000\du}
\pgfsetbuttcap
\pgfsetmiterjoin
\pgfsetdash{}{0pt}
\definecolor{dialinecolor}{rgb}{0.925490, 0.925490, 0.905882}
\pgfsetstrokecolor{dialinecolor}
\pgfpathmoveto{\pgfpoint{19.745821\du}{11.308393\du}}
\pgfpathlineto{\pgfpoint{20.309479\du}{11.308393\du}}
\pgfusepath{stroke}
\pgfsetlinewidth{0.001000\du}
\pgfsetbuttcap
\pgfsetmiterjoin
\pgfsetdash{}{0pt}
\definecolor{dialinecolor}{rgb}{0.478431, 0.478431, 0.352941}
\pgfsetfillcolor{dialinecolor}
\pgfpathmoveto{\pgfpoint{21.288380\du}{13.916166\du}}
\pgfpathlineto{\pgfpoint{21.526264\du}{13.676177\du}}
\pgfpathlineto{\pgfpoint{21.526264\du}{10.700000\du}}
\pgfpathlineto{\pgfpoint{21.288380\du}{10.939463\du}}
\pgfpathlineto{\pgfpoint{21.288380\du}{13.916166\du}}
\pgfusepath{fill}
\pgfsetbuttcap
\pgfsetmiterjoin
\pgfsetdash{}{0pt}
\definecolor{dialinecolor}{rgb}{0.286275, 0.286275, 0.211765}
\pgfsetstrokecolor{dialinecolor}
\pgfpathmoveto{\pgfpoint{21.288380\du}{13.916166\du}}
\pgfpathlineto{\pgfpoint{21.514686\du}{13.688282\du}}
\pgfusepath{stroke}
\pgfsetbuttcap
\pgfsetmiterjoin
\pgfsetdash{}{0pt}
\definecolor{dialinecolor}{rgb}{0.286275, 0.286275, 0.211765}
\pgfsetstrokecolor{dialinecolor}
\pgfpathmoveto{\pgfpoint{21.514686\du}{10.712105\du}}
\pgfpathlineto{\pgfpoint{21.288380\du}{10.939463\du}}
\pgfpathlineto{\pgfpoint{21.288380\du}{13.916166\du}}
\pgfusepath{stroke}
\pgfsetlinewidth{0.030000\du}
\pgfsetbuttcap
\pgfsetmiterjoin
\pgfsetdash{}{0pt}
\definecolor{dialinecolor}{rgb}{0.925490, 0.925490, 0.905882}
\pgfsetstrokecolor{dialinecolor}
\pgfpathmoveto{\pgfpoint{19.551093\du}{13.719859\du}}
\pgfpathlineto{\pgfpoint{21.287854\du}{13.719859\du}}
\pgfusepath{stroke}
\pgfsetbuttcap
\pgfsetmiterjoin
\pgfsetdash{}{0pt}
\definecolor{dialinecolor}{rgb}{0.000000, 0.000000, 0.000000}
\pgfsetstrokecolor{dialinecolor}
\pgfpathmoveto{\pgfpoint{19.551093\du}{12.134144\du}}
\pgfpathlineto{\pgfpoint{21.287854\du}{12.134144\du}}
\pgfusepath{stroke}
\pgfsetbuttcap
\pgfsetmiterjoin
\pgfsetdash{}{0pt}
\definecolor{dialinecolor}{rgb}{0.286275, 0.286275, 0.211765}
\pgfsetstrokecolor{dialinecolor}
\pgfpathmoveto{\pgfpoint{19.528989\du}{13.698281\du}}
\pgfpathlineto{\pgfpoint{21.286275\du}{13.698281\du}}
\pgfusepath{stroke}
\pgfsetbuttcap
\pgfsetmiterjoin
\pgfsetdash{}{0pt}
\definecolor{dialinecolor}{rgb}{0.000000, 0.000000, 0.000000}
\pgfsetstrokecolor{dialinecolor}
\pgfpathmoveto{\pgfpoint{19.528989\du}{12.112040\du}}
\pgfpathlineto{\pgfpoint{21.286275\du}{12.112040\du}}
\pgfusepath{stroke}
\pgfsetlinewidth{0.001000\du}
\pgfsetbuttcap
\pgfsetmiterjoin
\pgfsetdash{}{0pt}
\definecolor{dialinecolor}{rgb}{0.925490, 0.925490, 0.905882}
\pgfsetstrokecolor{dialinecolor}
\pgfpathmoveto{\pgfpoint{19.637405\du}{11.482069\du}}
\pgfpathlineto{\pgfpoint{19.637405\du}{11.112612\du}}
\pgfpathlineto{\pgfpoint{20.418947\du}{11.112612\du}}
\pgfusepath{stroke}
\pgfsetlinewidth{0.050000\du}
\pgfsetdash{}{0pt}
\pgfsetdash{}{0pt}
\pgfsetmiterjoin
\pgfsetbuttcap
{
\definecolor{dialinecolor}{rgb}{0.000000, 0.000000, 0.000000}
\pgfsetfillcolor{dialinecolor}
% was here!!!
\pgfsetarrowsend{to}
{\pgfsetcornersarced{\pgfpoint{0.000000\du}{0.000000\du}}\definecolor{dialinecolor}{rgb}{0.000000, 0.000000, 0.000000}
\pgfsetstrokecolor{dialinecolor}
\draw (19.578989\du,12.427814\du)--(19.600000\du,12.427814\du)--(19.600000\du,20.300000\du)--(16.100000\du,20.300000\du);
}}
\pgfsetlinewidth{0.100000\du}
\pgfsetdash{}{0pt}
\pgfsetdash{}{0pt}
\pgfsetbuttcap
\pgfsetmiterjoin
\pgfsetlinewidth{0.001000\du}
\pgfsetbuttcap
\pgfsetmiterjoin
\pgfsetdash{}{0pt}
\definecolor{dialinecolor}{rgb}{0.647059, 0.647059, 0.521569}
\pgfsetfillcolor{dialinecolor}
\pgfpathmoveto{\pgfpoint{26.924626\du}{12.362935\du}}
\pgfpathlineto{\pgfpoint{26.922680\du}{12.338037\du}}
\pgfpathlineto{\pgfpoint{26.917234\du}{12.313528\du}}
\pgfpathlineto{\pgfpoint{26.907897\du}{12.289020\du}}
\pgfpathlineto{\pgfpoint{26.895059\du}{12.264900\du}}
\pgfpathlineto{\pgfpoint{26.879109\du}{12.240780\du}}
\pgfpathlineto{\pgfpoint{26.859269\du}{12.217049\du}}
\pgfpathlineto{\pgfpoint{26.835927\du}{12.193707\du}}
\pgfpathlineto{\pgfpoint{26.809084\du}{12.170755\du}}
\pgfpathlineto{\pgfpoint{26.778740\du}{12.148191\du}}
\pgfpathlineto{\pgfpoint{26.745283\du}{12.126794\du}}
\pgfpathlineto{\pgfpoint{26.708714\du}{12.105398\du}}
\pgfpathlineto{\pgfpoint{26.669422\du}{12.084390\du}}
\pgfpathlineto{\pgfpoint{26.626240\du}{12.064550\du}}
\pgfpathlineto{\pgfpoint{26.580724\du}{12.045487\du}}
\pgfpathlineto{\pgfpoint{26.532484\du}{12.027203\du}}
\pgfpathlineto{\pgfpoint{26.481132\du}{12.009697\du}}
\pgfpathlineto{\pgfpoint{26.427835\du}{11.993357\du}}
\pgfpathlineto{\pgfpoint{26.371815\du}{11.977796\du}}
\pgfpathlineto{\pgfpoint{26.313849\du}{11.963402\du}}
\pgfpathlineto{\pgfpoint{26.253161\du}{11.949786\du}}
\pgfpathlineto{\pgfpoint{26.191305\du}{11.936948\du}}
\pgfpathlineto{\pgfpoint{26.127115\du}{11.925666\du}}
\pgfpathlineto{\pgfpoint{26.061369\du}{11.914773\du}}
\pgfpathlineto{\pgfpoint{25.994845\du}{11.905826\du}}
\pgfpathlineto{\pgfpoint{25.926376\du}{11.898045\du}}
\pgfpathlineto{\pgfpoint{25.856740\du}{11.891432\du}}
\pgfpathlineto{\pgfpoint{25.786715\du}{11.885596\du}}
\pgfpathlineto{\pgfpoint{25.715522\du}{11.881706\du}}
\pgfpathlineto{\pgfpoint{25.643941\du}{11.878205\du}}
\pgfpathlineto{\pgfpoint{25.571970\du}{11.875870\du}}
\pgfpathlineto{\pgfpoint{25.500000\du}{11.875870\du}}
\pgfpathlineto{\pgfpoint{25.500000\du}{11.875870\du}}
\pgfpathlineto{\pgfpoint{25.427641\du}{11.875870\du}}
\pgfpathlineto{\pgfpoint{25.355670\du}{11.878205\du}}
\pgfpathlineto{\pgfpoint{25.284089\du}{11.881706\du}}
\pgfpathlineto{\pgfpoint{25.212896\du}{11.885596\du}}
\pgfpathlineto{\pgfpoint{25.142871\du}{11.891432\du}}
\pgfpathlineto{\pgfpoint{25.073235\du}{11.898045\du}}
\pgfpathlineto{\pgfpoint{25.005155\du}{11.905826\du}}
\pgfpathlineto{\pgfpoint{24.938242\du}{11.914773\du}}
\pgfpathlineto{\pgfpoint{24.872496\du}{11.925666\du}}
\pgfpathlineto{\pgfpoint{24.808306\du}{11.936948\du}}
\pgfpathlineto{\pgfpoint{24.746450\du}{11.949786\du}}
\pgfpathlineto{\pgfpoint{24.686151\du}{11.963402\du}}
\pgfpathlineto{\pgfpoint{24.628185\du}{11.977796\du}}
\pgfpathlineto{\pgfpoint{24.572165\du}{11.993357\du}}
\pgfpathlineto{\pgfpoint{24.518479\du}{12.009697\du}}
\pgfpathlineto{\pgfpoint{24.467127\du}{12.027203\du}}
\pgfpathlineto{\pgfpoint{24.419276\du}{12.045487\du}}
\pgfpathlineto{\pgfpoint{24.373371\du}{12.064550\du}}
\pgfpathlineto{\pgfpoint{24.330578\du}{12.084390\du}}
\pgfpathlineto{\pgfpoint{24.290897\du}{12.105398\du}}
\pgfpathlineto{\pgfpoint{24.254328\du}{12.126794\du}}
\pgfpathlineto{\pgfpoint{24.220871\du}{12.148191\du}}
\pgfpathlineto{\pgfpoint{24.190527\du}{12.170755\du}}
\pgfpathlineto{\pgfpoint{24.163684\du}{12.193707\du}}
\pgfpathlineto{\pgfpoint{24.140342\du}{12.217049\du}}
\pgfpathlineto{\pgfpoint{24.120502\du}{12.240780\du}}
\pgfpathlineto{\pgfpoint{24.104552\du}{12.264900\du}}
\pgfpathlineto{\pgfpoint{24.091714\du}{12.289020\du}}
\pgfpathlineto{\pgfpoint{24.082377\du}{12.313528\du}}
\pgfpathlineto{\pgfpoint{24.077320\du}{12.338037\du}}
\pgfpathlineto{\pgfpoint{24.075374\du}{12.362935\du}}
\pgfpathlineto{\pgfpoint{24.075374\du}{12.362935\du}}
\pgfpathlineto{\pgfpoint{24.077320\du}{12.387833\du}}
\pgfpathlineto{\pgfpoint{24.082377\du}{12.411953\du}}
\pgfpathlineto{\pgfpoint{24.091714\du}{12.436851\du}}
\pgfpathlineto{\pgfpoint{24.104552\du}{12.460971\du}}
\pgfpathlineto{\pgfpoint{24.120502\du}{12.485090\du}}
\pgfpathlineto{\pgfpoint{24.140342\du}{12.508821\du}}
\pgfpathlineto{\pgfpoint{24.163684\du}{12.532163\du}}
\pgfpathlineto{\pgfpoint{24.190527\du}{12.555116\du}}
\pgfpathlineto{\pgfpoint{24.220871\du}{12.577290\du}}
\pgfpathlineto{\pgfpoint{24.254328\du}{12.599076\du}}
\pgfpathlineto{\pgfpoint{24.290897\du}{12.620473\du}}
\pgfpathlineto{\pgfpoint{24.330578\du}{12.641480\du}}
\pgfpathlineto{\pgfpoint{24.373371\du}{12.661321\du}}
\pgfpathlineto{\pgfpoint{24.419276\du}{12.680383\du}}
\pgfpathlineto{\pgfpoint{24.467127\du}{12.698668\du}}
\pgfpathlineto{\pgfpoint{24.518479\du}{12.716174\du}}
\pgfpathlineto{\pgfpoint{24.572165\du}{12.732513\du}}
\pgfpathlineto{\pgfpoint{24.628185\du}{12.748074\du}}
\pgfpathlineto{\pgfpoint{24.686151\du}{12.762468\du}}
\pgfpathlineto{\pgfpoint{24.746450\du}{12.776084\du}}
\pgfpathlineto{\pgfpoint{24.808306\du}{12.788922\du}}
\pgfpathlineto{\pgfpoint{24.872496\du}{12.800204\du}}
\pgfpathlineto{\pgfpoint{24.938242\du}{12.810708\du}}
\pgfpathlineto{\pgfpoint{25.005155\du}{12.820045\du}}
\pgfpathlineto{\pgfpoint{25.073235\du}{12.827825\du}}
\pgfpathlineto{\pgfpoint{25.142871\du}{12.834439\du}}
\pgfpathlineto{\pgfpoint{25.212896\du}{12.839885\du}}
\pgfpathlineto{\pgfpoint{25.284089\du}{12.844165\du}}
\pgfpathlineto{\pgfpoint{25.355670\du}{12.847277\du}}
\pgfpathlineto{\pgfpoint{25.427641\du}{12.849611\du}}
\pgfpathlineto{\pgfpoint{25.500000\du}{12.850000\du}}
\pgfpathlineto{\pgfpoint{25.500000\du}{12.850000\du}}
\pgfpathlineto{\pgfpoint{25.571970\du}{12.849611\du}}
\pgfpathlineto{\pgfpoint{25.643941\du}{12.847277\du}}
\pgfpathlineto{\pgfpoint{25.715522\du}{12.844165\du}}
\pgfpathlineto{\pgfpoint{25.786715\du}{12.839885\du}}
\pgfpathlineto{\pgfpoint{25.856740\du}{12.834439\du}}
\pgfpathlineto{\pgfpoint{25.926376\du}{12.827825\du}}
\pgfpathlineto{\pgfpoint{25.994845\du}{12.820045\du}}
\pgfpathlineto{\pgfpoint{26.061369\du}{12.810708\du}}
\pgfpathlineto{\pgfpoint{26.127115\du}{12.800204\du}}
\pgfpathlineto{\pgfpoint{26.191305\du}{12.788922\du}}
\pgfpathlineto{\pgfpoint{26.253161\du}{12.776084\du}}
\pgfpathlineto{\pgfpoint{26.313849\du}{12.762468\du}}
\pgfpathlineto{\pgfpoint{26.371815\du}{12.748074\du}}
\pgfpathlineto{\pgfpoint{26.427835\du}{12.732513\du}}
\pgfpathlineto{\pgfpoint{26.481132\du}{12.716174\du}}
\pgfpathlineto{\pgfpoint{26.532484\du}{12.698668\du}}
\pgfpathlineto{\pgfpoint{26.580724\du}{12.680383\du}}
\pgfpathlineto{\pgfpoint{26.626240\du}{12.661321\du}}
\pgfpathlineto{\pgfpoint{26.669422\du}{12.641480\du}}
\pgfpathlineto{\pgfpoint{26.708714\du}{12.620473\du}}
\pgfpathlineto{\pgfpoint{26.745283\du}{12.599076\du}}
\pgfpathlineto{\pgfpoint{26.778740\du}{12.577290\du}}
\pgfpathlineto{\pgfpoint{26.809084\du}{12.555116\du}}
\pgfpathlineto{\pgfpoint{26.835927\du}{12.532163\du}}
\pgfpathlineto{\pgfpoint{26.859269\du}{12.508821\du}}
\pgfpathlineto{\pgfpoint{26.879109\du}{12.485090\du}}
\pgfpathlineto{\pgfpoint{26.895059\du}{12.460971\du}}
\pgfpathlineto{\pgfpoint{26.907897\du}{12.436851\du}}
\pgfpathlineto{\pgfpoint{26.917234\du}{12.411953\du}}
\pgfpathlineto{\pgfpoint{26.922680\du}{12.387833\du}}
\pgfpathlineto{\pgfpoint{26.924626\du}{12.362935\du}}
\pgfusepath{fill}
\pgfsetbuttcap
\pgfsetmiterjoin
\pgfsetdash{}{0pt}
\definecolor{dialinecolor}{rgb}{0.286275, 0.286275, 0.211765}
\pgfsetstrokecolor{dialinecolor}
\pgfpathmoveto{\pgfpoint{26.907897\du}{12.355155\du}}
\pgfpathlineto{\pgfpoint{26.905952\du}{12.330257\du}}
\pgfpathlineto{\pgfpoint{26.900895\du}{12.306526\du}}
\pgfpathlineto{\pgfpoint{26.891947\du}{12.282406\du}}
\pgfpathlineto{\pgfpoint{26.879498\du}{12.258675\du}}
\pgfpathlineto{\pgfpoint{26.862770\du}{12.234556\du}}
\pgfpathlineto{\pgfpoint{26.842540\du}{12.211603\du}}
\pgfpathlineto{\pgfpoint{26.820366\du}{12.188261\du}}
\pgfpathlineto{\pgfpoint{26.793134\du}{12.166475\du}}
\pgfpathlineto{\pgfpoint{26.763178\du}{12.144301\du}}
\pgfpathlineto{\pgfpoint{26.730111\du}{12.122515\du}}
\pgfpathlineto{\pgfpoint{26.693542\du}{12.101897\du}}
\pgfpathlineto{\pgfpoint{26.654250\du}{12.081278\du}}
\pgfpathlineto{\pgfpoint{26.611846\du}{12.061826\du}}
\pgfpathlineto{\pgfpoint{26.567108\du}{12.042375\du}}
\pgfpathlineto{\pgfpoint{26.518090\du}{12.024869\du}}
\pgfpathlineto{\pgfpoint{26.467127\du}{12.007751\du}}
\pgfpathlineto{\pgfpoint{26.413830\du}{11.991412\du}}
\pgfpathlineto{\pgfpoint{26.358588\du}{11.976240\du}}
\pgfpathlineto{\pgfpoint{26.301011\du}{11.962235\du}}
\pgfpathlineto{\pgfpoint{26.241101\du}{11.948619\du}}
\pgfpathlineto{\pgfpoint{26.178856\du}{11.936559\du}}
\pgfpathlineto{\pgfpoint{26.115444\du}{11.925277\du}}
\pgfpathlineto{\pgfpoint{26.050088\du}{11.914773\du}}
\pgfpathlineto{\pgfpoint{25.983953\du}{11.905826\du}}
\pgfpathlineto{\pgfpoint{25.916261\du}{11.898045\du}}
\pgfpathlineto{\pgfpoint{25.846625\du}{11.891821\du}}
\pgfpathlineto{\pgfpoint{25.776989\du}{11.885596\du}}
\pgfpathlineto{\pgfpoint{25.706186\du}{11.882095\du}}
\pgfpathlineto{\pgfpoint{25.635382\du}{11.878983\du}}
\pgfpathlineto{\pgfpoint{25.563023\du}{11.877038\du}}
\pgfpathlineto{\pgfpoint{25.491441\du}{11.875870\du}}
\pgfpathlineto{\pgfpoint{25.491441\du}{11.875870\du}}
\pgfpathlineto{\pgfpoint{25.419860\du}{11.877038\du}}
\pgfpathlineto{\pgfpoint{25.348279\du}{11.878983\du}}
\pgfpathlineto{\pgfpoint{25.276697\du}{11.882095\du}}
\pgfpathlineto{\pgfpoint{25.206672\du}{11.885596\du}}
\pgfpathlineto{\pgfpoint{25.136647\du}{11.891821\du}}
\pgfpathlineto{\pgfpoint{25.067399\du}{11.898045\du}}
\pgfpathlineto{\pgfpoint{24.999319\du}{11.905826\du}}
\pgfpathlineto{\pgfpoint{24.932795\du}{11.914773\du}}
\pgfpathlineto{\pgfpoint{24.867827\du}{11.925277\du}}
\pgfpathlineto{\pgfpoint{24.804415\du}{11.936559\du}}
\pgfpathlineto{\pgfpoint{24.742171\du}{11.948619\du}}
\pgfpathlineto{\pgfpoint{24.682260\du}{11.962235\du}}
\pgfpathlineto{\pgfpoint{24.625073\du}{11.976240\du}}
\pgfpathlineto{\pgfpoint{24.569053\du}{11.991412\du}}
\pgfpathlineto{\pgfpoint{24.515367\du}{12.007751\du}}
\pgfpathlineto{\pgfpoint{24.464793\du}{12.024869\du}}
\pgfpathlineto{\pgfpoint{24.416553\du}{12.042375\du}}
\pgfpathlineto{\pgfpoint{24.371815\du}{12.061826\du}}
\pgfpathlineto{\pgfpoint{24.329022\du}{12.081278\du}}
\pgfpathlineto{\pgfpoint{24.289341\du}{12.101897\du}}
\pgfpathlineto{\pgfpoint{24.253550\du}{12.122515\du}}
\pgfpathlineto{\pgfpoint{24.220482\du}{12.144301\du}}
\pgfpathlineto{\pgfpoint{24.189749\du}{12.166475\du}}
\pgfpathlineto{\pgfpoint{24.163295\du}{12.188261\du}}
\pgfpathlineto{\pgfpoint{24.140342\du}{12.211603\du}}
\pgfpathlineto{\pgfpoint{24.120502\du}{12.234556\du}}
\pgfpathlineto{\pgfpoint{24.104163\du}{12.258675\du}}
\pgfpathlineto{\pgfpoint{24.091325\du}{12.282406\du}}
\pgfpathlineto{\pgfpoint{24.082377\du}{12.306526\du}}
\pgfpathlineto{\pgfpoint{24.077320\du}{12.330257\du}}
\pgfpathlineto{\pgfpoint{24.075374\du}{12.355155\du}}
\pgfpathlineto{\pgfpoint{24.075374\du}{12.355155\du}}
\pgfpathlineto{\pgfpoint{24.077320\du}{12.379274\du}}
\pgfpathlineto{\pgfpoint{24.082377\du}{12.403005\du}}
\pgfpathlineto{\pgfpoint{24.091325\du}{12.427514\du}}
\pgfpathlineto{\pgfpoint{24.104163\du}{12.451245\du}}
\pgfpathlineto{\pgfpoint{24.120502\du}{12.474976\du}}
\pgfpathlineto{\pgfpoint{24.140342\du}{12.498317\du}}
\pgfpathlineto{\pgfpoint{24.163295\du}{12.521270\du}}
\pgfpathlineto{\pgfpoint{24.189749\du}{12.543445\du}}
\pgfpathlineto{\pgfpoint{24.220482\du}{12.565230\du}}
\pgfpathlineto{\pgfpoint{24.253550\du}{12.587405\du}}
\pgfpathlineto{\pgfpoint{24.289341\du}{12.608413\du}}
\pgfpathlineto{\pgfpoint{24.329022\du}{12.628253\du}}
\pgfpathlineto{\pgfpoint{24.371815\du}{12.648094\du}}
\pgfpathlineto{\pgfpoint{24.416553\du}{12.667156\du}}
\pgfpathlineto{\pgfpoint{24.464793\du}{12.685052\du}}
\pgfpathlineto{\pgfpoint{24.515367\du}{12.702169\du}}
\pgfpathlineto{\pgfpoint{24.569053\du}{12.718119\du}}
\pgfpathlineto{\pgfpoint{24.625073\du}{12.733680\du}}
\pgfpathlineto{\pgfpoint{24.682260\du}{12.747685\du}}
\pgfpathlineto{\pgfpoint{24.742171\du}{12.761301\du}}
\pgfpathlineto{\pgfpoint{24.804415\du}{12.773750\du}}
\pgfpathlineto{\pgfpoint{24.867827\du}{12.784643\du}}
\pgfpathlineto{\pgfpoint{24.932795\du}{12.794758\du}}
\pgfpathlineto{\pgfpoint{24.999319\du}{12.803706\du}}
\pgfpathlineto{\pgfpoint{25.067399\du}{12.811486\du}}
\pgfpathlineto{\pgfpoint{25.136647\du}{12.818100\du}}
\pgfpathlineto{\pgfpoint{25.206672\du}{12.823935\du}}
\pgfpathlineto{\pgfpoint{25.276697\du}{12.827825\du}}
\pgfpathlineto{\pgfpoint{25.348279\du}{12.830938\du}}
\pgfpathlineto{\pgfpoint{25.419860\du}{12.832883\du}}
\pgfpathlineto{\pgfpoint{25.491441\du}{12.833661\du}}
\pgfpathlineto{\pgfpoint{25.491441\du}{12.833661\du}}
\pgfpathlineto{\pgfpoint{25.563023\du}{12.832883\du}}
\pgfpathlineto{\pgfpoint{25.635382\du}{12.830938\du}}
\pgfpathlineto{\pgfpoint{25.706186\du}{12.827825\du}}
\pgfpathlineto{\pgfpoint{25.776989\du}{12.823935\du}}
\pgfpathlineto{\pgfpoint{25.846625\du}{12.818100\du}}
\pgfpathlineto{\pgfpoint{25.916261\du}{12.811486\du}}
\pgfpathlineto{\pgfpoint{25.983953\du}{12.803706\du}}
\pgfpathlineto{\pgfpoint{26.050088\du}{12.794758\du}}
\pgfpathlineto{\pgfpoint{26.115444\du}{12.784643\du}}
\pgfpathlineto{\pgfpoint{26.178856\du}{12.773750\du}}
\pgfpathlineto{\pgfpoint{26.241101\du}{12.761301\du}}
\pgfpathlineto{\pgfpoint{26.301011\du}{12.747685\du}}
\pgfpathlineto{\pgfpoint{26.358588\du}{12.733680\du}}
\pgfpathlineto{\pgfpoint{26.413830\du}{12.718119\du}}
\pgfpathlineto{\pgfpoint{26.467127\du}{12.702169\du}}
\pgfpathlineto{\pgfpoint{26.518090\du}{12.685052\du}}
\pgfpathlineto{\pgfpoint{26.567108\du}{12.667156\du}}
\pgfpathlineto{\pgfpoint{26.611846\du}{12.648094\du}}
\pgfpathlineto{\pgfpoint{26.654250\du}{12.628253\du}}
\pgfpathlineto{\pgfpoint{26.693542\du}{12.608413\du}}
\pgfpathlineto{\pgfpoint{26.730111\du}{12.587405\du}}
\pgfpathlineto{\pgfpoint{26.763178\du}{12.565230\du}}
\pgfpathlineto{\pgfpoint{26.793134\du}{12.543445\du}}
\pgfpathlineto{\pgfpoint{26.820366\du}{12.521270\du}}
\pgfpathlineto{\pgfpoint{26.842540\du}{12.498317\du}}
\pgfpathlineto{\pgfpoint{26.862770\du}{12.474976\du}}
\pgfpathlineto{\pgfpoint{26.879498\du}{12.451245\du}}
\pgfpathlineto{\pgfpoint{26.891947\du}{12.427514\du}}
\pgfpathlineto{\pgfpoint{26.900895\du}{12.403005\du}}
\pgfpathlineto{\pgfpoint{26.905952\du}{12.379274\du}}
\pgfpathlineto{\pgfpoint{26.907897\du}{12.355155\du}}
\pgfusepath{stroke}
\pgfsetbuttcap
\pgfsetmiterjoin
\pgfsetdash{}{0pt}
\definecolor{dialinecolor}{rgb}{0.647059, 0.647059, 0.521569}
\pgfsetfillcolor{dialinecolor}
\pgfpathmoveto{\pgfpoint{24.075374\du}{11.345623\du}}
\pgfpathlineto{\pgfpoint{24.075374\du}{12.371494\du}}
\pgfpathlineto{\pgfpoint{26.907897\du}{12.371494\du}}
\pgfpathlineto{\pgfpoint{26.907897\du}{11.345623\du}}
\pgfpathlineto{\pgfpoint{24.075374\du}{11.345623\du}}
\pgfusepath{fill}
\pgfsetbuttcap
\pgfsetmiterjoin
\pgfsetdash{}{0pt}
\definecolor{dialinecolor}{rgb}{0.788235, 0.788235, 0.713726}
\pgfsetfillcolor{dialinecolor}
\pgfpathmoveto{\pgfpoint{26.924626\du}{11.337065\du}}
\pgfpathlineto{\pgfpoint{26.922680\du}{11.311778\du}}
\pgfpathlineto{\pgfpoint{26.917234\du}{11.287269\du}}
\pgfpathlineto{\pgfpoint{26.907897\du}{11.263149\du}}
\pgfpathlineto{\pgfpoint{26.895059\du}{11.239029\du}}
\pgfpathlineto{\pgfpoint{26.879109\du}{11.214521\du}}
\pgfpathlineto{\pgfpoint{26.859269\du}{11.190790\du}}
\pgfpathlineto{\pgfpoint{26.835927\du}{11.167059\du}}
\pgfpathlineto{\pgfpoint{26.809084\du}{11.144884\du}}
\pgfpathlineto{\pgfpoint{26.778740\du}{11.122710\du}}
\pgfpathlineto{\pgfpoint{26.745283\du}{11.100146\du}}
\pgfpathlineto{\pgfpoint{26.708714\du}{11.079527\du}}
\pgfpathlineto{\pgfpoint{26.669422\du}{11.058520\du}}
\pgfpathlineto{\pgfpoint{26.626240\du}{11.038290\du}}
\pgfpathlineto{\pgfpoint{26.580724\du}{11.019617\du}}
\pgfpathlineto{\pgfpoint{26.532484\du}{11.001332\du}}
\pgfpathlineto{\pgfpoint{26.481132\du}{10.983437\du}}
\pgfpathlineto{\pgfpoint{26.427835\du}{10.967098\du}}
\pgfpathlineto{\pgfpoint{26.371815\du}{10.951926\du}}
\pgfpathlineto{\pgfpoint{26.313849\du}{10.937143\du}}
\pgfpathlineto{\pgfpoint{26.253161\du}{10.923527\du}}
\pgfpathlineto{\pgfpoint{26.191305\du}{10.911078\du}}
\pgfpathlineto{\pgfpoint{26.127115\du}{10.899796\du}}
\pgfpathlineto{\pgfpoint{26.061369\du}{10.889292\du}}
\pgfpathlineto{\pgfpoint{25.994845\du}{10.879955\du}}
\pgfpathlineto{\pgfpoint{25.926376\du}{10.872175\du}}
\pgfpathlineto{\pgfpoint{25.856740\du}{10.865561\du}}
\pgfpathlineto{\pgfpoint{25.786715\du}{10.860115\du}}
\pgfpathlineto{\pgfpoint{25.715522\du}{10.855835\du}}
\pgfpathlineto{\pgfpoint{25.643941\du}{10.852334\du}}
\pgfpathlineto{\pgfpoint{25.571970\du}{10.850389\du}}
\pgfpathlineto{\pgfpoint{25.500000\du}{10.850000\du}}
\pgfpathlineto{\pgfpoint{25.500000\du}{10.850000\du}}
\pgfpathlineto{\pgfpoint{25.427641\du}{10.850389\du}}
\pgfpathlineto{\pgfpoint{25.355670\du}{10.852334\du}}
\pgfpathlineto{\pgfpoint{25.284089\du}{10.855835\du}}
\pgfpathlineto{\pgfpoint{25.212896\du}{10.860115\du}}
\pgfpathlineto{\pgfpoint{25.142871\du}{10.865561\du}}
\pgfpathlineto{\pgfpoint{25.073235\du}{10.872175\du}}
\pgfpathlineto{\pgfpoint{25.005155\du}{10.879955\du}}
\pgfpathlineto{\pgfpoint{24.938242\du}{10.889292\du}}
\pgfpathlineto{\pgfpoint{24.872496\du}{10.899796\du}}
\pgfpathlineto{\pgfpoint{24.808306\du}{10.911078\du}}
\pgfpathlineto{\pgfpoint{24.746450\du}{10.923527\du}}
\pgfpathlineto{\pgfpoint{24.686151\du}{10.937143\du}}
\pgfpathlineto{\pgfpoint{24.628185\du}{10.951926\du}}
\pgfpathlineto{\pgfpoint{24.572165\du}{10.967098\du}}
\pgfpathlineto{\pgfpoint{24.518479\du}{10.983437\du}}
\pgfpathlineto{\pgfpoint{24.467127\du}{11.001332\du}}
\pgfpathlineto{\pgfpoint{24.419276\du}{11.019617\du}}
\pgfpathlineto{\pgfpoint{24.373371\du}{11.038290\du}}
\pgfpathlineto{\pgfpoint{24.330578\du}{11.058520\du}}
\pgfpathlineto{\pgfpoint{24.290897\du}{11.079527\du}}
\pgfpathlineto{\pgfpoint{24.254328\du}{11.100146\du}}
\pgfpathlineto{\pgfpoint{24.220871\du}{11.122710\du}}
\pgfpathlineto{\pgfpoint{24.190527\du}{11.144884\du}}
\pgfpathlineto{\pgfpoint{24.163684\du}{11.167059\du}}
\pgfpathlineto{\pgfpoint{24.140342\du}{11.190790\du}}
\pgfpathlineto{\pgfpoint{24.120502\du}{11.214521\du}}
\pgfpathlineto{\pgfpoint{24.104552\du}{11.239029\du}}
\pgfpathlineto{\pgfpoint{24.091714\du}{11.263149\du}}
\pgfpathlineto{\pgfpoint{24.082377\du}{11.287269\du}}
\pgfpathlineto{\pgfpoint{24.077320\du}{11.311778\du}}
\pgfpathlineto{\pgfpoint{24.075374\du}{11.337065\du}}
\pgfpathlineto{\pgfpoint{24.075374\du}{11.337065\du}}
\pgfpathlineto{\pgfpoint{24.077320\du}{11.361574\du}}
\pgfpathlineto{\pgfpoint{24.082377\du}{11.386472\du}}
\pgfpathlineto{\pgfpoint{24.091714\du}{11.410202\du}}
\pgfpathlineto{\pgfpoint{24.104552\du}{11.435100\du}}
\pgfpathlineto{\pgfpoint{24.120502\du}{11.458831\du}}
\pgfpathlineto{\pgfpoint{24.140342\du}{11.482951\du}}
\pgfpathlineto{\pgfpoint{24.163684\du}{11.506293\du}}
\pgfpathlineto{\pgfpoint{24.190527\du}{11.528856\du}}
\pgfpathlineto{\pgfpoint{24.220871\du}{11.551420\du}}
\pgfpathlineto{\pgfpoint{24.254328\du}{11.573206\du}}
\pgfpathlineto{\pgfpoint{24.290897\du}{11.594602\du}}
\pgfpathlineto{\pgfpoint{24.330578\du}{11.615221\du}}
\pgfpathlineto{\pgfpoint{24.373371\du}{11.635061\du}}
\pgfpathlineto{\pgfpoint{24.419276\du}{11.654124\du}}
\pgfpathlineto{\pgfpoint{24.467127\du}{11.672797\du}}
\pgfpathlineto{\pgfpoint{24.518479\du}{11.689914\du}}
\pgfpathlineto{\pgfpoint{24.572165\du}{11.706643\du}}
\pgfpathlineto{\pgfpoint{24.628185\du}{11.722204\du}}
\pgfpathlineto{\pgfpoint{24.686151\du}{11.736598\du}}
\pgfpathlineto{\pgfpoint{24.746450\du}{11.750214\du}}
\pgfpathlineto{\pgfpoint{24.808306\du}{11.763052\du}}
\pgfpathlineto{\pgfpoint{24.872496\du}{11.773945\du}}
\pgfpathlineto{\pgfpoint{24.938242\du}{11.784838\du}}
\pgfpathlineto{\pgfpoint{25.005155\du}{11.793785\du}}
\pgfpathlineto{\pgfpoint{25.073235\du}{11.801566\du}}
\pgfpathlineto{\pgfpoint{25.142871\du}{11.808568\du}}
\pgfpathlineto{\pgfpoint{25.212896\du}{11.813626\du}}
\pgfpathlineto{\pgfpoint{25.284089\du}{11.818294\du}}
\pgfpathlineto{\pgfpoint{25.355670\du}{11.821406\du}}
\pgfpathlineto{\pgfpoint{25.427641\du}{11.823351\du}}
\pgfpathlineto{\pgfpoint{25.500000\du}{11.824130\du}}
\pgfpathlineto{\pgfpoint{25.500000\du}{11.824130\du}}
\pgfpathlineto{\pgfpoint{25.571970\du}{11.823351\du}}
\pgfpathlineto{\pgfpoint{25.643941\du}{11.821406\du}}
\pgfpathlineto{\pgfpoint{25.715522\du}{11.818294\du}}
\pgfpathlineto{\pgfpoint{25.786715\du}{11.813626\du}}
\pgfpathlineto{\pgfpoint{25.856740\du}{11.808568\du}}
\pgfpathlineto{\pgfpoint{25.926376\du}{11.801566\du}}
\pgfpathlineto{\pgfpoint{25.994845\du}{11.793785\du}}
\pgfpathlineto{\pgfpoint{26.061369\du}{11.784838\du}}
\pgfpathlineto{\pgfpoint{26.127115\du}{11.773945\du}}
\pgfpathlineto{\pgfpoint{26.191305\du}{11.763052\du}}
\pgfpathlineto{\pgfpoint{26.253161\du}{11.750214\du}}
\pgfpathlineto{\pgfpoint{26.313849\du}{11.736598\du}}
\pgfpathlineto{\pgfpoint{26.371815\du}{11.722204\du}}
\pgfpathlineto{\pgfpoint{26.427835\du}{11.706643\du}}
\pgfpathlineto{\pgfpoint{26.481132\du}{11.689914\du}}
\pgfpathlineto{\pgfpoint{26.532484\du}{11.672797\du}}
\pgfpathlineto{\pgfpoint{26.580724\du}{11.654124\du}}
\pgfpathlineto{\pgfpoint{26.626240\du}{11.635061\du}}
\pgfpathlineto{\pgfpoint{26.669422\du}{11.615221\du}}
\pgfpathlineto{\pgfpoint{26.708714\du}{11.594602\du}}
\pgfpathlineto{\pgfpoint{26.745283\du}{11.573206\du}}
\pgfpathlineto{\pgfpoint{26.778740\du}{11.551420\du}}
\pgfpathlineto{\pgfpoint{26.809084\du}{11.528856\du}}
\pgfpathlineto{\pgfpoint{26.835927\du}{11.506293\du}}
\pgfpathlineto{\pgfpoint{26.859269\du}{11.482951\du}}
\pgfpathlineto{\pgfpoint{26.879109\du}{11.458831\du}}
\pgfpathlineto{\pgfpoint{26.895059\du}{11.435100\du}}
\pgfpathlineto{\pgfpoint{26.907897\du}{11.410202\du}}
\pgfpathlineto{\pgfpoint{26.917234\du}{11.386472\du}}
\pgfpathlineto{\pgfpoint{26.922680\du}{11.361574\du}}
\pgfpathlineto{\pgfpoint{26.924626\du}{11.337065\du}}
\pgfusepath{fill}
\pgfsetbuttcap
\pgfsetmiterjoin
\pgfsetdash{}{0pt}
\definecolor{dialinecolor}{rgb}{0.286275, 0.286275, 0.211765}
\pgfsetstrokecolor{dialinecolor}
\pgfpathmoveto{\pgfpoint{26.907897\du}{11.329284\du}}
\pgfpathlineto{\pgfpoint{26.905952\du}{11.304386\du}}
\pgfpathlineto{\pgfpoint{26.900895\du}{11.280656\du}}
\pgfpathlineto{\pgfpoint{26.891947\du}{11.256925\du}}
\pgfpathlineto{\pgfpoint{26.879498\du}{11.232805\du}}
\pgfpathlineto{\pgfpoint{26.862770\du}{11.208296\du}}
\pgfpathlineto{\pgfpoint{26.842540\du}{11.185343\du}}
\pgfpathlineto{\pgfpoint{26.820366\du}{11.163169\du}}
\pgfpathlineto{\pgfpoint{26.793134\du}{11.140216\du}}
\pgfpathlineto{\pgfpoint{26.763178\du}{11.118430\du}}
\pgfpathlineto{\pgfpoint{26.730111\du}{11.096256\du}}
\pgfpathlineto{\pgfpoint{26.693542\du}{11.076026\du}}
\pgfpathlineto{\pgfpoint{26.654250\du}{11.055408\du}}
\pgfpathlineto{\pgfpoint{26.611846\du}{11.035567\du}}
\pgfpathlineto{\pgfpoint{26.567108\du}{11.017283\du}}
\pgfpathlineto{\pgfpoint{26.518090\du}{10.998998\du}}
\pgfpathlineto{\pgfpoint{26.467127\du}{10.982270\du}}
\pgfpathlineto{\pgfpoint{26.413830\du}{10.965931\du}}
\pgfpathlineto{\pgfpoint{26.358588\du}{10.950370\du}}
\pgfpathlineto{\pgfpoint{26.301011\du}{10.935975\du}}
\pgfpathlineto{\pgfpoint{26.241101\du}{10.922748\du}}
\pgfpathlineto{\pgfpoint{26.178856\du}{10.910689\du}}
\pgfpathlineto{\pgfpoint{26.115444\du}{10.899407\du}}
\pgfpathlineto{\pgfpoint{26.050088\du}{10.889292\du}}
\pgfpathlineto{\pgfpoint{25.983953\du}{10.879955\du}}
\pgfpathlineto{\pgfpoint{25.916261\du}{10.872175\du}}
\pgfpathlineto{\pgfpoint{25.846625\du}{10.865561\du}}
\pgfpathlineto{\pgfpoint{25.776989\du}{10.860115\du}}
\pgfpathlineto{\pgfpoint{25.706186\du}{10.855835\du}}
\pgfpathlineto{\pgfpoint{25.682455\du}{10.854668\du}}
\pgfpathlineto{\pgfpoint{25.300428\du}{10.854668\du}}
\pgfpathlineto{\pgfpoint{25.276697\du}{10.855835\du}}
\pgfpathlineto{\pgfpoint{25.206672\du}{10.860115\du}}
\pgfpathlineto{\pgfpoint{25.136647\du}{10.865561\du}}
\pgfpathlineto{\pgfpoint{25.067399\du}{10.872175\du}}
\pgfpathlineto{\pgfpoint{24.999319\du}{10.879955\du}}
\pgfpathlineto{\pgfpoint{24.932795\du}{10.889292\du}}
\pgfpathlineto{\pgfpoint{24.867827\du}{10.899407\du}}
\pgfpathlineto{\pgfpoint{24.804415\du}{10.910689\du}}
\pgfpathlineto{\pgfpoint{24.742171\du}{10.922748\du}}
\pgfpathlineto{\pgfpoint{24.682260\du}{10.935975\du}}
\pgfpathlineto{\pgfpoint{24.625073\du}{10.950370\du}}
\pgfpathlineto{\pgfpoint{24.569053\du}{10.965931\du}}
\pgfpathlineto{\pgfpoint{24.515367\du}{10.982270\du}}
\pgfpathlineto{\pgfpoint{24.464793\du}{10.998998\du}}
\pgfpathlineto{\pgfpoint{24.416553\du}{11.017283\du}}
\pgfpathlineto{\pgfpoint{24.371815\du}{11.035567\du}}
\pgfpathlineto{\pgfpoint{24.329022\du}{11.055408\du}}
\pgfpathlineto{\pgfpoint{24.289341\du}{11.076026\du}}
\pgfpathlineto{\pgfpoint{24.253550\du}{11.096256\du}}
\pgfpathlineto{\pgfpoint{24.220482\du}{11.118430\du}}
\pgfpathlineto{\pgfpoint{24.189749\du}{11.140216\du}}
\pgfpathlineto{\pgfpoint{24.163295\du}{11.163169\du}}
\pgfpathlineto{\pgfpoint{24.140342\du}{11.185343\du}}
\pgfpathlineto{\pgfpoint{24.120502\du}{11.208296\du}}
\pgfpathlineto{\pgfpoint{24.104163\du}{11.232805\du}}
\pgfpathlineto{\pgfpoint{24.091325\du}{11.256925\du}}
\pgfpathlineto{\pgfpoint{24.082377\du}{11.280656\du}}
\pgfpathlineto{\pgfpoint{24.077320\du}{11.304386\du}}
\pgfpathlineto{\pgfpoint{24.075374\du}{11.329284\du}}
\pgfpathlineto{\pgfpoint{24.075374\du}{11.329284\du}}
\pgfpathlineto{\pgfpoint{24.077320\du}{11.353015\du}}
\pgfpathlineto{\pgfpoint{24.082377\du}{11.377524\du}}
\pgfpathlineto{\pgfpoint{24.091325\du}{11.401644\du}}
\pgfpathlineto{\pgfpoint{24.104163\du}{11.425374\du}}
\pgfpathlineto{\pgfpoint{24.120502\du}{11.449105\du}}
\pgfpathlineto{\pgfpoint{24.140342\du}{11.472447\du}}
\pgfpathlineto{\pgfpoint{24.163295\du}{11.495011\du}}
\pgfpathlineto{\pgfpoint{24.189749\du}{11.517963\du}}
\pgfpathlineto{\pgfpoint{24.220482\du}{11.539749\du}}
\pgfpathlineto{\pgfpoint{24.253550\du}{11.561535\du}}
\pgfpathlineto{\pgfpoint{24.289341\du}{11.582542\du}}
\pgfpathlineto{\pgfpoint{24.329022\du}{11.602383\du}}
\pgfpathlineto{\pgfpoint{24.371815\du}{11.621834\du}}
\pgfpathlineto{\pgfpoint{24.416553\du}{11.641286\du}}
\pgfpathlineto{\pgfpoint{24.464793\du}{11.659181\du}}
\pgfpathlineto{\pgfpoint{24.515367\du}{11.675909\du}}
\pgfpathlineto{\pgfpoint{24.569053\du}{11.691860\du}}
\pgfpathlineto{\pgfpoint{24.625073\du}{11.707421\du}}
\pgfpathlineto{\pgfpoint{24.682260\du}{11.721426\du}}
\pgfpathlineto{\pgfpoint{24.742171\du}{11.735042\du}}
\pgfpathlineto{\pgfpoint{24.804415\du}{11.747880\du}}
\pgfpathlineto{\pgfpoint{24.867827\du}{11.757995\du}}
\pgfpathlineto{\pgfpoint{24.932795\du}{11.768887\du}}
\pgfpathlineto{\pgfpoint{24.999319\du}{11.777446\du}}
\pgfpathlineto{\pgfpoint{25.067399\du}{11.785616\du}}
\pgfpathlineto{\pgfpoint{25.136647\du}{11.792618\du}}
\pgfpathlineto{\pgfpoint{25.206672\du}{11.798065\du}}
\pgfpathlineto{\pgfpoint{25.276697\du}{11.802344\du}}
\pgfpathlineto{\pgfpoint{25.348279\du}{11.805067\du}}
\pgfpathlineto{\pgfpoint{25.419860\du}{11.806623\du}}
\pgfpathlineto{\pgfpoint{25.491441\du}{11.807790\du}}
\pgfpathlineto{\pgfpoint{25.491441\du}{11.807790\du}}
\pgfpathlineto{\pgfpoint{25.563023\du}{11.806623\du}}
\pgfpathlineto{\pgfpoint{25.635382\du}{11.805067\du}}
\pgfpathlineto{\pgfpoint{25.706186\du}{11.802344\du}}
\pgfpathlineto{\pgfpoint{25.776989\du}{11.798065\du}}
\pgfpathlineto{\pgfpoint{25.846625\du}{11.792618\du}}
\pgfpathlineto{\pgfpoint{25.916261\du}{11.785616\du}}
\pgfpathlineto{\pgfpoint{25.983953\du}{11.777446\du}}
\pgfpathlineto{\pgfpoint{26.050088\du}{11.768887\du}}
\pgfpathlineto{\pgfpoint{26.115444\du}{11.757995\du}}
\pgfpathlineto{\pgfpoint{26.178856\du}{11.747880\du}}
\pgfpathlineto{\pgfpoint{26.241101\du}{11.735042\du}}
\pgfpathlineto{\pgfpoint{26.301011\du}{11.721426\du}}
\pgfpathlineto{\pgfpoint{26.358588\du}{11.707421\du}}
\pgfpathlineto{\pgfpoint{26.413830\du}{11.691860\du}}
\pgfpathlineto{\pgfpoint{26.467127\du}{11.675909\du}}
\pgfpathlineto{\pgfpoint{26.518090\du}{11.659181\du}}
\pgfpathlineto{\pgfpoint{26.567108\du}{11.641286\du}}
\pgfpathlineto{\pgfpoint{26.611846\du}{11.621834\du}}
\pgfpathlineto{\pgfpoint{26.654250\du}{11.602383\du}}
\pgfpathlineto{\pgfpoint{26.693542\du}{11.582542\du}}
\pgfpathlineto{\pgfpoint{26.730111\du}{11.561535\du}}
\pgfpathlineto{\pgfpoint{26.763178\du}{11.539749\du}}
\pgfpathlineto{\pgfpoint{26.793134\du}{11.517963\du}}
\pgfpathlineto{\pgfpoint{26.820366\du}{11.495011\du}}
\pgfpathlineto{\pgfpoint{26.842540\du}{11.472447\du}}
\pgfpathlineto{\pgfpoint{26.862770\du}{11.449105\du}}
\pgfpathlineto{\pgfpoint{26.879498\du}{11.425374\du}}
\pgfpathlineto{\pgfpoint{26.891947\du}{11.401644\du}}
\pgfpathlineto{\pgfpoint{26.900895\du}{11.377524\du}}
\pgfpathlineto{\pgfpoint{26.905952\du}{11.353015\du}}
\pgfpathlineto{\pgfpoint{26.907897\du}{11.329284\du}}
\pgfusepath{stroke}
\pgfsetbuttcap
\pgfsetmiterjoin
\pgfsetdash{}{0pt}
\definecolor{dialinecolor}{rgb}{0.000000, 0.000000, 0.000000}
\pgfsetstrokecolor{dialinecolor}
\pgfpathmoveto{\pgfpoint{24.075374\du}{11.329284\du}}
\pgfpathlineto{\pgfpoint{24.075374\du}{12.354377\du}}
\pgfusepath{stroke}
\pgfsetbuttcap
\pgfsetmiterjoin
\pgfsetdash{}{0pt}
\definecolor{dialinecolor}{rgb}{0.000000, 0.000000, 0.000000}
\pgfsetstrokecolor{dialinecolor}
\pgfpathmoveto{\pgfpoint{26.907897\du}{11.329284\du}}
\pgfpathlineto{\pgfpoint{26.907897\du}{12.354377\du}}
\pgfusepath{stroke}
\pgfsetlinewidth{0.050000\du}
\pgfsetdash{}{0pt}
\pgfsetdash{}{0pt}
\pgfsetmiterjoin
\pgfsetbuttcap
{
\definecolor{dialinecolor}{rgb}{0.000000, 0.000000, 0.000000}
\pgfsetfillcolor{dialinecolor}
% was here!!!
\pgfsetarrowsend{to}
{\pgfsetcornersarced{\pgfpoint{0.000000\du}{0.000000\du}}\definecolor{dialinecolor}{rgb}{0.000000, 0.000000, 0.000000}
\pgfsetstrokecolor{dialinecolor}
\draw (21.526712\du,12.306822\du)--(23.800000\du,12.300000\du);
}}
% setfont left to latex
\definecolor{dialinecolor}{rgb}{0.000000, 0.000000, 0.000000}
\pgfsetstrokecolor{dialinecolor}
\node[anchor=west] at (24.900000\du,10.150000\du){Bdd};
% setfont left to latex
\definecolor{dialinecolor}{rgb}{0.000000, 0.000000, 0.000000}
\pgfsetstrokecolor{dialinecolor}
\node[anchor=west] at (12.350000\du,20.400000\du){Génération};
% setfont left to latex
\definecolor{dialinecolor}{rgb}{0.000000, 0.000000, 0.000000}
\pgfsetstrokecolor{dialinecolor}
\node[anchor=west] at (12.350000\du,21.200000\du){code HTML};
\pgfsetlinewidth{0.050000\du}
\pgfsetdash{}{0pt}
\pgfsetdash{}{0pt}
\pgfsetmiterjoin
\pgfsetbuttcap
{
\definecolor{dialinecolor}{rgb}{0.000000, 0.000000, 0.000000}
\pgfsetfillcolor{dialinecolor}
% was here!!!
\pgfsetarrowsend{to}
{\pgfsetcornersarced{\pgfpoint{0.000000\du}{0.000000\du}}\definecolor{dialinecolor}{rgb}{0.000000, 0.000000, 0.000000}
\pgfsetstrokecolor{dialinecolor}
\draw (12.150000\du,20.250000\du)--(12.150000\du,20.200000\du)--(5.550000\du,20.200000\du)--(5.550000\du,15.000000\du);
}}
% setfont left to latex
\definecolor{dialinecolor}{rgb}{0.000000, 0.000000, 0.000000}
\pgfsetstrokecolor{dialinecolor}
\node[anchor=west] at (12.400000\du,10.600000\du){Client demande};
% setfont left to latex
\definecolor{dialinecolor}{rgb}{0.000000, 0.000000, 0.000000}
\pgfsetstrokecolor{dialinecolor}
\node[anchor=west] at (12.400000\du,11.400000\du){une page};
\pgfsetlinewidth{0.050000\du}
\pgfsetdash{}{0pt}
\pgfsetdash{}{0pt}
\pgfsetbuttcap
{
\definecolor{dialinecolor}{rgb}{0.000000, 0.000000, 0.000000}
\pgfsetfillcolor{dialinecolor}
% was here!!!
\pgfsetarrowsend{to}
\definecolor{dialinecolor}{rgb}{0.000000, 0.000000, 0.000000}
\pgfsetstrokecolor{dialinecolor}
\draw (23.800000\du,12.950000\du)--(21.700000\du,12.900000\du);
}
\end{tikzpicture}
	
		\subsection{Syntaxe}
			Le code doit toujours être placé entre les balises <?php ?> dans une 
			page .php, le html peut se méler au php\\
			Toutes les instructions se terminent par ;\\
			Les commenaires se font avec // où /* */ \\
		\lstinputlisting[language=php, caption="Exemple hello wolrd"]{1.php}	
		\subsection{Variables}
			Toutes les variables doivent commencer par un \$, l'affectation
			se fait comme en C/C++.\\
			Pas de déclaration, les types sont gérés automatiquement.
		\lstinputlisting[language=php]{2.php}	
			\subsection{Opérations arithmétiques}
				\begin{tabular}{|c|c|}
					\hline
						+ & Addition\\
					\hline
						- & Soustraction  \\
					\hline
						/ & Division\\
					\hline	
						\% & Modulo\\
					\hline	
				\end{tabular}\\ \\
		\lstinputlisting[language=php, caption=Exemple]{3.php}	
		\subsection{Conditions}
			Les conditions ressemblent beaucoup au C/C++.
			\subsubsection{Opération de comparaison}
				\begin{tabular}{|c|c|c|}
					\hline	
						== & & Égalité\\
					\hline	
						!= & & Différence\\
					\hline	
						< & & Inférieur\\
					\hline	
						> & & Supérieur\\
					\hline	
						>= & & Supérieur ou égal\\
					\hline	
						<= & & Inférieur ou égal\\
					\hline	
				\end{tabular}\\ \\
			\subsubsection{Opérateurs logiques}
				\begin{tabular}{|c|c|c|}
					\hline
						|| & OR & ou\\
					\hline
						\&\& & AND & et \\
					\hline
						! & & Négation\\
					\hline
				\end{tabular}\\ \\
			\subsubsection{Concaténation}
			\lstinputlisting[language=php, caption=Concaténation]{10.php}	
					
			\subsubsection{if}
			\lstinputlisting[language=php, caption=if]{4.php}	
			\subsubsection{switch}
			\lstinputlisting[language=php, caption=switch]{5.php}	

		\subsection{Boucles}
			\subsubsection{While}
			\lstinputlisting[language=php, caption=while]{6.php}	
			\subsubsection{do...While}
			\lstinputlisting[language=php, caption=do...while]{7.php}	
			\subsubsection{for}
			\lstinputlisting[language=php, caption=for]{8.php}	
			\subsubsection{foreach}
			\lstinputlisting[language=php, caption=foreach]{9.php}	
					
		\subsection{Fonction}
			\subsubsection{Déclaration}
			\lstinputlisting[language=php, caption=Déclaration fonction]{11.php}	
			\lstinputlisting[language=php, caption=fonction]{12.php}	
			Pour déclarer une une procédure, il suffit de ne pas mettre 
			d'instruction return.
			\paragraph{Remarque} L'appel de la fonction doit se situer après la
			déclaration de la fonction.
			\subsubsection{include};
			\lstinputlisting[language=php, caption=Include]{13.php}	
			La fonction include permet d'insérer le code contenu dans un autre
			fichier.\\
			Permet par exemple d'insérer les corps des fonctions.\\
			\paragraph{Remarque} Il est possible d'inclure un fichier ne contenant
			que du HTML, cela peut être utile pour inclure un menu par exemple, 
			qui s'affiche sur toutes les pages, ainsi pour modifier le menu on 
			n'aura a modifier que le fichier contenant le menu.
		\subsection{Tableau}
			\subsubsection{Tableau à index numérique}
			\paragraph{} Ce sont les tableaux classiques.
			\lstinputlisting[language=php, caption=Index numérique]{14.php}	
			\paragraph{Remarque}Un tableau à index numérique commence à l'indice 0
			\subsubsection{Tableau à index associatif}
			\paragraph{} Très utilisé avec les bases de données.
			\lstinputlisting[language=php, caption=Index associatif]{15.php}	
			\paragraph{array\_key\_exist} Pour savoir si un array existe
			\paragraph{in\_array} Pour savoir si c'est dans un array 
		\subsection{Formulaires}
			\lstinputlisting[language=HTML, caption=Formulaire]{19.html}	
			\lstinputlisting[language=php, caption=Traitement]{16.php}	
			\subsubsection{Passage de variable dans l'URL}
			\lstinputlisting[language=HTML, caption=GET]{20.html}	
			\lstinputlisting[language=php, caption=GET]{17.php}	
		\subsection{Fichier}
			\subsubsection{fopen}
			\lstinputlisting[language=php, caption=fopen]{18.php}	
			\subsubsection{fclose}
			\lstinputlisting[language=php, caption=fclose]{19.php}	
			\subsubsection{Mode d'ouverture}
			%%%%%%%%% A COMPLETER %%%%%%%%%%%%
				\begin{itemize}
					\item r lecture seule
					\item r+ lecture + ecriture
					\item w ecriture
					\item w+
					\item a
					\item a+
				\end{itemize}
			\subsubsection{fgets}
			\lstinputlisting[language=php, caption=fgets]{20.php}	
			\lstinputlisting[language=php, caption=fread]{21.php}	

			\subsubsection{Autres fonction}
				\begin{itemize}
					\item feof
					\item filesize
					\item file\_exists
					\item unlink
				\end{itemize}	
			
			

\end{document}


