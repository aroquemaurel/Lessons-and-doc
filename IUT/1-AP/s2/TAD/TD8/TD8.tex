\documentclass{article}

\usepackage{lmodern}
\usepackage{xcolor}
\usepackage[utf8]{inputenc}
\usepackage[T1]{fontenc}
\usepackage[francais]{babel}
\usepackage[top=1.7cm, bottom=1.7cm, left=1.7cm, right=1.7cm]{geometry}
%\usepackage[frenchb]{babel}
%\usepackage{layout}
%\usepackage{setspace}
%\usepackage{soul}
%\usepackage{ulem}
%\usepackage{eurosym}
%\usepackage{bookman}
%\usepackage{charter}
%\usepackage{newcent}
%\usepackage{lmodern}
%\usepackage{mathpazo}
%\usepackage{mathptmx}
%\usepackage{url}
\usepackage{verbatim}
%\usepackage{moreverb}
%\usepackage{wrapfig}
%\usepackage{amsmath}
%\usepackage{mathrsfs}
%\usepackage{asmthm}
%\usepackage{makeidx}
\usepackage{tikz} %Vectoriel
\usepackage{listings}
\usepackage{fancyhdr}
\usepackage{multido}
\usepackage{amssymb}


\input{/home/satenske/cours/listings.tex} %prise en charge du langage algo

\title{TD 8\\ Implémentation dynamique d'une liste}
\date{TAD\\ Semestre 2}

\lhead{TD8: Implémentation dynamique d'une liste}
\chead{}
\rhead{\thepage}

\lfoot{Université Paul Sabatier Toulouse III}
\cfoot{\thepage}
\rfoot{tad2}

\pagestyle{fancy}
\begin{document}
	\maketitle
	\section{}
		La liste $ l = (e_{1}, e_{2} \dots e_{i}, e_{i+1} \dots e_{n})$
		\begin{verbatim}
   
 +-----+     +-----+-----+      +-----+-----+                     +-----+-----+        +-----+-----+       
 | --------->|     |  --------->|     |  ---------> ... --------->|     |   ---------> |     |     |
 |     |     | e1  |     |      |  e2 |     |                     | ei  |     |        |ei+1 |  |  |
l+-----+     +-----+-----+      +-----+-----+                     +-----+-----+        +-----+--|--+              
                                                                                                |
                                                                                                v
                                                                                             +-----+-----+       
                                                                                             |     |     |
                                                                                             | en  |NULL |
                                                                                             +-----+-----+
 

		\end{verbatim}

	\section{}
	\lstinputlisting[language=algo, caption=Longueur]{2-1.algo}	
	\lstinputlisting[language=algo, caption=creerListe]{2-2.algo}	
	\lstinputlisting[language=algo, caption=ieme]{2-3.algo}	
	\subsection{inserer}
	\subsubsection{Cas généra}
		\begin{verbatim}
   
 +-----+     +-----+-----+      +-----+-----+                     +-----+-----+        +-----+-----+       
 | --------->|     |  --------->|     |  ---------> ... --------->|     |   ---------> |     |     |
 |     |     | e1  |     |      |  e2 |     |                     | ei  |     |        |en   |NULL |
l+-----+     +-----+-----+      +-----+-----+                     +-----+-----+        +-----+-----+              
              ^                  ^                                   ^                  ^
         	    x                  x                                   x                  |
   +-----+    |                  |                                   |                  | 
aux|  --------+------------------+-----------------------------------+------------------+
   |     |
   +-----+

   +-----+
k  |     |
   |     |
   +-----+
		\end{verbatim}
	\lstinputlisting[language=algo]{2-4-a.algo}	
	\subsubsection{Cas particulier i=1 pour insérer}
	% Graphic for TeX using PGF
% Title: /home/satenske/cours/AP/tad2/TD8/2-4-b.tex
% Creator: Dia v0.97.1
% CreationDate: Mon May 23 08:47:38 2011
% For: satenske
% \usepackage{tikz}
% The following commands are not supported in PSTricks at present
% We define them conditionally, so when they are implemented,
% this pgf file will use them.
\ifx\du\undefined
  \newlength{\du}
\fi
\setlength{\du}{15\unitlength}
\begin{tikzpicture}
\pgftransformxscale{1.000000}
\pgftransformyscale{-1.000000}
\definecolor{dialinecolor}{rgb}{0.000000, 0.000000, 0.000000}
\pgfsetstrokecolor{dialinecolor}
\definecolor{dialinecolor}{rgb}{1.000000, 1.000000, 1.000000}
\pgfsetfillcolor{dialinecolor}
\pgfsetlinewidth{0.020000\du}
\pgfsetdash{}{0pt}
\pgfsetdash{}{0pt}
\pgfsetmiterjoin
\definecolor{dialinecolor}{rgb}{1.000000, 1.000000, 1.000000}
\pgfsetfillcolor{dialinecolor}
\fill (4.000000\du,3.200000\du)--(4.000000\du,5.500000\du)--(6.550000\du,5.500000\du)--(6.550000\du,3.200000\du)--cycle;
\definecolor{dialinecolor}{rgb}{0.000000, 0.000000, 0.000000}
\pgfsetstrokecolor{dialinecolor}
\draw (4.000000\du,3.200000\du)--(4.000000\du,5.500000\du)--(6.550000\du,5.500000\du)--(6.550000\du,3.200000\du)--cycle;
\pgfsetlinewidth{0.020000\du}
\pgfsetdash{}{0pt}
\pgfsetdash{}{0pt}
\pgfsetmiterjoin
\definecolor{dialinecolor}{rgb}{1.000000, 1.000000, 1.000000}
\pgfsetfillcolor{dialinecolor}
\fill (9.040000\du,3.185000\du)--(9.040000\du,5.485000\du)--(11.590000\du,5.485000\du)--(11.590000\du,3.185000\du)--cycle;
\definecolor{dialinecolor}{rgb}{0.000000, 0.000000, 0.000000}
\pgfsetstrokecolor{dialinecolor}
\draw (9.040000\du,3.185000\du)--(9.040000\du,5.485000\du)--(11.590000\du,5.485000\du)--(11.590000\du,3.185000\du)--cycle;
\pgfsetlinewidth{0.020000\du}
\pgfsetdash{}{0pt}
\pgfsetdash{}{0pt}
\pgfsetmiterjoin
\definecolor{dialinecolor}{rgb}{1.000000, 1.000000, 1.000000}
\pgfsetfillcolor{dialinecolor}
\fill (11.580000\du,3.220000\du)--(11.580000\du,5.520000\du)--(14.130000\du,5.520000\du)--(14.130000\du,3.220000\du)--cycle;
\definecolor{dialinecolor}{rgb}{0.000000, 0.000000, 0.000000}
\pgfsetstrokecolor{dialinecolor}
\draw (11.580000\du,3.220000\du)--(11.580000\du,5.520000\du)--(14.130000\du,5.520000\du)--(14.130000\du,3.220000\du)--cycle;
\pgfsetlinewidth{0.020000\du}
\pgfsetdash{}{0pt}
\pgfsetdash{}{0pt}
\pgfsetmiterjoin
\definecolor{dialinecolor}{rgb}{1.000000, 1.000000, 1.000000}
\pgfsetfillcolor{dialinecolor}
\fill (16.765000\du,3.265000\du)--(16.765000\du,5.565000\du)--(19.315000\du,5.565000\du)--(19.315000\du,3.265000\du)--cycle;
\definecolor{dialinecolor}{rgb}{0.000000, 0.000000, 0.000000}
\pgfsetstrokecolor{dialinecolor}
\draw (16.765000\du,3.265000\du)--(16.765000\du,5.565000\du)--(19.315000\du,5.565000\du)--(19.315000\du,3.265000\du)--cycle;
\pgfsetlinewidth{0.020000\du}
\pgfsetdash{}{0pt}
\pgfsetdash{}{0pt}
\pgfsetmiterjoin
\definecolor{dialinecolor}{rgb}{1.000000, 1.000000, 1.000000}
\pgfsetfillcolor{dialinecolor}
\fill (19.330000\du,3.250000\du)--(19.330000\du,5.550000\du)--(21.880000\du,5.550000\du)--(21.880000\du,3.250000\du)--cycle;
\definecolor{dialinecolor}{rgb}{0.000000, 0.000000, 0.000000}
\pgfsetstrokecolor{dialinecolor}
\draw (19.330000\du,3.250000\du)--(19.330000\du,5.550000\du)--(21.880000\du,5.550000\du)--(21.880000\du,3.250000\du)--cycle;
\pgfsetlinewidth{0.020000\du}
\pgfsetdash{}{0pt}
\pgfsetdash{}{0pt}
\pgfsetmiterjoin
\definecolor{dialinecolor}{rgb}{1.000000, 1.000000, 1.000000}
\pgfsetfillcolor{dialinecolor}
\fill (27.920000\du,3.320000\du)--(27.920000\du,5.620000\du)--(30.470000\du,5.620000\du)--(30.470000\du,3.320000\du)--cycle;
\definecolor{dialinecolor}{rgb}{0.000000, 0.000000, 0.000000}
\pgfsetstrokecolor{dialinecolor}
\draw (27.920000\du,3.320000\du)--(27.920000\du,5.620000\du)--(30.470000\du,5.620000\du)--(30.470000\du,3.320000\du)--cycle;
\pgfsetlinewidth{0.020000\du}
\pgfsetdash{}{0pt}
\pgfsetdash{}{0pt}
\pgfsetmiterjoin
\definecolor{dialinecolor}{rgb}{1.000000, 1.000000, 1.000000}
\pgfsetfillcolor{dialinecolor}
\fill (30.460000\du,3.355000\du)--(30.460000\du,5.655000\du)--(33.010000\du,5.655000\du)--(33.010000\du,3.355000\du)--cycle;
\definecolor{dialinecolor}{rgb}{0.000000, 0.000000, 0.000000}
\pgfsetstrokecolor{dialinecolor}
\draw (30.460000\du,3.355000\du)--(30.460000\du,5.655000\du)--(33.010000\du,5.655000\du)--(33.010000\du,3.355000\du)--cycle;
\pgfsetlinewidth{0.020000\du}
\pgfsetdash{}{0pt}
\pgfsetdash{}{0pt}
\pgfsetmiterjoin
\definecolor{dialinecolor}{rgb}{1.000000, 1.000000, 1.000000}
\pgfsetfillcolor{dialinecolor}
\fill (28.025000\du,9.975000\du)--(28.025000\du,12.275000\du)--(30.575000\du,12.275000\du)--(30.575000\du,9.975000\du)--cycle;
\definecolor{dialinecolor}{rgb}{0.000000, 0.000000, 0.000000}
\pgfsetstrokecolor{dialinecolor}
\draw (28.025000\du,9.975000\du)--(28.025000\du,12.275000\du)--(30.575000\du,12.275000\du)--(30.575000\du,9.975000\du)--cycle;
\pgfsetlinewidth{0.020000\du}
\pgfsetdash{}{0pt}
\pgfsetdash{}{0pt}
\pgfsetmiterjoin
\definecolor{dialinecolor}{rgb}{1.000000, 1.000000, 1.000000}
\pgfsetfillcolor{dialinecolor}
\fill (30.565000\du,10.010000\du)--(30.565000\du,12.310000\du)--(33.115000\du,12.310000\du)--(33.115000\du,10.010000\du)--cycle;
\definecolor{dialinecolor}{rgb}{0.000000, 0.000000, 0.000000}
\pgfsetstrokecolor{dialinecolor}
\draw (30.565000\du,10.010000\du)--(30.565000\du,12.310000\du)--(33.115000\du,12.310000\du)--(33.115000\du,10.010000\du)--cycle;
% setfont left to latex
\definecolor{dialinecolor}{rgb}{0.000000, 0.000000, 0.000000}
\pgfsetstrokecolor{dialinecolor}
\node[anchor=west] at (2.700000\du,5.400000\du){l};
\pgfsetlinewidth{0.020000\du}
\pgfsetdash{}{0pt}
\pgfsetdash{}{0pt}
\pgfsetbuttcap
{
\definecolor{dialinecolor}{rgb}{0.000000, 0.000000, 0.000000}
\pgfsetfillcolor{dialinecolor}
% was here!!!
\pgfsetarrowsend{stealth}
\definecolor{dialinecolor}{rgb}{0.000000, 0.000000, 0.000000}
\pgfsetstrokecolor{dialinecolor}
\draw (4.909108\du,4.244884\du)--(8.937500\du,4.237500\du);
}
\pgfsetlinewidth{0.020000\du}
\pgfsetdash{}{0pt}
\pgfsetdash{}{0pt}
\pgfsetbuttcap
{
\definecolor{dialinecolor}{rgb}{0.000000, 0.000000, 0.000000}
\pgfsetfillcolor{dialinecolor}
% was here!!!
\pgfsetarrowsend{stealth}
\definecolor{dialinecolor}{rgb}{0.000000, 0.000000, 0.000000}
\pgfsetstrokecolor{dialinecolor}
\draw (13.657518\du,4.226287\du)--(17.685911\du,4.218903\du);
}
\pgfsetlinewidth{0.020000\du}
\pgfsetdash{}{0pt}
\pgfsetdash{}{0pt}
\pgfsetbuttcap
{
\definecolor{dialinecolor}{rgb}{0.000000, 0.000000, 0.000000}
\pgfsetfillcolor{dialinecolor}
% was here!!!
\pgfsetarrowsend{stealth}
\definecolor{dialinecolor}{rgb}{0.000000, 0.000000, 0.000000}
\pgfsetstrokecolor{dialinecolor}
\draw (20.190018\du,4.308787\du)--(23.275000\du,4.300000\du);
}
% setfont left to latex
\definecolor{dialinecolor}{rgb}{0.000000, 0.000000, 0.000000}
\pgfsetstrokecolor{dialinecolor}
\node[anchor=west] at (23.800000\du,4.250000\du){...};
\pgfsetlinewidth{0.020000\du}
\pgfsetdash{}{0pt}
\pgfsetdash{}{0pt}
\pgfsetbuttcap
{
\definecolor{dialinecolor}{rgb}{0.000000, 0.000000, 0.000000}
\pgfsetfillcolor{dialinecolor}
% was here!!!
\pgfsetarrowsend{stealth}
\definecolor{dialinecolor}{rgb}{0.000000, 0.000000, 0.000000}
\pgfsetstrokecolor{dialinecolor}
\draw (24.937518\du,4.206287\du)--(27.875100\du,4.200000\du);
}
% setfont left to latex
\definecolor{dialinecolor}{rgb}{0.000000, 0.000000, 0.000000}
\pgfsetstrokecolor{dialinecolor}
\node[anchor=west] at (28.695000\du,11.320000\du){en};
% setfont left to latex
\definecolor{dialinecolor}{rgb}{0.000000, 0.000000, 0.000000}
\pgfsetstrokecolor{dialinecolor}
\node[anchor=west] at (31.164687\du,11.347187\du){NULL};
\pgfsetlinewidth{0.020000\du}
\pgfsetdash{}{0pt}
\pgfsetdash{}{0pt}
\pgfsetbuttcap
{
\definecolor{dialinecolor}{rgb}{0.000000, 0.000000, 0.000000}
\pgfsetfillcolor{dialinecolor}
% was here!!!
\pgfsetarrowsend{stealth}
\definecolor{dialinecolor}{rgb}{0.000000, 0.000000, 0.000000}
\pgfsetstrokecolor{dialinecolor}
\draw (5.269049\du,4.784332\du)--(5.725000\du,9.418750\du);
}
\pgfsetlinewidth{0.020000\du}
\pgfsetdash{}{0pt}
\pgfsetdash{}{0pt}
\pgfsetmiterjoin
\definecolor{dialinecolor}{rgb}{1.000000, 1.000000, 1.000000}
\pgfsetfillcolor{dialinecolor}
\fill (5.637500\du,9.387500\du)--(5.637500\du,11.687500\du)--(8.187500\du,11.687500\du)--(8.187500\du,9.387500\du)--cycle;
\definecolor{dialinecolor}{rgb}{0.000000, 0.000000, 0.000000}
\pgfsetstrokecolor{dialinecolor}
\draw (5.637500\du,9.387500\du)--(5.637500\du,11.687500\du)--(8.187500\du,11.687500\du)--(8.187500\du,9.387500\du)--cycle;
\pgfsetlinewidth{0.020000\du}
\pgfsetdash{}{0pt}
\pgfsetdash{}{0pt}
\pgfsetmiterjoin
\definecolor{dialinecolor}{rgb}{1.000000, 1.000000, 1.000000}
\pgfsetfillcolor{dialinecolor}
\fill (8.177500\du,9.422500\du)--(8.177500\du,11.722500\du)--(10.727500\du,11.722500\du)--(10.727500\du,9.422500\du)--cycle;
\definecolor{dialinecolor}{rgb}{0.000000, 0.000000, 0.000000}
\pgfsetstrokecolor{dialinecolor}
\draw (8.177500\du,9.422500\du)--(8.177500\du,11.722500\du)--(10.727500\du,11.722500\du)--(10.727500\du,9.422500\du)--cycle;
\pgfsetlinewidth{0.020000\du}
\pgfsetdash{}{0pt}
\pgfsetdash{}{0pt}
\pgfsetbuttcap
{
\definecolor{dialinecolor}{rgb}{0.000000, 0.000000, 0.000000}
\pgfsetfillcolor{dialinecolor}
% was here!!!
\pgfsetarrowsend{stealth}
\definecolor{dialinecolor}{rgb}{0.000000, 0.000000, 0.000000}
\pgfsetstrokecolor{dialinecolor}
\draw (9.446092\du,10.212291\du)--(9.425000\du,5.643750\du);
}
\pgfsetlinewidth{0.020000\du}
\pgfsetdash{}{0pt}
\pgfsetdash{}{0pt}
\pgfsetbuttcap
{
\definecolor{dialinecolor}{rgb}{0.000000, 0.000000, 0.000000}
\pgfsetfillcolor{dialinecolor}
% was here!!!
\definecolor{dialinecolor}{rgb}{0.000000, 0.000000, 0.000000}
\pgfsetstrokecolor{dialinecolor}
\draw (8.200000\du,3.443750\du)--(7.400000\du,5.193750\du);
}
\pgfsetlinewidth{0.020000\du}
\pgfsetdash{}{0pt}
\pgfsetdash{}{0pt}
\pgfsetbuttcap
{
\definecolor{dialinecolor}{rgb}{0.000000, 0.000000, 0.000000}
\pgfsetfillcolor{dialinecolor}
% was here!!!
\definecolor{dialinecolor}{rgb}{0.000000, 0.000000, 0.000000}
\pgfsetstrokecolor{dialinecolor}
\draw (6.915752\du,3.459502\du)--(8.375000\du,4.968750\du);
}
% setfont left to latex
\definecolor{dialinecolor}{rgb}{0.000000, 0.000000, 0.000000}
\pgfsetstrokecolor{dialinecolor}
\node[anchor=west] at (10.315000\du,4.335000\du){e1};
% setfont left to latex
\definecolor{dialinecolor}{rgb}{0.000000, 0.000000, 0.000000}
\pgfsetstrokecolor{dialinecolor}
\node[anchor=west] at (18.040000\du,4.415000\du){e2};
% setfont left to latex
\definecolor{dialinecolor}{rgb}{0.000000, 0.000000, 0.000000}
\pgfsetstrokecolor{dialinecolor}
\node[anchor=west] at (29.195000\du,4.470000\du){ei};
\pgfsetlinewidth{0.020000\du}
\pgfsetdash{}{0pt}
\pgfsetdash{}{0pt}
\pgfsetmiterjoin
\pgfsetbuttcap
{
\definecolor{dialinecolor}{rgb}{0.000000, 0.000000, 0.000000}
\pgfsetfillcolor{dialinecolor}
% was here!!!
\pgfsetarrowsend{to}
{\pgfsetcornersarced{\pgfpoint{0.000000\du}{0.000000\du}}\definecolor{dialinecolor}{rgb}{0.000000, 0.000000, 0.000000}
\pgfsetstrokecolor{dialinecolor}
\draw (31.835000\du,4.670752\du)--(31.835000\du,6.874219\du)--(29.175100\du,6.874219\du)--(29.175100\du,9.874219\du);
}}
\end{tikzpicture}

	\lstinputlisting[language=algo]{2-4-b.algo}	
	\subsubsection{Cas particulier fin de liste pour insérer}
	Ce cas se traite de la même manière qu'un cas général.
	\subsubsection{Programmation de la fonction}
	\lstinputlisting[language=algo, caption=inserer]{2-4.algo}	
	\subsection{supprimer}
	\lstinputlisting[language=algo, caption=supprimer]{2-5.algo}	

			
\end{document}
