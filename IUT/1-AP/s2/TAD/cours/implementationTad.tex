\section{Implémentation d'un type concret}
	\paragraph{Définition}
		Mise en œuvre informatique de la spécification algorithmique du type concret.\\
		Tâche du ressort du programmeur du type.\\
		L'implémentation doit respecter la spécification (spécification = cahier des charges pour le
			programmeur)

	\subsection{Tâche du programmeur}
		\begin{enumerate}
			\item Choisir une représentation mémoire pour coder les opérations du type
			\item Coder les corps des sous-programmes conformément à la spécification.
			\item Regrouper au sein d'un module la représentation mémoire et le codage des opérations. 
		\end{enumerate}

	\subsection{Conteneur d'un module d'implémentation}
		Un module peut contenir: 
		\begin{itemize}
			\item des déclarations de constantes
			\item des déclarations de types dont l'un au moins correspond à la définition du type étudié.
			\item les corps des sous-programmes définies par la spécification.
			\item tout sous-programmes nécessaire à la mise en œuvre du type. 
		\end{itemize}

	\subsection{Exemple}
		\subsubsection{1}
		Pour le TAD Point en représentation statique.
		\lstinputlisting{4-1.algo}

	\subsection{Implémentation du TAD Point en représentation dynamique}
		En représentation statique: \\
%			\relax 
\@setckpt{annexes/tachesRedmine}{
\setcounter{page}{2}
\setcounter{equation}{0}
\setcounter{enumi}{0}
\setcounter{enumii}{0}
\setcounter{enumiii}{0}
\setcounter{enumiv}{0}
\setcounter{footnote}{0}
\setcounter{mpfootnote}{0}
\setcounter{part}{0}
\setcounter{chapter}{0}
\setcounter{section}{0}
\setcounter{subsection}{0}
\setcounter{subsubsection}{0}
\setcounter{paragraph}{0}
\setcounter{subparagraph}{0}
\setcounter{figure}{0}
\setcounter{table}{0}
\setcounter{Item}{0}
\setcounter{Hfootnote}{2}
\setcounter{lstnumber}{1}
\setcounter{float@type}{8}
\setcounter{su@anzahl}{0}
\setcounter{DTLrowi}{0}
\setcounter{DTLrowii}{0}
\setcounter{DTLrowiii}{0}
\setcounter{DTLrow}{0}
\setcounter{section@level}{0}
\setcounter{lstlisting}{0}
}

		En représentation dynamique \\
%			\relax 
\@setckpt{annexes/tachesRedmine}{
\setcounter{page}{2}
\setcounter{equation}{0}
\setcounter{enumi}{0}
\setcounter{enumii}{0}
\setcounter{enumiii}{0}
\setcounter{enumiv}{0}
\setcounter{footnote}{0}
\setcounter{mpfootnote}{0}
\setcounter{part}{0}
\setcounter{chapter}{0}
\setcounter{section}{0}
\setcounter{subsection}{0}
\setcounter{subsubsection}{0}
\setcounter{paragraph}{0}
\setcounter{subparagraph}{0}
\setcounter{figure}{0}
\setcounter{table}{0}
\setcounter{Item}{0}
\setcounter{Hfootnote}{2}
\setcounter{lstnumber}{1}
\setcounter{float@type}{8}
\setcounter{su@anzahl}{0}
\setcounter{DTLrowi}{0}
\setcounter{DTLrowii}{0}
\setcounter{DTLrowiii}{0}
\setcounter{DTLrow}{0}
\setcounter{section@level}{0}
\setcounter{lstlisting}{0}
}

		(Un pointeur vers un enregistrement)
		Dans cette représentation dynamique		
		\lstinputlisting{4-2.algo}


\section{Sémantique de valeur et sémantique de référence}
	Dans le type concret, deux nouvelles opération s'ajoutent: l'affectation(<-) et l'égalité(=).
	\paragraph{Problème} Quelle signification (sémantique) donner à ces opérations pour un client? 
	\paragraph{Exemple} Pour un client
		\lstinputlisting{4-3.algo}

	\subsection{Sémantique de valeurs}
		\paragraph{Définition}
			Une affectation x<-y à une sémantique de valeurs si le conteneur (la valeur) de y est 
				recopiée dans x.
		\paragraph{Remarque}
			Avec une sémantique de valeur, toute opération de modification sur y après l'affectation y<-y
			n'affecte pas la valeur de x.
		\subsubsection{Exemple}
			Soit le TAD Point en représentation statique. (c'est-à-dire un enregistrement)\\
			Cette représentation statique à une sémantique de valeur pour le TAD Point
			Car la copie (<-) et la comparaison (= et /=) de deux enregistrements travaillant sur les
			valeur est un enregistrement.
			de champs d'un enregistrement. \\
			% Graphic for TeX using PGF
% Title: /usr/home/satenske/Diagram1.dia
% Creator: Dia v0.97.1
% CreationDate: Wed Feb 16 09:47:41 2011
% For: satenske
% \usepackage{tikz}
% The following commands are not supported in PSTricks at present
% We define them conditionally, so when they are implemented,
% this pgf file will use them.
\ifx\du\undefined
  \newlength{\du}
\fi
\setlength{\du}{15\unitlength}
\begin{tikzpicture}
\pgftransformxscale{1.000000}
\pgftransformyscale{-1.000000}
\definecolor{dialinecolor}{rgb}{0.000000, 0.000000, 0.000000}
\pgfsetstrokecolor{dialinecolor}
\definecolor{dialinecolor}{rgb}{1.000000, 1.000000, 1.000000}
\pgfsetfillcolor{dialinecolor}
\definecolor{dialinecolor}{rgb}{1.000000, 1.000000, 1.000000}
\pgfsetfillcolor{dialinecolor}
\fill (4.650000\du,9.100000\du)--(4.650000\du,13.300000\du)--(18.300000\du,13.300000\du)--(18.300000\du,9.100000\du)--cycle;
\pgfsetlinewidth{0.100000\du}
\pgfsetdash{}{0pt}
\pgfsetdash{}{0pt}
\pgfsetmiterjoin
\definecolor{dialinecolor}{rgb}{0.000000, 0.000000, 0.000000}
\pgfsetstrokecolor{dialinecolor}
\draw (4.650000\du,9.100000\du)--(4.650000\du,13.300000\du)--(18.300000\du,13.300000\du)--(18.300000\du,9.100000\du)--cycle;
% setfont left to latex
\definecolor{dialinecolor}{rgb}{0.000000, 0.000000, 0.000000}
\pgfsetstrokecolor{dialinecolor}
\node at (11.475000\du,11.395000\du){};
\definecolor{dialinecolor}{rgb}{1.000000, 1.000000, 1.000000}
\pgfsetfillcolor{dialinecolor}
\fill (5.791250\du,10.200000\du)--(5.791250\du,12.100000\du)--(7.908750\du,12.100000\du)--(7.908750\du,10.200000\du)--cycle;
\pgfsetlinewidth{0.100000\du}
\pgfsetdash{}{0pt}
\pgfsetdash{}{0pt}
\pgfsetmiterjoin
\definecolor{dialinecolor}{rgb}{0.000000, 0.000000, 0.000000}
\pgfsetstrokecolor{dialinecolor}
\draw (5.791250\du,10.200000\du)--(5.791250\du,12.100000\du)--(7.908750\du,12.100000\du)--(7.908750\du,10.200000\du)--cycle;
% setfont left to latex
\definecolor{dialinecolor}{rgb}{0.000000, 0.000000, 0.000000}
\pgfsetstrokecolor{dialinecolor}
\node at (6.850000\du,11.345000\du){0.0};
\definecolor{dialinecolor}{rgb}{1.000000, 1.000000, 1.000000}
\pgfsetfillcolor{dialinecolor}
\fill (9.000000\du,10.155000\du)--(9.000000\du,12.055000\du)--(11.117500\du,12.055000\du)--(11.117500\du,10.155000\du)--cycle;
\pgfsetlinewidth{0.100000\du}
\pgfsetdash{}{0pt}
\pgfsetdash{}{0pt}
\pgfsetmiterjoin
\definecolor{dialinecolor}{rgb}{0.000000, 0.000000, 0.000000}
\pgfsetstrokecolor{dialinecolor}
\draw (9.000000\du,10.155000\du)--(9.000000\du,12.055000\du)--(11.117500\du,12.055000\du)--(11.117500\du,10.155000\du)--cycle;
% setfont left to latex
\definecolor{dialinecolor}{rgb}{0.000000, 0.000000, 0.000000}
\pgfsetstrokecolor{dialinecolor}
\node at (10.058750\du,11.300000\du){1.0};
\definecolor{dialinecolor}{rgb}{1.000000, 1.000000, 1.000000}
\pgfsetfillcolor{dialinecolor}
\fill (11.890000\du,10.110000\du)--(11.890000\du,12.010000\du)--(14.227500\du,12.010000\du)--(14.227500\du,10.110000\du)--cycle;
\pgfsetlinewidth{0.100000\du}
\pgfsetdash{}{0pt}
\pgfsetdash{}{0pt}
\pgfsetmiterjoin
\definecolor{dialinecolor}{rgb}{0.000000, 0.000000, 0.000000}
\pgfsetstrokecolor{dialinecolor}
\draw (11.890000\du,10.110000\du)--(11.890000\du,12.010000\du)--(14.227500\du,12.010000\du)--(14.227500\du,10.110000\du)--cycle;
% setfont left to latex
\definecolor{dialinecolor}{rgb}{0.000000, 0.000000, 0.000000}
\pgfsetstrokecolor{dialinecolor}
\node at (13.058750\du,11.255000\du){noir};
\definecolor{dialinecolor}{rgb}{1.000000, 1.000000, 1.000000}
\pgfsetfillcolor{dialinecolor}
\fill (15.500000\du,10.165000\du)--(15.500000\du,12.065000\du)--(17.617500\du,12.065000\du)--(17.617500\du,10.165000\du)--cycle;
\pgfsetlinewidth{0.100000\du}
\pgfsetdash{}{0pt}
\pgfsetdash{}{0pt}
\pgfsetmiterjoin
\definecolor{dialinecolor}{rgb}{0.000000, 0.000000, 0.000000}
\pgfsetstrokecolor{dialinecolor}
\draw (15.500000\du,10.165000\du)--(15.500000\du,12.065000\du)--(17.617500\du,12.065000\du)--(17.617500\du,10.165000\du)--cycle;
% setfont left to latex
\definecolor{dialinecolor}{rgb}{0.000000, 0.000000, 0.000000}
\pgfsetstrokecolor{dialinecolor}
\node at (16.558750\du,11.310000\du){3.0};
% setfont left to latex
\definecolor{dialinecolor}{rgb}{0.000000, 0.000000, 0.000000}
\pgfsetstrokecolor{dialinecolor}
\node[anchor=west] at (3.000000\du,10.850000\du){P1};
\end{tikzpicture}
\\
			% Graphic for TeX using PGF
% Title: /usr/home/satenske/Diagram1.dia
% Creator: Dia v0.97.1
% CreationDate: Wed Feb 16 09:48:49 2011
% For: satenske
% \usepackage{tikz}
% The following commands are not supported in PSTricks at present
% We define them conditionally, so when they are implemented,
% this pgf file will use them.
\ifx\du\undefined
  \newlength{\du}
\fi
\setlength{\du}{15\unitlength}
\begin{tikzpicture}
\pgftransformxscale{1.000000}
\pgftransformyscale{-1.000000}
\definecolor{dialinecolor}{rgb}{0.000000, 0.000000, 0.000000}
\pgfsetstrokecolor{dialinecolor}
\definecolor{dialinecolor}{rgb}{1.000000, 1.000000, 1.000000}
\pgfsetfillcolor{dialinecolor}
\definecolor{dialinecolor}{rgb}{1.000000, 1.000000, 1.000000}
\pgfsetfillcolor{dialinecolor}
\fill (4.650000\du,9.100000\du)--(4.650000\du,13.300000\du)--(18.300000\du,13.300000\du)--(18.300000\du,9.100000\du)--cycle;
\pgfsetlinewidth{0.100000\du}
\pgfsetdash{}{0pt}
\pgfsetdash{}{0pt}
\pgfsetmiterjoin
\definecolor{dialinecolor}{rgb}{0.000000, 0.000000, 0.000000}
\pgfsetstrokecolor{dialinecolor}
\draw (4.650000\du,9.100000\du)--(4.650000\du,13.300000\du)--(18.300000\du,13.300000\du)--(18.300000\du,9.100000\du)--cycle;
% setfont left to latex
\definecolor{dialinecolor}{rgb}{0.000000, 0.000000, 0.000000}
\pgfsetstrokecolor{dialinecolor}
\node at (11.475000\du,11.395000\du){};
\definecolor{dialinecolor}{rgb}{1.000000, 1.000000, 1.000000}
\pgfsetfillcolor{dialinecolor}
\fill (5.791250\du,10.200000\du)--(5.791250\du,12.100000\du)--(7.908750\du,12.100000\du)--(7.908750\du,10.200000\du)--cycle;
\pgfsetlinewidth{0.100000\du}
\pgfsetdash{}{0pt}
\pgfsetdash{}{0pt}
\pgfsetmiterjoin
\definecolor{dialinecolor}{rgb}{0.000000, 0.000000, 0.000000}
\pgfsetstrokecolor{dialinecolor}
\draw (5.791250\du,10.200000\du)--(5.791250\du,12.100000\du)--(7.908750\du,12.100000\du)--(7.908750\du,10.200000\du)--cycle;
% setfont left to latex
\definecolor{dialinecolor}{rgb}{0.000000, 0.000000, 0.000000}
\pgfsetstrokecolor{dialinecolor}
\node at (6.850000\du,11.345000\du){1.5};
\definecolor{dialinecolor}{rgb}{1.000000, 1.000000, 1.000000}
\pgfsetfillcolor{dialinecolor}
\fill (9.000000\du,10.155000\du)--(9.000000\du,12.055000\du)--(11.117500\du,12.055000\du)--(11.117500\du,10.155000\du)--cycle;
\pgfsetlinewidth{0.100000\du}
\pgfsetdash{}{0pt}
\pgfsetdash{}{0pt}
\pgfsetmiterjoin
\definecolor{dialinecolor}{rgb}{0.000000, 0.000000, 0.000000}
\pgfsetstrokecolor{dialinecolor}
\draw (9.000000\du,10.155000\du)--(9.000000\du,12.055000\du)--(11.117500\du,12.055000\du)--(11.117500\du,10.155000\du)--cycle;
% setfont left to latex
\definecolor{dialinecolor}{rgb}{0.000000, 0.000000, 0.000000}
\pgfsetstrokecolor{dialinecolor}
\node at (10.058750\du,11.300000\du){3.5};
\definecolor{dialinecolor}{rgb}{1.000000, 1.000000, 1.000000}
\pgfsetfillcolor{dialinecolor}
\fill (11.502500\du,10.110000\du)--(11.502500\du,12.010000\du)--(14.615000\du,12.010000\du)--(14.615000\du,10.110000\du)--cycle;
\pgfsetlinewidth{0.100000\du}
\pgfsetdash{}{0pt}
\pgfsetdash{}{0pt}
\pgfsetmiterjoin
\definecolor{dialinecolor}{rgb}{0.000000, 0.000000, 0.000000}
\pgfsetstrokecolor{dialinecolor}
\draw (11.502500\du,10.110000\du)--(11.502500\du,12.010000\du)--(14.615000\du,12.010000\du)--(14.615000\du,10.110000\du)--cycle;
% setfont left to latex
\definecolor{dialinecolor}{rgb}{0.000000, 0.000000, 0.000000}
\pgfsetstrokecolor{dialinecolor}
\node at (13.058750\du,11.255000\du){vert};
\definecolor{dialinecolor}{rgb}{1.000000, 1.000000, 1.000000}
\pgfsetfillcolor{dialinecolor}
\fill (15.500000\du,10.165000\du)--(15.500000\du,12.065000\du)--(17.617500\du,12.065000\du)--(17.617500\du,10.165000\du)--cycle;
\pgfsetlinewidth{0.100000\du}
\pgfsetdash{}{0pt}
\pgfsetdash{}{0pt}
\pgfsetmiterjoin
\definecolor{dialinecolor}{rgb}{0.000000, 0.000000, 0.000000}
\pgfsetstrokecolor{dialinecolor}
\draw (15.500000\du,10.165000\du)--(15.500000\du,12.065000\du)--(17.617500\du,12.065000\du)--(17.617500\du,10.165000\du)--cycle;
% setfont left to latex
\definecolor{dialinecolor}{rgb}{0.000000, 0.000000, 0.000000}
\pgfsetstrokecolor{dialinecolor}
\node at (16.558750\du,11.310000\du){1.0};
% setfont left to latex
\definecolor{dialinecolor}{rgb}{0.000000, 0.000000, 0.000000}
\pgfsetstrokecolor{dialinecolor}
\node[anchor=west] at (3.000000\du,10.850000\du){P2};
% setfont left to latex
\definecolor{dialinecolor}{rgb}{0.000000, 0.000000, 0.000000}
\pgfsetstrokecolor{dialinecolor}
\node[anchor=west] at (3.650000\du,10.500000\du){};
\end{tikzpicture}

			\lstinputlisting{4-4-1.algo}
			\lstinputlisting{4-4.algo}
			On obtient l'affichage: \\
			3.5\\1.0

	\subsection{Sémantique de référence}
		\paragraph{Définition}
			L'affectation x <- y désigne le même contenu mémoire pouvant être référencé a la fois par x et 
			par y.
		\paragraph{Remarque}
			Après l'affectation x <- y avec un sémantique de référence toute modification sur y se 
			répercute sur x! \\
			On dit que x et y sont des alias pour désigner le même contenu.	

		\subsubsection{Exemple}
			Soit le TAD Point en représentation dynamique (c'est-à-dire par un pointeur vers un
			enregistrement).\\
			$\Rightarrow$ Sémantique de référence pour le TAD Point\\
			(Car affectation et comparaison de deux pointeurs!)\\
			% Graphic for TeX using PGF
% Title: /usr/home/satenske/cours/AP/tad2/cours/Diagram4.dia
% Creator: Dia v0.97.1
% CreationDate: Wed Feb 16 10:05:29 2011
% For: satenske
% \usepackage{tikz}
% The following commands are not supported in PSTricks at present
% We define them conditionally, so when they are implemented,
% this pgf file will use them.
\ifx\du\undefined
  \newlength{\du}
\fi
\setlength{\du}{15\unitlength}
\begin{tikzpicture}
\pgftransformxscale{1.000000}
\pgftransformyscale{-1.000000}
\definecolor{dialinecolor}{rgb}{0.000000, 0.000000, 0.000000}
\pgfsetstrokecolor{dialinecolor}
\definecolor{dialinecolor}{rgb}{1.000000, 1.000000, 1.000000}
\pgfsetfillcolor{dialinecolor}
\definecolor{dialinecolor}{rgb}{1.000000, 1.000000, 1.000000}
\pgfsetfillcolor{dialinecolor}
\fill (4.650000\du,9.100000\du)--(4.650000\du,13.300000\du)--(18.300000\du,13.300000\du)--(18.300000\du,9.100000\du)--cycle;
\pgfsetlinewidth{0.050000\du}
\pgfsetdash{}{0pt}
\pgfsetdash{}{0pt}
\pgfsetmiterjoin
\definecolor{dialinecolor}{rgb}{0.000000, 0.000000, 0.000000}
\pgfsetstrokecolor{dialinecolor}
\draw (4.650000\du,9.100000\du)--(4.650000\du,13.300000\du)--(18.300000\du,13.300000\du)--(18.300000\du,9.100000\du)--cycle;
% setfont left to latex
\definecolor{dialinecolor}{rgb}{0.000000, 0.000000, 0.000000}
\pgfsetstrokecolor{dialinecolor}
\node at (11.475000\du,11.395000\du){};
\definecolor{dialinecolor}{rgb}{1.000000, 1.000000, 1.000000}
\pgfsetfillcolor{dialinecolor}
\fill (5.791250\du,10.200000\du)--(5.791250\du,12.100000\du)--(7.908750\du,12.100000\du)--(7.908750\du,10.200000\du)--cycle;
\pgfsetlinewidth{0.050000\du}
\pgfsetdash{}{0pt}
\pgfsetdash{}{0pt}
\pgfsetmiterjoin
\definecolor{dialinecolor}{rgb}{0.000000, 0.000000, 0.000000}
\pgfsetstrokecolor{dialinecolor}
\draw (5.791250\du,10.200000\du)--(5.791250\du,12.100000\du)--(7.908750\du,12.100000\du)--(7.908750\du,10.200000\du)--cycle;
% setfont left to latex
\definecolor{dialinecolor}{rgb}{0.000000, 0.000000, 0.000000}
\pgfsetstrokecolor{dialinecolor}
\node at (6.850000\du,11.345000\du){1.5};
\definecolor{dialinecolor}{rgb}{1.000000, 1.000000, 1.000000}
\pgfsetfillcolor{dialinecolor}
\fill (9.000000\du,10.155000\du)--(9.000000\du,12.055000\du)--(11.117500\du,12.055000\du)--(11.117500\du,10.155000\du)--cycle;
\pgfsetlinewidth{0.050000\du}
\pgfsetdash{}{0pt}
\pgfsetdash{}{0pt}
\pgfsetmiterjoin
\definecolor{dialinecolor}{rgb}{0.000000, 0.000000, 0.000000}
\pgfsetstrokecolor{dialinecolor}
\draw (9.000000\du,10.155000\du)--(9.000000\du,12.055000\du)--(11.117500\du,12.055000\du)--(11.117500\du,10.155000\du)--cycle;
% setfont left to latex
\definecolor{dialinecolor}{rgb}{0.000000, 0.000000, 0.000000}
\pgfsetstrokecolor{dialinecolor}
\node at (10.058750\du,11.300000\du){3.5};
\definecolor{dialinecolor}{rgb}{1.000000, 1.000000, 1.000000}
\pgfsetfillcolor{dialinecolor}
\fill (11.502500\du,10.110000\du)--(11.502500\du,12.010000\du)--(14.615000\du,12.010000\du)--(14.615000\du,10.110000\du)--cycle;
\pgfsetlinewidth{0.050000\du}
\pgfsetdash{}{0pt}
\pgfsetdash{}{0pt}
\pgfsetmiterjoin
\definecolor{dialinecolor}{rgb}{0.000000, 0.000000, 0.000000}
\pgfsetstrokecolor{dialinecolor}
\draw (11.502500\du,10.110000\du)--(11.502500\du,12.010000\du)--(14.615000\du,12.010000\du)--(14.615000\du,10.110000\du)--cycle;
% setfont left to latex
\definecolor{dialinecolor}{rgb}{0.000000, 0.000000, 0.000000}
\pgfsetstrokecolor{dialinecolor}
\node at (13.058750\du,11.255000\du){vert};
\definecolor{dialinecolor}{rgb}{1.000000, 1.000000, 1.000000}
\pgfsetfillcolor{dialinecolor}
\fill (15.500000\du,10.165000\du)--(15.500000\du,12.065000\du)--(17.617500\du,12.065000\du)--(17.617500\du,10.165000\du)--cycle;
\pgfsetlinewidth{0.050000\du}
\pgfsetdash{}{0pt}
\pgfsetdash{}{0pt}
\pgfsetmiterjoin
\definecolor{dialinecolor}{rgb}{0.000000, 0.000000, 0.000000}
\pgfsetstrokecolor{dialinecolor}
\draw (15.500000\du,10.165000\du)--(15.500000\du,12.065000\du)--(17.617500\du,12.065000\du)--(17.617500\du,10.165000\du)--cycle;
% setfont left to latex
\definecolor{dialinecolor}{rgb}{0.000000, 0.000000, 0.000000}
\pgfsetstrokecolor{dialinecolor}
\node at (16.558750\du,11.310000\du){1.0};
% setfont left to latex
\definecolor{dialinecolor}{rgb}{0.000000, 0.000000, 0.000000}
\pgfsetstrokecolor{dialinecolor}
\node[anchor=west] at (-0.350000\du,13.550000\du){P2};
% setfont left to latex
\definecolor{dialinecolor}{rgb}{0.000000, 0.000000, 0.000000}
\pgfsetstrokecolor{dialinecolor}
\node[anchor=west] at (3.650000\du,10.500000\du){};
\definecolor{dialinecolor}{rgb}{1.000000, 1.000000, 1.000000}
\pgfsetfillcolor{dialinecolor}
\fill (-0.550000\du,10.000000\du)--(-0.550000\du,11.900000\du)--(1.450000\du,11.900000\du)--(1.450000\du,10.000000\du)--cycle;
\pgfsetlinewidth{0.050000\du}
\pgfsetdash{}{0pt}
\pgfsetdash{}{0pt}
\pgfsetmiterjoin
\definecolor{dialinecolor}{rgb}{0.000000, 0.000000, 0.000000}
\pgfsetstrokecolor{dialinecolor}
\draw (-0.550000\du,10.000000\du)--(-0.550000\du,11.900000\du)--(1.450000\du,11.900000\du)--(1.450000\du,10.000000\du)--cycle;
% setfont left to latex
\definecolor{dialinecolor}{rgb}{0.000000, 0.000000, 0.000000}
\pgfsetstrokecolor{dialinecolor}
\node at (0.450000\du,11.145000\du){};
\pgfsetlinewidth{0.050000\du}
\pgfsetdash{}{0pt}
\pgfsetdash{}{0pt}
\pgfsetmiterjoin
\pgfsetbuttcap
{
\definecolor{dialinecolor}{rgb}{0.000000, 0.000000, 0.000000}
\pgfsetfillcolor{dialinecolor}
% was here!!!
\pgfsetarrowsend{to}
{\pgfsetcornersarced{\pgfpoint{0.000000\du}{0.000000\du}}\definecolor{dialinecolor}{rgb}{0.000000, 0.000000, 0.000000}
\pgfsetstrokecolor{dialinecolor}
\draw (1.475250\du,10.950000\du)--(3.062625\du,10.950000\du)--(3.062625\du,11.200000\du)--(4.650000\du,11.200000\du);
}}
\definecolor{dialinecolor}{rgb}{1.000000, 1.000000, 1.000000}
\pgfsetfillcolor{dialinecolor}
\fill (4.275000\du,-0.932500\du)--(4.275000\du,3.267500\du)--(17.925000\du,3.267500\du)--(17.925000\du,-0.932500\du)--cycle;
\pgfsetlinewidth{0.050000\du}
\pgfsetdash{}{0pt}
\pgfsetdash{}{0pt}
\pgfsetmiterjoin
\definecolor{dialinecolor}{rgb}{0.000000, 0.000000, 0.000000}
\pgfsetstrokecolor{dialinecolor}
\draw (4.275000\du,-0.932500\du)--(4.275000\du,3.267500\du)--(17.925000\du,3.267500\du)--(17.925000\du,-0.932500\du)--cycle;
% setfont left to latex
\definecolor{dialinecolor}{rgb}{0.000000, 0.000000, 0.000000}
\pgfsetstrokecolor{dialinecolor}
\node at (11.100000\du,1.362500\du){};
\definecolor{dialinecolor}{rgb}{1.000000, 1.000000, 1.000000}
\pgfsetfillcolor{dialinecolor}
\fill (5.416250\du,0.167500\du)--(5.416250\du,2.067500\du)--(7.533750\du,2.067500\du)--(7.533750\du,0.167500\du)--cycle;
\pgfsetlinewidth{0.050000\du}
\pgfsetdash{}{0pt}
\pgfsetdash{}{0pt}
\pgfsetmiterjoin
\definecolor{dialinecolor}{rgb}{0.000000, 0.000000, 0.000000}
\pgfsetstrokecolor{dialinecolor}
\draw (5.416250\du,0.167500\du)--(5.416250\du,2.067500\du)--(7.533750\du,2.067500\du)--(7.533750\du,0.167500\du)--cycle;
% setfont left to latex
\definecolor{dialinecolor}{rgb}{0.000000, 0.000000, 0.000000}
\pgfsetstrokecolor{dialinecolor}
\node at (6.475000\du,1.312500\du){0.0};
\definecolor{dialinecolor}{rgb}{1.000000, 1.000000, 1.000000}
\pgfsetfillcolor{dialinecolor}
\fill (8.625000\du,0.122500\du)--(8.625000\du,2.022500\du)--(10.742500\du,2.022500\du)--(10.742500\du,0.122500\du)--cycle;
\pgfsetlinewidth{0.050000\du}
\pgfsetdash{}{0pt}
\pgfsetdash{}{0pt}
\pgfsetmiterjoin
\definecolor{dialinecolor}{rgb}{0.000000, 0.000000, 0.000000}
\pgfsetstrokecolor{dialinecolor}
\draw (8.625000\du,0.122500\du)--(8.625000\du,2.022500\du)--(10.742500\du,2.022500\du)--(10.742500\du,0.122500\du)--cycle;
% setfont left to latex
\definecolor{dialinecolor}{rgb}{0.000000, 0.000000, 0.000000}
\pgfsetstrokecolor{dialinecolor}
\node at (9.683750\du,1.267500\du){1.0};
\definecolor{dialinecolor}{rgb}{1.000000, 1.000000, 1.000000}
\pgfsetfillcolor{dialinecolor}
\fill (11.127500\du,0.077500\du)--(11.127500\du,1.977500\du)--(14.240000\du,1.977500\du)--(14.240000\du,0.077500\du)--cycle;
\pgfsetlinewidth{0.050000\du}
\pgfsetdash{}{0pt}
\pgfsetdash{}{0pt}
\pgfsetmiterjoin
\definecolor{dialinecolor}{rgb}{0.000000, 0.000000, 0.000000}
\pgfsetstrokecolor{dialinecolor}
\draw (11.127500\du,0.077500\du)--(11.127500\du,1.977500\du)--(14.240000\du,1.977500\du)--(14.240000\du,0.077500\du)--cycle;
% setfont left to latex
\definecolor{dialinecolor}{rgb}{0.000000, 0.000000, 0.000000}
\pgfsetstrokecolor{dialinecolor}
\node at (12.683750\du,1.222500\du){Noir};
\definecolor{dialinecolor}{rgb}{1.000000, 1.000000, 1.000000}
\pgfsetfillcolor{dialinecolor}
\fill (15.125000\du,0.132500\du)--(15.125000\du,2.032500\du)--(17.242500\du,2.032500\du)--(17.242500\du,0.132500\du)--cycle;
\pgfsetlinewidth{0.050000\du}
\pgfsetdash{}{0pt}
\pgfsetdash{}{0pt}
\pgfsetmiterjoin
\definecolor{dialinecolor}{rgb}{0.000000, 0.000000, 0.000000}
\pgfsetstrokecolor{dialinecolor}
\draw (15.125000\du,0.132500\du)--(15.125000\du,2.032500\du)--(17.242500\du,2.032500\du)--(17.242500\du,0.132500\du)--cycle;
% setfont left to latex
\definecolor{dialinecolor}{rgb}{0.000000, 0.000000, 0.000000}
\pgfsetstrokecolor{dialinecolor}
\node at (16.183750\du,1.277500\du){3.0};
% setfont left to latex
\definecolor{dialinecolor}{rgb}{0.000000, 0.000000, 0.000000}
\pgfsetstrokecolor{dialinecolor}
\node[anchor=west] at (-0.725000\du,3.517500\du){P1};
% setfont left to latex
\definecolor{dialinecolor}{rgb}{0.000000, 0.000000, 0.000000}
\pgfsetstrokecolor{dialinecolor}
\node[anchor=west] at (3.275000\du,0.467500\du){};
\definecolor{dialinecolor}{rgb}{1.000000, 1.000000, 1.000000}
\pgfsetfillcolor{dialinecolor}
\fill (-0.925000\du,-0.032500\du)--(-0.925000\du,1.867500\du)--(1.075000\du,1.867500\du)--(1.075000\du,-0.032500\du)--cycle;
\pgfsetlinewidth{0.050000\du}
\pgfsetdash{}{0pt}
\pgfsetdash{}{0pt}
\pgfsetmiterjoin
\definecolor{dialinecolor}{rgb}{0.000000, 0.000000, 0.000000}
\pgfsetstrokecolor{dialinecolor}
\draw (-0.925000\du,-0.032500\du)--(-0.925000\du,1.867500\du)--(1.075000\du,1.867500\du)--(1.075000\du,-0.032500\du)--cycle;
% setfont left to latex
\definecolor{dialinecolor}{rgb}{0.000000, 0.000000, 0.000000}
\pgfsetstrokecolor{dialinecolor}
\node at (0.075000\du,1.112500\du){};
\pgfsetlinewidth{0.050000\du}
\pgfsetdash{}{0pt}
\pgfsetdash{}{0pt}
\pgfsetmiterjoin
\pgfsetbuttcap
{
\definecolor{dialinecolor}{rgb}{0.000000, 0.000000, 0.000000}
\pgfsetfillcolor{dialinecolor}
% was here!!!
\pgfsetarrowsend{to}
{\pgfsetcornersarced{\pgfpoint{0.000000\du}{0.000000\du}}\definecolor{dialinecolor}{rgb}{0.000000, 0.000000, 0.000000}
\pgfsetstrokecolor{dialinecolor}
\draw (0.650250\du,0.817500\du)--(2.237625\du,0.817500\du)--(2.237625\du,1.067500\du)--(3.825000\du,1.067500\du);
}}
\pgfsetlinewidth{0.050000\du}
\pgfsetdash{}{0pt}
\pgfsetdash{}{0pt}
\pgfsetbuttcap
{
\definecolor{dialinecolor}{rgb}{0.000000, 0.000000, 0.000000}
\pgfsetfillcolor{dialinecolor}
% was here!!!
\definecolor{dialinecolor}{rgb}{0.000000, 0.000000, 0.000000}
\pgfsetstrokecolor{dialinecolor}
\draw (3.500000\du,-0.612500\du)--(2.200000\du,2.287500\du);
}
\pgfsetlinewidth{0.050000\du}
\pgfsetdash{}{0pt}
\pgfsetdash{}{0pt}
\pgfsetbuttcap
{
\definecolor{dialinecolor}{rgb}{0.000000, 0.000000, 0.000000}
\pgfsetfillcolor{dialinecolor}
% was here!!!
\definecolor{dialinecolor}{rgb}{0.000000, 0.000000, 0.000000}
\pgfsetstrokecolor{dialinecolor}
\draw (1.616078\du,-0.741422\du)--(3.400000\du,2.637500\du);
}
\pgfsetlinewidth{0.050000\du}
\pgfsetdash{}{0pt}
\pgfsetdash{}{0pt}
\pgfsetmiterjoin
\pgfsetbuttcap
{
\definecolor{dialinecolor}{rgb}{0.000000, 0.000000, 0.000000}
\pgfsetfillcolor{dialinecolor}
% was here!!!
\pgfsetarrowsend{to}
{\pgfsetcornersarced{\pgfpoint{0.000000\du}{0.000000\du}}\definecolor{dialinecolor}{rgb}{0.000000, 0.000000, 0.000000}
\pgfsetstrokecolor{dialinecolor}
\draw (0.325000\du,1.467500\du)--(2.262500\du,1.467500\du)--(2.262500\du,9.987500\du)--(4.200000\du,9.987500\du);
}}
\end{tikzpicture}
				
			\lstinputlisting{4-4-1.algo}
			\lstinputlisting{4-5.algo}
			On obtient à l'affichage\\ 3.5\\3.5	
			\paragraph{Remarque}
				\subparagraph{1- }
					Avec une représentation par pointeur (et donc avec un sémantique de référence) 
					on peut définir un type de sémantique de valeur\\
					$\Rightarrow$ ajouter dans le TAD une opération de copie et une opération de 
					comparaison.\\
					Pour le TAD Point en représentation dynamique on définit
					\lstinputlisting{4-6.algo}
				\subparagraph{2- }
					En C, un tableau est représenté par un pointeur constant vers son premier élément. 	\\
					$\Rightarrow$ Affectation de deux tableaux n'est pas autorisée!\\
					Comparaison de deux tableaux sont autorisés: comparaisons de deux adresses (différent
					de la comparaison des élément des deux tableaux).


\section{Exportation des opérateurs <-, = et /=}
	\subsection{Implémentation sans exportation des opérateurs}
		La spécification du TAD limite les opérations en interdisant l'usage de l'affectation et des 
		comparaisons.\\
		$\Rightarrow$ la spécification n'inclut pas d'en tête pour <-, = et /=
		\subsubsection{Exemple}
			Pour le TAD Point en implémentation statique ou dynamique. \\
			Pas d'indication de la spécification vis-à-vis des opérateurs implique l'interdiction au client
			d'utiliser les opérateurs. (voir spécification du TAD du chapitre 2) 
			\lstinputlisting[caption=Pour un client]{5-1.algo}
		\paragraph{Remarque} 
			Le statut d'exportation permet de protéger les données d'un type abstrait 
			(renforce l'encapsulation)\\
			Par exemple soit le TAD compteInformatique représente
			\lstinputlisting{5-2.algo}
	\subsection{Implémentation sans exportation}
		La spécification du type abstrait avec un en-tête par opérateur selon la syntaxe pour le type T.	
			\lstinputlisting[caption=Spécification]{5-3.algo}
			\lstinputlisting[caption=Cotès client]{5-4.algo}
			Pour l'implémentation du type deux possibilités.
			\begin{itemize}
				\item Pas de corps pour l'opérateur si la sémantique donné par le langage correspond à
					celle du type.
				\item Écriture d'un corps pour l'opérateur à la sémantique du langage ne correspond plus
					à celle du type abstrait
			\end{itemize}	
	\subsection{Synthèse}
		\begin{tabular}{|p{3cm}|p{7cm}|p{7cm}|}
			\hline
				& Implémentation sans exportations & Implémentation avec exportations\\ % à fusionner.... 
			\hline
				Utilisation (client) & opérateur non définie & Opérateur définie\\
			\hline 
				Spécification (concepteur) & Pas d'entête pour l'opérateur & Entête pour l'opérateur \\ 
			\hline
				Implémentation (programmeur) & Pas de corps pour l'opérateur (il n'y a pas d'entête) &pas de corps opérateur supporté par le langage. Présence d'un corps, redéfinition de l'opérateur\\
			\hline
		\end{tabular}
		\subsubsection{Remarque}
			\paragraph{1- } 
				On peut accorder un statut d'exportation différent selon l'opérateur. 
			\paragraph{2- } 
				L'opérateur inégalité est toujours défini implicitement à partir de l'opérateur égalité.	


\section{Type fonctionnelle V.S type impératif}
	\subsection{Type fonctionnelle}
		Type qui ne propose que des fonctions algorithmique. En général par d'affectation.  
	\subsection{Type impératif}
		Type qui possède au moins une opération de modification avec une procédure et un mode mise à jour pour une 
		variable du type. En général l'affectation est autorisée.
