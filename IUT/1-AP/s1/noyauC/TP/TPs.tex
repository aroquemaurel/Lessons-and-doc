\documentclass{article}

\usepackage{lmodern}
\usepackage{xcolor}
\usepackage[utf8]{inputenc}
\usepackage[T1]{fontenc}
\usepackage[francais]{babel}
\usepackage[top=1.7cm, bottom=1.7cm, left=1.7cm, right=1.7cm]{geometry}
%\usepackage[frenchb]{babel}
%\usepackage{layout}
%\usepackage{setspace}
%\usepackage{soul}
%\usepackage{ulem}
%\usepackage{eurosym}
%\usepackage{bookman}
%\usepackage{charter}
%\usepackage{newcent}
%\usepackage{lmodern}
%\usepackage{mathpazo}
%\usepackage{mathptmx}
%\usepackage{url}
%\usepackage{verbatim}
%\usepackage{moreverb}
%\usepackage{wrapfig}
%\usepackage{amsmath}
%\usepackage{mathrsfs}
%\usepackage{asmthm}
%\usepackage{makeidx}
\usepackage{listings}
\usepackage{fancyhdr}
\usepackage{multido}
\usepackage{amssymb}

\definecolor{gris1}{gray}{0.40}
\definecolor{gris2}{gray}{0.55}
\definecolor{gris3}{gray}{0.65}
\definecolor{gris4}{gray}{0.50}


\lstdefinelanguage{algo}{%
   morekeywords={%
    %%% couleur 1
		importer, programme, glossaire, fonction, procedure, constante, type, 
	%%% IMPORT & Co.
		si, sinon, alors, fin, tantque, debut, faire, lorsque, fin lorsque, declancher, enregistrement, tableau, retourne, retourner, =, /=, <, >, traite,exception, 
	%%% types 
		Entier, Reel, Booleen, Caractere,
	%%% types 
		entree, maj, sortie,	
	%%% types 
		et, ou, non,
	},
  sensitive=true,
  morecomment=[l]{--},
  morestring=[b]',
}

%\lstset{language=algo,
    %%% BOUCLE, TEST & Co.
%      emph={importer, programme, glossaire, fonction, procedure, constante, type},
%      emphstyle=\color{gris2},
    %%% IMPORT & Co.
%      emph={[2]si, sinon, alors, fin , tantque, debut, faire, lorsque, fin lorsque, declancher, retourner, et, ou, non,enregistrement, retourner, retourne, tableau, /=, <, =, >, traite,exception},
%      emphstyle=[2]\color{gris1},
    %%% FONCTIONS NUMERIQUES
%      emph={[3]Entier, Reel, Booleen, Caractere},
%      emphstyle=[3]\color{gris3},
    %%% FONCTIONS NUMERIQUES
%      emph={[4]entree, maj, sortie},	
%      emphstyle=[4]\color{gris4},
%}
\lstset{ % general style for listings 
   numbers=left 
	, extendedchars=\true
   , tabsize=2 
   , frame=single 
   , breaklines=true 
   , basicstyle=\ttfamily 
   , numberstyle=\tiny\ttfamily 
   , framexleftmargin=13mm 
   , xleftmargin=12mm 
   , captionpos=b 
	, language=algo
	, keywordstyle=\color{blue}
	, commentstyle=\color{green}
	, showstringspaces=false
	, extendedchars=true
	, mathescape=true
} 
 %prise en charge du langage algo

\date{Noyau C+\\ Semestre 1}
\lhead{TP : Noyau C+}
\chead{}
\rhead{\thepage}

\lfoot{Université paul sabatier Toulouse III}
\cfoot{\thepage}
\rfoot{noc1}

\pagestyle{fancy}
\begin{document}
	\maketitle
	\section{TP3}
		\subsection{Exercice 1}
			Écrire un programme qui permet à l'utilisateur de saisir la largeur et la longueur d'un champ et qui affiche le périmètre et la surface de celui-ci.
			\begin{itemize}
				\item Les variables largeur et longueur seront de type réels double précision
				\item Le calcul du périmètre se fera dans une fonction appelée perimetre, de même que le calcul de la surface se fera dans une fonction appelée surface.
			\end{itemize}		
			\lstinputlisting[caption=Périmètre et surface, language=C++]{TP1/perimetre.cpp}
		\newpage
		\subsection{Exercice 2}
			Écrire un programme qui demande à l'utilisateur de renseigner 5 entiers et qui affiche leur moyenne.
			\begin{itemize}
				\item Le programme ne devra utiliser que 2 variables.
			\end{itemize}			
			\lstinputlisting[caption=Moyenne, language=C]{TP1/moyenne.cpp}
		\subsection{Exercice 3}
			Écrire un programme qui demande à l'utilisateur de renseigner 2 entiers et qui affiche leur somme.
			\begin{itemize}
				\item Le calcul de la somme se fera dans une procédure appelée calculerSomme ayant les 2 entiers lus en entrée et l'entier résultat en sortie.
			\end{itemize}				
			\lstinputlisting[caption=Somme, language=C]{TP1/somme.cpp}
	\section{TP4}
		\subsection{Exercice 1}
			Ecrire un programme qui demande à un utilisateur de saisir un caractère et qui affiche si c'est un caractère minuscule, majuscule ou un chiffre
			\begin{itemize}
				\item La fonction estMinuscule prend un caractère en entrée et renvoie un booléen. Même chose pour les fonctions estMajuscule et estChiffre
				\item Les trois fonctions précédentes s'écrivent en une seule instruction !
			\end{itemize}
			\lstinputlisting[caption=Détermine si c'est une majuscule, minusucule ou un chiffre, language=C++]{TP2/maj.cpp}
		\subsection{Exercice 2}
			Ecrire un programme qui demande à un utilisateur de saisir un caractère et qui affiche son code en ASCII, puis qui demande à un utilisateur de saisir un code ASCII sous forme entière et affiche le caractère correspondant
			\begin{itemize}
				\item Les codes des conversions entre type caractère et type entier ne doivent pas dépasser une ligne de C++.
			\end{itemize}				
			\lstinputlisting[caption=Détermine le code ASCII, language=C++]{TP2/ascii.cpp}
		\subsection{Exercice 3}
			Ecrire une fonction changerMinusculesMajuscules qui a un paramètre de type chaîne de caractères (en mise à jour) et qui transforme toutes les minuscules de la chaîne en majuscules et toutes les majuscules en minuscules. Tester en demandant à un utilisateur de saisir une chaîne et afficher alors la chaîne modifiée.
			\begin{itemize}
				\item Vous utiliserez le type Chaine de chaine.h
				\item Les traitements sur les caractères de la chaîne seront encapsulés dans les fonctions minusculeEnMajuscule et majusculeEnMinuscule
				\item Les chaînes de caractères sont indicées de 0 à longueur(chaine) -1			
				\item Vous utiliserez une boucle for pour traiter tous les caractères de la chaîne.			
			\end{itemize}				
			\lstinputlisting[caption=fonction changerMinusculeMajuscules, language=C++]{TP2/minToMaj.cpp}
		\subsection{Exercice 4}
			Ecrire un programme qui demande à l'utilisateur de taper une chaîne de caractères et qui affiche la ou les lettres les plus fréquentes.			Ecrire un programme qui demande à l'utilisateur de taper une chaîne de caractères et qui affiche la ou les lettres les plus fréquentes.			
			\lstinputlisting[caption=Lettre la plus fréquente, language=C++]{TP2/frequente.cpp}
	\section{TP5}
		\subsection{Exercice 1}
			Ecrire un programme qui demande à un utilisateur de saisir 4 entiers A, B, C et D, représentant respectivement les intervalles d'entiers [A,B] et [C,D] puis qui affiche quelle est l'intersection des intervalles [AB] et [CD].

			\begin{itemize}
				\item Déclarez et utilisez une structure Intervalle comprenant deux entiers inf et sup.
				\item Ecrire une procédure chargée de la saisie d'un intervalle.
				\item Ecrire une procédure chargée de l'affichage d'un intervalle.
				\item Ecrire une fonction intersection prenant deux intervalles en entrée et renvoyant comme résultat l'intervalle représentant l'intersection.
				\item La constante INTERVALLE\_VIDE égale à [0,0] représentera l'intersection vide.			
			\end{itemize}		
			\lstinputlisting[caption=Intervalles, language=C++]{TP3/intervalles.cpp}
		\subsection{Exercice 2}
			Sachant que : l'eau gèle à $0^{\circ} C$, le fuel gèle à $-5^{\circ} C$, le super gèle à $-23^{\circ} C$, l'ordinaire gèle à $-13^{\circ} C$ et l'eau salée gèle à $-3^{\circ} C$. 
			Ecrire un programme qui demande à un utilisateur de saisir une température et qui affiche la liste des liquides gelés à cette température.
			Ecrire un programme qui gère un menu. L'affichage de ce menu comporte 4 choix d'actions possible. La dernière de ces actions est la sortie du programme. A chaque action sélectionnée par un utilisateur, 
			il est affiché le nom de l'action considérée, par exemple action1. Puis l'utilisateur tape sur la touche entrée et le menu est de nouveau affiché si le choix n'est pas la sortie du programme.

			\begin{itemize}
				\item La gestion des actions en fonction des choix effectués par l'utilisateur se programmera à l'aide d'une instruction switch(choix).
				\item L'affichage du menu se répètera tant que l'utilisateur n'a pas décidé de sortir du programme.
				\item La gestion des erreurs de saisie sera prise en compte. On supposera que l'utilisateur saisit toujours un entier.	
			\end{itemize}
			\lstinputlisting[caption=Gèle des liquides, language=C++]{TP3/gele.cpp}
		\subsection{Exercice 3}
			Ecrire un programme qui gère un menu. L'affichage de ce menu comporte 4 choix d'actions possible. La dernière de ces actions est la sortie du programme. A chaque action sélectionnée par un utilisateur, 
			il est affiché le nom de l'action considérée, par exemple action1. Puis l'utilisateur tape sur la touche entrée et le menu est de nouveau affiché si le choix n'est pas la sortie du programme.

			\begin{itemize}
				\item La gestion des actions en fonction des choix effectués par l'utilisateur se programmera à l'aide d'une instruction switch(choix).
				\item L'affichage du menu se répètera tant que l'utilisateur n'a pas décidé de sortir du programme.
				\item La gestion des erreurs de saisie sera prise en compte. On supposera que l'utilisateur saisit toujours un entier.	
			\end{itemize}				
			\lstinputlisting[caption=Gestion d'un menu, language=C++]{TP3/menu.cpp}					
	\section{TP6}
	 Ecrire un programme qui demande à un utilisateur de saisir un entier N et qui affiche la figure suivante.Vous utiliserez uniquement des boucles for () {...} imbriquées.
	N=1\\
	*\\
	N=2\\
	**\\
	*\\
	N=3\\
	***\\
	**\\
	*\\		
	\lstinputlisting[language=C++]{TP4/figure.cpp}		

	\section{TP7}
		\subsection{Exercice 1}
		Déclarez un tableau de 20 entiers et écrivez deux procédures permettant de renseigner et afficher les éléments de ce tableau.
		\begin{itemize}			
			\item La procédure permettant de saisir une suite de 20 valeurs entières et de les affecter à chaque poste de tableau s'appellera saisirTableau.
			\item La procédure permettant d'afficher les 20 valeurs entières contenues dans le tableau s'appelera afficherTableau.
			\item Testez vos programmes dans un programme principal
		\end{itemize}
		\lstinputlisting[language=C++]{TP5/tableau_ex1.cpp}		
		\subsection{Exercice 3 \& 4}
			fonction trierParSelection et trierABulles.
			\lstinputlisting[language=C++]{TP5/tri.cpp}		
	\section{TP8}
		définir une date sous la forme 6 janvier 2010 par une structure à trois champs. Ecrire des sous-programmes de saisie et d'affichage d'une date. Vous pourriez gérer la validité d'une date saisie ! Réutilisez cette date pour définir une personne comme une structrure composée d'un nom, d'un prénom et d'une date de naissance. Ecrire des sous-programmes de saisie et d'affichage d'une personne.
	\lstinputlisting[language=C++]{TP6/struc.cpp}		
\end{document}
