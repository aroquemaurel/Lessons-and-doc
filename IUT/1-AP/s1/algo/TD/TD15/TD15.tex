\documentclass{article}

\usepackage{lmodern}
\usepackage{xcolor}
\usepackage[utf8]{inputenc}
\usepackage[T1]{fontenc}
\usepackage[francais]{babel}
\usepackage[top=1.7cm, bottom=1.7cm, left=1.7cm, right=1.7cm]{geometry}
%\usepackage[frenchb]{babel}
%\usepackage{layout}
%\usepackage{setspace}
%\usepackage{soul}
%\usepackage{ulem}
%\usepackage{eurosym}
%\usepackage{bookman}
%\usepackage{charter}
%\usepackage{newcent}
%\usepackage{lmodern}
%\usepackage{mathpazo}
%\usepackage{mathptmx}
%\usepackage{url}
%\usepackage{verbatim}
%\usepackage{moreverb}
%\usepackage{wrapfig}
%\usepackage{amsmath}
%\usepackage{mathrsfs}
%\usepackage{asmthm}
%\usepackage{makeidx}
\usepackage{listings}
\usepackage{fancyhdr}
\usepackage{multido}
\usepackage{amssymb}

\definecolor{gris1}{gray}{0.40}
\definecolor{gris2}{gray}{0.55}
\definecolor{gris3}{gray}{0.65}
\definecolor{gris4}{gray}{0.50}


\lstdefinelanguage{algo}{%
   morekeywords={%
    %%% couleur 1
		importer, programme, glossaire, fonction, procedure, constante, type, 
	%%% IMPORT & Co.
		si, sinon, alors, fin, tantque, debut, faire, lorsque, fin lorsque, declancher, enregistrement, tableau, retourne, retourner, =, /=, <, >, traite,exception, 
	%%% types 
		Entier, Reel, Booleen, Caractere,
	%%% types 
		entree, maj, sortie,	
	%%% types 
		et, ou, non,
	},
  sensitive=true,
  morecomment=[l]{--},
  morestring=[b]',
}

%\lstset{language=algo,
    %%% BOUCLE, TEST & Co.
%      emph={importer, programme, glossaire, fonction, procedure, constante, type},
%      emphstyle=\color{gris2},
    %%% IMPORT & Co.
%      emph={[2]si, sinon, alors, fin , tantque, debut, faire, lorsque, fin lorsque, declancher, retourner, et, ou, non,enregistrement, retourner, retourne, tableau, /=, <, =, >, traite,exception},
%      emphstyle=[2]\color{gris1},
    %%% FONCTIONS NUMERIQUES
%      emph={[3]Entier, Reel, Booleen, Caractere},
%      emphstyle=[3]\color{gris3},
    %%% FONCTIONS NUMERIQUES
%      emph={[4]entree, maj, sortie},	
%      emphstyle=[4]\color{gris4},
%}
\lstset{ % general style for listings 
   numbers=left 
	, extendedchars=\true
   , tabsize=2 
   , frame=single 
   , breaklines=true 
   , basicstyle=\ttfamily 
   , numberstyle=\tiny\ttfamily 
   , framexleftmargin=13mm 
   , xleftmargin=12mm 
   , captionpos=b 
	, language=algo
	, keywordstyle=\color{blue}
	, commentstyle=\color{green}
	, showstringspaces=false
	, extendedchars=true
	, mathescape=true
} 
 %prise en charge du langage algo

\title{TD 15\\ Tri par insertion}
\date{Algorithmique\\ Semestre 1}

\lhead{TD 15: Tri par insertion}
\chead{}
\rhead{\thepage}

\lfoot{Université paul sabatier Toulouse III}
\cfoot{\thepage}
\rfoot{Alg1}

\pagestyle{fancy}
\begin{document}
	\maketitle
	\section{Tri par insértion d'une suite de valeurs}
		\subsection{}
			\lstinputlisting[caption=En-tête de trierSuiteParInsertion]{1-1.algo}
		\subsection{}
			\lstinputlisting[caption=Algorithme général de la procédure trierSuiteParInsertion]{1-2.algo}
		\subsection{}
			\lstinputlisting[caption=Procédure trierSuiteParInsertion]{1-3.algo}
		\subsection{}
			\lstinputlisting[caption=Programme triSuiteParInsertion]{1-4.algo}
	\section{Tri par insertion d'un tableau}
		\subsection{}
			\lstinputlisting[caption=Entête de trierTableauParInsertion]{2-1.algo}
		\subsection{}
			\lstinputlisting[caption=Procédure trierTableauParInsertion]{2-2.algo}
		\subsection{trace}	
			\begin{tabular}{|c|c|c|c|c|c|c|c|c|}
				\hline
					\textbf{Situation} & \textbf{i} & \textbf{1} & \textbf{2} & \textbf{3}
						 & \textbf{4} & \textbf{5} & \textbf{6} & \textbf{7}\\
				\hline 
					1 & 2 & 10 & 25 & 40 & 30 & 50 & 5& 20\\
				\hline
					2 & 2& 10 & 25 & 40 & 30 & 50 & 5 & 20\\
				\hline
					3 & 3& 10 & 25& 40& 30& 50& 5& 20\\
				\hline
					2 & 3& 10 & 25& 40& 30& 50 &5& 20\\
				\hline
					3  & 1 & 10 & 02 &10 & 30 & 50 & 5 & 20\\
				\hline
					2 & 4 & 10 & 25 & 30 & 40 & 50 & 5 & 20\\
				\hline
					3 & 5& 10 & 25 & 30 & 40 & 50 & 5 & 20\\ 
				\hline
					2& 5& 10 & 25 & 30 & 40 & 50 & 5 & 20\\
				\hline
					3 & 6 & 10 & 25 & 30 & 40 & 50 & 5 & 20\\
				\hline
					2& 6 & 5 & 10 & 25 & 30 & 40 & 50&  20\\
				\hline
			\end{tabular}						
				

\end{document}
