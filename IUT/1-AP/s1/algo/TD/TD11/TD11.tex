\documentclass{article}

\usepackage{lmodern}
\usepackage{xcolor}
\usepackage[utf8]{inputenc}
\usepackage[T1]{fontenc}
\usepackage[francais]{babel}
\usepackage[top=1.7cm, bottom=1.7cm, left=1.7cm, right=1.7cm]{geometry}
%\usepackage[frenchb]{babel}
%\usepackage{layout}
%\usepackage{setspace}
%\usepackage{soul}
%\usepackage{ulem}
%\usepackage{eurosym}
%\usepackage{bookman}
%\usepackage{charter}
%\usepackage{newcent}
%\usepackage{lmodern}
%\usepackage{mathpazo}
%\usepackage{mathptmx}
%\usepackage{url}
%\usepackage{verbatim}
%\usepackage{moreverb}
%\usepackage{wrapfig}
%\usepackage{amsmath}
%\usepackage{mathrsfs}
%\usepackage{asmthm}
%\usepackage{makeidx}
\usepackage{listings}
\usepackage{fancyhdr}
\usepackage{multido}
\usepackage{amssymb}

\definecolor{gris1}{gray}{0.40}
\definecolor{gris2}{gray}{0.55}
\definecolor{gris3}{gray}{0.65}
\definecolor{gris4}{gray}{0.50}


\lstdefinelanguage{algo}{%
   morekeywords={%
    %%% couleur 1
		importer, programme, glossaire, fonction, procedure, constante, type, 
	%%% IMPORT & Co.
		si, sinon, alors, fin, tantque, debut, faire, lorsque, fin lorsque, declancher, enregistrement, tableau, retourne, retourner, =, /=, <, >, traite,exception, 
	%%% types 
		Entier, Reel, Booleen, Caractere,
	%%% types 
		entree, maj, sortie,	
	%%% types 
		et, ou, non,
	},
  sensitive=true,
  morecomment=[l]{--},
  morestring=[b]',
}

%\lstset{language=algo,
    %%% BOUCLE, TEST & Co.
%      emph={importer, programme, glossaire, fonction, procedure, constante, type},
%      emphstyle=\color{gris2},
    %%% IMPORT & Co.
%      emph={[2]si, sinon, alors, fin , tantque, debut, faire, lorsque, fin lorsque, declancher, retourner, et, ou, non,enregistrement, retourner, retourne, tableau, /=, <, =, >, traite,exception},
%      emphstyle=[2]\color{gris1},
    %%% FONCTIONS NUMERIQUES
%      emph={[3]Entier, Reel, Booleen, Caractere},
%      emphstyle=[3]\color{gris3},
    %%% FONCTIONS NUMERIQUES
%      emph={[4]entree, maj, sortie},	
%      emphstyle=[4]\color{gris4},
%}
\lstset{ % general style for listings 
   numbers=left 
	, extendedchars=\true
   , tabsize=2 
   , frame=single 
   , breaklines=true 
   , basicstyle=\ttfamily 
   , numberstyle=\tiny\ttfamily 
   , framexleftmargin=13mm 
   , xleftmargin=12mm 
   , captionpos=b 
	, language=algo
	, keywordstyle=\color{blue}
	, commentstyle=\color{green}
	, showstringspaces=false
	, extendedchars=true
	, mathescape=true
} 
 %prise en charge du langage algo

\title{TD 11\\ Recherche séquentielle }
\date{Algorithmique\\ Semestre 1}

\lhead{TD 11: Recherche séquentielle}
\chead{}
\rhead{\thepage}

\lfoot{Université paul sabatier Toulouse III}
\cfoot{\thepage}
\rfoot{Alg1}

\pagestyle{fancy}

\begin{document}
	\maketitle
	\section{Sous-programme incideMinimum}
		\subsection{En-tête du sous-programme}
			\lstinputlisting[caption=En-tête de incideMinimum]{1-1.algo}
		\subsection{Corps du sous-programme}
			\lstinputlisting[caption=corps de indiceMinimum]{1-2.algo}
	\section{Sous-programme rechercherOccurence}		
		\subsection{En-tête du sous-programme}
			\lstinputlisting[caption=Entête de la procédure rechercherOccurence]{2-1.algo}
		\subsection{Préconditions}
			\lstinputlisting[caption=Entête de la procédure rechercherOccurence avec les préconditions]{2-2.algo}
		\subsection{Postconditions}
			a) Ne retourne pas faux si x non trouvé. \\ Ne garantit pas que ça soit la première occurence\\ \\
			b) Ne retourne pas faux mais garanti que ça soit la première occurence \\ \\
			c) Implique l'unciité de l'élément recherché \\ \\
			d) Post-condition\\ \\
			e) $tab[i] = x$ implique que tous les les éléménts soient égales à $x$
		\subsection{Corps du sous-programme}
			\lstinputlisting[caption=Corps de la procédure rechercherOccurence (version 1)]{2-4_v1.algo}
			\lstinputlisting[caption=Corps de la procédure rechercherOccurence (version 2)]{2-4_v2.algo}

		\subsection{Exemple d'appel du sous-programme}
			\subsubsection{}	
				C'est un tableau
			\subsubsection{}	
				\lstinputlisting[caption=Programme]{2-5-2.algo}
\end{document}


