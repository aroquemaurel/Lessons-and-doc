\chapter{Les classes et les objets}
\paragraph{Clef moodle} $\$$noj11b$\$$\\
2 facteurs de qualité en génie logiciel
	\paragraph{Extensibilité} Capacité pour un logiciel à intégrer de nouvelles fonctionnalités
	\paragraph{Réutilisabilité} Capacité pour un logiciel à être réexploité, tout en partie, 
		pour de nouvelles applications
\section{L'objet}
	\paragraph{Définition} Structure de données présente à l'exécution formée de champs et 
	d'opérations applicable à ces champs.\\
	Un objet est une identité, un état et un comportement

	\paragraph{État} Ensemble des valeurs des champs de l'objet
	\paragraph{Comportement} L'ensemble de ces opérations
\section{La classe}	
	\paragraph{Définition} Description d'une famille d'objets ayant même comportement. \\
	Deux composantes dans une classe:
	\begin{itemize}
		\item La description des données appelés \textbf{attributs}
		\item La description des opérations appelés \textbf{méthodes}
	\end{itemize}
	\newpage
	\subsection{Exemple}
	La classe \textit{article} associée aux des objets \textit{laLacoste} et \textit{laBadoit}
	\lstinputlisting[language=java]{ex1.java}	
	\subsection{Remarques}
	\paragraph{1} Les méthodes de la classe Article (getPrixHT, setPrixHT, \ldots) ne possèdent
	pas en paramètre un article
	\paragraph{2 Principe de l'encapsulation} une classe n'exporte (mot-clé \textbf{public}) 
	que ses services (méthode applicable à un article). Tout ce qui n'est pas exporté est 
	considéré privé (mot-clé \textbf{private})

	\paragraph{3} L'ensemble des services d'une classe (ses méthodes publiques) sont spécifiés 
	par une interface.
\section{Les objets instances de classe}
	Classe = modèle \\
	Objet = Représentant de ce modèle (appelé l'\textbf{instance})
	\newpage
	\subsection{Relation d'instanciation}
		Relation entre la classe et l'objet. \\
		écriture d'une méthode spécifique à la classe appelée \textbf{constructeur} 
		(création d'une instance à partir du modèle)
		\paragraph{Convention} le nom du constructeur est le nom de la classe et ne renvoit
		aucun résultat.
		\lstinputlisting[language=java]{ex2.java}	
	\subsection{Création d'objet}
		Pour créer un objet, on invoque le constructeur par l'opération \textbf{new}
		\subsubsection{Exemple}
		\lstinputlisting[language=java]{ex3.java}	
		\subsubsection{Remarques}
		\begin{itemize}
			\item Classe = Type	
			\item Classe = Entité statique et Objet = Entité statique (créé par l'opérateur \textbf{new})
			\item Une classe comprend en général les méthodes suivantes: 
				\begin{itemize}
					\item Un (ou plusieurs) constructeur 
					\item Des destructeurs (pour récupérer la mémoire des objets)
					\item Des sélecteurs ou opération de consultation pour accéder aux champs 
							de l'objet (souvent préfixés par get)
					\item Des modificateurs pour modifier l'état d'un objet (préfixés par set)
					\item Des itérateurs permettant de balayer une collection d'attributs
				\end{itemize}
			\item L'interface de la classe Article s'enrichit du constructeur 
		\end{itemize}

