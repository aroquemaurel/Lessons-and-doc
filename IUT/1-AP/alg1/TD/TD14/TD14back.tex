\documentclass{article}

\usepackage{lmodern}
\usepackage{xcolor}
\usepackage[utf8]{inputenc}
\usepackage[T1]{fontenc}
\usepackage[francais]{babel}
\usepackage[top=1.7cm, bottom=1.7cm, left=1.7cm, right=1.7cm]{geometry}
%\usepackage[frenchb]{babel}
%\usepackage{layout}
%\usepackage{setspace}
%\usepackage{soul}
%\usepackage{ulem}
%\usepackage{eurosym}
%\usepackage{bookman}
%\usepackage{charter}
%\usepackage{newcent}
%\usepackage{lmodern}
%\usepackage{mathpazo}
%\usepackage{mathptmx}
%\usepackage{url}
%\usepackage{verbatim}
%\usepackage{moreverb}
%\usepackage{wrapfig}
%\usepackage{amsmath}
%\usepackage{mathrsfs}
%\usepackage{asmthm}
%\usepackage{makeidx}
\usepackage{listings}
\usepackage{fancyhdr}
\usepackage{multido}
\usepackage{amssymb}

\definecolor{gris1}{gray}{0.40}
\definecolor{gris2}{gray}{0.55}
\definecolor{gris3}{gray}{0.65}
\definecolor{gris4}{gray}{0.50}


\lstdefinelanguage{algo}{%
   morekeywords={%
    %%% couleur 1
		importer, programme, glossaire, fonction, procedure, constante, type, 
	%%% IMPORT & Co.
		si, sinon, alors, fin, tantque, debut, faire, lorsque, fin lorsque, declancher, enregistrement, tableau, retourne, retourner, =, /=, <, >, traite,exception, 
	%%% types 
		Entier, Reel, Booleen, Caractere,
	%%% types 
		entree, maj, sortie,	
	%%% types 
		et, ou, non,
	},
  sensitive=true,
  morecomment=[l]{--},
  morestring=[b]',
}

%\lstset{language=algo,
    %%% BOUCLE, TEST & Co.
%      emph={importer, programme, glossaire, fonction, procedure, constante, type},
%      emphstyle=\color{gris2},
    %%% IMPORT & Co.
%      emph={[2]si, sinon, alors, fin , tantque, debut, faire, lorsque, fin lorsque, declancher, retourner, et, ou, non,enregistrement, retourner, retourne, tableau, /=, <, =, >, traite,exception},
%      emphstyle=[2]\color{gris1},
    %%% FONCTIONS NUMERIQUES
%      emph={[3]Entier, Reel, Booleen, Caractere},
%      emphstyle=[3]\color{gris3},
    %%% FONCTIONS NUMERIQUES
%      emph={[4]entree, maj, sortie},	
%      emphstyle=[4]\color{gris4},
%}
\lstset{ % general style for listings 
   numbers=left 
	, extendedchars=\true
   , tabsize=2 
   , frame=single 
   , breaklines=true 
   , basicstyle=\ttfamily 
   , numberstyle=\tiny\ttfamily 
   , framexleftmargin=13mm 
   , xleftmargin=12mm 
   , captionpos=b 
	, language=algo
	, keywordstyle=\color{blue}
	, commentstyle=\color{green}
	, showstringspaces=false
	, extendedchars=true
	, mathescape=true
} 
 %prise en charge du langage algo

\title{TD 14\\ Tri à bulles (Bubble sort)}
\date{Algorithmique\\ Semestre 1}

\lhead{TD 14: Tri à bulles (Bubble sort)}
\chead{}
\rhead{\thepage}

\lfoot{Université paul sabatier Toulouse III}
\cfoot{\thepage}
\rfoot{Alg1}

\pagestyle{fancy}

\begin{document}
	\maketitle
	\section{}
		\subsection{Algo niveau 1}
			\lstinputlisting[caption=sous-programme ramenerValeurMax Niveau 1]{1-niveau1.algo}
		\subsection{Algo niveau 2}
			\lstinputlisting[caption=sous-programme ramenerValeurMax Niveau 2]{1-niveau2.algo}
		\subsection{Trace du sous-programme}
			\begin{tabular}{|c|c|c|c|c|c|}
				\hline
					2 & -12 & 7 & 3 & 7 & 5\\
				\hline
			\end{tabular}\\ \\ 
			\begin{tabular}{|c|c|c|c|c|c|}
				\hline
					-12 & 2 & 7 & 3 & 7 & 5\\
				\hline
			\end{tabular}\\ \\ 
			\begin{tabular}{|c|c|c|c|c|c|}
				\hline
					-12 & 2 & 7 & 3 & 7 & 5\\
				\hline
			\end{tabular}\\ \\ 
			\begin{tabular}{|c|c|c|c|c|c|}
				\hline
					-12 & 2 & 3 & 7 & 7 & 5\\
				\hline
			\end{tabular}\\ \\ 
			\begin{tabular}{|c|c|c|c|c|c|}
				\hline
					-12 & 2 & 3 & 7 & 7 & 5\\
				\hline
			\end{tabular}\\ \\ 
			\begin{tabular}{|c|c|c|c|c|c|}
				\hline
					-12 & 2 & 3 & 7 & 5 & 5\\
				\hline
			\end{tabular}\\ \\ 
			\begin{tabular}{|c|c|c|c|c|c|}
				\hline
					2 & -12 & 7 & 3 & 7 & 5\\
				\hline
			\end{tabular} 
		\section{En-tête de ramenerValeurMax}	
			\lstinputlisting[caption=En-tête de ramenerValeurMax]{1-entete.algo}
		\subsection{Trace}		
			\begin{tabular}{|c|c|c|c|c|c|c|c|c|c|c|c|c|c|}
				\hline
					\textbf{situation} & 
					\textbf{1} &
					\textbf{2} &
					\textbf{3} &
					\textbf{4} &
					\textbf{5} &
					\textbf{6} &
					\textbf{7} &
					\textbf{...} &
					\textbf{$n$} &
					\textbf{$i$} &
					\textbf{$aux$} &
					\textbf{$i \ne n$} &
					\textbf{$tab[i] > tab[i+1]$} \\
				\hline
					0 & %situation
					2 & %1
					-12 &%2
					7 &%3
					3 &%4
					5 &%5
					7 &%6
					? &%7
					&%...
					6 &%n
					?& %i
					? &%aux
					?& %i /= n
					i\\%tab[i] > tab[i+n]
				\hline
					1 & %situation
					 & %1
					 &%2
					 &%3
					 &%4
					 &%5
					 &%6
					 &%7
					 &%...
					 &%n
					1& %i
					 &%aux
					& %i /= n
					\\%tab[i] > tab[i+n]
				\hline
					2 & %situation
					 & %1
					 &%2
					 &%3
					 &%4
					 &%5
					 &%6
					 &%7
					 &%...
					 &%n
					& %i
					 &%aux
					$1\ne 6$ %i /= n
					\\%tab[i] > tab[i+n]
				\hline
					 3& %situation
					 & %1
					 &%2
					 &%3
					 &%4
					 &%5
					 &%6
					 &%7
					 &%...
					 &%n
					& %i
					 &%aux
					?& %i /= n
					2 > -12 \\%tab[i] > tab[i+n]
				\hline
					 4& %situation
					 -12& %1
					 &%2
					 &%3
					 &%4
					 &%5
					 &%6
					 &%7
					 &%...
					 &%n
					& %i
					2 &%aux
					& %i /= n
					? \\%tab[i] > tab[i+n]
				\hline
					 5& %situation
					 & %1
					 &%2
					 &%3
					 &%4
					 &%5
					 &%6
					 &%7
					 &%...
					 &%n
					2& %i
					 &%aux
					& %i /= n
					\\
					 %tab[i] > tab[i+n]
				\hline
					 2& %situation
					 & %1
					 &%2
					 &%3
					 &%4
					 &%5
					 &%6
					 &%7
					 &%...
					 &%n
					& %i
					 &%aux
					$2 \ne 6$& %i /= n
					 \\%TAB[I] > tab[i+n]
				\hline
					 3& %situation
					 & %1
					 &%2
					 &%3
					 &%4
					 &%5
					 &%6
					 &%7
					 &%...
					 &%n
					& %i
					 &%aux
					& %i /= n
					$2 \ne 7$ \\%tab[i] > tab[i+n]
				\hline
					 5& %situation
					 & %1
					 &%2
					 &%3
					 &%4
					 &%5
					 &%6
					 &%7
					 &%...
					 &%n
					& %i
					 &%aux
					$3 \ne 6$& %i /= n
					 \\%tab[i] > tab[i+n]
				\hline
					2 & %situation
					 & %1
					 &%2
					 &%3
					 &%4
					 &%5
					 &%6
					 &%7
					 &%...
					 &%n
					& %i
					 &%aux
					& %i /= n
					7> 3 \\%tab[i] > tab[i+n]
				\hline
					 3& %situation
					 & %1
					 &%2
					 3&%3
					 7&%4
					 &%5
					 &%6
					 &%7
					 &%...
					 &%n
					& %i
					7 &%aux
					& %i /= n
					 \\%tab[i] > tab[i+n]
				\hline
					 4& %situation
					 & %1
					 &%2
					 &%3
					 &%4
					 &%5
					 &%6
					 &%7
					 &%...
					 &%n
					4& %i
					 &%aux
					& %i /= n
					 \\%tab[i] > tab[i+n]
				\hline
					 5& %situation
					 & %1
					 &%2
					 &%3
					 &%4
					 &%5
					 &%6
					 &%7
					 &%...
					 &%n
					& %i
					 &%aux
					$4 \ne 6$& %i /= n
					 \\%tab[i] > tab[i+n]
				\hline
					 2& %situation
					 & %1
					 &%2
					 &%3
					 &%4
					 &%5
					 &%6
					 &%7
					 &%...
					 &%n
					& %i
					 &%aux
					& %i /= n
					 \\%tab[i] > tab[i+n]
				\hline
					 3& %situation
					 & %1
					 &%2
					 &%3
					 &%4
					 &%5
					 &%6
					 &%7
					 &%...
					 &%n
					5& %i
					 &%aux
					& %i /= n
					 \\%tab[i] > tab[i+n]
				\hline
					 5& %situation
					 & %1
					 &%2
					 &%3
					 &%4
					 &%5
					 &%6
					 &%7
					 &%...
					 &%n
					& %i
					 &%aux
					$5 \ne 6$& %i /= n
					 \\%tab[i] > tab[i+n]
				\hline
					 2& %situation
					 & %1
					 &%2
					 &%3
					 &%4
					 &%5
					 &%6
					 &%7
					 &%...
					 &%n
					& %i
					 &%aux
					& %i /= n
					7>5 \\%tab[i] > tab[i+n]
				\hline
					 3& %situation
					 & %1
					 &%2
					 &%3
					 &%4
					 5&%5
					 7&%6
					 &%7
					 &%...
					 &%n
					& %i
					7 &%aux
					& %i /= n
					 \\%tab[i] > tab[i+n]
				\hline
					 4& %situation
					 & %1
					 &%2
					 &%3
					 &%4
					 &%5
					 &%6
					 &%7
					 &%...
					 &%n
					6& %i
					 &%aux
					& %i /= n
					 \\%tab[i] > tab[i+n]
				\hline
					 5& %situation
					 & %1
					 &%2
					 &%3
					 &%4
					 &%5
					 &%6
					 &%7
					 &%...
					 &%n
					& %i
					 &%aux
					$6 \ne 6$& %i /= n
					 \\%tab[i] > tab[i+n]
				\hline
					 2& %situation
					 -12& %1
					 2&%2
					 3&%3
					 7&%4
					 5&%5
					 7&%6
					 &%7
					 &%...
					 &%n
					& %i
					 &%aux
					& %i /= n
					 \\%tab[i] > tab[i+n]
				\hline
							
					\end{tabular} 
					%6						
	\subsection{}
			\lstinputlisting[caption=sous-programme ramenerValeurMax Niveau 1]{2.algo}
	\section{}
			\lstinputlisting[caption=sous-programme ramenerValeurMax Niveau 1]{3.algo}
	\section{}
		\subsection{Algo général}
			\lstinputlisting[caption=Algorithme général du tri à bulles]{4-1.algo}
		\subsection{Programme}	
			\lstinputlisting[caption=Programme trierParBulles]{4-2.algo}
		\section{}
			\begin{tabular}{|c|c|c|c|c|c|c|c|}
				\hline
					\textbf{Situations} &
					\textbf{nbElements} &
					\textbf{1} &
					\textbf{2} &
					\textbf{3} &
					\textbf{4} &
					\textbf{5} &
					\textbf{6}\\ 
				\hline
					1 & 6 & 2 &-12 & 7 & 3 & 7 & 5\\
				\hline
					2 & 6 & -12 &2 & 3 & 7 & 7 & 7\\
				\hline
					3 & 5 & -12 &2 & 3 & 7 & 7 & 7\\
				\hline
					2 & 5 & -12 &2 & 3 & 3 & 5 & 7\\
				\hline
			\end{tabular}
		\section{}	
			$$ \forall i \in [1,n-1], tab[i] < tab[i+1]$$ 
		\section{Comparaisons}
			$$(n-1)+(1-2)+(1-3)+\dots+1=\sum^{n-1}_{c=1}{(n-1)}$$
			$$=\sum^{n-1}_{i=n}-\sum^{n-1}_{i=1}{i} = \frac{(n-1)(n+2)}{2} - (n-1) = $$
			$$\frac{n(n-1)}{2}=O(n^{2})$$
			Donc complexisté en $O(n^{2})$
		\begin{eqnarray*}
			(n-1)+(1-2)+(1-3)+\dots+1&=&\sum^{n-1}_{c=1}{(n-1)} \\
			&=&\sum^{n-1}_{i=n}-\sum^{n-1}_{i=1}{i} \\
			&=& \frac{(n-1)(n+2)}{2} - (n-1) = 
			\frac{n(n-1)}{2}=O(n^{2})
		\end{eqnarray*} 
	
	\end{document}
