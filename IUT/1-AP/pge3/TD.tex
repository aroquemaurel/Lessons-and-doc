\documentclass[12pt,a4paper,openany]{report}

\usepackage{lmodern}
\usepackage{xcolor}
\usepackage[utf8]{inputenc}
\usepackage[T1]{fontenc}
\usepackage[francais]{babel}
\usepackage[top=1.7cm, bottom=1.7cm, left=1.7cm, right=1.7cm]{geometry}
\usepackage{verbatim}
\usepackage{tikz} %Vectoriel
\usepackage{listings}
\usepackage{fancyhdr}
\usepackage{multido}
\usepackage{amssymb}

\newcommand{\titre}{Programmation  événementielle}

\newcommand{\module}{Programmation événementielle}
\newcommand{\sigle}{pge}

\newcommand{\semestre}{3}

\input{/home/satenske/cours/listings.tex} %prise en charge du langage algo
\input{/home/satenske/cours/entete_iut-cours.tex}

\begin{document}
	\maketitle
	\section{Vocabulaire}
	\paragraph{widget} = composants graphiques de l'interface (exemple bouton)
	\paragraph{sélécteur} = bouton radio (l'un ou l'autre)
	\paragraph{interrupteurs} (checkboxe) = plusieurs choix possible
	\paragraph{combo} = liste déroulante, un choix possible
	\paragraph{zone de texte/spin} = pour taper du texte
	\section{Ergnomie}
	\paragraph{Affordance} Propriétés physique d'un objet qui font penser à des actions qu'on pourrait faire dessus.
	S'il faut lire un manuel, ce n'est pas affordant. 
	\section{Développement d'ihm en 4 étapes}
	\begin{enumerate}
		\item Aspect dynamique => Diagramme états-transitions
		\item Aspect statique => fenêtre et widgets
		\item Programmation => Code windeb
		\begin{enumerate}
			\item Déclarations globale de la fenêtre
			\item Initialisation de la fenetre
			\item  Génération d'événements
			\item  Procédure GDE
		\end{enumerate}
	\end{enumerate}
\end{document}







