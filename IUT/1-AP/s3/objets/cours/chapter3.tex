\chapter{L'envoi de message et le polymorphisme}
Envoie de message demandé par un objet de l'exécution d'un service de la classe. \\
C'est l'unique de manière de dialoguer entre objets.

Envoi de message = (Objet receveur, sélecteur de méthode, une liste de paramètre)

\section{L'envoi de message}
\subsection{Cas général}
L'envoi de message est réalisé par la primitive \textbf{send} ou la notation pointée

\subsubsection{Exemple}
Soit l'objet laLacoste, instance de la classe chemise.\\
prix = laLacoste.prixDeVente();\\
Avec un prixHT de 50\euro{}, prix = 59.8euros (car TVA = 19.6\%)

\paragraph{}Soit a de type Article et prix = a.prixDeVente()
Si a à l'exécution désigne une chemise (instance de la classe Chemise). \\
prix = 59.8 euros (TVA = 19.6)
\lstinputlisting[language=java]{ex8.java}
\subsection{Désignation de l'objet receveur (objet self)}
Dans les exemples de la section 3-1-1, on connait l'identité de l'objet receveur.
L'objet laLacoste, l'objet a, l'objet out\ldots\\
lorsque l'objet receveur n'est pas connu, on le désigne par l'objet courant receveur du message:
l'objet \textbf{self} (en Java, self = \textbf{this}).\\
Cas code d'un code d'une classe = code de l'instance (l'objet) courant

\paragraph{Exemple}Dans la classe Article, ajout d'une méthode prixAvecLivraison = prixDeVente + coutLIvraison\\
Le même code doit s'appliquer à une chemise avec un taux de TVA = 19.6\% et une eau gazeuse avec TVA = 5.5\%
\lstinputlisting[language=java]{ex9.java}
\paragraph{Remarque}
Cet objet courant \textbf{this} est implicite dans la première version de la classe Article
(chapitre 1 section 2).

\subsection{Accès à la superméthode (mot-clé \textbf{super})}
Héritage par substitution $\Rightarrow$ la méthode héritée est masquée par la redéfinition. \\
La méthode masquée est appelée superméthode.
\lstinputlisting[language=java]{ex8.java}
\paragraph{pb} Comment réutiliser le code de la superméthode ? 
\paragraph{} On utilise la superméthode par un envoi de message en utilisant le mot clé \textbf{super}
(qui n'est pas un objet)
\paragraph{Exemple} Dans la classe Chaine-hifi, redéfinition de la méthode prixDeVente définie par :
$$prixDeVente_{materielHifi} = prixDeVente_{Article} + coutGarantie_{MaterielHifi}$$

\paragraph{Remarque} Attention à l'usage de \textbf{super} ! Un envoi de message avec \textbf{super}
ne doit s'appliquer pour l'accès d'une superméthode masquée par l'héritage. Dans les autres cas utiliser
\textbf{this}.
\lstinputlisting[language=java]{ex10.java}
\subsection{La délégation}
Mécanisme de communication entre objets (par envoi de message) qui permet à un objet composite
de déléguer à un de ses composant la requête à satisfaire.
\paragraph{Exemple} Un produit laitier confie à un de ses composants conditionnement le soi d'emballer le produit.
Pour mettre en place la relation non structurelle, on communique à l'emballage la référence \textbf{this} du produit laitier.
\begin{center}
% Graphic for TeX using PGF
% Title: /home/satenske/cours/AP/obj3/uml20.dia
% Creator: Dia v0.97.1
% CreationDate: Thu Sep 29 10:17:27 2011
% For: satenske
% \usepackage{tikz}
% The following commands are not supported in PSTricks at present
% We define them conditionally, so when they are implemented,
% this pgf file will use them.
\ifx\du\undefined
  \newlength{\du}
\fi
\setlength{\du}{15\unitlength}
\begin{tikzpicture}
\pgftransformxscale{1.000000}
\pgftransformyscale{-1.000000}
\definecolor{dialinecolor}{rgb}{0.000000, 0.000000, 0.000000}
\pgfsetstrokecolor{dialinecolor}
\definecolor{dialinecolor}{rgb}{1.000000, 1.000000, 1.000000}
\pgfsetfillcolor{dialinecolor}
\pgfsetlinewidth{0.100000\du}
\pgfsetdash{}{0pt}
\definecolor{dialinecolor}{rgb}{1.000000, 1.000000, 1.000000}
\pgfsetfillcolor{dialinecolor}
\fill (9.500000\du,-10.487500\du)--(9.500000\du,-9.087500\du)--(16.252500\du,-9.087500\du)--(16.252500\du,-10.487500\du)--cycle;
\definecolor{dialinecolor}{rgb}{0.000000, 0.000000, 0.000000}
\pgfsetstrokecolor{dialinecolor}
\draw (9.500000\du,-10.487500\du)--(9.500000\du,-9.087500\du)--(16.252500\du,-9.087500\du)--(16.252500\du,-10.487500\du)--cycle;
% setfont left to latex
\definecolor{dialinecolor}{rgb}{0.000000, 0.000000, 0.000000}
\pgfsetstrokecolor{dialinecolor}
\node at (12.876250\du,-9.537500\du){ProduitLaitier};
\definecolor{dialinecolor}{rgb}{1.000000, 1.000000, 1.000000}
\pgfsetfillcolor{dialinecolor}
\fill (9.500000\du,-9.087500\du)--(9.500000\du,-8.687500\du)--(16.252500\du,-8.687500\du)--(16.252500\du,-9.087500\du)--cycle;
\definecolor{dialinecolor}{rgb}{0.000000, 0.000000, 0.000000}
\pgfsetstrokecolor{dialinecolor}
\draw (9.500000\du,-9.087500\du)--(9.500000\du,-8.687500\du)--(16.252500\du,-8.687500\du)--(16.252500\du,-9.087500\du)--cycle;
\definecolor{dialinecolor}{rgb}{1.000000, 1.000000, 1.000000}
\pgfsetfillcolor{dialinecolor}
\fill (9.500000\du,-8.687500\du)--(9.500000\du,-8.287500\du)--(16.252500\du,-8.287500\du)--(16.252500\du,-8.687500\du)--cycle;
\definecolor{dialinecolor}{rgb}{0.000000, 0.000000, 0.000000}
\pgfsetstrokecolor{dialinecolor}
\draw (9.500000\du,-8.687500\du)--(9.500000\du,-8.287500\du)--(16.252500\du,-8.287500\du)--(16.252500\du,-8.687500\du)--cycle;
\pgfsetlinewidth{0.100000\du}
\pgfsetdash{}{0pt}
\definecolor{dialinecolor}{rgb}{1.000000, 1.000000, 1.000000}
\pgfsetfillcolor{dialinecolor}
\fill (30.880000\du,-10.472500\du)--(30.880000\du,-9.072500\du)--(39.152500\du,-9.072500\du)--(39.152500\du,-10.472500\du)--cycle;
\definecolor{dialinecolor}{rgb}{0.000000, 0.000000, 0.000000}
\pgfsetstrokecolor{dialinecolor}
\draw (30.880000\du,-10.472500\du)--(30.880000\du,-9.072500\du)--(39.152500\du,-9.072500\du)--(39.152500\du,-10.472500\du)--cycle;
% setfont left to latex
\definecolor{dialinecolor}{rgb}{0.000000, 0.000000, 0.000000}
\pgfsetstrokecolor{dialinecolor}
\node at (35.016250\du,-9.522500\du){Conditionnement};
\definecolor{dialinecolor}{rgb}{1.000000, 1.000000, 1.000000}
\pgfsetfillcolor{dialinecolor}
\fill (30.880000\du,-9.072500\du)--(30.880000\du,-8.672500\du)--(39.152500\du,-8.672500\du)--(39.152500\du,-9.072500\du)--cycle;
\definecolor{dialinecolor}{rgb}{0.000000, 0.000000, 0.000000}
\pgfsetstrokecolor{dialinecolor}
\draw (30.880000\du,-9.072500\du)--(30.880000\du,-8.672500\du)--(39.152500\du,-8.672500\du)--(39.152500\du,-9.072500\du)--cycle;
\definecolor{dialinecolor}{rgb}{1.000000, 1.000000, 1.000000}
\pgfsetfillcolor{dialinecolor}
\fill (30.880000\du,-8.672500\du)--(30.880000\du,-8.272500\du)--(39.152500\du,-8.272500\du)--(39.152500\du,-8.672500\du)--cycle;
\definecolor{dialinecolor}{rgb}{0.000000, 0.000000, 0.000000}
\pgfsetstrokecolor{dialinecolor}
\draw (30.880000\du,-8.672500\du)--(30.880000\du,-8.272500\du)--(39.152500\du,-8.272500\du)--(39.152500\du,-8.672500\du)--cycle;
\pgfsetlinewidth{0.100000\du}
\pgfsetdash{}{0pt}
\definecolor{dialinecolor}{rgb}{1.000000, 1.000000, 1.000000}
\pgfsetfillcolor{dialinecolor}
\fill (16.700000\du,-0.237500\du)--(27.395000\du,-0.237500\du)--(27.995000\du,0.362500\du)--(27.995000\du,1.462500\du)--(16.700000\du,1.462500\du)--cycle;
\definecolor{dialinecolor}{rgb}{0.000000, 0.000000, 0.000000}
\pgfsetstrokecolor{dialinecolor}
\draw (16.700000\du,-0.237500\du)--(27.395000\du,-0.237500\du)--(27.995000\du,0.362500\du)--(27.995000\du,1.462500\du)--(16.700000\du,1.462500\du)--cycle;
\pgfsetlinewidth{0.050000\du}
\definecolor{dialinecolor}{rgb}{0.000000, 0.000000, 0.000000}
\pgfsetstrokecolor{dialinecolor}
\draw (27.395000\du,-0.237500\du)--(27.395000\du,0.362500\du)--(27.995000\du,0.362500\du);
% setfont left to latex
\definecolor{dialinecolor}{rgb}{0.000000, 0.000000, 0.000000}
\pgfsetstrokecolor{dialinecolor}
\node[anchor=west] at (17.050000\du,1.007500\du){dépendance non structurelle};
\pgfsetlinewidth{0.100000\du}
\pgfsetbuttcap
\pgfsetdash{}{0pt}
{
\definecolor{dialinecolor}{rgb}{0.000000, 0.000000, 0.000000}
\pgfsetfillcolor{dialinecolor}
% was here!!!
\pgfsetarrowsend{to}
\definecolor{dialinecolor}{rgb}{0.000000, 0.000000, 0.000000}
\pgfsetstrokecolor{dialinecolor}
\draw (16.302918\du,-9.387500\du)--(23.566331\du,-9.387500\du)--(23.566331\du,-9.372500\du)--(30.829744\du,-9.372500\du);
}
% setfont left to latex
\pgfsetlinewidth{0.100000\du}
\pgfsetdash{{1.000000\du}{1.000000\du}}{0\du}
\pgfsetdash{{0.400000\du}{0.400000\du}}{0\du}
\pgfsetmiterjoin
\pgfsetbuttcap
{
\definecolor{dialinecolor}{rgb}{0.000000, 0.000000, 0.000000}
\pgfsetfillcolor{dialinecolor}
% was here!!!
\pgfsetarrowsend{to}
\definecolor{dialinecolor}{rgb}{0.000000, 0.000000, 0.000000}
\pgfsetstrokecolor{dialinecolor}
\draw (35.016250\du,-8.222185\du)--(35.016250\du,-3.387500\du)--(12.876250\du,-3.387500\du)--(12.876250\du,-8.237231\du);
}
% setfont left to latex
\pgfsetlinewidth{0.050000\du}
\pgfsetdash{{0.400000\du}{0.400000\du}}{0\du}
\pgfsetdash{{1.000000\du}{1.000000\du}}{0\du}
\pgfsetbuttcap
{
\definecolor{dialinecolor}{rgb}{0.000000, 0.000000, 0.000000}
\pgfsetfillcolor{dialinecolor}
% was here!!!
\definecolor{dialinecolor}{rgb}{0.000000, 0.000000, 0.000000}
\pgfsetstrokecolor{dialinecolor}
\draw (23.070498\du,-0.287775\du)--(25.600000\du,-3.437500\du);
}
% setfont left to latex
\definecolor{dialinecolor}{rgb}{0.000000, 0.000000, 0.000000}
\pgfsetstrokecolor{dialinecolor}
\node[anchor=west] at (17.250000\du,-8.237500\du){1};
% setfont left to latex
\definecolor{dialinecolor}{rgb}{0.000000, 0.000000, 0.000000}
\pgfsetstrokecolor{dialinecolor}
\node[anchor=west] at (29.950000\du,-8.237500\du){1};
\end{tikzpicture}

	\lstinputlisting[language=java]{ex11.java}
\end{center}

\paragraph{Remarque}
\begin{enumerate}
	\item Le code de méthodes conditionnement de ProduitLaitier et conditionnement est valide, quels que soient les
		produits laitiers et le conditionnement (ch chapitre 2 section 3.2)
	\item Une classe A dépend d'une classe B (relation non structurelle).
		\begin{itemize}
			\item Lorsque une méthode de A possède un paramètre de type ou un retour de type B
			\item Lorsque une méthode de A possède un objet local de type B
		\end{itemize}
	\item On peut coder la classe Chemise du chapitre 2 par par un mécanisme de délégation. On indique
		alors qu'une chemise possède (relation avoir) la caractéristique d'un Article.

\end{enumerate}


\section{Le polymorphisme}
\paragraph{Polymorphisme} Plusieurs (poly) formes (morphes) possible d'une même entité.
Deux types de molymorphisme lié:
\begin{itemize}
	\item Le polymorphisme d'inclusions
	\item Le polymorphisme de redéfinition(ou polymorphisme) d'héritage
\end{itemize}

\subsection{L'attachement polymorphe}
Tout objet de classe d'appartenance (de type) A peut désigner à l'exécution un objet, 
sous-classe (sous type) de A. \\D'après le principe d'inclusion des extensions (chapitre 2, section 1) 
\paragraph{Exemple}
	La classe Article avec Chemise et EauGazeuse.
	\begin{lstlisting}[language=java]
		Article a;
		Chemise c;
		c = new Chemise("Chemise", 50.0, 1, rouge);

		// Ecriture valide: 
		a = c;
		a = new EauGazeuse("Eau Gazeuse", 0.70);
	\end{lstlisting}

	\paragraph{Remarque}
	\begin{enumerate}
		\item Les attachements polymorphes (cf exemple avec a et c) interviennent lors des relations paramètre effectif.
	Paramètre formel- pour les paramètres de retour, pour les objets locaux et les objets courant \textbf{this}
		\item L'attachement est autorisé dans le sens $objet_{ancetre} = objet_{descendant}$
	L'affectation inverse n'a en général pas de sens!
	\end{enumerate}
	\subsection{Le polymorphisme d'inclusions}
	Ce polymorphisme résulte de l'héritage.\\
	À l'exécution, un objet d'une classe peut désigner un objet d'une sous classe.
	Traiter une famille d'ojet en ignorant pour chacun sa classe d'appartenance. (ici des chemises ch et des eaux gazeuses eg)\\
	Polymorphisme à condition de définir un héritage d'interface.\\
	\begin{center} % Graphic for TeX using PGF
% Title: /home/satenske/Diagram1.dia
% Creator: Dia v0.97.1
% CreationDate: Thu Oct  6 09:50:55 2011
% For: satenske
% \usepackage{tikz}
% The following commands are not supported in PSTricks at present
% We define them conditionally, so when they are implemented,
% this pgf file will use them.
\ifx\du\undefined
  \newlength{\du}
\fi
\setlength{\du}{15\unitlength}
\begin{tikzpicture}
\pgftransformxscale{1.000000}
\pgftransformyscale{-1.000000}
\definecolor{dialinecolor}{rgb}{0.000000, 0.000000, 0.000000}
\pgfsetstrokecolor{dialinecolor}
\definecolor{dialinecolor}{rgb}{1.000000, 1.000000, 1.000000}
\pgfsetfillcolor{dialinecolor}
\pgfsetlinewidth{0.000000\du}
\pgfsetdash{}{0pt}
\pgfsetdash{}{0pt}
\pgfsetbuttcap
\pgfsetmiterjoin
\pgfsetlinewidth{0.000000\du}
\pgfsetbuttcap
\pgfsetmiterjoin
\pgfsetdash{}{0pt}
\definecolor{dialinecolor}{rgb}{1.000000, 1.000000, 1.000000}
\pgfsetfillcolor{dialinecolor}
\pgfpathellipse{\pgfpoint{19.837500\du}{17.787500\du}}{\pgfpoint{6.837500\du}{0\du}}{\pgfpoint{0\du}{6.837500\du}}
\pgfusepath{fill}
\definecolor{dialinecolor}{rgb}{0.000000, 0.000000, 0.000000}
\pgfsetstrokecolor{dialinecolor}
\pgfpathellipse{\pgfpoint{19.837500\du}{17.787500\du}}{\pgfpoint{6.837500\du}{0\du}}{\pgfpoint{0\du}{6.837500\du}}
\pgfusepath{stroke}
\pgfsetbuttcap
\pgfsetmiterjoin
\pgfsetdash{}{0pt}
\definecolor{dialinecolor}{rgb}{0.000000, 0.000000, 0.000000}
\pgfsetstrokecolor{dialinecolor}
\pgfpathellipse{\pgfpoint{19.837500\du}{17.787500\du}}{\pgfpoint{6.837500\du}{0\du}}{\pgfpoint{0\du}{6.837500\du}}
\pgfusepath{stroke}
\pgfsetlinewidth{0.000000\du}
\pgfsetdash{}{0pt}
\pgfsetdash{}{0pt}
\pgfsetmiterjoin
\definecolor{dialinecolor}{rgb}{1.000000, 1.000000, 1.000000}
\pgfsetfillcolor{dialinecolor}
\fill (17.550000\du,17.150000\du)--(17.550000\du,18.150000\du)--(19.550000\du,18.150000\du)--(19.550000\du,17.150000\du)--cycle;
\definecolor{dialinecolor}{rgb}{0.000000, 0.000000, 0.000000}
\pgfsetstrokecolor{dialinecolor}
\draw (17.550000\du,17.150000\du)--(17.550000\du,18.150000\du)--(19.550000\du,18.150000\du)--(19.550000\du,17.150000\du)--cycle;
\pgfsetlinewidth{0.000000\du}
\pgfsetdash{}{0pt}
\pgfsetdash{}{0pt}
\pgfsetmiterjoin
\definecolor{dialinecolor}{rgb}{1.000000, 1.000000, 1.000000}
\pgfsetfillcolor{dialinecolor}
\fill (18.380000\du,12.920000\du)--(18.380000\du,13.920000\du)--(20.380000\du,13.920000\du)--(20.380000\du,12.920000\du)--cycle;
\definecolor{dialinecolor}{rgb}{0.000000, 0.000000, 0.000000}
\pgfsetstrokecolor{dialinecolor}
\draw (18.380000\du,12.920000\du)--(18.380000\du,13.920000\du)--(20.380000\du,13.920000\du)--(20.380000\du,12.920000\du)--cycle;
\pgfsetlinewidth{0.000000\du}
\pgfsetdash{}{0pt}
\pgfsetdash{}{0pt}
\pgfsetmiterjoin
\definecolor{dialinecolor}{rgb}{1.000000, 1.000000, 1.000000}
\pgfsetfillcolor{dialinecolor}
\fill (20.960000\du,19.940000\du)--(20.960000\du,20.940000\du)--(22.960000\du,20.940000\du)--(22.960000\du,19.940000\du)--cycle;
\definecolor{dialinecolor}{rgb}{0.000000, 0.000000, 0.000000}
\pgfsetstrokecolor{dialinecolor}
\draw (20.960000\du,19.940000\du)--(20.960000\du,20.940000\du)--(22.960000\du,20.940000\du)--(22.960000\du,19.940000\du)--cycle;
\pgfsetlinewidth{0.000000\du}
\pgfsetdash{}{0pt}
\pgfsetdash{}{0pt}
\pgfsetmiterjoin
\definecolor{dialinecolor}{rgb}{1.000000, 1.000000, 1.000000}
\pgfsetfillcolor{dialinecolor}
\fill (15.140000\du,19.960000\du)--(15.140000\du,20.960000\du)--(17.140000\du,20.960000\du)--(17.140000\du,19.960000\du)--cycle;
\definecolor{dialinecolor}{rgb}{0.000000, 0.000000, 0.000000}
\pgfsetstrokecolor{dialinecolor}
\draw (15.140000\du,19.960000\du)--(15.140000\du,20.960000\du)--(17.140000\du,20.960000\du)--(17.140000\du,19.960000\du)--cycle;
\pgfsetlinewidth{0.000000\du}
\pgfsetdash{}{0pt}
\pgfsetdash{}{0pt}
\pgfsetmiterjoin
\definecolor{dialinecolor}{rgb}{1.000000, 1.000000, 1.000000}
\pgfsetfillcolor{dialinecolor}
\fill (14.420000\du,15.430000\du)--(14.420000\du,16.430000\du)--(16.420000\du,16.430000\du)--(16.420000\du,15.430000\du)--cycle;
\definecolor{dialinecolor}{rgb}{0.000000, 0.000000, 0.000000}
\pgfsetstrokecolor{dialinecolor}
\draw (14.420000\du,15.430000\du)--(14.420000\du,16.430000\du)--(16.420000\du,16.430000\du)--(16.420000\du,15.430000\du)--cycle;
\pgfsetlinewidth{0.000000\du}
\pgfsetdash{}{0pt}
\pgfsetdash{}{0pt}
\pgfsetmiterjoin
\definecolor{dialinecolor}{rgb}{1.000000, 1.000000, 1.000000}
\pgfsetfillcolor{dialinecolor}
\fill (22.200000\du,15.850000\du)--(22.200000\du,16.850000\du)--(24.200000\du,16.850000\du)--(24.200000\du,15.850000\du)--cycle;
\definecolor{dialinecolor}{rgb}{0.000000, 0.000000, 0.000000}
\pgfsetstrokecolor{dialinecolor}
\draw (22.200000\du,15.850000\du)--(22.200000\du,16.850000\du)--(24.200000\du,16.850000\du)--(24.200000\du,15.850000\du)--cycle;
% setfont left to latex
\definecolor{dialinecolor}{rgb}{0.000000, 0.000000, 0.000000}
\pgfsetstrokecolor{dialinecolor}
\node[anchor=west] at (18.830000\du,13.695000\du){ch2};
% setfont left to latex
\definecolor{dialinecolor}{rgb}{0.000000, 0.000000, 0.000000}
\pgfsetstrokecolor{dialinecolor}
\node[anchor=west] at (14.715000\du,16.155000\du){ch1};
% setfont left to latex
\definecolor{dialinecolor}{rgb}{0.000000, 0.000000, 0.000000}
\pgfsetstrokecolor{dialinecolor}
\node[anchor=west] at (17.915000\du,17.880000\du){eg3};
% setfont left to latex
\definecolor{dialinecolor}{rgb}{0.000000, 0.000000, 0.000000}
\pgfsetstrokecolor{dialinecolor}
\node[anchor=west] at (22.630000\du,16.540000\du){eg2};
% setfont left to latex
\definecolor{dialinecolor}{rgb}{0.000000, 0.000000, 0.000000}
\pgfsetstrokecolor{dialinecolor}
\node[anchor=west] at (21.245000\du,20.725000\du){ch3};
% setfont left to latex
\definecolor{dialinecolor}{rgb}{0.000000, 0.000000, 0.000000}
\pgfsetstrokecolor{dialinecolor}
\node[anchor=west] at (15.535000\du,20.685000\du){eg1};
\end{tikzpicture}
 \end{center}
	\subsubsection{Exemple}
	\paragraph{1} Dans la classe HyperMarché, définition d'un attribut de type Tableau<Article>. Le table stock est un tableau
	polymorphe car il peut recevoir tout type d'article (des chemises, des eaux gazeuses,\ldots)
	\paragraph{2} Autre exemple: \\
	\begin{lstlisting}[language=java]
ProduitLaitier p;
Conditionnement c;

c = new Bouteille();
p = new Lait(c);
p.conditionner ();
	\end{lstlisting}

	\subsection{Polymorphisme de redéfinition ou polymorphisme d'héritage}
	Polymorphisme qui s'appuie sur la redéfinition de méthode. \\
	À l'exécution, le système choisit la méthode la plus spécialisée (redéfinie) en fonction du type 
	de l'objet.
	Appliquer une même méthode à des objets en ignorant pour chacun sa classe d'appartenance (ici la méthode m des chemises et des eaux gazeuses)
	\\
% Graphic for TeX using PGF
% Title: /home/satenske/Diagram1.dia
% Creator: Dia v0.97.1
% CreationDate: Thu Oct  6 10:47:28 2011
% For: satenske
% \usepackage{tikz}
% The following commands are not supported in PSTricks at present
% We define them conditionally, so when they are implemented,
% this pgf file will use them.
\ifx\du\undefined
  \newlength{\du}
\fi
\setlength{\du}{15\unitlength}
\begin{tikzpicture}
\pgftransformxscale{1.000000}
\pgftransformyscale{-1.000000}
\definecolor{dialinecolor}{rgb}{0.000000, 0.000000, 0.000000}
\pgfsetstrokecolor{dialinecolor}
\definecolor{dialinecolor}{rgb}{1.000000, 1.000000, 1.000000}
\pgfsetfillcolor{dialinecolor}
\pgfsetlinewidth{0.000000\du}
\pgfsetdash{}{0pt}
\pgfsetdash{}{0pt}
\pgfsetbuttcap
\pgfsetmiterjoin
\pgfsetlinewidth{0.000000\du}
\pgfsetbuttcap
\pgfsetmiterjoin
\pgfsetdash{}{0pt}
\definecolor{dialinecolor}{rgb}{1.000000, 1.000000, 1.000000}
\pgfsetfillcolor{dialinecolor}
\pgfpathellipse{\pgfpoint{22.600000\du}{19.625000\du}}{\pgfpoint{7.325000\du}{0\du}}{\pgfpoint{0\du}{7.325000\du}}
\pgfusepath{fill}
\definecolor{dialinecolor}{rgb}{0.000000, 0.000000, 0.000000}
\pgfsetstrokecolor{dialinecolor}
\pgfpathellipse{\pgfpoint{22.600000\du}{19.625000\du}}{\pgfpoint{7.325000\du}{0\du}}{\pgfpoint{0\du}{7.325000\du}}
\pgfusepath{stroke}
\pgfsetbuttcap
\pgfsetmiterjoin
\pgfsetdash{}{0pt}
\definecolor{dialinecolor}{rgb}{0.000000, 0.000000, 0.000000}
\pgfsetstrokecolor{dialinecolor}
\pgfpathellipse{\pgfpoint{22.600000\du}{19.625000\du}}{\pgfpoint{7.325000\du}{0\du}}{\pgfpoint{0\du}{7.325000\du}}
\pgfusepath{stroke}
\pgfsetlinewidth{0.000000\du}
\pgfsetdash{}{0pt}
\pgfsetdash{}{0pt}
\pgfsetmiterjoin
\definecolor{dialinecolor}{rgb}{1.000000, 1.000000, 1.000000}
\pgfsetfillcolor{dialinecolor}
\fill (19.875000\du,14.100000\du)--(19.875000\du,15.100000\du)--(24.175000\du,15.100000\du)--(24.175000\du,14.100000\du)--cycle;
\definecolor{dialinecolor}{rgb}{0.000000, 0.000000, 0.000000}
\pgfsetstrokecolor{dialinecolor}
\draw (19.875000\du,14.100000\du)--(19.875000\du,15.100000\du)--(24.175000\du,15.100000\du)--(24.175000\du,14.100000\du)--cycle;
% setfont left to latex
\definecolor{dialinecolor}{rgb}{0.000000, 0.000000, 0.000000}
\pgfsetstrokecolor{dialinecolor}
\node[anchor=west] at (20.375000\du,14.775000\du){m chemise};
\pgfsetlinewidth{0.000000\du}
\pgfsetdash{}{0pt}
\pgfsetdash{}{0pt}
\pgfsetmiterjoin
\definecolor{dialinecolor}{rgb}{1.000000, 1.000000, 1.000000}
\pgfsetfillcolor{dialinecolor}
\fill (17.055000\du,17.695000\du)--(17.055000\du,18.695000\du)--(21.355000\du,18.695000\du)--(21.355000\du,17.695000\du)--cycle;
\definecolor{dialinecolor}{rgb}{0.000000, 0.000000, 0.000000}
\pgfsetstrokecolor{dialinecolor}
\draw (17.055000\du,17.695000\du)--(17.055000\du,18.695000\du)--(21.355000\du,18.695000\du)--(21.355000\du,17.695000\du)--cycle;
% setfont left to latex
\definecolor{dialinecolor}{rgb}{0.000000, 0.000000, 0.000000}
\pgfsetstrokecolor{dialinecolor}
\node[anchor=west] at (17.555000\du,18.420000\du){m chemise};
\pgfsetlinewidth{0.000000\du}
\pgfsetdash{}{0pt}
\pgfsetdash{}{0pt}
\pgfsetmiterjoin
\definecolor{dialinecolor}{rgb}{1.000000, 1.000000, 1.000000}
\pgfsetfillcolor{dialinecolor}
\fill (23.185000\du,21.415000\du)--(23.185000\du,22.415000\du)--(27.485000\du,22.415000\du)--(27.485000\du,21.415000\du)--cycle;
\definecolor{dialinecolor}{rgb}{0.000000, 0.000000, 0.000000}
\pgfsetstrokecolor{dialinecolor}
\draw (23.185000\du,21.415000\du)--(23.185000\du,22.415000\du)--(27.485000\du,22.415000\du)--(27.485000\du,21.415000\du)--cycle;
% setfont left to latex
\definecolor{dialinecolor}{rgb}{0.000000, 0.000000, 0.000000}
\pgfsetstrokecolor{dialinecolor}
\node[anchor=west] at (23.685000\du,22.090000\du){m chemise};
\pgfsetlinewidth{0.000000\du}
\pgfsetdash{}{0pt}
\pgfsetdash{}{0pt}
\pgfsetmiterjoin
\definecolor{dialinecolor}{rgb}{1.000000, 1.000000, 1.000000}
\pgfsetfillcolor{dialinecolor}
\fill (22.655000\du,18.195000\du)--(22.655000\du,19.195000\du)--(28.225000\du,19.195000\du)--(28.225000\du,18.195000\du)--cycle;
\definecolor{dialinecolor}{rgb}{0.000000, 0.000000, 0.000000}
\pgfsetstrokecolor{dialinecolor}
\draw (22.655000\du,18.195000\du)--(22.655000\du,19.195000\du)--(28.225000\du,19.195000\du)--(28.225000\du,18.195000\du)--cycle;
% setfont left to latex
\definecolor{dialinecolor}{rgb}{0.000000, 0.000000, 0.000000}
\pgfsetstrokecolor{dialinecolor}
\node[anchor=west] at (23.155000\du,18.870000\du){m EauGazeuse};
% setfont left to latex
\definecolor{dialinecolor}{rgb}{0.000000, 0.000000, 0.000000}
\pgfsetstrokecolor{dialinecolor}
\node[anchor=west] at (25.625000\du,18.675000\du){};
\pgfsetlinewidth{0.000000\du}
\pgfsetdash{}{0pt}
\pgfsetdash{}{0pt}
\pgfsetmiterjoin
\definecolor{dialinecolor}{rgb}{1.000000, 1.000000, 1.000000}
\pgfsetfillcolor{dialinecolor}
\fill (16.605000\du,20.595000\du)--(16.605000\du,21.595000\du)--(22.175000\du,21.595000\du)--(22.175000\du,20.595000\du)--cycle;
\definecolor{dialinecolor}{rgb}{0.000000, 0.000000, 0.000000}
\pgfsetstrokecolor{dialinecolor}
\draw (16.605000\du,20.595000\du)--(16.605000\du,21.595000\du)--(22.175000\du,21.595000\du)--(22.175000\du,20.595000\du)--cycle;
% setfont left to latex
\definecolor{dialinecolor}{rgb}{0.000000, 0.000000, 0.000000}
\pgfsetstrokecolor{dialinecolor}
\node[anchor=west] at (17.105000\du,21.270000\du){m EauGazeuse};
\pgfsetlinewidth{0.000000\du}
\pgfsetdash{}{0pt}
\pgfsetdash{}{0pt}
\pgfsetmiterjoin
\definecolor{dialinecolor}{rgb}{1.000000, 1.000000, 1.000000}
\pgfsetfillcolor{dialinecolor}
\fill (20.035000\du,23.815000\du)--(20.035000\du,24.815000\du)--(25.605000\du,24.815000\du)--(25.605000\du,23.815000\du)--cycle;
\definecolor{dialinecolor}{rgb}{0.000000, 0.000000, 0.000000}
\pgfsetstrokecolor{dialinecolor}
\draw (20.035000\du,23.815000\du)--(20.035000\du,24.815000\du)--(25.605000\du,24.815000\du)--(25.605000\du,23.815000\du)--cycle;
% setfont left to latex
\definecolor{dialinecolor}{rgb}{0.000000, 0.000000, 0.000000}
\pgfsetstrokecolor{dialinecolor}
\node[anchor=west] at (20.535000\du,24.490000\du){m EauGazeuse};
\end{tikzpicture}

	\subsection{Exemple}
	\paragraph{1}
	Calcul du prixDeVente d'un article, fonction  la TVA (19.6\%) en général et 5.5\% pour les eaux gazeuses)\\
	Ce calcul s'effectue en fonction du ième article du tableau stock de type Table<Article>.
	\begin{lstlisting}[language=java, caption=Dans Article]
public Float prixDeVente(){
	return this.prixHT * 1.196);
}
	\end{lstlisting}

	\begin{lstlisting}[language=java, caption=Dans eau gazeuse]
public Float prixDeVente(){
	return this.prixHT * 1.055);
}
	\end{lstlisting}
	\begin{lstlisting}[language=java, caption=Dans supermarché]
prix = this.stock.getIème(i).prixDeVente();
\end{lstlisting}

\paragraph{2}Calcul de la masse salariale de la classe HyperMarché avec un attribut personnel de type Liste<Employé>.\\
Remarque: La méthode salaire de la classe employée est retardée!
\begin{lstlisting}[language=java, caption=Dans supermarché]
public Float masseSalariale(){
	Float s=0.0
	for (int i = 0; i < this.personnel.longueur(); i++)
		s = s + this.personnel.getIeme.salaire();

	return (s);
}
\end{lstlisting}

\paragraph{3}Recharche séquentielle d'un élément dans un tableau
\lstinputlisting[language=java]{12.java}
