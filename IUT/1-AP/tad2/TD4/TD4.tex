\documentclass{article}

\usepackage{lmodern}
\usepackage{xcolor}
\usepackage[utf8]{inputenc}
\usepackage[T1]{fontenc}
\usepackage[francais]{babel}
\usepackage[top=1.7cm, bottom=1.7cm, left=1.7cm, right=1.7cm]{geometry}
%\usepackage[frenchb]{babel}
%\usepackage{layout}
%\usepackage{setspace}
%\usepackage{soul}
%\usepackage{ulem}
%\usepackage{eurosym}
%\usepackage{bookman}
%\usepackage{charter}
%\usepackage{newcent}
%\usepackage{lmodern}
%\usepackage{mathpazo}
%\usepackage{mathptmx}
%\usepackage{url}
%\usepackage{verbatim}
%\usepackage{moreverb}
%\usepackage{wrapfig}
%\usepackage{amsmath}
%\usepackage{mathrsfs}
%\usepackage{asmthm}
%\usepackage{makeidx}
%\usepackage{tikz} %Vectoriel
\usepackage{listings}
\usepackage{fancyhdr}
\usepackage{multido}
\usepackage{amssymb}


\input{/home/satenske/cours/listings.tex} %prise en charge du langage algo

\title{TD 4\\ Le Type Abstrait de Données Durée}
\date{TAD\\ Semestre 2}

\lhead{TD4: Le Type	Abstrait de Données Durée}
\chead{}
\rhead{\thepage}

\lfoot{Université Paul Sabatier Toulouse III}
\cfoot{\thepage}
\rfoot{tad2}

\pagestyle{fancy}
\begin{document}
	\maketitle
	\section{Spécification algorithmique du Type Abstrait de Données Durée}
		\subsection{}
			\lstinputlisting{1.algo}
		\subsection{}
			\lstinputlisting{1-2.algo}
%		\lstinputlisting[caption=Programme]{2-2.algo}
	\section{Implémentation du Type Abstrait de Données Durée}
		\subsection{}
			\lstinputlisting{2-1.algo}
		\subsection{}
			\lstinputlisting{2-2.algo}
		\subsection{}
			Implémentation du TAD Durée: 
			\begin{itemize}
				\item type Durée (2-1)
				\item COrps des sous programmes (2-2)
				\item Le corps du sous programme estValide 
			\end{itemize}	
			\lstinputlisting{2-3.algo}
		\subsection{}
			\lstinputlisting{2-4.algo}
		\subsection{}
			\lstinputlisting{2-5.algo}
		\subsection{}
			Spécification cf question 1.2(valide pour l'implémentation 2.3)\\
			Implémentation -> cd 2.4 et 2.5
			
\end{document}
