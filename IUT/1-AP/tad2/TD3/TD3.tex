\documentclass{article}

\usepackage{lmodern}
\usepackage{xcolor}
\usepackage[utf8]{inputenc}
\usepackage[T1]{fontenc}
\usepackage[francais]{babel}
\usepackage[top=1.7cm, bottom=1.7cm, left=1.7cm, right=1.7cm]{geometry}
%\usepackage[frenchb]{babel}
%\usepackage{layout}
%\usepackage{setspace}
%\usepackage{soul}
%\usepackage{ulem}
%\usepackage{eurosym}
%\usepackage{bookman}
%\usepackage{charter}
%\usepackage{newcent}
%\usepackage{lmodern}
%\usepackage{mathpazo}
%\usepackage{mathptmx}
%\usepackage{url}
%\usepackage{verbatim}
%\usepackage{moreverb}
%\usepackage{wrapfig}
%\usepackage{amsmath}
%\usepackage{mathrsfs}
%\usepackage{asmthm}
%\usepackage{makeidx}
%\usepackage{tikz} %Vectoriel
\usepackage{listings}
\usepackage{fancyhdr}
\usepackage{multido}
\usepackage{amssymb}


\input{/home/satenske/cours/listings.tex} %prise en charge du langage algo

\title{TD 3\\ Evalusation d'une expression algébrique postfixée}
\date{TAD\\ Semestre 2}

\lhead{TD 3: Evaluation d'une expression algébrique postfixée}
\chead{}
\rhead{\thepage}

\lfoot{Université paul sabatier Toulouse III}
\cfoot{\thepage}
\rfoot{tad2}

\pagestyle{fancy}
\begin{document}
	\maketitle
	\paragraph{Une pile}
		Liste particulière dont les elements sont gérés selon la politique LIFO
			(Last In - First Out)\\
		\textbf{Exemple} une pile de livre
		
	\section{Spéficication algorithmique du Type Abstrait de Données Pile[T]}
		\lstinputlisting{1.algo}
	\section{Utilisation du Type Abstrait de Données Pile[T]}		
		\paragraph{Exemple} $a b + . $\\
		\begin{itemize}
			\item empiler la valeur de a (a)
			\item empiler la valeur de b (b)
			\item dépiler a et b (+)
			\item empiler le résultat de $a + b$ (+)
		\end{itemize}
		\paragraph{Exemple} $d e * a b + / c - . $\\
		\begin{itemize}
			\item empiler la valeur de d (d)
			\item empiler la valeur de e (e)
			\item dépiler d et e (*)
			\item empiler le résultat de $d \times e$ (*)
			\item empiler la valeur de a (a)
			\item empiler la valeur de b (b)
			\item dépiler pour obtenir a et b (+) 
			\item empiler la valeur de $a + b$ (/) 
			\item dépiler pour obtenir de * et ab+ (/)
			\item empiler le résultat $(d \times e) / (a + b)$
			\item Empiler la valeur de c
			\item dépiler les trois opérandes de -
			\item empiler la valeur de $(d \times e)/(a + b) - c$
		\end{itemize}
		\lstinputlisting[caption=Algorithme général]{2.algo}
		\lstinputlisting[caption=Programme]{2-2.algo}
\end{document}
