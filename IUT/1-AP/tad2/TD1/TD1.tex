\documentclass{article}

\usepackage{lmodern}
\usepackage{xcolor}
\usepackage[utf8]{inputenc}
\usepackage[T1]{fontenc}
\usepackage[francais]{babel}
\usepackage[top=1.7cm, bottom=1.7cm, left=1.7cm, right=1.7cm]{geometry}
%\usepackage[frenchb]{babel}
%\usepackage{layout}
%\usepackage{setspace}
%\usepackage{soul}
%\usepackage{ulem}
%\usepackage{eurosym}
%\usepackage{bookman}
%\usepackage{charter}
%\usepackage{newcent}
%\usepackage{lmodern}
%\usepackage{mathpazo}
%\usepackage{mathptmx}
%\usepackage{url}
%\usepackage{verbatim}
%\usepackage{moreverb}
%\usepackage{wrapfig}
%\usepackage{amsmath}
%\usepackage{mathrsfs}
%\usepackage{asmthm}
%\usepackage{makeidx}
\usepackage{tikz} %Vectoriel
\usepackage{listings}
\usepackage{fancyhdr}
\usepackage{multido}
\usepackage{amssymb}

\definecolor{gris1}{gray}{0.40}
\definecolor{gris2}{gray}{0.55}
\definecolor{gris3}{gray}{0.65}
\definecolor{gris4}{gray}{0.50}


%\lstset{language=algo,
    %%% BOUCLE, TEST & Co.
%      emph={importer, programme, glossaire, fonction, procedure, constante, type},
%      emphstyle=\color{gris2},
    %%% IMPORT & Co.
%      emph={[2]si, sinon, alors, fin , tantque, debut, faire, lorsque, fin lorsque, declancher, retourner, et, ou, non,enregistrement, retourner, retourne, tableau, /=, <, =, >, traite,exception},
%      emphstyle=[2]\color{gris1},
    %%% FONCTIONS NUMERIQUES
%      emph={[3]Entier, Reel, Booleen, Caractere},
%      emphstyle=[3]\color{gris3},
    %%% FONCTIONS NUMERIQUES
%      emph={[4]entree, maj, sortie},	
%      emphstyle=[4]\color{gris4},
%}
\lstset{ % general style for listings 
   numbers=left 
   , tabsize=2 
   , frame=single 
   , breaklines=true 
   , basicstyle=\ttfamily 
   , numberstyle=\tiny\ttfamily 
   , framexleftmargin=13mm 
   , xleftmargin=12mm 
   , captionpos=b 
	, keywordstyle=\color{blue}
	, commentstyle=\color{green}
	, showstringspaces=false
	, extendedchars=true
} 
 %prise en charge du langage algo

\title{TD 1\\ Construction dynamique d'une liste}
\date{TAD\\ Semestre 1}

\lhead{TD 1: Construction dynamique d'une liste}
\chead{}
\rhead{\thepage}

\lfoot{Université Paul sabatier Toulouse III}
\cfoot{\thepage}
\rfoot{tad2}

\pagestyle{fancy}
\begin{document}
	\maketitle
	\section{Représentation de la liste}		
		% Graphic for TeX using PGF
% Title: /usr/home/satenske/Diagram1.dia
% Creator: Dia v0.97.1
% CreationDate: Tue Feb  1 08:26:44 2011
% For: satenske
% \usepackage{tikz}
% The following commands are not supported in PSTricks at present
% We define them conditionally, so when they are implemented,
% this pgf file will use them.
\ifx\du\undefined
  \newlength{\du}
\fi
\setlength{\du}{15\unitlength}
\begin{tikzpicture}
\pgftransformxscale{1.000000}
\pgftransformyscale{-1.000000}
\definecolor{dialinecolor}{rgb}{0.000000, 0.000000, 0.000000}
\pgfsetstrokecolor{dialinecolor}
\definecolor{dialinecolor}{rgb}{1.000000, 1.000000, 1.000000}
\pgfsetfillcolor{dialinecolor}
\pgfsetlinewidth{0.100000\du}
\pgfsetdash{}{0pt}
\pgfsetdash{}{0pt}
\pgfsetbuttcap
\pgfsetmiterjoin
\pgfsetlinewidth{0.100000\du}
\pgfsetbuttcap
\pgfsetmiterjoin
\pgfsetdash{}{0pt}
\definecolor{dialinecolor}{rgb}{1.000000, 1.000000, 1.000000}
\pgfsetfillcolor{dialinecolor}
\fill (1.382258\du,9.600000\du)--(1.382258\du,11.600000\du)--(3.317742\du,11.600000\du)--(3.317742\du,9.600000\du)--cycle;
\definecolor{dialinecolor}{rgb}{0.000000, 0.000000, 0.000000}
\pgfsetstrokecolor{dialinecolor}
\draw (1.382258\du,9.600000\du)--(1.382258\du,11.600000\du)--(3.317742\du,11.600000\du)--(3.317742\du,9.600000\du)--cycle;
\pgfsetbuttcap
\pgfsetmiterjoin
\pgfsetdash{}{0pt}
\definecolor{dialinecolor}{rgb}{0.000000, 0.000000, 0.000000}
\pgfsetstrokecolor{dialinecolor}
\draw (1.382258\du,9.600000\du)--(1.382258\du,11.600000\du)--(3.317742\du,11.600000\du)--(3.317742\du,9.600000\du)--cycle;
\pgfsetlinewidth{0.100000\du}
\pgfsetdash{}{0pt}
\pgfsetdash{}{0pt}
\pgfsetbuttcap
\pgfsetmiterjoin
\pgfsetlinewidth{0.100000\du}
\pgfsetbuttcap
\pgfsetmiterjoin
\pgfsetdash{}{0pt}
\definecolor{dialinecolor}{rgb}{1.000000, 1.000000, 1.000000}
\pgfsetfillcolor{dialinecolor}
\fill (6.032258\du,9.600000\du)--(6.032258\du,11.600000\du)--(7.967742\du,11.600000\du)--(7.967742\du,9.600000\du)--cycle;
\definecolor{dialinecolor}{rgb}{0.000000, 0.000000, 0.000000}
\pgfsetstrokecolor{dialinecolor}
\draw (6.032258\du,9.600000\du)--(6.032258\du,11.600000\du)--(7.967742\du,11.600000\du)--(7.967742\du,9.600000\du)--cycle;
\pgfsetbuttcap
\pgfsetmiterjoin
\pgfsetdash{}{0pt}
\definecolor{dialinecolor}{rgb}{0.000000, 0.000000, 0.000000}
\pgfsetstrokecolor{dialinecolor}
\draw (6.032258\du,9.600000\du)--(6.032258\du,11.600000\du)--(7.967742\du,11.600000\du)--(7.967742\du,9.600000\du)--cycle;
\pgfsetlinewidth{0.100000\du}
\pgfsetdash{}{0pt}
\pgfsetdash{}{0pt}
\pgfsetbuttcap
{
\definecolor{dialinecolor}{rgb}{0.000000, 0.000000, 0.000000}
\pgfsetfillcolor{dialinecolor}
% was here!!!
\pgfsetarrowsend{stealth}
\definecolor{dialinecolor}{rgb}{0.000000, 0.000000, 0.000000}
\pgfsetstrokecolor{dialinecolor}
\draw (2.867688\du,10.692395\du)--(5.900000\du,10.750000\du);
}
\pgfsetlinewidth{0.100000\du}
\pgfsetdash{}{0pt}
\pgfsetdash{}{0pt}
\pgfsetbuttcap
\pgfsetmiterjoin
\pgfsetlinewidth{0.100000\du}
\pgfsetbuttcap
\pgfsetmiterjoin
\pgfsetdash{}{0pt}
\definecolor{dialinecolor}{rgb}{1.000000, 1.000000, 1.000000}
\pgfsetfillcolor{dialinecolor}
\fill (7.953011\du,9.605000\du)--(7.953011\du,11.605000\du)--(9.888495\du,11.605000\du)--(9.888495\du,9.605000\du)--cycle;
\definecolor{dialinecolor}{rgb}{0.000000, 0.000000, 0.000000}
\pgfsetstrokecolor{dialinecolor}
\draw (7.953011\du,9.605000\du)--(7.953011\du,11.605000\du)--(9.888495\du,11.605000\du)--(9.888495\du,9.605000\du)--cycle;
\pgfsetbuttcap
\pgfsetmiterjoin
\pgfsetdash{}{0pt}
\definecolor{dialinecolor}{rgb}{0.000000, 0.000000, 0.000000}
\pgfsetstrokecolor{dialinecolor}
\draw (7.953011\du,9.605000\du)--(7.953011\du,11.605000\du)--(9.888495\du,11.605000\du)--(9.888495\du,9.605000\du)--cycle;
\pgfsetlinewidth{0.100000\du}
\pgfsetdash{}{0pt}
\pgfsetdash{}{0pt}
\pgfsetbuttcap
{
\definecolor{dialinecolor}{rgb}{0.000000, 0.000000, 0.000000}
\pgfsetfillcolor{dialinecolor}
% was here!!!
\pgfsetarrowsend{stealth}
\definecolor{dialinecolor}{rgb}{0.000000, 0.000000, 0.000000}
\pgfsetstrokecolor{dialinecolor}
\draw (9.288661\du,10.645714\du)--(12.475000\du,10.588125\du);
}
% setfont left to latex
\definecolor{dialinecolor}{rgb}{0.000000, 0.000000, 0.000000}
\pgfsetstrokecolor{dialinecolor}
\node[anchor=west] at (1.950000\du,10.700000\du){};
% setfont left to latex
\definecolor{dialinecolor}{rgb}{0.000000, 0.000000, 0.000000}
\pgfsetstrokecolor{dialinecolor}
\node[anchor=west] at (2.350000\du,10.600000\du){};
% setfont left to latex
\definecolor{dialinecolor}{rgb}{0.000000, 0.000000, 0.000000}
\pgfsetstrokecolor{dialinecolor}
\node[anchor=west] at (6.900000\du,10.750000\du){e1};
% setfont left to latex
\definecolor{dialinecolor}{rgb}{0.000000, 0.000000, 0.000000}
\pgfsetstrokecolor{dialinecolor}
\node[anchor=west] at (2.350000\du,10.600000\du){};
\pgfsetlinewidth{0.100000\du}
\pgfsetdash{}{0pt}
\pgfsetdash{}{0pt}
\pgfsetbuttcap
\pgfsetmiterjoin
\pgfsetlinewidth{0.100000\du}
\pgfsetbuttcap
\pgfsetmiterjoin
\pgfsetdash{}{0pt}
\definecolor{dialinecolor}{rgb}{1.000000, 1.000000, 1.000000}
\pgfsetfillcolor{dialinecolor}
\fill (12.425000\du,9.793125\du)--(12.425000\du,11.793125\du)--(14.360484\du,11.793125\du)--(14.360484\du,9.793125\du)--cycle;
\definecolor{dialinecolor}{rgb}{0.000000, 0.000000, 0.000000}
\pgfsetstrokecolor{dialinecolor}
\draw (12.425000\du,9.793125\du)--(12.425000\du,11.793125\du)--(14.360484\du,11.793125\du)--(14.360484\du,9.793125\du)--cycle;
\pgfsetbuttcap
\pgfsetmiterjoin
\pgfsetdash{}{0pt}
\definecolor{dialinecolor}{rgb}{0.000000, 0.000000, 0.000000}
\pgfsetstrokecolor{dialinecolor}
\draw (12.425000\du,9.793125\du)--(12.425000\du,11.793125\du)--(14.360484\du,11.793125\du)--(14.360484\du,9.793125\du)--cycle;
\pgfsetlinewidth{0.100000\du}
\pgfsetdash{}{0pt}
\pgfsetdash{}{0pt}
\pgfsetbuttcap
\pgfsetmiterjoin
\pgfsetlinewidth{0.100000\du}
\pgfsetbuttcap
\pgfsetmiterjoin
\pgfsetdash{}{0pt}
\definecolor{dialinecolor}{rgb}{1.000000, 1.000000, 1.000000}
\pgfsetfillcolor{dialinecolor}
\fill (14.345753\du,9.798125\du)--(14.345753\du,11.798125\du)--(16.281237\du,11.798125\du)--(16.281237\du,9.798125\du)--cycle;
\definecolor{dialinecolor}{rgb}{0.000000, 0.000000, 0.000000}
\pgfsetstrokecolor{dialinecolor}
\draw (14.345753\du,9.798125\du)--(14.345753\du,11.798125\du)--(16.281237\du,11.798125\du)--(16.281237\du,9.798125\du)--cycle;
\pgfsetbuttcap
\pgfsetmiterjoin
\pgfsetdash{}{0pt}
\definecolor{dialinecolor}{rgb}{0.000000, 0.000000, 0.000000}
\pgfsetstrokecolor{dialinecolor}
\draw (14.345753\du,9.798125\du)--(14.345753\du,11.798125\du)--(16.281237\du,11.798125\du)--(16.281237\du,9.798125\du)--cycle;
\pgfsetlinewidth{0.100000\du}
\pgfsetdash{}{0pt}
\pgfsetdash{}{0pt}
\pgfsetbuttcap
{
\definecolor{dialinecolor}{rgb}{0.000000, 0.000000, 0.000000}
\pgfsetfillcolor{dialinecolor}
% was here!!!
\pgfsetarrowsend{stealth}
\definecolor{dialinecolor}{rgb}{0.000000, 0.000000, 0.000000}
\pgfsetstrokecolor{dialinecolor}
\draw (15.681403\du,10.838839\du)--(18.867742\du,10.781250\du);
}
% setfont left to latex
\definecolor{dialinecolor}{rgb}{0.000000, 0.000000, 0.000000}
\pgfsetstrokecolor{dialinecolor}
\node[anchor=west] at (13.292742\du,10.943125\du){e1};
\pgfsetlinewidth{0.100000\du}
\pgfsetdash{}{0pt}
\pgfsetdash{}{0pt}
\pgfsetbuttcap
\pgfsetmiterjoin
\pgfsetlinewidth{0.100000\du}
\pgfsetbuttcap
\pgfsetmiterjoin
\pgfsetdash{}{0pt}
\definecolor{dialinecolor}{rgb}{1.000000, 1.000000, 1.000000}
\pgfsetfillcolor{dialinecolor}
\fill (18.375000\du,9.743125\du)--(18.375000\du,11.743125\du)--(20.310484\du,11.743125\du)--(20.310484\du,9.743125\du)--cycle;
\definecolor{dialinecolor}{rgb}{0.000000, 0.000000, 0.000000}
\pgfsetstrokecolor{dialinecolor}
\draw (18.375000\du,9.743125\du)--(18.375000\du,11.743125\du)--(20.310484\du,11.743125\du)--(20.310484\du,9.743125\du)--cycle;
\pgfsetbuttcap
\pgfsetmiterjoin
\pgfsetdash{}{0pt}
\definecolor{dialinecolor}{rgb}{0.000000, 0.000000, 0.000000}
\pgfsetstrokecolor{dialinecolor}
\draw (18.375000\du,9.743125\du)--(18.375000\du,11.743125\du)--(20.310484\du,11.743125\du)--(20.310484\du,9.743125\du)--cycle;
\pgfsetlinewidth{0.100000\du}
\pgfsetdash{}{0pt}
\pgfsetdash{}{0pt}
\pgfsetbuttcap
\pgfsetmiterjoin
\pgfsetlinewidth{0.100000\du}
\pgfsetbuttcap
\pgfsetmiterjoin
\pgfsetdash{}{0pt}
\definecolor{dialinecolor}{rgb}{1.000000, 1.000000, 1.000000}
\pgfsetfillcolor{dialinecolor}
\fill (20.295753\du,9.748125\du)--(20.295753\du,11.748125\du)--(22.231237\du,11.748125\du)--(22.231237\du,9.748125\du)--cycle;
\definecolor{dialinecolor}{rgb}{0.000000, 0.000000, 0.000000}
\pgfsetstrokecolor{dialinecolor}
\draw (20.295753\du,9.748125\du)--(20.295753\du,11.748125\du)--(22.231237\du,11.748125\du)--(22.231237\du,9.748125\du)--cycle;
\pgfsetbuttcap
\pgfsetmiterjoin
\pgfsetdash{}{0pt}
\definecolor{dialinecolor}{rgb}{0.000000, 0.000000, 0.000000}
\pgfsetstrokecolor{dialinecolor}
\draw (20.295753\du,9.748125\du)--(20.295753\du,11.748125\du)--(22.231237\du,11.748125\du)--(22.231237\du,9.748125\du)--cycle;
% setfont left to latex
\definecolor{dialinecolor}{rgb}{0.000000, 0.000000, 0.000000}
\pgfsetstrokecolor{dialinecolor}
\node[anchor=west] at (19.242742\du,10.893125\du){e1};
\pgfsetlinewidth{0.100000\du}
\pgfsetdash{}{0pt}
\pgfsetdash{}{0pt}
\pgfsetmiterjoin
\pgfsetbuttcap
{
\definecolor{dialinecolor}{rgb}{0.000000, 0.000000, 0.000000}
\pgfsetfillcolor{dialinecolor}
% was here!!!
\pgfsetarrowsend{stealth}
{\pgfsetcornersarced{\pgfpoint{0.000000\du}{0.000000\du}}\definecolor{dialinecolor}{rgb}{0.000000, 0.000000, 0.000000}
\pgfsetstrokecolor{dialinecolor}
\draw (21.625000\du,10.813125\du)--(21.275000\du,10.813125\du)--(21.275000\du,18.113125\du)--(17.025000\du,18.113125\du);
}}
\pgfsetlinewidth{0.100000\du}
\pgfsetdash{}{0pt}
\pgfsetdash{}{0pt}
\pgfsetbuttcap
\pgfsetmiterjoin
\pgfsetlinewidth{0.100000\du}
\pgfsetbuttcap
\pgfsetmiterjoin
\pgfsetdash{}{0pt}
\definecolor{dialinecolor}{rgb}{1.000000, 1.000000, 1.000000}
\pgfsetfillcolor{dialinecolor}
\fill (10.482258\du,17.018125\du)--(10.482258\du,19.018125\du)--(12.417742\du,19.018125\du)--(12.417742\du,17.018125\du)--cycle;
\definecolor{dialinecolor}{rgb}{0.000000, 0.000000, 0.000000}
\pgfsetstrokecolor{dialinecolor}
\draw (10.482258\du,17.018125\du)--(10.482258\du,19.018125\du)--(12.417742\du,19.018125\du)--(12.417742\du,17.018125\du)--cycle;
\pgfsetbuttcap
\pgfsetmiterjoin
\pgfsetdash{}{0pt}
\definecolor{dialinecolor}{rgb}{0.000000, 0.000000, 0.000000}
\pgfsetstrokecolor{dialinecolor}
\draw (10.482258\du,17.018125\du)--(10.482258\du,19.018125\du)--(12.417742\du,19.018125\du)--(12.417742\du,17.018125\du)--cycle;
\pgfsetlinewidth{0.100000\du}
\pgfsetdash{}{0pt}
\pgfsetdash{}{0pt}
\pgfsetbuttcap
\pgfsetmiterjoin
\pgfsetlinewidth{0.100000\du}
\pgfsetbuttcap
\pgfsetmiterjoin
\pgfsetdash{}{0pt}
\definecolor{dialinecolor}{rgb}{1.000000, 1.000000, 1.000000}
\pgfsetfillcolor{dialinecolor}
\fill (12.422366\du,17.063125\du)--(12.422366\du,19.023125\du)--(14.319140\du,19.023125\du)--(14.319140\du,17.063125\du)--cycle;
\definecolor{dialinecolor}{rgb}{0.000000, 0.000000, 0.000000}
\pgfsetstrokecolor{dialinecolor}
\draw (12.422366\du,17.063125\du)--(12.422366\du,19.023125\du)--(14.319140\du,19.023125\du)--(14.319140\du,17.063125\du)--cycle;
\pgfsetbuttcap
\pgfsetmiterjoin
\pgfsetdash{}{0pt}
\definecolor{dialinecolor}{rgb}{0.000000, 0.000000, 0.000000}
\pgfsetstrokecolor{dialinecolor}
\draw (12.422366\du,17.063125\du)--(12.422366\du,19.023125\du)--(14.319140\du,19.023125\du)--(14.319140\du,17.063125\du)--cycle;
% setfont left to latex
\definecolor{dialinecolor}{rgb}{0.000000, 0.000000, 0.000000}
\pgfsetstrokecolor{dialinecolor}
\node[anchor=west] at (10.950000\du,18.368125\du){ei};
% setfont left to latex
\definecolor{dialinecolor}{rgb}{0.000000, 0.000000, 0.000000}
\pgfsetstrokecolor{dialinecolor}
\node[anchor=west] at (15.882258\du,18.213125\du){.....};
\pgfsetlinewidth{0.100000\du}
\pgfsetdash{}{0pt}
\pgfsetdash{}{0pt}
\pgfsetmiterjoin
\pgfsetbuttcap
{
\definecolor{dialinecolor}{rgb}{0.000000, 0.000000, 0.000000}
\pgfsetfillcolor{dialinecolor}
% was here!!!
\pgfsetarrowsend{stealth}
{\pgfsetcornersarced{\pgfpoint{0.000000\du}{0.000000\du}}\definecolor{dialinecolor}{rgb}{0.000000, 0.000000, 0.000000}
\pgfsetstrokecolor{dialinecolor}
\draw (13.338124\du,19.072678\du)--(13.232258\du,22.413125\du);
}}
\pgfsetlinewidth{0.100000\du}
\pgfsetdash{}{0pt}
\pgfsetdash{}{0pt}
\pgfsetbuttcap
\pgfsetmiterjoin
\pgfsetlinewidth{0.100000\du}
\pgfsetbuttcap
\pgfsetmiterjoin
\pgfsetdash{}{0pt}
\definecolor{dialinecolor}{rgb}{1.000000, 1.000000, 1.000000}
\pgfsetfillcolor{dialinecolor}
\fill (10.482258\du,22.068125\du)--(10.482258\du,24.068125\du)--(12.417742\du,24.068125\du)--(12.417742\du,22.068125\du)--cycle;
\definecolor{dialinecolor}{rgb}{0.000000, 0.000000, 0.000000}
\pgfsetstrokecolor{dialinecolor}
\draw (10.482258\du,22.068125\du)--(10.482258\du,24.068125\du)--(12.417742\du,24.068125\du)--(12.417742\du,22.068125\du)--cycle;
\pgfsetbuttcap
\pgfsetmiterjoin
\pgfsetdash{}{0pt}
\definecolor{dialinecolor}{rgb}{0.000000, 0.000000, 0.000000}
\pgfsetstrokecolor{dialinecolor}
\draw (10.482258\du,22.068125\du)--(10.482258\du,24.068125\du)--(12.417742\du,24.068125\du)--(12.417742\du,22.068125\du)--cycle;
\pgfsetlinewidth{0.100000\du}
\pgfsetdash{}{0pt}
\pgfsetdash{}{0pt}
\pgfsetbuttcap
\pgfsetmiterjoin
\pgfsetlinewidth{0.100000\du}
\pgfsetbuttcap
\pgfsetmiterjoin
\pgfsetdash{}{0pt}
\definecolor{dialinecolor}{rgb}{1.000000, 1.000000, 1.000000}
\pgfsetfillcolor{dialinecolor}
\fill (12.422366\du,22.113125\du)--(12.422366\du,24.073125\du)--(14.319140\du,24.073125\du)--(14.319140\du,22.113125\du)--cycle;
\definecolor{dialinecolor}{rgb}{0.000000, 0.000000, 0.000000}
\pgfsetstrokecolor{dialinecolor}
\draw (12.422366\du,22.113125\du)--(12.422366\du,24.073125\du)--(14.319140\du,24.073125\du)--(14.319140\du,22.113125\du)--cycle;
\pgfsetbuttcap
\pgfsetmiterjoin
\pgfsetdash{}{0pt}
\definecolor{dialinecolor}{rgb}{0.000000, 0.000000, 0.000000}
\pgfsetstrokecolor{dialinecolor}
\draw (12.422366\du,22.113125\du)--(12.422366\du,24.073125\du)--(14.319140\du,24.073125\du)--(14.319140\du,22.113125\du)--cycle;
% setfont left to latex
\definecolor{dialinecolor}{rgb}{0.000000, 0.000000, 0.000000}
\pgfsetstrokecolor{dialinecolor}
\node[anchor=west] at (10.950000\du,23.418125\du){en};
% setfont left to latex
\definecolor{dialinecolor}{rgb}{0.000000, 0.000000, 0.000000}
\pgfsetstrokecolor{dialinecolor}
\node[anchor=west] at (12.570753\du,23.343125\du){NULL};
\pgfsetlinewidth{0.100000\du}
\pgfsetdash{}{0pt}
\pgfsetdash{}{0pt}
\pgfsetbuttcap
{
\definecolor{dialinecolor}{rgb}{0.000000, 0.000000, 0.000000}
\pgfsetfillcolor{dialinecolor}
% was here!!!
\pgfsetarrowsend{stealth}
\definecolor{dialinecolor}{rgb}{0.000000, 0.000000, 0.000000}
\pgfsetstrokecolor{dialinecolor}
\draw (15.632258\du,18.213125\du)--(14.319140\du,18.043125\du);
}
% setfont left to latex
\definecolor{dialinecolor}{rgb}{0.000000, 0.000000, 0.000000}
\pgfsetstrokecolor{dialinecolor}
\node[anchor=west] at (2.232258\du,12.513125\du){l};
\end{tikzpicture}

	\section{Définition du type de la liste}		
		\lstinputlisting[caption=Type de la liste]{2.algo}
	\section{Construction de la liste}
		\subsection{}
		\subsection{}
			\subsubsection{Avant la première insertion}
		\subsection{}
			\lstinputlisting{3-3.algo}
	\section{Impression de la liste en détruisant au fur et à mesure 
				ses éléments}
		\subsection{}
			\lstinputlisting{4-1.algo}
		\subsection{}
			\subsubsection{Après la première insertion}
			\subsubsection{Suppresion autre que la première}
				même séquence de code que a (car suppression toujours en tête 
					de la liste)
		\subsection{}
			\lstinputlisting{4-3.algo}
\end{document}
