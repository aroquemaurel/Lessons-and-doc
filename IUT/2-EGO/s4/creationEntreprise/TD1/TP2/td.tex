\documentclass{article}


\usepackage{lmodern}
\usepackage{xcolor}
\usepackage[utf8]{inputenc}
\usepackage[T1]{fontenc}
\usepackage[francais]{babel}
\usepackage[top=1.7cm, bottom=1.7cm, left=1.7cm, right=1.7cm]{geometry}
\usepackage{verbatim}
\usepackage{tikz} %Vectoriel
\usepackage{listings}
\usepackage{fancyhdr}
\usepackage{multido}
\usepackage{amssymb}
\usepackage{float}

\newcommand{\titre}{Les emprunts et les prêts}
\newcommand{\numTD}{1}

\newcommand{\module}{Création d'entreprise}
\newcommand{\sigle}{ce}

\newcommand{\semestre}{4}

\input{/home/satenske/cours/listings.tex} %prise en charge du langage algo

\usepackage{ifthen}
\date{\today}

\chead{}
\rhead{TD\no\numTD}
\lhead{\titre}
%\makeindex

\lfoot{Université Paul Sabatier Toulouse III}
\rfoot{\sigle\semestre}
%\rfoot{}
\cfoot{--~~\thepage~~--}

\makeglossary
\makeatletter
\def\clap#1{\hbox to 0pt{\hss #1\hss}}%

\def\haut#1#2#3{%
	\hbox to \hsize{%
		\rlap{\vtop{\raggedright #1}
	}%
	\hss
	\clap{\vtop{\centering #2}
}%
\hss
\llap{\vtop{\raggedleft #3}}}}%
\def\bas#1#2#3{%
	\hbox to \hsize{%
		\rlap{\vbox{
			\raggedright #1
		}
	}%
	\hss \clap{\vbox{\centering #2}}%
	\hss
	\llap{\vbox{\raggedleft #3}}}
}%
\def\maketitle{%
	\thispagestyle{empty}{%
		\haut{}{\@blurb}{}
		%	
		%\vfill

		\begin{center}
			\vspace{-1.5cm}
			\usefont{OT1}{ptm}{m}{n}
			\huge \@numeroTD \@title
		\end{center}
		\par
		\hrule height 1pt
		\par
		\vspace{1cm}
		\bas{}{}{}
}%
}
\def\date#1{\def\@date{#1}}
\def\author#1{\def\@author{#1}}
\def\numeroTD#1{\def\@numeroTD{#1}}
\def\title#1{\def\@title{#1}}
\def\location#1{\def\@location{#1}}
\def\blurb#1{\def\@blurb{#1}}
\date{\today}
\newboolean{monBool}
\setboolean{monBool}{true}
\author{}
\title{}
\ifthenelse{\equal{\numTD}{}}{
\numeroTD{}
}
{
	\numeroTD{TD \no\numTD~--- }
}
\location{Amiens}\blurb{}
%\makeatother
\title{\titre}
\author{%Semestre \semestre
}

\location{Toulouse}
\blurb{%
\vspace{-35px}
\begin{flushleft}
	Université Paul Sabatier -- Toulouse III\\
	IUT A - Toulouse Rangueil\\
\end{flushleft}
\begin{flushright}
	\vspace{-45px}
	\Large \textbf \module \\
	\normalsize \textit \today\\
	Semestre \semestre
	\vspace{30px}
\end{flushright}
}%



%\title{Cours \\ \titre}
%\date{\today\\ Semestre \semestre}

%\lhead{Cours: \titre}
%\chead{}
%\rhead{\thepage}

%\lfoot{Université Paul Sabatier Toulouse III}
%\cfoot{\thepage}
%\rfoot{\sigle\semestre}

\pagestyle{fancy}





%----------------------------------------------------------------------------------------
%	DEFINITION OF COLORED BOXES
%----------------------------------------------------------------------------------------

\RequirePackage[framemethod=default]{mdframed} % Required for creating the theorem, definition, exercise and corollary boxes

% Theorem box
\newmdenv[skipabove=7pt,
skipbelow=7pt,
backgroundcolor=black!5,
linecolor=ocre,
innerleftmargin=5pt,
innerrightmargin=5pt,
innertopmargin=5pt,
leftmargin=0cm,
rightmargin=0cm,
innerbottommargin=5pt]{tBox}

% Exercise box	  
\newmdenv[skipabove=7pt,
skipbelow=7pt,
rightline=false,
leftline=true,
topline=false,
bottomline=false,
backgroundcolor=ocre!10,
linecolor=ocre,
innerleftmargin=5pt,
innerrightmargin=5pt,
innertopmargin=5pt,
innerbottommargin=5pt,
leftmargin=0cm,
rightmargin=0cm,
linewidth=4pt]{eBox}	

% Definition box
\newmdenv[skipabove=10pt,
skipbelow=10pt,
rightline=false,
leftline=true,
topline=false,
bottomline=false,
linecolor=ocre,
innerleftmargin=5pt,
innerrightmargin=5pt,
innertopmargin=0pt,
leftmargin=0cm,
rightmargin=0cm,
linewidth=4pt,
innerbottommargin=0pt]{dBox}	

% Corollary box
\newmdenv[skipabove=7pt,
skipbelow=7pt,
rightline=false,
leftline=true,
topline=false,
bottomline=false,
linecolor=gray,
backgroundcolor=black!5,
innerleftmargin=5pt,
innerrightmargin=5pt,
innertopmargin=5pt,
leftmargin=0cm,
rightmargin=0cm,
linewidth=4pt,
innerbottommargin=5pt]{cBox}		

% Corollary box
\newmdenv[skipabove=7pt,
skipbelow=7pt,
rightline=true,
leftline=false,
topline=false,
bottomline=true,
linecolor=gray,
backgroundcolor=black!5,
innerleftmargin=5pt,
innerrightmargin=5pt,
innertopmargin=5pt,
leftmargin=0cm,
rightmargin=0cm,
linewidth=1pt,
innerbottommargin=5pt]{rBox}				  
		  

% Creates an environment for each type of theorem and assigns it a theorem text style from the "Theorem Styles" section above and a colored box from above
\newenvironment{theorem}{\begin{tBox}\begin{theoremeT}}{\end{theoremeT}\end{tBox}}
\newenvironment{example}{\begin{exampleT}}{\hfill{\tiny\ensuremath{\blacksquare}}\end{exampleT}}
\newenvironment{definition}{\begin{dBox}\begin{definitionT}}{\end{definitionT}\end{dBox}}
\newenvironment{attention}{\begin{eBox}\small}{\end{eBox}}				  	
\newenvironment{exemple}{\begin{cBox}\small}{\end{cBox}}	

%----------------------------------------------------------------------------------------
%	REMARK ENVIRONMENT
%----------------------------------------------------------------------------------------

\newenvironment{remarque}{\par\vskip10pt\small
\begin{rBox}
\begin{list}{}{
\leftmargin=35pt % Indentation on the left
\rightmargin=25pt}\item\ignorespaces % Indentation on the right
\makebox[-2.5pt]{\begin{tikzpicture}[overlay]
\node[draw=ocre!60,line width=1pt,circle,fill=ocre!25,font=\sffamily\bfseries,inner sep=2pt,outer sep=0pt] at (-15pt,0pt){\textcolor{ocre}{R}};\end{tikzpicture}} % Orange R in a circle
\advance\baselineskip -1pt}
{\end{list}\vskip1mm\end{rBox}\vskip5pt} % Tighter line spacing and white space after remark




\begin{document}
	\maketitle
	\section{Intérêts simples}
	\begin{verbatim}
       47500
---///---]------------------------------------------------]--///-------------------------->	
	   14/10/N                                         28/03/N+1
	\end{verbatim}
	\paragraph{Nombre de jours} $17+30+31+31+28+28 = 165$	
	\begin{eqnarray*}
		\frac{165}{360} \times 47500 \times 0.125 = 2721.35
	\end{eqnarray*}
	\remarque{
		\textbf{Intérêts simples} $Intérets = C \times Taux \times duree$\newline
		\textbf{Intérêts composés}	$C_n = C_0 \times (1+i)^n $
	}
	\section{Valeurs acquises à intérêts simples}
	\begin{eqnarray*}
		I &=& \frac{6 + 30 + 31 + 30 + 31 + 16}{360} \times 38000 \times (0.016 \times 12) = \textbf{2918.4}\\
		C_n &=& 38000+2918.4 = \textbf{40918.4}
	\end{eqnarray*}
	\section{Valeur acquise à intérêts composés}
	\begin{eqnarray*}
		C_n &=& C_0 \times (1+i)^n\\
		C_8 &=& 10000 \times (1 + 0.125)^{8} = \textbf{25657.85}
	\end{eqnarray*}
	\section{Capitalisation semestrielle}
		\begin{eqnarray*}
			C_n &=& C_0 \times (1+i)^n\\
			C_7 &=& 10000 \times (1 + 0.06)^{7\times2} = \textbf{22609.04}
		\end{eqnarray*}
	\section{Valeur actuelle à intérêts composés}

	\begin{eqnarray*}
		535897 &=& C_1 \times (1 + 0.10)^{8}\\
		C_1 &=& \frac{535897}{(1 + 0.10)^8} = \textbf{249999.91}  
	\end{eqnarray*}
	\section{Valeur actuelle d'un capital}
	On ramène toutes les dépenses en $0$ c'est-à-dire au 1$^{er}$ Janvier 2010, on se pose donc la question \textit{``Combien fallait il placer en $0$ pour pouvoir ensuite faire ces dépenses ?''}. \newline
	On choisira donc la solution, qui demandera le placement le plus faible.
	\paragraph{Solution A}
	$400000 - (3\%~de~400000)~en~0 = 388000$
	\paragraph{Solution B}
	$215000 \times (1.09)^0 + 215000 \times (1.09)^{-1} = 412247.7$
	\paragraph{Solution C}
	$150000 \times (1.09)^{-1} + 150000 \times (1.09)^{-2} + 150000 \times (1.09)^{-3} = 379694.18$
	\paragraph{Conclusion}
	On constate que la solution qui demande le moins d'argent en $0$ est la solution C.  C'est donc la plus intéressante pour l'entreprise.
	\section{Emprunt remboursé par amortissements constants}
	\begin{table}[H]
		\centering
		\begin{tabular}{|p{2cm}|p{3.6cm}|c|c|c|p{3.2cm}|}
			\hline
			\textbf{Échéance} & \textbf{Capital restant dû en début de période} & \textbf{Intérêts} & \textbf{Amortissements}& \textbf{Semestrialité} & \textbf{Capital restant dû en fin de période }\\
			\hline
			31/08/N & 100000& 10000&20000&30000&80000\\
			\hline
			28/02/N+1 & 80000&8000&20000&28000 & 60000\\
			\hline
			31/08/N+1 & 60000&6000&20000&26000&40000\\
			\hline
			28/02/N+2 & 40000& 4000&20000&24000&20000\\
			\hline
			31/08/N+2 &20000&2000&20000&22000 & 0\\ 
			\hline
		\end{tabular}
		\caption{Tableau d'amortissement}
	\end{table}
	\section{Emprunt remboursé par annuités constantes}
	\remarque{
			\textbf{Formule}:  

			$ a = V_0 \times \frac{i}{1-(1 \times i)^{-n}} $
	}
\begin{eqnarray*}
	a = 150\;000 \times \frac{0.10}{1-1.10^{-5}} = 39\;569.62 &\approx& 39\;570
\end{eqnarray*}
	\begin{table}[H]
		\centering
		\begin{tabular}{|p{2cm}|p{2.8cm}|c|c|c|p{2.8cm}|}
			\hline
			\textbf{Échéance} & \textbf{Capital restant dû en début } & \textbf{Intérêts annuels} & \textbf{Amortissements}& \textbf{Annuités} \\
			\hline
			1/3/N+1 & $200\;000$&$28\;000$& $40\;641$ & $68\;641$\\ 
			\hline
			1/3/N+2 & $159\;359$&$22\;310.26$& $46\;330.74$ & $68\;641$\\ 	
			\hline
			1/3/N+3 & $113\;028.26$&$15\;823.95$& $52\;817641.04$ & $68\;641$\\ 
			\hline
			1/3/N+4 & $60\;211.21$&$8\;429.57$& $60\;211.21$ & $68\;640.78$\\ 
			\hline
		\end{tabular}
		\caption{Tableau d'amortissements}
	\end{table}
	\section{Étude d'un financement}
	\section{Financement de l'investissement}
	\begin{table}[H]
		\centering
		\begin{tabular}{|p{2cm}|p{2.8cm}|c|c|c|p{2.8cm}|}
			\hline
			\textbf{Échéance} & \textbf{Capital restant dû en début } & \textbf{Intérêts annuels} & \textbf{Amortissements}& \textbf{Annuités} \\
			\hline
			01/01/2011 & $80\;000$		& $84\;000$		&	$17\;111$		&	$25\;511$\\
			\hline
			01/01/2012 & $62\;889$		& $6\;603.35$	&	$18\;907.65$	&	$25\;511$\\
			\hline
			01/01/2013 & $43\;981.35$	& $4\;618.04$	&	$20\;892.96$	&	$25\;511$\\
			\hline
			01/01/2014 & $23\;088.39$	& $2\;424.28$	&	$23\;088.39$	&	$25\;512.67$\\
			\hline
		\end{tabular}
		\caption{Tableau d'amortissements}
	\end{table}

\end{document}






