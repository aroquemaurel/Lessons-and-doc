\chapter*{Document de synthèse}
	\addcontentsline{toc}{chapter}{Document de synthèse}
	\paragraph{}
	La société \K{} (Logo fig. \ref{fig:logo}) conçoit des solutions logicielles pour les 
	\glo{TPE}{TPE}{Les très petites entreprises (TPE) sont en France une appellation des entreprises de moins de 20 salariés}\footnote{\textbf{T}rès \textbf{P}etites \textbf{Entreprises}} et 
	les \glo{PME}{PME}{Les petites et les moyennes entreprises sont des entreprises dont la taille, définie à partir du nombre d'employés, du bilan ou du chiffre d'affaires, ne dépasse pas certaines limites ; les définitions de ces limites diffèrent selon les pays.}\footnote{\textbf{P}etites et \textbf{M}oyennes \textbf{E}ntreprises} du \glo{génie climatique}{Génie climatique}{
Le génie climatique est une branche de la physique qui traite du domaine du chauffage, de la plomberie, de la climatisation et du gaz et de ses applications. L'étude de domaine se réalise en physique, l'application se fait dans le domaine industriel.}. 
	Une des problématiques de ce secteur est la réalisation d’un bilan thermique précis.\\
	À l'heure actuelle, les professionnels n'ont à leu disposition que peu d'outils :
	\begin{itemize}
		\item Tableurs Excel réalisés en interne
		\item Logiciels réglementaires très couteux (plusieurs milliers d'euros)
	\end{itemize}

	Dans le premiers cas, les résultats sont approximatifs et dans le contexte actuel de maitrise 
	de l'énergie, il devient primordial de pouvoir réaliser ces calculs de façon précise.\newline
	Au niveau des logiciels payants, les calculs sont précis mais leur prix reste un obstacle à 
	leur utilisation pour la quasi totalité des entreprises visées.  De plus, l'ergonomie de ces 
	outils les rend difficile d’usage.

	\paragraph{}
	L'idée de ce projet est née à travers diverses expériences dans le secteur du génie climatique.
	Une première tentative avait été menée durant l'été 2009. Ce projet avait été soutenu par des 
	professionnels (entreprise A.S.O à Toulouse) qui avait mis à disposition des locaux et avait 
	fait une proposition d’actionnariat. Cependant, le projet n’a pas pu être mené à son terme par 
	manque de temps et d’organisation.\\
	Le projet a été relancé en octobre 2010, à la suite de divers échanges avec des professionnels.

	\paragraph{}
	Aujourd'hui une nouvelle équipe a été formée et une nouvelle méthode de calcul a été adoptée.

	\paragraph{}
	L'objectif est de distribuer gratuitement les solutions logicielles proposées par \K{} qui 
	seront téléchargeables sur le site internet de la société.  En contre partie, des encarts 
	publicitaires seront réservés aux fabricants de système de chauffage et de climatisation sur 
	la fenêtre d’accueil (splach screen) et sur la fenêtre principale du logiciel.

  \begin{figure}[H]
	  \begin{center}
	\includegraphics[width=3.6cm]{images/logo.png}
	  \end{center}
	  \caption{Logo de \K{}}
	  \label{fig:logo}
  \end{figure}

