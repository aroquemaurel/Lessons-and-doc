	\chapter{Analyse des risques}
  \begin{table}[H]
	  \centering
	  \rowcolors{2}{grisclair}{grisfonce}
	  \begin{tabular}{|p{4.8cm}|c|p{4.8cm}|p{4.8cm}|}
		  \hline
		  \textbf{Risques} & \textbf{Pertinence} & \textbf{Coût} & \textbf{Solution} \\
			Retard dans l'aboutissement du logiciel & 	
			Haute & 
			Tant que le logiciel n'est pas abouti, aucune rentrée d'argent ne peut être envisagée.	&
			Utiliser une méthode de développement avec des rendus réguliers et des réunions de développement fréquentes, afin qu'il n'y ai pas de temps mort.\\
			Limite des compétences des membres de l'équipe & Moyenne & Impact sur la qualité du logiciel. Retard. &
			Analyser les compétences requises. Prévoir un temps de documentation et d'auto formation.\\

			Nombre de téléchargements insuffisants & 	
			Moyenne & 
			Les annonceurs ne sont pas intéressés $\Rightarrow$ Moins de rentrées d'argent.& Proposer une offre gratuite et appuyer la communication.\\

			Offre média non attrayante & 	
			Moyenne & Moins de publicité $\Rightarrow$ Moins d'argent 
			& Utiliser un support innovant et interactif avec un temps de visibilité plus important que les offres actuelles. Possibilité de cibler la clientèle.\\
			Dégâts dans les locaux & Faible & Perte de données. Plus de bâtiments & Faire des sauvegardes externes régulièrement. Prévoir un lieu de secours pour pouvoir travailler.\\
			Impossibilité d'un membre de l'équipe à travailler & Moyenne & Retard dans l'avancée du projet 
			& Prévenir les clients d'un éventuel retard. Nouvelle répartition du travail dans l'équipe.\\ 
			Arrivée d'un nouveau concurrent & Forte & Perte de parts de marché. Perte du monopole. & Réagir rapidement et adapter les tarifs. Trouver des solutions innovantes.\\
			\hline
	  \end{tabular}
	  \caption{Les différents risques qui pourrait gêner la bonne avancée du projet}
	  \label{tab:risques}
  \end{table}
