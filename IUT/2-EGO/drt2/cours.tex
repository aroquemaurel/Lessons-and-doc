\documentclass{article}

\usepackage{lmodern}
\usepackage{xcolor}
\usepackage[utf8]{inputenc}
\usepackage[T1]{fontenc}
\usepackage[francais]{babel}
\usepackage[top=1.7cm, bottom=1.7cm, left=1.7cm, right=1.7cm]{geometry}
%\usepackage[frenchb]{babel}
%\usepackage{layout}
%\usepackage{setspace}
%\usepackage{soul}
%\usepackage{ulem}
%\usepackage{eurosym}
%\usepackage{bookman}
%\usepackage{charter}
%\usepackage{newcent}
%\usepackage{lmodern}
%\usepackage{mathpazo}
%\usepackage{mathptmx}
%\usepackage{url}
%\usepackage{verbatim}
%\usepackage{moreverb}
%\usepackage{wrapfig}
%\usepackage{amsmath}
%\usepackage{mathrsfs}
%\usepackage{asmthm}
%\usepackage{makeidx}
\usepackage{listings}
\usepackage{fancyhdr}
\usepackage{multido}
\usepackage{amssymb}

\definecolor{gris1}{gray}{0.40}
\definecolor{gris2}{gray}{0.55}
\definecolor{gris3}{gray}{0.65}
\definecolor{gris4}{gray}{0.50}


\lstdefinelanguage{algo}{%
   morekeywords={%
    %%% couleur 1
		importer, programme, glossaire, fonction, procedure, constante, type, 
	%%% IMPORT & Co.
		si, sinon, alors, fin, tantque, debut, faire, lorsque, fin lorsque, declancher, enregistrement, tableau, retourne, retourner, =, /=, <, >, traite,exception, 
	%%% types 
		Entier, Reel, Booleen, Caractere,
	%%% types 
		entree, maj, sortie,	
	%%% types 
		et, ou, non,
	},
  sensitive=true,
  morecomment=[l]{--},
  morestring=[b]',
}

%\lstset{language=algo,
    %%% BOUCLE, TEST & Co.
%      emph={importer, programme, glossaire, fonction, procedure, constante, type},
%      emphstyle=\color{gris2},
    %%% IMPORT & Co.
%      emph={[2]si, sinon, alors, fin , tantque, debut, faire, lorsque, fin lorsque, declancher, retourner, et, ou, non,enregistrement, retourner, retourne, tableau, /=, <, =, >, traite,exception},
%      emphstyle=[2]\color{gris1},
    %%% FONCTIONS NUMERIQUES
%      emph={[3]Entier, Reel, Booleen, Caractere},
%      emphstyle=[3]\color{gris3},
    %%% FONCTIONS NUMERIQUES
%      emph={[4]entree, maj, sortie},	
%      emphstyle=[4]\color{gris4},
%}
\lstset{ % general style for listings 
   numbers=left 
	, extendedchars=\true
   , tabsize=2 
   , frame=single 
   , breaklines=true 
   , basicstyle=\ttfamily 
   , numberstyle=\tiny\ttfamily 
   , framexleftmargin=13mm 
   , xleftmargin=12mm 
   , captionpos=b 
	, language=algo
	, keywordstyle=\color{blue}
	, commentstyle=\color{green}
	, showstringspaces=false
	, extendedchars=true
	, mathescape=true
} 
 %prise en charge du langage algo

\title{Droit}
\date{Droit\\ Semestre 2}

\lfoot{Université paul sabatier Toulouse III}
\cfoot{\thepage}
\rfoot{Drt2}

\pagestyle{fancy}
\begin{document}
	\maketitle
	Le droit est les règles permettant de vivre en société, il y a longtemps elles étaient fixés par l'église, maintenant elles sont fixés par les états. \\
	« ub societas, ubi jus » là ou il y a des sociétés il y a des droits! \\
	le droit à plusieurs fonctions:
	\begin{itemize}	
		\item \textbf{permettre} (ex: vote, divorce)
		\item \textbf{interdir} (ex: meurture)
		\item \textbf{préscrire} (ex: assurance obligatoire)
	\end{itemize}
	C'est le pouvoir legistlatif qui décide les droits, que ça soit le droit économique ou familliale. \\
	Un droit n'est pas forcément écrit, il peut être oral, cependant les français se sont appuyés sur le droit Romain et est donc toujours écrit.

	\section{Les branches du droit}
		\subsection{Le droit public (intervention de l'état)}
			\subsubsection{International}
				C'est l'\textbf{ONU} depuis 1946 qui est en est chargé cependant il est très difficilement applicable, en effet il n'y a pas `vraiment` de sanctions, les différents états ne le respecte pas. 
					(ex: interdit de faire la guerre sauf pour se défence)\\
				Tribunal ad-hoc: c'est un tribunal international qui est supprimé après le jugement, il est créer pour des cas exceptionnel (ex: après la IIe guerre mondial pour les nazis).
			\subsubsection{National}
				\paragraph{Droit constitutionnel}
					Il indique comment fonctionne le pouvoir dans un pays, il est régit par la Ve république (en France)
				\paragraph{Droit administratif}
					C'est le droit entre l'état et ses citoyens.\\
					Il contient le \textbf{droit fiscal} qui concerne les impots.
		\subsection{Droit privé (particuliers)}
			\subsubsection{International}
				Il gère les rapports entre les particuliers de pays différents. (ex: divorce entre conjoints de nationalité différents)
			\subsubsection{National}
				\paragraph{Droit civil}
					Le droit entre les particuliers (familles(mariage...))
				\paragraph{Droit des sociétés}
					Par exemple quand on veut monter une entreprise.
				\paragraph{Droit du travail}
					Les rapports entre salariés et employeurs
				\paragraph{Droit de l'informatique}
					Droit d'auteurs, téléchargement etc...
				\paragraph{Droit de l'immobilier}
				\paragraph{Droit banquaire}
				\paragraph{Droit pénal}
					C'est le droit des sanctions.\\
					Il y a trois types de violation de la loie faisant référence dans le code pénal:
					\begin{itemize}
						\item \textit{Infractions}. On est pas jugé. On recoit une contravention. en cas de littige \textit{Tribunal de Police}.
						\item \textit{Délits}. jugés en \textit{Tribunal correctionnel}. 
						\item \textit{Crime}. jugés en \textit{cours d'assise}.
					\end{itemize}				
					
\end{document}
