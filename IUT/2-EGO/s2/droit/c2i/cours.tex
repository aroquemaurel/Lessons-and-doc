\documentclass{article}

\usepackage{lmodern}
\usepackage{xcolor}
\usepackage[utf8]{inputenc}
\usepackage[T1]{fontenc}
\usepackage[francais]{babel}
\usepackage[top=1.7cm, bottom=1.7cm, left=1.7cm, right=1.7cm]{geometry}
%\usepackage[frenchb]{babel}
%\usepackage{layout}
%\usepackage{setspace}
%\usepackage{soul}
%\usepackage{ulem}
%\usepackage{eurosym}
%\usepackage{bookman}
%\usepackage{charter}
%\usepackage{newcent}
%\usepackage{lmodern}
%\usepackage{mathpazo}
%\usepackage{mathptmx}
%\usepackage{url}
%\usepackage{verbatim}
%\usepackage{moreverb}
%\usepackage{wrapfig}
%\usepackage{amsmath}
%\usepackage{mathrsfs}
%\usepackage{asmthm}
%\usepackage{makeidx}
\usepackage{listings}
\usepackage{fancyhdr}
\usepackage{multido}
\usepackage{amssymb}

\definecolor{gris1}{gray}{0.40}
\definecolor{gris2}{gray}{0.55}
\definecolor{gris3}{gray}{0.65}
\definecolor{gris4}{gray}{0.50}


\lstdefinelanguage{algo}{%
   morekeywords={%
    %%% couleur 1
		importer, programme, glossaire, fonction, procedure, constante, type, 
	%%% IMPORT & Co.
		si, sinon, alors, fin, tantque, debut, faire, lorsque, fin lorsque, declancher, enregistrement, tableau, retourne, retourner, =, /=, <, >, traite,exception, 
	%%% types 
		Entier, Reel, Booleen, Caractere,
	%%% types 
		entree, maj, sortie,	
	%%% types 
		et, ou, non,
	},
  sensitive=true,
  morecomment=[l]{--},
  morestring=[b]',
}

%\lstset{language=algo,
    %%% BOUCLE, TEST & Co.
%      emph={importer, programme, glossaire, fonction, procedure, constante, type},
%      emphstyle=\color{gris2},
    %%% IMPORT & Co.
%      emph={[2]si, sinon, alors, fin , tantque, debut, faire, lorsque, fin lorsque, declancher, retourner, et, ou, non,enregistrement, retourner, retourne, tableau, /=, <, =, >, traite,exception},
%      emphstyle=[2]\color{gris1},
    %%% FONCTIONS NUMERIQUES
%      emph={[3]Entier, Reel, Booleen, Caractere},
%      emphstyle=[3]\color{gris3},
    %%% FONCTIONS NUMERIQUES
%      emph={[4]entree, maj, sortie},	
%      emphstyle=[4]\color{gris4},
%}
\lstset{ % general style for listings 
   numbers=left 
	, extendedchars=\true
   , tabsize=2 
   , frame=single 
   , breaklines=true 
   , basicstyle=\ttfamily 
   , numberstyle=\tiny\ttfamily 
   , framexleftmargin=13mm 
   , xleftmargin=12mm 
   , captionpos=b 
	, language=algo
	, keywordstyle=\color{blue}
	, commentstyle=\color{green}
	, showstringspaces=false
	, extendedchars=true
	, mathescape=true
} 
 %prise en charge du langage algo

\title{Droit de l'informatique}
\date{Droit\\ Semestre 2}
\lhead{Droit de l'informatique}
\chead{}
\rhead{}
\lfoot{Université Paul Sabatier Toulouse III}
\cfoot{\thepage}
\rfoot{C2I}

\pagestyle{fancy}
\begin{document}
	\maketitle
	Le droit de l'informatique est très récent, il est survenu avec l'arrivée
	des nouvelles technologies.\\
	\section{Cybercriminalité}
		\subsection{données}
		Plusieurs domaines:
		\begin{itemize}
			\item Données (sites pouvant revendre les données)
			\item Arnaques au fishing.
		\end{itemize}
		Chiffre d'affaires fait par les pirates constants.\\
		Nouvelle forme de droit, donc impact économique(Délit: payer/réparer)
		\subsection{Lutte}
			Mise en place de laboratoires spécialisés, test d'antivirus, 
			d'antispam.
			\subsubsection{Tribunaux}
				Quel droit doit il être appliqué sur internet? \\
				Le droit appliqué est le droit de la victime, 
				cependant, retrouver le coupable sur internet reste difficile.
				La sanction est donc très dure. 
				\paragraph{Exemple contrefaçon} le tarif est de 45000 euros et 3 ans 
				de prison.
	\section{Libertés sur le net}
		La liberté dépend beaucoup du pays. 62 pays qui censurent, 10 qui se sont
		déclarés ennemies d'Internet (ex Chine et cuba).\\
		Cela dépend beaucoup de l'état, aucune loi commune.\\
		La liberté d'expression n'est pas universelle, dépend de référence
		politique, idéologique et religieuse. \\
		Les libertés dépendent donc des pays. 
		\begin{itemize}
			\item Pornographie interdit partout. 	
			\item Problème du terrorisme
			\item Diffamation interdite
			\item Injures 
		\end{itemize}
		Ce qui est donc interdis en général sera interdis en général.\\
		Protection des mineurs sur internet.\\
		Aucune loi unique sur internet, attention à la vitesse de transfert
		des informations.
	\section{Données}
		Professeur en informatique à prouvé que grâce à Twitter il a put
		faire un résumé de la vie des gens, Google le fait lui aussi.\\
		La publicité est ciblée grâce aux données récupérés (ex: Facebook)\\
		On a le droit d'avoir des dossiers ou mails personnel sur notre ordinateur
		professionnel, l'employeur ne peut regarder le dossier, sauf s'il pense 
		que cela peut causer un dommage pour l'entreprise. Il peut connaître les 
		sites que l'on a visité, ainsi il y a souvent présence d'une charte 
		informatique. \\
		Toutes les données sont sensibles, tout dépend de quelle façon on va 
		l'utiliser.\\
		Google stock énormément de données.\\
		Présence en France d'une Commission National de l'Informatique et Libertés.\\
	\section{Droits d'auteur}
		Mise en place de Hadopi qui essaye de valoriser les artistes et 
		réprimander les personnes ne respectant pas les auteurs.\\
		Projet de mise en place de plateformes de téléchargement, légal, qui 
		rémunèrera les auteurs de musique, avec une paye mensuel.\\
		Les droits d'auteurs sont toutes les droits sur toutes les œuvres de 
		l'esprit. On a le droit de copier une œuvre si on dispose de l'original
		et que elle reste dans le cadre privé.
	\section{Protection des consommateurs}
		Commerce électronique se développe,	problème d'acheter un objet avec 
		une photo (taille...). Problème des sécurités des données (CB...).
		Quelles est le droit applicable? 
	\section{Responsabilité des fournisseurs}
	Trois types de fournisseurs, fournisseur d'accès, fournisseur de contenu et
	hébergeurs. Si l'hébergeur est au courant des données illicites sa sécurité 
	peut être engagés.
	\section{Sécurités}
	Enjeux politiques, enjeux économiques.\\
	Sécurité de l'intégrité des données. Contrôle, prévention, filtrage.
	\paragraph{Exemple}Coupure d'Internet pendant 24h
	\section{Brevets}		
	Protection de la propriété intellectuelle. 

\end{document}

