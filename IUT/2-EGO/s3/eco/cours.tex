\documentclass[12pt,a4paper,openany]{report}

\usepackage{lmodern}
\usepackage{xcolor}
\usepackage[utf8]{inputenc}
\usepackage[T1]{fontenc}
\usepackage[francais]{babel}
\usepackage[top=1.7cm, bottom=1.7cm, left=1.7cm, right=1.7cm]{geometry}
\usepackage{verbatim}
\usepackage{tikz} %Vectoriel
\usepackage{listings}
\usepackage{fancyhdr}
\usepackage{multido}
\usepackage{amssymb}

\newcommand{\titre}{Contrat de travail}

\newcommand{\module}{Économie et droit}
\newcommand{\sigle}{eco}

\newcommand{\semestre}{3} 

\input{/home/satenske/cours/listings.tex} %prise en charge du langage algo
\input{/home/satenske/cours/entete_iut-cours.tex}

\begin{document}
	\maketitle

	\paragraph{Definition} Un accord pour lequel une personne (salarié) s'engage à travailler pour le compte et sous la direction d'une autre personne (employeur) qui en contrepartie lui verse une rémunération (salaire)
licenciement 
	\section{Formation}
	\paragraph{Pour le CDI} prestation + rémunération + lien de subordination. 
	Pas de trace écrite nécessaire si les 3 conditions ci-dessus sont réunies

	\paragraph{Pour le CDD} Un contrat écris est obligatoire, sinon il sera pris en compte comme un CDI

	\section{Contenu}
	\subsection{ Obligation du salarié}
		\begin{itemize}
			\item Exécuter le travail convenu
			\item Prendre soin du matériel et de l'outillage qui lui sont confiés
			\item Ne pas révéler les "secrets de fabrication" (au sens large) dont il pourrait avoir connaissance
			\item Respecter les ordres reçus et le réglement interrieur
		\end{itemize}

		\subsection{Obligations de l'employeur}
		\begin{itemize}
			\item  Fournir le travail convenu
			\item  Payer le salaire
			\item  Respecter la législation sociale, les conventions collectives et accords de l'entreprise
		\end{itemize}

	\subsection{Clauses particulières}
	\begin{itemize}
		\item Periode d'essai : l'employeur et le salarié peuvent à tout moment rompre le contrat de travail sans raison.
		\item Non concurrence : ne peut pas travailler simultanement chez le concurrent (meme pendant 1 ans)
		\item Resultat : dangeureuse car peut entrainer un licenciment immediat (si non respect des objectifs)
		\item Mobilité : l'employeur peut demander a aller bosser a Budapest
		\item Propriétée intelectuelle : 
	\end{itemize}

	\subsection{Forme du contrat}
		Le CDD est  possible si: 
		\begin{itemize}
			\item remplacement d'un salarié absent
			\item accroissement temporaire de l'entreprise
			\item travaux saisonniers
			\item si le métier s'y prête (intermittent du spectacle, enseignement)
		\end{itemize}
			Le CDD ne peut pas excéder 2 ans
			CTT=Contrat de Travail Temporaire

			\subsection{Évènements affectant le contrat de travail (suspension)}
			\begin{itemize}
				\item Maladie (payé par la sécurité sociale)
				\item  Maternité, 6 semaines avant l'accouchement, 10 semaines après
				\item  Grève
				\item  Appel du drapeau
				\item Cas de forces majeures comme destruction des batiments (11 septembre par exemple)
				\item  Entreprise rachetée et donc restructuration de l'entreprise
			\end{itemize}
		\section{Les pouvoirs de l'employeur}
		\begin{itemize}
			\item Direction 
			\item Réglementation
			\item Disciplinaire 
		\end{itemize}
		\section{Rupture du contrat}
		\begin{itemize}
			\item Licenciement pour fait personnel du salarié: Raison réél et justifiée
			\item Licenciement pour motif économique
		\end{itemize}

\end{document}

