\documentclass[12pt,a4paper,openany]{report}
%%%% JNLP 
\usepackage{lmodern}
\usepackage{xcolor}
\usepackage[utf8]{inputenc}
\usepackage[T1]{fontenc}
\usepackage[francais]{babel}
\usepackage[top=1.7cm, bottom=1.7cm, left=1.5cm, right=1.5cm]{geometry}
\usepackage{pdfpages}
\usepackage{listingsutf8}
\usepackage{fancyhdr}
\usepackage{multido}
\usepackage{amssymb}
\usepackage{tikz}
\usepackage{makeidx}
\usepackage[urlbordercolor={1 1 1}, linkbordercolor={1 1 1}, colorlinks=none, linktoc=all,urlcolor=white]{hyperref}
\usepackage{eurosym}

\newcommand{\titre}{Contrôle de gestion}
\newcommand{\module}{Contrôle de gestion}
\newcommand{\sigle}{ctg}
\newcommand{\semestre}{3}
\date{\today}

\chead{}
\rhead{--~ \thepage ~--}
\lhead{\titre}
\makeindex
\lfoot{Université Paul Sabatier Toulouse III}
\rfoot{\sigle\semestre}
%\rfoot{}
\cfoot{}
\makeglossary
\makeatletter
\def\clap#1{\hbox to 0pt{\hss #1\hss}}%
\def\ligne#1{%
\hbox to \hsize{%
\vbox{\centering #1}}}%
\def\haut#1#2#3{%
\hbox to \hsize{%
\rlap{\vtop{\raggedright #1}}%
\hss
\clap{\vtop{\centering #2}}%
\hss
\llap{\vtop{\raggedleft #3}}}}%
\def\bas#1#2#3{%
\hbox to \hsize{%
\rlap{\vbox{\raggedright #1}}%
\hss \clap{\vbox{\centering #2}}%
\hss
\llap{\vbox{\raggedleft #3}}}}%
\def\maketitle{%
\thispagestyle{empty}\vbox to \vsize{%
\haut{}{\@blurb}{}

\vfill
\vspace{1cm}
\begin{flushleft}
\usefont{OT1}{ptm}{m}{n}
\huge \@title
\end{flushleft}
\par
\hrule height 4pt
\par
\begin{flushright}
\usefont{OT1}{phv}{m}{n}
\Large \@author
\par
\end{flushright}
\vspace{1cm}
\vfill
\vfill
\bas{}{\@location, le \@date}{}
}%
\cleardoublepage
}
\def\date#1{\def\@date{#1}}
\def\author#1{\def\@author{#1}}
\def\title#1{\def\@title{#1}}
\def\location#1{\def\@location{#1}}
\def\blurb#1{\def\@blurb{#1}}
\date{\today}
\author{}
\title{}
\location{Amiens}\blurb{}
\makeatother
\title{\titre}
\author{Semestre \semestre}

\location{Toulouse}
\blurb{%
Université Paul Sabatier -- Toulouse III\\
IUT A - Toulouse Rangueil\\
}%



%\title{Cours \\ \titre}
%\date{\today\\ Semestre \semestre}

%\lhead{Cours: \titre}
%\chead{}
%\rhead{\thepage}

%\lfoot{Université Paul Sabatier Toulouse III}
%\cfoot{\thepage}
%\rfoot{\sigle\semestre}

\pagestyle{fancy}


\pagestyle{fancy}
\begin{document}
	\maketitle
	\chapter{Seuil de rentabilité (SR)}
	La méthode du seuil de rentabilité pose la question à partir de quel chiffre on a un résultat positif.
	\begin{eqnarray*}
		CA = Px \times QtéVendue\\
	\end{eqnarray*}
	\section{Charges fixes et charges variables}
	\paragraph{Charges} Flux fixes entrantes sauf les immobilisations

	\subsection{Les charges fixes (Charges de structure)}
	Le montant est indépendant de l'activité
	\subparagraph{Exemple} Amortissement, loyer du bâtiment, éclairage
	\subparagraph{Remarque} La somme des charges fixes est appelée le coût fixe \\ $\sum{chargesFixes} = coûtFixe$ 

	\subsection{Charges variables (Charges opérationnelles)}
	Le montant est proportionnelle au niveau d'activité
	\paragraph{Remarque} La somme des charges fixes est appelée le coût variable \\ $\sum{chargesVariables} = coûtVariable$
	\paragraph{Niveau d'activité} Les entreprises qui ne vendent qu'un seul produit peuvent mesurer leur activité par le nombre de produits vendus
	\subparagraph{Exemple} Le producteur de lait. 
	\paragraph{} Ces mesures présentent l'avantage d'être indépendant du prix de vente, cependant dans la plupart des entreprises, la production est diversifiée, 
	on prend alors le chiffre d'affaire pour mesurer la variation d'activités.
	\section{La marge sur coût variable (MCV)}
	Une marge est toujours la différence entre un chiffre d'affaire et un coût. 
	\begin{eqnarray*}
		MCV = CA - CV
	\end{eqnarray*}
	\begin{center}
	\begin{tabular}{|c|}
		\hline
	CV      \\
		\hline
	CF  \\
		\hline
	Resultat\\ 
		\hline
	\end{tabular}
	\hspace{1cm}
	\begin{tabular}{|c|}
		\hline
		~ \\	
		CA \\
		~ 	\\
		\hline
	\end{tabular}
	\end{center}
	\paragraph{Remarque} La marge sur coût variable correspondrait au résultat, si il n'y avait pas de coût fixe. 
	\section{Le Taux de Marge sur coût Variable (TMCV)} 
	\begin{eqnarray*}
		TMCV = \frac{MCV}{CA} = \frac{CA - CV}{CA} = Constante
	\end{eqnarray*}
	Il correspond au montant de la marge sur coût variable pour un euro de chiffre d'affaire. 
	\section{Détermination du seuil de rentabilité}
	Le seuil de rentabilité est un chiffre d'affaire, en effet il correspond au chiffre d'affaire pour lequel le résultat est égal = 0
	\begin{eqnarray*}
		TMCV &=& \frac{MCV}{CA}\\
		\Leftrightarrow MCV &=& TMCV \times CA\\
		lqR &=& 0\\
		MCV &=& CF\\
		CA &=& SR\\
		CF &=& TMCV \times CA\\
		CF &=& TMCV \times SR\\
		SR &=& \frac{CF}{TMCV}\\
	\end{eqnarray*}
	\paragraph{Remarque}La différence entre le chiffre d'affaire et le seuil de rentabilité correspond au montant du chiffre
	d'affaire qui pourrait diminuer avant que l'on soit en perte, on l'appelle la \textbf{marge de sécurité}
	\begin{eqnarray*}
	Ms = CA-SR
\end{eqnarray*}
	\paragraph{Détermination de la date du seuil de rentabilité}
	Lorsque nous avons le choix entre plusieurs projets, on peut déterminer le projet le moins risqué.
	Ce sera bien évidemment celui qui atteint le SR \textit{le plus tôt}.
	\begin{eqnarray*}
		DateSR_{en~mois} &=& \frac{SR}{CA} * 12\\
	\end{eqnarray*}
	\paragraph{Remarque} En supposant que la chiffre d'affaire est constant. 
	\chapter{La fixation du prix de vente par la méthode des coûts complet}
	\section{La notion de coût complet} \label{coutcomplet}
	\subsection{La formation des coûts}
	\subsubsection{Le reclassement des charges}
	Dans cette méthode on ne distingue plus les charges variables et les charges fixes comme dans la méthode du seuil de rentabilité.
	On les regroupe maintenant selon la fonction à laquelle on considère qu'elles appartiennent.
	\begin{itemize}
		\item fonction acquisition
		\item fonction production
		\item fonction distribution
		\item fonction administration
	\end{itemize}
	\subsubsection{Le calcul des différents coûts}
	\paragraph{Exemple} Une menuiserie on va essayer de déterminer le coût de revient d'une table et donc
	son prix minimum de vente.
	\paragraph{Le coût d'acquisition}
	On a répertorié les charges suivantes qui appartiennent à la fonction d'acquisition pour une table
	\begin{itemize}
		\item Prix d'achat du bois: 10\euro
		\item Coût d'acquisition des fournitures (peintures, colles, \ldots): 6\euro
		\item Charges d'approvisionnement : 8\euro (transport du bois, loyer du local où est stocké
			le bois, salaire du magasinier, etc\ldots)
	\end{itemize}
	\begin{center}$\Rightarrow$ Coût d'acquisition: 24\euro~pour une table.\end{center}
		\paragraph{Le coût de production}
		\subparagraph{Attention} le coût de production ne correspond pas uniquement aux charges de production !\\
		On doit lui additionner le coût d'acquisition. Ainsi le coût d'acquisition est inclus dans le coût de production.
		\begin{eqnarray*} Cout_{acquisition} + charges_{production} = cout_{production} \end{eqnarray*}
			\begin{center}
			\begin{tabular}{|c|c|}
				\hline
				coût d'acquisition & charges de production \\
				\hline
				\multicolumn{2}{|c|}{coût de production}\\
				\hline
			\end{tabular}
		\end{center}
		Dans notre exemple, le coût de production d'une table: 
		\begin{itemize}
			\item Coût acquisition : 24\euro
			\item Charges de production 50\euro~(salaires des employés dans les ateliers, éclairages des ateliers, loyers des ateliers, \ldots)
		\end{itemize}
	\begin{center}$\Rightarrow$ Coût de production: 74\euro~pour une table.\end{center}
		\paragraph{Le coût de distribution}
		Le coût de distribution d'une table comprend uniquement les charges de distribution comme les salaires
		des vendeurs, les charges d'emballages, la publicité \ldots

		Dans notre exemple on va les estimer pour une table à 20\euro .
	\begin{center}$\Rightarrow$ Coût de distribution: 20\euro~pour une table.\end{center}
		\paragraph{Le coût d'administration}
		Pour ce dernier coût on parle aussi de ``charges de structure''. 

		Dans notre exemple on va les estimer pour une table à 6\euro .
	\begin{center}$\Rightarrow$ Coût d'administration: 6\euro~pour une table.\end{center}
		\paragraph{Le coût complet aussi appelé coût de revient)}
		Le coût de revient est un coût complet puisqu'il comprend toutes les charges.

		Coût de revient d'une table : 
		\begin{itemize}
			\item Coût de production: 74\euro
			\item Coût de distribution: 20\euro 
			\item Charges administratives: 6\euro
		\end{itemize}
	\begin{center}$\Rightarrow$ Coût de revient: 100\euro.\end{center}
	\section{Conclusion}
	\begin{tabular}{|c|c|c|c|}
		\hline
		Coût d'acquisition & Charge de production &\multicolumn{2}{c|}{}\\
		\hline
		\multicolumn{2}{|c|}{Coût de production} & Coût de distribution & Charges administratives\\
		\hline
		\multicolumn{4}{|c|}{Coût de revient}\\
		\hline
	\end{tabular}
	\section{Le résultat analytique et les marges}
	 \subsection{Le résultat analytique (RA}
	 \begin{eqnarray*}
		 RA &=& CA - Coût_{revient~total} ~ ~ où\\ 
		 RA_{unitaire} &=& prix_{vente} - coût_{revient~unitaire}
	 \end{eqnarray*}
	 \subsection{Les marges}
	 \paragraph{Rappel} Une marge est la différence entre un prix de vente et un coût partiel.
	 \subparagraph{Exemple} $Marge~sur~coût~de~distribution = prix_{vente} - coût_{distribution} $
	 \section{Les charges directes et indirectes}
	 \subsection{La distinction}
	 Les charges directes sont celles qui sont affectées à un coût précis, sans aucun doute ni arbitraire; cf section \ref{coutcomplet}.
	 \paragraph{Exemple} Les commissions des représentants $\Rightarrow$ coût de distribution.\\
	 Les charges ``indirectes'' sont celles qui concernent plusieurs coût à la fois ; il n'est pas possible de les
	 affecter à un coût particulier.
	 \paragraph{Exemple} Les charges du chauffage central, concernent les ateliers et l'administration. 
	 \textit{ Coût de production ou coût d'administration ? }

	 Les charges indirectes seront réparties dans un $1^{er}$ temps dans des divisions comptables de l'entreprise: ``les centres d'analyses''.\\
	 Puis dans un $2^{nd}$ temps, elles seront affectées à un coût.\\ \\
	 \begin{tikzpicture}
		\draw (0,0) rectangle (3.5,2) ;
		 \node at(1.7,1.5) {Charges directes} ;
		 \draw (0,1) -- (3.5,1);
		 \node at(1.7,0.5) {Charges indirectes} ;

		\draw (15.5,0) rectangle (11.6,2) ;
		 \node at(13.5,1.0) {Coûts des produits} ;
		 \draw[>=latex, ->, thick] (3.5,1.5) -- (11.6,1.5);

		 \draw[>=latex, ->, thick] (3.5,0.5) -- (5.6,0.5);
		 \node at(7.2,0.5) {Centre d'analyses} ;
		 \draw[>=latex, ->, thick] (9.0,0.5) -- (11.6,0.5);
	 \end{tikzpicture}
	 \subsection{Les centres d'analyse}
	 Un centre d'analyse correspond à une division de l'entreprise où sont réparties les charges indirectes \textit{avant} leur affectation
	 aux différents coûts des produits.
	 \section{La répartition des charges indirectes}
	 La répartition est présentée dans la réalité sous la forme d'un table ``tableau d'analyse des charges indirectes''
	 \begin{itemize}
		 \item ``La répartition primaire'' consiste à répartir les charges indirectes entre tous les centres d'analyses (y compris les centres auxiliaires)
		 \item ``La répartition secondaire'' consiste à répartir les charges des centres auxiliaires entre les autres centres d'analyses (centre principaux : 
			 approv, prod, distribution et les centres de structure).
	 \end{itemize}
	 \begin{table}
		 \begin{center}
	 \begin{tabular}{|p{3.5cm}|p{1.3cm}|p{2.96cm}|p{1.4cm}|p{1.4cm}|p{2.1cm}|p{3.7cm}|}
		 \hline
		 {Charges} & {Totaux} &{ Centre auxiliaire }& \multicolumn{3}{|c|}{{Centres principaux}} & {Centre de structure}\\
		 \hline
		   & &Prestation annexes & Découpe & Finition & Distribution & Administration \\
		 \hline
		 Achats non stockés (énergie\ldots) & $19800$ & $5000$ & $10000$ & $1500$ & $300$ & $3000$\\
		 \hline
		 Services extérieurs (assurances\ldots) & 45150 & 15000 & 2000 & 3850 &3800 &2500\\
		 \hline
		 Autres services extérieurs & 48000 & 15000& 5000& 7000&7000&14000\\
		 \hline
		 Impôts & 12000&2000&4000&2000&2000&2000\\
		 \hline
		 Charges de personnel & 57200&3000&20000&14000&14500&5700\\
		 \hline
		 Dotations aux amortissements & 22300&10000&1000&1400&9000&900\\
		 \hline
		 Total de la répartition primaire & 50000&60000&22750&36600&28100\\
		 \hline
		 Prestations annexes & & 20\% & 30\%& 40\%&10\%\\
		 \hline
		 Total de la répartition secondaire & & 70000&44750&56600&33100\\
		 \hline
		 Unité d'\oe{}uvre ou assiette des frais & & m$^2$ de bois découpés & Heures machines & Chiffre d'affaires global & Coût de production global \\
		 \hline
		 Nombre d'unités d'\oe{}uvre ou montant de l'assiette des frais & & 5000&\ldots&\ldots&\ldots\\
		 \hline
		 Coût d'une unité d'\oe{}uvre ou taux de frais & \ldots&\ldots&14\euro~par m$^2$ de bois découpé&lcdots&\ldots&\ldots\\
		 \hline

	 \end{tabular}
		 \caption{Répartition des charges indirectes}
		 \end{center}
	 \end{table}
	 \newpage
	 \section{L'attribution aux coûts (imputation}
	 On veut déterminer le coût de revient d'une chaine afin de connaître son prix de vente minimal.

	 Si on avait qu'un seul produit on additionnerait toutes les charges et on diviserait par le nombre de produit.

	 Le problème est qu'en fait comme la plupart des entreprises notre menuiserie fabrique des produits très différents les uns des autres (tables, escaliers, 
	 chaises, parquets, \ldots). On va donc essayer de mesurer le coût d'activité de chaque centre en connaissant la consommation d'activité de chaque produit.
	 \subsection{Les unités d'\oe{}uvre}
	 \subsubsection{Définition}
	 L'activité d'un centre d'analyse est mesurée si cela est possible par une unité physique appelée ``unité d'\oe{}uvre (\oe{}uvre synonyme de activité)
	 \paragraph{Exemple} Dans la menuiserie
	 \begin{itemize}
		 \item Atelier découpe: L'unité d'\oe{}uvre peut-être le volume de bois découpé afin de mesurer son activité.
		 \item Atelier finition : l'activité de ce centre d'analyse peut-être meusrée par le temps de travail des employés qui
			 procèdent aux finitions sur les meubles
	 \end{itemize}
	 \subsection{Le coût d'une unité d'\oe{}uvre}
	 \paragraph{Exemple} Quel est le coût de l'unité d'\oe{}uvre dans l'atelier ''découpé``, sachant que ce centre a découpé 5000m$^2$ de bois
	 au cours du mois de novembre.
	 Le total des charges du centre ''découpe`` est d'un montant de 70000\euro~\\
	 $\Rightarrow$ Coût d'une unité d'\oe{}uvre: $\frac{70000}{5000}=14$\euro\\
	 Ainsi on a la formule: $$coût_{unité~oeuvre~centre~analyse} = \frac{Total_{charges~centre}}{Nombre~unité~oeuvre}$$
	 \subsubsection{L'attribution (l'imputation)}
	 \textit{Sachant que l'atelier a découpé 600m$^2$ de bois pour obtenir les chaises, que peut on en conclure ?}

	 $\Rightarrow 14\euro \times 600 = 8400\euro$ de charges indirectes sont attribués au coût des chaises en provenance de l'atelier découpe.
	 \subsection{Les taux de frais}
	 \subsubsection{Assiette de frais}
	 Il arrive pour certains centres de ne pas pouvoir mesurer leur activité par une unité physique. 
	 \paragraph{Exemple} Centre de distribution
	 On va prendre comme base de calcul, une valeur monétaire : ex : le chiffre d'affaires, le montant des achats, \ldots
	 toute valeur monétaire ayant une forte corrélation avec l'activité du centre. On ne parle plus dans ce cas d'unité d'\oe{}uvre mais d'assiette de frais.
	 \subsubsection{Le taux de frais}
	 On établi le même raisonnement qu'avec les unités d'\oe{}uvre. On a la formule:
	 $$Taux_{frais~centre~analyse} = \frac{total_{charges~du~centre}}{assiette~de~frais}$$
	 \section{La pris en compte des stocks}
	 Il faut distinguer: 
	 \begin{enumerate}
		 \item Coût de production de la période
		 \item Coût d'acquisition de la matière première utilisée
		 \item Coût de production des produits vendus 
		 \item Cout d'acquisition de la matière première achetée \\
	 \end{enumerate}
	 \begin{tikzpicture}
		\draw (0,0) rectangle (2.7,0.7) ;
		 \node at(1.3,0.4) {Fournisseurs} ;
		 \node at(3.1,0.7) {\textbf{(4)}} ;
		 \draw[>=latex, ->, thick] (2.7,0.4) -- (4.0,0.40);
		 \node at(5.0,0.4) {Stock MP} ;
		 \node at(6.4,0.7) {\textbf{(2)}} ;
		 \draw[>=latex, ->, thick] (6.0,0.4) -- (7.0,0.40);
		\draw (7,0) rectangle (9.3,0.7) ;
		\node at(9.7,0.7) {\textbf{(1)}} ;
		 \draw[>=latex, ->, thick] (9.3,0.4) -- (10.8,0.40);
		 \node at(8.0,0.4) {Atelier} ;
		 \node at(15.6,0.7) {\textbf{(3)}} ;
		 \draw[>=latex, ->, thick] (15.2,0.4) -- (17.0,0.40);
		\draw (17,0) rectangle (19,0.7) ;
		 \node at(13,0.4) {Stock de produits finis} ;
		 \node at(18.0,0.4) {Client} ;
	 \end{tikzpicture}
	 \subsection{Les différentes méthodes de valorisation des stocks}
	 Dans notre menuiserie nous utilisons 4 planches de chêne pour une série de chaises. 
	 On achète une planche tous les jours : 
	 \begin{itemize}
		 \item Le 1/12 : 100\euro~la planche
		 \item Le 2/12 : 101\euro~la planche
		 \item Le 3/12 : 102\euro~la planche
		 \item Le 4/12 : 103\euro~la planche
		 \item etc\ldots
	 \end{itemize}
	 Le 15 nous prenons 4 planches pour notre série.
	 \paragraph{Question} quelle valeur attribuer à ces 4 planches ?  3 méthodes principales:
	 \begin{itemize}
		 \item Méthode du premier entrée, premier sortie\footnote{PEPS}
		 \item Méthode du dernier entrée, premier sortie\footnote{LIFO}
		 \item Méthode du coût unitaire moyen pondéré\footnote{CUMP}
	 \end{itemize}
	 \paragraph{Exemple}
	 Stock initiali : 1$^{er}$ Juin 400 m$^2$ dont coût d'acquisition : 10400\euro
	 \subparagraph{Entrées}
	 \begin{itemize}
		 \item 5 Juin : 800 m$^2$ (CU: 29,02\euro)
		 \item 19 Juin : 1000m$^2$ (CU: 31\euro)
	 \end{itemize}
	 \subparagraph{Sorties}
	 \begin{itemize}
		 \item Le 10 Juin: 300m$^2$ destinés à la fabrication de table
		 \item Le 21 juin : 1200m$^2$ destinés à la fabrication d'armoire
	 \end{itemize}
	 \subparagraph{Question : Quelle est la valeur de la matière première utilisée ?}
	 \subsubsection{Méthode du premier entrée, premier sorti}
	 \begin{tabular}{|c|c|c|c| |c|c|c|c|}
		 \hline
		 \multicolumn{4}{|c||}{Entrées} & \multicolumn{4}{c|}{Sorties}\\
		 \hline
		 Date & Quantité & CU & Montant & Date & QUantité & CU & Montant\\
		 \hline
		 1/06 & 400 & 26 & 10400 & 10/06&300&26&7800\\
		 5/06&800&29.02&23216&21/06&100&26&\\
		 &&&&&800&29.02&35116\\
		 19/06&1000&31&31000&&300&31&\\
		 &&&&30/06(SF)&700&31&21700\\
		 \hline
		 &2200&&64616&&2200&&64616 \\
		 \hline
	 \end{tabular}
	 \paragraph{Conclusion} 
	 \begin{itemize}
		 \item Les 300m$^2$ destinés à la fabrication de tables sont évalués à 7800\euro
		 \item Les 1200m$^2$ destinés à la fabrication d'armoires sont évalués à 35116\euro
		 \item Le stock final est évalué à 21700\euro 
	 \end{itemize}
	 \subsubsection{Méthode du dernier entrée, premier sorti}
	 \begin{tabular}{|c|c|c|c| |c|c|c|c|}
		 \hline
		 \multicolumn{4}{|c||}{Entrées} & \multicolumn{4}{c|}{Sorties}\\
		 \hline
		 Date & Quantité & CU & Montant & Date & QUantité & CU & Montant\\
		 \hline
		 1/06 & 400 & 26 & 10400 & 10/06&300&29.02&8706\\
		 5/06&800&29.02&23216&21/06&1000&31&\\
		 &&&&&200&29.02&36804\\
		 19/06&1000&31&31000&&&&\\
		 &&&&30/06(SF)&300&29.02&19106\\
		 &&&&&400&26&\\
		 \hline
		 &2200&&64616&&2200&&64616 \\
		 \hline
	 \end{tabular}
	 \paragraph{Conclusion}
	 \begin{itemize}
		 \item Les 300 m$^2$ destinés à la fabrication de tables sont évalués à 8706\euro (au lieu de 7800\euro avec PEPS)
		 \item Les 1200 m$^2$ destinés à la fabrication d'armoires sont évalués à 36804\euro (au lieu de 35116\euro avec PEPS)
		 \item Le stock final est évalué à 19106\euro (au lieu de 21700\euro avec PEPS)
	 \end{itemize}
	 \subsubsection{Méthode du coût unitaire moyen pondéré}
	 Les sorties sont évalués au $CUMP = \frac{SI_{en~valeur} + entree~du~mois_{en~valeur}}{SI_{en~quantité} + entrees~du~mois_{en~quantité}}$\\
	 \begin{tabular}{|c|c|c|c| |c|c|c|c|}
		 \hline
		 \multicolumn{4}{|c||}{Entrées} & \multicolumn{4}{c|}{Sorties}\\
		 \hline
		 Date & Quantité & CU & Montant & Date & QUantité & CU & Montant\\
		 \hline
		 1/06(SI) & 400 & &10400&10/06&300&29.37&\textbf{8811}\\
		 5/06&800&&23216&21/06&1200&29.37&\textbf{35244}\\
		 19/06&1000&&31000&30/06(SF)&700&29.37&\textbf{20559}\\
		 \hline
		 &2200&29.37&64616&&2200&&\textbf{64616}\\
		 \hline
	 \end{tabular}
	 \paragraph{Conclusion}
	 Les planches utilisées ont un coût d'acquisition à nouveau différent. Il en est de même pour l'évaluation du stock final.
	 \paragraph{Remarque}
	 On ne peut évaluer ici la valeur des sorties que rétroactivement, c'est-à-dire une fois que l'on connaît toutes les entrées du mois.
	 \subsection{Conséquences du choix de la méthode de valorisation des sorties}
	 \begin{tabular}{|c|c|c|c|}
		 \hline
		 & \textbf{PEPS} & \textbf{CUMP}& \textbf{LIFO}\\
		 \hline
		 Sortie de stock & 42916 & 44056 & 45510\\
		 \hline
		 Stock final & 21700 & 20559 & 19106\\
		 \hline
	 \end{tabular}
	 \\
	 Dans notre exemple, nous avons eu une augmentation du prix et donc une hausse du coût d'acquisition (29.02\euro $\rightarrow$ 31\euro)

	 Dans cette situation inflationniste, la méthode PEPS minore les coûts de sortie et majore les valeurs du stock final (méthode donc à adopter si on vend notre
	 entreprise en période inflationniste)

	 Pour la méthode LIFO, c'est le contraire, elle minore le stock final. En période de baisse des prix, le choix des méthodes aurait
	 des conséquences opposées.


\end{document}
