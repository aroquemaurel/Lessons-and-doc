\chapter{Les emprunts et les prêts}
	\section{Emprunts et interêts}
	\subsection{Principes}
	Quand une personne (prêteur) prête un capital à une autre personne (emprunteur), il est habituellement convenu que l'emprunteur rembourse à l'écheance, non seulement le montant du prêt, mais un 
	supplément, appelé interêt du prêt.

	L'interêt est fonction
	\begin{itemize}
		\item Du capital prêté (C)
		\item Du taux d'intérêt qui a été convenu entre le prêteur et l'emprunteur (t)
		\item De la durée du prêt (n)
	\end{itemize}
	\subsection{Les systèmes de calcul des interêts}
	\subsubsection{Le système des interêts simples}
	Dans ce système l'interêt est égal a : $ I = C \times t \times n$

	Les intérêts sont payés à la fin du prêt, ils ne s'ajoutent pas au capital pour produire eux-mêmes des intérêts.\\
	Ce système est en général utilisé pour les prêts de courte durée.

	\subsubsection{Le sysème des intérêts composés}
	Dans ce système les intérêts \textbf{s'ajoutent périodiquement au capital pour porter eux-mêmes intérêts}.
	\exemple{
		Supposons que l'on effectue un emprunt de 500 000\euro ~au taux de 9\% sur 10 ans.\\
		Intérêts de la 1ère année: $500000 \times 0.09 = 45000$\\
		Intérêts de la 2nd année: $545000 \times 0.09 = 49050$\\
	}
	\section{Valeur acquise et valeur actuelle d'un capital}
	\subsection{Valeur acquise}
	\subsubsection{La notion de valeur acquise}
	La valeur acquise à une date donnée par un capital est égale au montant du capital augmenté du montant des intérêts à cette date.
	$$ValeurAcquise = C + I$$
	\subsubsection{Valeur acquise d'un capital place à intérêts composés}
	Les périodes sont des intervalles de temps égaux. À la fin de chaque période les intérêts s'ajoutente au capitale : ces intérêts produisent à leur tour des intérêts lors de la période suivant et ainsi de suite.
	\exemple{Supposons que l'on effectue un emprunt de 500 000\euro ~au taux de 9\% sur 10 ans le 01/01/N\\
	\begin{itemize}
		\item La valeur acquise après 1 an, soit le 1/1/N+1 sera de : 
			\begin{eqnarray*}
			C_1 &=& 500 000+(500000 \times 0.09)\\
			 &=& 500 000+(1 \times 0.009)\\
			 &=& 500 000 \times 1.09\\
			\end{eqnarray*}

		\item La valeur acquise après deux ans, soit le 1/1/N+2 sera de:
			\begin{eqnarray*}
				C_2 &=& 500000 \times (1.09) + (500000 \times (1.09)) \times 0.09 \\
				&=& 500000 \times 1.09^2\\ 
			\end{eqnarray*}
		\item La valeur acquise après trois ans, soit le 1/1/N+3 sera de:
			\begin{eqnarray*}
				C_3 &=& 500000 \times (1.09^2 + 500000 \times 1.09^2 \times 0.09^2\\
				&=& 500000 \times 1.09^3
			\end{eqnarray*}
		\item À l'échance du prêt soit 10 ans plus tard, le 1er Janvier N+10, la valeur acquise sera de 
			\begin{eqnarray*}
				C_{10} &=& 500000 \times 1.09^{10}\\
				&=& 1183681 
			\end{eqnarray*}
	\end{itemize}
	}
	\paragraph{Généralisation}
	Désignons par $C_n = C_0 \times (1 \times i)^n$
	\begin{itemize}
		\item $C_0$ le capital emprunté à l'époque 0, soit 500 000\euro ~(dans l'exemple)
		\item $C_n$ la valeur acquise ç l'époque n soit 10 ans après (dans l'exemple)
		\item $i$ le taux d'intérêt relatif à une période soit 0.09 (dans l'exemple)
	\end{itemize}
	\subsection{Valeur actuelle d'un capitale à intérêts composés}
	Si on a la possibilité de placer ses capitaux au taux d'intérêts composés i, il est équivalent :
	\begin{itemize}
		\item de recevoir de 10 ans ces capitaux
		\item ou de recevoir aujourd'hui l'argent que l'on aurait placé
	\end{itemize}
	Le capital $C_0$ est apelé \textbf{valeur actuelle} à l'époque 0 du capital $C_n$ échéant à l'époque n.

	Quand on calcule la valeur actuelle d'un capital, en tenant compte des intérêts composés, on dit qu'on actualise ce capital. Le taux d'intérêts composés prend le nom de taux d'actualisation.
	\subsubsection{Calcul de la valeur actuelle}
	La relation $C_n=C_0 \times (1+i)^n$ permet d'écrire, $C_0=\frac{C_n}{(1+i)^n}$
	\exemple{Supposons qu'au 1er janvier N+10 nous ayons besoin d'un capital de 100 000\euro ~pour financer l'acquisition d'une nouvelle machine.\\
	Combient faut-il que nous placions au 1er janvier N, à intérêt composés au taux annuel de 8\% afin d'obtenir ces 100 000 au 1er Janvier N+10.\\ 
	La valeur acquise est donc 100 000, on cherche la valeur actualisée.\\
	$\Rightarrow C_0 = 100000 \times (1.08)^{-10} = 46319$
	}
	\section{Les cas des emprunts remboursables au moyen de versements périodiques}
	Jusqu'à présent on a fait l'hypothèse implicite que l'emprunt était remboursable en totalité à la find ela période.\\
	Mais d'autre modalités de remboursement sont possibles, en particulier celles d'un remboursement périodique (chaque fin de mois, d'années, etc\dots)\\
	À chaque échéance il faut alors:
	\begin{itemize}
		\item Rembourser une fraction de l'emprunt (le montant ainsi remboursé est apelé amortissement de l'emprunt\footnote{A ne pas confondre avec l'amortissement des immobilisations}.
		\item Verser des intérêts stipulés au contrat sur le montant du capital restant dû.
	\end{itemize}
	\begin{eqnarray*}
		\Rightarrow &=& Interêts+ammortissement_{emprunt}
	\end{eqnarray*}
	\paragraph{}Deux modalités de calcul des annuités sont possibles:

	\subsection{1ère possibilité : Amortissement constants}
	Le remboursement de l'emprunt est régulièrement étalé sur la durée de l'emprunt. Les intérêts sont alors calculés à chaque échéance sur le capital restant dû, ainsi leur montant diminie à chaque échéance.
	\exemple{
	Déterminer les remboursements annuels de l'emprunt contracté le 1er janvier N pour un montant initial de 500 000\euro ~sur cinq ans au taux annuel de 9\%
	\\
	\begin{tabular}{|p{0.5cm}|p{2.5cm}|p{1.6cm}|p{3cm}|p{1.7cm}|p{2.5cm}|}
	\hline
	\textbf{An} &\textbf{Capital restant dû en début de periode}& \textbf{Intérêts annuels} & \textbf{Amortissement annuel} & \textbf{Annuité}&\textbf{Capital restant dû en fin de période}\\
	& $a$ & $b=a \times 0.09 $ & $c$ & $d = b+c$ & $e=a-c$ \\
	\hline
		1 & 500000 & 45000 & 100000 & 145000 & 400000\\
		\hline
		2 & 400000 & 36000 & 100000 & 136000& 300000\\
		\hline
		3 & 300000 & 27000 & 100000 & 127000& 200000\\
		\hline
		4 & 200000 & 18000 & 100000 & 118000& 100000\\
		\hline
		5 & 100000 & 9000 & 100000 & 109000& 0\\
		\hline
	\end{tabular}
	\\
	Ce tableau est appelé \textit{tableau d'amortissement}
	}
	\subsection{2nd possibilité : Tableau d'amortissement}
	La suite dest annuités s'échelonnant sur plusieurs années, les opérations concernant ces annuités sont donc effectuées avec des intérêts composés.

	Considérons la suite des annuités constantes : $a_1 = a_2 = a_3 = a_4 = \dots = a_n$ 

	On peut écrire aussi: $V_0 = actualisation~a_1 + actualisation~a_2 + \dots + actualisation~a_n =$ somme d'agent qu'il fallait en 0 pour obtenir ces annuités. \\
	$$\Leftrightarrow V_0 = a_1(1+i)^{-1} + a_2(1+i)^{-2} + a_3(1+i)^{-3} + \dots + a_n(1+i)^{-n}$$
	\paragraph{}
	On remarque que le deuxième membre constitue une suite géométrique dont la raison est égale à $(1+i)^{-1}$.\\
	Or la somme des termes d'une suite géométrique est donnée par l'expression $S=p \times \frac{q^n -1}{q - 1}$\\
	Avec p = premier terme ; q = raison ; n = nombre de termes. On obtient donc:
	$$a(1+i)^{-1} \times \frac{1+i^{-n}-1}{(1+i)^{-1} - 1}$$
	On peut simplifier l'expression en divisant le premier terme et le dénominateur par $(1+i)$. D'où
	$$a = V_0 \times \frac{i}{1-(i+i)^n}$$
	\exemple{Déterminer les remboursements annuels de l'emprunt contracté le 1er Janvier N pour un montant initial de 500 000\euro ~sur cinq ans au taux annuel de 9\%.\\
	Première étape : déterminer le montant de l'annuité constante.
	$$a = 128 546 $$
	\begin{tabular}{|p{0.5cm}|p{2.5cm}|p{1.6cm}|p{1.7cm}|p{3cm}|p{2.5cm}|}
	\hline
	\textbf{An} &\textbf{Capital restant dû en début de periode}& \textbf{Annuité constant} & \textbf{Intérêts annuels} & \textbf{Amortissement annuel}&\textbf{Capital restant dû en fin de période}\\
	& $a$ & $b$ & $c = a \times 0.09$ & $d = b-c$ & $e=a-d$ \\
	\hline
		1 & 500000 & 128546 & 45000 & 83546& 416454\\
		\hline
		2 & 416454& 128456 & 37481& 91065& 325389 \\
		\hline
		3 & 325389 & 128546 & 29285 & 99261 & 226128 \\
		\hline
		4 & 226128 & 128546 & 20352 & 108194 & 117934 \\
		\hline
		5 & 117934 & 128548 & 10614 & \textbf{117934} & 0\\
		\hline
	\end{tabular}
	}
	\remarque{Le dernier amortissement doit être obligatoirement égal au capital restant dû au début de la période (117934) $\Rightarrow$ la dernière annuité peut être légèrement différnetes des autres.
	Ceci dû à l'arrondissement des calculs.
	}


