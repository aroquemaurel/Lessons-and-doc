\chapter{La politique d'investissement}
	\section{La rentabilité des investissements}
		\subsection{Les investissements}
		Les investissements sont des dépenses destinés à: 
		\begin{itemize}
			\item Conserver dans l'état la capacité de production --- \textbf{ investissements de remplacement} 
			\item Diminuer les coûts de production --- \textbf{investissements de modernisation}
			\item Accroitre la capacité de production --- \textbf{investissements de croissance}
		\end{itemize}
		\subsection{Les dépenses et les recettes liés à l'investissement}
		Pour juger si un investissement est rentable, on compare les dépenses et les recettes entrainés par le projet.\\
		Apriori, un projet est rentable si les $Recette > Dépenses$, mais de quel recette et de quelle dépense doit on tenir compte ?
		\subsubsection{La dépense d'investissement ($C$)}
		La dépense initiale est désigné par $C$, elle correspond au prix d'achat des immobilisations.
		\remarque{
			\begin{enumerate}
				\item Le prix est Hors Taxe
				\item Dans le cas des investissements de croissance, il faut tenir compte aussi de l'augmentation du besoin en fond de roulement\footnote{Appelle BFR}.
					$$BFR = Stocks + Creances - dettes$$
			\end{enumerate}
		}
		\exemple{
			On achète une une machine $90~000$\euro~payé le 31/12/N. Cet investissement permettrai une croissance de vente,  ceci engendrerai une augmentation du BFR de $30~000$\euro , on considère donc que le montant
			de l'investissement est de $90~000 + 30~000 = 120~000$\euro 
		}
		\paragraph{Les recettes nettes d'exploitations après impôts\newline}
		Elle correspondent à la différence entre le chiffre d'affaires\footnote{Abréviation CA} et les charges d'exploitation dégressives.
		$$ CA = Px \times Quantité$$
		Deux termes à préciser: 
		\begin{itemize}
			\item \textbf{Les charges d'exploitation} correspondent à toutes les charges sauf les charges financières et les charges exceptionnel.
			\item \textbf{Les charges décaissés} correspondent aux charges qui ont provoqués une sortie d'argent\footnote{Ceux sont en générale toutes les charges, les amortissements n'en font pas partis}
		\end{itemize}
		\exemple{Le chiffre d'affaire, résultat de l'achat de la nouvelle machine s'élèverait de $171~000$\euro~par an pendant les cinq ans à venir.
			Les charges d'exploitations prévus suites à l'achat de cette nouvelle machine pour chacune des cinq années sont les suivantes: 
			\begin{itemize}
				\item \textbf{Charges décaissés}
					\begin{itemize}
						\item Achat de matière première: $40~000$\euro
						\item Service extérieur: $28~000$\euro
						\item Charge de personnel: $50~000$\euro
					\end{itemize}
				\item \textbf{Charges d'exploitations non décaissés} 
					\begin{itemize}
						\item Amortissement linéaire: durée 8 ans. $90~000 \div 8 = 11~250$\euro.\newline 
					\end{itemize}
			\end{itemize}
			
			\textit{Déterminer le montant des recettes nettes annuels d'exploitation après impôt.}
			\begin{eqnarray*}
				IS &=& \frac{171~000 - (60~000 + 28~000 + 50~000 + 11~250)}{3}\\
				&=& 7250\\ \\
				RNE &=& 171~000 - 60~000 - 28~000 - 50~000 - 7250\\
				&=&25~750
			\end{eqnarray*}
		}
		\paragraph{La valeur résiduelle de l'investissement ($V$)\newline}
		À la fin du projet, certaines immobilisations ont encore de la valeur, c'est le cas des immobilisations dont l'amortissement n'est pas achevé. De même, les sommes d'argents qui avait été
		employés dans l'augmentation du Besoin en Fond de Roulement redeviennent disponible, c'est deux points constituent donc une recette appelée valeur résiduelle à la fin. On la désigne par $V$.
		\exemple{
		Dans notre exemple on considère qu'a la fin de la période des cinq ans la valeur nette véritable de l'immobilisation est de $33~750$\euro .
		Ainsi la valeur résiduelle de l'investissement est de $33~750 + 30~000 = 63~750$\euro.
		}
		\subsection{Actualisation des recettes prévisionnelles}
		Il ne suffit pas que le montant des recettes futures soit supérieur au montant de l'investissement pour que ce dernier soit déclaré rentable. En effet, il faut considérer que les sommes dépensés pour réaliser
		l'investissement auraient pus être utilisé à un placement et donc elles auraient pus elle aussi rapporter de l'argent. On va donc actualiser chaque recette afin de déterminer le montant initial
		qu'il aurait fallut placer pour obtenir ces recettes.

		On actualise les recettes c'est-à-dire on calcule la somme d'argent qu'il fallait placer en 0 pour obtenir ces recettes.		
		\begin{eqnarray*}
			\sum argent &=& R_1 \times (1+i)^{-1} + R_2 \times (1+i)^{-2} + \cdots + (R_n + V) \times (1+i)^{-n}\\
		Si\ C > \sum argent &\Longrightarrow& Ce\ n'est\ pas\ rentable \\
		Si\ -C + \sum argent > 0 &\Longrightarrow& Rentable. 
		\end{eqnarray*}
		

		La différence entre la valeur actuelle des recettes et la valeur actuelle des dépenses est appelée \textit{valeur actuelle nette} du projet. Ainsi
		un projet est rentable si sa $V_n$ est positive.
		\begin{eqnarray*}
			VAN = -C + (R_1 \times (1 + i)^{-1} + \cdots + R_n \times (1+i)^{-n})
		\end{eqnarray*}
		\exemple{\textbf{Calcul de la VAN}

		\textit{Calculons la VAN sachant qu'en moyenne le taux du marché est de 12\%.}
		\begin{eqnarray*}
			VAN &=& -C + (R_1 \times (1 + i)^{-1} + \cdots + R_n \times (1+i)^{-n})\\
			&=& -120\; 000 + 25\; 750 \times (1.12)^{-1} + \cdots + 89\; 500 \times (1.12)^{-5}\\
			VAN &=& \textbf{8996\euro{}}
		\end{eqnarray*}
		}
		\remarque{
		Une entreprise peut avoir à choisir entre plusieurs projets rentables. On suppose qu'elle choisira le plus rentable, ce sera donc celui qui à la plus grande VAN
		}
		\newpage
	\section{Le plan de financement}
		Le plan de financement est un document financier prévisionnel qui permet d'étudier les effets des projets de l'entreprise sur la trésorerie des années à venir. 
		
		On peut avoir un projet définit comme rentable\footnote{$VAN > 0$} tout en ayant un problème de trésorerie. 
		Le plan de financement comporte pour chaque année : 
		\begin{itemize}
			\item Une prévision des ressources\footnote{La ressource c'est ce qui permet d'obtenir l'emploi}
				\begin{itemize}
					\item Apport des associés (Capital)
					\item Les emprunts
					\item Trésorerie initiale
					\item Capacité d'auto financement\footnote{Appelé CAF -- Classe 7 encaissable et Classe 6 décaissable}
					\item Vente d'immobilisation 
		\remarque{La restitution du BFR\footnote{Besoin en fond de Roulement} à la fin du projet n'est pas prise en compte dans le plan de financement car elle ne représente pas un apport d'argent liquide\\
		alors que cette restitution est prise en compte dans le calcul de la vente.}
				\end{itemize}
			\item Une prévision des emplois
				\begin{itemize}
					\item Achat d'immobilisation
					\item Hausse du BFR
					\item Remboursement d'emprunts
					\item Distribution de dividendes
				\end{itemize}
		\end{itemize}
		\exemple{
		On décide de procéder à l'investissement de $90\;000$\euro{} le 31/12/N. On dispose d'une trésorerie initiale de $10\;000$\euro{}. On emprunte $90\;000$\euro{} aux taux de 15\% remboursable en 
		5 amortissements annuel constant. Ce financement sera complété par un apport de $20\; 000$\euro{} des associés.
			\paragraph{Amortissements}
			\paragraph{}
			\begin{table}[H]
				\centering
				\begin{tabular}{|c|c|c|p{3.5cm}|p{1.5cm}|p{1.5cm}|}
					\hline
					\textbf{Année} & \textbf{Capital} & \textbf{intérêt} & \textbf{Amortissement} & \textbf{Annuité} & \textbf{Capital}\\
					\hline
					N & $90\;000$&$13\;500$&$18\;000$&$31\;500$&$72\;000$\\
					\hline
					N+1&$72\;000$& $10\;800$&$180\;000$ &$28\;800$ & $54\;000$\\
					\hline
					N+2 & $54\;000$&$8\;100$&$18\;000$&$26\;100$&$36\;000$\\
					\hline
					N+3&$36\;000$&$5\;400$&$18\;000$&$23\;400$ & $18\;000$\\
					\hline
					N+4 & $18\;000$&$2\;700$&$18\;000$&$20\;700$&0\\
					\hline
				\end{tabular}
				\caption{Amortissements}
			\end{table}
			\subparagraph{Capacité d'autofinancement}
				\begin{eqnarray*}
					CAF &=& produit_{encaissable} - charges_{decaissables}\\
					CAF_{projet} &=& EBE - charges_{financieres} - IS \\
					&=& EBE - charges_{financieres} - \frac{1}{3} \times \\
					&&~ ~ ~(Resultat_{exploitation} - charges_{financieres}) \times N+4\\
					&=& EBE - charges_{financières} - \frac{1}{3} \times Resultat_{exploitation} + \\
					&&~ ~ ~\frac{1}{3} \times charges_{financières}\\
					&=& Resultat~nettes~exploitation - \frac{2}{3} \times charges_{financieres}
				\end{eqnarray*}
			}





