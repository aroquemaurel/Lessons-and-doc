	  \chapter{Hypoth\`eses financi\`eres et pr\'evisions}
		Chaque exercice d\'ebutera le $1^{er}$ janvier de chaque ann\'ee et se terminera le 31 d\'ecembre de cette même ann\'ee.
		Par exception, le premer exercice d\'ebutera le $1^{er}$ octobre et finira le 31 d\'ecembre.\\

		Nous avons fait le choix de suivre tantôt des hypoth\`eses hautes et tantôt des
		hypoth\`eses basses afin de se pr\'eparer à investir ou au contraire à trouver
		les points faibles du budget.

		L'ensmble des tableaux ont été ajoutés en annexe (Annexe \ref{tableaux} p.\pageref{tableaux}) pour une meilleur lisibilité.

		Le plan de tr\'esorerie utilisant les hypoth\`eses hautes est repr\'esent\'e par le tableau
		\ref{tab:planTresoH} et celui utilisant les hypoth\`eses basses est repr\'esent\'e par le tableau
		\ref{tab:planTresoB}.
		
		Les comptes de r\'esultat \ref{tab:hypHN},
		\ref{tab:hypHN1} et \ref{tab:hypHN2} repr\'esentent les hypoth\`eses hautes. Les
		comptes de r\'esultat \ref{tab:hypBN}, \ref{tab:hypBN1} et \ref{tab:hypBN2} repr\'esentent
		les hypoth\`eses basses.
