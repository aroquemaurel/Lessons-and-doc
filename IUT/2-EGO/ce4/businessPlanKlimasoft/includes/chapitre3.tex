\chapter{Marché et concurence}
	\section{Présentation du marché}\label{marché}
		% présenter le marché
		\subsection{Prospects}
			% dimension
			\chiffresToulouse{}
			% volume du marché
			Le marché des bilans thermiques est évalué à environ $360\;000$ études par an,
			soit 3 bilans thermiques par mois pour les $10\;000$ TPE et PME du \gHabitat{}.
			Sur ces $10\;000$ entreprises, seule environ $1\;000$ d'entre elles sont équipées 
			d'un logiciel de calcul payant\footnote{Voir \ref{concurence}. Concurence}.
			Les autres structures ont recours à des méthodes simplificatrices
			(tableurs excel, approximations sur expérience, etc.).
			% evolution du marché
	
		\subsection{Annonceurs}
		Les fabricants qui représentent ici les clients de la société \K{} ont une part de leurs budget allouée à la publicité à l'heure actuelle seul les magasines spécialisés leurs
		offre un support de publicité spécifiques et attrayant. Des magasines tel le ``JDC''\footnote{\textbf{J}ournal \textbf{d}u \textbf{C}hauffage} offrent 25 pages d'emplacements publicitaires
		à $10\;000$\euro{}{} par mois sur un nombre de page totales du magasine de 75.
		% dimension
		% volume du marché
		% evolution du marché (qualité - quantité)
		% reglementé en france ? à l'étrangé ?
		% pourcentage visé par l'entreprise
		% taux de croissance à 5 ans ? JUSTIFIER !!
		
	\section{Segmentation du marché}
		\subsection{Prospects}
		Les logiciels proposés par \K{} sont principalement à destination des PME/TPE dans le secteur du génie climatique.
		Sont visés les bureaux d'étude et les prestataires de services réalisant des bilans thermiques.
		Les attentes de ce segment du marché ne sont pas dans la précision et la performance des calculs mais dans
		rapidité de saisie d'un grand nombre de données et dans l'acquisition d'un outils précis et performant
		à moidre coût.\\

		De plus, la gratuité de notre produit nous aidera à toucher des particuliers souhaitant
		dimenionner de petites structures ou l'aider à mieux dimensionner une installation existante
		afin de lui faire réaliser des économies d'énergie. Ce segment de marché n'a pas d'attentes particulières
		car jusque là, ces partculiers faisaient appel à des professionnels et leur faisaient confiance. Grâce à notre
		logiciel, les particuliers pourront réaliser de petites études chez eux, sans avoir de grosses dépenses
		à réaliser. Ces prospects seront les plus intéressés par les publicités proposées qui les aiderons
		à faire leur choix non pas en fonction de références constructeurs mais en fonction des résultats que calculera le
		logiciel et des produits partenaires qui lui seront proposés.

		\subsection{Annonceurs}
			Une liste de 47 annonceurs\footnote{La liste de tous les annonceurs potentiels est disponible en Annexe \ref{listeAnnonceurs} page \pageref{listeAnnonceurs}}  
			potentiels à été déterminée selon les différents types de matériel proposés 
			pour chauffer ou climatiser les bâtiments.

			La réussite de ce projet est totalement dépendante de l'intérêt des fabricants à communiquer à travers les solutions logicielles proposés par \K{}.
			% segmentation possible ? comment ?
			% attentes différentes selon les segments ?
			% concurence identique selon les segments ?
			% nouvelles technologies qui déterminent l'évolution du marché ?
		
	
	\section{Concurence}\label{concurence}
	% étudier la concurence
	L'offre publicitaire proposée par \K{} n'a pas de concurrence directe
	car c'est une solution marketing innovante dans le secteur de génie
	climatique. En effet, les logiciels actuellement
	présent sur le marché sont payants et très coûteux et n'intègre pas la
	publicité dans leur interface.
	En revanche, il existe une concurrence indirecte incarnée par les
	magazines spécialisés (``Le moniteur'', ``JDC'', ``CVC''\ldots).

	L'avantage du support publicitaire dans les magazines spécialisés
	est qu'il offre aux fabricants une visibilité auprès des professionnels
	du génie climatique.

	La visibilité qu'offre les support de la société \K{} sera la même.

	La différence se fait au niveau des supports publicitaires : les notres
	seront intéractifs. Ainsi l'utilisateur pourra cliquer sur l'encart publicitaire
	et être redirigé vers le site du fabricant, voir même vers un page précise
	adaptée au dimensionnement qu'il vient de réaliser.

	De plus, les annonceurs pourront cibler les utilisateurs qu'ils souhaiteront toucher
	grâce aux renseignements saisies lors du téléchargement (localisation géographique,
	structure et taille de l'entreprise,\ldots). Ce système fait contraste avec la
	publicité des magasines dite ``statique'', qui joue s'implement sur l'attraction
	visuelle et qui nécessite plus de volonté côté lecteur pour aller chercher des
	informations complémentaires.
	
