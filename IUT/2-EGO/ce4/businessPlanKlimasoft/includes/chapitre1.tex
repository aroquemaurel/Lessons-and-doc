\chapter{Projet et Équipe de management}
	\section{Présentation des créateurs}
		\begin{itemize}
			\item \Bonte{}, Ingénieur en \gHabitat{}
			\item \Ben{}, Ingénieur en \gHabitat{}
			\item \Drm{}, Développeur Web
			\item \Soum{}, Développeur \texttt{\glo{C++}{C++}{4e langage de programmation
				le plus utilisé au monde. Il est compilé, permettant de produire un programme
				s'éxecutant le plus rapidement possible.}/\glo{Qt}{Qt}{Bibliothèque programmée
				en C++ permettant de créer des interfaces graphiques.}}
			\item \Clem{}, Développeur \texttt{C++/Qt}, Administrateur Système et Réseau
		\end{itemize}
		
		\subsection{Formations}
			\Bonte{} et \Ben{} ont obtenu un Master Pro \gHabitat{} à l'INSA de Toulouse et
			font actuellement une thèse.

			\Drm{}, \Soum{} et \Clem{} sont en deuxième année de DUT Informatique 
			à l'IUT Paul Sabatier de Toulouse. Ces derniers sont
			également autodidactes et ont pu acquérir de nombreuses conpétences lors de projets personnels.
	
		\subsection{Expériences professionnelles}
			\Bonte{} a passé un an dans un bureau d'étude à réaliser des bilans thermique. C'est pendant cette periode 
			que lui est venue l'idée de notre projet après avoir constaté le manque dramatique d'affordance des solutions disponibles.
			Le reste de l'équipe s'est constitué autour de ce constat global.
	
	\section{Atouts}
		%atouts qui font qu'on a des facilités a créer l'entreprise

		%facultés particulières
		Ayant déja eu une expérience professionnelle, \bonte{} et \ben{} connaissent les difficultés et les besoins des TPE et PME du \gHabitat{}.
		%contacts
		Ainsi, nous disposons déja de contacts dans ce secteur,
		nottamment au sein d'établissements universitaires et de bureaux d'étude.
		Ce premier carnet d'adresse peut être facilement étoffé car notre clientelle est,
		par sa taille humaine, facilement abordable et particulièrement à l'écoute pour trouver des solutions gratuites et efficaces.
		%connaissances pratiques théo
		Grâce aux cours généraux enseignés à l'université et en DUT, 
		tout les associés ont des connaissances en Comptabilité, Gestion et Droit des Entreprises,
		ce qui permet de faciliter les échanges avec les professionnels (comptables, avocats...) que nous ne manquerons pas de contacter.
		%part à des orga assoc
		\bonte, actuellement thésard, interviens dans des promotions de \gHabitat{} et a, 
		de ce fait, la possibilité de présenter des produits de \K{} aux étudiants. \\
		\clem{} est quant à lui impliqué dans diveres associations et a de ce fait rencontré plusieurs personnes ayant fondé ou travaillant dans une \glo{SCOP}{SCOP}{Société soumise à l’impératif de rentabilité comme toute entreprise.
Ses salariés-coopérateurs y sont en effet associés (ou « co-entrepreneurs ») majoritaires et détiennent au moins 51\% du capital et 65\% des droits de vote. Par ailleurs, quelle que soit la quantité du capital détenu, chaque coopérateur ne dispose que d'une seule voix lors de l'assemblée générale de l'entreprise.
}\footnotesouvenir{scop}{\textbf{S}ociété \textbf{CO}opérative et \textbf{P}articipative}. Cela nous permet d'avoir des réponses rapides et un premier contact avec le réseau des SCOPs\footnoterappel{scop}, qui permet aux jeunes entreprises de bénéficier d'avantages divers afin de se développer.
		%aide famille...


	\section{L'idée}
		% societe de dev de logiciel et de prestation de services informatiques dans le génie de l'habitat
		\K{} est une Société de Développement de Logiciels et de Prestation de Services Informatiques dans le secteur du \gHabitat{}.

		% Comment est venue l'idée
		% Secteur du génie de l'habitat
		L'idée de ce projet est née à travers diverses expériences dans le domaine du génie climatique.
		À l'heure actuelle, les professionnels n'ont à leurs disposition que peu d'outils : 
		\begin{itemize}
			\item Tableurs Excel réalisés en interne, aux résultats approximatifs dans un contexte de maîtrise de l'énergie
				et dans lesquels la saisie des données est peu aisée.
			\item Logiciels réglementaires coûtant plusieurs milliers d'euros, à l'ergonomie souvent douteuse et peu adaptés a de petites structures telles que les PME\footnote{\textbf{P}etites et \textbf{M}oyennes \textbf{E}ntreprises} et les TPE\footnote{\textbf{T}rès \textbf{P}etite \textbf{E}ntreprise}
		\end{itemize}
		
		% C'est une création
		Nous souhaitons donc créer une société a l'écoute des besoins de ces petites structures, afin de leurs permettre d'économiser leurs ressources lors de leurs projets grâce à des outils adaptés à leurs échelle.
		% Sur Toulouse
		Notre équipe s'étant formée à Toulouse, et le secteur du \gHabitat{} y étant largement développé\footnote{Voir Chapitre \ref{marché}. Marché}, c'est donc dans cette ville que nous implanterons notre société.

	\section{Objectifs du projet}
		%quel objectif ? expansion, retabilité ? autonomie ?
		%prépondérent
		%d'autres ?
		La société \K{} repose sur des valeurs et des principes communautaires
		où le seul objectif est de rendre accessible au plus grand nombre
		l'accès a des outils ergonomiques, intuitifs et performants
		afin de fournir les résultats les plus précis possible 
		dans une optique de maîtrise de l'énergie et de développement durable.

	\section{Taille de l'entreprise}
	%dimension (effectif, CA, capitaux 10890, parts de marché
	%taille max ? min ?
	L'entreprise \K{} souhaite rester une entreprise à taille Humaine, ainsi elle sera composé d'un maximum de 20 personnes afin que tout le monde soit impliqué dans l'entreprise.

	Le capital de départ sera de 10090\euro{} et pourra évoluer selon les activités de l'entreprise. 

	Notre but serai d'avoir un chiffre d'affaire de $104\;400$\euro{} au bout de deux ans d'activités et atteindre les 2~220~000\euro{} dans les 4 ans après la création de notre entreprise. 

	Dans un premier temps, \K{} favorisera l'évolution au sein de Midi-Pyrénnés, si celle-ci fonctionne
	convenablement, elle s'étendra au reste de la France dans un second temps.
