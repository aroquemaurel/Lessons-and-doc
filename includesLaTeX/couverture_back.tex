\newsavebox{\fmbox}
\newenvironment{fmpage}[1]
     {\begin{lrbox}{\fmbox}\begin{minipage}{#1}}
	  {\end{minipage}\end{lrbox}\fbox{\usebox{\fmbox}}}


\makeatletter

\title{\titreDocument}

\def\top#1{\def\@top{#1}}

\def\sousTitre#1{\def\@sousTitre{#1}}
\sousTitre{Bibliothèque d'objets graphiques UML}

\def\location#1{\def\@location{#1}}
\location{Toulouse}

\date{\today}

%% Entete (Université..)  
\def\clap#1{
	\hbox to 0pt{\hss #1\hss}
}%
\def\ligne#1{%
	\hbox to \hsize{%
		\vbox{\centering #1}
	}
}%

% définition du haut de la couverture %
\def\haut#1#2#3{%
	\hbox to \hsize{%
		\rlap{
			\vtop{\raggedright #1}
		}%
		\hss
		\clap{	
			\vtop{\centering #2}
		}%
		\hss
		\llap{
			\vtop{\raggedleft #3}
		}
	}
}%

% Définition du bas de la couverture %
\def\bas#1#2#3{%
	\hbox to \hsize{%
		\hss \clap{\vbox{\centering #2}}%
		\hss
	}
}%

% Zou, on peut construire la page de garde 
\def\maketitle{%
	\thispagestyle{empty}\vbox to \vsize{%
		\vspace{-25px}
		\vspace{5px}
		\haut{}{\@top}{}
		\begin{flushleft}
			\Drm{}\\
\Soum{}\\
\Clem{}\\

		\end{flushleft}

		\begin{flushright}
			\vspace{-3cm}
			\begin{tabular}{r@{~}l}
				\newcommand{\footCentre}{}
\newcommand{\premierDestinataire}{Monsieur Thierry Millan}
\newcommand{\rolePremierDestinataire}{Client}

\newcommand{\secondDestinaire}{Madame Caroline Kross}
\newcommand{\roleSecondDestinaire}{Tutrice}

\newcommand{\troisiemeDestinaire}{}
\newcommand{\roleTroisiemeDestinaire}{}

\newcommand{\quatriemeDestinaire}{}
\newcommand{\roleQuatriemeDestinaire}{}

\newcommand{\cinquiemeDestinaire}{}
\newcommand{\roleCinquiereDestinaire}{}

\newcommand{\titreDocument}{Installation et exploitation}

\newcommand{\nouveauChapitre}{ \thispagestyle{fancy} }
\def\sectionautorefname{Section}


%\ifthenelse{\equal{\premierDestinataire}{}}{
%}
%{
%	Pour \premierDestinataire & (\rolePremierDestinataire) \\
%}
%\ifthenelse{\equal{\secondDestinaire}{}}{
%}
%{
%\secondDestinaire & (\roleSecondDestinaire) \\
%}
%\ifthenelse{\equal{\troisiemeDestinaire}{}}{
%}
%{
%	\troisiemeDestinaire & (\roleTroisiemeDestinaire) \\
%}
%\ifthenelse{\equal{\quatriemeDestinaire}{}}{
%}
%{
%	\quatriemeDestinaire & (\roleQuatriemeDestinaire) \\
%}
%\ifthenelse{\equal{\cinquiemeDestinaire}{}}{
%}
%{
%	\cinquiemeDestinaire & (\roleCinquiereDestinaire) \\
%}

			\end{tabular}
		\end{flushright}
		\vfill
		\vspace{1cm}
		\begin{flushleft}
			\policeTitre{\huge \@title}
		\end{flushleft}

		\par
		\hrule height 4pt
		\par

		\begin{flushright}
			\policeTitre{\Large \@sousTitre}
			\par
		\end{flushright}

		\vspace{1cm}
		\vfill
		\vfill
		\bas{}{\@location, le \@date}{}
		\vspace{5px}
		\vspace{-15px}
	}%
	\cleardoublepage
}

\makeatother

\top{%
	Université Paul Sabatier -- Toulouse III\\
	IUT A - Toulouse Rangueil\\
	\textbf{Projet tuteuré \#20}\\[1em]
}%

\makeatletter
