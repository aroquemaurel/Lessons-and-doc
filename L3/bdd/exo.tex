\chapter{Exercices}
\section{Algèbres relationnel}
\subsection{Opérateurs algébriques}
\texttt{ vins(numv, cru, mill, region, degré) } \\
\texttt{ buveurs(nums, nom, prenom, ville) } \\	
\texttt{ abus(nomv, nomb, date, quantite, place) } \\

	
\begin{attention}
Avec les opérateurs algébriques et uniquement les opérateurs de base. 
\end{attention}

\subsubsection{ Donner le degré des vins de cru Morgon et Millésime 2001 }
\begin{eqnarray*}
	T_1 &=& \sigma_{cru='morgon'~and~mill=2001}(vins)\\
	T_2 &=& \pi_{degré}(T_2)
\end{eqnarray*}
\subsubsection{Numéro des buveurs de Chenas}
\begin{eqnarray*}
	T_1 &=& \sigma_{cru='chenas'}\\
	T_2 &=&  pc(T_1, abus)\\
	T_3 &=& \sigma_{T_1.numV = abus.numV}(T_2)\\
	Res &=& \Pi_{numB}(T3)
\end{eqnarray*}

\subsubsection{Nom et prénom des buveurs de chénon et de Tariquet}
\remarque{Autorisation d'utiliser le join\\~}

\begin{eqnarray*}
	T_1 &=&  \sigma_{cru='chenon' ou cru='tariquet'}(vins)\
	T_2&=&   join(T_1, abus, T_2.numV= abus.numV)\\
	T_3 &=& join(T_2, buveurs, T_2.numB=buveurs.numB)\\
	Res &=& \Pi_{nom, prenom}(T_3)
\end{eqnarray*}
