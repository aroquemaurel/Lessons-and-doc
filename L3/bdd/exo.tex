\chapter{Exercices}
\section{Algèbres relationnel}
\subsection{Opérateurs algébriques}
\texttt{ vins(numv, cru, mill, region, degré) } \\
\texttt{ buveurs(nums, nom, prenom, ville) } \\	
\texttt{ abus(nomv, nomb, date, quantite, place) } \\

	
\begin{attention}
Avec les opérateurs algébriques et uniquement les opérateurs de base. 
\end{attention}

\subsubsection{ Donner le degré des vins de cru Morgon et Millésime 2001 }
\begin{eqnarray*}
	T_1 &=& \sigma_{cru='morgon'~and~mill=2001}(vins)\\
	T_2 &=& \pi_{degré}(T_1)
\end{eqnarray*}
\subsubsection{Numéro des buveurs de Chenas}
\begin{eqnarray*}
	T_1 &=& \sigma_{cru='chenas'}\\
	T_2 &=&  pc(T_1, abus)\\
	T_3 &=& \sigma_{T_1.numV = abus.numV}(T_2)\\
	Res &=& \Pi_{numB}(T3)
\end{eqnarray*}

\subsubsection{Nom et prénom des buveurs de chénon et de Tariquet}
\begin{remarque}{Autorisation d'utiliser le join\\~}\end{remarque}

\begin{eqnarray*}
	T_1 &=&  \sigma_{cru='chenon'\ ou\ cru='tariquet'}(vins)\\
	T_2&=&   join(T_1, abus, T_2.numV= abus.numV)\\
	T_3 &=& join(T_2, buveurs, T_2.numB=buveurs.numB)\\
	Res &=& \Pi_{nom, prenom}(T_3)
\end{eqnarray*}

\subsubsection{Noms des buveurs ayant bu uniquement du Tavel}\label{q4}
\begin{eqnarray*}
	T_1 &=& \sigma_{cru = 'Tavel'}(vins)\\
	T_2 &=& \sigma_{cru \neq tavel}(vins)\\
	T_3 &=& join(abus, T_1, t_1.numv=abus.numv)\\
	T_4 &=& join(abus, T_2, T_2.numv=abus.numv)\\
	T_5 &=& \Pi_{numB}(T_3)\\
	T_6 &=& \Pi_{numB}(T_4)\\
	T_7 &=& difference(T_5, T_6)\\
	T_8 &=& join(T_7,\ buveurs,\ T_7.numB = buveurs.num7)\\
	Res &=& \Pi_{nom}(T_8)
\end{eqnarray*}
\subsubsection{Écrire \ref{q4} en langage algébrique} 
\begin{eqnarray*}
	&&\Pi_{nom}(join(difference(\Pi_{numB}(\\
	&&join(abus a, \sigma_{cru='Tavel'}(vins), abus.numv=vins.numv)), \\
	&&\Pi_{numB}(join(abus, \sigma_{cru\neq 'tavel'(vins)}, abus.numv=vins.numv))),\\
	&& buveurs, a.numB=buveurs.numB))
\end{eqnarray*}
\subsection{Arbres algébriques}
\texttt{Trafic(nTrain, nLige, gare)} \\
\texttt{Trains(nTrains, nRegion)} \\	
\texttt{lignes(nLignes, rang, gare)} \\
\texttt{wagons(nWagon, type, poidsVide, capacie, etat)} \\
\subsubsection{Types de wagons du train 4001}
\subsubsection{Numéro et types des wagons des ligne l11 et l12}
\subsubsection{Numéro des wagons communcs aux lignes l11 e l12}
\subsubsection{Numéro des trains ayant au moins 2 wagons vides}
\subsubsection{Liste des gares de dessertes pour le train 4001}
\subsubsection{Numéro des lignes qui partent de la gare de Toulouse}
\section{Dépendances fonctionnelles}
\subsection{}
\begin{eqnarray*}
R1 &=&  <u1,F1>\\
u1 &=&  \{nuPiece, pri, tva, libellé, catégorie\}\\
F1 &=&  \{d1: nuPiece\rightarrow prix, d2: nuPiece \rightarrow libelle, \\&&d3: nF \rightarrow catalogue, d4: catalogue \rightarrow tva, d5: nP \rightarrow tva\}
\end{eqnarray*}
\subsubsection{Forme normale de $R_1$ ?}
\subsubsection{Proposer un schéma en 3NF}
