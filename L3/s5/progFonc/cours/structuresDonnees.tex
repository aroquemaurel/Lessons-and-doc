\chapter{Structures de données}
\section{N-uplets}
Permet de rassembler des informations dont on connait à l'avance le nombre et le type éventuellement hétérogène.
	\begin{lstlisting}[language=Caml, caption=Syntaxe de n-uplets, numbers=none]
(x1, x2, ..., xn)		
tx1 * tx2 * ... * txn
\end{lstlisting}

\begin{exemple}
	\begin{lstlisting}[language=Caml, caption=Exemple n-uplets, numbers=none, framerule=0pt]
# (1, 2, 3);;
-: int * int * int = 1, 2, 3
# (1, (2,3));; 
int * (int * int)
# let first = fun (x,y) -> x ;;
val first = ('a*'b) -> 'a = <fun>
# let consCple = fun x -> fun y -> (x, y);;
val consCple = 'a -> 'b -> ('a*'b) = <fun>
\end{lstlisting}
\end{exemple}

Ils permettent de mettre des informations hétérogènes, cependant la dimension doit être connue à l'avance. L'avantage étant l'accès aux informations
par filtrage.

Il faut que N soit raisonnable, pour un grand nombre d'informations ce n'est pas adapté

\section{Listes}
Elles permettent de rassembler N informations avec un N quelconque, cependant les informations doivent être homogènes.

On définie une liste par induction : la liste est vide, ou l'ajout d'un élément.

En Caml le type est list, noté \texttt{'a list}, correspondant à une liste dont tous les éléments sont de type a. Le constructeur d'une liste vide se
fait avec \texttt{[]}.  L'ajout d'un élément se fait systématiquement en tête, à l'aide de l'opérateur \texttt{::}, \texttt{e::l} ajoute l'élément e en tête de la liste.

	\begin{lstlisting}[language=Caml, numbers=none, framerule=0pt]
e1::e2::...::en::[]	
[e1;e2;...;e3]
	\end{lstlisting}
	\begin{exemple}
	\begin{lstlisting}[language=Caml, caption=Exemple de listes, numbers=none, framerule=0pt]
#[];;
- : 'a list = []
#let ex = 1::[];;
val ex : int list = [1]
#let ex1 = 1 ::2::3::[];;
val ex1 : int list = [1;2;3]
#let ex2 = 0::ex1
val ex2 : int list = [0;1;2;3]
#let ex3 = [1,2]::({3,4]::[]);;
[[(1,2)];[(3,4)]]
let ex4 = (1,tue)::[(2,false)}
ex4 : (int * bool) lt = [(1,true);(2,false)]
	\end{lstlisting}
	\end{exemple}

	On peut utiliser le filtrage sur les listes avec les filtrage \texttt[] et \texttt{::l} permettant respectivement de savoir si une liste est vide ou
	d'accéder aux informations: tête(e), reste(l)

	\subsection{Fonctions récursives sur les listes}
	Le cas d'arrêt est 1 ou plusieurs élément, le cas général correspond souvent à diminuer la taille de la liste.

\lstinputlisting[language=Caml, caption=Fonctions sur les listes]{listes.ml}

\subsubsection{La fonction \texttt{append}}
La fonction \texttt{append} est une fonction qui concatène deux listes, elle pourrait être implémenter comme ceci: 
\begin{lstlisting}[language=Caml, caption=Exemple d'implémentation de \texttt{append}]
(* Concatène deux liste, version naive *)
let rec append = fun l1 l2 
    match (l1,l2) with
     ([],[]) -> [] 
    |([]::e, t::q) -> t::q
    |([t1:::q1, []) -> t1::q1
    |(t1::q1, t2::q2) -> t1::(append q1 (t2::q2));;

(* Bonne version de append *)
let rec append = fun l1 l2 ->
    match l1 with
     [] -> l2
    |t::q -> t::(append q l2);;
\end{lstlisting}

\begin{remarque}
Cette fonction est déjà disponible dans le langage OCaml, pour cela on peut utiliser l'opérateur \texttt{@} : \texttt{l1@l2}
\end{remarque}

\subsection{La fonction \texttt{reverse}}
La fonction \texttt{reverse} retourne la liste construite à l'envers.
\lstinputlisting[language=Caml, caption=Fonction \texttt{reverse}]{reverse.ml}

\subsubsection{La fonction \texttt{nbOcc}}
Fonction qui compte le nombre d'occurrence d'un élément dans une liste.
\lstinputlisting[language=Caml, caption=Fonction \texttt{nbOcc}]{nbOcc.ml}
\subsubsection{La fonction \texttt{remove}}
Fonction qui supprime toutes les occurrences du premier paramètre de la liste présente dans le second paramètre.
\lstinputlisting[language=Caml, caption=Fonction \texttt{remove}]{remove.ml}

\subsection{Fonctions de tris}
\subsubsection{Tri par insertion}
\begin{description}
	\item[Fonction \texttt{inserer}] Cette fonction insère un élément dans une liste déjà trié.\\
\lstinputlisting[language=Caml, caption=Fonction \texttt{inserer}]{inserer.ml}
	\item[Fonction \texttt{triInsertion}] Tri la liste avec l'algorithme du tri par insertion\\
		\lstinputlisting[language=Caml, caption=Fonction \texttt{triInsertion}]{triInsertion.ml}
\end{description}

\subsubsection{Tri par fusion}
\begin{description}
	\item[Fonction \texttt{partage}]Partage une liste en deux. Prend une liste en paramètre et retourne deux lites.
	\lstinputlisting[language=Caml, caption=Fonction \texttt{partage}]{partage.ml}
	\item[Fonction \texttt{merge}] Fusionne deux listes en une seule.
	\lstinputlisting[language=Caml, caption=Fonction \texttt{merge}]{merge.ml}
\end{description}

\section{Types utilisateurs}
L'objectif est de pouvoir définir un type représentant l'union de types existants.
Par exemple comment faire pour manipuler des nombres qui
soient des entiers, des réels ou des complexes ? Définition d'un type par l'utilisateur qui représente une
union de type

\subsection{Définition d'un type}
\lstinputlisting[language=Caml, caption=Définition d'un type utilisateur]{defType.ml}
\begin{exemple}
	\begin{lstlisting}[language=Caml, caption=Exemple de listes, numbers=none, framerule=0pt]
#[];;
- : 'a list = []
#let ex = 1::[];;
val ex : int list = [1]
#let ex1 = 1 ::2::3::[];;
val ex1 : int list = [1;2;3]
#let ex2 = 0::ex1
val ex2 : int list = [0;1;2;3]
#let ex3 = [1,2]::({3,4]::[]);;
[[(1,2)];[(3,4)]]
let ex4 = (1,tue)::[(2,false)}
ex4 : (int * bool) lt = [(1,true);(2,false)]
	\end{lstlisting}
	\end{exemple}

	On peut utiliser le filtrage sur les listes avec les filtrage \texttt[] et \texttt{::l} permettant respectivement de savoir si une liste est vide ou
	d'accéder aux informations: tête(e), reste(l)

	\subsection{Fonctions récursives sur les listes}
	Le cas d'arrêt est 1 ou plusieurs élément, le cas général correspond souvent à diminuer la taille de la liste.

\lstinputlisting[language=Caml, caption=Fonctions sur les listes]{listes.ml}

\subsubsection{La fonction \texttt{append}}
La fonction \texttt{append} est une fonction qui concatène deux listes, elle pourrait être implémenter comme ceci: 
\begin{lstlisting}[language=Caml, caption=Exemple d'implémentation de \texttt{append}]
(* Concatène deux liste, version naive *)
let rec append = fun l1 l2 
    match (l1,l2) with
     ([],[]) -> [] 
    |([]::e, t::q) -> t::q
    |([t1:::q1, []) -> t1::q1
    |(t1::q1, t2::q2) -> t1::(append q1 (t2::q2));;

(* Bonne version de append *)
let rec append = fun l1 l2 ->
    match l1 with
     [] -> l2
    |t::q -> t::(append q l2);;
\end{lstlisting}

\begin{remarque}
Cette fonction est déjà disponible dans le langage OCaml, pour cela on peut utiliser l'opérateur \texttt{@} : \texttt{l1@l2}
\end{remarque}

\subsection{La fonction \texttt{reverse}}
La fonction \texttt{reverse} retourne la liste construite à l'envers.
\lstinputlisting[language=Caml, caption=Fonction \texttt{reverse}]{reverse.ml}

\subsubsection{La fonction \texttt{nbOcc}}
Fonction qui compte le nombre d'occurrence d'un élément dans une liste.
\lstinputlisting[language=Caml, caption=Fonction \texttt{nbOcc}]{nbOcc.ml}
\subsubsection{La fonction \texttt{remove}}
Fonction qui supprime toutes les occurrences du premier paramètre de la liste présente dans le second paramètre.
\lstinputlisting[language=Caml, caption=Fonction \texttt{remove}]{remove.ml}

\subsection{Fonctions de tris}
\subsubsection{Tri par insertion}
\begin{description}
	\item[Fonction \texttt{inserer}] Cette fonction insère un élément dans une liste déjà trié.\\
\lstinputlisting[language=Caml, caption=Fonction \texttt{inserer}]{inserer.ml}
	\item[Fonction \texttt{triInsertion}] Tri la liste avec l'algorithme du tri par insertion\\
		\lstinputlisting[language=Caml, caption=Fonction \texttt{triInsertion}]{triInsertion.ml}
\end{description}

\subsubsection{Tri par fusion}
\begin{description}
	\item[Fonction \texttt{partage}]Partage une liste en deux. Prend une liste en paramètre et retourne deux lites.
	\lstinputlisting[language=Caml, caption=Fonction \texttt{partage}]{partage.ml}
	\item[Fonction \texttt{merge}] Fusionne deux listes en une seule.
	\lstinputlisting[language=Caml, caption=Fonction \texttt{merge}]{merge.ml}
\end{description}

\section{Types utilisateurs}
L'objectif est de pouvoir définir un type représentant l'union de types existants.
Par exemple comment faire pour manipuler des nombres qui
soient des entiers, des réels ou des complexes ? Définition d'un type par l'utilisateur qui représente une
union de type

\subsection{Définition d'un type}
\lstinputlisting[language=Caml, caption=Définition d'un type utilisateur]{defType.ml}
\begin{exemple}
	\begin{itemize}
		\item Définir le type \texttt{couleur} avec les 3 couleurs : bleu, blanc et rouge
	\begin{lstlisting}[language=Caml, numbers=none, framerule=0pt]
type couleur = Bleu | Blanc | Rouge;;		
	\end{lstlisting}
\item Définir le type \texttt{carte} à jouer avec les 3 figures (roi, dame et valet) et les cartes numérotées de 1 à 10
	\begin{lstlisting}[language=Caml, numbers=none, framerule=0pt]
type carte = Valet | Dame | Roi | Valeur of int
	\end{lstlisting}
\item Définir le type \texttt{nombre} comme étant soit un entier, soit un réel, soit un complexe (couple de 2 réels)
	\begin{lstlisting}[language=Caml, numbers=none, framerule=0pt]
type nombre = Entier of int | Reel of float | Complexe of float * float;;	
	\end{lstlisting}
\item Définir le type \texttt{automateFini} non déterministe comme étant un triplet avec son état initial, la fonction de transition qui associe à un
	état et un symbole la liste es états successeurs, et le prédicat qui indique si un état est final ou non.
	\begin{lstlisting}[language=Caml, numbers=none, framerule=0pt]
type ('e, 's) automate = ('e * 
						 ('e -> 's -> 'e list) *
						 ('e -> bool));;
	\end{lstlisting}
\item Définir le type de \texttt{monAutomate}, instanciation du type générique précédent, avec des entiers pour représenter les états et des caractères
	pour représenter les symboles
	\begin{lstlisting}[language=Caml, numbers=none, framerule=0pt]
type monAutomate = (int,int) automate;;
	\end{lstlisting}
\item Définir le type des \texttt{arbres binaires} comme étant soit une feuille contenant une information, soit un nœud avec 2 sous arbres.
	\begin{lstlisting}[language=Caml, numbers=none, framerule=0pt]
type 'a arbreBinaire = 
	 Feuille of 'a 
	|Noeud of 'a arbreBinaire * 'a arbreBinaire;;
	\end{lstlisting}
\item Définir le type des \texttt{arbres n-aires} comme étant un noeud contenant une information et un nombre quelconque d'arbres n-aire comme fils
	\begin{lstlisting}[language=Caml, numbers=none, framerule=0pt]
type 'a arbreNaire = 
Node of 'a * 'a arbreNaire lits
	\end{lstlisting}
\item Définir le type \texttt{jour} de la semaine
	\begin{lstlisting}[language=Caml, numbers=none, framerule=0pt]
type jour = Lundi|Mardi|Mercredi|Jeudi|Vendredi|Samedi|Dimanche;;
	\end{lstlisting}
\item Définir le type Résultats comme étant soit un entiers, soit une erreur
	\begin{lstlisting}[language=Caml, numbers=none, framerule=0pt]
	type resultat = Entier of int | Erreur;;
	\end{lstlisting}
\item Définir le type option paramètre cmme étant soit rien soit un élément du type paramètre
	\begin{lstlisting}[language=Caml, numbers=none, framerule=0pt]
	type option = Rien | Quelque chose of 'a;;
	\end{lstlisting}
\item Définir le type action du langage LOGO sachant qu'une action peut être: tourner d'un certain nombre de degrés, avancer d'une certaine distance,
	lever ou poser le crayon, une séquence de 2 actions ou la répétition d'une action un certain nombre de fois.
	\begin{lstlisting}[language=Caml, numbers=none, framerule=0pt]
type actionLogo = 
	 Tourner of int
	|Avancer of int
	|Lever
	|Poser
	|Seq of actionLogo * actionLogo
	|Rep of actionLogo * int;;
	\end{lstlisting}
	\end{itemize}
\end{exemple}

\begin{attention}
	Un constructeur doit toujours commencer par une majuscule.
\end{attention}

\subsection{Instanciation de type}
Cela se fait à l'aide de l'appel au constructeur
\begin{exemple}
\begin{itemize}
	\item Définir un objet bleu du type \texttt{couleur}
	\begin{lstlisting}[language=Caml, numbers=none, framerule=0pt]
let c = Bleu;
	\end{lstlisting}
\item Définir les objets valets, 3 et 10 du type \texttt{carte}
	\begin{lstlisting}[language=Caml, numbers=none, framerule=0pt]
let c1 = Valet and c2 = Valeur 3 and c3 = Valeur 10;;
	\end{lstlisting}
\item Définir un complexe et un réel du type \texttt{nombre}
	\begin{lstlisting}[language=Caml, numbers=none, framerule=0pt]
let nbCom = Complexe(1.0, 2.0) and nbR = Reel 5.6;;
	\end{lstlisting}
\item Définir un arbre binaire du type \texttt{ab} contenant les entiers 4,5,6
	\begin{lstlisting}[language=Caml, numbers=none, framerule=0pt]
let monArbre = Noeud (Feuille 4,
					  Noeud (Feuille 5, Feuille 6)
					  );;
	\end{lstlisting}
	\item Définir l'arbre n-aire du type \texttt{arbreNaire} suivant
	\begin{lstlisting}[language=Caml, numbers=none, framerule=0pt]
let monArbre = Node(1 [	Node (2,[]) ; 
						Node (3, [Node (5, []) ; Node(6, [])]) ;
						Node (4, []]
					);;
	\end{lstlisting}
	~\\~\\~
	\item Définir un objet de type \texttt{actionLogo} qui dessine un carré e 10 cm de côté à l'endroit ou se trouve le crayon.
	\begin{lstlisting}[language=Caml, numbers=none, framerule=0pt]
let carre = ActionLogo Seq(Poser, 
							Rep(Seq(Avancer 1à, Tourner 90),
								4)
							);
	\end{lstlisting}
\end{itemize}
\end{exemple}

\subsection{Fonctions manipulant des objets de type utilisateur}
\begin{exemple}
	\begin{itemize}
		\item Écrire la fonction qui associe un numéro (1,2 ou 3) à chaque couleur ; donner son type.
	\begin{lstlisting}[language=Caml, numbers=none, framerule=0pt]
let assoCouleur = function 
	 Bleu -> 1
	|Blanc -> 2
	|Rouge -> 3
(* var assoCouleur : couleur -> int = <fun> *)	
	\end{lstlisting}
\item Écrire la fonction qui donne le lendemain d'un jour de la semaine ; donner son type.
	\begin{lstlisting}[language=Caml, numbers=none, framerule=0pt]
let lendemain = function
	 Lundi -> Mardi
	|Mardi -> Mercredi
	|Mercredi -> Jeudi
	|Jeudi -> Vendredi
	|Vendredi -> Samedi
	|Samedi -> Dimanche
	|Dimanche -> Lundi;;
(* val lendemain  : jour -> jour = <fun> *)	
	\end{lstlisting}
\item Écrire la somme des points d'une liste de cartes (les figures valent 10)
	\begin{lstlisting}[language=Caml, numbers=none, framerule=0pt]
let valeurCarte = function
	 Valeur n -> n 
	|_ -> 10;; (* figures forcément *)
let rec calculPointsPaqets = function ->
	 [] -> 0	
	|t::q ->  valeurCarte t + (sommePts q);;
	\end{lstlisting}
\item Écrire la fonction qui fait la somme de 2 objets de types nombres : on peut faire la somme de 2 entiers, de 2 réels, de 2 complexes, sinon c'est
	une erreur.
	\begin{lstlisting}[language=Caml, numbers=none, framerule=0pt]
let sommeNombre = fun nb1 -> fun nb2 ->
	match(nb1,nb2) with
	 (Entier n, Entier m) -> Entier (n + m)
	|(Reel r1, Reel r2) -> Reel(r1 +. r2)
	|(Complexe(r1,i1), Complexe(r2,i2)) -> Complexe(r1 +. r2, i1 +. i2);;
	|_ failith "On ne mélange pas !";
	(* val somme: nombre -> nombre -> nombre = <fun> *)
	\end{lstlisting}
	\end{itemize}
\end{exemple}

