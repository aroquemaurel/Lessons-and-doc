\chapter{Syntaxe de base}
\section{Action}
\begin{lstlisting}[language=Caml, caption=Syntaxe de base]
# expression ;;
-: valeur : type
#
\end{lstlisting}
\begin{itemize}
	\item Lire l'expression jusqu'au \texttt{;;}
	\item Typer
		\begin{itemize}
			\item Si ko $\Rightarrow$ Message d'erreur
			\item Si ok $\Rightarrow$ Évaluation $\Rightarrow$ << Réduire, calculer >> $\Rightarrow$ Résultat / Valeur
		\end{itemize}
\end{itemize}
\section{Types de base}
\begin{tabular}{c|c|c|c|c}
	\textbf{Type} & \textbf{Mot clé }& \textbf{Opération} & \textbf{Comparaison} & \textbf{Exemple} \\
	\hline
	Entiers($\mathbb Z$) & int & \texttt{+, -, *, /, mod}& \texttt{=, >, <, >=, <=, <>}&2013\\
	Flottants & float &\texttt{+., -., *., /., sqrt, **} &Polymorphe  &2013.0\\
	Chaines & string & \texttt{"\_\_", \^} & Polymorphe &\texttt{"coucou"}\\
	Caractères & char &\texttt{'\_'} &\texttt{}Polymorphe & 'c'\\
	Booléens & bool & \texttt{true, false, \&\&, ||, not}&\texttt{}Polymorphe &\\
\end{tabular}
\section{Structures de contrôles}
\subsection{Conditions}
\begin{lstlisting}[language=Caml, caption=Syntaxe de la condition]
# if condition then action else alternative ;;		
\end{lstlisting}
\begin{remarque}
	\begin{itemize}
		\item La condition doit être un booléen.
		\item L'action et l'alternative doit être du même type
	\end{itemize}
\end{remarque}
\subsection{Filtrage (pattern matching)}
	Filtre ou motif, permet d'exprimer la syntaxe d'une donnée. On écrit la fonction par cas, c'est-à-dire on filtre la donnée avec un
	filtre\footnote{ou pattern}.
	\begin{lstlisting}[language=Caml, caption=Syntaxe du filtrage, numbers=none]
match expr with
pat1 -> expr1
| pat2 -> expr2
| pat31|pat32|pat33 -> expr3 (* un des pattern retourne expr3 *)
| patn -> exprn
| _ -> not b;; (* default *)
	\end{lstlisting}
	On examine en séquence et essaye de filtrer successivement l'expression avec le pattern i, le premier à marcher sera appliqué. 

	\begin{attention}
		Les pattern doivent tous être de même type afin que cela fonctionne.\\~
	\end{attention}
\begin{exemple}
	Les exemples ci-dessous utilisent des fonctions, celles-ci sont détaillées dans le chapitre \ref{fonctions}.
	\begin{lstlisting}[language=Caml, caption=Exemple filtrage, numbers=none, framerule=0pt]
# let nand = fun a -> fun b -> 
match a with false -> true
			| _ -> not b;;

\end{lstlisting}
Écrire la fonction d'implication.
\begin{tabular}{c|c||c}
	A&B&A$\rightarrow$B\\
	T&T&T\\
	T&F&F\\
	F&T&T\\
	F&F&T\\
\end{tabular}
	\begin{lstlisting}[language=Caml, caption=Exemple filtrage -- Implication, numbers=none, framerule=0pt]
# let impl = fun a -> fun b -> 
	match (a,b) with 
	 (true,true) -> true
	|(true,false) -> false
	|(false,true) -> true
	|(false,false) -> true;;
(* Autre manière plus élégante *)
# let imp = fun a -> fun b -> 
	match (a,b) with
	 (true, false) -> false
	|_ -> true;;
\end{lstlisting}
\end{exemple}

\section{Variables}
Un définition peut être de plusieurs type : 
\begin{itemize}
	\item Globale
	\item Locale
	\item Simultanée
\end{itemize}
~
\subsection{Définition globale}
\begin{lstlisting}[language=Caml, caption=Définition de variable]
# let variable = expression;;
\end{lstlisting}
L'interpréteur va évaluer la valeur et donner un type à la variable, il effectue une liaison \texttt{<var,val>}, ceci peut aussi s'appeler une
fermeture.

On ajoute la liaison à l'environnement, un environnement est donc un ensemble ordonné de liaisons.
\subsection{Définition Locale}
\begin{lstlisting}[language=Caml, caption=Définition de variable]
# let variable = expression 1
in expression2 ;;
\end{lstlisting}
La définition et temporaire
\begin{enumerate}
	\item Évaluer l'expression dans l'environnement courant
	\item Ajouter à l'environnement courant la nouvelle. Liaison \texttt{var,val1}
	\item Évaluer l'expression 2 dans ce nouvel environnement augmenté $\Rightarrow$ Résultat
	\item Restituer environnement de départ
\end{enumerate}

\subsection{Définitions simultanées}
\begin{lstlisting}[language=Caml, caption=Définition de variable]
# let var1 = expression1 
and var2 = expression2
and var3 = expression3;;
\end{lstlisting}

\section{Exceptions}
Dans le cas où on ne veut pas rattraper une exception, celles-ci peuvent s'effectuer simplement à l'aide de \texttt{failwith} suivi du message
d'erreur.

