\chapter{Exercices}
\section{Fonctions}
\subsection{Récursivité sur les entiers}
\subsubsection{Parité}
Donner la définition récursive de la fonction pair qui retourne un booléen si n est pair : 
\begin{lstlisting}[language=algo]
si n = 0 alors true
si n = 1 alors false
si n > 1 alors pair(n-2)
si n < 0 alors erreur
\end{lstlisting}
\begin{lstlisting}[language=Caml, caption=Exercice -- Fonction pair en récursif]
# let pair = fun n ->
	let rec verifMult2 = fun x ->
		match x with
			 0 -> true
			|1 -> false
			|p -> verifMult2(p-2)
	in if n < 0 then verifMult2(-n)
		else verifMult2 n;;
int -> bool = <fun>
\end{lstlisting}

\subsubsection{\texttt{sommeCarres}}
Écriture d'une fonction qui effectue la somme des carrés : $1^2+2^2+3^2+\cdots+n^2$
\begin{lstlisting}[language=Caml, caption=Exercice -- Fonction pair en récursif]
# let sommeCarres = fun n ->
	let rec funRecCarres = fun x -> 
		if x = 0 then 0
		else (x*x) + funRecCarres(x-1)
	in if n < 0 then funRecCarres (-n)
		else funRecCarres n;;
\end{lstlisting}

\subsubsection{\texttt{sommeFonctions}}
\begin{lstlisting}[language=Caml, caption=Exercice -- Fonction pair en récursif]
# let sommeFonctions = fun f -> fun n ->
	let rec calcul = fun x ->
		if x = 0 then 0
		else (f x) + calcul(x-1)
	in if n < 0 then calcul(-n)
		else calcul n;;
\end{lstlisting}
\subsection{\texttt{PGCD}}
Écrire la fonction pgcd tel que \texttt{pgcd a b} est égale au lus grand diviseur de a et de b.

Par soustraction successive, le pgcd de a et b est le pgcd du plus petit des 2 et de la valeur absolue de leur différence $a>0$ et
$b>0$.
\begin{lstlisting}[language=algo, caption=Exercices -- Algorithme pgcd]
	si a = b alors a
	si a < b alrs pgcd a (b-a)
	si a > b alors pgcd (a-b) b
\end{lstlisting}

\begin{lstlisting}[language=Caml, caption=Exercice -- Fonction pgcd]
let pgcd = fun a -> fun b ->
	let rec trait = fun x -> fun y -> (* x > 0 et y > 0*)
		if x = y then x (* cas d'arrêt *)
		else if x < y then trait x (y-x) (* Appel récursif *)
		else trait (x-y) y (* Appel récursif *)
	in if(a > 0) && (b > 0) then trait a b
	   else failwith "PGCD, entiers négatifs ou nuls";;
\end{lstlisting}

\subsection{Exercices diverses sur les fonctions}
\subsubsection{dernierChiffre}
\subsubsection{Son argument privé de son dernier chiffre}
\subsubsection{nombre d'occurence d'un chiffre}
Compte le nombre d'occurrence d'un chiffre dans l'écriture décimale d'un entier
\section{Structures de données}
\subsection{Liste}
\lstinputlisting[language=Caml, caption=Exercice -- listes]{exos/listes.ml}

\lstinputlisting[language=Caml, caption=Exercice -- fonctions sur les listes]{exos/listes2.ml}
