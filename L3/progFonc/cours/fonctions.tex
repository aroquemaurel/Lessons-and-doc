\chapter{Fonctions}
\label{fonctions}
\section{Définition}
	\begin{lstlisting}[language=Caml, caption=Sytaxe d'une définition de fonction]
# fun param -> corps;;		
# function param -> corps;;
\end{lstlisting}
\begin{remarque}Il existe une différence entre \texttt{fun} et \texttt{function}. \texttt{function} permet d'alléger l'écriture en cas
	de pattern matching. En effet 
	\begin{lstlisting}[language=Caml, numbers=none, framerule=0pt]
	fun x -> match x with (* contenu du match ... *);;
\end{lstlisting}

	est équivalent à l'écriture suivante : 	
	\begin{lstlisting}[language=Caml, numbers=none, framerule=0pt]
	function (* contenu du match ... *);;
	\end{lstlisting}
	\end{remarque}
			\begin{exemple}
	\begin{lstlisting}[language=Caml, caption=Sytaxe d'une définition de fonction, numbers=none, framerule=0pt]
# fun x -> x + 1;;		
int -> int = <fun>
# let succ = func x -> x + 1;; 
succ: int -> int = <fun>
\end{lstlisting}
				\end{exemple}
\begin{remarque}
Il est possible de tracer une fonction, ceci- s'effectue à l'aide de la commande suivante, cette trace est succincte mais permet de savoir ce qui se
passe lors de l'exécution de la fonction:  \texttt{\#trace fonction}

Afin d'enlever la trace, il suffit d'utiliser \texttt{\#untrace fonction}

	Cela peut s'avérer particulièrement utile pour les fonctions récursives, cf \ref{recurseInt}.
\end{remarque}

				\section{Application}
				\subsection{Valeur d'une fonction dans un environnement $\Gamma$}
				Évaluer une fonction \texttt{fun x -> corps} dans $\Gamma$ nous donne la fermeture suivante $<\Gamma, x, cors>$
				\subsection{Application d'une fonction à un argument dans $\Gamma_2$}
				\begin{itemize}
					\item Évaluer f dans $\Gamma_1$
					\item Évaluer a dans $\Gamma_1$ Soit v la valeur de a dans $\Gamma_1$
					\item Soit $<\Gamma, x, corps>$ la valeur de f dans $\Gamma_1$
					\item On << branche x et v >> et on évalue le corps de la fonction dans l'environnement où x est lié à v a été ajouté à $\Gamma$
					\item Résultat
				\end{itemize}
			\begin{exemple}
	\begin{lstlisting}[language=Caml, caption=Exemple d'utilisation de fonctions, numbers=none]
# let x = 2013;;
val x : int = 2013
# let y = x + 10
	and z = x * 10;;
val y : int = 2013 z = int : 20130
# let f = func x -> x + z + y;;
val f: int -> int = <fun>
# f(y+1);;
\end{lstlisting}
			\end{exemple}
			\section{Fonction à n paramètres}
	\begin{lstlisting}[language=Caml, caption=Sytaxe d'une définition de fonction à n paramètres, numbers=none]
# fun x1 -> fun x2 -> fun x3 -> ... -> fun xn -> corps;;
# fun x1 x2 x3 ... xn -> corps;;
\end{lstlisting}

Une fonction à N paramètre fonctionne à l'aide de N fonctions à 1 paramètre.  
\section{Application partielle}
	\begin{lstlisting}[language=Caml, caption=Application partielle, numbers=none]
# let creerPredPGQ = fun x -> fun y -> x > y;;
# let plusGrandQue10 = creePredPGQ 10;
# plusGrandQue10 5;;
- : bool = false 
\end{lstlisting}

\section{Fonctions récursives sur les entiers}\label{recurseInt}
Une récursive implique qu'il y ai une référence au nom de la fonction dans le corps de cette même fonction. Systématiquement un cas d'arrêt de la
fonction doit être présent, ceci afin que la fonction se termine, éventuellement des cas d'erreurs peuvent être gérés.
\begin{exemple}
	Fonction qui étant donné un entier n, calcule sa factorielle c'est-à-dire $n!$ en sachant que : 
	\begin{eqnarray*}
		n &\leq& 0\\\ &&\\
		0 &\neq& 1\\
		n &\neq& n\times (n-1)! = 1\times 2\times 3\times\cdots\times (n-1)\times n
	\end{eqnarray*}
	\begin{lstlisting}[language=Caml, caption=Exemple de la fonction factorielle en récursif, numbers=none, framerule=0pt]
let rec fact = fun n -> 
	if n = 0 then 1 (* arrêt *)
	else n * fact(n-1);; (* cas général*)
	\end{lstlisting}
	 
	La solution en utilisant le filtrage : 
	\begin{lstlisting}[language=Caml, caption=Exemple de la fonction factorielle avec filtrage, numbers=none, framerule=0pt]
let rec fact = fun n ->
	match n with
		 0 -1 1
		|p -> p*fact(p-1);;
	\end{lstlisting}
\end{exemple}

Les vérifications du domaine d'application doivent se faire en dehors de la fonction récursives. En effet, il est inutile de tester le cas d'erreur à
chaque appel récursif. Ceci peut s'effectuer de la manière suivante : 
	\begin{lstlisting}[language=Caml, caption=Exemple de la fonction factorielle avec un cas d'erreur, numbers=none]
let fact = fun n ->
	let rec calcul = fun x ->
		if x = 0 then 1 (* cas d'arrêt *)
		else x * calcul(x-1) (* cas général *)
	in if n < 0 then failwith "erreur nombre négatif"
		else calcul n;;
	\end{lstlisting}

	\begin{remarque}
		Il faut faire attention, dans ce cas il est donc impossible de tracer le cas d'erreur, en effet la fonction n'est pas visible de l'extérieur.
	\end{remarque}

	\subsection{Fonction mutuellement récursives}
	Il est également possible d'avoir deux fonctions que l'on appelle mutuellement récursives. C'est-à-dire que les fonctions s'appellent les une les
	autres, la première fonction appelle la seconde et la seconde appelle la première.

	\begin{exemple}
		Écriture de la fonction pair avec deux fonctions mutuellement récursives.	
\begin{lstlisting}[language=algo, numbers=none, framerule=0pt, caption=Algorithme de la récursivité mutuelle pour pair]
fonction pair	
	si n = 0 alors true
	sinon impair(n-1)

-- Avec la fonction impair comme ceci : 
fonction impair
	si n = 0 alors false 
	sinon pair(n-1)
\end{lstlisting}
Ce qui se traduirait de la façon suivante en Caml, les deux fonctions doivent être déclarés en même temps, pour ceci on utilise le \texttt{and}.
\begin{lstlisting}[language=Caml, caption=Récursivité mutuelle pour pair, numbers=none, framerule=0pt]
# let rec pair = fun n ->
	if n = 0 then true
	else impair(n-1)
and impair = fun n -> 
	if n = 0 then false
	else pair (n-1);;
\end{lstlisting}

	\end{exemple}

	\begin{tabular}{c|c}
		Cas d'arrêt & Diminution taille du problème\\
		\hline
		$n=0$ & $n - 1$\\
		$n=1$ & $n - 2$\\
		$a=b$ & $a-b$\\
		$n<10$ & $\frac{n}{10}$
	\end{tabular}

