\chapter{La programmation fonctionnelle}
\section{Différents paradigmes de programmation}
\begin{itemize}
	\item Impératif : C, Java, Ada, \ldots
	\item Objet: Java, C++, \ldots
	\item Fonctionnel : Lisp, Shceme, ML, Caml, Haskell, \ldots
	\item Déclarative ou logique : Prolog
\end{itemize}
\section{Le fonctionnel}
L'outil de base de la programmation fonctionnel est les fonctions. On peut les définir, les appliquer et les composer. Il n'y a pas d'affectation
en fonctionnel.

Le fonctionnel est partis d'une base théorique avec le $\lambda$ calcul en 1936,c'est un langage sûr. C'était d'abord non typé\footnote{Comme le
lisp ou, Scheme}, les langages typés sont arrivés ensuite avec la famille Ocaml vers les années 2000.

Un langage fonctionnel typé possède plusieurs propriétés.
\begin{description}
	\item[Inférence de type] On ne déclare pas le type expressément.
	\item[Vérification de type] Vérifier à la compilation, pas de risque de problème lors de l'execution
	\item[Polymorphe]~ 
	\item[Syntaxe simple] Syntaxe non verbeuse, sémantique solide, environnement de développement solide, mise au point facilitée et programmation sûre
\end{description}

\subsection{Mode de compilation}
Le Caml peut être soit compilé soit interprété, l'avantage de la compilation étant l'efficacité et l'interprétation « convivial ». Historiquement
ceux-ci étaient uniquement compilés.
