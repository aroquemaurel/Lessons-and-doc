\documentclass[12pt,a4paper,openany]{article}

\usepackage{lmodern}
\usepackage[svgnames]{xcolor} % Required to specify font color
\input{/home/aroquemaurel/cours/includesLaTeX/couleurs.tex}

\usepackage[utf8]{inputenc}
\usepackage[T1]{fontenc}
\usepackage[francais]{babel}
\usepackage[top=1.7cm, bottom=1.7cm, left=1.7cm, right=1.7cm]{geometry}
\usepackage{verbatim}
\usepackage[urlbordercolor={1 1 1}, linkbordercolor={1 1 1}, linkcolor=vert1, urlcolor=bleu, colorlinks=true]{hyperref}
\usepackage{tikz} %Vectoriel
\usepackage{listings}
\usepackage{fancyhdr}
\usepackage{multido}
\usepackage{float}
\usepackage{amssymb}
\usepackage{graphicx} % Required for box manipulation

\newcommand{\titre}{Observation bas niveau de protocoles}
\newcommand{\module}{Réseau}
\newcommand{\subtitle}{Observation bas niveau de protocoles}
\newcommand{\auteur}{Antoine de \bsc{Roquemaurel}}
\newcommand{\formation}{L3 Informatique}
\newcommand{\semestre}{5}
\newcommand{\annee}{2013}
\newcommand{\prof}{\bsc{Kacimi}}
\newcommand{\numero}{1}
\newcommand{\typeDoc}{TP}


\newcommand{\pole}{}
\newcommand{\sigle}{reseau}

\usepackage{ifthen}
\date{\today}

\chead{Antoine de \bsc{Roquemaurel}}
\rhead{TP\no\typeDoc}
\lhead{\titre}
%\makeindex

\lfoot{Université Toulouse III -- Paul Sabatier}
\rfoot{\sigle\semestre}
%\rfoot{}
\cfoot{--~~\thepage~~--}

\makeglossary
\makeatletter
\def\clap#1{\hbox to 0pt{\hss #1\hss}}%

\def\haut#1#2#3{%
	\hbox to \hsize{%
		\rlap{\vtop{\raggedright #1}
	}%
	\hss
	\clap{\vtop{\centering #2}
}%
\hss
\llap{\vtop{\raggedleft #3}}}}%
\def\bas#1#2#3{%
	\hbox to \hsize{%
		\rlap{\vbox{
			\raggedright #1
		}
	}%
	\hss \clap{\vbox{\centering #2}}%
	\hss
	\llap{\vbox{\raggedleft #3}}}
}%
\def\maketitle{%
	\thispagestyle{empty}{%
		\haut{}{\@blurb}{}
		%	
		%\vfill

		\begin{center}
			\vspace{-2.0cm}
			\usefont{OT1}{ptm}{m}{n}
			\huge \@type \@title
		\end{center}
		\par
		\hrule height 1pt
		\par
		\vspace{1cm}
		\bas{}{}{}
}%
}
\def\date#1{\def\@date{#1}}
\def\author#1{\def\@author{#1}}
\def\type#1{\def\@type{#1}}
\def\title#1{\def\@title{#1}}
\def\location#1{\def\@location{#1}}
\def\blurb#1{\def\@blurb{#1}}
\date{\today}
\newboolean{monBool}
\setboolean{monBool}{true}
\author{}
\title{}
\ifthenelse{\equal{\typeDoc}{}}{
\numeroTD{}
}
{
	\type{\typeDoc~--- }
}
\location{Amiens}\blurb{}
%\makeatother
\title{\titre}
\author{%Semestre \semestre
}

\location{Toulouse}
\blurb{%
\vspace{-35px}
\begin{flushleft}
	Université Toulouse III -- Paul Sabatier\\
	L2 Informatique\\
\end{flushleft}
\begin{flushright}
	\vspace{-45px}
	\Large \textbf \module \\
	\normalsize \textit \today\\
	Semestre \semestre
	\vspace{30px}
\end{flushright}
Antoine de \bsc{Roquemaurel}
}%



%\title{Cours \\ \titre}
%\date{\today\\ Semestre \semestre}

%\lhead{Cours: \titre}
%\chead{}
%\rhead{\thepage}

%\lfoot{Université Paul Sabatier Toulouse III}
%\cfoot{\thepage}
%\rfoot{\sigle\semestre}

\pagestyle{fancy}





%----------------------------------------------------------------------------------------
%	DEFINITION OF COLORED BOXES
%----------------------------------------------------------------------------------------

\RequirePackage[framemethod=default]{mdframed} % Required for creating the theorem, definition, exercise and corollary boxes

% Theorem box
\newmdenv[skipabove=7pt,
skipbelow=7pt,
backgroundcolor=black!5,
linecolor=ocre,
innerleftmargin=5pt,
innerrightmargin=5pt,
innertopmargin=5pt,
leftmargin=0cm,
rightmargin=0cm,
innerbottommargin=5pt]{tBox}

% Exercise box	  
\newmdenv[skipabove=7pt,
skipbelow=7pt,
rightline=false,
leftline=true,
topline=false,
bottomline=false,
backgroundcolor=ocre!10,
linecolor=ocre,
innerleftmargin=5pt,
innerrightmargin=5pt,
innertopmargin=5pt,
innerbottommargin=5pt,
leftmargin=0cm,
rightmargin=0cm,
linewidth=4pt]{eBox}	

% Definition box
\newmdenv[skipabove=10pt,
skipbelow=10pt,
rightline=false,
leftline=true,
topline=false,
bottomline=false,
linecolor=ocre,
innerleftmargin=5pt,
innerrightmargin=5pt,
innertopmargin=0pt,
leftmargin=0cm,
rightmargin=0cm,
linewidth=4pt,
innerbottommargin=0pt]{dBox}	

% Corollary box
\newmdenv[skipabove=7pt,
skipbelow=7pt,
rightline=false,
leftline=true,
topline=false,
bottomline=false,
linecolor=gray,
backgroundcolor=black!5,
innerleftmargin=5pt,
innerrightmargin=5pt,
innertopmargin=5pt,
leftmargin=0cm,
rightmargin=0cm,
linewidth=4pt,
innerbottommargin=5pt]{cBox}		

% Corollary box
\newmdenv[skipabove=7pt,
skipbelow=7pt,
rightline=true,
leftline=false,
topline=false,
bottomline=true,
linecolor=gray,
backgroundcolor=black!5,
innerleftmargin=5pt,
innerrightmargin=5pt,
innertopmargin=5pt,
leftmargin=0cm,
rightmargin=0cm,
linewidth=1pt,
innerbottommargin=5pt]{rBox}				  
		  

% Creates an environment for each type of theorem and assigns it a theorem text style from the "Theorem Styles" section above and a colored box from above
\newenvironment{theorem}{\begin{tBox}\begin{theoremeT}}{\end{theoremeT}\end{tBox}}
\newenvironment{example}{\begin{exampleT}}{\hfill{\tiny\ensuremath{\blacksquare}}\end{exampleT}}
\newenvironment{definition}{\begin{dBox}\begin{definitionT}}{\end{definitionT}\end{dBox}}
\newenvironment{attention}{\begin{eBox}\small}{\end{eBox}}				  	
\newenvironment{exemple}{\begin{cBox}\small}{\end{cBox}}	

%----------------------------------------------------------------------------------------
%	REMARK ENVIRONMENT
%----------------------------------------------------------------------------------------

\newenvironment{remarque}{\par\vskip10pt\small
\begin{rBox}
\begin{list}{}{
\leftmargin=35pt % Indentation on the left
\rightmargin=25pt}\item\ignorespaces % Indentation on the right
\makebox[-2.5pt]{\begin{tikzpicture}[overlay]
\node[draw=ocre!60,line width=1pt,circle,fill=ocre!25,font=\sffamily\bfseries,inner sep=2pt,outer sep=0pt] at (-15pt,0pt){\textcolor{ocre}{R}};\end{tikzpicture}} % Orange R in a circle
\advance\baselineskip -1pt}
{\end{list}\vskip1mm\end{rBox}\vskip5pt} % Tighter line spacing and white space after remark



\input{/home/aroquemaurel/cours/includesLaTeX/listings.tex} %prise en charge du langage algo
\input{/home/aroquemaurel/cours/includesLaTeX/polices.tex}
\input{/home/aroquemaurel/cours/includesLaTeX/l3/affichageChapitre_td.tex}
\makeatother
%
% kacimi@irit.fr
% [L3][G1.1][Rx] TP1
%
\begin{document}
	\maketitle
	\section{DHCP}
	Les messages DHCP sont envoyés au dessus d'UDP, cf figure \ref{fig:dhcpudp}.
	%% Figure, screen
	\begin{figure}[H]
		\centering
		\includegraphics[width=15cm]{capture1.jpeg}
		\includegraphics[width=15cm]{capture2.jpeg}
		\label{fig:dhcpudp}
		\caption{Messages DHCP}
	\end{figure}
	%%% Diagramme

	\textbf{IP du serveur DHCP}: 192.168.0.254, cela correspond à la source du second message, la réponse.

	Le message \No{}2, à l'instant 0.013422 contient la nouvelle IP, il fait également office d'acquittement, cette IP est 192.168.0.10.

	L'adresse IP à une durée de vie de 10 jours, cette durée de vie est transmise dans le protocole bootstrap via le champ \textit{Address Lease Time}.

	\section{Telnet}
	Ils sont envoyés sur TCP. Cf figure \ref{fig:telnettcp}.
	
	Le login est << guest >> et le mot de passe est << trivial >>, cela permet de se connecter au serveur d'adresse IP \texttt{192.168.0.6}.

	La commande tapée est \texttt{pwd} ayant pour réponse \texttt{/home/guest}. 

	%% Diagramme
	\begin{figure}[H]
		\centering
		\includegraphics[width=10cm]{capture3.jpeg}
		\caption{Messages telnet}
		\label{fig:telnettcp}
	\end{figure}
	
	Figure \ref{fig:telnetdiag} est disponible le diagramme temporel d'échange.
	\begin{figure}[H]
		\centering
		\includegraphics[width=13cm]{capture4.jpeg}
		\caption{Diagramme temporel d'échange}
		\label{fig:telnetdiag}
	\end{figure}

	\begin{remarque}
		C'est le fait que le protocole telnet ne soit pas sécurisé qui nous permet de pouvoir analyser le login et le mot de passe de l'utilisateur,
		ainsi l'utilisation d'un autre protocole tel que SSH permettrai que personne utilisant un analyseur réseau puisse obtenir ces informations
	\end{remarque}
	\section{DNS}
	Les messages sont envoyés en UDP, c'est une requête de type << standad query A >>.

	Plusieurs réponses sont situées dans la requête 114, celle-ci répondant avec les adresses IP et sous domaines de kernel.org\footnote{c'est à dire
	ftp.kernel.org, pub.kernel.org qui correspondent au adresses ip de pub.us.kernel.org, 204.152.191.37 et 204.152.191.5}. Cf figure \ref{fig:kernel}.

	Le protocole DNS permet de faire le lien entre un domaine (ftp.kernel.org) et une adresse ip (204.152.191.37).

	\begin{figure}[H]
		\centering
		\includegraphics[width=13cm]{Capture5.jpeg}
		\caption{Les messages DNS}
		\label{fig:kernel}
	\end{figure}

	\section{Ping}
	Le protocole du ping est le protocole ICMP.

	Pour le second message, contrairement à la requête 101, le client ne connait pas l'adresse, ainsi avant d'effectuer le ping, il effectue une
	résolution de nom à l'aide du serveur DNS. Dans le premier cas, le ping s'effectue immédiatement.
	
	Le protocole ICMP n'a pas de numéro de port car il est situé au même niveau que TCP et UDP.

	\section{FTP}
	Les échanges s'effectuent en TCP, j'utilise le filtre \texttt{ftp || ftp-data} afin d'avoir tous les échanges ftp, y compris les données.

	L'utilisateur est << anonymous >> et le mot de passe << toto@titi.fr >>, il a téléchargé le fichier welcom.msg d'une taille de 1912 octets.
	
	Le message est dans l'échange 218, de protocole ftp-data, le message commence par « Welcome to the\ldots » 

	\begin{figure}[H]
		\centering
		\includegraphics[width=13cm]{Capture6.jpeg}
		\caption{Diagramme temporel d'échange FTP}
		\label{fig:kernel}
	\end{figure}

	\begin{remarque}
		De la même manière que Telnet, FTP n'est pas un protocole sécurisé, ainsi l'utilisation du protocole FTP encapsulé dans du SSH (utilisation de
		sftp) semblerai meilleur. Ainsi il n'aurai pas été possible d'obtenir le contenu des échanges et les logins et mot de passe.
	\end{remarque}

	\section{HTTP}
	\subsection{Image}
	Le client demande la version 1.1 de http, le serveur renvoit bien la version 1.1 avec le statut 200 (signifiant que tout est ok).

	Le fichier a été modifié le Lundi 21 Janvier 2008 a 10h55 et 2 secondes.

	La requête HTTP à une taille totale de 23754 octets.

	Le fichier est grand, ainsi lors de la demande du fichier par l'utilisateur, le serveur HTTP envoie un acquittement, et découpe le fichier en 36
	trames\footnote{jusqu'à la trame 272}. Pour chaque trame, le serveur envoie la trame numérotée, et le client un acquittement (protocole TCP), une
	fois que toutes les trames sont envoyées et acquittées, le serveur peut envoyer \texttt{HTTP/1.1 200 ok} afin de signaler la fin de la
	communication et le bon envoie du fichier.

	\subsection{Site Web}
	Le client est sur le système Linux(Debian 2.0). 

	Le nombre maximal de fichier demandé simultanément est 

	Cette image à été mise en cache dans le navigateur, elle n'est ainsi pas redemandée.

	\subsection{Traceroute}
	Traceroute envoie des messages aux serveurs jusqu'à que ceux-ci atteignent un TTL de 1, ainsi les messages envoyés ont un TTL de plus en plus grand
	afin de passer de plu en plus de routeurs. Une fois qu'un routeur atteind le TTL de 1, celui-ci envoie « time to lie excedeed ».

	\appendix
	\listoffigures
	\end{document}


