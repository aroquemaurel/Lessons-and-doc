\chapter{Introduction}	
\section{Définitions préalables}
	\subsection{Economie}
		<< règles de conduites des affaires domestiques >>, élargit à la nation
		\subsection{Science économique}
		Se définit par un objet d'étude et une méthodologie.

	On définit le besoin économique comme un manque qui peut être satisfait par l'acquisition ou la consommation de besoin et service sachant qu'ils
	sont en produit en quantité limité.

	Comme les ressources en matière première sont rares, il est impossible de produire tous les biens nécessaire à la satisfaction des besoins
	nécessaire, ainsi la science économique va essayer de régler ce problème. Pour appréhender la réalité la science économique va reposer sur des
	modélisations.

	La sciences économique est une approche positive et non une approche normative.

	\subsection{Macro-économie}
	Étude de l'économie en général, elle porte l'intérêt sur des variables (agrégats), meusrées par la comptabilité nationale. 
	\begin{exemple}
		Augmentation du niveau général des prix dans l'économie	
	\end{exemple}
	\subsection{Micro-économie}
	Analyse des comportements individuels des agents économique dans le but de comprendre leur processus de décision
	\subsection{Méso-économie}
	Analyse des groupes (une branches, une filières, un secteur) qui détiennent un pouvoir suffisant pour influencer l'économie nationale.

	\subsection{Questions récurrentes aujourd'hui}
	\begin{itemize}
		\item Quel est le niveau de la croissance ? Comment est-elle répartie ? 
		\item La crise pourra-t-elle est enrayée ? 
		\item La mondialisation est-elle responsable de tous les maux ? 
	\end{itemize}

	Un bref retour sur << le >> contexte en France:
	\begin{itemize}
		\item Phénomène de désindustrialisation : Certains facteur traduisent l'affaiblissement de la France pare qu'il y  eu une industrialisation du
		\item Faiblesse d'entreprise de taille intermédiaire et donc un manque de PME exportatrice.  bas de gamme.
		\item Le taux de chômage
		\item La responsabilité sociétale des entreprises
		\item Un nouveau modèle économique : l'économie du partage  production, distribution, échange, consommation, \ldots
	\end{itemize}

	\subsection{Les types d'entreprises}
	\begin{itemize}
		\item Micro entreprise : moins de 10 personnes, un CA annuel ou un total de bilan n'excédant pas 2 millions d'euros.
		\item PME : moins de 250 personnes, un CA annuel n'excédant par 50 millions d'euros ou un total de bilan n'excédant pas 43 millions.
		\item Entreprise de traille intermédiaire : moins de 5000 personnes
		\item Grandes entreprises : plus de 5000 salariés
	\end{itemize}

	En 2012 environ 3.5 millions d'entreprise en France, 67\% des entreprises française n'ont aucun salarié, 94\% ont de 0 à 9 salariés, et 99\% ont de
	0 à 49 salariés.

	\subsection{L'activité économique et ses agents}
	\begin{definition}
		Un agent économique est un individu ou groupement d'individus qui s'organise et prend des décisions économiques pour satisfaire ses besoins et
		lutter contre la rareté.
	\end{definition}

	On distingue 6 catégories d'agents économiques de par leur fonction économique principale : 
	\begin{description}
		\item[Les entreprises]Production de biens et services marchants et distribution de la richesse grâce à la combinaison de facteurs de
			production tel que le travail est la capital
		\item[Les ménages] Consommation de bien et services rendue possible grâce aux revenus que le ménage perçois, revenu qui est obtenir pour
			l'essentiel en échange de leur travail mais également de revenus secondaires.\\~
			$$revenuPrimaire + revenuSecondaires - Impots = RevenuDisponibleBrut$$

			\begin{remarque}
				\begin{itemize}
					\item La consommation des ménages est un des agrégats les plus stable de l'économie.
					\item La consommation ne dépend pas toujours du niveau de revenus.
					\item La partie non consommée par les revenus est appelé l'épargne
				\end{itemize}
			\end{remarque}
		\item[Les institutions financières] Gestion des produits financiers, opérations de crédit, gestion des moyens de paiement.\\
			La banque centrale auquel les banques se refinance, qui émet de la monnaie et qui limite la création monétaire plus qu'elle va contrôler le
			montant des crédits par exemple.
		\item[Le secteur public] Fournitures des services non marchands, redistribution, régulation. Il se compose des administrations publiques,
			l'état, les collectivités locale, la sécurité sociale, \ldots 
		\item[L'extérieur] Opérations entre entités résidentes et non résidentes.
	\end{description}
\chapter{Qu'est-ce que l'économie ? }

	\section{Le circuit économique}
	Le circuit économique permet de représenter les relations qui existent entre les agents économiques. Les relations s'effectue par des flux réels et
	des flux monétaires.

	\subsection{La comptabilité nationale et la mesure de l'activité}
	Les agents économiques regroupés en catégories d'acteurs sont rassemblés dans un cadre: la comptabilité nationale.

	La comptabilité nationale est une représentation chiffrée de l'économie nationale qui sert à faire des simulations sur la croissance économique et
	donc des prévisions sur le budget de 'état.

	Elle produit: 
	\begin{itemize}
		\item Le tableau économique d'ensemble (TEE)
		\item Le tableau des entrées/sorties (TES)
	\end{itemize}
	\section{Le marché économique}
	\subsection{Introduction}
	\begin{definition}
		L'économie de marché se définit par opposition à une économie d'endettement où les banquessont au cœur du financement de l'économie.
	\end{definition}
	\begin{definition}
		Rôle du marché des capitaux : mutation du financement de l'économie.
	\end{definition}
	\subsection{Les marchés économique}
	\subsection{Le marché des biens et sevices}
	\subsection{Biens matériels}
	Pour les entreprises : bien intérmédiaires (matières premières)/biens de production(appareils électroniques).

	Pour les ménages : bien de consommation courante (alimntaire) / biens de consommation durable (automobile)

	Biens de production + biens de consommation durable = bien d 'équipement.

	\subsection{Services}
	Services marchands(transports, commerce, hotellerie) / services marchands (éducation, police, jusice, armée).

	Regroupés en branches et secteur d'activités.

	\chapter{Les politiques économiques}

