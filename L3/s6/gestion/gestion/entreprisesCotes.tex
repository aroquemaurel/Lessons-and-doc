\chapter{Gouvernance des entreprises}
L'entreprise appartient à ses actionnaires, le chef d'entreprise ne fait que décider en leurs noms.
\begin{itemize}
	\item Droit de propriété
	\item Droit de votes, via les assemblées générales ordinaires
	\item Droit de dividendes si l'entreprise dégage des résultats et si l'entreprise souhaite distribuer des dividendes.
	\item Droit de liquidation
\end{itemize}

\begin{remarque}
	Bien que la distribution des dividendes n'est pas obligatoire, la plupart des entreprises décident d'en
	verser afin que les actionnaires continuent d'apporter des fonds à l'entreprise.
\end{remarque}

\begin{itemize}
	\item Apporte des ressources/fond à l'entreprise
	\item Possède une part de l'entreprise : droit de vote, de propriété
	\item Ils assurent/supportent le risque au sein de l'entreprise
	\item Rémunération possible des actionnaires uniquement si l'entreprise fait des bénéfices
	\item En cas de bénéfices, c'est l'entreprise qui choisit l'argent qu'elle remet aux actionnaires\footnote{Non obligatoire, même en cas de bénéfice}
		Généralement, 40\% des bénéfices vont aux salariés dans la société et 60\% aux actionnaires. 
		Principe de fidélisation, l'actionnaire nous a permis d’avoir des fonds qui nous a permis de faire
		davantage de bénéfices donc on fidélise l'actionnaire pour qu'il continue à fournir des fonds
\end{itemize}

\begin{remarque}
	Un PDG est un Président + Directeur Général\\~
\end{remarque}
\begin{itemize}
	\item Actionnaire ont le droit de vote à l’Assemblée Générale
	\item Il existe des associations de minoritaires (ex: Colette Neuville) pour les actionnaires apportant moins d’argent et ayant donc moins de d’influence sur
		la gestion de l’entreprise. Souvent il n’apporte pas leur droit de vote, vote donnée à un acteur majeur
		Si actionnaire mécontent de la gestion, vend généralement ses titres
	\item Il vote la rémunération des dirigeants
	\item Il vote sur la révocation des dirigeants
\end{itemize}

\begin{definition}
	\textbf{Parachute doré}: Nom donné à la prime de départ prenant la forme d’une clause contractuelle entre un dirigeant d’une société anonyme et l’entreprise qui l’emploie. Elle fixe
	les indemnités versées lors d’une éviction à la suite d’un licenciement, d’une restructuration, d’une fusion avec une autre société ou même lors d’un départ
	programmé de l’intéressé.
\end{definition}

\begin{definition}
	\textbf{Loi NRE (2001)}: 
	\begin{itemize}
		\item Sépération des onctions Président et Directeur Général
		\item Limite de cumul de mandats dans l'administration
	\end{itemize}
\end{definition}

\begin{remarque}
Pour fonctionner correctement le conseil d’administration droit être composé de  10 à 20 personnes.\\~
\end{remarque}

\textbf{Mécanisme incitatif}
\begin{itemize}
	\item Rémunération
	\item Stock\footnote{en anglais = action} option : alignement des intérêts des dirigeants sur ceux des actionnaires (on aligne nos objectifs sur ceux des actionnaires)
\end{itemize}
