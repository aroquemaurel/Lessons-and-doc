\documentclass[a4paper, 11pt]{article}

\usepackage{xcolor}
\input{/home/aroquemaurel/cours/includesLaTeX/couleurs.tex}
\usepackage{lmodern}
\usepackage[utf8]{inputenc}
\usepackage[T1]{fontenc}
\usepackage[francais]{babel}
\usepackage[top=1.7cm, bottom=1.7cm, left=2.5cm, right=2.5cm]{geometry}
\usepackage{verbatim}
\usepackage{tikz} %Vectoriel
\usepackage{listings}
\usepackage{fancyhdr}
\usepackage{multido}
\usepackage{amssymb}
\usepackage{multicol}
\usepackage{float}
\usepackage[urlbordercolor={1 1 1}, linkbordercolor={1 1 1}, linkcolor=vert1, urlcolor=bleu, colorlinks=true]{hyperref}

\newcommand{\titre}{Analyse de couverture de coe}
\newcommand{\numero}{2 \& 3}
\newcommand{\typeDoc}{TP}
\newcommand{\module}{Tests et maintenance logiciel}
\newcommand{\sigle}{TML}
\newcommand{\semestre}{6}


\usepackage{ifthen}
\date{\today}

\chead{Antoine de \bsc{Roquemaurel}}
\rhead{TP\no\typeDoc}
\lhead{\titre}
%\makeindex

\lfoot{Université Toulouse III -- Paul Sabatier}
\rfoot{\sigle\semestre}
%\rfoot{}
\cfoot{--~~\thepage~~--}

\makeglossary
\makeatletter
\def\clap#1{\hbox to 0pt{\hss #1\hss}}%

\def\haut#1#2#3{%
	\hbox to \hsize{%
		\rlap{\vtop{\raggedright #1}
	}%
	\hss
	\clap{\vtop{\centering #2}
}%
\hss
\llap{\vtop{\raggedleft #3}}}}%
\def\bas#1#2#3{%
	\hbox to \hsize{%
		\rlap{\vbox{
			\raggedright #1
		}
	}%
	\hss \clap{\vbox{\centering #2}}%
	\hss
	\llap{\vbox{\raggedleft #3}}}
}%
\def\maketitle{%
	\thispagestyle{empty}{%
		\haut{}{\@blurb}{}
		%	
		%\vfill

		\begin{center}
			\vspace{-2.0cm}
			\usefont{OT1}{ptm}{m}{n}
			\huge \@type \@title
		\end{center}
		\par
		\hrule height 1pt
		\par
		\vspace{1cm}
		\bas{}{}{}
}%
}
\def\date#1{\def\@date{#1}}
\def\author#1{\def\@author{#1}}
\def\type#1{\def\@type{#1}}
\def\title#1{\def\@title{#1}}
\def\location#1{\def\@location{#1}}
\def\blurb#1{\def\@blurb{#1}}
\date{\today}
\newboolean{monBool}
\setboolean{monBool}{true}
\author{}
\title{}
\ifthenelse{\equal{\typeDoc}{}}{
\numeroTD{}
}
{
	\type{\typeDoc~--- }
}
\location{Amiens}\blurb{}
%\makeatother
\title{\titre}
\author{%Semestre \semestre
}

\location{Toulouse}
\blurb{%
\vspace{-35px}
\begin{flushleft}
	Université Toulouse III -- Paul Sabatier\\
	L2 Informatique\\
\end{flushleft}
\begin{flushright}
	\vspace{-45px}
	\Large \textbf \module \\
	\normalsize \textit \today\\
	Semestre \semestre
	\vspace{30px}
\end{flushright}
Antoine de \bsc{Roquemaurel}
}%



%\title{Cours \\ \titre}
%\date{\today\\ Semestre \semestre}

%\lhead{Cours: \titre}
%\chead{}
%\rhead{\thepage}

%\lfoot{Université Paul Sabatier Toulouse III}
%\cfoot{\thepage}
%\rfoot{\sigle\semestre}

\pagestyle{fancy}

\input{/home/aroquemaurel/cours/includesLaTeX/listings.tex} %prise en charge du langage C 




%----------------------------------------------------------------------------------------
%	DEFINITION OF COLORED BOXES
%----------------------------------------------------------------------------------------

\RequirePackage[framemethod=default]{mdframed} % Required for creating the theorem, definition, exercise and corollary boxes

% Theorem box
\newmdenv[skipabove=7pt,
skipbelow=7pt,
backgroundcolor=black!5,
linecolor=ocre,
innerleftmargin=5pt,
innerrightmargin=5pt,
innertopmargin=5pt,
leftmargin=0cm,
rightmargin=0cm,
innerbottommargin=5pt]{tBox}

% Exercise box	  
\newmdenv[skipabove=7pt,
skipbelow=7pt,
rightline=false,
leftline=true,
topline=false,
bottomline=false,
backgroundcolor=ocre!10,
linecolor=ocre,
innerleftmargin=5pt,
innerrightmargin=5pt,
innertopmargin=5pt,
innerbottommargin=5pt,
leftmargin=0cm,
rightmargin=0cm,
linewidth=4pt]{eBox}	

% Definition box
\newmdenv[skipabove=10pt,
skipbelow=10pt,
rightline=false,
leftline=true,
topline=false,
bottomline=false,
linecolor=ocre,
innerleftmargin=5pt,
innerrightmargin=5pt,
innertopmargin=0pt,
leftmargin=0cm,
rightmargin=0cm,
linewidth=4pt,
innerbottommargin=0pt]{dBox}	

% Corollary box
\newmdenv[skipabove=7pt,
skipbelow=7pt,
rightline=false,
leftline=true,
topline=false,
bottomline=false,
linecolor=gray,
backgroundcolor=black!5,
innerleftmargin=5pt,
innerrightmargin=5pt,
innertopmargin=5pt,
leftmargin=0cm,
rightmargin=0cm,
linewidth=4pt,
innerbottommargin=5pt]{cBox}		

% Corollary box
\newmdenv[skipabove=7pt,
skipbelow=7pt,
rightline=true,
leftline=false,
topline=false,
bottomline=true,
linecolor=gray,
backgroundcolor=black!5,
innerleftmargin=5pt,
innerrightmargin=5pt,
innertopmargin=5pt,
leftmargin=0cm,
rightmargin=0cm,
linewidth=1pt,
innerbottommargin=5pt]{rBox}				  
		  

% Creates an environment for each type of theorem and assigns it a theorem text style from the "Theorem Styles" section above and a colored box from above
\newenvironment{theorem}{\begin{tBox}\begin{theoremeT}}{\end{theoremeT}\end{tBox}}
\newenvironment{example}{\begin{exampleT}}{\hfill{\tiny\ensuremath{\blacksquare}}\end{exampleT}}
\newenvironment{definition}{\begin{dBox}\begin{definitionT}}{\end{definitionT}\end{dBox}}
\newenvironment{attention}{\begin{eBox}\small}{\end{eBox}}				  	
\newenvironment{exemple}{\begin{cBox}\small}{\end{cBox}}	

%----------------------------------------------------------------------------------------
%	REMARK ENVIRONMENT
%----------------------------------------------------------------------------------------

\newenvironment{remarque}{\par\vskip10pt\small
\begin{rBox}
\begin{list}{}{
\leftmargin=35pt % Indentation on the left
\rightmargin=25pt}\item\ignorespaces % Indentation on the right
\makebox[-2.5pt]{\begin{tikzpicture}[overlay]
\node[draw=ocre!60,line width=1pt,circle,fill=ocre!25,font=\sffamily\bfseries,inner sep=2pt,outer sep=0pt] at (-15pt,0pt){\textcolor{ocre}{R}};\end{tikzpicture}} % Orange R in a circle
\advance\baselineskip -1pt}
{\end{list}\vskip1mm\end{rBox}\vskip5pt} % Tighter line spacing and white space after remark



\input{/home/aroquemaurel/cours/includesLaTeX/polices.tex}
\input{/home/aroquemaurel/cours/includesLaTeX/affichageChapitre.tex}
\makeatother
\begin{document}
	\maketitle

	\section{Algorithme}
	L'algorithme à été développé en java. Afin de créer la liste, deux classes ont été créés : \texttt{List} et \texttt{Cell}. 

	\subsection{\texttt{Cell}}
	\texttt{Cell} est une cellule de la liste, elle contient donc la cellule précédente, la suivante et la valeur concernée\footnote{La valeur est de
	type \texttt{int}}. 

	Cette classe possède la visibilité package afin que celle si ne soit pas utilisée à mauvais escient.

	\lstinputlisting[language=Java, caption=Classe \texttt{Cell}]{src/list/Cell.java}

	\subsection{\texttt{List}}
	La classe \texttt{List} qui est publique, permet à l'utilisateur de créer une liste triée et d'y ajouter des éléments\footnote{Possibilité d'ajouter
	un élément depuis la méthode \texttt{add} ou depuis le constructeur}.

	L'algorithme proposé à donc été implémenté dans la méthode \texttt{add}.
	\lstinputlisting[language=Java, caption=Classe \texttt{List}]{src/list/List.java}

	\section{Instructions couvrant une insertion en position intermédiaire}
	\subsection{Ajout des deux premiers entiers}
	Figure \ref{fig:1erEntiers} est disponible la couverture pour l'insertion des deux premiers entiers, avant l'insertion de l'entier qui nous
	intéresse.

	\begin{figure}[H]
		\centering
		\includegraphics[width=7.5cm]{screens/ajout1-5.png}
		\label{fig:1erEntiers}
	\end{figure}

	\subsection{Ajout de l'entier 4}
	Afin d'insérer l'entier 4, les entiers 1 et 5 devaient être déjà présents dans la liste, afin
	d'éviter que le rapport d'emma soit faussé par ces deux insertions, une nouvelle méthode à été créé insérant ces deux entiers, cette méthode est
	disponible ci-dessous.

	% TODO
	\begin{figure}[H]
		\centering
		\includegraphics[width=7.5cm]{screens/ajout4.png}
		\label{fig:entier4}
	\end{figure}
	
	Comme le montre la figure \ref{fig:entier4}, l'insertion du 4 ne passe que par une portion du code, en effet, les instructions d'insertion en tête
	ou en queue ne sont pas nécessaire ici, c'est ainsi que la condition \texttt{current != null} du while et le test \texttt{previous == null} ne sont
	pas nécessaire pour l'insertion intermédiaire. Nous pouvons observer avec la figure \ref{fig:1erEntiers} que l'insertion en tête ou en queue à
	l'inverse passe dans ces deux instructions mais n'a pas besoin du test vérifiant si \texttt{current.getData() > data}.

	On peut donc en déduire que c'est ce test qui permet à l'insertion intermédiaire de fonctionner : ça permet d'insérer notre élément à une place
	correct dans la liste triée.
\end{document}
