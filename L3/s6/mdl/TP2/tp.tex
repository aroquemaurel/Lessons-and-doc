\documentclass[a4paper, 11pt]{article}

\usepackage{xcolor}
\input{/home/aroquemaurel/cours/includesLaTeX/couleurs.tex}
\usepackage{lmodern}
\usepackage[utf8]{inputenc}
\usepackage[T1]{fontenc}
\usepackage[francais]{babel}
\usepackage[top=2.7cm, bottom=2.7cm, left=2.5cm, right=2.5cm]{geometry}
\usepackage{verbatim}
\usepackage{tikz} %Vectoriel
\usepackage{listings}
\usepackage{fancyhdr}
\usepackage{multido}
\usepackage{amssymb}
\usepackage{multicol}
\usepackage{float}
\usepackage[urlbordercolor={1 1 1}, linkbordercolor={1 1 1}, linkcolor=vert1, urlcolor=bleu, colorlinks=true]{hyperref}

\newcommand{\titre}{OpenUP}
\newcommand{\numero}{2}
\newcommand{\typeDoc}{TDM}
\newcommand{\module}{Méthodologies de Dév. Logiciel}
\newcommand{\sigle}{MDL}
\newcommand{\semestre}{6}


\usepackage{ifthen}
\date{\today}

\chead{Antoine de \bsc{Roquemaurel}}
\rhead{TP\no\typeDoc}
\lhead{\titre}
%\makeindex

\lfoot{Université Toulouse III -- Paul Sabatier}
\rfoot{\sigle\semestre}
%\rfoot{}
\cfoot{--~~\thepage~~--}

\makeglossary
\makeatletter
\def\clap#1{\hbox to 0pt{\hss #1\hss}}%

\def\haut#1#2#3{%
	\hbox to \hsize{%
		\rlap{\vtop{\raggedright #1}
	}%
	\hss
	\clap{\vtop{\centering #2}
}%
\hss
\llap{\vtop{\raggedleft #3}}}}%
\def\bas#1#2#3{%
	\hbox to \hsize{%
		\rlap{\vbox{
			\raggedright #1
		}
	}%
	\hss \clap{\vbox{\centering #2}}%
	\hss
	\llap{\vbox{\raggedleft #3}}}
}%
\def\maketitle{%
	\thispagestyle{empty}{%
		\haut{}{\@blurb}{}
		%	
		%\vfill

		\begin{center}
			\vspace{-2.0cm}
			\usefont{OT1}{ptm}{m}{n}
			\huge \@type \@title
		\end{center}
		\par
		\hrule height 1pt
		\par
		\vspace{1cm}
		\bas{}{}{}
}%
}
\def\date#1{\def\@date{#1}}
\def\author#1{\def\@author{#1}}
\def\type#1{\def\@type{#1}}
\def\title#1{\def\@title{#1}}
\def\location#1{\def\@location{#1}}
\def\blurb#1{\def\@blurb{#1}}
\date{\today}
\newboolean{monBool}
\setboolean{monBool}{true}
\author{}
\title{}
\ifthenelse{\equal{\typeDoc}{}}{
\numeroTD{}
}
{
	\type{\typeDoc~--- }
}
\location{Amiens}\blurb{}
%\makeatother
\title{\titre}
\author{%Semestre \semestre
}

\location{Toulouse}
\blurb{%
\vspace{-35px}
\begin{flushleft}
	Université Toulouse III -- Paul Sabatier\\
	L2 Informatique\\
\end{flushleft}
\begin{flushright}
	\vspace{-45px}
	\Large \textbf \module \\
	\normalsize \textit \today\\
	Semestre \semestre
	\vspace{30px}
\end{flushright}
Antoine de \bsc{Roquemaurel}
}%



%\title{Cours \\ \titre}
%\date{\today\\ Semestre \semestre}

%\lhead{Cours: \titre}
%\chead{}
%\rhead{\thepage}

%\lfoot{Université Paul Sabatier Toulouse III}
%\cfoot{\thepage}
%\rfoot{\sigle\semestre}

\pagestyle{fancy}

\input{/home/aroquemaurel/cours/includesLaTeX/listings.tex} %prise en charge du langage C 




%----------------------------------------------------------------------------------------
%	DEFINITION OF COLORED BOXES
%----------------------------------------------------------------------------------------

\RequirePackage[framemethod=default]{mdframed} % Required for creating the theorem, definition, exercise and corollary boxes

% Theorem box
\newmdenv[skipabove=7pt,
skipbelow=7pt,
backgroundcolor=black!5,
linecolor=ocre,
innerleftmargin=5pt,
innerrightmargin=5pt,
innertopmargin=5pt,
leftmargin=0cm,
rightmargin=0cm,
innerbottommargin=5pt]{tBox}

% Exercise box	  
\newmdenv[skipabove=7pt,
skipbelow=7pt,
rightline=false,
leftline=true,
topline=false,
bottomline=false,
backgroundcolor=ocre!10,
linecolor=ocre,
innerleftmargin=5pt,
innerrightmargin=5pt,
innertopmargin=5pt,
innerbottommargin=5pt,
leftmargin=0cm,
rightmargin=0cm,
linewidth=4pt]{eBox}	

% Definition box
\newmdenv[skipabove=10pt,
skipbelow=10pt,
rightline=false,
leftline=true,
topline=false,
bottomline=false,
linecolor=ocre,
innerleftmargin=5pt,
innerrightmargin=5pt,
innertopmargin=0pt,
leftmargin=0cm,
rightmargin=0cm,
linewidth=4pt,
innerbottommargin=0pt]{dBox}	

% Corollary box
\newmdenv[skipabove=7pt,
skipbelow=7pt,
rightline=false,
leftline=true,
topline=false,
bottomline=false,
linecolor=gray,
backgroundcolor=black!5,
innerleftmargin=5pt,
innerrightmargin=5pt,
innertopmargin=5pt,
leftmargin=0cm,
rightmargin=0cm,
linewidth=4pt,
innerbottommargin=5pt]{cBox}		

% Corollary box
\newmdenv[skipabove=7pt,
skipbelow=7pt,
rightline=true,
leftline=false,
topline=false,
bottomline=true,
linecolor=gray,
backgroundcolor=black!5,
innerleftmargin=5pt,
innerrightmargin=5pt,
innertopmargin=5pt,
leftmargin=0cm,
rightmargin=0cm,
linewidth=1pt,
innerbottommargin=5pt]{rBox}				  
		  

% Creates an environment for each type of theorem and assigns it a theorem text style from the "Theorem Styles" section above and a colored box from above
\newenvironment{theorem}{\begin{tBox}\begin{theoremeT}}{\end{theoremeT}\end{tBox}}
\newenvironment{example}{\begin{exampleT}}{\hfill{\tiny\ensuremath{\blacksquare}}\end{exampleT}}
\newenvironment{definition}{\begin{dBox}\begin{definitionT}}{\end{definitionT}\end{dBox}}
\newenvironment{attention}{\begin{eBox}\small}{\end{eBox}}				  	
\newenvironment{exemple}{\begin{cBox}\small}{\end{cBox}}	

%----------------------------------------------------------------------------------------
%	REMARK ENVIRONMENT
%----------------------------------------------------------------------------------------

\newenvironment{remarque}{\par\vskip10pt\small
\begin{rBox}
\begin{list}{}{
\leftmargin=35pt % Indentation on the left
\rightmargin=25pt}\item\ignorespaces % Indentation on the right
\makebox[-2.5pt]{\begin{tikzpicture}[overlay]
\node[draw=ocre!60,line width=1pt,circle,fill=ocre!25,font=\sffamily\bfseries,inner sep=2pt,outer sep=0pt] at (-15pt,0pt){\textcolor{ocre}{R}};\end{tikzpicture}} % Orange R in a circle
\advance\baselineskip -1pt}
{\end{list}\vskip1mm\end{rBox}\vskip5pt} % Tighter line spacing and white space after remark



\input{/home/aroquemaurel/cours/includesLaTeX/polices.tex}
\input{/home/aroquemaurel/cours/includesLaTeX/affichageChapitre.tex}
\makeatother
\begin{document}
	\maketitle
	\section{Sujet du projet}
	Le projet était le développement de personnage pouvant interagir en réseau en utilisant le serveur fourni, ceci en utilisant le langage Java.

	Les personnages pouvaient ensuite se déplacer, ramasser des objets et se battre, chaque personnage et/ou objet possédant ses propres caractéristiques. Le
	serveur ne devait pas être modifié.
	\section{Phases du projet}
	\begin{description}
		\item[Inception] Effectué par l'université, en effet la conception était déjà faite
		\item[Elaboration] Effectué par l'université
		\item[Construction] Développement de nos personnages
		\item[Transition] Tests de notre projet en réseau, afin de vérifier que tout fonctionne correctement
	\end{description}

	À l'issue de ce projet, nous avons livrés le logiciel contenant les sources et le fichier jar, ainsi qu'une documentation au format PDF et
	HTML\footnote{Générée à l'aide de Doxygen} et enfin, un rapport expliquant nos choix de conception et expliquant le fonctionnement du logiciel.

	La documentation à été effectué durant le développement, les choix de conception ont été posé au début de la phase de construction, quant au source, elles
	ont été écrites durant la phase de construction.
	\section{Difficultés}
	Les difficultés de ce projet ont principalement été de la compréhension plutôt que de l'organisation, en effet, ayant travaillé plusieurs fois ensemble, nous
	nous connaissions et avons l'habitude de notre organisation. De plus, mon expérience du DUT Informatique m'a apparis un certain nombre de chose, notamment
	la gestion de petits projets : Utilisation de Redmine pour la répartition des tâches, de Sonar pour l'analyse du code, de Git pour le travail en équipe, mise
	en place de conventions d'écritures, \ldots

	Cependant, notre principale difficulté à été de comprendre le code fournis par l'université, notamment parce que celui-ci était assez mal conçu, et donc difficilement
	compréhensible afin de le modifier au mieux, d'autant plus que son adaptation n'était pas envisageable : nous avons ainsi passé trop de temps sur la
	compréhension de l'existant par rapport au temps que nous aurions du y passer.

	Afin de pouvoir réaliser notre projet plus rapidement et simplement, il aurait été agréable que le projet existant possède des tests unitaires et des tests
	d'intégration : ainsi nous aurions facilement pus éviter certaines régression du projet ceci en lançant les tests régulièrement, et en utilisant un outil
	d'intégration continue.

	\newpage
	\section{Conclusion}
	L'utilisation de la méthode OpenUP aurait été inutile dans ce projet. Et ceci pour plusieurs raisons.

	Tout d'abord, la taille du projet était trop petit, que ça soit dans le temps ou en charge de travail, ainsi l'utilisation de cette méthode nous aurait
	plutôt gêné, notamment car ne nous la connaissions pas parfaitement : une formation aurait été nécessaire.\\
	OpenUP à un fonctionnement incrémental, chaque incrément ayant une durée assez conséquente, en une semaine sa mise en place aurait été laborieuse.
	
	Également, nous n'étions que deux, l'utilisation d'OpenUP pour deux personnes est quasiment inutile, et pose de nombreux problèmes, comme la répartition des
	rôles.

	Enfin, la méthode OpenUP nécessite l'implication du client, or notre client n'était pas clairement définis.
\end{document}
