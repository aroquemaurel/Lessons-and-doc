\documentclass[a4paper, 11pt]{article}

\usepackage{xcolor}
\input{/home/aroquemaurel/cours/includesLaTeX/couleurs.tex}
\usepackage{lmodern}
\usepackage[utf8]{inputenc}
\usepackage[T1]{fontenc}
\usepackage[francais]{babel}
\usepackage[top=1.7cm, bottom=1.7cm, left=2.5cm, right=2.5cm]{geometry}
\usepackage{verbatim}
\usepackage{tikz} %Vectoriel
\usepackage{listings}
\usepackage{fancyhdr}
\usepackage{multido}
\usepackage{amssymb}
\usepackage{multicol}
\usepackage{float}
\usepackage[urlbordercolor={1 1 1}, linkbordercolor={1 1 1}, linkcolor=vert1, urlcolor=bleu, colorlinks=true]{hyperref}

\newcommand{\titre}{OpenUP}
\newcommand{\numero}{1}
\newcommand{\typeDoc}{TDM}
\newcommand{\module}{Méthodologies de Dév. Logiciel}
\newcommand{\sigle}{MDL}
\newcommand{\semestre}{6}


\usepackage{ifthen}
\date{\today}

\chead{Antoine de \bsc{Roquemaurel}}
\rhead{TP\no\typeDoc}
\lhead{\titre}
%\makeindex

\lfoot{Université Toulouse III -- Paul Sabatier}
\rfoot{\sigle\semestre}
%\rfoot{}
\cfoot{--~~\thepage~~--}

\makeglossary
\makeatletter
\def\clap#1{\hbox to 0pt{\hss #1\hss}}%

\def\haut#1#2#3{%
	\hbox to \hsize{%
		\rlap{\vtop{\raggedright #1}
	}%
	\hss
	\clap{\vtop{\centering #2}
}%
\hss
\llap{\vtop{\raggedleft #3}}}}%
\def\bas#1#2#3{%
	\hbox to \hsize{%
		\rlap{\vbox{
			\raggedright #1
		}
	}%
	\hss \clap{\vbox{\centering #2}}%
	\hss
	\llap{\vbox{\raggedleft #3}}}
}%
\def\maketitle{%
	\thispagestyle{empty}{%
		\haut{}{\@blurb}{}
		%	
		%\vfill

		\begin{center}
			\vspace{-2.0cm}
			\usefont{OT1}{ptm}{m}{n}
			\huge \@type \@title
		\end{center}
		\par
		\hrule height 1pt
		\par
		\vspace{1cm}
		\bas{}{}{}
}%
}
\def\date#1{\def\@date{#1}}
\def\author#1{\def\@author{#1}}
\def\type#1{\def\@type{#1}}
\def\title#1{\def\@title{#1}}
\def\location#1{\def\@location{#1}}
\def\blurb#1{\def\@blurb{#1}}
\date{\today}
\newboolean{monBool}
\setboolean{monBool}{true}
\author{}
\title{}
\ifthenelse{\equal{\typeDoc}{}}{
\numeroTD{}
}
{
	\type{\typeDoc~--- }
}
\location{Amiens}\blurb{}
%\makeatother
\title{\titre}
\author{%Semestre \semestre
}

\location{Toulouse}
\blurb{%
\vspace{-35px}
\begin{flushleft}
	Université Toulouse III -- Paul Sabatier\\
	L2 Informatique\\
\end{flushleft}
\begin{flushright}
	\vspace{-45px}
	\Large \textbf \module \\
	\normalsize \textit \today\\
	Semestre \semestre
	\vspace{30px}
\end{flushright}
Antoine de \bsc{Roquemaurel}
}%



%\title{Cours \\ \titre}
%\date{\today\\ Semestre \semestre}

%\lhead{Cours: \titre}
%\chead{}
%\rhead{\thepage}

%\lfoot{Université Paul Sabatier Toulouse III}
%\cfoot{\thepage}
%\rfoot{\sigle\semestre}

\pagestyle{fancy}

\input{/home/aroquemaurel/cours/includesLaTeX/listings.tex} %prise en charge du langage C 




%----------------------------------------------------------------------------------------
%	DEFINITION OF COLORED BOXES
%----------------------------------------------------------------------------------------

\RequirePackage[framemethod=default]{mdframed} % Required for creating the theorem, definition, exercise and corollary boxes

% Theorem box
\newmdenv[skipabove=7pt,
skipbelow=7pt,
backgroundcolor=black!5,
linecolor=ocre,
innerleftmargin=5pt,
innerrightmargin=5pt,
innertopmargin=5pt,
leftmargin=0cm,
rightmargin=0cm,
innerbottommargin=5pt]{tBox}

% Exercise box	  
\newmdenv[skipabove=7pt,
skipbelow=7pt,
rightline=false,
leftline=true,
topline=false,
bottomline=false,
backgroundcolor=ocre!10,
linecolor=ocre,
innerleftmargin=5pt,
innerrightmargin=5pt,
innertopmargin=5pt,
innerbottommargin=5pt,
leftmargin=0cm,
rightmargin=0cm,
linewidth=4pt]{eBox}	

% Definition box
\newmdenv[skipabove=10pt,
skipbelow=10pt,
rightline=false,
leftline=true,
topline=false,
bottomline=false,
linecolor=ocre,
innerleftmargin=5pt,
innerrightmargin=5pt,
innertopmargin=0pt,
leftmargin=0cm,
rightmargin=0cm,
linewidth=4pt,
innerbottommargin=0pt]{dBox}	

% Corollary box
\newmdenv[skipabove=7pt,
skipbelow=7pt,
rightline=false,
leftline=true,
topline=false,
bottomline=false,
linecolor=gray,
backgroundcolor=black!5,
innerleftmargin=5pt,
innerrightmargin=5pt,
innertopmargin=5pt,
leftmargin=0cm,
rightmargin=0cm,
linewidth=4pt,
innerbottommargin=5pt]{cBox}		

% Corollary box
\newmdenv[skipabove=7pt,
skipbelow=7pt,
rightline=true,
leftline=false,
topline=false,
bottomline=true,
linecolor=gray,
backgroundcolor=black!5,
innerleftmargin=5pt,
innerrightmargin=5pt,
innertopmargin=5pt,
leftmargin=0cm,
rightmargin=0cm,
linewidth=1pt,
innerbottommargin=5pt]{rBox}				  
		  

% Creates an environment for each type of theorem and assigns it a theorem text style from the "Theorem Styles" section above and a colored box from above
\newenvironment{theorem}{\begin{tBox}\begin{theoremeT}}{\end{theoremeT}\end{tBox}}
\newenvironment{example}{\begin{exampleT}}{\hfill{\tiny\ensuremath{\blacksquare}}\end{exampleT}}
\newenvironment{definition}{\begin{dBox}\begin{definitionT}}{\end{definitionT}\end{dBox}}
\newenvironment{attention}{\begin{eBox}\small}{\end{eBox}}				  	
\newenvironment{exemple}{\begin{cBox}\small}{\end{cBox}}	

%----------------------------------------------------------------------------------------
%	REMARK ENVIRONMENT
%----------------------------------------------------------------------------------------

\newenvironment{remarque}{\par\vskip10pt\small
\begin{rBox}
\begin{list}{}{
\leftmargin=35pt % Indentation on the left
\rightmargin=25pt}\item\ignorespaces % Indentation on the right
\makebox[-2.5pt]{\begin{tikzpicture}[overlay]
\node[draw=ocre!60,line width=1pt,circle,fill=ocre!25,font=\sffamily\bfseries,inner sep=2pt,outer sep=0pt] at (-15pt,0pt){\textcolor{ocre}{R}};\end{tikzpicture}} % Orange R in a circle
\advance\baselineskip -1pt}
{\end{list}\vskip1mm\end{rBox}\vskip5pt} % Tighter line spacing and white space after remark



\input{/home/aroquemaurel/cours/includesLaTeX/polices.tex}
\input{/home/aroquemaurel/cours/includesLaTeX/affichageChapitre.tex}
\makeatother
\begin{document}
	\maketitle
	\section{Que représente cet ensemble de pages web ?}
	Cet ensemble de site web est la documentation de la méthode OpenUP, comment celle-ci fonctionne. Une équipe utilisant cette méthode devrait se
	référer à cet ensemble de pages comme référence.
	
	\section{Phases de openUP} 
	La méthode OpenUp est composé de 4 phases exécuté consécutivement : 
	\begin{description}
		\item[Inception] Compréhension du projet, de ses objectifs et obtention
			d'informations pour confirmer la faisabilité du
			projet.
		\item[Elaboration] Établir les bases de l'architecture du logiciel
		\item[Construction] Implémenter et tester les fonctions du logiciel pour développer un système complet. 
		\item[Transition] Phase finale permettant de s'assurer que le logiciel est prêt pour le déploiement du logiciel. 
	\end{description}
%	1 Inception
%	2 Elaboration
%	3 Construction
%	4 Transition
%> Dans quel ordre sont-elles effectuées ? 
%	Cf num.
%> Quels sont les objectifs de chacune d'elles ?

\section{Les itérations}
%> Qu’est-ce qu’une itération ?
Une itération est une période de temps pendant laquelle une fonctionnalité du programme est développé. 
Elle amène à un produit stable, avec une version exécutable du produit possédant une documentation.

%> Quels buts vise-t-on lors de la définition du contenu d’une itération ?
\subsection{Durée}
Une itération dure 4 à 6 semaines.

\subsection{Résultat}
Une itération donne un exécutable stable du produit, avec sa documentation et les éventuels scripts d'installations.
%> Quel est le résultat d’une itération ?
%	Une démonstration (Demo-able or Shippable build)
	
\section{Rôles principaux}
7 rôles sont présent dans une équipe OpenUP : 
\begin{description}
	\item[Stackeholder (Parties prenantes)] Représente les intérêts du groupe dont les besoins doivent être satisfait par le projet.
	\item[Analyst (Analyste)] Représente le client et l'utilisateur final préoccupé par le recueillement des commentaires des
		parties prenantes pour comprendre le problème à résoudre. 
	\item[Architect (Architecte)] Il est responsable de la conception du logiciel. 
	\item[Developer (Développeur)] Il doit développer une partie du système, incluant la conception s'insérant dans l'architecture.
	\item[Tester (Testeur)] Le responsable des tests, comme identifié, implémenté et conduire les tests nécessaires. 
	\item[Project Manager (Chef de projet)] Conduit la planification en collaboration avec les parties prenantes et l'équipe.
	\item[Any role (Tous les rôles)] Représente n'importe qui dans l'équipe qui peut effectuer des tâches d'ordre général.
\end{description}
\begin{table}[H]
	\begin{tabular}{l |p{10cm}}
	\textbf{Acteur}  & \textbf{Produit}\\
\hline
Stackeholder& \\
Analyst&  Glossary, spécifications, use case, vision, \\
Architect& Architecture notebook, define vision\\
Developer&  Implementation, build, Developer Test, Design\\
Tester& Test case, Test script, Test log\\
Project Manager& Iteration plan, Project plan, work items list\\
\end{tabular}
\caption{Responsabilité des acteurs}
\end{table}
\begin{table}
	\begin{tabular}{p{5cm} |p{11cm}|}
	\textbf{Activité} & \textbf{Participant}\\
	Initiate Project & Analyst, Architect, Developer, Project Manager, Stakeholder, Tester\\
	\hline
	Plan and manage iteration &Analyst, Architect, Developer, Project Manager, Stakeholder, Tester\\
	\hline
	Identify and refine requirements &Analyst, Architect, Developer, Stakeholder, Tester\\
	\hline
	Agree on technical approach&Analyst, Architect, Developer, Project Manager, Stakeholder\\
	\hline
	Develop teh architecture&Analyst, Architect, Developer, Project Manager, stakeholder, Tester\\
	\hline
	Develop solution increment&Analyst, Architect, Developer, Stakeholder, Tester\\
	\hline
	Test solution&Analyst, Developer, Stakeholder, Tester\\
	\hline
	Ongoing tasks&Any role\\
	\hline
	Develop product Documentation and training&Developer\\
	\hline
	Finalize product documentation and training&Developer\\
	\hline
	Prepare for release&Developer\\
	\hline
	Deploy release to production&Developer\\
	\hline
\end{tabular}
\caption{Activités et leur participants}
\end{table}

%Analyst &
%Any Role &
%Architect &
%Developer &
%Project Manager &
%Stakeholder &
%Tester &
%\end{tabular}
%> Quels sont les rôles principaux ?
\section{Build}
Un build est un livrable incrémental pour l'utilisateur et le client, il fourni un outil testable pour vérification.

Cette version fonctionnelle du système ou partie du système est le résultat de l'itération.
 
%> Qui le produit ?
\subsection{Qui le produit ? }
Le build est produit par les développeurs.

%> Lors de quelle activité ?
\subsection{Lors de quelle activité ?}
Lors de l'activité "Develop Solution Increment", présent dans les phases 2,3,4.

\section{Définitions du logiciel à développer}
\subsection{Où définit-on le logiciel à développer ?}
Lors de la première phase, dans tous les produits concernant les << Requirements >> (Exigences/Besoins), c'est-à-dire Glossary, Vision, System-Wide Requirement,
Use-case Model et Use Case.

\subsection{Rôles participant à cette définition} 
Les rôles participant à la définition du logiciel sont le chef de projet, l'analyste, l'architecte, le développeur, le stakeholder et le testeur : soit tous les
rôles principaux.

\subsection{Activités correspondantes}
Les activités de la première phase : Initiate project, Plan and manage iteration, identify and refine requirements, agree on technical approach.

%> Quelle forme prend cette définition ?
\subsection{Forme de la définition}
La méthode se sert des diagrammes Use Case(Diagramme de cas d'utilisations), ça permet de facilement définir les besoins du client.
	
\end{document}
