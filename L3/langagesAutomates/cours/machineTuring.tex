\chapter{Machine de Turing}
\begin{description}
	\item[Langage régulier] $a^*b^*$
	\item[Langage hors contexte] 
		\begin{itemize}
			\item $a^nb^n$ avec $n\leq 0$ : Automate à pile déterministe
			\item $w_1.c.\widetilde{w_1}$ avec $n\leq 0$ : Automate à pile déterministe
			\item $w_1.\widetilde{w_1}$ avec $n\leq 0$ : Automate à pile non déterministe
		\end{itemize}
	\item[Langage context sensitive] $a^nb^nc^n$ : Machine de Turing
\end{description}
La machine de Turing permet de conserver l'information plus longtemps qu'une pile, pour cela on utilise un ruban pour lire et conserver l'information écrite.

\begin{remarque}
C'est le plus puissant des automates, bien que sont fonctionnement soit relativement simple.\\~
\end{remarque}

\section{Informellement}
Alan \bsc{Turing} est un mathématicien qui invente en 1936 la machine de Turing dont le principe était simple : Posséder un ruban très grand, qu'il le
considère d'une taille infini vers la droite\footnote{Celui-ci possède cependant une butée sur la gauche}, avec la possibilité de se déplacer vers la
droite, vers la gauche d'écrire et de lire sur ce ruban.

\begin{remarque}
	Encore de nos jours, n'importe quel algorithme peut être résolu à l'aide d'une machine de Turing\\~
\end{remarque}
\section{Définition}
Une machine de Turing est un 7-uplet: $<Q,X,F,\Gamma, B, q_0, \delta>$
\begin{description}
	\item[$Q$]Ensemble d'états ; $q_0 \in Q$ état initial
	\item[$F$] Ensemble d'états accepteurs ; $F \subseteq Q$
	\item[$X$] Alphabet(lecture)
	\item[$\Gamma$] Alphabet de lecture et d'écriture
	\item[$B$] << Symbole blanc >> que nous noterons \#
	\item[$\delta$] Fonction de transition qui avec un état courant et le symbole lu donne le prochain état, le symbole écrit et le sens de déplacement
		$Q \times \delta \rightarrow Q \times \Gamma \times \{L,R\}$
\end{description}
\section{Configuration}
La configuration est un triplet $(q, \alpha_1, \alpha_2)$.
\begin{description}
	\item[$q$] $q \in Q$, l'état courant.
	\item[$\alpha_1$]Tout ce qui est avant la tête de lecture.$\alpha_1 \in \Gamma^*$
	\item[$\alpha_2$] Tout ce qui est derrière la tête de lecture jusqu'au premier caractère blanc. 
		$\alpha_2 \in\{\lambda\} \cup \Gamma^*.(\Gamma -\{\beta\})$
\end{description}<++>
\section{Relation entre configuration}
\section{Langages acceptés ; langages décidés}
\section{Introduction à la calculabilité}
