%%%%%%%%%%%%%%%%%%%%%%%%%%%%%%%%%%%%%%%%%%%%%%%%%%%%%%%%%%%%%%%%%%%%%%%
%
%   Presentation of Beamer UNL Theme
%   Beamer Presentation by Chris Bourke
%
%%%%%%%%%%%%%%%%%%%%%%%%%%%%%%%%%%%%%%%%%%%%%%%%%%%%%%%%%%%%%%%%%%%%%%%

\documentclass{beamer}

\usetheme[hideothersubsections]{UNLTheme}


\title{UNL Beamer Theme}
\author{Christopher M. Bourke} %
\institute{University Of Nebraska -- Lincoln}
\date{Spring 2005}

\begin{document}

%{% open a Local TeX Group
%\setbeamertemplate{sidebar}{}
\begin{frame}
        \titlepage
        \begin{center}
    \href{mailto:cbourke@cse.unl.edu}{\color{blue}{\texttt{cbourke@cse.unl.edu}}}
        \end{center}
\end{frame}
%}% end Local TeX Group


\section{Introduction}

\begin{frame}
    \frametitle{Introduction}
    \framesubtitle{}

    \texttt{Beamer} is a wonderful \LaTeX\ document class that produces
    high quality PDF slide presentations.
    
    Moreover, you can customize and develop your own themes!

\end{frame}

\section{UNL Theme}    

\begin{frame}
    \frametitle{UNL Theme}
    \framesubtitle{}

    This theme was developed for the University of Nebraska--Lincoln.
    
    The theme itself was developed from the PaloAlto, sidebar and sidbartab
    themes available by default in beamer.
    
    However, this theme has several unique features and customizations:
    \begin{itemize}
      \item The color theme uses UNL's ``scarlet and creme'' colors.
      \item Improved spacing.
      \item Math mode preserves \LaTeX's serif font.
      \item Incorporates UNL's Logo automatically.
      \item Unique drop shadows on the top and side bars!
    \end{itemize}

\end{frame}

\begin{frame}  %---------------------------------------------------------------
    \frametitle{Other Features}
    \framesubtitle{}
    
    For convenience, if you are in handout mode, all features, including the
    navigation symbols at the bottom right are shut off!


    Beamer boxes are by default, rounded and have a drop shadow.  All beamer
    boxes (definition, theorem, etc) have the same color scheme.

    \begin{definition}
      This is my definition
      $$A = \{p \mid \textrm{$p$ is prime }\}$$
    \end{definition}
    
    \begin{theorem}
      The set $A$ is countable.
    \end{theorem}

\end{frame}

\section{Using The Theme}

\begin{frame}[fragile]
    \frametitle{How To Use The Theme}
    \framesubtitle{Theme Options}
    
    To use the UNL Theme, after your document class declaration, simply use:
    
    \begin{verbatim}\usetheme{UNLTheme}\end{verbatim}
    
    To pass options to the package, use
    
    \verb"\usetheme[left,hideothersubsections,width=2.5cm]{UNLTheme}"
    
\end{frame}

\begin{frame}[fragile]
    \frametitle{How To Use The Theme}
    \framesubtitle{Theme Options}
        
    There are also several options that you can pass:
    \begin{itemize}
      \item \texttt{hideothersubsections} -- This hides subsections in the 
            sidebar \emph{other than} the subsections of the current section.
      \item \texttt{hideallsubsections} -- This option doesn't print \emph{any}
            subsections in the sidebar.
      \item \texttt{width} -- sets the width of the sidebar, default is 
      	    \verb"2.5\baselineskip"
      \item \texttt{height} -- sets the height of the header, default is 
      	    \verb"2.5\baselineskip"
      \item \texttt{left} -- sets the sidebar to the left of the slide (default).
      \item \texttt{right} -- sets the sidebar to the right of the slide.
    \end{itemize}
    
\end{frame}

\begin{frame}
    \frametitle{How To Get The Theme}
    \framesubtitle{}
    
    The theme is available from my home page,
    \textcolor{blue}{\url{http://www.cse.unl.edu/~cbourke/}}
    

    You need \texttt{You need beamerthemeUNLTheme.sty} and \texttt{UNL.pdf}
    (the UNL logo).
    
    Place them in the working directory or add them to \texttt{beamer/themes/theme}
    or somewhere in your \LaTeX\ path and you're good to go!
    
\end{frame}
    
\end{document}
